%%AmSTeX
%\documentstyle{amsppt}
\input amstex
\input amsppt.sty


\magnification=\magstep1
\pagewidth{5.5in}
\pageheight{19.5cm}


\def \emph#1{\underbar{#1}}
\def \Prob{\operatorname{Prob}}
\def \var{\operatorname{var}}

\heading
Datov\'e struktury
\endheading

\subhead
\'Uvod
\endsubhead

\flushpar Z\'akladn\'\i\ probl\'em: Reprezentace mno\v zin a 
operace s nimi. V \v rad\v e \'uloh a algoritm\accent23u je tento 
podprobl\'em rozhoduj\'\i c\'\i\ pro slo\v zitost \v re\v sen\'\i , proto\v ze 
tyto operace se mnohokr\'at opakuj\'\i . Proto je t\v reba 
navrhnout pro tyto \'ulohy co nejefektivn\v ej\v s\'\i\ algoritmy 
(ka\v zd\'y u\v set\v ren\'y \v cas mnohon\'asobn\'ym opakov\'an\'\i m za\v cne 
hr\'at d\accent23ule\v zitou roli). To vede k detailn\'\i\ 
anal\'yze slo\v zitosti v z\'avislosti na vn\v ej\v s\'\i ch okolnostech. 
Nelze \v r\'\i ct, 
\v ze n\v ekter\'y algoritmus je nejlep\v s\'\i , proto\v ze za ur\v cit\'ych 
okolnost\'\i\ m\accent23u\v ze b\'yt `m\'en\v e efektivn\'\i ' algoritmus 
v\'yhodn\v ej\v s\'\i .
\medskip

\flushpar \v Re\v s\'\i me tzv. slovn\'\i kov\'y probl\'em: D\'ano 
univerzum $U$, m\'ame reprezentovat $S\subseteq U$ a navrhnout algoritmy pro 
n\'asle\-duj\'\i c\'\i\ operace\newline 
{\bf MEMBER$(x)$} -- zjist\'\i , zda $x\in S$, a nalezne jeho 
ulo\v zen\'\i\newline 
{\bf INSERT$(x)$} -- kdy\v z $x\notin S$, pak vlo\v z\'\i\ $x$ do struktury 
repre\-zentuj\'\i c\'\i\ $S$\newline 
{\bf DELETE$(x)$} -- kdy\v z $x\in S$, pak odstran\'\i\ $x$ ze struktury 
reprezentuj\'\i c\'\i\ $S$.\newline 
Efektivita algoritmu: \v casov\'a slo\v zitost, prostorov\'a 
slo\v zitost;\newline 
vy\-\v set\v ren\'e bu\v d v nejhor\v s\'\i m p\v r\'\i pad\v e nebo v 
pr\accent23um\v ern\'em p\v r\'\i pad\v e nebo amortizovan\v e.
\medskip

\flushpar Literatura:\newline 
K.~Mehlhorn: Data Structures and Algorithms 1: Sorting 
and Searching, Springer 1984\newline 
http://www.mpi-sb.mpg.de/$^{\thicksim}$mehlhorn/DatAlgbooks.html
\smallskip

\flushpar J.~S.~Vitter, W.-Ch.~Chen: Design and Analysis of 
Coalesced Hashing, Oxford Univ. Press, 1987
\newpage

\heading
Ha\v sov\'an\'\i
\endheading

\flushpar Pomoc\'\i\ bitov\'eho pole m\accent23u\v zeme rychle 
implementovat operace {\bf MEMBER}, {\bf INSERT} a {\bf DELETE}. 

\flushpar Nev\'yhoda: kdy\v z je velk\'e univerzum, pak je prostorov\'a 
slo\v zitost v nejlep\v s\'\i m p\v r\'\i pad\v e ohrom\-n\'a, ve \v spatn\'em 
p\v r\'\i pad\v e nelze pole zadat do po\v c\'\i ta\v ce.\newline 
Ha\v sov\'an\'\i\ chce zachovat rychlost operac\'\i , ale odstranit 
pam\v e\v to\-vou n\'aro\v cnost. Prvn\'\i\ publikovan\'y \v cl\'anek o 
ha\v sov\'an\'\i\ je od Dumney z roku 1956, prvn\'\i\ anal\'yza ha\v sov\'an\'\i\ 
poch\'az\'\i\ od Petersona z roku 1957, ale existuje technick\'a  
zpr\'ava od IBM o ha\v sov\'an\'\i\ z roku 1953.
\medskip

\flushpar Z\'akladn\'\i\ idea: D\'ano univerzum $U$ a mno\v zina 
$S\subseteq U$ tak, \v ze $|S|<<|U|$. M\'ame funkci 
$h:U@>>>\{0,1,\dots,m-1\}$ a mno\v zinu $S$ reprezentujeme 
tabulkou (polem) s $m$ \v r\'adky tak, \v ze $s\in S$ je ulo\v zen na 
\v r\'adku $h(s)$.\newline 
Nev\'yhoda: mohou existovat r\accent23uzn\'a $s,t\in S$ takov\'a, 
\v ze $h(s)=h(t)$ - tento jev se naz\'yv\'a \emph{kolize}.\newline 
Hlavn\'\i\ probl\'em: \v re\v sen\'\i\ koliz\'\i .\newline 
Z\'akladn\'\i\ \v re\v sen\'\i : pou\v zijeme pole o velikosti $[
0..m-1]$ a 
$i$-t\'a polo\v zka pole bude spojov\'y seznam obsahuj\'\i c\'\i\ 
v\v sechny prvky $s\in S$ takov\'e, \v ze $h(s)=i$. Toto \v re\v sen\'\i\ se 
naz\'yv\'a \emph{ha\v sov\'an\'\i} \emph{se} \emph{separovan\'ymi} 
\emph{\v ret\v ezci}.

\flushpar P\v r\'\i klad: $U=\{1,2,\dots,1000\}$, $S=\{1,7,11,53,
73,141,161\}$ a 
funkce je $h(x)=x\bmod10$. Pak 
$$\gather P(0)=P(2)=P(4)=P(5)=P(6)=P(8)=P(9)=\emptyset,\\  
P(7)=<7>,\quad P(3)=<53,73>,\quad P(1)=<1,141,11,161>.\endgather$$
Seznamy nemus\'\i\ b\'yt uspo\v r\'adan\'e.  
\medskip

\subhead
Algoritmy operac\'\i
\endsubhead
\smallskip

\flushpar{\bf MEMBER$(x)$}:\newline 
Spo\v c\'\i t\'ame $i:=h(x)$, $t:=NIL$\newline 
{\bf if} $i$-t\'y seznam je nepr\'azdn\'y {\bf then\newline 
\phantom{{\rm ---}}$t:=$}prvn\'\i\ prvek $i$-t\'eho seznamu\newline 
\phantom{---}{\bf while} $t\ne x$ a $t\ne$posledn\'\i\ prvek $i$-t\'eho seznamu {\bf do\newline 
\phantom{{\rm ------}}$t:=$}n\'asleduj\'\i c\'\i\ prvek $i$-t\'eho seznamu\newline 
\phantom{---}{\bf enddo\newline 
endif\newline 
if} $t=x$ {\bf then V\'ystup}: $x\in S$ {\bf else V\'ystup}: $x\notin 
S$ {\bf endif
\medskip

\flushpar INSERT$(x)$}:\newline 
Spo\v c\'\i t\'ame $i:=h(x)$, $t:=NIL$\newline 
{\bf if} $i$-t\'y seznam je nepr\'azdn\'y {\bf then\newline 
\phantom{{\rm ---}}$t:=$}prvn\'\i\ prvek $i$-t\'eho seznamu\newline 
\phantom{---}{\bf while} $t\ne x$ a $t\ne$posledn\'\i\ prvek $i$-t\'eho seznamu {\bf do\newline 
\phantom{{\rm ------}}$t:=$}n\'asleduj\'\i c\'\i\ prvek $i$-t\'eho seznamu\newline 
\phantom{---}{\bf enddo\newline 
endif\newline 
if} $t\ne x$ {\bf then} vlo\v z\'\i me $x$ do $i$-t\'eho seznamu {\bf endif
\medskip

\flushpar DELETE$(x)$}:\newline 
Spo\v c\'\i t\'ame $i:=h(x)$, $t:=NIL$\newline 
{\bf if} $i$-t\'y seznam je nepr\'azdn\'y {\bf then\newline 
\phantom{{\rm ---}}$t:=$}prvn\'\i\ prvek $i$-t\'eho seznamu\newline 
\phantom{---}{\bf while} $t\ne x$ a $t\ne$posledn\'\i\ prvek $i$-t\'eho seznamu {\bf do\newline 
\phantom{{\rm ------}}$t:=$}n\'asleduj\'\i c\'\i\ prvek $i$-t\'eho seznamu\newline 
\phantom{---}{\bf enddo\newline 
endif\newline 
if} $t=x$ {\bf then} odstran\'\i me $x$ z $i$-t\'eho seznamu {\bf endif}
\bigskip

\flushpar V n\'asleduj\'\i c\'\i\ anal\'yze p\v redpokl\'ad\'ame, \v ze hodnota funkce $
h(x)$ 
je spo\v citateln\'a v \v case $O(1)$.\newline 
V nejhor\v s\'\i m p\v r\'\i pad\v e operace vy\v zaduj\'\i\ \v cas $
O(|S|)$ 
(v\v sechny prv\-ky jsou v jednom seznamu). \newline 
Po\v zadovan\'a pam\v e\v tov\'a n\'aro\v cnost $O(m+|S|)$ 
(p\v redpokl\'ad\'ame, \v ze reprezentace prvku $s\in S$ vy\v zaduje pam\v e\v t $
O(1)$)\newline 
pam\v e\v t nen\'\i\ efektivn\v e vyu\v zit\'a
\smallskip

\flushpar Spo\v c\'\i t\'ame o\v cek\'avanou d\'elku \v ret\v ezc\accent23u za 
p\v redpoklad\accent23u
\roster
\item 
$h$ je rychle spo\v citateln\'a (tj. $O(1)$) a nem\v enn\'a b\v ehem 
v\'ypo\v ctu;
\item
$h$ rozd\v eluje univerzum $U$ rovnom\v ern\v e (tj. 
$-1\le |h^{-1}(i)|-|h^{-1}(j)|\le 1$ pro $i,j\in \{0,1,\dots,m-1\}$);
\item 
$S$ je n\'ahodn\v e vybran\'a z univerza $U$ (tj. pro dan\'e $
n=|S|$ 
jsou v\v sechny podmno\v zi\-ny $U$ o velikosti $n$ reprezentovanou 
mno\v zinou $S$ se stejnou pravd\v epodobnost\'\i );
\item
ka\v zd\'y prvek z $U$ m\'a stejnou pravd\v epodobnost b\'yt 
argumentem operace;
\item
velikost reprezentovan\'e mno\v ziny je v\'yrazn\v e men\v s\'\i\ ne\v z 
velikost univerza.
\endroster


\flushpar Pou\v zit\'e zna\v cen\'\i : $|S|=n$, $m=$po\v cet 
\v ret\v ezc\accent23u, $|U|=N$,\newline 
$\ell (i)=$d\'elka $i$-t\'eho \v ret\v ezce, $\alpha =\frac nm$ faktor napln\v en\'\i\ (load factor)
\smallskip

\flushpar D\accent23usledky p\v redpoklad\accent23u:
\roster
\item"{}"
$\Prob(h(x)=i)=\frac 1m$ pro v\v sechna $x\in U$ a v\v sechna 
$i=0,1,\dots,m-1$; 
\item"{}"
$\Prob(\ell (i)=l)=p_{n,l}=\binom nl(\frac 1m)^l(1-\frac 1m)^{n-l}$ 
pro v\v sechna $i=0,1,\dots,m-1$.
\endroster
\flushpar Vysv\v etlen\'\i : $i$-t\'y \v ret\v ezec m\'a d\'elku $l$, pr\'av\v e kdy\v z existuje 
pod\-mno\-\v zina $A\subseteq S$ takov\'a, \v ze $|A|=l$ (t\v echto mo\v znost\'\i\ je 
$\binom nl$), pro ka\v zd\'e $x\in A$ plat\'\i\ $h(x)=i$ (pravd\v epodobnost 
tohoto jevu je $(\frac 1m)^l$) a pro ka\v zd\'e $x\in S\setminus 
A$ plat\'\i\ $h(x)\ne i$ 
(pravd\v epodobnost tohoto jevu je $(1-\frac 1m)^{n-l}$). To znamen\'a, \v ze jev m\'a 
binomi\'aln\'\i\ rozd\v elen\'\i.
\medskip

\subhead
O\v cek\'avan\'a d\'elka \v ret\v ezc\accent23u
\endsubhead
$$\align E(l)=&\sum_{l=0}^nlp_{n,l}=\sum_{l=0}^nl\binom nl(\frac 
1m)^l(1-\frac 1m)^{n-l}=\\
&\sum_{l=0}^nl\frac {n!}{l!(n-l)!}(\frac 1m)^l(1-\frac 1m)^{n-l}=\\
&\frac nm\sum_{l=1}^n\frac {(n-1)!}{(l-1)!(n-l)!}(\frac 1m)^{l-1}
(1-\frac 1m)^{n-l}=\\
&\frac nm\sum_{l=1}^n\binom {n-1}{l-1}(\frac 1m)^{l-1}(1-\frac 1m
)^{(n-1)-(l-1)}=\\
&\frac nm\sum_{l=0}^{n-1}\binom {n-1}l(\frac 1m)^l(1-\frac 1m)^{n
-1-l}=\\
&\frac nm(\frac 1m+1-\frac 1m)^{n-1}=\frac nm.\endalign$$
Toto je standardn\'\i\ element\'arn\'\i\ v\'ypo\v cet 
o\v cek\'avan\'e hodnoty binomi\'aln\'\i ho rozd\v elen\'\i.  
\medskip

\subhead
V\'ypo\v cet druh\'eho momentu
\endsubhead

$$\align E(l^2)=&E(l(l-1))+E(l),\\
E(l(l-1&))=\sum_{l=0}^nl(l-1)\binom nl(\frac 1m)^l(1-\frac 1m)^{n
-l}=\\
&\frac {n(n-1)}{m^2}\sum_{l=2}^n\binom {n-2}{l-2}(\frac 1m)^{l-2}
(1-\frac 1m)^{(n-2)-(l-2)}=\\
&\frac {n(n-1)}{m^2}\sum_{l=0}^{n-2}\binom {n-2}l(\frac 1m)^l(1-\frac 
1m)^{n-2-l}=\\
&\frac {n(n-1)}{m^2},\\
E(l^2)=&\frac {n(n-1)}{m^2}+\frac nm=\frac nm(1+\frac {n-1}m).\endalign$$
\subhead
V\'ypo\v cet rozptylu
\endsubhead

$$\align\var(l)=&E(l-E(l))^2=E(l^2)-(E(l))^2=\\
&\frac nm(1+\frac {n-1}m)-(\frac nm)^2=\frac nm(1-\frac 1m).\endalign$$
\flushpar Shrneme v\'ysledky:\newline 
O\v cek\'avan\'a d\'elka \v ret\v ezc\accent23u je $\frac nm$ a rozptyl d\'elky 
\v ret\v ezc\accent23u je $\frac nm(1-\frac 1m)$.
Toto jsou standardn\'\i\ element\'arn\'\i\ odvozen\'\i\ druh\'eho momentu
a rozptylu binomi\'aln\'\i ho rozd\v elen\'\i.
\medskip

\subhead
O\v cek\'avan\'y nejhor\v s\'\i\ p\v r\'\i pad
\endsubhead
\smallskip

\flushpar Spo\v c\'\i t\'ame $E(NP)$ o\v cek\'avanou d\'elku maxim\'aln\'\i ho 
\v ret\v ezce.\newline 
O\-zna\v c\-me $\ell (i)$ d\'elku $i$-t\'eho \v ret\v ezce. Pak
$$\Prob(\max_i\ell (i)=j)=\Prob(\max_i\ell (i)\ge j)-\Prob(\max_i
\ell (i)\ge j+1).$$

\flushpar Pak m\accent23u\v zeme po\v c\'\i tat:
$$\align E(NP)=&\sum_jj\Prob(\max_i\ell (i)=j)=\\
&\sum_jj(\Prob(\max_i\ell (i)\ge j)-\Prob(\max_i\ell (i)\ge j+1))
=\\
&\sum_jj\Prob(\max_i\ell (i)\ge j)-\sum_jj\Prob(\max_i\ell (i)\ge 
j+1)=\\
&\sum_jj\Prob(\max_i\ell (i)\ge j)-\sum_j(j-1)\Prob(\max_i\ell (i
)\ge j)=\\
&\sum_j(j-j+1)\Prob(\max_i\ell (i)\ge j)=\\
&\sum_j\Prob(\max_i\ell (i)\ge j).\endalign$$
Vysv\v etlen\'\i : P\v ri \v ctvrt\'e rovnosti se v druh\'e sum\v e zv\v et\v sil 
index, p\v res kter\'y s\v c\'\i t\'ame, o $1$, v p\'at\'e rovnosti se k sob\v e 
daly koeficienty p\v ri stejn\'ych pravd\v epodobnostech ve dvou 
sum\'ach. 

\flushpar Odtud
$$\align\Prob(\max_i&(\ell (i))\ge j)=\\
&\Prob(\ell (1)\ge j\vee\ell (2)\ge j\vee\dots\vee\ell (m-1)\ge j
)\le\\
&\sum_i\Prob(\ell (i)\ge j)\le m\binom nj(\frac 1m)^j=\\
&\frac {\prod_{k=0}^{j-1}(n-k)}{j!}(\frac 1m)^{j-1}\le n(\frac nm
)^{j-1}\frac 1{j!}.\endalign$$
Vysv\v etlen\'\i :  Prvn\'\i\ nerovnost plyne z toho, \v ze 
pravd\v epodobnost disjunkce jev\accent23u je men\v s\'\i\ ne\v z 
sou\v cet pravd\v epodobnost\'\i\ jev\accent23u, dru\-h\'a nerovnost 
plyne z toho, \v ze $i$-t\'y \v ret\v ezec m\'a d\'elku alespo\v n $
j$, jakmile existuje 
podmno\v zina $A\subseteq S$ takov\'a, \v ze $|A|=j$ (t\v echto mo\v znost\'\i\ je 
$\binom nj$) a pro ka\v zd\'e $x\in A$ plat\'\i\ $h(x)=i$ (pravd\v epo\-dob\-nost 
tohoto jevu je $(\frac 1m)^j$). 
\proclaim{D\accent23usledek}
$$\Prob(\max_i(\ell (i))\ge j)\le\min\{1,n(\frac nm)^{j-1}\frac 1{
j!}\}.$$
\endproclaim

\flushpar P\v redpoklad: $\alpha =\frac nm\le 1$. Uk\'a\v zeme, \v ze pro dostate\v cn\v e 
velk\'a $n$ pro $$j_0=\min\{j\mid n(\frac nm)^{j-1}\frac 1{j!}\le 1\}$$  
plat\'\i\ $j_0\le\frac {8\log n}{\log\log n}$.
Z $\frac nm\le 1$ a z $(\frac j2)^{\frac j2}\le j!$ plyne
$$\min\{j\mid n(\frac nm)^{j-1}\frac 1{j!}\le 1\}\le\min\{
j\mid\frac n{j!}\le 1\}\le\min\{j\mid n\le (\frac j2)^{\frac j2}\}$$
pro ka\v zd\'e $n\ge 1$, kde $j$ prob\'\i h\'a p\v rirozen\'a \v c\'\i sla. Pro pevn\'e 
$n$ ozna\v cme $$k+1=\min\{j\mid n\le (\frac j2)^{\frac j2}\},$$ 
pak 
$$(\frac k2)^{\frac k2}<n\le (\frac {k+1}2)^{\frac {k+1}2}.$$
Nyn\'\i\ toto dvakr\'at zlogaritmujeme a proto\v ze logaritmus  
je rostouc\'\i\ funkce, dost\'av\'a\-me 
$$\gather(\frac k2)\log(\frac k2)<\log n\le\frac {k+1}2\log(\frac {
k+1}2)\\\log(\frac k2)+\log\log(\frac k2)<
\log\log n\le\log(\frac {k+1}2)+\log\log(\frac {k+1}2).\endgather$$
Za p\v redpokladu, \v ze $k\ge 3$, tak dost\'av\'ame, \v ze $\log\log 
n<2\log(\frac {k+1}2)$, 
a odtud plyne 
$$\frac k8<\frac k4\frac {\log(\frac k2)}{\log(\frac {k+1}2)}<\frac {\log 
n}{\log\log n},$$
proto\v ze pro $k\ge 3$ je 
$\frac 12<\frac {\log(\frac k2)}{\log(\frac {k+1}2)}$.
P\v ri sofistikovan\v ej\v s\'\i\ metod\v e, kdy\v z se pou\v zije Stirlingova 
aproximace $\log j!$, lze dok\'azat, \v ze $j_0<(1+a_j)\frac {\log 
n}{\log\log n}$, kde 
$\lim_{j\mapsto\infty}a_j=0$.

\flushpar Toto pou\v zijeme p\v ri odhadu $E(NP)$.
$$\align E(NP)=&\sum_j\Prob(\max_i(\ell (i))\ge j)\le\\
&\sum_j\min\{1,n(\frac nm)^{j-1}\frac 1{j!}\}=\\
&\sum_{j=1}^{j_0}1+\sum_{j=j_0+1}^{\infty}\big(n(\frac nm)^{j-1}\frac 
1{j!}\big)\le j_0+\sum_{j=j_0+1}^{\infty}\frac n{j!}=\\
&j_0+\frac n{j_0!}\sum_{j=j_0+1}^{\infty}\frac {j_0!}{j!}\le j_0+
\sum_{j=j_0+1}(\frac 1{j_0+1})^{j-j_0}=\\
&j_0+\frac {\frac 1{j_0+1}}{-\frac 1{j_0+1}+1}=j_0+\frac 1{j_0}=O
(j_0).\endalign$$
Vysv\v etlen\'\i : P\v ri druh\'e nerovnosti jsme pou\v zili, \v ze $\frac 
nm\le 1$, 
p\v ri t\v ret\'\i\ nerovnosti jsme pou\v zili, \v ze $\frac n{j_
0!}\le 1$ a 
$$\frac {j_0!}{j!}=\frac 1{\prod_{k=j_{0+1}}^jk}\le (\frac 1{j_0+
1})^{j-j_0}.$$
Shrneme z\'\i skan\'y v\'ysledek

\proclaim{V\v eta}Za p\v redpokladu $\alpha =\frac nm\le 1$ je p\v ri ha\v sov\'an\'\i\ 
se se\-parovan\'ymi \v ret\v ezci horn\'\i\ odhad o\v cek\'avan\'e d\'elky 
maxim\'aln\'\i ho \v re\-t\v ez\-ce $O(\frac {\log n}{\log\log n}
)$.  

\flushpar Kdy\v z $0.5\le\alpha\le 1$, je to z\'arove\v n i doln\'\i\ odhad.
\endproclaim

\subhead
O\v cek\'avan\'y po\v cet test\accent23u
\endsubhead
\smallskip

\flushpar Test je porovn\'an\'\i\ argumentu operace s 
prvkem na dan\'em m\'\i st\v e \v ret\v ezce nebo zji\v st\v e\-n\'\i , 
\v ze vy\v set\v rovan\'y \v ret\v ezec je pr\'azdn\'y.
\medskip

\flushpar Budeme rozli\v sovat dva p\v r\'\i pady:\newline 
\phantom{---}\emph{\'usp\v e\v sn\'e} \emph{vyhled\'av\'an\'\i} -- argument 
operace je mezi prvky reprezentovan\'e mno\v ziny,\newline
\phantom{---}\emph{ne\'usp\v e\v sn\'e} \emph{vyhled\'av\'an\'\i} -- argument 
operace nen\'\i\ mezi prv\-ky reprezentovan\'e mno\-\v ziny.
\medskip

\subhead
Ne\'usp\v e\v sn\'e vyhled\'av\'an\'\i
\endsubhead
\smallskip

\flushpar O\v cek\'avan\'y po\v cet test\accent23u:
$$\align E(T)=&\Prob(\ell (i)=0)+\sum_ll\Prob(\ell (i)=l)=\\
&p_{n,0}+\sum_llp_{n,l}=\\
&(1-\frac 1m)^n+\frac nm\approx e^{-\alpha}+\alpha .\endalign$$
Vysv\v etlen\'\i : Zji\v st\v en\'\i , zda \v ret\v ezec je pr\'azdn\'y, vy\v zaduje 
jeden test, tj. $\Prob(\ell (i)=0)$ nen\'\i\ s koeficientem $0$, ale $
1$.
Proto\v ze pravd\v epodobnosti jsou stejn\'e pro v\v sechny 
\v ret\v ezce, nemus\'\i me specifikovat \v ret\v ezec, kter\'y 
vy\v set\v rujeme, sta\v c\'\i\ ps\'at obecn\v e $i$. $\sum_llp_{
n,l}$ jsme spo\v c\'\i tali 
p\v ri v\'ypo\v ctu o\v cek\'avan\'e d\'elky \v ret\v ezce.
\medskip

\subhead
\'Usp\v e\v sn\'e vyhled\'av\'an\'\i
\endsubhead
\smallskip

\flushpar Zvolme jeden \v ret\v ezec prvk\accent23u o d\'elce $l$. 
Po\v cet test\accent23u p\v ri vyhled\'an\'\i\ v\v sech prvk\accent23u 
v tomto \v ret\v ezci je
$$1+2+\dots+l=\binom {l+1}2.$$
\medskip

\flushpar O\v cek\'avan\'y po\v cet test\accent23u p\v ri vyhled\'an\'\i\ v\v sech 
prvk\accent23u v n\v ejak\'em \v ret\v ezci je 
$$\sum_l\binom {l+1}2\Prob(\ell (i)=l)=\sum_l\binom {l+1}2p_{n,l}.$$
\medskip

\flushpar O\v cek\'an\'y po\v cet test\accent23u p\v ri vyhled\'an\'\i\ v\v sech 
prvk\accent23u v tabulce je $m\sum_l\binom {l+1}2p_{n,l}$.
\medskip

\flushpar O\v cek\'avan\'y po\v cet test\accent23u pro 
vyhled\'an\'\i\ jednoho prvku je 
$$\align\frac mn\sum_{l=0}^n\binom {l+1}2p_{n,l}=&\frac m{2n}\big
(\sum_{l=0}^nl^2p_{n,l}+\sum_{l=0}^nlp_{n,l}\big)=\\
&\frac m{2n}\big(\sum_{l=1}^nl(l-1)p_{n,l}+2\sum_{l=1}^nlp_{n,l}\big
)=\\
&\frac m{2n}(\frac {n(n-1)}{m^2}+\frac {2n}m)=\frac {n-1}{2m}+1\approx\\
&1+\frac {\alpha}2.\endalign$$

\flushpar Jin\'y postup: Po\v cet test\accent23u p\v ri \'usp\v e\v sn\'em 
vyhled\'av\'an\'\i\ prvku $x\in S$ je $1+$po\v cet porovn\'an\'\i\ 
kl\'\i\v c\accent23u p\v ri ne\'usp\v e\v sn\'em vyhled\'av\'an\'\i\ $
x$ v operaci 
{\bf INSERT$(x)$}. Po\v cet porovn\'an\'\i\ kl\'\i\v c\accent23u je d\'elka 
\v ret\v ezce, a proto o\v cek\'ava\-n\'y po\v cet porovn\'an\'\i\ kl\'\i\v c\accent23u 
je o\v cek\'ava\-n\'a d\'elka \v ret\v ezce. Tedy o\v cek\'avan\'y po\v cet 
test\accent23u p\v ri \'usp\v e\v sn\'em vyhled\'av\'an\'\i\ $x$ je  
$1+$o\v cek\'avan\'a d\'elka \v ret\v ezce v okam\v ziku vkl\'ad\'an\'\i\ $
x$, neboli 
$$\frac 1n\sum_{i=0}^{n-1}(1+\frac im)=1+\frac {n-1}{2m}.$$

\proclaim{V\v eta}P\v ri ha\v sov\'an\'\i\ se separovan\'ymi \v ret\v ezci je 
o\v cek\'avan\'y po\-\v cet test\accent23u p\v ri ne\'uspe\v s\-n\'em 
vyhled\'avan\'\i\ p\v ribli\v zn\v e $e^{-\alpha}+\alpha$ a p\v ri \'us\-p\v e\v s\-n\'em 
vyhled\'av\'an\'\i\ p\v ribli\v zn\v e $1+\frac {\alpha}2$.
\endproclaim

\flushpar N\'asleduj\'\i c\'\i\ tabulka d\'av\'a p\v rehled o\v cek\'avan\'eho 
po\v ctu test\accent23u pro r\accent23uzn\'e hodnoty $\alpha$
$$\vtop{\hbox{\vbox {\offinterlineskip\halign{\strut\vrule\ # & \vrule\ # & \vrule\ # & \vrule\ # & \vrule\ # & \vrule\ # & \vrule\ # & \vrule\hfill # \hfill \vrule\cr\noalign{\hrule} $
\alpha$ & 0 & 0.1 & 0.2 & 0.3 & 0.4 & 0.5 & 0.6 \cr\noalign{\hrule}ne\'usp. vyh. & 1 & 1.005 & 1.019 & 1.041 & 1.07 & 1.107 & 1.149 \cr\noalign{\hrule} \'usp\v e\v s. vyh. & 1 & 1.05 & 1.1 & 1.15 & 1.2 & 1.25 & 1.3 \cr\noalign{\hrule}}}}\hbox{\vbox {\offinterlineskip\halign{\strut\vrule\ # & \vrule\ # & \vrule\ # & \vrule\ # & \vrule\ # & \vrule\ # & \vrule\hfill # \hfill \vrule\cr\noalign{\hrule} $
\alpha$ & 0.7 & 0.8 & 0.9 & 1 & 2 & 3 \cr\noalign{\hrule}ne\'usp. vyh. & 1.196 & 1.249 & 1.307 & 1.368 & 2.135 & 3.05 \cr\noalign{\hrule}\'usp\v e\v s. vyh. & 1.35 & 1.4 & 1.45 & 1.5 & 2 & 2.5 \cr\noalign{\hrule}}}}}$$

\flushpar V\v simn\v ete si, \v ze o\v cek\'avan\'y po\v cet test\accent23u 
p\v ri ne\'usp\v e\v sn\'em vy\-hled\'av\'an\'\i\ je men\v s\'\i\ ne\v z o\v cek\'avan\'y 
po\v cet test\accent23u p\v ri \'usp\v e\v sn\'em vy\-hled\'av\'an\'\i , kdy\v z 
$\alpha\le 1$. Na prvn\'\i\ pohled vypad\'a tento v\'ysledek nesmysln\v e, 
ale d\accent23uvod je, \v ze po\v cet test\accent23u p\v ri 
\'usp\v e\v sn\'em vy\-hled\'av\'an\'\i\ pr\accent23um\v erujeme proti $
n$, 
kde\v zto p\v ri ne\'usp\v e\v sn\'em vy\-hled\'av\'an\'\i\ proti $
m$. Ilustrujeme 
to na n\'asleduj\'\i c\'\i m p\v r\'\i kladu:\newline 
Nech\v t $n=\frac m2$ a nech\v t polovina nepr\'azdn\'ych 
\v ret\v ezc\accent23u m\'a d\'elku $1$ a polovina m\'a d\'elku $
2$. \newline 
O\v cek\'avan\'y po\v cet test\accent23u p\v ri ne\'usp\v e\v sn\'em 
vyhled\'avan\'\i :
\roster
\item"{}"
$1$ test pro pr\'azdn\'e \v ret\v ezce a \v ret\v ezce 
d\'elky $1$ -- t\v echto p\v r\'\i pad\accent23u je $\frac {5m}6$
\item"{}"
$2$ testy pro \v ret\v ezce d\'elky $2$ -- t\v echto p\v r\'\i pad\accent23u je 
$\frac m6$.
\endroster
O\v cek\'avan\'y po\v cet test\accent23u je 
$\frac 1m(1\frac {5m}6+2\frac m6)=\frac 76$.\newline 
O\v cek\'avan\'y po\v cet test\accent23u p\v ri \'usp\v e\v sn\'em 
vyhled\'av\'an\'\i: 
\roster
\item"{}"
$1$ test pro prvky na prvn\'\i m m\'\i st\v e \v ret\v ezce -- t\v echto 
p\v r\'\i pad\accent23u je $\frac {2n}3$
\item"{}"
$2$ testy pro prvky, kter\'e jsou na druh\'em m\'\i st\v e \v ret\v ezce -- t\v echto 
p\v r\'\i pad\accent23u je $\frac n3$.
\endroster
O\v cek\'avan\'y po\v cet test\accent23u je $\frac 1n(1\frac {2n}
3+2\frac n3)=\frac 43$.

\flushpar Velikost $\alpha$ je doporu\v cov\'ana men\v s\'\i\ ne\v z $
1$, ale nem\'a b\'yt 
hodn\v e mal\'a, proto\v ze by pam\v e\v t nebyla efektivn\v e vyu\v zita. 


\heading
Ha\v sov\'an\'\i\ s uspo\v r\'adan\'ymi separovan\'ymi \v ret\v ezci
\endheading

\flushpar Vylep\v sen\'\i\ metody: ha\v sov\'an\'\i\ s uspo\v r\'adan\'ymi \v ret\v ezci.
Rozd\'\i l proti p\accent23uvodn\'\i\ metod\v e -- \v ret\v ezce jsou 
uspo\v r\'adan\'e ve vzr\accent23usta\-j\'\i c\'\i m po\v rad\'\i . Proto\v ze 
\v ret\v ezce obsahuj\'\i\ tyt\'e\v z prvky, je po\v cet o\v cek\'a\-va\-n\'ych test\accent23u 
p\v ri \'usp\v e\v sn\'em vyhled\'av\'an\'\i\ stejn\'y jako u ne\-uspo\v r\'adan\'ych 
\v ret\v ezc\accent23u. P\v ri ne\'usp\v e\v sn\'em vyhled\'av\'an\'\i\ kon\v c\'\i me, 
kdy\v z argument operace je men\v s\'\i\ ne\v z vy\v set\v rovan\'y prvek 
v \v ret\v ez\-ci, tedy kon\v c\'\i me d\v r\'\i v. N\'asleduj\'\i c\'\i\ v\v eta (bez 
d\accent23ukazu) uv\'ad\'\i\ o\v cekavan\'y po\v cet test\accent23u v 
ne\'usp\v e\v sn\'em p\v r\'\i pad\v e.

\proclaim{V\v eta}O\v cek\'avan\'y po\v cet test\accent23u p\v ri 
ne\'usp\v e\v sn\'em vyhled\'av\'an\'\i\ pro ha\v sov\'an\'\i\ s uspo\v r\'a\-dan\'ymi 
\v ret\v ezci je p\v ribli\v zn\v e $e^{-\alpha}+1+\frac {\alpha}
2-\frac 1{\alpha}(1-e^{-\alpha})$. O\v cek\'avan\'y 
po\v cet test\accent23u p\v ri \'usp\v e\v sn\'em vyhled\'av\'an\'\i\ pro 
ha\v sov\'an\'\i\ s uspo\v r\'adan\'ymi \v ret\v ezci je 
p\v ribli\v zn\v e $1+\frac {\alpha}2.$
\endproclaim

\flushpar Uvedeme algoritmy pro operace s uspo\v r\'adan\'ymi 
\v re\-t\v ez\-ci.
\medskip

\subhead
Algoritmy
\endsubhead
\smallskip

\flushpar {\bf MEMBER$(x)$}:\newline 
Spo\v c\'\i t\'ame $i:=h(x)$, $t:=NIL$\newline 
{\bf if} $i$-t\'y seznam je nepr\'azdn\'y {\bf then\newline 
\phantom{{\rm ---}}$t:=$}prvn\'\i\ prvek $i$-t\'eho seznamu\newline 
\phantom{---}{\bf while} $t<x$ a $t\ne$posledn\'\i\ prvek $i$-t\'eho seznamu {\bf do\newline 
\phantom{{\rm ------}}$t:=$}n\'asleduj\'\i c\'\i\ prvek $i$-t\'eho seznamu\newline
\phantom{---}{\bf enddo\newline 
endif \newline 
if} $t=x$ {\bf then} $x\in S$ {\bf else} $x\notin S$ {\bf endif
\medskip

\flushpar INSERT$(x)$}:\newline 
Spo\v c\'\i t\'ame $i:=h(x)$, $t:=NIL$\newline 
{\bf if} $i$-t\'y seznam je nepr\'azdn\'y {\bf then\newline 
\phantom{{\rm ---}}$t:=$}prvn\'\i\ prvek $i$-t\'eho seznamu\newline 
\phantom{---}{\bf while} $t<x$ a $t\ne$posledn\'\i\ prvek $i$-t\'eho seznamu {\bf do\newline 
\phantom{{\rm ------}}$t:=$}n\'asleduj\'\i c\'\i\ prvek $i$-t\'eho seznamu\newline 
\phantom{---}{\bf enddo\newline 
endif \newline 
if} $t\ne x$ {\bf then\newline 
\phantom{{\rm ---}}if} $x<t$ {\bf then\newline 
\phantom{{\rm ------}}}vlo\v z\'\i me $x$ do $i$-t\'eho seznamu p\v red prvek $
t$\newline 
\phantom{---}{\bf else\newline} 
\phantom{------}vlo\v z\'\i me $x$ do $i$-t\'eho seznamu za prvek $
t$\newline 
\phantom{---}{\bf endif\newline 
endif
\medskip

\flushpar DELETE$(x)$}:\newline 
Spo\v c\'\i t\'ame $i:=h(x)$, $t:=NIL$
{\bf if} $i$-t\'y seznam je nepr\'azdn\'y {\bf then\newline 
\phantom{{\rm ---}}$t:=$}prvn\'\i\ prvek $i$-t\'eho seznamu\newline 
\phantom{---}{\bf while} $t<x$ a $t\ne$posledn\'\i\ prvek $i$-t\'eho seznamu {\bf do\newline 
\phantom{{\rm ------}}$t:=$}n\'asleduj\'\i c\'\i\ prvek $i$-t\'eho seznamu\newline 
\phantom{---}{\bf enddo\newline 
endif\newline 
if} $t=x$ {\bf then} odstran\'\i me $x$ z $i$-t\'eho seznamu {\bf endif}
\bigskip

\flushpar Nev\'yhody ha\v sovan\'\i\ se separovan\'ymi \v ret\v ezci --\newline 
\phantom{---}nevyu\v zit\'\i\ alokovan\'e pam\v eti (nehospod\'arn\'e)\newline 
\phantom{---}pou\v z\'\i v\'an\'\i\ ukazatel\accent23u (cache).\newline 

\flushpar\v Re\v sen\'\i : vyu\v z\'\i t pro \v ret\v ezce p\accent23uvodn\'\i\ tabulku. 
\medskip
\flushpar Polo\v zky tabulky:\newline 
\phantom{---}key,\newline 
\phantom{---}odkaz na ulo\v zen\'a data,\newline 
\phantom{---}polo\v zky pro pr\'aci s tabulkou.
\medskip

\flushpar P\v redpokl\'ad\'ame, \v ze data jsou velk\'a, v tom p\v r\'\i pad\v e se 
ukl\'adaj\'\i\ mimo tabulku. V tabulce je jen odkaz na ulo\v zen\'a data. 
P\v ri popisu pr\'ace s tabulkou tuto \v c\'ast budeme vynech\'avat (tj. 
data budou pouze kl\'\i\v c).
\medskip

\flushpar Podle \v re\v sen\'\i\ kolize d\v el\'\i me d\'al ha\v sov\'an\'\i :
\roster
\item"{}"
ha\v sov\'an\'\i\ s p\v rem\'\i s\v tov\'an\'\i m, ha\v sov\'an\'\i\ s dv\v ema 
ukazateli, 
\item"{}"
sr\accent23ustaj\'\i c\'\i\ ha\v sov\'an\'\i,
\item"{}"
dvojit\'e ha\v sov\'an\'\i\ a ha\v sov\'an\'\i\ s line\'arn\'\i m 
p\v rid\'av\'an\'\i m.
\endroster 

\heading
Ha\v sov\'an\'\i\ s p\v rem\'\i s\v tov\'an\'\i m
\endheading

\flushpar Polo\v zky pro pr\'aci s tabulkou: next, previous\newline 
\phantom{---}polo\v zka next -- \v c\'\i slo \v r\'adku tabulky 
obsahuj\'\i c\'\i\ n\'asleduj\'\i c\'\i\ polo\v z\-ku seznamu\newline 
\phantom{---}polo\v zka previous -- \v c\'\i slo \v r\'adku tabulky obsahuj\'\i c\'\i\ 
p\v redch\'azej\'\i\ polo\v zku seznamu.
\medskip

\flushpar P\v r\'\i klad: $U=\{1,2,\dots,1000\}$, $h(x)=x\bmod10$,\newline 
ulo\v zen\'a mno\v zina $S=\{1,7,11,53,73,141,161\}$,\newline 
\v ret\v ezce: $P(1)=(1,141,11,161)$, 
$P(3)=(73,53)$, $P(7)=(7)$.\newline 
Ha\v sovac\'\i\ tabulka:
$$\vtop{\offinterlineskip\halign {\strut\vrule\ # & \vrule\ # & \vrule\ # & \vrule \hfil # \hfil \vrule\cr\noalign{\hrule}\v r\'adek & key & next & previous\hfil \cr\noalign{\hrule}P(0) & & & \hfill \cr\noalign{\hrule}P(1) & 1 & 9 & \hfill \cr\noalign{\hrule}P(2) & & & \hfill \cr\noalign{\hrule}P(3) & 73 & 6 & \hfill \cr\noalign{\hrule}P(4) & & &\hfill \cr\noalign{\hrule}P(5) & 161 & & 8 \cr\noalign{\hrule}P(6) & 53 & & 3 \cr\noalign{\hrule}P(7) & 7 & &\hfill \cr\noalign{\hrule}P(8) & 11 & 5 & 9 \cr\noalign{\hrule}P(9) & 141 & 8 & 1 \cr\noalign{\hrule}}}$$
\flushpar Tabulka vznikla n\'asleduj\'\i c\'\i\ posloupnost\'\i\ 
operac\'\i :\newline 
{\bf INSERT$(1)$}, {\bf INSERT$(141)$}, {\bf INSERT$(11)$}, {\bf INSERT$(73)$}, 
{\bf INSERT$(53)$},\newline {\bf INSERT$(7)$}, {\bf INSERT$(161)$}. 
\medskip

\subhead
Algoritmy
\endsubhead
\smallskip

\flushpar {\bf MEMBER$(x)$}:\newline 
Spo\v c\'\i t\'ame $i:=h(x)$\newline 
{\bf if} $i.previous\ne$pr\'azdn\'e nebo $i.key=$pr\'azdn\'e {\bf then V\'ystup}: $
x\notin S$, stop {\bf endif \newline 
while} $i.next\ne$pr\'azdn\'e a $i.key\ne x$ {\bf do} $i:=i.next$ {\bf enddo}\newline 
{\bf if} $i.key=x$ {\bf then V\'ystup}: $x\in S$ {\bf else V\'ystup}: $
x\notin S$ {\bf endif
\medskip

\flushpar DELETE$(x)$}:\newline 
Spo\v c\'\i t\'ame $i:=h(x)$\newline 
{\bf if} $i.previous\ne$pr\'azdn\'e nebo $i.key=$pr\'azdn\'e {\bf then} stop {\bf endif\newline 
while} $i.next\ne$pr\'azdn\'e a $i.key\ne x$ {\bf do} $i:=i.next$ {\bf enddo \newline 
if} $i.key=x$ {\bf then\newline} 
\phantom{---}{\bf if} $i.previous\ne$pr\'azdn\'e {\bf then\newline}
\phantom{------}$(i.previous).next:=i.next$\newline 
\phantom{------}{\bf if} $i.next\ne$pr\'azdn\'e {\bf then} $(i.n
ext).previous:=i.previous$ {\bf endif\newline} 
\phantom{------}$i.key:=i.next:=i.previous:=$ pr\'azdn\'e \newline 
\phantom{---}{\bf else}\newline 
\phantom{------}{\bf if} $i.next\ne$pr\'azdn\'e {\bf then\newline}
\phantom{---------}$i.key:=(i.next).key$, $i.next:=(i.next).next$\newline 
\phantom{---------}{\bf if} $((i.next).next)\ne$pr\'azdn\'e {\bf then} $
((i.next).next).previous:=i$ {\bf endif \newline} 
\phantom{---------}$(i.next).key:=(i.next).next:=(i.next).previous:=$ pr\'azdn\'e \newline 
\phantom{------}{\bf else}\newline 
\phantom{---------}$i.key:=$ pr\'azdn\'e \newline 
\phantom{------}{\bf endif}\newline 
\phantom{---}{\bf endif\newline 
endif
\medskip

\flushpar INSERT$(x)$}:\newline 
Spo\v c\'\i t\'ame $i:=h(x)$\newline 
{\bf if} $i.previous\ne NIL$ nebo $i.key= NIL$ {\bf then}\newline
\phantom{---}{\bf if} $i.key=NIL$ {\bf then}\newline 
\phantom{------}$i.key:=x$\newline 
\phantom{---}{\bf else}\newline 
\phantom{------}{\bf if} neexistuje pr\'azdn\'y \v r\'adek tabulky {\bf then}\newline 
\phantom{---------}{\bf V\'ystup}: p\v repln\v en\'\i , stop\newline 
\phantom{------}{\bf else}\newline 
\phantom{---------}nech\v t $j$ je voln\'y \v r\'adek tabulky\newline 
\phantom{---------}$j.key:=i.key$, $j.previous:=i.previous$, $j.next
:=i.next$\newline 
\phantom{---------}$(j.previous).next:=j$\newline 
\phantom{---------}{\bf if} $j.next\ne NIL$ {\bf then} $(j.next).previous:=j$ {\bf endif}\newline 
\phantom{---------}$i,key:=x$, $i.next:=i.previous:=$pr\'azdn\'e\newline 
\phantom{------}{\bf endif}\newline 
\phantom{---}{\bf endif}\newline
\phantom{---}stop\newline
{\bf endif\newline
while} $i.next\ne NIL$ a $i.key\ne x$ {\bf do} $i:=i.next$ {\bf enddo\newline 
if} $i.key\ne x$ {\bf then}\newline 
\phantom{---}{\bf if} neexistuje pr\'azdn\'y \v r\'adek tabulky {\bf then}\newline 
\phantom{------}{\bf V\'ystup}: p\v repln\v en\'\i , stop\newline 
\phantom{---}{\bf else}\newline 
\phantom{------}nech\v t $j$ je voln\'y \v r\'adek tabulky\newline 
\phantom{------}$i.next:=j$, $j.key:=x$, $j.previous:=i$, stop\newline 
\phantom{---}{\bf endif\newline 
endif\newline 
endif}
\bigskip

\flushpar V p\v r\'\i kladu provedeme {\bf INSERT$(28)$}, nov\'y \v r\'adek je 4. \v r\'adek\newline 
-- v\'ysledn\'a ha\v sovac\'\i\ tabulka
$$\vtop{\offinterlineskip\halign {\strut\vrule\ # & \vrule\ # & \vrule\ # & \vrule \hfil # \hfil \vrule\cr\noalign{\hrule}\v r\'adek & key & next & previous\hfil \cr\noalign{\hrule}P(0) & & &\hfill \cr\noalign{\hrule}P(1) & 1 & 9 &\hfill \cr\noalign{\hrule}P(2) & & &\hfill \cr\noalign{\hrule}P(3) & 73 & 6 &\hfill \cr\noalign{\hrule}P(4) & 11 & 5 & 9 \cr\noalign{\hrule}P(5) & 161 & & 4 \cr\noalign{\hrule}P(6) & 53 & & 3 \cr\noalign{\hrule}P(7) & 7 & &\hfill \cr\noalign{\hrule}P(8) & 28 &  &\hfill  \cr\noalign{\hrule}P(9) & 141 & 4 & 1 \cr\noalign{\hrule}}}$$

\flushpar O\v cek\'avan\'y po\v cet test\accent23u je stejn\'y jako pro 
ha\v sov\'an\'\i\ se se\-pa\-ro\-va\-n\'y\-mi \v ret\v ezci:\newline 
\phantom{---}\'usp\v e\v sn\'e vyhled\'av\'an\'\i : $\frac {n-1}{2m}+1\approx 
1+\frac {\alpha}2$\newline 
\phantom{---}ne\'usp\v e\v sn\'e vyhled\'av\'an\'\i : $(1-\frac 1m)^n+\frac nm
\approx e^{-\alpha}+\alpha$,\newline 
kde $m=$ velikost tabulky, $n=$ velikost $S$ tj. po\v cet ulo\v zen\'ych 
prvk\accent23u, $\alpha =\frac nm=$ faktor zapln\v en\'\i.

\heading
Ha\v sov\'an\'\i\ s dv\v ema ukazateli
\endheading

\flushpar Nev\'yhoda ha\v sov\'an\'\i\ s p\v rem\'\i s\v tov\'an\'\i m je krok 5) v operaci 
{\bf INSERT}. Vy\v zaduje v\'\i ce \v casu -- operace s p\v rem\'\i st\v en\'\i m 
polo\v zky. Toto odstra\v nuje dal\v s\'\i\ implementace ha\v sov\'an\'\i\ se 
separuj\'\i c\'\i mi \v ret\v ezci.
\smallskip

\flushpar Polo\v zky pro pr\'aci s tabulkou -- next, 
begin\newline 
\phantom{---}Polo\v zka next -- \v c\'\i slo \v r\'adku tabulky obsahuj\'\i c\'\i\ 
n\'asleduj\'\i c\'\i\ polo\v z\-ku seznamu\newline 
\phantom{---}Polo\v zka begin -- \v c\'\i slo \v r\'adku tabulky obsahuj\'\i c\'\i\ prvn\'\i\ polo\v zku seznamu 
s touto adresou\newline 

\flushpar Stejn\'a data jako v minul\'em p\v r\'\i pad\v e\newline 

\flushpar Ha\v sovac\'\i\ tabulka:
$$\vtop{\offinterlineskip\halign {\strut\vrule\ # & \vrule\ # & \vrule\ # & \vrule \hfil # \hfil \vrule\cr\noalign{\hrule}\v r\'adek & key & next & begin \hfil \cr\noalign{\hrule}P(0) & & & \hfill \cr\noalign{\hrule}P(1) & 1 & 9 & 1 \cr\noalign{\hrule}P(2) & & & \hfill \cr\noalign{\hrule}P(3) & 73 & 7 & 3 \cr\noalign{\hrule}P(4) & & &\hfill \cr\noalign{\hrule}P(5) & 161 & & \hfill \cr\noalign{\hrule}P(6) & 7 & & \hfill \cr\noalign{\hrule}P(7) & 53 & & 6 \cr\noalign{\hrule}P(8) & 11 & 5 & \hfill \cr\noalign{\hrule}P(9) & 141 & 8 & \hfill \cr\noalign{\hrule}}}$$

\flushpar Tabulka vznikla n\'asleduj\'\i c\'\i\ posloupnost\'\i\ 
operac\'\i :\newline 
{\bf INSERT$(1)$}, {\bf INSERT$(141)$}, {\bf INSERT$(11)$}, {\bf INSERT$
(73)$}, 
{\bf INSERT$(53)$}, {\bf INSERT$(7)$}, {\bf INSERT$(161)$}. 
\medskip

\subhead
Algoritmy
\endsubhead
\smallskip

\flushpar {\bf MEMBER$(x)$}:\newline 
Spo\v c\'\i t\'ame $i:=h(x)$\newline 
{\bf if} $i.begin=$pr\'azdn\'e {\bf then V\'ystup}: $x\notin S$, stop {\bf else} i:=i.begin {\bf endif}\newline 
{\bf while} $i.next\ne$pr\'azdn\'e a $i.key\ne x$ {\bf do} $i:=i.next$ {\bf enddo \newline 
if} $i.key=x$ {\bf then V\'ystup}: $x\in S$ {\bf else V\'ystup}: $
x\notin S$ {\bf endif
\medskip

\flushpar DELETE$(x)$}:\newline 
Spo\v c\'\i t\'ame $i:=h(x)$\newline 
{\bf if} $i.begin=$pr\'azdn\'e {\bf then} stop {\bf else} $j:=i$, $
i:=i.begin$ {\bf endif\newline 
while} $i.next\ne$pr\'azdn\'e a $i.key\ne x$ {\bf do} $j:=i$, $i:=i.next$ {\bf enddo \newline 
if} $i.key=x$ {\bf then}\newline 
\phantom{---}{\bf if} $i=j.begin$ {\bf then}\newline 
\phantom{------}{\bf if} $i.next\ne$pr\'azdn\'e {\bf then}\newline 
\phantom{---------}$j.begin:=i.next$\newline 
\phantom{------}{\bf else}\newline 
\phantom{---------}$j.begin:=$pr\'azdn\'e\newline 
\phantom{------}{\bf endif}\newline 
\phantom{---}{\bf else}\newline 
\phantom{------}$j.next:=i.next$\newline 
\phantom{---}{\bf endif}\newline 
\phantom{---}$i.key:=i.next:=$pr\'azdn\'e\newline
{\bf endif
\medskip

\flushpar INSERT$(x)$}:\newline 
Spo\v c\'\i t\'ame $i:=h(x)$\newline
{\bf if} $i.begin=$pr\'azdn\'e {\bf then}\newline
\phantom{---}{\bf if} $i.key=$pr\'azdn\'e {\bf then}\newline 
\phantom{------}$i.key:=x$, $i.begin:=i$\newline 
\phantom{---}{\bf else}\newline 
\phantom{------}{\bf if} neexistuje pr\'azdn\'y \v r\'adek tabulky {\bf then}\newline 
\phantom{---------}{\bf V\'ystup}: p\v repln\v en\'\i , stop\newline 
\phantom{------}{\bf else}\newline 
\phantom{---------}nech\v t $j$ je voln\'y \v r\'adek tabulky\newline 
\phantom{---------}$j.key=x$, $i.begin:=j$\newline 
\phantom{------}{\bf endif}\newline
\phantom{---}{\bf endif\newline 
else}\newline
\phantom{---}$i:=i.begin$
\phantom{---}{\bf while} $i.next\ne$pr\'azdn\'e a $i.key\ne x$ {\bf do} $i:=i.next$ {\bf enddo}\newline 
\phantom{---}{\bf if} $i.key\ne x$ {\bf then}\newline 
\phantom{------}{\bf if} neexistuje pr\'azdn\'y \v r\'adek tabulky {\bf then}\newline 
\phantom{---------}{\bf V\'ystup}: p\v repln\v en\'\i , stop\newline 
\phantom{------}{\bf else}\newline 
\phantom{---------}nech\v t $j$ je voln\'y \v r\'adek tabulky\newline 
\phantom{---------}$i.next:=j$, $j.key:=x$, stop\newline 
\phantom{------}{\bf endif}\newline
\phantom{---}{\bf endif\newline 
endif}
\bigskip

\flushpar V p\v r\'\i kladu provedeme {\bf INSERT$(28)$}, nov\'y \v r\'adek je 4. 
\v r\'adek\newline 
-- v\'ysledn\'a ha\v sovac\'\i\ tabulka
$$\vtop{\offinterlineskip\halign {\strut\vrule\ # & \vrule\ # & \vrule\ # & \vrule \hfil # \hfil \vrule\cr\noalign{\hrule}\v r\'adek & key & next & begin\hfil \cr\noalign{\hrule}P(0) & & &\hfill \cr\noalign{\hrule}P(1) & 1 & 9 & 1 \cr\noalign{\hrule}P(2) & & &\hfill \cr\noalign{\hrule}P(3) & 73 & 7 & 3 \cr\noalign{\hrule}P(4) & 28 &  &\hfill \cr\noalign{\hrule}P(5) & 161 & & \hfill \cr\noalign{\hrule}P(6) & 7 & &\hfill \cr\noalign{\hrule}P(7) & 53 & & 6 \cr\noalign{\hrule}P(8) & 11 & 5 & 4  \cr\noalign{\hrule}P(9) & 141 & 8 & \hfill \cr\noalign{\hrule}}}$$

\flushpar Algoritmus p\v ri pr\'aci s polo\v zkami je rychlej\v s\'\i\ ne\v z p\v ri 
ha\v sov\'an\'\i\ s p\v rem\'\i s\v tov\'an\'\i m, ale za\v c\'atek \v ret\v ezce v 
jin\'em m\'\i st\v e tabulky p\v rid\'av\'a jeden test. V\'ysledek bez 
odvozov\'an\'\i :\newline 
O\v cek\'avan\'y po\v cet test\accent23u:\newline 
\phantom{---}\'usp\v e\v sn\'y p\v r\'\i pad: $1+\frac {(n-1)(n-2)}{6m^2}+\frac {
n-1}{2m}\approx 1+\frac {\alpha^2}6+\frac {\alpha}2$\newline 
\phantom{---}ne\'usp\v e\v sn\'y p\v r\'\i pad: $\approx 1+\frac {\alpha^2}2+\alpha 
+e^{-\alpha}(2+\alpha )-2$.

\heading
Sr\accent23ustaj\'\i c\'\i\ ha\v sov\'an\'\i
\endheading

\flushpar Sr\accent23ustaj\'\i c\'\i\ ha\v sov\'an\'\i\ se d\v el\'\i\ podle pr\'ace s 
pam\v et\'\i\ na standardn\'\i\ a na sr\accent23ustaj\'\i c\'\i\ ha\v sov\'an\'\i\ 
s pomocnou pam\v et\'\i\ (kter\'e se naz\'yva jen 
sr\accent23ustaj\'\i c\'\i\ ha\v sov\'an\'\i ) a podle zp\accent23usobu 
p\v rid\'av\'an\'\i\ dal\v s\'\i ho prvku.
\medskip

\flushpar Pop\'\i\v seme metody:\newline 
\phantom{---}Standardn\'\i\ sr\accent23ustaj\'\i c\'\i\ ha\v sov\'an\'\i: LISCH, EISCH,\newline 
\phantom{---}Sr\accent23ustaj\'\i c\'\i\ ha\v sov\'an\'\i : LICH, VICH, EICH.
\medskip

\flushpar V\v sechny metody pro pr\'aci s tabulkou pou\v z\'\i vaj\'\i\ jen 
polo\v zku next -- \v c\'\i slo \v r\'adku tabulky obsahuj\'\i c\'\i\ 
n\'asleduj\'\i c\'\i\ polo\v zku seznamu. 

\flushpar Z\'akladn\'\i\ idea: \v ret\v ezec za\v c\'\i n\'a na sv\'em m\'\i st\v e, ale 
pokud u\v z tam byl ulo\v zen 
n\v ejak\'y \'udaj, pak \v ret\v ezec tohoto \'udaje sroste s \v ret\v ezcem 
za\v c\'\i naj\'\i c\'\i m na tomto \v r\'adku. To znamen\'a, \v ze prvky \v ret\v ezce, kter\'y za\v c\'\i n\'a na tomto m\'\i st\v e budou ulo\v zeny v \v ret\v ezci, kter\'y u\v z je ulo\v zen na tomto m\'\i st\v e, ale jen od tohoto m\'\i sta d\'al.
\medskip

\subhead
Metody EISCH a LISCH
\endsubhead

\phantom{---}EISCH -- early-insertion standard coalesced hashing\newline 
\phantom{---}LISCH -- late-insertion standard coalesced hashing.
 
\flushpar Organizace tabulky je stejn\'a jako v p\v redchoz\'\i ch 
p\v r\'\i padech.\newline 
Z\'akladn\'\i\ ideje: LISCH p\v rid\'av\'a nov\'y prvek na konec 
\v ret\v ezce, \newline 
EISCH p\v rid\'av\'a nov\'y prvek $x$ do \v ret\v ezce za 
\v r\'adkem $h(x)$.\newline 

\flushpar P\v r\'\i klad: $U=\{1,2,\dots,1000\}$, $h(x)=x\bmod10$\newline 
mno\v zina $S=\{1,7,11,53,73,141,171\}$ je ulo\v zena 
v ha\v sovac\'\i\ tabulce
$$\vtop{\offinterlineskip\halign {\strut\vrule\ # & \vrule\ # & \vrule \hfil # \hfil \vrule\cr\noalign{\hrule}\v r\'adek & key & next \hfil \cr\noalign{\hrule}P(0) & &\hfill \cr\noalign{\hrule}P(1) & 1 & 9 \cr\noalign{\hrule}P(2) & &\hfill \cr\noalign{\hrule}P(3) & 73 & 6 \cr\noalign{\hrule}P(4) &  &\hfill \cr\noalign{\hrule}P(5) & 7 & \hfill \cr\noalign{\hrule}P(6) & 53 &\hfill \cr\noalign{\hrule}P(7) & 161 & 5 \cr\noalign{\hrule}P(8) & 11 & 7  \cr\noalign{\hrule}P(9) & 141 & 8 \cr\noalign{\hrule}}}$$

\flushpar Tabulka pro metodu LISCH vznikla n\'asleduj\'\i c\'\i\ posloupnost\'\i\ 
operac\'\i :\newline 
{\bf INSERT$(1)$}, {\bf INSERT$(141)$}, {\bf INSERT$(11)$}, {\bf INSERT$
(73)$}, 
{\bf INSERT$(53)$},\newline {\bf INSERT$(161)$}, {\bf INSERT$(7)$}.\newline 
Pro metodu EISCH tabulka vznikla n\'asleduj\'\i c\'\i\ posloupnost\'\i\ 
operac\'\i :\newline 
{\bf INSERT$(1)$}, {\bf INSERT$(161)$}, {\bf INSERT$(11)$}, {\bf INSERT$
(73)$}, 
{\bf INSERT$(53)$}, {\bf INSERT$(7)$}, {\bf INSERT$(141)$}. 

\flushpar Provedeme {\bf INSERT$(28)$}, p\v rid\'av\'ame do \v cvrt\'eho r\'adku, v\'y\-sled\-n\'a tabulka vlevo je pro meto\-du 
LISCH, vpravo pro metodu EISCH.
$$\vbox {\hbox{\vtop{\offinterlineskip\halign{\strut\vrule\ # & \vrule\ # & \vrule\hfil# \hfil\vrule\cr\noalign{\hrule}\v r\'adek & key & next \hfil\cr\noalign{\hrule}P(0) & &\hfill \cr\noalign{\hrule}P(1) & 1 & 9 \cr\noalign{\hrule}P(2) & &\hfill \cr\noalign{\hrule}P(3) & 73 & 6 \cr\noalign{\hrule}P(4) & 28 &\hfill \cr\noalign{\hrule}P(5) & 7 & 4 \cr\noalign{\hrule}P(6) & 53 &\hfill \cr\noalign{\hrule}P(7) & 161 & 5 \cr\noalign{\hrule}P(8) & 11 & 7 \cr\noalign{\hrule}P(9) & 141 & 8 \cr\noalign{\hrule}}}\qquad\vtop{\offinterlineskip\halign {\strut\vrule\ # & \vrule\ # & \vrule \hfil # \hfil \vrule\cr\noalign{\hrule}\v r\'adek & key & next \hfil \cr\noalign{\hrule}P(0) & &\hfill \cr\noalign{\hrule}P(1) & 1 & 9 \cr\noalign{\hrule}P(2) & &\hfill \cr\noalign{\hrule}P(3) & 73 & 6 \cr\noalign{\hrule}P(4) & 28 & 7 \cr\noalign{\hrule}P(5) & 7 & \hfill \cr\noalign{\hrule}P(6) & 53 & \hfill \cr\noalign{\hrule}P(7) & 161 & 5 \cr\noalign{\hrule}P(8) & 11 & 4  \cr\noalign{\hrule}P(9) & 141 & 8 \cr\noalign{\hrule}}}}}$$
\medskip

\subhead
Algoritmy
\endsubhead
\smallskip

\flushpar Algoritmus operace {\bf MEMBER} je pro ob\v e metody stejn\'y.
\smallskip

\flushpar {\bf MEMBER$(x)$}:\newline 
Spo\v c\'\i t\'ame $i:=h(x)$\newline 
{\bf while} $i.next\ne$pr\'azdn\'e a $i.key\ne x$ {\bf do} $i:=i.next$ {\bf enddo \newline 
if} $i.key=x$ {\bf then V\'ystup}: $x\in S$ {\bf else V\'ystup}: $
x\notin S$ {\bf endif
\medskip

}\flushpar Metoda LISCH -- {\bf INSERT$(x)$}:\newline 
Spo\v c\'\i t\'ame $i:=h(x)$\newline 
{\bf while} $i.next\ne$pr\'azdn\'e a $i.key\ne x$ {\bf do} $i:=i.next$ {\bf enddo \newline 
if} $i.key\ne x$ {\bf then\newline 
\phantom{{\rm ---}}if} neexistuje pr\'azdn\'y \v r\'adek tabulky {\bf then\newline 
\phantom{{\rm ------}}V\'ystup}: p\v repln\v en\'\i , stop\newline 
\phantom{---}{\bf else\newline 
\phantom{{\rm ------}}}nech\v t $j$ je pr\'azdn\'y \v r\'adek $j.
key:=x$, 
$i.next:=j$\newline 
\phantom{---}{\bf endif\newline 
endif
\medskip

}\flushpar Metoda EISCH -- {\bf INSERT$(x)$}:\newline 
Spo\v c\'\i t\'ame $k:=i:=h(x)$\newline 
{\bf while} $i.next\ne$pr\'azdn\'e a $i.key\ne x$ {\bf do} $i:=i.next$ {\bf enddo \newline 
if} $i.key\ne x$ {\bf then\newline 
\phantom{{\rm ---}}if} neexistuje pr\'azdn\'y \v r\'adek tabulky {\bf then\newline 
\phantom{{\rm ------}}V\'ystup}: p\v repln\v en\'\i , stop\newline 
\phantom{---}{\bf else\newline 
\phantom{{\rm ------}}}nech\v t $j$ je voln\'y \v r\'adek tabulky\newline 
\phantom{------}$j.next:=k.next$, $k.next:=j$, $j.key:=x$\newline 
\phantom{---}{\bf endif\newline 
endif
\medskip

}\flushpar Efektivn\'\i\ operace {\bf DELETE}  nen\'\i\ zn\'am\'a, ale i primitivn\'\i\ 
algoritmy pro operaci {\bf DELETE} maj\'\i\ rozumnou o\v cek\'avanou 
\v casovou slo\v zitost.
\medskip

\subhead
Anal\'yza slo\v zitosti t\v echto algoritm\accent23u.
\endsubhead
\smallskip

\flushpar Popis situace: Ulo\v zena mno\v zina 
$S=\{s_1,s_2,\dots,s_n\}$ do ta\-bulky velikosti $m$, je d\'an 
prvek $s_{n+1}$ a m\'ame zjistit, zda $s_{n+1}\in S$. Ozna\v cme 
$a_i=h(s_i)$ pro $i=1,2,\dots,n+1$, kde $h$ je pou\v zit\'a ha\v sovac\'\i\ 
funkce.\newline 
P\v redpoklad: v\v sechny posloupnosti $a_1,a_2,\dots,a_{n+1}$ jsou 
{\bf stejn\v e pravd\v epodobn\'e}. V\'yb\v er pr\'azdn\'eho \v r\'adku je 
pevn\v e dan\'y, to znamen\'a, \v ze p\v ri stejn\v e obsazen\'ych 
\v radc\'\i ch dostaneme v\v zdy stejn\'y pr\'azdn\'y \v r\'adek.  
\medskip

\subhead
Ne\'usp\v e\v sn\'e vyhled\'av\'an\'\i\ ($s_{n+1}\notin S$). 
\endsubhead
\smallskip

\flushpar Ozna\v cen\'\i : $C(a_1,a_2,\dots,a_n;a_{n+1})$ ozna\v cuje po\v cet 
test\accent23u pro zji\v s\-t\v en\'\i , \v ze $s_{n+1}\notin S$. Plat\'\i : 
o\v cek\'avan\'y po\v cet test\accent23u p\v ri 
ne\'usp\v e\v sn\'em vy\-hled\'av\'an\'\i\ v mno\v zin\v e $S$ je 
$$\frac {\sum C(a_1,a_2,\dots,a_n;a_{n+1})}{m^{n+1}},$$
kde se s\v c\'\i t\'a p\v res v\v sechny posloupnosti $a_1,a_2,\dots
,a_{n+1}$ -- 
a t\v ech je $m^{n+1}$. \newline 
\v Ret\v ezec d\'elky $l$ v mno\v zin\v e $S$ je maxim\'aln\'\i\ posloupnost 
adres $(b_1,b_2,\dots,b_l)$ takov\'a, \v ze $b_i.next=b_{i+1}$ pro 
$i=1,2,\dots,l-1$.
Kdy\v z adresa $a_{n+1}$ je $i$-t\'y prvek v \v ret\v ezci, pak po\v cet 
test\accent23u je $l-i+1$. \v Ret\v ezec d\'elky $l$ p\v risp\v el k sou\v ctu 
$\sum C(a_1,a_2,\dots,a_n;a_{n+1})$ $1+2+\dots+l=l+\binom l2$ testy. 

\flushpar$c_n(l)=$ po\v cet v\v sech \v ret\v ezc\accent23u d\'elky $
l$ ve v\v sech 
reprezentac\'\i ch $n$-prvkov\'ych mno\v zin (ztoto\v z\v nuje\-me dv\v e 
mno\v ziny, kter\'e m\v ely stej\-nou posloupnost adres p\v ri 
ukl\'adan\'\i\ prvk\accent23u), pak 
$$\align\sum C(a_1,a_2,&\dots,a_n;a_{n+1})=c_n(0)+\sum_{l=1}^n(l+\binom 
l2)c_n(l)\\
&=c_n(0)+\sum_{l=1}^nlc_n(l)+\sum_{l=1}^n\binom l2c_n(l),\endalign$$
kde $c_n(0)$ je po\v cet pr\'azdn\'ych \v r\'adk\accent23u ve v\v sech 
reprezentac\'\i ch. \newline 
Reprezentace $S$ m\'a $m-n$ pr\'azdn\'ych \v r\'adk\accent23u,\newline 
v\v sech posloupnost\'\i\ $n$-adres je $m^n$, proto 
$$c_n(0)=(m-n)m^n.$$
$\sum_{l=1}^nlc_n(l)$ je celkov\'a d\'elka v\v sech \v ret\v ezc\accent23u ve 
v\v sech tabulk\'ach reprezentuj\'\i c\'\i ch v\v sechny $n$-prvkov\'e 
mno\v ziny a proto 
$$\sum_{l=1}^nlc_n(l)=nm^n.$$
Spo\v c\'\i t\'ame $S_n=\sum_{l=1}^n\binom l2c_n(l)$.  Nejprve rekurentn\'\i\ vztah 
pro $c_n(l)$.  P\v rid\'av\'ame prvek s adresou $a_{n+1}$.  Pak \v ret\v ezec 
d\'elky $l$ v reprezentaci $S$ z\accent23ustal stejn\'y, kdy\v z 
adresa $a_{n+1}$ nele\v zela v tomto \v ret\v ezci, v opa\v cn\'em p\v r\'\i pad\v e 
se d\'elka \v ret\v ezce zv\v et\v sila na $l+1$.  Proto p\v rid\'an\'\i\ jednoho 
prvku vytvo\v rilo z \v ret\v ezce d\'elky $l$ celkem $m-l$ 
\v ret\v ezc\accent23u d\'elky $l$ a $l$ \v ret\v ezc\accent23u d\'elky $
l+1$.  
Vys\v c\'\i t\'an\'\i m p\v res v\v sechny $n$-prvkov\'e posloupnosti adres 
dost\'av\'ame 
$$c_{n+1}(l)=(m-l)c_n(l)+(l-1)c_n(l-1).$$

\flushpar Odtud
$$\align S_n=&\sum_{l=1}^n\binom l2c_n(l)=\\
&\sum_{l=1}^n\big(\binom l2(m-l)c_{n-1}(l)+\binom l2(l-1)c_{n-1}(
l-1)\big)=\\
&\big(\sum_{l=1}^n\binom l2(m-l)c_{n-1}(l)\big)+\big(\sum_{l=0}^{
n-1}\binom {l+1}2lc_{n-1}(l)\big)=\\
&\binom n2(m-n)c_{n-1}(n)+\\
&\big(\sum_{l=1}^{n-1}(\binom l2(m-l)+\binom {l+1}2l)c_{n-1}(l)\big
)+\binom 120c_{n-1}(0)=\\
&\sum_{l=1}^{n-1}\binom l2(m+2)c_{n-1}(l)+\sum_{l=1}^{n-1}lc_{n-1}
(l)=\\
&(m+2)S_{n-1}+(n-1)m^{n-1},\endalign$$
kde jsme pou\v zili, \v ze $c_{n-1}(n)=0$, a identitu
$$\align(m-l)\binom l2+l\binom {l+1}2=&\frac 12(l^2m-lm-l^3+l^2+l^
3+l^2)=\\
&\frac 12(l^2m-lm+2l^2)=\\
&\frac 12(l^2m-lm+2(l^2-l))+l=\\
&(m+2)\binom l2+l.\endalign$$

\flushpar Rekurence pro $S_n$ d\'av\'a 
$$\align S_n=&(m+2)S_{n-1}+(n-1)m^{n-1}=\\
&(m+2)^2S_{n-2}+(m+2)(n-2)m^{n-2}+(n-1)m^{n-1}=\\
&(m+2)^3S_{n-3}+(m+2)^2(n-3)m^{n-3}+\\
&(m+2)(n-2)m^{n-2}+(n-1)m^{n-1}=\\
&(m+2)^{n-1}S_0+\sum_{i=0}^{n-1}(m+2)^i(n-1-i)m^{n-1-i}=\\
&(m+2)^{n-1}\sum_{i=0}^{n-1}(n-1-i)\big(\frac m{m+2}\big)^{n-1-i}=\\
&(m+2)^{n-1}\sum_{i=1}^{n-1}i\big(\frac m{m+2}\big)^i,\endalign$$
kde jsme vyu\v zili, \v ze $S_0=0$. Spo\v c\'\i t\'ame sou\v cet 
$T_c^n=\sum_{i=1}^nic^i$ pro $n=1,2,\dots$ a $c\ne 1$. Z
$cT_c^n=\sum_{i=1}^nic^{i+1}$ plyne 
$$\align(c-1)T_c^n=&cT_c^n-T_c^n=\sum_{i=2}^{n+1}(i-1)c^i-\sum_{i
=1}^nic^i=\\
&nc^{n+1}+\big(\sum_{i=2}^n((i-1)c^i-ic^i)\big)-c=\\
&nc^{n+1}+\big(\sum_{i=2}^n-c^i\big)-c=\\
&nc^{n+1}-\sum_{i=1}^nc^i=nc^{n+1}-\frac {c^{n+1}-c}{c-1}=\\
&\frac {nc^{n+2}-(n+1)c^{n+1}+c}{c-1}.\endalign$$

\flushpar Tedy plat\'\i\ 
$$T_c^n=\frac {nc^{n+2}-(n+1)c^{n+1}+c}{(c-1)^2}.$$
Proto\v ze $\frac m{m+2}\ne 1$, dost\'av\'ame, \v ze 
$$\align S_n=&(m+2)^{n-1}\frac {(n-1)\big(\frac m{m+2}\big)^{n+1}
-n\big(\frac m{m+2}\big)^n+\frac m{m+2}}{\big(\frac m{m+2}-1\big)^
2}=\\
&\frac 14(m+2)^{n+1}\big[(n-1)\big(\frac m{m+2}\big)^{n+1}-
n\big(\frac m{m+2}\big)^n+\frac m{m+2}\big]=\\
&\frac 14\big[(n-1)m^{n+1}-n(m+2)m^n+m(m+2)^n\big]=\\
&\frac 14\big(m(m+2)^n-m^{n+1}-2nm^n\big).\endalign$$
O\v cek\'avan\'y po\v cet test\accent23u p\v ri ne\'usp\v e\v sn\'em 
vyhled\'av\'an\'\i\ je  
$$\align&\frac {(m-n)m^n+nm^n+\frac 14\big(m(m+2)^n-m^{n+1}-2nm^n\big
)}{m^{n+1}}=\\
&\frac {m^{n+1}+\frac 14\big(m(m+2)^n-m^{n+1}-2nm^n\big)}{m^{n+1}}
=\\
&1+\frac 14\big((1+\frac 2m)^n-1-\frac {2n}m\big)\sim 1+\frac 14(
e^{2\alpha}-1-2\alpha ).\endalign$$
Tento odhad je stejn\'y pro ob\v e metody -- LISCH i EISCH, 
proto\v ze maj\'\i\ stejn\'e posloupnosti adres (li\v s\'\i\ se jen po\v rad\'\i m 
prvk\accent23u v jednotliv\'ych \v ret\v ezc\'\i ch).
\medskip

\subhead
\'Usp\v e\v sn\'y p\v r\'\i pad ($s_{n+1}\in S$)
\endsubhead
\smallskip

\flushpar O\v cek\'avan\'y po\v cet test\accent23u p\v ri \'usp\v e\v sn\'em 
vyhled\'av\'an\'\i\ v modelu LISCH spo\v c\'\i t\'ame stejnou meto\-dou jako 
pro ha\v sov\'an\'\i\ se separuj\'\i c\'\i mi \v ret\v ezci.  Pro vyhled\'an\'\i\ prvku 
$s_{n+1}\in S$ je po\v cet test\accent23u roven $1+$po\v cet porovn\'an\'\i\ 
kl\'\i\v c\accent23u p\v ri operaci {\bf INSERT$(s_{n+1})$.}  Kdy\v z $
s_{n+1}$ je vlo\v zen 
na m\'\i sto $h(s_{n+1})$, nebyl porovn\'av\'an \v z\'adn\'y kl\'\i\v c a test bude $
1$, 
kdy\v z $h(s_{n+1})$ byl na na $i$-t\'em m\'\i st\v e v \v ret\v ezci d\'elky $
l$, pak bylo 
p\v ri operaci {\bf INSERT$(s_{n+1})$} pou\v zito $l-i+1$ porovn\'an\'\i\ kl\'\i\v c\accent23u 
a te\v d se pou\v zije $l-i+2$ test\accent23u. Podle p\v redchoz\'\i\ \v c\'asti anal\'yzy dostaneme, \v ze o\v cek\'avan\'y po\v cet 
porovn\'an\'\i\ kl\'\i\v c\accent23u p\v ri ne\'usp\v e\v sn\'em vyhled\'av\'an\'\i\ je 
$$\align&\frac 1{m^{n+1}}(\sum_{l=1}^n(l+\binom l2)c_n(l))=\\
&\frac 1{m^{n+1}}(nm^n+\frac 14\big(m(m+2)^n-m^{n+1}-2nm^n\big))=\\
&\frac 14\big((1+\frac 2m)^n-1+\frac {2n}m\big).\endalign$$
Tedy o\v cek\'avan\'y po\v cet test\accent23u p\v ri \'usp\v e\v sn\'em 
vy\-hled\'av\'an\'\i\ v $n$-prvkov\'e mno\v zin\v e je podle 
p\v red\-cho\-\'\i\ anal\'yzy 
roven $1+n$-tina sou\v ctu o\v cek\'avan\'eho po\v ctu porovn\'an\'\i\ 
kl\'\i\v c\accent23u p\v ri ne\'usp\v e\v sn\'em vy\-hled\'av\'an\'\i\ v $
i$-prvkov\'e 
mno\v zin\v e, kde $i$ prob\'\i h\'a \v c\'\i sla $0,1,\dots,n-1$.  Podle p\v redchoz\'\i ch 
v\'ysledk\accent23u je hledan\'y sou\v cet 
$$\align\sum_{i=0}^{n-1}\frac 14\big[(1+\frac 2m)^i-&1+\frac {2i}
m\big]=\frac 14\frac {(1+\frac 2m)^n-1}{1+\frac 2m-1}-\frac n4+\frac {\binom 
n2}{2m}=\\
&\frac m8\big((1+\frac 2m)^n-1-\frac {2n}m\big)+\frac {n^2-n}{4m}
.\endalign$$
Tedy o\v cek\'avan\'y po\v cet test\accent23u v \'usp\v e\v sn\'em 
p\v r\'\i pad\v e pro $n$-prvko\-vou mno\v zinu je 
$$1+\frac m{8n}\big((1+\frac 2m)^n-1-\frac {2n}m\big)+\frac {n-1}{
4m}\sim 1+\frac 1{8\alpha}(e^{2\alpha}-1-2\alpha )+\frac {\alpha}
4.$$
Pro metodu EISCH je o\v cek\'av\'an\'y po\v cet test\accent23u v 
\'usp\v e\v sn\'em p\v r\'\i pa\-d\v e 
$$\frac mn\big((1+\frac 1m)^n-1\big)\sim\frac 1{\alpha}(e^{\alpha}
-1).$$
V\'ypo\v cet je ale komplikovan\v ej\v s\'\i\, mus\'\i\ se pou\v z\'\i t slo\v zit\v ej\v s\'\i\ metoda (metoda EISCH d\'av\'a nov\'y prvek hned za m\'\i sto, kde m\'a b\'yt ulo\v zen). Chyba aproximace 
pro tyto odhady je $O(\frac 1m)$. 
\medskip

\subhead
Metody LICH, EICH, VICH
\endsubhead

\phantom{---}LICH -- late-insertion coalesced hashing\newline 
\phantom{---}EICH -- early-insertion coalesced hashing\newline 
\phantom{---}VICH -- varied-insertion coalesced hashing.
\medskip

\flushpar Z\'akladn\'\i\ idea: Metody pou\v z\'\i vaj\'\i\ pomocnou pam\v e\v t. 
Tabulka je rozd\v elen\'a na adresovac\'\i\ \v c\'ast a na pomocnou 
pam\v e\v t, kter\'a nen\'\i\ dostupn\'a pomoc\'\i\ ha\v sovac\'\i\ funkce, 
ale pom\'ah\'a p\v ri \v re\v sen\'\i\ koliz\'\i . Metody se li\v s\'\i\ operac\'\i\ 
{\bf INSERT}. V\v sechny metody p\v ri kolizi nejprve pou\v zij\'\i\ \v r\'adek 
tabulky z pomocn\'e \v c\'asti a teprve, kdy\v z je pomocn\'a \v c\'ast 
zapln\v ena, pou\v z\'\i vaj\'\i\ adresovac\'\i\ \v c\'ast.\newline 
Metoda LICH: p\v ri {\bf INSERT}u vkl\'ad\'a prvek v\v zdy na konec 
\v re\-t\v ez\-ce.\newline 
Metoda EICH: p\v ri {\bf INSERT}u vkl\'ad\'a prvek $x$ do \v ret\v ezce v\v zdy 
na m\'\i sto hned za \v r\'adkem $h(x)$.\newline 
Metoda VICH: P\v ri {\bf INSERT}u, kdy\v z nov\'y \v r\'adek je z 
pomocn\'e \v c\'asti, tak je vlo\v zen s nov\'ym prvkem na konec 
\v ret\v ezce, kdy\v z je pomocn\'a \v c\'ast pam\v eti vy\v cer\-p\'a\-na, tak se 
\v r\'adek s nov\'ym prvkem vkl\'ad\'a do \v ret\v ezce za posled\-n\'\i\ 
\v r\'adek z pomocn\'e \v c\'asti tabulky. Kdy\v z \v ret\v ezec neobsahuje 
\v z\'adn\'y \v r\'adek z pomocn\'e pam\v eti, tak se \v r\'adek s nov\'ym  prvkem $x$ vkl\'ad\'a hned za \v r\'adek $h(x)$. 
\medskip

\flushpar Idea: pomocn\'a \v c\'ast m\'a zabr\'anit rychl\'emu 
sr\accent23ust\'an\'\i\ \v ret\v ezc\accent23u.
\medskip

\flushpar Tyto metody nepodporuj\'\i\ p\v rirozen\'e efektivn\'\i\ algoritmy 
pro ope\-raci {\bf DELETE}.

\flushpar P\v r\'\i klad: $U=\{1,2,\dots,1000\}$, $h(x)=x\bmod10$,\newline 
$S=\{1,7,11,53,73,141,161\}$. Tabulka m\'a 12 \v r\'adk\accent23u a 
m\'a tvar
$$\vtop{\offinterlineskip\halign {\strut\vrule\ # & \vrule\ # & \vrule \hfil # \hfil \vrule\cr\noalign{\hrule}\v r\'adek & key & next \hfil \cr\noalign{\hrule}P(0) & &\hfill \cr\noalign{\hrule}P(1) & 1 & 10 \cr\noalign{\hrule}P(2) & &\hfill \cr\noalign{\hrule}P(3) & 73 & 11 \cr\noalign{\hrule}P(4) &  &\hfill \cr\noalign{\hrule}P(5) & 7 & \hfill \cr\noalign{\hrule}P(6) & &\hfill \cr\noalign{\hrule}P(7) & 161 & 5 \cr\noalign{\hrule}P(8) & 11 & 7 \cr\noalign{\hrule}P(9) & &\hfill \cr\noalign{\hrule}\noalign{\hrule}P(10) & 141 & 8 \cr\noalign{\hrule}P(11) & 53 & \cr\noalign{\hrule}}}$$

\flushpar Ha\v sovac\'\i\ tabulka vznikla posloupnostmi operac\'\i :\newline 
Pro metodu LICH:\newline 
{\bf INSERT$(1)$}, {\bf INSERT$(73)$}, {\bf INSERT$(141)$}, 
{\bf INSERT$(53)$}, {\bf INSERT$(11)$},\newline {\bf INSERT$(161)$}, {\bf INSERT$
(7)$}.\newline 
Pro metodu EICH:\newline 
{\bf INSERT$(1)$}, {\bf INSERT$(73)$}, {\bf INSERT$(161)$}, 
{\bf INSERT$(53)$}, {\bf INSERT$(11)$},\newline {\bf INSERT$(141)$}, {\bf INSERT$
(7)$},

\flushpar ale nedodr\v zovalo se, \v ze se nejd\v r\'\i v zapl\v nuj\'\i\ \v r\'adky z 
pomocn\'e \v c\'asti.  P\v ri dodr\v zov\'an\'\i\ tohoto pravidla takov\'ato tabulka 
nem\accent23u\v ze vznik\-nout.\newline 
Pro metodu VICH:\newline 
{\bf INSERT$(1)$}, {\bf INSERT$(73)$}, {\bf INSERT$(141)$}, 
{\bf INSERT$(53)$}, {\bf INSERT$(161)$},\newline {\bf INSERT$(11)$}, {\bf INSERT$(7)$}. 

\flushpar Aplikujeme operace {\bf INSERT$(28)$} a {\bf INSERT$(31
)$}, nov\'e \v r\'ad\-ky 
budou \v r\'adky \v c\'\i slo 4 a 9. Tabulka vytvo\v ren\'a pomoc\'\i\ 
metody LICH je na lev\'e stran\v e, metodou VICH je v 
prost\v redku a metodou EICH je na prav\'e stran\v e.

$\vbox {\hbox{\vtop{\offinterlineskip\halign{\strut\vrule\ # & \vrule\ # & \vrule\hfil # \hfil\vrule\cr\noalign{\hrule}\v r\'adek & key & next \hfil\cr\noalign{\hrule}P(0) & &\hfill \cr\noalign{\hrule}P(1) & 1 & 10 \cr\noalign{\hrule}P(2) & &\hfill \cr\noalign{\hrule}P(3) & 73 & 11 \cr\noalign{\hrule}P(4) & 28 & 9 \cr\noalign{\hrule}P(5) & 7 & 4 \cr\noalign{\hrule}P(6) &  &\hfill \cr\noalign{\hrule}P(7) & 161 & 5 \cr\noalign{\hrule}P(8) & 11 & 7 \cr\noalign{\hrule}P(9) & 31 &\hfill \cr\noalign{\hrule}\noalign{\hrule}P(10) & 141 & 8 \cr\noalign{\hrule}P(11) & 53 & \hfill \cr\noalign{\hrule}}}\,\vtop{\offinterlineskip\halign{\strut\vrule\ # & \vrule\ # & \vrule\hfil# \hfil\vrule\cr\noalign{\hrule}\v r\'adek & key & next \hfil\cr\noalign{\hrule}P(0) & &\hfill \cr\noalign{\hrule}P(1) & 1 & 10 \cr\noalign{\hrule}P(2) & &\hfill \cr\noalign{\hrule}P(3) & 73 & 11 \cr\noalign{\hrule}P(4) & 28 & 7 \cr\noalign{\hrule}P(5) & 7 & \hfill \cr\noalign{\hrule}P(6) &  &\hfill \cr\noalign{\hrule}P(7) & 161 & 5 \cr\noalign{\hrule}P(8) & 11 & 4 \cr\noalign{\hrule}P(9) & 31 & 8 \cr\noalign{\hrule}\noalign{\hrule}P(10) & 141 & 9 \cr\noalign{\hrule}P(11) & 53 & \hfill \cr\noalign{\hrule}}}\,\vtop{\offinterlineskip\halign {\strut\vrule\ # & \vrule\ # & \vrule \hfil # \hfil \vrule\cr\noalign{\hrule}\v r\'adek & key & next \hfil \cr\noalign{\hrule}P(0) & &\hfill \cr\noalign{\hrule}P(1) & 1 & 9 \cr\noalign{\hrule}P(2) & &\hfill \cr\noalign{\hrule}P(3) & 73 & 11 \cr\noalign{\hrule}P(4) & 28 & 7 \cr\noalign{\hrule}P(5) & 7 & \hfill \cr\noalign{\hrule}P(6) & & \hfill \cr\noalign{\hrule}P(7) & 161 & 5 \cr\noalign{\hrule}P(8) & 11 & 4  \cr\noalign{\hrule}P(9) & 31 & 10 \cr\noalign{\hrule}\noalign{\hrule}P(10) & 141 & 8 \cr\noalign{\hrule}P(11) & 53 & \hfill \cr\noalign{\hrule}}}}}$
\medskip

\subhead
Algoritmy
\endsubhead
\smallskip

\flushpar Algoritmus operace {\bf MEMBER} je pro tyto metody 
stejn\'y jako pro LISCH a EISCH
\bigskip

\flushpar {\bf MEMBER$(x)$}:\newline 
Spo\v c\'\i t\'ame $i:=h(x)$\newline 
{\bf while} $i.next\ne$pr\'azdn\'e a $i.key\ne x$ {\bf do} $i:=i.
next$ {\bf enddo\newline 
if} $i.key=x$ {\bf then V\'ystup}: $x\in S$ {\bf else V\'ystup}: $
x\notin S$ {\bf endif
\medskip

}\flushpar Algoritmus operace {\bf INSERT} je pro metodu LICH stejn\'y 
jako pro metodu LISCH a pro metodu EICH je stejn\'y jako 
pro metodu EISCH s jedin\'ym dopl\v nkem, pokud existuje 
pr\'azdn\'y \v r\'adek v pomocn\'e \v c\'asti, tak $j$-t\'y \v r\'adek je z 
pomocn\'e \v c\'asti. Tento p\v redpoklad je i pro algoritmus 
{\bf INSERT} pro metodu VICH.
\medskip

\flushpar Metoda LICH -- {\bf INSERT$(x)$}:\newline 
Spo\v c\'\i t\'ame $i:=h(x)$\newline 
{\bf if} $i.next=NIL$ {\bf then} $i.next=x$, stop {\bf endif\newline 
while} $i.next\ne NIL$ a $i.key\ne x$ {\bf do} $i:=i.next$ {\bf enddo\newline 
if} $i.key\ne x$ {\bf then}\newline 
\phantom{---}{\bf if} neexistuje pr\'azdn\'y \v r\'adek tabulky {\bf then}\newline 
\phantom{------}p\v repln\v en\'\i, stop\newline 
\phantom{---}{\bf else}\newline 
\phantom{------}nech\v t $j$ je pr\'azdn\'y \v r\'adek, $j.key:=x$, $i.next:=j$\newline 
\phantom{---}{\bf endif\newline 
endif}
\medskip

\flushpar Metoda EICH -- {\bf Insert$(x)$}:\newline 
Spo\v c\'\i t\'ame $k:=i:=h(x)$\newline 
{\bf if} $i.next=NIL$ {\bf then} $i.next=x$, stop {\bf endif\newline 
while} $i.next\ne NIL$ a $i.key\ne x$ {\bf do} $i:=i.next$ {\bf enddo\newline 
if} $i.key\ne x$ {\bf then}\newline 
\phantom{---}{\bf if} neexistuje pr\'azdn\'y \v r\'adek tabulky {\bf then}\newline 
\phantom{------}p\v repln\v en\'\i, stop\newline 
\phantom{---}{\bf else}\newline 
\phantom{------}nech\v t $j$ je voln\'y \v r\'adek tabulky\newline 
\phantom{------}$j.next:=k.next$, $k.next:=j$, $j.key:=x$\newline 
\phantom{---}{\bf endif\newline 
endif
\medskip

}\flushpar Metoda VICH -- {\bf INSERT$(x)$}:\newline 
Spo\v c\'\i t\'ame $i:=h(x)$\newline 
{\bf if} $i.next=NIL$ {\bf then} $i.next=x$, stop {\bf endif\newline 
while} $i.next\ne NIL$ a $i.key\ne x$ {\bf do}\newline 
\phantom{---}{\bf if} $k$ nen\'\i\ definov\'ano a $i.next<m$ {\bf then} $
k:=i$ {\bf endif\newline }
Pozn\'amka: Podm\'\i nka pro $k$ je spln\v ena, kdy\v z jsme byli na 
za\v c\'atku nebo v pomocn\'e \v c\'asti, podm\'\i nka na $i.next$ je spln\v ena, 
kdy\v z $i.next$ nen\'\i\ v pomocn\'e \v c\'asti.\newline 
\phantom{---}$i:=i.next$\newline 
{\bf enddo\newline 
if} $i.key\ne x$ {\bf then}\newline 
\phantom{---}{\bf if} neexistuje pr\'azdn\'y \v r\'adek {\bf then}\newline 
\phantom{------}p\v repln\v en\'\i, stop\newline 
\phantom{---}{\bf else}\newline 
\phantom{------}nech\v t $j$ je voln\'y \v r\'adek, $j.key
:=x$\newline 
\phantom{------}{\bf if} $k$ nen\'\i\ definov\'ano {\bf then}\newline 
\phantom{---------}$i.next:=j$\newline 
\phantom{------}{\bf else}\newline 
\phantom{---------}$j.next:=k.next$, $k.next:=j$\newline 
\phantom{------}{\bf endif}\newline 
\phantom{---}{\bf endif\newline 
endif}
\bigskip

\flushpar Slo\v zitost algoritm\accent23u pro sr\accent23ustaj\'\i c\'\i\ ha\v sov\'an\'\i .\newline 
Zna\v cen\'\i : $n$ -- velikost ulo\v zen\'e mno\v ziny,\newline 
$m$ -- velikost adresovac\'\i\ \v c\'asti tabulky,\newline 
$m'$ -- velikost tabulky,\newline 
$\alpha =\frac n{m'}$ -- faktor zapln\v en\'\i ,\newline 
$\beta =\frac m{m'}$ -- adresovac\'\i\ faktor,\newline 
$\lambda$ -- jedin\'e nez\'aporn\'e \v re\v sen\'\i\ rovnice $e^{
-\lambda}+\lambda =\frac 1{\beta}$.
\medskip

\flushpar O\v cek\'avan\'y po\v cet test\accent23u pro metodu LICH\newline 
ne\'usp\v e\v sn\'y p\v r\'\i pad:\newline 
\phantom{---}$e^{-\frac {\alpha}{\beta}}+\frac {
\alpha}{\beta}$, kdy\v z  $\alpha\le\lambda\beta$, \newline 
\phantom{---}$\frac 1{\beta}+\frac 14(e^{2(\frac {\alpha}{\beta}-\lambda )}-1)
(3-\frac 2{\beta}+2\lambda )-\frac 12(\frac {\alpha}{\beta}-\lambda 
)$, kdy\v z $\alpha\ge\lambda\beta$\newline 
\'usp\v e\v sn\'y p\v r\'\i pad:\newline 
\phantom{---}$1+\frac {\alpha}{2\beta}$, kdy\v z $
\alpha\le\lambda\beta$, \newline 
\phantom{---}$1+\frac {\beta}{8\alpha}(e^{2(\frac {\alpha}{\beta}-\lambda )}-1
-2(\frac {\alpha}{\beta}-\lambda ))(3-\frac 2{\beta}+2\lambda )+\frac 
14(\frac {\alpha}{\beta}+\lambda )+\frac {\lambda}4(1-\frac {\lambda
\beta}{\alpha})$, 
kdy\v z $\alpha\ge\lambda\beta$.
\medskip

\flushpar O\v cek\'avan\'y po\v cet test\accent23u pro metodu EICH\newline 
ne\'usp\v e\v sn\'y p\v r\'\i pad:\newline
\phantom{---} $e^{-\frac {\alpha}{\beta}}+\frac {
\alpha}{\beta}$, kdy\v z  $\alpha\le\lambda\beta$, \newline 
\phantom{---}$e^{2(\frac {\alpha}{\beta}-\lambda )}(\frac 34+\frac {\lambda}2-\frac 
1{2\beta})+e^{\frac {\alpha}{\beta}-\lambda}(\frac 1{\beta}-1)+(\frac 
14-\frac {\alpha}{2\beta}+\frac 1{2\beta})$, kdy\v z $\alpha\ge\lambda
\beta$\newline 
\'usp\v e\v sn\'y p\v r\'\i pad:\newline 
\phantom{---}$1+\frac {\alpha}{2\beta}$, kdy\v z $
\alpha\le\lambda\beta$, \newline 
\phantom{---}$1+\frac {\alpha}{2\beta}+\frac {\beta}{\alpha}((e^{\frac {\alpha}{
\beta}-\lambda}-1)(1+\lambda )-(\frac {\alpha}{\beta}-\lambda ))(
1+\frac {\lambda}2+\frac {\alpha}{2\beta}))$, kdy\v z $\alpha\ge\lambda
\beta$.
\medskip

\flushpar O\v cek\'avan\'y po\v cet test\accent23u pro metodu VICH\newline 
ne\'usp\v e\v sn\'y p\v r\'\i pad:\newline 
\phantom{---}$e^{-\frac {\alpha}{\beta}}+\frac {
\alpha}{\beta}$, kdy\v z  $\alpha\le\lambda\beta$, \newline 
\phantom{---}$\frac 1{\beta}+\frac 14(e^{2(\frac {\alpha}{\beta}-\lambda )}-1)
(3-\frac 2{\beta}+2\lambda )-\frac 12(\frac {\alpha}{\beta}-\lambda 
)$, kdy\v z $\alpha\ge\lambda\beta$\newline 
\'usp\v e\v sn\'y p\v r\'\i pad:\newline 
\phantom{---}$1+\frac {\alpha}{2\beta}$, kdy\v z $
\alpha\le\lambda\beta$, \newline 
\phantom{---}$1+\frac {\alpha}{2\beta}+\frac {\beta}{\alpha}((e^{\frac {\alpha}{
\beta}-\lambda}-1)(1+\lambda )-(\frac {\alpha}{\beta}-\lambda ))(
1+\frac {\lambda}2+\frac {\alpha}{2\beta}))+\frac {1-\beta}{\alpha}
(\frac {\alpha}{\beta}-\lambda -e^{\frac {\alpha}{\beta}-\lambda}
+1)$, 
kdy\v z $\alpha\ge\lambda\beta$.
\medskip
\flushpar Chyba aproximace pro tyto odhady je $O(\log\frac {m'}{\sqrt {
m'}})$.


\heading
Ha\v sov\'an\'\i\ s line\'arn\'\i m p\v rid\'av\'an\'\i m
\endheading

\flushpar Tabulka m\'a jedinou polo\v zku -- key
\medskip

\flushpar Z\'akladn\'\i\ idea:  P\v ri operaci {\bf INSERT$(x)$} vlo\v z\'\i me $
x$ 
na \v r\'adek $h(x)$, kdy\v z je pr\'azdn\'y, v opa\v cn\'em p\v r\'\i pad\v e 
nalezneme nejmen\v s\'\i\ $i$ takov\'e, \v ze \v r\'adek $h(x)+i\bmod 
m$ je 
pr\'azdn\'y, a tam vlo\v z\'\i me $x$. Tato metoda byla motivov\'ana snahou o co nejv\v et\v s\'\i\ vyu\v zit\'\i\ pam\v eti. 
\medskip

\flushpar Koment\'a\v r:  Metoda vy\v zaduje minim\'aln\'\i\ velikost pam\v eti.  
V tabulce se vytv\'a\v rej\'\i\ shluky pou\v zit\'ych \v r\'adk\accent23u, a proto 
p\v ri velk\'em zapln\v en\'\i\ metoda vy\v zaduje velk\'e mno\v zstv\'\i\ \v casu.  
Metoda nepodporuje efektivn\'\i\ implementaci operace {\bf DELETE}.  
P\v ri vy\-hle\-d\'a\-v\'a\-n\'\i\ je t\v reba testovat, zda nevy\v set\v rujeme 
podruh\'e prvn\'\i\ vy\v set\v rovan\'y \v r\'adek a pro zji\v st\v en\'\i\ p\v repln\v en\'\i\ je 
vhodn\'e m\'\i t ulo\v zen po\v cet vypln\v en\'ych \v r\'adk\accent23u v tabulce. Pro standarn\'\i\ pam\v eti nen\'\i\ v\'yhodn\'a. P\v ri pou\v ziti cache-pam\v eti se v\'yrazn\v e m\v en\'\i\ jej\'\i\ ohodnocen\'\i, proto\v ze minimalizuje po\v cet p\v rechod\accent23u mezi r\accent23uzn\'ymi typy pam\v et\'\i. Proto se tato metoda doporu\v cuje pro po\v c\'\i ta\v ce s cache-pam\v et\'\i.
\medskip

\flushpar {\bf MEMBER$(x)$}:\newline 
Spo\v c\'\i t\'ame $i:=h(x)$, $h:=i$\newline 
{\bf if} $i.key=x$ {\bf then V\'ystup} $x\in S$, stop {\bf endif\newline 
if} $i.key=$pr\'azdn\'y {\bf then {{\rm V\'ystup}}}: $x\notin S$, stop {\bf endif\newline 
$i:=i+1$\newline 
while} $i.key\ne$pr\'azdn\'y a $i.key\ne x$ a $i\ne h$ {\bf do} $
i:=i+1\bmod m$ {\bf enddo\newline 
if} $i.key=x$ {\bf then V\'ystup}: $x\in S$ {\bf else V\'ystup}: $
x\notin S$ {\bf endif
\medskip

\flushpar INSERT$(x)$}:\newline 
Spo\v c\'\i t\'ame $i:=h(x)$, $j:=0$\newline 
{\bf while} $i.key\ne$pr\'azdn\'y a $i.key\ne x$ a $j<m$ {\bf do} $
i:=i+1\bmod m${\bf ,} $j:=j+1$ {\bf enddo\newline 
if} $j=m$ {\bf then V\'ystup}: p\v repln\v en\'\i , stop {\bf endif\newline 
if} $i.key=$pr\'azdn\'y {\bf then} $i.key:=x$ {\bf endif
\bigskip

}\flushpar P\v r\'\i klad: M\'ame universum $U=\{1,2,\dots,1000\}$, 
ha\v sovac\'\i\ funk\-ci $h(x)=x\bmod10$ a mno\v zinu  
$S=\{1,7,11,53,73,141,161\}$. Tato mno\v zina je ulo\v zena v lev\'e tabulce. 
Provedeme operaci {\bf INSERT$(35)$}. V\'ysledek je ulo\v zen v 
prav\'e tabulce.
$$\vbox {\hbox{\vtop{\offinterlineskip\halign{\strut\vrule\ # & \vrule\hfil# \hfil\vrule\cr\noalign{\hrule}\v r\'adek & key \hfil\cr\noalign{\hrule}P(0) & \hfill \cr\noalign{\hrule}P(1) & 1 \cr\noalign{\hrule}P(2) & 11 \cr\noalign{\hrule}P(3) & 73 \cr\noalign{\hrule}P(4) & 141 \cr\noalign{\hrule}P(5) & 161 \cr\noalign{\hrule}P(6) & 53 \cr\noalign{\hrule}P(7) & 7 \cr\noalign{\hrule}P(8) & \hfill\cr\noalign{\hrule}P(9) & \hfill \cr\noalign{\hrule}}}\qquad\vtop{\offinterlineskip\halign {\strut\vrule\ # & \vrule \hfil # \hfil \vrule\cr\noalign{\hrule}\v r\'adek & key \hfil \cr\noalign{\hrule}P(0) & \hfill \cr\noalign{\hrule}P(1) & 1 \cr\noalign{\hrule}P(2) & 11 \cr\noalign{\hrule}P(3) & 73 \cr\noalign{\hrule}P(4) & 141 \cr\noalign{\hrule}P(5) & 161 \cr\noalign{\hrule}P(6) & 53 \cr\noalign{\hrule}P(7) & 7 \cr\noalign{\hrule}P(8) & 35  \cr\noalign{\hrule}P(9) & \hfill \cr\noalign{\hrule}}}}}$$

\flushpar Tabulka vznikla posloupnost\'\i\ operac\'\i :\newline 
{\bf INSERT$(1)$}, {\bf INSERT$(11)$}, {\bf INSERT$(73)$}, 
{\bf INSERT$(141)$}, {\bf INSERT$(161)$},\newline {\bf INSERT$(53)$}, {\bf INSERT$
(7)$}. 

\flushpar Na z\'av\v er uvedeme slo\v zitost t\'eto metody. O\v cek\'avan\'y 
po\v cet test\accent23u:\newline 
\phantom{---}ne\'usp\v e\v sn\'y p\v r\'\i pad: $\approx\frac 12(1+\big(\frac 
1{1-\alpha}\big)^2)$, \newline 
\phantom{---}\'usp\v e\v sn\'y p\v r\'\i pad: $\approx\frac 12(1+\frac 1{1-\alpha}
)$.

\heading
Dvojit\'e ha\v sov\'an\'\i
\endheading

\flushpar Z\'akladn\'\i\ nev\'yhoda p\v redchoz\'\i\ metody je zp\accent23usob 
v\'yb\v eru dal\-\v s\'\i\-ho \v r\'adku. Je velmi determinov\'an a 
d\accent23usledkem je vznik shluku \v r\'adk\accent23u, kter\'y 
vede k v\'yrazn\'emu zpomalen\'\i\ metody. \newline 
Idea jak odstranit tuto nev\'yhodu: Pou\v zijeme dv\v e 
ha\v sovac\'\i\ funkce $h_1$ a $h_2$ a p\v ri operaci {\bf INSERT$
(x)$ }
nalezneme nejmen\v s\'\i\ $i=0,1,\dots$ takov\'e, \v ze 
$(h_1(x)+ih_2(x))\bmod m$ je pr\'azdn\'y \v r\'adek, a tam ulo\v z\'\i me 
prvek $x$.\newline 
Tabulka m\'a jedinou polo\v zku -- key.\newline 
Po\v zadavky na korektnost: Pro ka\v zd\'e $x$ mus\'\i\ b\'yt $h_
2(x)$ a 
$m$ nesoud\v eln\'e (jinak prvek $x$ nem\accent23u\v ze b\'yt ulo\v zen 
na libovoln\'em \v r\'adku tabulky). \newline 
P\v redpoklad pro v\'ypo\v cet o\v cekavan\'eho po\v ctu test\accent23u: posloupnost 
$\{h_1(x)+ih_2(x)\}_{i=0}^{m-1}$ je n\'ahodn\'a permutace mno\v ziny 
\v r\'adk\accent23u tabulky.\newline 
Nev\'yhoda: Uveden\'a metoda nepodporuje operaci {\bf DELETE}. \newline 
P\v repln\v en\'\i\ se \v re\v s\'\i\ stejn\'ym zp\accent23usobem jako v 
metod\v e ha\v sov\'an\'\i\ s line\'arn\'\i m p\v rid\'av\'an\'\i m.\newline 
Pozn\'amka: Metoda ha\v sov\'an\'\i\ s line\'arn\'\i m p\v rid\'av\'an\'\i m je 
speci\'al\-n\'\i\ p\v r\'\i pad dvojit\'eho ha\v sov\'an\'\i , kde $
h_2(x)=1$ pro 
ka\v zd\'e $x\in U$.
\medskip

\subhead
Algoritmy
\endsubhead
\smallskip

\flushpar {\bf MEMBER$(x)$}:\newline 
Spo\v c\'\i t\'ame $i:=h_1(x)$, $h:=h_2(x)$, $j:=0$\newline 
{\bf while} $i.key\ne$pr\'azdn\'y a $i.key\ne x$ a $j<m$ {\bf do} $
i:=i+h\bmod m$, $j:=j+1$ {\bf enddo\newline 
if} $i.key=x$ {\bf then V\'ystup}: $x\in S$ {\bf else V\'ystup}: $
x\notin S$ {\bf endif
\medskip

\flushpar INSERT$(x)$}:\newline 
Spo\v c\'\i t\'ame $i:=h_1(x)$, $h:=h_2(x)$, $j:=0$\newline 
{\bf while} $i.key\ne$pr\'azdn\'y a $i.key\ne x$ a $j<m$ {\bf do} $
i:=i+h\bmod m$, $j:=j+1$ {\bf enddo\newline 
if} $j=m$ {\bf then V\'ystup}: p\v repln\v en\'\i , stop {\bf endif\newline 
if} $i.key=$pr\'azdn\'y {\bf then} $i.key:=x$ {\bf endif
\medskip

}\flushpar P\v r\'\i klad: M\v ejme universum $U=\{1,2,\dots,1000
\}$. 
Ha\v sovac\'\i\ funkce jsou 
$h_1(x)=x\bmod10$ a $h_2(x)=1+2(x\bmod4)$, kdy\v z 
$x\bmod4\in \{0,1\}$, $h_2(x)=3+2(x\bmod4)$, kdy\v z $x\bmod4\in 
\{2,3\}$. 
Mno\v zina je  $S=\{1,7,11,53,73,141,161\}$. Tato mno\v zina je ulo\v zena v 
lev\'e tabulce. Aplikujme {\bf INSERT$(35)$}. Pak $h_2(35)=9$, tedy 
posloupnost pro $x=35$ je 
$$(5,4,3,2,1,0,9,8,7,6).$$
V\'ysledek je ulo\v zen v prav\'e tabulce. 
$$\vbox {\hbox{\vtop{\offinterlineskip\halign{\strut\vrule\ # & \vrule\hfil# \hfil\vrule\cr\noalign{\hrule}\v r\'adek & key \hfil\cr\noalign{\hrule}P(0) & 11 \cr\noalign{\hrule}P(1) & 1 \cr\noalign{\hrule}P(2) & \hfill \cr\noalign{\hrule}P(3) & 73 \cr\noalign{\hrule}P(4) & 141 \cr\noalign{\hrule}P(5) & 7 \cr\noalign{\hrule}P(6) & 53 \cr\noalign{\hrule}P(7) & 161 \cr\noalign{\hrule}P(8) & \hfill\cr\noalign{\hrule}P(9) & \hfill \cr\noalign{\hrule}}}\qquad\vtop{\offinterlineskip\halign {\strut\vrule\ # & \vrule \hfil # \hfil \vrule\cr\noalign{\hrule}\v r\'adek & key \hfil \cr\noalign{\hrule}P(0) & 11 \cr\noalign{\hrule}P(1) & 1 \cr\noalign{\hrule}P(2) & 35 \cr\noalign{\hrule}P(3) & 73 \cr\noalign{\hrule}P(4) & 141 \cr\noalign{\hrule}P(5) & 7 \cr\noalign{\hrule}P(6) & 53 \cr\noalign{\hrule}P(7) & 161 \cr\noalign{\hrule}P(8) & \hfill \cr\noalign{\hrule}P(9) & \hfill \cr\noalign{\hrule}}}}}$$

\flushpar Tabulka vznikla posloupnost\'\i\ operac\'\i :\newline 
{\bf INSERT$(1)$}, {\bf INSERT$(73)$}, {\bf INSERT$(53)$}, 
{\bf INSERT$(141)$}, {\bf INSERT$(161)$},\newline {\bf INSERT$(11)$}, {\bf INSERT$
(7)$}. 

\flushpar Anal\'yza vyhled\'av\'an\'\i\ v dvojit\'em ha\v sov\'an\'\i .
\medskip
\subhead
Ne\'usp\v e\v sn\'y p\v r\'\i pad
\endsubhead
\smallskip
 
Zna\v cen\'\i :  $q_i(n,m)$ -- kdy\v z tabulka m\'a $m$ \v r\'adk\accent23u a 
je v n\'\i\ obsazeno $n$ \v r\'adk\accent23u, tak je to 
pravd\v epodobnost, \v ze pro ka\v zd\'e $j=0,1,\dots,i-1$ je 
\v r\'adek $h_1(x)+jh_2(x)$ obsazen. Pak $q_0(n,m)=1$, $q_1(n,m)=\frac 
nm$, 
$q_2(n,m)=\frac {n(n-1)}{m(m-1)}$ a obecn\v e 
$$q_i(n,m)=\frac {\prod_{j=0}^{i-1}(n-j)}{\prod_{j=0}^{i-1}(m-j)}
.$$
$C(n,m)$ -- o\v cek\'avan\'y po\v cet test\accent23u v ne\'usp\v e\v sn\'em 
vyhled\'av\'an\'\i , kdy\v z tabulka m\'a $m$ \v r\'adk\accent23u a $
n$ jich 
je obsazeno. Podle definice plat\'\i :
$$C(n,m)=\sum_{j=0}^n(j+1)(q_j(n,m)-q_{j+1}(n,m))=\sum_{j=0}^nq_j
(n,m).$$
D\'ale plat\'\i\ $C(0,m)=1$ pro ka\v zd\'e $m$ a 
$q_j(n,m)=\frac nmq_{j-1}(n-1,m-1)$ pro v\v sechna $j,n>0$ a $m>1$. 
Odtud
$$C(n,m)=\sum_{j=0}^nq_j(n,m)=1+\frac nm(\sum_{j=0}^{n-1}
q_j(n-1,m-1))=1+\frac nmC(n-1,m-1).$$

\flushpar Indukc\'\i\ uk\'a\v zeme, \v ze $C(n,m)=\frac {m+1}{m-n
+1}$. Kdy\v z $n=0$, pak 
$C(0,m)=\frac {m+1}{m-0+1}=1$ a tvrzen\'\i\ plat\'\i . P\v redpokl\'ad\'ame, \v ze tvrzen\'\i\ 
plat\'\i\ pro $n-1\ge 0$ a pro ka\v zd\'e $m\ge n-1$ a dok\'a\v zeme tvrzen\'\i\ 
pro $n$ a $m\ge n$. Plat\'\i
$$\align C(n,m)=&1+\frac nmC(n-1,m-1)=\\
&1+\frac {n((m-1)+1)}{m((m-1)-(n-1)+1)}=\\
&1+\frac n{m-n+1}=\frac {m+1}{m-n+1}.\endalign$$
O\v cek\'avan\'y po\v cet dotaz\accent23u p\v ri ne\'usp\v e\v sn\'em 
vyhled\'av\'an\'\i\ v tabulce s $m$ \v r\'adky, z nich\v z $n$ je obsazeno, 
je $\frac {m+1}{m-n+1}$.
\medskip

\subhead
\'Usp\v e\v sn\'y p\v r\'\i pad
\endsubhead
\smallskip
 
Pou\v zijeme metodu ze separuj\'\i c\'\i ch \v ret\v ezc\accent23u. 
Po\v cet dotaz\accent23u p\v ri vyhled\'av\'an\'\i\ $x$ pro $x\in 
S$ je 
stejn\'y jako byl po\v cet dotaz\accent23u p\v ri vkl\'ad\'an\'\i\ $
x$ do 
tabulky. Tedy o\v cek\'avan\'y po\v cet dotaz\accent23u p\v ri 
\'usp\v e\v sn\'em vyhled\'av\'an\'\i\ v tabulce s $m$ \v r\'adky, z nich\v z $
n$ je 
obsazeno, je
$$\align\frac 1n\sum_{i=0}^{n-1}C(i,m)=&\frac 1n\sum_{i=0}^{n-1}\frac {
m+1}{m-i+1}=\\
&\frac {m+1}n\big(\sum_{j=1}^{m+1}\frac 1j-\sum_{j=1}^{m-n+1}\frac 
1j\big)\approx\\
&\frac 1{\alpha}\ln(\frac {m+1}{m-n+1})\approx\frac 1{\alpha}\ln(\frac 
1{1-\alpha}).\endalign$$
N\'asleduj\'\i c\'\i\ tabulka ukazuje tyto hodnoty v z\'avislosti na 
velikosti $\alpha$. 
$$\vtop{\offinterlineskip\halign {\strut\vrule\ # & \vrule\ # & \vrule\ # & \vrule\ # & \vrule\ # & \vrule\ # & \vrule \hfil # \hfil \vrule\cr\noalign{\hrule}hodnota $\alpha$ & 0.5 & 0.7 & 0.9 & 0.95 & 0.99 & 0.999 \hfill \cr\noalign{\hrule}$\frac 1{1-\alpha}$ & 2 & 3.3 & 10 & 20 & 100 & 1000 \cr\noalign{\hrule} $\frac 1{\alpha}\ln(\frac 1{1-\alpha})$ & 1.38 & 1.70 & 2.55 & 3.15 & 4.65 & 6.9 \cr\noalign{\hrule}}}$$

\heading
Porovn\'an\'\i\ efektivity
\endheading

\flushpar Po\v rad\'\i\ metod ha\v sov\'an\'\i\ podle o\v cek\'avan\'eho po\v ctu 
test\accent23u:
\subhead
Ne\'usp\v e\v sn\'e vyhled\'av\'an\'\i
\endsubhead 

\phantom{---}Ha\v sov\'an\'\i\ s uspo\v r\'adan\'ymi \v ret\v ezci,\newline 
\phantom{---}Ha\v sovan\'\i\ s \v ret\v ezci=Ha\v sov\'an\'\i\ s p\v rem\'\i s\v tov\'an\'\i m,\newline 
\phantom{---}Ha\v sov\'an\'\i\ s dv\v ema ukazateli,\newline 
\phantom{---}VICH=LICH,\newline 
\phantom{---}EICH,\newline 
\phantom{---}LISCH=EISCH,\newline 
\phantom{---}Dvojit\'e ha\v sov\'an\'\i ,\newline 
\phantom{---}Ha\v sov\'an\'\i\ s line\'arn\'\i m p\v rid\'av\'an\'\i m.
\medskip

\subhead
\'Usp\v e\v sn\'e vyhled\'av\'an\'\i 
\endsubhead

\phantom{---}Ha\v sov\'an\'\i\ s uspo\v r\'adan\'ymi \v ret\v ezci=Ha\v sov\'an\'\i\ s 
\v ret\v ezci=Ha\v so\-v\'a\-n\'\i\ s p\v rem\'\i s\v tov\'an\'\i m,\newline 
\phantom{---}Ha\v sov\'an\'\i\ s dv\v ema ukazateli,\newline 
\phantom{---}VICH,\newline 
\phantom{---}LICH,\newline 
\phantom{---}EICH,\newline 
\phantom{---}EISCH,\newline 
\phantom{---}LISCH,\newline 
\phantom{---}Dvojit\'e ha\v sov\'an\'\i ,\newline 
\phantom{---}Ha\v sov\'an\'\i\ s line\'arn\'\i m p\v rid\'av\'an\'\i m.
\medskip

\flushpar Pozn\'amka: Metoda VICH p\v ri ne\'usp\v e\v sn\'em vyhled\'av\'an\'\i\ pro 
$\alpha <0.72$ a p\v ri \'usp\v e\v sn\'em vy\-hled\'av\'an\'\i\ pro $
\alpha <0.92$ 
vy\v zaduje men\v s\'\i\ o\v cek\'avan\'y po\v cet test\accent23u ne\v z 
metoda s dv\v ema ukazateli.

\flushpar P\v ri ne\'usp\v e\v sn\'em vyhled\'av\'an\'\i\ jsou metody VICH a LICH stejn\'e a jsou o 8\% lep\v s\'\i\ ne\v z 
EICH a o 15\% ne\v z metody LISCH a EISCH. P\v ri \'usp\v e\v sn\'em vyhled\'av\'an\'\i\ 
je VICH nepatrn\v e lep\v s\'\i\ ne\v z LICH a EICH o 3\% lep\v s\'\i\ ne\v z 
EISCH a o 7\% lep\v s\'\i\ ne\v z LISCH.
\medskip
\subhead
O\v cek\'avan\'y po\v cet test\accent23u p\v ri \'upln\v e zapln\v en\'e 
tabulce
\endsubhead 

\phantom{---}Metoda s p\v rem\'\i s\v tov\'an\'\i m: ne\'usp\v e\v sn\'e vyhled\'av\'an\'\i\ 1.5, \'usp\v e\v sn\'e 
vyhled\'av\'an\'\i\ 1.4.\newline 
\phantom{---}Metoda s dv\v ema ukazateli: \'usp\v e\v sn\'e i ne\'usp\v e\v sn\'e vyhled\'av\'an\'\i\  1.6.\newline 
\phantom{---}VICH: ne\'usp\v e\v sn\'e vyhled\'av\'an\'\i\ 1.79, \'usp\v e\v sn\'e vyhled\'av\'an\'\i\ 
1.67.\newline 
\phantom{---}LICH: ne\'usp\v e\v sn\'e vyhled\'av\'an\'\i\ 1.79, \'usp\v e\v sn\'e vyhled\'av\'an\'\i\ 
1.69.\newline 
\phantom{---}EICH: ne\'usp\v e\v sn\'e vyhled\'av\'an\'\i\ 1.93, \'usp\v e\v sn\'e vyhled\'av\'an\'\i\ 
1.69.\newline 
\phantom{---}EISCH: ne\'usp\v e\v sn\'e vyhled\'av\'an\'\i\ 2.1, \'usp\v e\v sn\'e vyhled\'av\'an\'\i\ 
1.72.\newline 
\phantom{---}LISCH: ne\'usp\v e\v sn\'e vyhled\'av\'an\'\i\ 2.1, \'usp\v e\v sn\'e vyhled\'av\'an\'\i\ 
1.8.
\medskip

\flushpar Metodu s line\'arn\'\i m p\v rid\'av\'an\'\i m je dobr\'e pou\v z\'\i t jen pro 
$\alpha <0.7$, metodu s dvojit\'ym ha\v sov\'an\'\i m pro $\alpha 
<0.9$, pak 
\v cas pro ne\'usp\v e\v sn\'e vyhled\'av\'an\'\i\ rychle nar\accent23ust\'a.
\medskip

\flushpar Vliv $\beta =\frac m{m'}$ p\v ri sr\accent23ustaj\'\i c\'\i m ha\v sov\'an\'\i .\newline 
P\v ri 
\'usp\v e\v sn\'em vy\-hled\'av\'an\'\i\ je optim\'aln\'\i\ hodnota $
\beta =0.85$, p\v ri 
ne\'usp\v e\v sn\'em vy\-hled\'av\'an\'\i\ je optim\'aln\'\i\ hodnota $
\beta =0.78$. 
V praxi se doporu\v cuje pou\v z\'\i t hodnotu $\beta =0.86$ (uveden\'e 
v\'ysledky byly pro tuto hodnotu $\beta ).$
\medskip

\flushpar Koment\'a\v r: Metody se separuj\'\i c\'\i mi \v ret\v ezci a 
sr\accent23ustaj\'\i c\'\i\ ha\v sov\'an\'\i\ pou\v z\'\i vaj\'\i\ v\'\i ce pam\v eti (p\v ri 
sr\accent23ustaj\'\i c\'\i m ha\v sov\'an\'\i\ sou\v cet adresovac\'\i\ a 
pomocn\'e \v c\'asti). Metoda s p\v rem\'\i s\v tov\'an\'\i m a metoda 
dvojit\'eho ha\v sov\'an\'\i\ vy\v zaduji v\'\i ce \v casu -- na p\v rem\'\i st\v en\'\i\ 
prvku a na v\'ypo\v cet druh\'e ha\v sovac\'\i\ funkce.

\heading
Dal\v s\'\i\ ot\'azky
\endheading

\flushpar Jak nal\'ezt voln\'y \v r\'adek.\newline 
Za nejlep\v s\'\i\ metodu se pova\v zuje m\'\i t seznam (z\'asobn\'\i k) 
voln\'ych \v r\'adk\accent23u a z jeho vrcholu br\'at voln\'y \v r\'adek 
a po \'usp\v e\v sn\'e operaci {\bf DELETE} tam zase \v r\'adek vlo\v zit 
(pozor p\v ri operaci {\bf DELETE} ve struktur\'ach kter\'e 
nepodporuj\'\i\ {\bf DELETE}).
\medskip

\flushpar Jak \v re\v sit p\v repln\v en\'\i .\newline 
Standardn\'\i\ model: D\'ana z\'akladn\'\i\ velikost tabulky $m$ a 
pracuje se s tabulkami s $2^im$ \v r\'adky pro vhodn\'e 
$i=0,1,\dots$. Vhodn\'e $i$ znamen\'a, \v ze faktor zapln\v en\'\i\ $
\alpha$ je v 
intervalu $<\frac 14,1>$ (s vyj\'\i mkou $i=0$, kde se uva\v zuje 
pouze horn\'\i\ mez). P\v ri p\v rekro\-\v ce\-n\'\i\ meze se zv\v et\v s\'\i\ nebo 
zmen\v s\'\i\ $i$ a v\v sechna data se p\v reha\v suj\'\i\ do nov\'e 
tabulky.\newline 
V\'yhoda: Po p\v reha\v sov\'an\'\i\ do nov\'e tabulky, po\v cet operac\'\i , 
kter\'e vedou k nov\'emu p\v reha\v sov\'a\-v\'a\-n\'\i , je alespo\v n polovina 
velikosti ulo\v zen\'e mno\v ziny.\newline 
Praktick\'e pou\v zit\'\i : Nedr\v zet se striktn\v e mez\'\i , pou\v z\'\i vat 
mal\'e pomocn\'e ta\-bul\-ky p\v ri p\v repln\v en\'\i\ a posunout velk\'e 
p\v reha\v sov\'an\'\i\ na dobu klidu (aby syst\'em nenechal u\v zivatele v 
norm\'aln\'\i\ dob\v e \v cekat).
\medskip

\flushpar Jak \v re\v sit {\bf DELETE} v metod\'ach, kter\'e ho 
nepodporuj\'\i .\newline 
Pou\v z\'\i t ideu tzv. `fale\v sn\'eho {\bf DELETE}'. 
Odstranit prvek, ale \v r\'adek neuvolnit (i v kl\'\i\v ci nechat 
n\v ejakou hodnotu, kter\'a bude znamenat, \v ze \v r\'adek je 
pr\'azdn\'y, polo\v zky podporuj\'\i c\'\i\ pr\'aci s tabulkami nem\v enit). 
\v R\'adek nebude v seznamu voln\'ych \v r\'adk\accent23u, ale 
operace {\bf INSERT}, kdy\v z testuje tento \v r\'adek, tak tam 
m\accent23u\v ze vlo\v zit nov\'y prvek. Kdy\v z je alespo\v n 
polovina pou\v zit\'ych \v r\'adk\accent23u takto blokov\'ana, je 
vhodn\'e celou strukturu p\v reha\v sovat. Prav\-d\v epodobnostn\'\i\ 
anal\'yzu tohoto modelu nezn\'am.
\medskip
\subhead
Otev\v ren\'e probl\'emy
\endsubhead
\smallskip

\flushpar Jak vyu\v z\'\i t ideje z ha\v sov\'an\'\i\ s uspo\v r\'adan\'ymi 
\v ret\v ezci pro ostatn\'\i\ metody \v re\v sen\'\i\ koliz\'\i\ 
(jmeno\-vit\v e pro sr\accent23ustaj\'\i c\'\i\ ha\v sov\'an\'\i ).
\medskip

\flushpar Jakou metodu pou\v z\'\i t pro operaci {\bf DELETE} ve 
sr\accent23ustaj\'\i c\'\i m ha\v sov\'an\'\i\ (probl\'em je zachovat n\'ahodnost 
ulo\v zen\'e mno\v ziny a t\'\i m platnost odhadu na slo\v zitost operac\'\i ). 
\medskip

\flushpar Jak nal\'ezt druhou ha\v sovac\'\i\ funkci pro metodu 
dvojit\'eho ha\v so\-v\'a\-n\'\i , aby vznikl\'e posloupnosti adres p\v ri operaci 
{\bf INSERT} se chovaly jako n\'ahodn\'e?
\medskip

\subhead
Z\'av\v er
\endsubhead
\smallskip

\flushpar P\v ripome\v nme si p\v redpoklady pro p\v redchoz\'\i\ uveden\'e 
v\'ysledky o ha\v sov\'an\'\i :
\roster
\item
Ha\v sovac\'\i\ funkce se rychle spo\v c\'\i t\'a (v \v case 
$O(1)$);
\item Ha\v sovac\'\i\ funkce rovnom\v ern\v e rozd\v eluje univerzum (to 
znamen\'a, \v ze pro dv\v e r\accent23uzn\'e hodnoty $i$ a $j$ 
ha\v sovac\'\i\ funkce plat\'\i\ $-1\le |h^{-1}(i)|-|h^{-1}(j)|\le 
1$); 
\item 
Vstupn\'\i\ data jsou rovnom\v ern\v e rozd\v elen\'a.
\endroster

\flushpar Diskutujme splnitelnost t\v echto p\v redpoklad\accent23u.\newline
P\v redpoklad 1) je jasn\'y.\newline 
P\v redpoklad 2) -- je v\'yhodn\'e, 
kdy\v z rozd\v elen\'\i\ univerza ha\v sovac\'\i\ funkc\'\i\ kop\'\i ruje zn\'am\'e 
rozd\v ele\-n\'\i\ vstupn\'\i ch dat. Pou\v zilo se p\v ri n\'avrhu 
p\v reklada\v ce pro FORTRAN (Lum 1971). V n\'asle\-du\-j\'\i\-c\'\i\ tabulce jsou uveden\'e spo\v c\'\i tan\'e a nam\v e\v ren\'e v\'ysledky. Pou\v zila se metoda separovan\'ych \v ret\v ezc\accent23u. Byly teoreticky 
spo\v c\'\i tan\'e za na\v sich p\v redpoklad\accent23u. Experiment byl 
prov\'ad\v en pomoc\'\i\ ha\v sovac\'\i\ funkce, kter\'a preferovala 
obvykl\'e n\'azvy identifik\'ator\accent23u. V\'ysledky byly m\v e\v reny, 
kdy\v z se p\v reklada\v c FORTRANu pou\v zil pro standardn\'\i\ v\'ypo\v cet.
Porovn\'an\'\i\ v\'ysledku:

$$\vtop{\offinterlineskip\halign {\strut\vrule\ # & \vrule\ # & \vrule\ # & \vrule\ # & \vrule\ # & \vrule \hfil # \hfil \vrule\cr\noalign{\hrule}hodnota $\alpha$ & 0.5 & 0.6 & 0.7 & 0.8 & 0.9 \hfill \cr\noalign{\hrule}experiment & 1.19 & 1.25 & 1.28 & 1.34 & 1.38 \cr\noalign{\hrule}teorie & 1.25 & 1.30 & 1.35 & 1.40 & 1.45 \cr\noalign{\hrule}}}$$
\medskip

\flushpar Z\'av\v er: Podm\'\i nky 1) a 2) m\accent23u\v zeme splnit, 
kdy\v z zn\'ame rozlo\v zen\'\i\ vstupn\'\i ch dat, m\accent23u\v zeme 
dos\'ahnout je\v st\v e lep\v s\'\i ch v\'ysledk\accent23u.
\medskip

\flushpar Nev\'yhoda: Rozlo\v zen\'\i\ vstupn\'\i ch dat 
nem\accent23u\v zeme ovlivnit a obvykle ho ani nezn\'ame. Je 
re\'aln\'e, \v ze rozd\v elen\'\i\ vstupn\'\i ch dat bude nevhodn\'e pro 
pou\v zitou ha\v sovac\'\i\ funkci. D\accent23usledek -- na po\v c\'atku 
70. let se za\v calo ustupovat od ha\v sov\'an\'\i . Hledal se postup, 
kter\'y by se vyhnul uveden\'emu probl\'emu s bodem 3). Nalezen\'emu 
\v re\v sen\'\i\ je v\v enov\'an n\'asleduj\'\i c\'\i\ text.
\medskip


\heading
Univerz\'aln\'\i\ ha\v sov\'an\'\i
\endheading

\flushpar\v Re\v sen\'\i\ navrhli Carter a Wegman (1977), kdy\v z p\v ri\v sli s 
metodou univerz\'aln\'\i ho ha\v sov\'an\'\i , kter\'a obch\'az\'\i\ po\v za\-davek 3).  
To vedlo k nov\'emu rozs\'ahl\'emu pou\v z\'\i v\'an\'\i\ ha\v sov\'an\'\i .  
\medskip

\flushpar Z\'akladn\'\i\ idea: M\'\i sto jedn\'e funkce m\'ame 
mno\v zinu $H$ funkc\'\i\ z univerza do tabulky velikosti $m$ 
takov\'ych, \v ze pro ka\v zdou mno\v zinu $S\subseteq U$, $|S|\le 
m$ se 
v\v et\v sina funkc\'\i\ chov\'a dob\v re v\accent23u\v ci $S$ (tj. $
S$ 
spl\v nuje po\v zadavek 3)). Ha\v sovac\'\i\ funkci 
zvol\'\i me n\'ahodn\v e z $H$ (s rovnom\v ern\'ym rozd\v elen\'\i m) a ha\v sujeme  
pomoc\'\i\ takto zvo\-len\'e funkce. 
\medskip

\flushpar Modifikace ideje. Ov\v e\v rov\'an\'\i\ vlastnost\'\i\ vy\v zaduje 
znalost velikosti mno\v ziny $H$. Rychl\'a vy\v c\'\i slitelnost $
h(x)$ 
vy\v zaduje ana\-lytick\'e zad\'an\'\i\ funkc\'\i\ v $H$, ale zji\v st\v en\'\i\ rovnosti 
dvou analy\-ticky zadan\'ych funkc\'\i\ na univerzu $U$ je 
problematick\'e. \v Re\v sen\'\i m probl\'emu je pou\v zit\'\i\ indexov\'e mno\v ziny. 
To znamen\'a, \v ze $H=\{h_i\mid i\in I\}$ a dv\v e funkce jsou r\accent23uzn\'e, 
kdy\v z maj\'\i\ r\accent23uzn\'e indexy. Pak velikost syst\'emu bude 
velikost indexov\'e mno\v ziny. M\'\i sto zvolen\'\i\ ha\v sovac\'\i\ funkce 
budeme volit n\'ahodn\v e index s rovnom\v ern\'ym rozlo\v zen\'\i m a kdy\v z 
zvol\'\i me index $i$, pak budeme pracovat s ha\v sovac\'\i\ funkc\'\i\ $
h_i$. 
O\v cek\'avan\'a hodnota n\'ahodn\'e prom\v enn\'e $f$ z mno\v ziny $
I$ do 
re\'aln\'ych \v c\'\i sel bude pr\accent23um\v er p\v res $I$, tj. $\frac {
\sum_{i\in I}f(i)}{|I|}$.

\flushpar Form\'aln\v e: Nech\v t  $U$ je univerzum. Soubor  
funkc\'\i\ $H=\{h_i\mid i\in I\}$ z univerza $U$ do mno\v ziny $\{
0,1,\dots,m-1\}$ se 
naz\'yv\'a $c$-\emph{univerz\'aln\'\i} ($c$ je kladn\'e re\'aln\'e \v c\'\i slo), 
kdy\v z 
$$\forall x,y\in U,\,x\ne y\text{\rm\ plat\'\i\ }|\{i\in I\mid h_
i(x)=h_i(y)\}|\le\frac {c|I|}m.$$
\medskip

Jako ekvivalentn\'\i\ definici lze pou\v z\'\i t toto tvrzen\'\i: syst\'em funkc\'\i\ $H$ z univerza $U$ do mno\v ziny $\{0,1,\dots,m-1\}$ je $c$-univerz\'aln\'\i ,
kdy\v z vyb\'\i r\'ame $h\in H$ s rovnom\v ern\'ym rozd\v elen\'\i m, pak pro 
ka\v zd\'e dv\v e r\accent23uzn\'a $x,y\in U$, plat\'\i\ 
$$\Prob(h(x)=h(y))\le\frac cm.$$


\flushpar Probl\'emy: existence $c$-univerz\'aln\'\i ch 
syst\'em\accent23u, \newline 
vlastnosti $c$-univerz\'aln\'\i ch syst\'em\accent23u (zda spl\v nuj\'\i\ 
po\v zadovan\'e ide\-je).
\medskip

\subhead
Existence univerz\'aln\'\i ch syst\'em\accent23u
\endsubhead
\smallskip

\flushpar Univerzum $U=\{0,1,\dots,N-1\}$ pro prvo\v c\'\i slo $N$,\newline 
$H=\{h_{a,b}\mid (a,b)\in U\times U\}$,\newline 
kde $h_{a,b}(x)=((ax+b)\bmod N)\bmod m$\newline 
(tj. indexov\'a mno\v zina je $U\times U$ a jej\'\i\ velikost je $
N^2$).
\smallskip
\flushpar V\'yhoda: funkce z mno\v ziny $H$ um\'\i me rychle 
vy\v c\'\i slit.
\smallskip

\flushpar Zvolme $x,y\in U$ takov\'a, \v ze $x\ne y$. Chceme nal\'ezt 
$(a,b)\in U\times U$ takov\'e, \v ze $h_{a,b}(x)=h_{a,b}(y)$. 
\medskip

\flushpar Mus\'\i\ existovat $i\in \{0,1,\dots,m-1\}$ a 
$r,s\in \{0,1,\dots,\lceil\frac Nm\rceil -1\}$ tak, \v ze plat\'\i
$$\align(ax+b&\equiv i+rm)\bmod N\\
(ay+b&\equiv i+sm)\bmod N\endalign$$
Kdy\v z $x$, $y$, $i$, $r$ a $s$ jsou konstanty a $a$ a $b$ jsou 
prom\v enn\'e, je to syst\'em line\'arn\'\i ch rovnic v t\v elese 
$\Bbb Z/\bmod N$, kde $\Bbb Z$ jsou cel\'a \v c\'\i sla. Matice soustavy  
$$\pmatrix x\quad 1\\
y\quad 1\endpmatrix $$
je regul\'arn\'\i , proto\v ze $x\ne y$. Tedy existuje 
jedin\'e \v re\v sen\'\i\ t\'eto soustavy pro fixovan\'a $x$, $y$, $
i$, $r$ a $s$.
Pro dan\'a $x$ a $y$, $i$ nab\'yv\'a $m$ hodnot, $r$ a $s$ nab\'yvaj\'\i\ 
$\lceil\frac Nm\rceil$ hodnot. 

\flushpar Z\'av\v er: pro ka\v zd\'a $x,y\in U$ takov\'a, \v ze 
$x\ne y$, existuje $m\big(\lceil\frac Nm\rceil\big)^2$ dvojic $(a
,b)\in U\times U$ takov\'ych, \v ze 
$h_{a,b}(x)=h_{a,b}(y)$.
\medskip

\proclaim{V\v eta}Mno\v zina $H$ je $c$-univerz\'aln\'\i\ pro 
$$c=\frac {\big(\lceil\frac Nm\rceil\big)^2}{\big(\frac Nm\big)^2}
.$$
\endproclaim

\flushpar Skute\v cn\v e, pro ka\v zd\'e $x,y\in U$, $x\ne y$, je po\v cet 
$(a,b)\in U\times U$ takov\'ych, \v ze $h_{a,b}(x)=h_{a,b}(y)$, nejv\'y\v se roven
$$m\big(\lceil\frac Nm\rceil\big)^2=\frac {\big(\lceil\frac Nm\rceil\big
)^2}{\big(\frac Nm\big)^2}\frac {N^2}m=\frac {\big(\lceil\frac Nm
\rceil\big)^2}{\big(\frac Nm\big)^2}\frac {|I|}m.$$
\medskip

\flushpar Z\'av\v er: Dok\'azali jsme existenci $c$-univerz\'aln\'\i ch 
syst\'em\accent23u pro $c$ bl\'\i zk\'e $1$. Sta\v c\'\i\ si uv\v edomit, \v ze  ka\v zd\'e univerzum m\accent23u\v zeme 
pova\v zovat za univerzum tvaru $\{0,1,\dots,N-1\}$ pro n\v ejak\'e 
$N$ a \v ze mezi \v c\'\i sly $N$ a $2N$ v\v zdy existuje n\v ejak\'e 
prvo\v c\'\i slo.
\medskip

\subhead
Vlastnosti univerz\'aln\'\i ho ha\v sov\'an\'\i
\endsubhead
\smallskip

\flushpar P\v redpoklad: $H=\{h_i\mid i\in I\}$ je $c$-univerz\'aln\'\i\ syst\'em funkc\'\i\newline 
Ozna\v cen\'\i : Pro $i\in I$ a prvky $x,y\in U$ 
ozna\v cme 
$$\delta_i(x,y)=\cases 1&\quad\text{\rm\ kdy\v z }x\ne y\text{\rm\ a }
h_i(x)=h_i(y),\\
0&\text{\rm\ kdy\v z }x=y\text{\rm\ nebo }h_i(x)\ne h_i(y).\endcases $$
Pro mno\v zinu $S\subseteq U$, $x\in U$ a $i\in I$ definujme
$$\delta_i(x,S)=\sum_{y\in S}\delta_i(x,y).$$
\flushpar Pro fixovanou mno\v zinu $S\subseteq U$ a pro fixovan\'e $
x\in U$ se\v cteme
$\delta_i(x,S)$ p\v res v\v sechna $i\in I$:
$$\align\sum_{i\in I}\delta_i(x,S)=&\sum_{i\in I}\sum_{y\in S}\delta_
i(x,y)=\sum_{y\in S}\sum_{i\in I}\delta_i(x,y)=\\
&\sum_{y\in S,y\ne x}|\{i\in I\mid h_i(x)=h_i(y)\}|\le\\
&\sum_{y\in S,y\ne x}c\frac {|I|}m=\cases (|S|-1)c\frac {|I|}m&\text{\rm\ kdy\v z }
x\in S,\\
|S|c\frac {|I|}m&\text{\rm\ kdy\v z }x\notin S.\endcases \endalign$$
\flushpar Proto\v ze $\delta_i(x,S)$ d\'av\'a odhad na velikost \v ret\v ezce $
h_i(x)$ p\v ri 
repre\-zentaci mno\v ziny $S$ pomoc\'\i\ funkce $h_i$, dost\'av\'ame, \v ze 
o\v cek\'avan\'a d\'elka \v ret\v ezce pro fixovanou mno\v zinu  
$S\subseteq U$ a fixovan\'e $x\in U$ p\v res $i\in I$ s rovnom\v ern\'ym 
rozd\v elen\'\i m je nejv\'y\v se
$$\frac 1{|I|}\sum_{i\in I}\delta_i(x,S)\le\cases c\frac {|S|-1}m&\text{\rm\ kdy\v z }
\quad x\in S,\\
c\frac {|S|}m&\text{\rm\ kdy\v z }\quad x\notin S.\endcases $$

\proclaim{V\v eta}O\v cek\'avan\'y \v cas operac\'\i\ {\bf MEMBER}, {\bf INSERT }
a {\bf DE\-LETE} p\v ri $c$-univerz\'aln\'\i m ha\v sov\'an\'\i\ je $
O(1+c\alpha )$, kde 
$\alpha$ je faktor napl\-n\v en\'\i\ (tj. $\alpha =\frac {|S|}m$).

\flushpar O\v cek\'avan\'y \v cas pro pevnou posloupnost $n$ operac\'\i\ 
{\bf MEMBER}, {\bf INSERT} a {\bf DELETE} aplikovan\'ych na pr\'azdnou ta\-bulku 
pro $c$-univerz\'aln\'\i\ ha\v sov\'an\'\i\ je $O((1+\frac c2\alpha 
)n)$, kde $\alpha =\frac nm$.  
\endproclaim

\flushpar V\'yznam v\'ysledku:  Vzorec se jen o multiplikativn\'\i\ 
konstantu $c$ li\v s\'\i\ od vzorce pro ha\v sov\'an\'\i\ se separovan\'ymi 
\v ret\v ezci.  P\v ritom $c$ m\accent23u\v ze b\'yt jen o m\'alo men\v s\'\i\ ne\v z $
1$ a 
ve v\v sech zn\'am\'ych p\v r\'\i kla\-dech je $c\ge 1$.  Tak\v ze, co jsme 
dos\'ahli?  Rozd\'\i l je v p\v redpokla\-dech.  Zde je p\v redpoklad 3) 
nahrazen p\v redpokladem, \v ze index $i\in I$ je vybr\'an s 
rovnom\v ern\'ym rozd\v elen\'\i m, a nen\'\i\ \v z\'adn\'y p\v redpoklad na vstupn\'\i\ 
data.  {\bf V\'yb\v er indexu} $i$ {\bf m\accent23u\v zeme ov\-liv\-nit, ale v\'yb\v er 
vstupn\'\i ch dat nikoliv.}  M\accent23u\v zeme zajistit rovno\-m\v er\-n\'e 
rozd\v elen\'\i\ v\'yb\v eru $i$ z $I$ nebo se k tomuto rozd\v elen\'\i\ hodn\v e 
p\v ribl\'\i\v zit.  
\medskip

\subhead
Markovova nerovnost
\endsubhead
\smallskip

\flushpar P\v redpoklady: Je d\'ana mno\v zina $S\subseteq U$, prvek $
x\in U$. 
O\v cek\'a\-va\-n\'a velikost $\delta_i(x,S)$, je $\mu$, a $t\ge 
1$.
\medskip

\flushpar Uk\'a\v zeme pro $t>1$, \v ze pravd\v epodobnost, \v ze $
\delta_i(x,S)\ge t\mu$ 
pro $i\in I$, je men\v s\'\i\ ne\v z $\frac 1t$ (p\v redpoklad\'ame, \v ze $
i$ je z $I$ vybr\'ano s 
rovnom\v ern\'ym rozd\v elen\'\i m).  
\medskip

\flushpar Ozna\v cme $I'=\{i\in I\mid\delta_i(x,S)\ge t\mu \}$. Pak plat\'\i
$$\mu =\frac {\sum_{i\in I}\delta_i(x,S)}{|I|}>\frac {\sum_{i\in 
I'}\delta_i(x,S)}{|I|}\ge\frac {\sum_{i\in I'}t\mu}{|I|}=\frac {|
I'|}{|I|}t\mu$$
\flushpar Odtud
$$|I'|<\frac {|I|}t.$$
\flushpar Z\'av\v er: Pravd\v epodobnost, \v ze $\delta_i(x,S)\ge 
t\mu$, je men\v s\'\i\ 
ne\v z $\frac 1t$, a odtud plyne po\v zadovan\'e tvrzen\'\i . 
\medskip

\flushpar Pozn\'amka: Toto tvrzen\'\i\ plat\'\i\ obecn\v e a naz\'yv\'a 
se Markovova nerovnost. Uveden\'y d\accent23ukaz 
ilustruje jednoduch\'e tvrzen\'\i\ pro ko\-ne\v c\-n\'y p\v r\'\i pad.
\medskip

\subhead
Probl\'emy
\endsubhead
\smallskip

\flushpar Hlavn\'\i\ probl\'em: Zaji\v st\v en\'\i\ rovnom\v ern\'eho rozd\v elen\'\i\ 
v\'yb\v eru $i$ z $I$.
\medskip

\flushpar Proveden\'\i\ v\'yb\v eru: Zak\'odovat indexy z mno\v ziny $
I$ do 
\v c\'\i sel $0,1,\dots,|I|-1$. Zvolit n\'ahodn\v e \v c\'\i slo $
i$ z tohoto 
intervalu s rovnom\v ern\'ym rozd\v elen\'\i m a pak pou\v z\'\i t 
funkci s indexem, jeho\v z k\'od je $i$. Abychom vybrali $i$, nalezneme 
nejmen\v s\'\i\ $j$ takov\'e, \v ze $2^j-1\ge |I|-1$. Pak \v c\'\i sla v 
intervalu $\{0,1,\dots,2^j-1\}$ jednozna\v cn\v e koresponduj\'\i\ s 
posloupnostmi $0$ a $1$ d\'elky $j$. Budeme vyb\'\i rat n\'ahodn\v e 
posloupnost $0$ a $1$ d\'elky $j$. K v\'yb\v eru posloupnosti 
pou\v zijeme n\'ahodn\'y gener\'ator rovnom\v ern\'eho rozd\v ele\-n\'\i .
\medskip

\flushpar Z\'avada: Skute\v cn\'y n\'ahodn\'y gener\'ator pro 
rovnom\v ern\'e rozd\v ele\-n\'\i\ je prakticky nedosa\v ziteln\'y 
(n\v ekter\'e fyzik\'aln\'\i\ procesy). K dispozici je pouze 
pseudogener\'ator. 
\medskip

\flushpar Jeho nev\'yhoda: \v C\'\i m je $j$ v\v et\v s\'\i , t\'\i m je 
posloupnost pravideln\v ej\v s\'\i\ (tj. m\'en\v e n\'ahodn\'a). 
\medskip

\flushpar D\accent23usledky: Nal\'ezt co nejmen\v s\'\i\ $c$-univerz\'aln\'\i\ 
syst\'emy. Nal\'ezt doln\'\i\ odhady na jejich velikost.
\medskip

\subhead
Doln\'\i\ odhady na velikost
\endsubhead
\smallskip

\flushpar P\v redpoklady: Nech\v t $U$ je universum velikosti $N$ a nech\v t 
$H=\{h_i\mid i\in I\}$ je $c$-univerz\'aln\'\i\ syst\'em funkc\'\i\ ha\v suj\'\i c\'\i ch do tabulky velikosti 
$m$. M\accent23u\v zeme p\v redpokl\'a\-dat, \v ze $$I=\{0,1,\dots
,|I|-1\}.$$
\medskip

\flushpar Indukc\'\i\ definujme mno\v ziny $U_0,U_1,\dots$ tak, \v ze: 
$U_0=U$.\newline 
Nech\v t $U_1$ je nejv\v et\v s\'\i\ podmno\v zina $U_0$ vzhledem k po\v ctu 
prvk\accent23u takov\'a, \v ze $h_0(U_1)$ je jednoprvkov\'a 
mno\v zina. \newline 
Nech\v t $U_2$ je nejv\v et\v s\'\i\ podmno\v zina $U_1$ vzhledem k po\v ctu 
prvk\accent23u takov\'a, \v ze $h_1(U_2)$ je jednoprvkov\'a 
mno\v zina. \newline 
Nech\v t $U_3$ je nejv\v et\v s\'\i\ podmno\v zina $U_2$ vzhledem k po\v ctu 
prvk\accent23u takov\'a, \v ze $h_2(U_3)$ je jednoprvkov\'a 
mno\v zina. \newline 
Obecn\v e, nech\v t $U_i$ je nejv\v et\v s\'\i\ podmno\v zina $U_{
i-1}$ vzhledem k po\v ctu 
prvk\accent23u takov\'a, \v ze $h_{i-1}(U_i)$ je jednoprvkov\'a 
mno\v zina. 
\medskip

\flushpar Proto\v ze ha\v sujeme do tabulky velikosti $m$, plat\'\i\ 
$|U_i|\ge\lceil\frac {|U_{i-1}|}m\rceil$. Proto\v ze $|U_0|=N$, dost\'av\'ame indukc\'\i , \v ze 
$|U_i|\ge\lceil\frac N{m^i}\rceil$ pro ka\v zd\'e $i$. Zvolme $i=
\lceil\log_mN\rceil -1$. Pak $i$ je 
nejv\v et\v s\'\i\ p\v rirozen\'e \v c\'\i slo takov\'e, \v ze $\frac 
N{m^i}>1$. Tedy $U_i$ m\'a 
aspo\v n dva prvky, zvolme $x,y\in U_i$ takov\'a, \v ze $x\ne y$. Pak 
$h_j(x)=h_j(y)$ pro $j=0,1,\dots,i-1$. Tedy 
$$i\le |\{j\in I\mid h_j(x)=h_j(y)\}|\le\frac {c|I|}m.$$
\medskip

\proclaim{V\v eta}Kdy\v z $H=\{h_i\mid i\in I\}$ je $c$-univerz\'aln\'\i\ syst\'em pro 
univerzum $U$ o velikosti $N$ ha\v suj\'\i c\'\i\ do tabulky s $m$ \v r\'adky, pak 
$$|I|\ge\frac mc(\lceil\log_mN\rceil -1).$$
Posloupnosti $0$ a $1$ p\v ri n\'ahodn\'e volb\v e $i$ z $I$ mus\'\i\ m\'\i t 
d\'elku alespo\v n $\lceil (\log m-\log c+\log\log N-\log\log m)\rceil$ (zde v\v sechny logaritmy 
jsou o z\'akladu $2$).
\endproclaim
\medskip

\subhead
Mal\'y univerz\'aln\'\i\ syst\'em
\endsubhead
\smallskip

\flushpar Zkonstruujeme $c$-univerz\'aln\'\i\ syst\'em takov\'y, 
\v ze logaritmus z velikosti jeho indexov\'e mno\v ziny pro velk\'a 
univerza je a\v z na aditivn\'\i\ konstantu men\v s\'\i\ ne\v z 
$4(\log m+\log\log N)$, kde $N$ je velikost univerza a $m$ je po\v cet 
\v r\'adk\accent23u v tabulce.  
\medskip

\flushpar Nech\v t $p_1,p_2,\dots$ je rostouc\'\i\ posloupnost v\v sech 
prvo\v c\'\i sel. M\v ej\-me velikost tabulky $m$ a univerzum $U=\{0,1,\dots,N-1\}$ pro 
n\v ejak\'e p\v rirozen\'e \v c\'\i slo $N$ (nemus\'\i\ b\'yt prvo\v c\'\i slo). Nech\v t $t$ je 
nejmen\v s\'\i\ \v c\'\i slo takov\'e, \v ze $t\ln p_t\ge m\ln N$. Definujme 
$$H_1=\{g_{c,d}(h_{\ell})\mid t<\ell\le 2t,\,c,d\in \{0,1,\dots,p_{
2t}-1\}\},$$
kde $h_{\ell}(x)=x\bmod p_{\ell}$ a 
$g_{c,d}(x)=((cx+d)\bmod p_{2t})\bmod m$.\newline 
Uk\'a\v zeme, \v ze kdy\v z $m(\ln m+\ln\ln m)<N$, pak $H_1$ je $3.25$-univerz\'aln\'\i\ syst\'em. 
\medskip

\flushpar Nejprve si p\v ripomeneme zn\'amou v\v etu o velikosti prvo\v c\'\i sel 
(zde $\ln$ je p\v rirozen\'y logaritmus, tj. o z\'akladu $e$).

\proclaim{V\v eta}Pro ka\v zd\'e $i=1,2,\dots$ plat\'\i\ $p_i>i\ln 
i$ a pro $i\ge 6$ 
plat\'\i\ $p_i<i(\ln i+\ln\ln i)$. \qed
\endproclaim

\flushpar Tedy pro $i\ge 6$ plat\'\i\ $p_i<2i\ln i$.
\medskip

\flushpar Velikost indexov\'e mno\v ziny $H_1$.  Indexov\'a mno\v zina $
H_1$ je 
$$I=\{(c,d,\ell )\mid c,d\in \{0,1,\dots,p_{2t}-1,\,t<\ell\le 2t\}\}.$$  
Tedy $
|I|=tp_{2t}^2$.   
Odtud plyne $|I|\le 16t^3\ln^22t$ a tedy 
$$\log(|I|)\le 4+3\log t+2\log\log t.$$
Pro dostate\v cn\v e velk\'e $t$ (takov\'e, \v ze $\log t\ge 2\log\log 
t$, tj. $t\ge 16$) plat\'\i , 
\v ze $\log(|I|)\le 4+4\log t$.  Z definice $t$ plyne, 
\v ze $t\le m\ln N$, kdy\v z $\ln p_t\ge 1$ (tj.  $p_t\ge 3$).\newline 
Z\'av\v er: $\log(|I|)\le 4+4(\log m+\log\log N)$. 
\medskip

\subhead
Univerzalita mal\'eho syst\'emu
\endsubhead
\smallskip

\flushpar Zvolme r\accent23uzn\'a $x$ a $y$ z univerza $U$. 
Ozna\v c\'\i me
$$\align G_1&=\{(c,d,\ell )\mid g_{c,d}(h_{\ell}(x))=g_{c,d}(h_{\ell}
(y)),\,h_{\ell}(x)\ne h_{\ell}(y)\},\\
G_2&=\{(c,d,\ell )\mid g_{c,d}(h_{\ell}(x))=g_{c,d}(h_{\ell}(y)),\,
h_{\ell}(x)=h_{\ell}(y)\}\endalign$$
a odhadneme velikost $G_1$ a $G_2$. 
\medskip

\flushpar Odhad velikosti $G_1$. Kdy\v z $(c,d,\ell )\in G_1$, pak 
existuj\'\i\ $r,s\in \{0,1,\dots,\lceil\frac {
p_{2t}}m\rceil -1\}$ a $i\in \{0,1,\dots,m-1\}$ 
takov\'a, \v ze 
$$\align(c(x\bmod p_{\ell})+d&\equiv i+rm)\bmod p_{2t}\\
(c(y\bmod p_{\ell})+d&\equiv i+sm)\bmod p_{2t}.\endalign$$
Kdy\v z $c$ a $d$ pova\v zujeme za nezn\'am\'e, pak je to  
soustava line\'ar\-n\'\i ch rovnic s regul\'arn\'\i\ matic\'\i\ (proto\v ze 
$x\bmod p_{\ell}\ne y\bmod p_{\ell}$), a tedy pro ka\v zd\'e $\ell$, $
i$, $r$ a $s$ 
existuje nejv\'y\v se jedna dvojice $(c,d)$. Proto 
$$|G_1|\le tm(\lceil\frac {p_{2t}}m\rceil )^2\le\frac {tp_{2t}^2}
m(1+\frac m{p_{2t}})^2=\frac {|I|}m(1+\frac m{p_{2t}})^2.$$

\flushpar Odhad velikosti $G_2$. Ozna\v cme 
$L=\{\ell\mid t<\ell\le 2t,\,x\bmod p_{\ell}=y\bmod p_{\ell}\}$ a $
P=\prod_{\ell\in L}p_{\ell}$. 
Proto\v ze $P$ d\v el\'\i\ $|x-y|$, dost\'av\'ame, \v ze $P\le N$. Proto\v ze 
$p_t<p_{\ell}$ pro ka\v zd\'e $\ell\in L$, dost\'av\'ame, \v ze $
P>p_t^{|L|}$. Tedy 
$|L|\le\frac {\ln N}{\ln p_t}\le\frac tm$ z definice $t$. Proto\v ze $
(c,d,\ell )\in G_2$, pr\'av\v e 
kdy\v z $\ell\in L$ a $c,d\in \{0,1,\dots,p_{2t}-1\}$, shrneme, \v ze 
$$|G_2|\le |L|p_{2t}^2\le\frac {tp_{2t}^2}m=\frac {|I|}m.$$
\medskip

\flushpar Abychom odhadli $(1+\frac m{p_{2t}})^2$, uk\'a\v zeme si nejd\v r\'\i v 
pomocn\'e lem\-ma.

\proclaim{Lemma}Kdy\v z $t\ge 6$ a $m(\ln m+\ln\ln m)<N$, pak $m<\frac {
p_t}{\ln t}$.
\endproclaim

\demo{D\accent23ukaz}P\v redpokl\'adejme, \v ze tvrzen\'\i\ neplat\'\i. pak$m\ge\frac {p_t}{\ln 
t}$. Z V\v ety o velikosti 
prvo\-\v c\'\i\-sel pak plyne $m\ge\frac {p_t}{\ln t}>\frac {t\ln 
t}{\ln t}=t$. Kdy\v z pou\v zijeme, \v ze 
$m(\ln m+\ln\ln m)<N$, tak dostaneme, \v ze 
$$\ln m+\ln(\ln m+\ln\ln m)<\ln N,$$
a odtud plyne, \v ze 
$$t\ln p_t<t\ln(t(\ln t+\ln\ln t))\le m(\ln m+\ln(\ln m+\ln\ln m))<m\ln N$$
a to je spor s definic\'\i\ $t$. Tedy $m<\frac {p_t}{\ln t}$. \qed
\enddemo

\flushpar Nyn\'\i\ zkombinujeme V\v etu o odhadu velikosti prvo\v c\'\i sel, 
p\v red\-cho\-z\'\i\ Lemma a fakt, \v ze 
$$\ln2t\ge\ln t\ge\ln\ln t\qquad\text{pro v\v sechna }t\ge 1$$
a dostaneme, \v ze
$$\frac m{p_{2t}}\le\frac {\frac {p_t}{\ln t}}{2t\ln2t}<\frac {t(\ln 
t+\ln\ln t)}{2t\ln t\ln2t}<\frac 1{\ln 2t}(1+\frac{\ln\ln t}{\ln t}).$$
Je z\v rejm\'e, \v ze tento v\'raz je men\v s\'\i\ ne\v z $\frac 12$ 
a kdy\v z konverguje k $+\infty$ pak tento v\'yraz konverguje k $0$.
\medskip

\flushpar Z toho plyne, \v ze $(1+\frac m{p_{2t}})^2\le 1.5^2=2.2
5$ a tedy 
$$\align|\{i\in I&\mid h_i(x)=h_i(y)\}|=|G_1|+|G_2|\le\\
&\frac {|I|}m(1+\frac m{p_{2t}})^2+\frac {|I|}m\le\frac {|I|}m(1+
2.25)=3.25\frac {|I|}m.\endalign$$
\medskip

\flushpar Shrnut\'\i : Kdy\v z $t\ge 6$ a $m\ln m\ln\ln m<N$, pak $
H_1$ je 
$3.25$-univerz\'aln\'\i . Bez jak\'ychkoliv p\v redpoklad\accent23u lze 
uk\'azat, \v ze $H_1$ je $5$-univerz\'aln\'\i .
\medskip

\subhead
Odhad na velikost $c$
\endsubhead
\smallskip

\proclaim{V\v eta}Kdy\v z $H$ je $c$-univerz\'aln\'\i\ 
syst\'em univerza $U$ o velikosti $N$ ha\v suj\'\i c\'\i\ do tabulky s $
m$ \v r\'adky, 
pak $c\ge 1-\frac mN$.
\endproclaim
\medskip

Nejprve dok\'a\v zeme technick\'e lemma.

\proclaim{Lemma}M\v ejme re\'aln\'a \v c\'\i sla $b_i$ pro $i=0,1
,\dots,m-1$ a 
nech\v t $b=\sum_{i=0}^{m-1}b_i$. Pak $$\sum_{i=0}^{m-1}b_i(b_i-1)
\ge b(\frac bm-1).$$
\endproclaim

\demo{D\accent23ukaz lemmatu}Z Cauchyho-Schwarzovy nerovnosti 
$$(\sum_{i=0}^{m-1}x_iy_i)^2\le (\sum_{i=0}^{m-1}x^2_i)(\sum_{i=0}^{
m-1}y_i^2)$$
plyne $(\sum_{i=0}^{m-1}b_i)^2=b^2\le m(\sum_{i=0}^{m-1}b_i^2)$, sta\v c\'\i\ polo\v zit 
$x_i=b_i$ a $y_i=1$, a tedy $\frac {b^2}m\le\sum_{i=0}^{m-1}b_i^2$. Odtud
$$\sum_{i=0}^{m-1}b_i(b_i-1)=\sum_{i=0}^{m-1}b_i^2-\sum_{i=0}^{m-
1}b_i=\sum_{i=0}^{m-1}b_i^2-b\ge\frac {b^2}m-b=b(\frac bm-1)$$
a lemma je dok\'az\'ano. \qed
\enddemo

\demo{D\accent23ukaz V\v ety}M\v ejme funkci $f:U@>>>S$, kde $U$ m\'a 
velikost $N$ a $S$ m\'a velikost $m$. Ozna\v cme $A$ mno\v zinu 
uspo\v r\'adan\'ych dvojic $u,v\in U$ takov\'ych, \v ze $u\ne v$ a $
f(u)=f(v)$.
Kdy\v z pro $s\in S$ ozna\v c\'\i me $k_s=|f^{-1}(s)|$, pak $|A|=
\sum_{s\in S}k_s(k_s-1)$.
Z lemmatu plyne, \v ze 
$$|A|=\sum_{s\in S}k_s(k_s-1)\ge N(\frac Nm-1)=N(\frac {N-m}m),$$
proto\v ze $\sum_{s\in S}k_s=N$.\newline 
Kdy\v z $H=\{h_i\mid i\in I\}$ je $c$-univerz\'aln\'\i\ syst\'em funkc\'\i\ z univerza $
U$ o 
velikosti $N$ do tabulky o velikosti $m$, pak pomoc\'\i\ 
lemmatu dost\'av\'ame 
$$\align&|I|N(\frac {N-m}m)\le\\
&\sum_{i\in I}|\{(x,y)\in U\times U\mid h_i(x)=h_i(y),\,x\ne y\}|
=\\
&\sum_{(x,y)\in U\times U,\,x\ne y}|\{i\in I\mid h_i(x)=h_i(y)\}|
\le\\
&\sum_{(x,y)\in U\times U,x\ne y}c\frac {|I|}m=N(N-1)c\frac {|I|}
m.\endalign$$
Odtud plyne, \v ze $N-m\le c(N-1)$, a tedy 
$$c\ge\frac {N-m}{N-1}>\frac {N-m}N=1-\frac mN.\qed$$
\enddemo


\subhead
Probl\'emy univerz\'aln\'\i ho ha\v sov\'an\'\i
\endsubhead
\smallskip

\flushpar Pou\v z\'\i t jin\'e metody na \v re\v sen\'\i\ koliz\'\i\ ne\v z 
separovan\'e \v ret\v ezce. Jak to ovlivn\'\i\ pou\v zitelnost 
univerz\'aln\'\i ho ha\v sov\'an\'\i ? Plat\'\i\ podobn\'e vztahy jako pro 
pevn\v e danou ha\v sovac\'\i\ funkci? Jak\'y vliv na efektivnost 
m\'a nep\v r\'\i tomnost operace {\bf DELETE}?
\medskip

\flushpar Existuje $c$-univerz\'aln\'\i\ ha\v sovac\'\i\ syst\'em pro 
$c<1$? Jak\'y je vztah mezi velikost\'\i\ $c$-univerz\'aln\'\i\-ho 
ha\v sovac\'\i ho syst\'emu a velikost\'\i\ $c$? Lze zkonstruovat 
mal\'y $c$-univerz\'aln\'\i\ syst\'em pro $c<3.25$? Zde hraje roli 
fakt, \v ze p\v ri $c=3.25$ se o\v cek\'avan\'a d\'elka \v ret\v ezce 
m\accent23u\v ze pohybovat a\v z kolem hodnoty $7$.
\medskip

\flushpar Pou\v zit\'\i\ \v Ceby\v sevovy nerovnosti m\'\i sto 
Markovovy nerovnosti d\'av\'a kvadratick\'y odhad 
prav\-d\v e\-podobnosti, \v ze d\'elka \v ret\v ezce je o $t$ v\v et\v s\'\i\ ne\v z 
o\v cek\'avan\'a hodnota. Za jak\'ych okolnost\'\i\ d\'av\'a lep\v s\'\i\ 
odhad? Lze pou\v z\'\i t i vy\v s\v s\'\i ch moment\accent23u?
\medskip

\flushpar Jak pou\v z\'\i t Markovou nerovnost a o\v cek\'avanou 
d\'elku maxim\'al\-n\'\i\-ho \v ret\v ezce pro odhad o\v cek\'a\-va\-n\'eho 
po\v ctu voleb ha\v sovac\'\i\ funk\-ce? 
Pro jak\'e parametry lze pou\v z\'\i t n\'asleduj\'\i c\'\i\ model?

\flushpar Je d\'ana z\'akladn\'\i\ velikost tabulky $m$ a d\'ale pro $
j=0,1,\dots$ 
\v c\'\i sla (parametry) $l_j$ a $c$-univer\-z\'al\-n\'\i\ ha\v sovac\'\i\ syst\'emy 
$H_j=\{h_i\mid i\in I_j\}$ z univerza do tabulky s $m2^j$ \v r\'adky.\newline 
Mno\v zina $S\subseteq U$ je reprezentov\'ana n\'asledovn\v e:  je d\'ano $
j$ 
tako\-v\'e, \v ze kdy\v z $j>0$, pak $m2^{j-2}\le |S|\le m2^j$, kdy\v z $
j=0$, pak 
$|S|\le m$, a je zvolen index $i\in I_j$.  D\'ale m\'ame prost\'e \v ret\v ezce 
$r_0,r_1,\dots,r_{m2^j-1}$, jejich\v z d\'elky jsou nejv\'y\v se $
l_j$, a \v ret\v ezec $r_k$ 
obsahuje prvky $\{s\in S\mid h_i(s)=k\}$.\newline 
Operace {\bf INSERT$(x)$} prohled\'a \v ret\v ezec $r_{h_i(x)}$ a kdy\v z tento 
\v re\-t\v e\-zec neobsahuje prvek $x$, pak ho p\v rid\'a.  Kdy\v z 
$m2^{j-2}\le |S|\le m2^j$ a d\'elka \v ret\v ezce $r_{h_i}(x)$ je nejv\'y\v se $
l_j$, pak 
operace kon\v c\'\i .  Kdy\v z $|S|>m2^j$, tak se nejd\v r\'\i ve zv\v et\v s\'\i\ $
j$ o $1$.  
Pak se n\'ahodn\v e zvol\'\i\ $i\in I_j$ a zkonstruuj\'\i\ se \v ret\v ezce 
reprezentuj\'\i c\'\i\ $S$.  Kdy\v z n\v ekter\'y z nich m\'a d\'elku v\v et\v s\'\i\ ne\v z $
l_j$, 
tak se volba a konstrukce \v ret\v ezc\accent23u opakuje tak dlouho, 
dokud se nepovede zvolit $i\in I_j$ takov\'e, \v ze v\v sechny 
zkonstruovan\'e \v ret\v ezce maj\'\i\ d\'elku nejv\'y\v se $l_j$.  Operace {\bf DELETE }
se \v re\v s\'\i\ analogicky.\newline 
Probl\'em:  Jak volit parametry $l_i$?  
\medskip


\flushpar V p\v r\'\i pad\v e \v re\v sen\'\i\ 
koliz\'\i\ dvojit\'ym ha\v sov\'an\'\i m nebo ha\v sov\'an\'\i m s 
line\'arn\'\i m p\v rid\'av\'an\'\i m je t\v reba d\'at siln\v ej\v s\'\i\ podm\'\i nky na velikost $|S|$. V posledn\'\i\ dob\v e se t\'eto t\'ematice 
v\v enuje pozornost a byla dosa\v zena \v rada zaj\'\i mav\'ych v\'ysledk\accent23u.

\heading
Perfektn\'\i\ ha\v sov\'an\'\i
\endheading

\flushpar Jin\'e \v re\v sen\'\i\ koliz\'\i\ je perfektn\'\i\ ha\v sov\'an\'\i . Idea 
je nal\'ezt pro danou mno\v zinu ha\v sovac\'\i\ funkci, kter\'a 
nevytv\'a\v r\'\i\ kolize.
\medskip

\flushpar Nev\'yhoda: Metoda nep\v ripou\v st\'\i\ operaci {\bf INSERT }
(pro nov\'y vstup nem\accent23u\v zeme zaru\v cit, \v ze nevznikne  
kolize). Metodu lze prakticky pou\v z\'\i t pro \'ulohy, kde lze 
o\v cek\'avat hodn\v e operac\'\i\ {\bf MEMBER} a operace {\bf INSERT} se 
t\'em\v e\v r nevyskytuje (kolize se \v re\v s\'\i\ pomoc\'\i\ mal\'e pomocn\'e 
tabulky, kam se ukl\'adaj\'\i\ koliduj\'\i\-c\'\i\ data). Tato metoda se 
pou\v z\'\i v\'a p\v ri navrhov\'an\'\i\ kompil\'ator\accent23u.
\medskip

\flushpar Zad\'an\'\i\ \'ulohy: Pro danou mno\v zinu $S\subseteq 
U$ chceme 
nal\'ezt ha\v so\-va\-c\'\i\ funkci $h$ takovou, \v ze
\roster
\item
pro $s,t\in S$ takov\'e, \v ze $s\ne t$, plat\'\i\ $h(s)\ne h(
t)$ (tj. $h$ je \emph{perfektn\'\i} \emph{ha\v sovac\'\i} \emph{funkce} pro 
$S$); 
\item 
$h$ ha\v suje do tabulky s $m$ \v r\'adky, kde $m$ je p\v ribli\v zn\v e 
stejn\v e velk\'e jako $|S|$ (nen\'\i\ praktick\'e ha\v sovat do p\v r\'\i li\v s 
velk\'ych tabulek -- ztr\'ac\'\i\ se jeden ze z\'akladn\'\i ch 
d\accent23uvod\accent23u pro ha\v sov\'an\'\i); 
\item 
$h$ mus\'\i\ b\'yt rychle spo\v citateln\'a -- jinak ha\v sov\'an\'\i\ 
nen\'\i\ rychl\'e;
\item 
ulo\v zen\'\i\ $h$ nesm\'\i\ vy\v zadovat moc pam\v eti, 
nejv\'yhodn\v ej\v s\'\i\ je ana\-lytick\'e zad\'an\'\i\ (kdy\v z zad\'an\'\i\ $
h$ bude 
vy\v zadovat moc pam\v eti, nap\v r. kdy\v z by byla d\'ana tabulkou, 
pak se ztr\'ac\'\i\ d\accent23uvod k pou\v zit\'\i\ stejn\v e jako v 
bod\v e 2).
\endroster


\flushpar Kompenzace: Nalezen\'\i\ ha\v sovac\'\i\ funkce 
m\accent23u\v ze spot\v rebovat v\'\i ce \v casu. Prov\'ad\'\i\ se jen na 
za\v c\'atku \'ulohy. 
\medskip

\flushpar Uveden\'e po\v zadavky motivuj\'\i\ zaveden\'\i\ 
n\'asleduj\'\i c\'\i ho pojmu.\newline 
M\v ejme univerzum $U=\{0,1,\dots,N-1\}$. Soubor funkc\'\i\ $H$ z 
$U$ do mno\v ziny $\{0,1,\dots,m-1)$ se naz\'yv\'a 
$(N,m,n)$-\emph{perfektn\'\i}, kdy\v z pro ka\v zdou $S\subseteq 
U$ 
takovou, \v ze $|S|=n$, existuje $h\in H$ perfektn\'\i\ pro $S$ (tj. 
$h(s)\ne h(t)$ pro ka\v zd\'a dv\v e r\accent23uzn\'a $s,t\in S$).

\flushpar Proto\v ze nev\'\i me, zda takov\'a $h$ existuj\'\i , nejprve 
vy\v set\v r\'\i me mno\-\v zi\-ny perfektn\'\i ch ha\v sovac\'\i ch 
funkc\'\i . Vy\v set\v r\'\i me vlastnosti $(N,m,n)$-perfektn\'\i ch 
soubor\accent23u funkc\'\i .
\medskip

\subhead
Doln\'\i\ odhady na velikost $(N,m,n)$-perfektn\'\i ho souboru
\endsubhead
\smallskip

\flushpar P\v redpokl\'adejme, \v ze $H$  je $(N,m,n)$-perfektn\'\i\ 
syst\'em pro $U=\{0,1,\dots,N-1\}$ a nejprve nalezneme doln\'\i\ 
odhady na velikost $|H|$.
\medskip

\flushpar M\v ejme funkci $h$ z $U$ do mno\v ziny $\{0,1,\dots,m-
1\}$. 
Nalezneme po\v cet mno\v zin $S\subseteq U$ takov\'ych, \v ze  
$h$ je perfektn\'\i\ funkce pro $S$ a $|S|=n$. Funkce $h$ je perfektn\'\i\ pro 
$S\subseteq U$, pr\'av\v e kdy\v z pro ka\v zd\'e $i=0,1,\dots,m-
1$ je 
$|h^{-1}(i)\cap S|\le 1$. Odtud po\v cet t\v echto mno\v zin je
$$\sum \{\prod_{j=0}^{n-1}|h^{-1}(i_j)|\mid 0\le i_0<i_1<\dots<i_{
n-1}<m\}.$$
\flushpar Vysv\v etlen\'\i : $h(S)=\{i_j\mid j=0,1,\dots,n-1\}$.
\medskip

\flushpar Toto \v c\'\i slo je maxim\'aln\'\i , kdy\v z $|h^{-1}(
i)|=\frac Nm$ pro ka\v zd\'e $i$. 
Tedy $h$ m\accent23u\v ze b\'yt perfektn\'\i\ nejv\'y\v se pro $\binom 
mn(\frac Nm)^n$ 
mno\v zin (\v c\'\i slo $\binom mn$ ur\v cuje po\v cet posloupnost\'\i\ 
$0\le i_0<i_1<\dots<i_{n-1}<m$). Proto\v ze $n$-prvkov\'ych 
podmno\v zin universa je $\binom Nn$, dost\'av\'ame, \v ze 
$$|H|\ge\frac {\binom Nn}{\binom mn(\frac Nm)^n}.$$

\flushpar Jin\'y odhad velikosti $(N,m,n)$-perfektn\'\i ho souboru.
\medskip

\flushpar P\v redpokl\'adejme, \v ze $H=\{h_1,\dots,h_t\}$ je 
$(N,m,n)$-perfektn\'\i\ soubor funkc\'\i .  Definujme indukc\'\i\ 
soubor mno\v zin $U_i$:\newline 
$U_0=U$ a pro 
$i>0$ je $U_i$ nejv\v et\v s\'\i\ podmno\v zina $U_{i-1}$, co do po\v ctu 
prvk\accent23u, 
takov\'a, \v ze $h_i$ je konstantn\'\i\ na $U_i$.  Pak $|U_i|\ge\frac {
|U_{i-1}|}m$ 
pro v\v sechna $i>0$.  Z $|U_0|=N$ plyne $|U_i|\ge\frac N{m^i}$.  Pro ka\v zd\'e 
$i=1,2,\dots,t$ je $h_j(U_i)$ jednobodov\'a mno\v zina pro ka\v zd\'e 
$j\le i$.  Proto \v z\'adn\'a $h_j$ pro $j\le i$ nen\'\i\ perfektn\'\i\ pro 
mno\v zinu $S\subseteq U$ takovou, \v ze $|S\cap U_i|\ge 2$.  Proto\v ze $
H$ je 
$(N,m,n)$-perfektn\'\i , mus\'\i\ b\'yt $|U_t|\le 1$, a tedy $\frac 
N{m^t}\le 1$.  Proto 
$t\ge\frac {\log N}{\log m}$.    
\medskip

\proclaim{V\v eta}Kdy\v z $H$ je $(N,m,n)$-perfektn\'\i\ soubor 
funkc\'\i , pak 
$$|H|\ge\max\{\frac {\binom Nn}{\binom mn(\frac Nm)^n},\frac {\log 
N}{\log m}\}.$$
\endproclaim

\subhead
Existence $(N,m,n)$-perfektn\'\i ho souboru
\endsubhead
\smallskip

\flushpar M\v ejme univerzum $U=\{0,1,\dots,N-1\}$ a soubor 
funkc\'\i\ $H=\{h_1,h_2,\dots,h_t\}$ z univerza $U$ do mno\v ziny 
$\{0,1,\dots,m-1\}$.  Reprezentujeme tento soubor pomoc\'\i\ 
matice $M(H)$ typu $N\times t$ s hodnotami $\{0,1,\dots,m-1\}$ tak, 
\v ze pro $x\in U$ a $i=1,2,\dots,t$ je 
v $x$-t\'em \v r\'adku a $i$-t\'em sloupci matice $M(H)$ 
hodnota $h_i(x)$. Pak \v z\'adn\'a funkce z $H$ 
nen\'\i\ perfektn\'\i\ pro mno\v zinu $S=\{s_1,s_2,\dots,s_n\}\subseteq 
U$, 
pr\'av\v e kdy\v z podmatice $M(H)$ tvo\v re\-n\'a 
\v r\'adky $s_1$, $s_2$, a\v z $s_n$ a v\v semi sloupci nem\'a prost\'y sloupec. 
Ta\-ko\-v\'ych matic je nejv\'y\v se 
$$(m^n-\prod_{i=0}^{n-1}(m-i))^tm^{(N-n)t}.$$
\flushpar Vysv\v etlen\'\i :  $m^n$ je po\v cet v\v sech funkc\'\i\ z $
S$ do 
$\{0,1,\dots,m-1\}$, $\prod_{i=0}^{n-1}(m-i)$ je po\v cet prost\'ych funkc\'\i\ z $
S$ 
do $\{0,1,\dots,m-1\}$, a tedy po\v cet v\v sech podmatic s $n$ 
\v r\'adky takov\'ych, \v ze \v z\'adn\'y jejich sloupec nen\'\i\ prost\'y, je 
$(m^n-\prod_{i=0}^{n-1}(m-i))^t$.  Tyto podmatice m\accent23u\v zeme 
libovoln\v e doplnit na matici typu $N\times n$ a pro ka\v zdou 
matici je t\v echto dopln\v en\'\i\ $m^{(N-n)t}$.  
\medskip

\flushpar Podmno\v zin $U$ velikosti $n$ je $\binom Nn$, tedy po\v cet v\v sech 
matic, kter\'e nereprezentuj\'\i\ $(N,m,n)$-perfektn\'\i\ syst\'em, je 
men\v s\'\i\ nebo rovno  
$$\binom Nn(m^n-\prod_{i=0}^{n-1}(m-i))^tm^{(N-n)t}.$$

\flushpar V\v sech matic je $m^{Nt}$ a kdy\v z 
$$\binom Nn(m^n-\prod_{i=0}^{n-1}(m-i))^tm^{(N-n)t}<m^{Nt},\tag *$$
\flushpar pak nutn\v e existuje $(N,m,n)$-perfektn\'\i\ syst\'em. 
N\'asleduj\'\i c\'\i\ v\'y\-ra\-zy jsou ekvivalentn\'\i\ s ne\-rov\-nost\'\i\ 
\thetag{$*$}
$$\binom Nn\big(1-\frac {\prod_{i=0}^{n-1}(m-i)}{m^n}\big)^t<1\,\Leftrightarrow\,
t\ge\frac {\ln\binom Nn}{-\ln(1-\frac {\prod_{i=0}^{n-1}(m-i)}{m^
n})}.$$
Proto\v ze $\ln\binom Nn\le n\ln N$ a proto\v ze $-\ln(1-x)\ge x$ pro $x\in (0,1)$ dost\'av\'ame
$$\align-\ln(1-\frac {\prod_{i=0}^{n-1}(m-i)}{m^n})\ge&\prod_{i=0}^{
n-1}(1-\frac im)=e^{\sum_{i=0}^{n-1}\ln(1-\frac im)}\ge\\
&e^{\int_0^n\ln(1-\frac xm)dx}\endalign$$
kde integr\'al m\accent23u\v zeme odhadnout 
$$m[(1-\frac nm)(1-\ln(1-\frac nm))-1]\ge m[(1-\frac nm)(1+\frac 
nm)-1]=-\frac {n^2}m,$$
dost\'av\'ame, \v ze kdy\v z $t\ge n(\ln N)e^{\frac {n^2}m}$, pak \thetag{*} plat\'\i , a tedy 
existuje $(N,m,n)$-perfektn\'\i\ soubor funkc\'\i . 
\medskip

\flushpar Existence $(N,m,n)$-perfektn\'\i ho souboru funkc\'\i\ ale 
nezaru\v cuje spln\v en\'\i\ po\v zadavk\accent23u 2), 3) a 4).  
Abychom usp\v eli, pou\v zijeme ideu z metody univerz\'aln\'\i ho 
ha\v sov\'an\'\i .  
\medskip

\subhead
Konstrukce perfektn\'\i\ ha\v sovac\'\i\ funkce
\endsubhead
\smallskip

\flushpar P\v redpoklady:  $U=\{0,1,\dots,N-1\}$, kde $N$ je 
prvo\v c\'\i slo.  M\v ejme $S\subseteq U$ o velikosti $n$.  Budeme 
uva\v zovat funkce 
$$h_k(x)=(kx\bmod N)\bmod m\qquad\text{\rm\ pro }k=1,2,\dots,N-1.$$
Pro $i=0,1,\dots,m-1$ a $k=1,2,\dots,N-1$ ozna\v cme 
$$b_i^k=|\{x\in S\mid (kx\bmod N)\bmod m=i\}|.$$
\medskip

\flushpar V\'yznam $b_i^k$: Hodnoty $b_i^k$ lze pova\v zovat za 
veli\v ciny, kter\'e ukazuj\'\i\ odchylku od perfektnosti. 
V\v simn\v eme si, \v ze 
$$\text{\rm kdy\v z }b_i^k\ge 2,\text{\rm\ pak }(b_i^k)^2-b_i^k\ge 
2,$$
\flushpar proto\v ze $a^2-a\ge 2$, kdy\v z $a\ge 2$. Na druhou stranu 
$$b_i^k\le 1\text{\rm\ implikuje }(b_i^k)^2-b_i^k=0.$$
\flushpar Tedy z $\sum_{i=0}^{m-1}b_i^k=n$ plyne 

\proclaim{V\v eta}Funkce $h_k$ je perfektn\'\i , pr\'av\v e kdy\v z 
$\sum_{i=0}^{m-1}(b_i^k)^2-n<2$.
\endproclaim

\flushpar Nyn\'\i\ odhadneme v\'yraz $\sum_{k=1}^{N-1}\big((\sum_{
i=0}^{m-1}(b_i^k)^2)-n\big)$.
$$\align\sum_{k=1}^{N-1}\big(&(\sum_{i=0}^{m-1}(b_i^k)^2)-n\big)=\\
&\sum_{k=1}^{N-1}\big((\sum_{i=0}^{m-1}|\{x\in S\mid h_k(x)=i\}|^
2)-n\big)=\\
&\sum_{k=1}^{N-1}|\{(x,y)\mid x,y\in S,\,x\ne y,\,h_k(x)=h_k(y)\}
|=\\
&\sum_{x,y\in S,x\ne y}|\{k\mid 1\le k<N,\,h_k(x)=h_k(y)\}|.\endalign$$
Zvolme $x,y\in S$ takov\'a, \v ze $x\ne y$. Pak $h_k(x)=h_k(y)$, 
pr\'av\v e kdy\v z existuje $i=0,1,\dots,m-1$ a $r,s=0,1,\dots,\lfloor\frac 
Nm\rfloor$ 
takov\'a, \v ze $kx\equiv i+rm\bmod N$ a $ky\equiv i+sm\bmod N$ a $
i+rm,i+sm<N$. Odtud dost\'av\'ame, \v ze $h_k(x)=h_k(y)$ implikuje $
kx-ky\equiv (r-s)m\bmod N$.
Proto\v ze $0<k<n$ a $x\ne y$, dost\'av\'ame, \v ze $kx-ky\ne 0$, a tedy 
$h_k(x)=h_k(y)$ implikuje existenci 
$q=-\lfloor\frac Nm\rfloor ,-\lfloor\frac Nm\rfloor +1,\dots,-1,1
,2,\dots,\lfloor\frac Nm\rfloor$ takov\'eho, \v ze 
$kx-ky\equiv qm\bmod N$,
a to je 
ekvivalentn\'\i\ s t\'\i m, \v ze $k(x-y)\equiv qm\bmod N$ pro n\v ejak\'e 
$q=-\lfloor\frac Nm\rfloor ,-\lfloor\frac Nm\rfloor +1,\dots,-1,1
,2,\dots,\lfloor\frac Nm\rfloor$.
\medskip

\flushpar Pro $x>y$ a pro jedno $q=0,1,\dots,\lfloor\frac Nm\rfloor$ e\-xis\-tuje 
pr\'av\v e jedno $k$ takov\'e, \v ze $k(x-y)\equiv qm\bmod N$, proto\v ze 
$\Bbb Z_N$ je t\v eleso (tato rovnice m\'a jedin\'e \v re\v sen\'\i ). 
Proto\v ze pro $q=-\lfloor\frac Nm\rfloor ,\dots,-1,0$ je rovnice $
k(x-y)\equiv qm\bmod N$ 
ekvivalentn\'\i\ s rovnic\'\i\ $k(x-y)\equiv N+qm\bmod N$, tak 
dost\'av\'ame, \v ze pro $x,y\in S$, $x<y$ existuje nejv\'y\v se 
$2\lfloor\frac Nm\rfloor =2\lfloor\frac {N-1}m\rfloor$ r\accent23uzn\'ych $
k=1,2,\dots,N-1$, \v ze $h_k(x)=h_k(y)$ (nen\'\i\ pravda, \v ze kdy\v z 
$k$ spl\v nuje rovnici $k(x-y)\equiv qm\bmod N$ pro n\v ejak\'e 
$q=-\lfloor\frac Nm\rfloor ,\dots,-1,1,\dots,\lfloor\frac Nm\rfloor$, 
pak $h_k(x)=h_k(y)$). Stejn\'y odhad analogicky dostaneme, kdy\v z $
x<y$ 
(ale dost\'av\'ame jin\'a \v re\v sen\'\i ). Odtud 
$$\sum_{k=1}^{N-1}\big((\sum_{i=0}^{m-1}(b_i^k)^2)-n\big)\le
\sum_{x,y\in S,x\ne y}2(\frac {N-1}m)\le2(N-1)\frac {n(n-1)}m.$$
\smallskip

\flushpar Tedy existuje $k$ takov\'e, \v ze $\sum_{i=0}^{m-1}(b_i^
k)^2\le 2\frac {n(n-1)}m+n$. 
\medskip

\flushpar Uk\'a\v zeme, \v ze existuje v\'\i ce ne\v z $\frac {N-
1}4$ takov\'ych 
$k$, \v ze plat\'\i
$$\sum_{i=0}^{m-1}(b_i^k)^2<3\frac {n(n-1)}m+n.$$
V opa\v cn\'em p\v r\'\i pad\v e dost\'av\'ame, \v ze 
$$\align\sum_{k=1}^{N-1}\big((\sum_{i=0}^{m-1}(b_i^k)^2)-n\big)\ge&\frac {
3(N-1)}4\frac {3n(n-1)}m=\\
&\frac {9(N-1)n(n-1)}{4m}>\\
&\frac {2(N-1)n(n-1)}m,\endalign$$
a to je spor s p\v redchoz\'\i m v\'ysledkem. Tedy p\v ri n\'ahodn\'em rovnom\v ern\'em 
v\'yb\v eru $k$ je 
$$\Prob\{\sum_{i=0}^{m-1}(b_i^k)^2<\frac {
3n(n-1)}m+n\mid k\in \{1,2,\dots,N-1\}\}\ge\frac 14.$$

\proclaim{Tvrzen\'\i}Kdy\v z $n=m$, pak 
\roster
\item"{(a)}"
existuje deterministick\'y algoritmus, jen\v z v \v case $O(nN)$ 
nalezne $k$ takov\'e, \v ze $$\sum_{i=0}^{m-1}(b_i^k)^2<3n;$$
\item"{(b)}"
existuje pravd\v epodobnostn\'\i\ algoritmus, kter\'y 
nalezne v \v case $O(n)$ takov\'e $k$, \v ze $\sum_{i=0}^{m-1}(b_
i^k)^2<4n$ -- o\v cek\'avan\'y po\v cet iterac\'\i\ v\'ypo\v ctu je nejv\'y\v se $
4$.
\endroster
D\'ale 
\roster
\item"{(c)}"
existuje deterministick\'y algoritmus, jen\v z v \v case $O(nN)$ pro 
$m=n(n-1)+1$ nalezne takov\'e $k$, \v ze $h_k$ je 
perfektn\'\i ;
\item"{(d)}"
existuje pravd\v epodobnostn\'\i\ algoritmus, kter\'y pro 
$m=2n(n-1)$ v \v case $O(n)$ nalezne $k$ takov\'e, \v ze $h_k$ je 
perfektn\'\i\ -- o\v cek\'avan\'y po\v cet iterac\'\i\ v\'ypo\v ctu 
je nejv\'y\v se $4$.
\endroster
\endproclaim

\demo{D\accent23ukaz}M\v ejme $n=m$. Proto\v ze spo\v c\'\i t\'an\'\i\ 
$\sum_{i=0}^{m-1}(b_i^k)^2$ pro pevn\'e $k$ vy\v zaduje \v cas $O
(n)$, prohled\'an\'\i m 
v\v sech mo\v znost\'\i\ nalez\-ne\-me $k$ takov\'e, \v ze 
$$\sum_{i=0}^{m-1}(b_i^k)^2\le\frac {2n(n-1)}n+n=3n-2<3n,$$
v \v case $O(nN)$. T\'\i m je 
dok\'az\'ano a). Pravd\v epodobnostn\'\i\ algoritmus dokazuj\'\i c\'\i\ b) vol\'\i\ 
n\'ahodn\v e $k$ a v \v case $O(n)$ ov\v e\v r\'\i , zda 
$\sum_{i=0}^{m-1}(b_i^k)^2\le 3\frac {n(n-1)}n+n=4n-3<4n$. Tuto akci opakuje dokud 
po\v zadavek nen\'\i\ spln\v en. Proto\v ze pravd\v epodobnost, \v ze $
k$ spl\v nuje 
po\v zadavek, je alespo\v n $\frac 14$, tak o\v cek\'avan\'y po\v cet ite\-rac\'\i\ 
akce je nejv\'y\v se 
$$\sum_{i=0}^{\infty}i(\frac 34)^{i-1}\frac 14=\frac 14\frac 1{(1
-\frac 34)^2}=4$$
a odtud plyne b). 

\flushpar Kdy\v z $m=n(n-1)+1$, pak prohled\'an\'\i m v\v sech mo\v znost\'\i\ 
nalezne\-me $k$ takov\'e, \v ze $$\sum_{i=0}^{m-1}(b_i^k)^2\le\frac {
2n(n-1)}{n(n-1)+1}+n<n+2,$$ v 
\v case $O(nN)$ a c) plyne z p\v redchoz\'\i\ v\v ety. Kdy\v z 
$m=2n(n-1)$, pak pro n\'ahodn\v e zvolen\'e $k$ plat\'\i\ s 
pravd\v epodobnost\'\i\ $\le\frac 14$, \v ze 
$$\sum_{i=0}^{m-1}(b_i^k)^2\le\frac {3n(n-1)}{2n(n-1)}+n<n+2.$$
Algoritmus spl\v nuj\'\i c\'\i\ tvrzen\'\i\ d) je stejn\'y jako 
v p\v r\'\i pad\v e b) (jen $m=2n(n-1)$). \qed
\enddemo

\flushpar Takto zkonstruovan\'e perfektn\'\i\ ha\v sovac\'\i\ funkce 
nespl\v nuj\'\i\ po\v za\-davek 2) (plat\'\i\ $m=\Theta (n^2)$).  Pou\v zijeme n\'asleduj\'\i c\'\i\ 
postup. 
\medskip

\flushpar1) Nalezneme $k$ takov\'e, \v ze pro $m=n$ plat\'\i\ 
$\sum_{i=0}^{m-1}(b_i^k)^2<3n$ (respektive $\sum_{i=0}^{m-1}(b_i^
k)^2<4n$).  Pro 
$i=0,1,\dots,m-1$ nalezneme mno\v ziny 
$S_i=\{s\in S\mid h_k(s)=i\}$;\newline 
2) Pro ka\v zd\'e $i=0,1\dots,m-1$ takov\'e, \v ze $S_i\ne\emptyset$, 
nalezneme pro $m=1+|S_i|(|S_i|-1)$ (respektive $m=1+2|S_i|(|S_i|-
1)$) 
takov\'e $k_i$, \v ze $h_{k_i}$ je perfektn\'\i\ na $S_i$. Definujme 
$c_i=1+|S_i|(|S_i|-1)$ (respektive $c_i=2|S_i|(|S_i|-1)$), kdy\v z 
$S_i\ne\emptyset$, a $c_i=0$, kdy\v z 
$S_i=\emptyset$.\newline 
3) Pro $i=0,1,\dots,m$ definujme $d_i=\sum_{j=0}^{i-1}c_j$ a pro $
x\in U$ 
ozna\v cme $h_k(x)=l$. Pak polo\v z\'\i me $g(x)=d_l+h_{k_l}(x)$.

\proclaim{V\v eta}Zkonstruovan\'a funkce $g$ je perfektn\'\i , 
hodnota $g(x)$ se pro ka\v zd\'e $x\in U$ spo\v c\'\i t\'a v \v case $
O(1)$, 
v deterministick\'em p\v r\'\i pa\-d\v e ha\v suje do tabulky velikosti 
$<3n$ a je nalezena v \v case $O(nN)$, v pravd\v epodobnostn\'\i m 
p\v r\'\i pa\-d\v e ha\v suje do tabulky velikosti $<6n$ a je nalezena 
v \v case $O(n)$. Pro jej\'\i\ zak\'odov\'an\'\i\ jsou t\v reba hodnoty $
k$ a 
$k_i$ pro $i=0,1,\dots,m-1$. Tyto hodnoty jsou v rozmez\'\i\ 
$1,2,\dots,N-1$, a tedy vy\v zaduj\'\i\ $O(n\log N)$ pam\v eti.
\endproclaim

\demo{D\accent23ukaz}Proto\v ze $g(S_i)$ pro $i=0,1,\dots,m-1$ 
jsou navz\'ajem disjunktn\'\i\ a $h_{k_i}$ je perfektn\'\i\ na $S_
i$, 
dost\'av\'ame, \v ze $g$ je perfektn\'\i . Pro v\'ypo\v cet hodnoty $
g(x)$ 
jsou t\v reba dv\v e n\'asoben\'\i , dvoj\'\i\ v\'ypo\v cet zbytku p\v ri 
d\v elen\'\i\ a jedno s\v c\'\i t\'an\'\i\ (hodnoty $d_i$ jsou ulo\v zeny v 
pam\v eti). Proto v\'ypo\v cet $g(x)$ vy\v zaduje \v cas $O(1)$. D\'ale 
$d_m$ je horn\'\i\ odhad na po\v cet \v radk\accent23u v tabulce. 
Proto\v ze pro $S_i\ne\emptyset$, m\'ame $|S_i|(|S_i|-1)+1\le |S_
i|^2=(b_i^k)^2$, 
dost\'av\'ame v deterministick\'em p\v r\'\i pad\v e 
$d_m=\sum_{i=0}^{m-1}c_i\le\sum_{i=0}^{m-1}(b_i^k)^2<3n$ a $k$ nalezneme v \v case 
$O(nN)$. Proto\v ze $k_i$ nalezneme v \v case $O(|S_i|N)$, lze $g$ 
zkonstruovat v \v case 
$O(nN+\sum_{i=0}^{m-1}|S_i|N)=O(nN+N\sum_{i=0}^{m-1}|S_i|)=O(2nN)
=O(nN)$. 
V pravd\v epodob\-nostn\'\i m p\v r\'\i pad\v e je 
$$d_m=\sum_{i=0}^{m-1}c_i\le\sum_{i=0}^{m-1}(2|S_i|^2-2|S_
i|)=2\sum_{i=0}^{m-1}(b_i^k)^2-2\sum_{i=0}^{m-1}b_i^k<8n-2n=6n$$
(proto\v ze $|S_i|=b_i^k$ a $\sum_{i=0}^{m-1}b_i^k=n$). Proto\v ze $
k$ nalezneme v \v case $O(n)$ a $k_i$ v \v case $O(|S_i|)$ dostaneme, \v ze 
\v ze $g$ nalezneme v \v case $O(n)$. Zbytek je jasn\'y. \qed
\enddemo

\flushpar Tedy zkonstruovan\'a ha\v sovac\'\i\ funkce spl\v nuje po\v zadavky 
1), 2) a 3), ale po\v zadavek 4) nen\'\i\ spln\v en. 
\medskip

\flushpar M\v ejme p\v rirozen\'e \v c\'\i slo $m$ a nech\v t $q$ je po\v cet v\v sech prvo\v c\'\i sel 
d\v el\'\i c\'\i ch $m$ ($p_1,p_2,\dots$ je rostouc\'\i\ posloupnost v\v sech 
prvo\v c\'\i sel). Pak
$$m\ge\prod_{i=1}^qp_i>q!=e^{\sum_{i=1}^q\ln i}\ge e^{\int_1^q\ln 
xdx}=e^{q\ln(\frac qe)+1}\ge (\frac qe)^q.$$
Proto existuje konstanta $c$, \v ze $q\le c\frac {\ln m}{\ln\ln m}$. Plat\'\i\ tedy  
\proclaim{V\v eta}Nech\v t $\delta (m)=$po\v cet prvo\v c\'\i sel, kter\'a d\v el\'\i\ 
$m$. Pak $\delta (m)=O(\frac {\log m}{\log\log m})$.
\endproclaim

\flushpar M\v ejme $S=\{s_1<s_2<\dots<s_n\}\subseteq U$. Ozna\v cme 
$d_{i,j}=s_j-s_i$ pro $1\le i<j\le n$. Pak $s_i\bmod p\ne s_j\bmod 
p$, 
pr\'av\v e kdy\v z $d_{i,j}\ne 0\bmod p$. Ozna\v cme 
$D=\prod_{1\le i<j\le n}d_{i,j}\le N^{(n^2)}$. Pak po\v cet prvo\v c\'\i seln\'ych 
d\v elitel\accent23u \v c\'\i sla $D$ je nejv\'y\v se $c\frac {\ln 
D}{\ln\ln D}$, a tedy mezi 
prvn\'\i mi $1+c\frac {\ln D}{\ln\ln D}$ prvo\v c\'\i sly existuje prvo\v c\'\i slo $
p$ 
takov\'e, \v ze $s_i\bmod p\ne s_j\bmod p$ pro ka\v zd\'e $1\le i
<j\le n$. 
To znamen\'a, \v ze funkce $\phi_p(x)=x\bmod p$ je perfektn\'\i\ pro 
$S$. Podle v\v ety o velikosti prvo\v c\'\i sel $p_t\le 2t\ln t$ pro ka\v zd\'e $
t\ge 6$, tedy
$$\align p\le&2(1+c\frac {\ln D}{\ln\ln D})\ln(1+c\frac {\ln D}{\ln\ln 
D})\le\\
&4c\frac {\ln D}{\ln\ln D}\ln(2c\frac {\ln D}{\ln\ln D})\le\\
&4c(\ln2c)\frac {\ln D}{\ln\ln D}+4c\frac {\ln D}{\ln\ln D}\ln(\frac {\ln 
D}{\ln\ln D})=\\
&4c\ln D+o(\ln D)=O(\ln D)=O(n^2\ln N).\endalign$$

\proclaim{V\v eta}Pro ka\v zdou $n$-prvkovou mno\v zinu $S\subseteq 
U$ 
existuje prvo\-\v c\'\i s\-lo $p$ o velikosti $O(n^2\ln N)$ takov\'e, \v ze 
funkce $\phi_p(x)=x\bmod p$ je perfektn\'\i\ pro $S$.  
\endproclaim
\medskip

\flushpar Test, zda funkce $\phi_p(x)=x\bmod p$ je perfektn\'\i\ 
pro $S$, 
vy\v zaduje \v cas $O(n\log n)$. Tedy systematick\'e hled\'an\'\i\ 
nejmen\v s\'\i ho $p$, \v ze $\phi_p$ je perfektn\'\i\ pro $S$, vy\v zaduje \v cas 
$O(n^3\log n\log N)$. Nejmen\v s\'\i\ $p$ takov\'e, \v ze $\phi_p$ je perfektn\'\i\ 
pro $S$, 
je prvo\v c\'\i slo. Navrhneme pravd\v epodob\-nostn\'\i\ algoritmus 
pro nalezen\'\i\ $p$. Pro dostate\v cn\v e velk\'e $n$ mezi 
prvn\'\i mi $9c\ln D$ prvo\v c\'\i sly je alespo\v n polovina tako\-v\'ych 
prvo\v c\'\i sel $p$, \v ze $\phi_p$ je perfektn\'\i\ pro $S$. Algoritmus pak 
opakuje n\'asleduj\'\i c\'\i\ krok, dokud nenalezne perfektn\'\i\ funkci
\roster
\item"{}"
vyberme n\'ahodn\v e \v c\'\i slo $p$ mezi prvn\'\i mi $9cn^2\ln 
N$ 
\v c\'\i sly a otestujme, zda $p$ je prvo\v c\'\i slo a $\phi_p$ je perfektn\'\i
\endroster
\flushpar Odhad o\v cek\'avan\'eho po\v ctu ne\'usp\v e\v sn\'ych krok\accent23u.\newline 
N\'ahodn\v e zvolen\'e \v c\'\i slo $p\le 9cn^2\ln N$ je prvo\v c\'\i slo s 
pravd\v epo\-dob\-nost\'\i\ $\Theta (\frac 1{\ln(9cn^2\ln N)})$ a pro prvo\v c\'\i slo $
p$ je $\phi_p$ 
perfektn\'\i\ s prav\-d\v e\-po\-dob\-nost\'\i\ $\ge\frac 12$. Tedy n\'ahodn\v e zvolen\'e 
\v c\'\i slo $p\le 9cn^2\ln N$ spl\v nuje test s pravd\v e\-po\-dob\-nost\'\i\ 
$\Theta (\frac 1{\ln(9cn^2\ln N)})$, a proto o\v cek\'a\-va\-n\'y po\v cet ne\'usp\v e\v sn\'ych 
test\accent23u je $O(\ln(9cn^2\ln N))$. Tedy o\v cek\'a\-van\'y \v cas 
algoritmu je $O(n\log n(\log n+\log\log N))$.

\proclaim{V\v eta}Pro danou mno\v zinu $S\subseteq U$ takovou, \v ze 
$|S|=n$, deterministick\'y algoritmus nalezne prvo\v c\'\i slo 
$p=O(n^2\log N)$ takov\'e, \v ze $\phi_p(x)=x\bmod p$ je perfektn\'\i\ 
pro $S$, a pracuje v \v case $O(n^3\log n\log N)$. Pravd\v epodobnostn\'\i\ 
algoritmus nalezne pr\-vo\v c\'\i slo $p=O(n^2\log N)$ takov\'e, \v ze 
$\phi_p$ je perfektn\'\i , v o\v cek\'ava\-n\'em \v case 
$O(n\log n(\log n+\log\log N))$.
\endproclaim

\flushpar Deterministick\'y algoritmus nalezne nejmen\v s\'\i\ prvo\v c\'\i slo s 
po\-\v za\-dovanou vlastnost\'\i .  Prav\-d\v e\-podobnostn\'\i\ algoritmus 
nalezne prvo\v c\'\i slo, kter\'e m\accent23u\v ze b\'yt podstatn\v e 
v\v et\v s\'\i , ale jeho velikost je omezena $9cn^2\log N$. 
\medskip

\flushpar Nyn\'\i\ navrhneme postup na konstrukci perfektn\'\i\ 
ha\v sovac\'\i\ funk\-ce pro mno\v zinu $S\subseteq U$.

\roster
\item 
Nalezneme prvo\v c\'\i slo $q_0\in O(n^2\log N)$ takov\'e, \v ze 
$\phi_{q_0}(x)=x\bmod q_0$ je perfektn\'\i\ funkce pro $S$. Polo\v zme 
$S_1=\{\phi_{q_0}(s)\mid s\in S\}$.
\item
Nalezneme prvo\v c\'\i slo $q_1$ takov\'e, \v ze 
$n(n-1)<q_1\le 2n(n-1)$. Pak existuje 
$l\in \{1,2,\dots,q_0-1\}$ takov\'e, \v ze $h_l(x)=((lx)\bmod q_0
)\bmod q_1$ 
je perfektn\'\i\ pro $S_1\subseteq \{0,1,\dots,q_0-1\}$. Polo\v zme 
$S_2=\{h_l(s)\mid s\in S_1\}$.
\item
D\'ale zkonstruujme perfektn\'\i\ ha\v sovac\'\i\ funkci 
$g$ pro mno\v zinu 
$S_2\subseteq \{0,1,\dots,q_1-1\}$ do tabulky s m\'en\v e ne\v z $
3n$ \v r\'adky. 
Polo\v zme $f(x)=g(h_l(\phi_{q_0}(x)))$. Konstruovan\'a ha\v sovac\'\i\ 
funkce je $f$.
\endroster

\flushpar V\'ysledek: $f$ je perfektn\'\i\ ha\v sovac\'\i\ funkce pro $
S$, proto\v ze 
slo\v zen\'\i\ perfektn\'\i ch ha\v sovac\'\i ch funkc\'\i\ je zase perfektn\'\i\ 
funkce, a tedy po\v zadavek 1) je spln\v en. \newline 
$f$ ha\v suje $S$ do tabulky s m\'en\v e ne\v z $3n$ \v r\'adky, a tedy 
spl\v nuje po\v zadavek 2). \newline 
Proto\v ze ka\v zd\'a z funkc\'\i\ $g$, $h_l$, $\phi_{q_0}$ se vy\v c\'\i sl\'\i\ v \v case $
O(1)$, 
i vy\v c\'\i slen\'\i\ funkce $f$ vy\v zaduje \v cas $O(1)$ a po\v zadavek 3) je 
spln\v en.\newline 
Funkce $\phi_{q_0}$ je jednozna\v cn\v e ur\v cena \v c\'\i slem $
q_0\in O(n^2\log N)$. 
Funkce $h_l$ je ur\v cena \v c\'\i sly $q_1\in O(n^2)$ a $l\in O(
q_0)$. Funkce $g$ 
je ur\v cena $n+1$ \v c\'\i sly velikosti $O(q_1)$. Tedy zad\'an\'\i\ $
f$ 
vy\v zaduje pam\v e\v t o velikosti 
$$O(\log n+\log\log N+n\log n)=O(n\log n+\log\log N).$$
Lze \v r\'\i ct, \v ze po\v zadavek 4) je spln\v en. 
\medskip

\flushpar V\'ypo\v cet $\phi_{q_0}$ vy\v zaduje \v cas $O(n^3\log 
n\log N)$. V\'ypo\v cet $h_l$ 
vy\v za\-du\-je \v cas $O(n(n^2\log N))=O(n^3\log N)$ (pou\v zit\'e univerzum je 
$\{0,1,\dots,q_0\}$). V\'ypo\v cet $g$ vy\v zaduje \v cas $O(nn^2
)=O(n^3)$ 
(zde univerzum je $\{0,1,\dots,q_1\}$). Celkov\v e, 
v\'ypo\v cet $f$ vy\v zaduje \v cas $O(n^3\log n\log N)$.
\medskip

\flushpar Lze pou\v z\'\i t i pravd\v epodobnostn\'\i\ algoritmy pro nalezen\'\i\ 
$g$, $h_l$ a $\phi_{q_0}$. Pak ha\v sujeme do tabulky s m\'en\v e ne\v z $
6n$ 
\v r\'adky, ale o\v cek\'avan\'y \v cas pro nalezen\'\i\ $f$ je 
$O(n\log n(\log n+\log\log N))$.
\bigskip

\flushpar Tuto metodu navrhli Fredman, Koml\'os a Szemer\'edi.
\medskip

\subhead
Dynamick\'e perfektn\'\i\ ha\v sov\'an\'\i
\endsubhead
\smallskip

\flushpar Jedna z velk\'ych nev\'yhod perfektn\'\i ho ha\v sovan\'\i\ je 
neznalost efektivn\'\i ch aktualiza\v cn\'\i ch ope\-rac\'\i . Existuj\'\i\ sice 
obecn\'e meto\-dy na dynamizaci deterministick\'ych operac\'\i\ -- viz 
letn\'\i\ p\v redn\'a\v s\-ka, ale tato metoda v tomto p\v r\'\i pad\v e neposkytuje 
efektivn\'\i\ dynamiza\v cn\'\i\ operace, proto\v ze deterministick\'y 
algoritmus pro \v re\-\v se\-n\'\i\ perfektn\'\i ho ha\v sov\'an\'\i\ je pro 
aktualiza\v cn\'\i\ ope\-race p\v rili\v s pomal\'y. To vedlo k n\'avrhu, 
kter\'y kombinuje pravd\v epodobnostn\'\i\ algoritmus pro perfektn\'\i\ 
ha\v sov\'an\'\i\ s obecnou metodou dynami\-za\-ce a tyto metody jsou 
upraveny pro konkr\'etn\'\i\ situaci. 
\medskip

\flushpar Nejprve uvedeme modifikaci v\'ysledk\accent23u z 
p\v redchoz\'\i\ \v c\'asti, na kter\'ych je tato metoda zalo\v zena.
P\v redpokl\'ad\'ame, \v ze $U=\{0,1,\dots,N-1\}$ je univerzum, kde $
N$ je 
prvo\v c\'\i slo, a \v ze je d\'ano \v c\'\i slo $s<N$. Ozna\v cme 
$\Cal H_s=\{h_k\mid k=1,2,\dots,N-1\}$ mno\v zinu funkc\'\i\ z $U$ do 
$\{0,1,\dots,s-1\}$, kde $h_k(x)=(kx\bmod N)\bmod s$ pro ka\v zd\'e $
x\in U$. 
Kdy\v z zvol\'\i me n\'ahodn\v e $k=1,2,\dots,N-1$, pak s pravd\v epodobnost\'\i\ 
ales\-po\v n $\frac 12$ plat\'\i\ 
$$\sum_{i=0}^{s-1}(b_i^k)^2<\frac {8n^2}s+2n.$$
Budeme p\v redpokl\'adat, 
\v ze takov\'e $k$ m\'ame, a pak pro ka\v zd\'e $i=0,1,\dots,s-1$ 
p\v redpokl\'ad\'ame, \v ze n\'ahodn\v e zvol\'\i me $j_i\in \Cal H_{
2(b_i^k)^2}$ takov\'e, \v ze 
$h_{j_i}$ je prost\'a na mno\v zin\v e $S_i=\{s\in S\mid h_k(s)=i
\}$ (z p\v redchoz\'\i ho textu v\'\i me, \v ze kdy\v z zvol\'\i me n\'ahodn\v e 
$j_i=0,1,\dots,N-1$, pak s pravd\v epodobnost\'\i\ ales\-po\v n $\frac 
12$ je $h_{j_i}$ 
prost\'a na $S_i$). Pro jednoduchost p\v redpokl\'ad\'ame, \v ze mno\v ziny $
S_i$ 
pro $i=0,1,\dots,s-1$ ulo\v z\'\i me do tabulek $T_i$ a tabulky 
$T_0,T_1,\dots,T_{s-1}$ budou ulo\v zeny v tabulce $T$. Kdy\v z $
s=O(|S|)$, 
pak tato metoda vy\v zaduje $O(|S|)$ prostoru. Abychom ur\v cili $
s$, 
zvolme $c>1$ a polo\v zme $s=\sigma (|S|)$, kde $\sigma (n)=\frac 
43\sqrt 6(1+c)n$ pro 
ka\v zd\'e $n$. Nyn\'\i\ pop\'\i\v seme algoritmy. 
\medskip

\subhead
Algoritmy
\endsubhead
\smallskip

\flushpar {\bf INSERT$(x)$}: \newline 
$n:=n+1$ \newline 
{\bf if} $n\le m$ {\bf then}\newline 
\phantom{---}$j:=h(x)$, $|S_j|:=|S_j|+1$\newline 
\phantom{---}{\bf if} $|S_j|\le m(j)$ a pozice $h_j(x)$ v $T_j$ je pr\'azdn\'a {\bf then}\newline 
\phantom{------}vlo\v z\'\i me $x$ do tabulky $T_j$  na pozici $
h_j(x)$\newline 
\phantom{---}{\bf else}\newline 
\phantom{------}{\bf if} $|S_j|\le m(j)$ a pozice $h_j(x)$ v $T_
j$ je obsazen\'a {\bf then}\newline 
\phantom{---------}vytvo\v r\'\i me seznam $S_j$ prvk\accent23u v tabulce $
T_j$\newline 
\phantom{---------}vypr\'azdn\'\i me tabulku $T_j$\newline 
\phantom{---------}zvol\'\i me n\'ahodn\v e funkci $h_j\in \Cal H_{2m(j)^2}$\newline 
\phantom{---------}{\bf while} $h_j$ nen\'\i\ prost\'a na mno\v zin\v e $S_j$ {\bf do}\newline
\phantom{------------}zvol\'\i me n\'ahodn\v e funkci $h_j\in \Cal H_{2m(j)^2}$\newline 
\phantom{---------}{\bf enddo}\newline
\phantom{---------}{\bf for every} $y\in S_j$ {\bf do} vlo\v z\'\i me $
y$ do $T_j$ na pozici $h_j(y)$ {\bf enddo}\newline 
\phantom{------}{\bf else}\newline 
\phantom{---------}$m(j):=2m(j)$\newline 
\phantom{---------}{\bf if} nen\'\i\ dost prostoru pro tabulku $
T_j$ nebo 
$$\sum_{i=0}^{\sigma (m)-1}2(m(i))^2\ge\frac {8m^2}{\sigma (m)}+2
m$$
\phantom{---------}{\bf then}\newline 
\phantom{------------}{\bf RehashAll}\newline 
\phantom{---------}{\bf else}\newline 
\phantom{------------}alokujeme prostor pro novou pr\'azdnou tabulku $
T_j$ \newline 
\phantom{------------}vytvo\v r\'\i me seznam $S_j$ prvk\accent23u ze star\'e tabulky $T_j$ a zru\v s\'\i me ji\newline 
\phantom{------------}zvol\'\i me n\'ahodn\v e funkci $
h_j\in \Cal H_{2m(j)^2}$\newline
\phantom{------------}{\bf while} $h_j$ nen\'\i\ prost\'a na mno\v zin\v e $
S_j$ {\bf do}\newline
\phantom{---------------}zvol\'\i me n\'ahodn\v e funkci $h_j\in \Cal H_{2m(j)^2}$\newline
\phantom{------------}{\bf enddo}\newline
\phantom{------------}{\bf for every} $y\in S_j$ {\bf do} vlo\v z\'\i me $
y$ do $T_j$ na pozici $h_j(y)$ {\bf enddo}\newline 
\phantom{---------}{\bf endif}\newline 
\phantom{------}{\bf endif}\newline 
\phantom{---}{\bf else}\newline 
\phantom{------}{\bf RehashAll}\newline
\phantom{---}{\bf endif\newline 
endif
\bigskip

\flushpar RehashAll}:\newline 
projdeme tabulku $T$ a tabulky $T_i$ a vytvo\v r\'\i me seznam 
prvk\accent23u z mno\v ziny $S$\newline 
$m:=(1+c)|S|$\newline 
zvolme n\'ahodn\v e $h\in \Cal H_{\sigma (m)}$ \newline 
{\bf for every} $i=0,1,\dots,\sigma (m)-1$ {\bf do} $S_i:=\{x\in S\mid h(x)=i\}$ {\bf enddo\newline
while} $\sum_{i=0}^{\sigma (m)-1}2(|S_i|)^2<\frac {8m^2}{\sigma (
m)}+2m$ {\bf do}\newline
\phantom{---}zvolme n\'ahodn\v e $h\in \Cal H_{\sigma (m)}$ \newline 
\phantom{---}{\bf for every} $i=0,1,\dots,\sigma (m)-1$ {\bf do} $S_i:=\{x\in S\mid h(x)=i\}$ {\bf enddo\newline
enddo}
Koment\'a\v r: zde $S_i$ jsou mno\v ziny vytvo\v ren\'e k n\'ahodn\v e zvolenou 
funkci $h$ \newline 
$n:=0$ \newline 
{\bf for every} $i=0,1,\dots,\sigma (m)-1$ {\bf do}\newline
\phantom{---}$m(i):=|S_i|$\newline 
\phantom{---}zvol\'\i me n\'ahodn\v e $h_i\in \Cal H_{2m(i)^2}$\newline 
\phantom{---}{\bf while} $h_i$ nen\'\i\ prost\'a na mno\v zin\v e $
S_i${\bf do}\newline
\phantom{------}zvol\'\i me n\'ahodn\v e $h_i\in \Cal H_{2m(i)^2}$\newline
\phantom{---}{\bf enddo\newline 
enddo\newline 
for every} $x\in S$ {\bf do INSERT$(x)$ enddo
\bigskip

\flushpar DELETE$(x)$}:\newline 
$j:=h(x)$, $n:=n-1$, $|S_j|:=|S_j|-1$\newline 
odstran\'\i me $x$ z pozice $h_j(x)$ v tabulce $T_j$, pozice bude pr\'azdn\'a \newline 
{\bf if} $n<\frac m{1+2c}$ {\bf then RehashAll endif
\bigskip

\flushpar MEMBER$(x)$}: \newline 
$j:=h(x)$\newline 
{\bf if} $x$ je na $h_j(x)$-t\'e pozici v tabulce $T_j$ {\bf then\newline 
\phantom{{\rm ---}}V\'ystup}: $x$ je prvek $S$ \newline 
{\bf else\newline 
\phantom{{\rm ---}}V\'ystup}: $x$ nen\'\i\ prvkem $S$\newline 
{\bf endif
\bigskip

}\flushpar Algoritmy p\v edpokl\'adaj\'\i , \v ze p\v ri operaci {\bf INSERT$
(x)$  }
prvek $x$ nepat\v r\'\i\ do $S$ a p\v ri operaci {\bf DELETE$(x)$} $
x$ je prvkem $S$. 
Pak $n$ znamen\'a velikost reprezentovan\'e mno\v ziny. 
\medskip

\flushpar Uvedu slo\v zitost t\'eto metody bez d\accent23ukazu.
\medskip

\proclaim{V\v eta}Popsan\'a metoda vy\v zaduje line\'arn\'\i\ pam\v e\v t 
(neuva\v zuje se pam\v e\v t pot\v rebn\'a pro zak\'odo\-v\'a\-n\'\i\ ha\v sovac\'\i ch funkc\'\i ), operace {\bf MEMBER }
v nej\-hor\v s\'\i m p\v r\'\i pad\v e vy\v zaduje \v cas $O(1)$ a o\v cek\'avan\'a 
amortizovan\'a slo\v zitost operac\'\i\ {\bf INSERT} a {\bf DELETE} je tak\'e $
O(1)$.
\endproclaim

\flushpar Toto zobecn\v en\'\i\ Fredman-Koml\'os-Szemer\'ediho metody navrhli 
Dietzfelbinger, Karlin, Mehl\-horn, Meyer auf der Heide, Rohnert 
a Tarjan.
\bigskip

\flushpar Dal\v s\'\i\ nev\'yhoda Fredman-Koml\'os-Szemer\'ediho metody:\newline 
Navr\v zen\'a metoda pracuje pro $m<3n$, ale nezajist\'\i\ $m=n$. 
Lze \v r\'\i ct, \v ze pam\v e\v t je efektivn\'\i\ vyu\v zit\'a? Existuje 
metoda, kter\'a by umo\v znila n\'avrh perfektn\'\i\ ha\v sovac\'\i\ 
funkce pro $m=n$? Z v\'ysledk\accent23u pro 
$(N,m,n)$-perfektn\'\i\ soubory funkc\'\i\ plyne existence 
$(N,n,n)$-perfektn\'\i ho souboru pro $n^N>e^{n+ln(n)}\ln(N)$. 
Zm\'\i n\'\i me se orienta\v cn\v e o parametrizovan\'e metod\v e, kter\'a 
navr\-huje perfektn\'\i\ ha\v sovac\'\i\ funkci pro $S\subseteq U$ a pro 
$|S|=n$. Parametr bude p\v rirozen\'e \v c\'\i slo $r$, kter\'e ur\v cuje, 
jak\'e hypergrafy jsou u\v zity p\v ri konstrukci funkce. Proto 
nejd\v r\'\i ve p\v ripomeneme n\v ekolik definic.

\flushpar Dvojice $(X,E)$, kde $X$ je mno\v zina a $E$ je syst\'em 
$r$-prvkov\'ych podmno\v zin $X$, se naz\'yv\'a $r$-\emph{hypergraf}.  
Prvky v $E$ se naz\'yvaj\'\i\ \emph{hrany} $r$-hypergrafu.  
\emph{Cyklus} je hypergraf $(X,E)$, kde ka\v zd\'y vrchol le\v z\'\i\ 
alespo\v n ve dvou r\accent23uzn\'ych hran\'ach.  Naopak 
$r$-hypergraf $(X,E)$ se naz\'yv\'a \emph{acyklick\'y}, kdy\v z 
\v z\'adn\'y jeho podhypergraf nen\'\i\ cyklus.  
\medskip

\flushpar Nyn\'\i\ pop\'\i\v seme metodu, kter\'a je rozd\v elena do 
dvou krok\accent23u. Je d\'ano $S\subseteq U$ takov\'e, \v ze 
$|S|=n$.
\medskip

\flushpar Krok 1) M\v ejme $r$-hypergraf $(V,E)$, kde $|E|=n$.  Nalezneme 
zobrazen\'\i\ $$g:V@>>>\{0,1,\dots,n-1\}$$ 
takov\'e, \v ze funkce 
$h:E@>>>\{0,1,\dots,n-1\}$ definovan\'a $h(e)=\sum_{i=1}^rg(v_i)\bmod n$, 
kde $e=\{v_1,v_2,\dots,v_r\}$, je prost\'a (m\'\i sto s\v c\'\i t\'an\'\i\ modulo $n$ 
m\accent23u\v zeme pou\v z\'\i t libovolnou grupovou operaci na 
mno\v zin\v e $\{0,1,\dots,n-1\}$).  Pro acyklick\'y $r$-hypergraf lze 
funkci $g$ zkonstruovat n\'asle\-duj\'\i c\'\i m postupem.  Zvol\'\i me 
bijekci $h:E@>>>\{0,1,\dots,n-1\}$ a pak definujeme $g$ 
n\'asledovn\v e:  kdy\v z $e=\{v_1,v_2,\dots,v_r\}$ a $g(v_i)$ je 
definov\'ano pro $i=2,3,\dots,r$, pak 
$$g(v_1)=h(e)-\sum_{i=2}^rg(v_i)\bmod n.$$
Proto\v ze pro ka\v zd\'y acyklick\'y $r$-hypergraf existuje vrchol, kter\'y 
le\v z\'\i\ v jedin\'e hran\v e, lze tento postup pou\v z\'\i t ke konstrukci $
g$ 
pomoc\'\i\ indukce (a tedy m\'ame algoritmus pro konstrukci $g$).  
\smallskip

\flushpar Krok 2) Nalezneme $r$ funkc\'\i\ $f_1,f_2,\dots,f_r:U@>>>
V$ takov\'ych, \v ze 
$(V,E)$, kde $$E=\{\{f_1(x),f_2(x),\dots,f_r(x)\}\mid x\in S\},$$ je acyklick\'y 
$r$-hypergraf.  Pak ha\v sovac\'\i\ funkce $f$ je definov\'ana 
$f(x)=\sum_{i=1}^rg(f_i(x))$ pro ka\v zd\'e $x\in U$.  Z konstrukce 
vypl\'yv\'a, \v ze je perfektn\'\i\ na mno\v zin\v e $S$.  
\medskip

\flushpar Auto\v ri dok\'azali, \v ze 
nejvhodn\v ej\v s\'\i\ alternativa je, kdy\v z zobrazen\'\i\ $f_1
,f_2,\dots,f_r$ 
jsou n\'ahodn\'a zob\-razen\'\i\ n\'ahodn\v e zvolen\'a. Bohu\-\v zel takov\'a 
zobrazen\'\i\ neum\'\i me zkonstruovat, ale auto\v ri uk\'azali, \v ze 
pro tyto \'u\v cely lze pou\v z\'\i t n\'ahodn\'y v\'yb\v er funkc\'\i\ z 
n\v ejak\'eho $c$-univerz\'aln\'\i ho souboru funk\-c\'\i . 
\medskip

\flushpar Auto\v ri uk\'azali, \v ze jejich algoritmus vy\v zaduje $
O(rn+|V|)$ 
\v casu a $O(n\log n+r\log|V|)$ pam\v eti. 
\medskip

\flushpar Tento metapostup navrhli Majewski, Wormald, Havas a Czech (1996).
\medskip

\flushpar Pro praktick\'e pou\v zit\'\i\ je problematick\'a 
reprezentace $r$-hyper\-gra\-fu a i n\'ahodn\'a volba funkc\'\i\ 
$f_1,f_2,\dots,f_r$ (viz p\v redchoz\'\i\ diskuze o $c$-univerzalit\v e).  Z 
po\v zadavk\accent23u na perfektn\'\i\ ha\v sovac\'\i\ funkci je op\v et 
probl\'emem spln\v en\'\i\ po\v za\-davku 4).  Nev\'\i m, jak je uveden\'a 
metoda prakticky pou\-\v zi\-teln\'a a zda se n\v ekde pou\v z\'\i v\'a.  

\heading
Extern\'\i\ ha\v sov\'an\'\i
\endheading

\flushpar Navr\v zen\'y postup je tak\'e zn\'am pod n\'azvem Fagin\accent23uv algoritmus. T\'\i mto probl\'emem se prvn\'\i asi zab\'yval Larsson.
\medskip

\flushpar\v Re\v s\'\i me jin\'y probl\'em -- ulo\v zen\'\i\ dat na extern\'\i\ 
pam\v e\v t. Hlavn\'\i\ probl\'em -- minimalizovat p\v r\'\i stup na 
extern\'\i\ pam\v e\v t. 
\medskip

\flushpar P\v redpoklady:  Extern\'\i\ pam\v e\v t je rozd\v elena na 
str\'anky, ka\v zd\'a str\'anka obsahuje $b$ polo\v zek (dat) 
(p\v redpokl\'ad\'ame, \v ze $b>1$, jinak to nem\'a smysl).  V\v zdy v 
jednom kroku na\v cteme celou str\'anku do intern\'\i\ pam\v eti 
nebo celou str\'anku v intern\'\i\ pam\v eti v jednom kroku 
zap\'\i\v seme na extern\'\i\ medium.  Tyto operace jsou \v r\'adov\v e 
pomalej\v s\'\i\ ne\v z oprace v intern\'\i\ pam\v eti.  \newline 
N\'a\v s c\'\i l:  
Nal\'ezt zp\accent23usob ukl\'ad\'an\'\i\ dat do str\'anek extern\'\i\ 
pam\v eti, aby se minimalizoval po\v cet operac\'\i\ s extern\'\i\ 
pam\v et\'\i .  
\medskip

\flushpar P\v redpokl\'adejme, \v ze $h:U@>>>\{0,1\}^{*}$ je prost\'e 
zobrazen\'\i\ takov\'e, \v ze d\'elka $h(u)$ je stejn\'a pro v\v sechny prvky 
univerza $U$.  Ozna\v cme $k$ d\'elku $h(u)$ pro $u\in U$.  Pak $
h$ je 
ha\v sovac\'\i\ funkce.  Nech\v t $S\subseteq U$, pak pro slovo $
\alpha$ d\'elky 
men\v s\'\i\ ne\v z $k$ definujme 
$$h^{-1}_S(\alpha )=\{s\in S\mid\alpha\text{\rm\ je prefix }h(s)\}.$$ 
\v Rekneme, \v ze $\alpha$ je \emph{kritick\'e} \emph{slovo}, kdy\v z 
$0<|h^{-1}_S(\alpha )|\le b$ a pro ka\v zd\'y vlastn\'\i\ prefix $
\alpha'$ slova $\alpha$ plat\'\i\ 
$|h^{-1}_S(\alpha')|>b$.  Pro ka\v zd\'e $s\in S$ existuje pr\'av\v e jedno 
kritick\'e slovo $\alpha$, kter\'e je prefixem $h(s)$.  Definujme $
d(s)$ 
pro $s\in S$ jako d\'elku kritick\'eho slova, kter\'e je prefixem 
$h(s)$ a 
$$d(S)=\max\{\text{\rm d\'elka}(\alpha )\mid\alpha\text{\rm\ je kritick\'e slovo}\}=\max\{d(s)\mid s\in S\}.$$
Mno\v zinu $S$ reprezentujeme tak, \v ze je jednozna\v cn\'a 
korespondence mezi kritick\'ymi slovy a str\'ankami extern\'\i\ 
pam\v eti slou\v z\'\i\-c\'\i\-mi k reprezentaci $S$.  Na str\'ance 
p\v r\'\i slu\v sej\'\i c\'\i\ kritick\'emu slovu $\alpha$ je reprezentov\'an soubor 
$h^{-1}_S(\alpha )$.  

\flushpar Probl\'em: jak nal\'ezt str\'anku kritick\'eho slova 
$\alpha$?\newline 
\v Re\v sen\'\i : Adres\'a\v r je funkce, kter\'a ka\v zd\'emu slovu $
\alpha$ o 
d\'elce $d(S)$ p\v ri\v rad\'\i\ adresu str\'anky p\v redpisem
\roster
\item"{}"kdy\v z kritick\'e slovo $\beta$ je prefixem $\alpha$, pak k $
\alpha$ je 
p\v ri\v razena str\'an\-ka koresponduj\'\i c\'\i\ s $\beta$, jinak je k $
\alpha$ 
p\v ri\v razena str\'an\-ka $NIL$ -- speci\'aln\'\i\ pr\'azdn\'a str\'anka.
\endroster
\flushpar Korektnost: Pro r\accent23uzn\'a kritick\'a slova 
$\beta$ a $\gamma$ plat\'\i\ $h^{-1}_S(\beta )\cap h^{-1}_S(\gamma 
)=\emptyset$, a tedy pro ka\v zd\'e slovo $\alpha$ 
d\'elky $d(S)$ existuje nejv\'y\v se jedno kritick\'e slovo, kter\'e 
je prefixem $\alpha$. Kdy\v z $\alpha$ je slovo d\'elky $d(S)$, pak nastane 
jeden z t\v echto t\v r\'\i\ p\v r\'\i pad\accent23u:
\roster
\item
$h^{-1}_S(\alpha )\ne\emptyset$, pak $0<|h^{-1}_S(\alpha )|\le b$ a existuje pr\'av\v e jedno kritick\'e slovo $
\beta$, 
kter\'e je prefixem $\alpha$; 
\item
$h^{-1}_S(\alpha )=\emptyset$ a existuje prefix $\alpha'$ slova $
\alpha$ takov\'y, \v ze 
$0<|h^{-1}_S(\alpha')|\le b$, pak existuje pr\'av\v e jedno kritick\'e slovo, kter\'e je 
prefixem $\alpha'$ (a tedy tak\'e prefixem $\alpha$);
\item
$h^{-1}_S(\alpha )=\emptyset$ a pro ka\v zd\'y prefix $\alpha'$ slova $
\alpha$ plat\'\i\ bu\v d 
$h^{-1}_S(\alpha')=\emptyset$ nebo $|h^{-1}_S(\alpha')|>b$ (pak k $
\alpha$ je p\v ri\v razena 
str\'anka $NIL$.
\endroster

\flushpar M\v ejme slovo $\alpha$ o d\'elce $d(S)$. Ozna\v cme $c
(\alpha )$ nejkrat\v s\'\i\ 
prefix $\alpha'$ slova $\alpha$ takov\'y, \v ze str\'anka p\v ri\v razen\'a slovu $
\beta$ o 
d\'elce $d(S)$, kter\'e m\'a $\alpha'$ za prefix, je stejn\'a jako str\'anka 
p\v ri\v razen\'a $\alpha$. V\v simn\v eme si, \v ze kdy\v z $h^{
-1}_S(\alpha )\ne\emptyset$, pak $c(\alpha )$ je 
kritick\'e slovo. Plat\'\i\ siln\v ej\v s\'\i\ tvrzen\'\i , kter\'e tvrd\'\i , \v ze 
n\'asleduj\'\i c\'\i\ podm\'\i nky jsou ekvivalentn\'\i :
\roster
\item
str\'anka p\v ri\v razen\'a slovu $\alpha$ je r\accent23uzn\'a od $NIL$;
\item
$c(\alpha )$ je kritick\'e slovo;
\item
n\v ejak\'y prefix $\alpha$ je kritick\'e slovo.
\endroster
V\v simn\v eme si, \v ze znalost adres\'a\v re umo\v z\v nuje nal\'ezt slovo 
$c(\alpha )$ pro ka\v zd\'e slovo o d\'elce $d(S)$.
\medskip

\flushpar Line\'arn\'\i\ uspo\v r\'ad\'an\'\i\ na slovech d\'elky $
n$ nazveme 
\emph{lexikografick\'e}, kdy\v z $\alpha <\beta$, pr\'av\v e kdy\v z 
$\alpha =\gamma 0\alpha'$ a $\beta =\gamma 1\beta'$ pro n\v ejak\'a slova $
\gamma$, $\alpha'$ a $\beta'$. 
Lexikografick\'e uspo\v r\'ad\'an\'\i\ v\v zdy existuje a je 
jednozna\v cn\'e.
\medskip

\flushpar Reprezentace adres\'a\v re: Je to seznam adres 
str\'anek o d\'elce $2^{d(S)}$ takov\'y, \v ze adresa na $i$-t\'em m\'\i st\v e 
odpov\'\i d\'a $i$-t\'emu slovu d\'elky $d(S)$ v lexikografick\'em 
uspo\v r\'ad\'an\'\i .

\flushpar P\v r\'\i klad: $U$ je mno\v zina v\v sech slov nad $\{
0,1\}$ o 
d\'elce $5$, $h$ 
je identick\'a funkce a $b=2$. Reprezentujme mno\v zinu  
$S=\{00000,\,00010,\,01000,\,10000\}$. Pak 
$d(00000)=d(00010)=d(01000)=2$, $d(10000)=1$, kritick\'a slova 
jsou $00$, $01$ a $1$ a adres\'a\v r je (m\'\i sto adresy str\'anky 
uvedeme mno\v zinu, kter\'a je na t\'eto str\'ance ulo\v zena)
$$00\mapsto \{00000,00010\},\quad 01\mapsto \{01000\},\quad 10\mapsto 
11\mapsto \{10000\}.$$
Tedy $c(00)=00$, $c(01)=01$ a $c(10)=c(11)=1$.
Kdy\v z odstran\'\i me prvek $10000$, pak $1$ p\v restane b\'yt kritick\'e 
slovo a adres\'a\v r bude m\'\i t tvar
$$00\mapsto \{00000,00010\},\quad 01\mapsto \{01000\},\quad 10\mapsto 
11\mapsto NIL.$$
Op\v et plat\'\i\ $c(00)=00$, $c(01)=01$ a $c(10)=c(11)=1$.
V adres\'a\v ri je tak\'e ulo\v zeno $d(S)$.
\medskip

\subhead
Algoritmy
\endsubhead
\smallskip

\flushpar Slovn\'\i\ popis operac\'\i . P\v redpokl\'ad\'ame, \v ze adres\'a\v r 
je ulo\v zen v extern\'\i\ pam\v eti na jedn\'e str\'ance.
\medskip

\flushpar {\bf MEMBER$(x)$}:\newline 
1) Spo\v c\'\i t\'ame $h(x)$ a {\bf na\v cteme} adres\'a\v r do intern\'\i\ pam\v eti. 
Vez\-me\-me prefix $\alpha$ slova $h(x)$ o d\'elce $d(S)$ a nalezneme 
adresu str\'anky p\v r\'\i slu\v sej\'\i c\'\i\ k $\alpha$. Kdy\v z je to str\'anka 
$NIL$, pak $x\notin S$ a konec, jinak pokra\v cujeme krokem 
2).\newline 
2) {\bf Na\v cteme} str\'anku p\v r\'\i slu\v sej\'\i c\'\i\ k $\alpha$ do intern\'\i\ 
pam\v eti. Prohle\-d\'a\-me ji a pokud neobsahuje $x$, pak $x\notin 
S$ a 
konec. Kdy\v z obsahuje $x$, pak provedeme po\v zadovan\'e 
zm\v eny a str\'anku {\bf ulo\-\v z\'\i\-me} do extern\'\i\ pam\v eti na jej\'\i\ 
p\accent23uvodn\'\i\ m\'\i sto. Konec.
\medskip

\flushpar {\bf INSERT$(x)$}:\newline 
1) Spo\v c\'\i t\'ame $h(x)$ a {\bf na\v cteme} adres\'a\v r do intern\'\i\ pam\v eti. 
Vez\-me\-me prefix $\alpha$ slova $h(x)$ o d\'elce $d(S)$ a nalezneme 
adresu str\'anky p\v r\'\i slu\v sej\'\i c\'\i\ k $\alpha$ a slovo $
c(\alpha )$. 
Kdy\v z str\'anka p\v ri\v razen\'a k $\alpha$ je $NIL$, pokra\v cu\-je\-me krokem 
3), v opa\v cn\'em p\v r\'\i pad\v e pokra\v cu\-je\-me krokem 2).\newline 
2) {\bf Na\v cteme} str\'anku p\v ri\v razenou slovu $\alpha$. Kdy\v z $
x$ je 
ulo\v zeno na t\'eto str\'ance, pak skon\v c\'\i me. Kdy\v z $x$ nen\'\i\ na 
t\'eto str\'ance, pak tam p\v rid\'ame slovo $x$. Pokud na str\'ance 
je nejv\'y\v se $b$ prvk\accent23u, pak {\bf ulo\v z\'\i me} str\'anku 
na jej\'\i\ p\accent23uvodn\'\i\ m\'\i sto a skon\v c\'\i me. Kdy\v z na 
str\'ance je v\'\i ce ne\v z $b$ prvk\accent23u, pak nalezneme nov\'a 
kritick\'a slova, 
kter\'a n\'am str\'anku rozd\v el\'\i , a vytvo\v r\'\i me dv\v e str\'anky -- 
jednu {\bf ulo\v z\'\i me} na m\'\i sto p\accent23uvodn\'\i\ str\'anky a 
druhou {\bf ulo\v z\'\i me} na novou str\'anku. Pokra\v cujeme krokem 
4).\newline 
3) Vytvo\v r\'\i me v intern\'\i\ pam\v eti novou str\'anku, kter\'a 
obsahuje $x$, nalezneme novou str\'anku v extern\'\i\ pam\v eti a 
tam {\bf ulo\v z\'\i me} vytvo\v renou str\'anku (v\v sem slov\accent23um, 
kter\'a maj\'\i\ $c(\alpha )$ za prefix, bude p\v ri\v razena tato str\'anka) a 
pokra\v cujeme krokem 4).\newline 
4) {\bf Na\v cteme} op\v et adres\'a\v r do intern\'\i\ pam\v eti, 
aktualizujeme adresy p\v ri\v razen\'ych str\'anek a p\v r\'\i padn\v e 
zv\v et\v s\'\i me adres\'a\v r (to nastane, kdy\v z n\v ejak\'e nov\'e kritick\'e 
slovo m\'a d\'elku v\v et\v s\'\i\ ne\v z $d(S)$, pak nov\'e $d(S
)$ je pr\'av\v e 
d\'elka tohoto slova -- ob\v e kritick\'a slova vznikl\'a v kroku 
2) maj\'\i\ stejnou d\'elku). Aktualizovan\'y adres\'a\v r {\bf ulo\v z\'\i me} do 
extern\'\i\ pam\v eti. Konec.
\medskip

\flushpar {\bf DELETE$(x)$}:\newline 
1) Spo\v c\'\i t\'ame $h(x)$ a {\bf na\v cteme} adres\'a\v r do intern\'\i\ pam\v eti. 
Vez\-me\-me prefix $\alpha$ slova $h(x)$ o d\'elce $d(S)$ a nalezneme 
adresu str\'anky p\v r\'\i slu\v sej\'\i c\'\i\ k $\alpha$ a slovo $
c(\alpha )$. 
Kdy\v z str\'anka p\v ri\v razen\'a k $\alpha$ je $NIL$, pak skon\v c\'\i me. 
Ozna\v cme $\beta'$ slovo, kter\'e m\'a stejnou d\'elku jako $c(\alpha 
)$ a 
li\v s\'\i\ se od $c(\alpha )$ pouze v posledn\'\i m bitu. Kdy\v z existuje 
slovo $\beta$ d\'elky $d(S)$ takov\'e, \v ze $c(\beta )=\beta'$, pak str\'anka 
p\v ri\v razen\'a k $\beta$ je \emph{kandid\'at}.\newline 
2) {\bf Na\v cteme} str\'anku p\v r\'\i slu\v snou k slovu $\alpha$ do intern\'\i\ 
pam\v eti. Kdy\v z tato str\'anka neobsahuje $x$, pak skon\v c\'\i me. 
Kdy\v z tato str\'anka obsahuje $x$, pak odstran\'\i me $x$ z t\'eto 
str\'anky. Kdy\v z neexistuje kandid\'at nebo kdy\v z nov\'a 
str\'anka a str\'anka kandid\'ata dohromady obsahuj\'\i\ v\'\i ce ne\v z $
b$ 
prvk\accent23u, pak novou str\'anku {\bf ulo\-\v z\'\i me} na jej\'\i\ 
p\accent23uvodn\'\i\ m\'\i sto a skon\v c\'\i me. \newline 
3) Kdy\v z nov\'a str\'anka a str\'anka kandid\'ata maj\'\i\ 
dohromady $b$ prvk\accent23u, pak {\bf na\v cteme} str\'anku 
kandid\'ata do intern\'\i\ pam\v eti. V intern\'\i\ pam\v eti tyto 
str\'anky spoj\'\i me do jedn\'e a tuto str\'anku pak 
{\bf ulo\-\v z\'\i me} do extern\'\i\ pam\v eti.\newline 
4) {\bf Na\v cteme} adres\'a\v r, kde zaktualizujeme adresy 
str\'anek. Po\-kud jsme slou\v cili dv\v e str\'anky, mus\'\i me nal\'ezt 
nov\'e $c(\alpha )$ (je to nejkrat\v s\'\i\ prefix $\alpha'$ slova $
\alpha$ takov\'y, \v ze 
ke ka\v zd\'emu slovu $\beta$ o d\'elce $d(S)$, kter\'e m\'a $\alpha'$ za prefix, 
je p\v ri\v razena jedna z t\v echto adres: adresa str\'anky 
p\v ri\v razen\'a k $\alpha$, adresa str\'anky kandid\'ata, $NIL$) a ka\v zd\'emu 
slovu o d\'elce $d(S)$, kter\'e m\'a nov\'e $c(\alpha )$ za prefix, bude 
p\v ri\v razena adresa nov\'e (spojen\'e) str\'anky. Otestujeme, zda 
se adres\'a\v r nem\accent23u\v ze zkr\'atit (to nastane, kdy\v z 
adresy str\'anek p\v ri\v razen\'e $(2i+1)$-\'\i mu slovu a $(2i+
2)$-\'emu 
slovu o d\'elce $d(S)$ jsou stejn\'e pro v\v sechna $i$, 
pak se tato slova spoj\'\i\ a $d(S)$ se zmen\v s\'\i\ o $1$). Upraven\'y 
adres\'a\v r {\bf ulo\v z\'\i me}. Konec.
\bigskip

\flushpar N\'asleduj\'\i c\'\i\ v\v eta ukazuje, \v ze jsme n\'a\v s hlavn\'\i\ c\'\i l splnili. Pro jednoduchost p\v redpokl\'ad\'ame, \v ze 
adres\'a\v r je tak\'e ulo\v zen na extern\'\i\ pam\v eti a \v ze v 
intern\'\i\ pam\v eti nem\accent23u\v ze b\'yt ulo\v zen spolu s n\v ejakou jinou str\'ankou. 

\proclaim{V\v eta}Operace {\bf MEMBER} vy\v zaduje nejv\'y\v se t\v ri 
operace s extern\'\i\ pam\v et\'\i . Operace {\bf INSERT} a {\bf DELETE }
vy\v zaduj\'\i\ nejv\'y\v se \v sest operac\'\i\ s extern\'\i\ pam\v et\'\i .
\endproclaim

\flushpar V na\v sem p\v r\'\i kladu provedeme operaci 
{\bf INSERT$(00001)$}. Po p\v rid\'an\'\i\ prvku str\'anka p\accent23uvodn\v e 
p\v ri\v razen\'a k slovu $00$ vypad\'a takto $\{00000,00001,0001
0\}$. Tuto 
str\'anku rozd\v el\'\i me na  
str\'an\-ky $\{00000,00001\}$ a $\{00010\}$. P\v ritom kritick\'e slovo 
prvn\'\i\ str\'an\-ky je $0000$ a druh\'e str\'an\-ky je $0001$. Tak\v ze 
$d(S)=4$ a adres\'a\v r vypad\'a 
$$\align&0000\mapsto \{00000,00001\},0001\mapsto \{00010\},\\
&0010\mapsto 0011\mapsto NIL,\\
&0100\mapsto 0101\mapsto 0110\mapsto 0111\mapsto \{0100\},\\
&1000\mapsto 1001\mapsto 1010\mapsto 1011\mapsto \{10000\},\\
&1100\mapsto 1101\mapsto 1110\mapsto 1111\mapsto \{10000\}.\endalign$$
To znamen\'a, \v ze krom\v e adresy $00$ se ostatn\'\i\ slova rozd\v elila 
na \v cty\v ri slova, ale adresy z\accent23ustaly stejn\'e. Jen u 
slova 00 vznikl\'a slova dostala r\accent23uzn\'e adresy.
\medskip

\flushpar V p\accent23uvodn\'\i m p\v r\'\i kladu 
provedeme operaci {\bf DELETE$(01000)$}. Pak kandid\'at je $00$ a po 
odstran\v en\'\i\ prvku $01000$ nastane spojen\'\i\ t\v echto dvou 
str\'anek. Po aktualizaci adres dostane adres\'a\v r tvar
$$00\mapsto 01\mapsto \{00000,00010\},10\mapsto 11\mapsto \{10000
\},$$
tj. k prvn\'\i mu a druh\'emu slovu je p\v ri\v razena stejn\'a 
str\'anka a stejn\v e tak k t\v ret\'\i mu a \v ctvrt\'emu slovu. Tak\v ze 
m\accent23u\v zeme adres\'a\v r zmen\v sit. Pak $d(S)=1$ a adres\'a\v r m\'a 
podobu  
$$0\mapsto \{00000,00010\},1\mapsto \{10000\}.$$

\flushpar Vznik\'a ot\'azka, jak je tato metoda efektivn\'\i . 
Hlavn\v e jak efektivn\v e vyu\v z\'\i v\'a pam\v e\v t. Plat\'\i

\proclaim{V\v eta}Kdy\v z velikost repre\-zentovan\'e mno\v ziny je $n$, pak 
o\v cek\'avan\'y po\v cet pou\v zit\'ych str\'anek je 
$\frac n{b\ln2}$ a o\v cek\'ava\-n\'a velikost adres\'a\v re je $\frac 
e{b\ln2}n^{1+\frac 1b}$. 
\endproclaim

\flushpar Prvn\'\i\ tvrzen\'\i\ \v r\'\i k\'a, \v ze o\v cek\'avan\'y po\v cet 
prvk\accent23u na str\'ance je $b\ln2\approx 0.69b$. Tedy zapln\v eno 
je asi 69\% m\'\i st. Tento v\'ysledek nen\'\i\ p\v rekvapuj\'\i c\'\i\ a je 
akceptovateln\'y. Hor\v s\'\i\ je to s adres\'a\v rem, jak ukazuje 
n\'asleduj\'\i c\'\i\ tabulka
\medskip
$$\vtop{\offinterlineskip\halign {\strut\vrule\ # & \vrule\ # & \vrule\ # & \vrule\ # & \vrule \hfil # \hfil \vrule\cr\noalign{\hrule}velikost $S$ & $10^5$ & $10^6$ & $10^8$ & $10^{10}$ \hfill \cr\noalign{\hrule} 2 & $6.2\cdot10^7$ & $1.96\cdot10^8$ & $1.96\cdot10^{11}$ & $1.96\cdot10^{14}$ \cr\noalign{\hrule} $10$ & $1.2\cdot10^5$ & $1.5\cdot10^6$ & $2.4\cdot10^8$ & $3.9\cdot10^{10}$ \cr\noalign{\hrule} $50$ & $9.8\cdot10^3$ & $1.0\cdot10^6$ & $1.1\cdot10^8$ & $1.2\cdot10^{10}$ \cr\noalign{\hrule} $100$ & $4.4\cdot10^3$ & $4.5\cdot10^4$ & $4.7\cdot10^6$ & $4.9\cdot10^8$ \cr\noalign{\hrule}}}$$
\flushpar kde jednotliv\'e \v r\'adky odpov\'\i daj\'\i\ hodnot\'am $
b$ 
uveden\'ych v prv\-n\'\i m sloupci. Proto\v ze o\v cek\'avan\'a velikost 
adres\'a\v re se zv\v et\v su\-je rychleji ne\v z line\'arn\v e (exponent u $
n$ je $1+\frac 1b$), 
tak nelze o\v cek\'avat, \v ze tuto metodu lze v\v zdy pou\v z\'\i t. 
V\'ypo\v cty i expe\-ri\-menty uka\-zu\-j\'\i , \v ze pou\v ziteln\'a je do 
velikosti $|S|=10^{10}$, kdy\v z $b\approx 100$. V tomto rozmez\'\i\ je 
n\'ar\accent23ust adres\'a\v re jen kolem 5\%. Pro v\v et\v s\'\i\ $
n$ je 
t\v reba, aby $b$ bylo je\v st\v e v\v et\v s\'\i .
\end

