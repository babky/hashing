\section*{AVL Tree}

\section*{Delete}

Works as in common binary search tree but height of a tree may be decreased and rebalancing may be required. 

Let $t$ be the node for which the height of its subtree decreased. Let $v$ be its parent and $u$ be the brother of $t$. Assume that the subtree rooted at the node $t$ is a correct AVL tree and the value $\omega$ stored inside the node $v$ is that before deletion.

\[
\begin{split}
& h(v) = a \\
& \omega(v) \in \{-1, 0, 1\} \\
& h(t) \in \{a - 1, a - 2\} \\
& h(u) \in \{a - 1, a - 2\} \\
\end{split}
\]

\vspace{1cm}

In our analysis assume that $t$ is the left son of $v$. Mirror case is the same as the mirror case with insertion.

Now we have tree cases distinguished by $\omega(v)$.

\pagebreak

\paragraph*{\textbf{Case 1}: $\omega(v) = 1$}
\[
\begin{split}
& h(v) = a \\
& \omega(v) = 1 \\
& h(t) = a - 2 \\
& h(u) = a - 1 \\
& h'(t) = a - 3 \\
\end{split}
\]

Now we have three subcases according to $\omega(u)$.
\paragraph*{\textbf{Case 1a}: $\omega(u) = 0$}
Rotate edge $u, v$.
\[
\begin{split}
& h(v) = a \\
& \omega(v) = 1 \\
& h(t) = a - 2 \\
& h(u) = a - 1 \\
& h(u_1) = a - 2 \\
& h(u_2) = a - 2 \\
& h'(t) = a - 3 \\
\end{split}
\]

Result:
\[
\begin{split}
& \omega'(v) = 1 \\
& \omega'(u) = -1 \\
& h'(v) = a - 1 \\
& h'(u) = a \\
& h'(u_1) = a - 2 \\
& h'(u_2) = a - 2 \\
& h'(t) = a - 3 \\
\end{split}
\]

Since the height remains the same we stop.

\paragraph*{\textbf{Case 1b}: $\omega(u) = 1$}
Rotate edge $u, v$.
\[
\begin{split}
& h(v) = a \\
& \omega(v) = 1 \\
& h(t) = a - 2 \\
& h(u) = a - 1 \\
& h(u_1) = a - 3 \\
& h(u_2) = a - 2 \\
& h'(t) = a - 3 \\
\end{split}
\]

Result:
\[
\begin{split}
& \omega'(v) = 0 \\
& \omega'(u) = 0 \\
& h'(v) = a - 2 \\
& h'(u) = a - 1\\
& h'(u_1) = a - 3 \\
& h'(u_2) = a - 2 \\
& h'(t) = a - 3 \\
\end{split}
\]
Since the height decreased we have to continue.

\paragraph*{\textbf{Case 1c}: $\omega(u) = -1$}
Perform double rotation $u_1, u, v$.
\[
\begin{split}
& h(v) = a \\
& \omega(v) = 1 \\
& h(t) = a - 2 \\
& h(u) = a - 1 \\
& h(u_1) = a - 2 \\
& h(u_2) = a - 3 \\
& h'(t) = a - 3 \\
\end{split}
\]

Analysis and the result depends on $\omega(u_1)$. In every case the height of the subtree is decreased and we have to continue with the node $u_1$.
\paragraph*{\textbf{Case 1cI}: $\omega(u_1) = -1$}
\[
\begin{split}
& h(v) = a \\
& \omega(v) = 1 \\
& h(t) = a - 2 \\
& h'(t) = a - 3 \\
& h(u) = a - 1 \\
& \omega(u) = -1 \\
& h(u_1) = a - 2 \\
& \omega(u_1) = -1 \\
& h(u_2) = a - 3 \\
& h(u_3) = a - 3 \\
& h(u_4) = a - 4 \\
\end{split}
\]
Result:
\[
\begin{split}
& \omega'(u_1) = 0 \\
& h'(u_1) = a - 1 \\
& h'(v) = a - 2 \\
& \omega'(v) = 0 \\
& h'(u) = a - 2 \\
& \omega'(u) = 1 \\
& h'(t) = a - 3 \\
& h'(u_3) = a - 3 \\
& h'(u_4) = a - 4 \\
& h'(u_2) = a - 3 \\
\end{split}
\]

\paragraph*{\textbf{Case 1cII}: $\omega(u_1) = 0$}
\[
\begin{split}
& h(v) = a \\
& \omega(v) = 1 \\
& h(t) = a - 2 \\
& h'(t) = a - 3 \\
& h(u) = a - 1 \\
& \omega(u) = -1 \\
& h(u_1) = a - 2 \\
& \omega(u_1) = 0 \\
& h(u_2) = a - 3 \\
& h(u_3) = a - 3 \\
& h(u_4) = a - 3 \\
\end{split}
\]
Result:
\[
\begin{split}
& \omega'(u_1) = 0 \\
& h'(u_1) = a - 1 \\
& h'(v) = a - 2 \\
& \omega'(v) = 0 \\
& h'(u) = a - 2 \\
& \omega'(u) = 0 \\
& h'(t) = a - 3 \\
& h'(u_3) = a - 3 \\
& h'(u_4) = a - 3 \\
& h'(u_2) = a - 3 \\
\end{split}
\]

\paragraph*{\textbf{Case 1cIII}: $\omega(u_1) = 1$}
\[
\begin{split}
& h(v) = a \\
& \omega(v) = 1 \\
& h(t) = a - 2 \\
& h'(t) = a - 3 \\
& h(u) = a - 1 \\
& \omega(u) = - 1 \\
& h(u_1) = a - 2 \\
& \omega(u_1) = 1 \\
& h(u_2) = a - 3 \\
& h(u_3) = a - 4 \\
& h(u_4) = a - 3 \\
\end{split}
\]
Result:
\[
\begin{split}
& \omega'(u_1) = 0 \\
& h'(u_1) = a - 1 \\
& h'(v) = a - 2 \\
& \omega'(v) = -1 \\
& h'(u) = a - 2 \\
& \omega'(u) = 0 \\
& h'(t) = a - 3 \\
& h'(u_3) = a - 4 \\
& h'(u_4) = a - 3 \\
& h'(u_2) = a - 3 \\
\end{split}
\]
\pagebreak

\paragraph*{\textbf{Case 2}: $\omega(v) = 0$}
\[
\begin{split}
& h(v) = a \\
& \omega(v) = 0 \\
& h(t) = a - 1 \\
& h(u) = a - 1 \\
& h'(t) = a - 2 \\
\end{split}
\]

Result:
\[
\begin{split}
& h'(v) = a \\
& \omega'(v) = 1 \\
& h'(t) = a - 2 \\
& h'(u) = a - 1 \\
\end{split}
\]
The height of the subtree did not change we can stop.

\paragraph*{\textbf{Case 3}: $\omega(v) = -1$}
\[
\begin{split}
& h(v) = a \\
& \omega(v) = 0 \\
& h(t) = a - 1 \\
& h(u) = a - 2 \\
& h'(t) = a - 2 \\
\end{split}
\]

Result:
\[
\begin{split}
& h'(v) = a - 1 \\
& \omega'(v) = 0 \\
& h'(t) = a - 2 \\
& h'(u) = a - 2 \\
\end{split}
\]

Continue with the node $v$.