% This file contains \TeX\ and PostScript definitions for Leo.  
% \input this file into any document produced by Leo which 
% uses Leo pictures or tables.
%
%Define PostScript macros for handling pictures.
\special{!
%%%%%%%%%%%%%%%%%%%%
/LDiff {
% Compute the difference between two points
% x0 y0 x1 y1
1 index 4 index sub 1 index 4 index sub
mark 7 3 roll cleartomark
} def
%%%%%%%%%%%%%%%%%%%%
/LAdd {
% Add offset to a point
% x0 y0 dx dy
exch 4 1 roll add 3 1 roll add exch
} def
%%%%%%%%%%%%%%%%%%%%
/LDist {
% Compute distance between two points.
% x0 y0 x1 y1
LDiff dup mul exch dup mul add sqrt
} def
%%%%%%%%%%%%%%%%%%%%
}
\special{!
/LXYScale {
% Multiply a coordinate by a scalar.
% x0 y0 t
dup 4 1 roll mul 3 1 roll mul exch
} def
%%%%%%%%%%%%%%%%%%%%
/LAng {
% Compute the angle of slope between two points.
% x0 y0 x1 y1
LDiff exch atan 
} def
%%%%%%%%%%%%%%%%%%%%
/LOnLine { 
% x0 y0 x1 y1 t 
5 copy pop LDiff
3 2 roll LXYScale
4 2 roll pop pop
LAdd
} def
%%%%%%%%%%%%%%%%%%%%
/LOnOrtho {
% x0 y0 x1 y1 t
5 copy pop LDiff
% rotate by 90 degrees
-1 mul exch
3 2 roll LXYScale
LAdd
4 2 roll pop pop
} def
%%%%%%%%%%%%%%%%%%%%
}
\special{!
/LOnCircle { 
% x0 y0 x1 y1 t 
180 mul 3.1416 div
5 copy pop LDist
% x0 y0 x1 y1 t r
exch dup cos exch sin 3 2 roll LXYScale
% x0 y0 x1 y1 x y
4 2 roll pop pop LAdd
} def
%%%%%%%%%%%%%%%%%%%%
/LOnBox {
% x0 y0 x1 y1 t
pop 3 1 roll pop pop
} def
%%%%%%%%%%%%%%%%%%%%%%%%%%%%%%%%%%%%%%%%
%%%%%%%%%%%%%%%%%%%%%%%%%%%%%%%%%%%%%%%%
% for use in LOnSpline
/LA { add 0.5 mul store } def
/LB { exch def } def
/LD { 0 def } def
%%%%%%%%%%%
/LE { 
/tt LB
/a3y LB % define parameters
/a3x LB
/a2y LB
/a2x LB
/a1y LB
/a1x LB
/a0y LB
/a0x LB
/b0x LD % define local variables
/b0y LD
/b1x LD
/b1y LD
/b2x LD
/b2y LD
/c0x LD
/c0y LD
/c1x LD
/c1y LD
/d0x LD
/d0y LD
} def
}
\special{!
/LF {
/b0x a0x a1x LA % compute coordinates of the new half splines
/b0y a0y a1y LA
/b1x a1x a2x LA
/b1y a1y a2y LA
/b2x a2x a3x LA
/b2y a2y a3y LA
/c0x b0x b1x LA
/c0y b0y b1y LA
/c1x b1x b2x LA
/c1y b1y b2y LA
/d0x c0x c1x LA
/d0y c0y c1y LA
} def
}
\special{!
/LOnSpline {
% a0x a0y a1x a1y a2x a2y a3x a3y tt
50 dict begin
LE % define parameters and local variables
7 {% iterate 7 times
LF % compute new half splines
tt 0.5 le { % on the left half
/tt tt 2 mul store
/a1x b0x store
/a1y b0y store
/a2x c0x store
/a2y c0y store
/a3x d0x store
/a3y d0y store }
{ % on the right half
/tt tt 2 mul 1 sub store
/a0x d0x store
/a0y d0y store
/a1x c1x store
/a1y c1y store
/a2x b2x store
/a2y b2y store } ifelse
} repeat
a0x a0y a3x a3y tt LOnLine % Interpolate between two end points
end
} def
}
\special{!
%%%%%%%%%%%%%%%%%%%%
/LLine { 
% x0 y0 x1 y1 
newpath moveto lineto 
stroke 
} def 
%%%%%%%%%%%%%%%%%%%%
/LUVec { % compute a unit vector from (x0,y0) to (x1,y1)
% x0 y0 x1 y1
4 copy LDist
5 1 roll LDiff 3 -1 roll 1 exch div LXYScale
} def
%%%%%%%%%%%%%%%%%%%%
/LArc { 
% x0 y0 x1 y1 x2 y2
newpath 
6 copy pop pop LDist
7 copy pop pop pop LAng
8 copy pop pop 4 2 roll pop pop LAng
% x0 y0  x1 y1 x2 y2 r th1 th2
7 3 roll pop pop moveto
% x0 y0 r th1 th2
arc stroke
} def 
}
\special{!
%%%%%%%%%%%%%%%%%%%%
/LCircle {
% x0 y0 x1 y1
4 copy LDist 3 1 roll pop pop
% x0 y0 r
newpath 3 copy 0 LAdd moveto
0 360 arc stroke
} def
%%%%%%%%%%%%%%%%%%%% 
/LDot { 
% x0 y0 
2 copy moveto 
currentlinewidth 1.7 mul 0 rmoveto currentlinewidth 1.7 mul 
0 360 arc 
closepath fill
} def 
%%%%%%%%%%%%%%%%% 
/LOdot {
% x0 y0
2 copy
gsave newpath 1 setgray % draw in white
2 copy moveto % move to center of circle
currentlinewidth 1.7 mul 0 rmoveto % move to radius of circle
currentlinewidth 1.7 mul % compute radius of circle
0 360 arc closepath fill % fill in the center
0 setgray % draw in black
2 copy moveto % move to center of circle
currentlinewidth 1.7 mul 0 rmoveto % move to the radius of circle
currentlinewidth 1.7 mul % compute radius of circle
0 360 arc LThin stroke % draw the circle
grestore } def
}
\special{!
%%%%%%%%%%%%%%%%% 
/LNarrow { 0.6 setlinewidth } def
%%%%%%%%%%%%%%%%% 
/LWide { 1.2 setlinewidth } def
%%%%%%%%%%%%%%%%%
/LThin { 0.3 setlinewidth } def
/LFine { 0.3 setlinewidth } def
%%%%%%%%%%%%%%%%%
/LDashed { [3] 0 setdash } def
%%%%%%%%%%%%%%%%%
/LNoDash { [] 0 setdash } def
%%%%%%%%%%%%%%%%%
/LSpline {
% x0 y0 x1 y1 x2 y2 x3 y3
8 -2 roll
moveto curveto stroke
} def
%%%%%%%%%%%%%%%%% 
/LTextloc {
% Remove arguemnts from stack, no Postscript drawing needed
% x0 y0 a
pop pop pop pop pop
} def
}
\special{!
%%%%%%%%%%%%%%%%%
/LTextlocx {
% remove arguments from stack, no Postscript drawing needed
% x0 y0 x1 y1 x2 y2 a
pop pop pop pop pop pop pop
} def
%%%%%%%%%%%%%%%%%
/LVecx { currentlinewidth 5 mul neg } def
/LVecxx { currentlinewidth 4 mul neg } def
/LVecy { currentlinewidth 3 mul } def
/LVector {
% Draw a vector from (x0,y0) to (x1,y1)
4 copy 4 copy LUVec LVecxx LXYScale LAdd LLine
% need to draw arrow head
% x0 y0 x1 y1
2 copy moveto
4 copy LAng
gsave currentpoint translate rotate
LVecx LVecy neg moveto LVecxx 0 lineto LVecx LVecy lineto 
0 0 lineto closepath fill grestore
pop pop pop pop
} def
}
\special{!
%%%%%%%%%%%%%%%%%
/LSArc {
% Draw a standard arc centered at (x0,y0) from (x1,y1) to (x2,y2)
% x0 y0 x1 y1 x2 y2
newpath
6 copy pop pop LAng
7 copy pop 4 2 roll pop pop LAng
% x0 y0 x1 y1 x2 y2 th1 th2
6 2 roll pop pop pop pop
% x0 y0 th1 th2
gsave 0.3 setlinewidth
4 copy pop dup cos 10 mul exch sin 10 mul LAdd moveto
10 3 1 roll arc stroke
grestore
} def
%%%%%%%%%%%%%%%%%
/LCorner {
% Draw a perpendicular indicator at the corner at x0 y0
% x0 y0 x1 y1 x2 y2
gsave
0.3 setlinewidth
newpath
6 copy pop pop LUVec 5 LXYScale
8 copy 6 4 roll pop pop pop pop LUVec 5 LXYScale
8 4 roll pop pop pop pop
% x0 y0 dx1 dy1 dx2 dy2
6 copy pop pop LAdd moveto 6 copy LAdd LAdd lineto
6 copy 4 2 roll pop pop LAdd lineto stroke
grestore
pop pop pop pop pop pop
} def
}
\special{!
%%%%%%%%%%%%%%%%%
/LBegin {
% size
20 add dict begin
} def
%%%%%%%%%%%%%%%%%%
/LEnd {
end
} def
}% End of special code
%%%%%%%%%%%%%%%%%%%%%%%%%%%%%%%%%%%%%%%%%%%%%%%%%%%%%%%%
%%%%%%%%%%%%%%%%%%%%%%%%%%%%%%%%%%%%%%%%%%%%%%%%%%%%%%%%
%%%%%%%%%%%%%%%%%%%%%%%%%%%%%%%%%%%%%%%%%%%%%%%%%%%%%%%%
%%%%%%%%%%%%%%%%%%%%%%%%%%%%%%%%%%%%%%%%%%%%%%%%%%%%%%%%
% Following are two definitions of \LPic.  The first definition
% is the standard Leo definition of \LPic.  The second definition
% can be used in conjunction with macros provided by Springer Verlag for 
% publishing books through submission of a \TeX\ document.
% You can rewrite \LPic for your own purposes.
%
% The first argument is the horizontal dimension of the picture in 
% centimeters.
% The second argument is the vertical dimension of the picture in
% centimeters.
% The third argument is the Postscript file name.
% The fourth argument is the Figure number.
% The fifth argument is the Figure caption.
% The sixth argument are the text labels in the picture.
%%%%%%%%%%%%%%%%%%%%%%%%%%%%%%%%%%%%%%%%%%%%%%%%%%%%%%%%
\def\LPic #1 #2 #3 #4#5#6{%
\hbox to #1cm{\vbox{%
\hbox to #1cm{\vbox to#2cm{%
\offinterlineskip\vfill\hbox to #1cm{%
\hbox{}\ignorespaces#6%
\hbox{}\special{psfile=#3}%
\hfil}%
}\hfil}%
\hbox to #1cm{\hfil#4 #5\hfil}}\hfil}}
%%%%%%%%%%%%
%%%%%%%%%%%% for Springer Verlag
%\def\LPic #1 #2 #3 #4#5#6{%
%\begfig #2cm%
%\Lpic #1 #2 #3 {#6}%
%\figure{#4}{#5}%
%\endfig
%}
%%%%%%%%%%%
%%%%%%%%%%% Used in definition of \LPic
\def\Lpic #1 #2 #3 #4{%
\vbox to 0cm%
{\offinterlineskip\vfill\hbox to\hsize{\hfill\hbox to#1cm{%
\hbox{}\hbox to0pt{\special{psfile=#3}}\hbox to0pt{\ignorespaces#4}%
\hfill}\hfill}}}
%%%%%%%%%%%
% These macros allow text labels to be drawn at given coordinates in a 
% picture.  We use "bp" (big points) to match the dvips default scale
% of 72 units per inch.
\def\atxy #1 #2 #3 {\smash{\rlap{\kern#1bp\raise#2bp\hbox to0pt{#3}}}}
\def\LText#1%
{{\vbox to 0pt{\vss\vbox{\hbox to0pt{#1\hss}\vbox{}}\vss}}}
\def\RText#1%
{\smash{\vbox to0pt{\vss\vbox{\hbox to0pt{\hss#1}\vbox{}}\vss}}}
\def\CText#1{\vbox to0pt{\vss{\hbox to0pt{\hss#1\hss}}\vss}}
\def\BText#1%
{\smash{\vbox to0pt{\vbox{\hbox to0pt{\hss#1\hss}\vbox{}}\vss}}}
\def\AText#1%
{\smash{\vbox to0pt{\vss\vbox{\hbox to0pt{\hss#1\hss}\vbox{}}}}}
%%%%%%%%%%%%
% Table macros
\def\LTable#1{\vbox{\tabskip=0pt\offinterlineskip\halign{#1}}}
\def\LTt{&\strut&}
\def\LWt{&\strut\vrule&}
\def\LHl{\noalign{\hrule}}
\def\LCc#1{\strut\hfil\quad#1\quad\hfil}
\def\LLc#1{\strut\quad#1\hfil}
\def\LRc#1{\strut\hfil#1\quad}
\def\LLFc#1{\strut#1\hfil}
\def\LRFc#1{\strut\hfil#1}
\def\LSp#1{\multispan{#1}}

