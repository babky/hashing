\documentclass{article}

\usepackage[utf8]{inputenc}

\usepackage{color}
\usepackage{amsmath}
\usepackage{amsfonts}
\usepackage{amssymb}
\usepackage{amsthm}
\usepackage{mathabx}
\usepackage{geometry}
\usepackage{graphicx}
\geometry{
	b5paper,
	margin=1.5cm,
	top=1.75cm,
	bottom=1.75cm
}

\newcommand{\llinnm}[2]{\operatorname{L}_{\operatorname{lin}}({#1}, {#2})}
\newcommand{\llinn}[1]{\llinnm{#1}{#1}}
\newcommand{\hlinr}[1]{\operatorname{H}_{\operatorname{lin}}^{#1}}
\newcommand{\hlin}{\operatorname{H}_{\operatorname{lin}}}
\newcommand{\leap}[3]{\operatorname{\mathbf{leap}}({#1}, {#2}, {#3})}
\newcommand{\hfact}[2]{\operatorname{H}_{\operatorname{factor}}({#1}, {#2})}
\newcommand{\rot}[2]{\operatorname{H}_{\operatorname{rot}}^{{#1}, {#2}}}

\newcommand{\bin}[3]{\operatorname{\mathbf{bin}}({#1}, {#2}, {#3})}
\newcommand{\lbin}[2]{\operatorname{\mathbf{lbin}}({#1}, {#2})}
\newcommand{\vbin}[2]{\operatorname{\mathbf{bin}}({#1}, {#2})}
\newcommand{\vlbin}[1]{\operatorname{\mathbf{lbin}}({#1})}


\newcommand{\probs}[2]{\operatorname{\mathbf{Pr}}_{{#1}}\left[{#2}\right]}
\newcommand{\prob}[1]{\probs{}{#1}}
\newcommand{\expects}[2]{\operatorname{\mathbf{E}}_{{#1}}\left[{#2}\right]}
\newcommand{\expect}[1]{\expects{}{#1}}
\newcommand{\inu}{\in_U}

\newtheorem{lemma}{Lemma}
\newtheorem{theorem}{Theorem}
\newtheorem{claim}{Claim}
\newtheorem{corollary}{Corollary}

\title{A probabilistic proof of existence of a small system of functions having small largest bins}
\author{Martin Babka, Michal Kouck\'y}

\begin{document}

\maketitle

\begin{abstract}
We show that there is a system of functions for the balls and bins problem achieving $O(\log n/ \log \log n)$ maximal bin with at least constant probability where $n$ is the number of balls (and bins). 
The size of the system is linear with respect to the number of balls.
The proof is probabilistic.
\end{abstract}

\section{Setting and notation}

By $[k]$ we denote the set $\{0, \dots, k - 1\}$. 
Let $n$ be a positive number.
Assume that the universe is $[n^2]$ and that we have to place $n$ bins chosen from the universe, i.e. we store a set of size $n$.
Let $\mathcal{H}$ be a system of all functions mapping $[n^2]$ to $[n]$.
By $\lbin{f}{S}$ we denote the size of the largest bin when the balls from the set $S$ are placed according to the function $f$.

\section{The construction}
\begin{lemma}
It holds $\probs{f\in_U\mathcal{H}}{\lbin f S \geq c \log n / \log \log n} \leq n^{-c + 1 + o(1)}$.
\end{lemma}
\begin{proof}
Union bound the probability that a single bucket has the required size. In the estimates use the fact that $\binom{n}{k} \leq (ne/k)^k$.
\end{proof}

\begin{corollary}
\label{corollary-single-function}
It holds $\probs{f\in_U\mathcal{H}}{\lbin f S \geq c \log n / \log \log n} \leq n^{-c + 2}$.
\end{corollary}
\begin{corollary}
\label{corollary-single-function-polylog}
It holds $\probs{f\in_U\mathcal{H}}{\lbin f S \geq c \log n} \leq n^{-c \log \log n + 1}$.
\end{corollary}

\begin{theorem}
\label{theorem-fixed-p-c}
Let $p \in (0, 1)$, $c > 2$ and $n \in \mathbb{N}$ be large enough. If $\frac{n(\log n + \log e)}{p(c \log n - \log e + \log{p})} < h$, then there is a system of functions $H$, which contains mappings from $[n^2]$ to $[n]$, of size $h$ such that 
\[
\forall S \in \binom{[n^2]}{n} \colon \probs{f \in_U H}{\lbin{f}{S} \geq (c + 2) \log n / \log \log n} \leq p.
\]
\end{theorem}
\begin{proof}
Let $h$ be a positive integer and it is the size of the constructed system of functions.
We create $H$ so that it consists of $h$ functions chosen uniformly from $\mathcal{H}$ with possible repetitions, i.e. $H$ is chosen uniformly at random from $\mathcal{H}^h$.

Now notice that a system $H$ of size $h$ having small large bins exists if the following statement holds:
\[
\probs{H \in_U \mathcal{H}^h}{\exists S \colon \probs{f \in_U H}{\lbin{f}{S} \geq (c + 2) \log n / \log \log n } > p} < 1.
\]

We can union bound the above probability as
\[
\binom{n^2}{n} \probs{H \in_U \mathcal{H}^h}{\probs{f \in_U H}{\lbin{f}{S} > (c + 2) \log n / \log \log n } > p}.
\]

Now for a fixed $S$ and a random $H$ we estimate \[\probs{f \in_U H}{\lbin{f}{S} \geq (c + 2) \log n / \log \log n } > p.\]
If there are at least $ph$ functions in $H$ such that they break the $(c + 2)\log n / \log \log n$ bound on the size of the largest bin the above event occurs. Since the functions are chosen independently and uniformly using Corollary~\ref{corollary-single-function} we upper bound the probability as follows.

\begin{alignat*}{1}
& \probs{H \in_U \mathcal{H}^h}{\probs{f \in_U H}{\lbin{f}{S} \geq (c + 2) \log n / \log \log n} > p} \\
	& \quad < \binom{h}{ph} \cdot \left(n^{-c}\right)^{ph} \\
	& \quad < \left(\frac{eh}{ph}\right)^{ph} \cdot n^{-cph} \\
	& \quad = 2^{ph(\log e - \log{p}) -cph \log n} \\
	& \quad = 2^{ph(\log e - \log{p} -c \log n)}
\end{alignat*}

Since $\binom{n^2}{n} < (ne)^n$ we may conclude:

\begin{alignat*}{1}
& \probs{H \in_U \mathcal{H}^h}{\exists S \in \binom{[n^2]}{n} \colon \probs{f \in_U H}{\lbin{f}{S} \geq (c + 2) \log n / \log \log n} > p} \\ 
	& \quad \leq (ne)^n \cdot 2^{ph(\log e - \log{p} -c \log n)} \\
	& \quad = 2^{n(\log n + \log e) + ph(\log e - \log{p} - c \log n)}
\end{alignat*}

We are interested in the case when the above probability is less than one, i.e. the exponent of the above expression is negative.

\begin{alignat*}{1}
n(\log n + \log e) + ph(\log e - \log{p} -c \log n) & < 0 \\
n(\log n + \log e) & < -ph(\log e - \log{p} - c \log n) \\
\frac{n(\log n + \log e)}{p(c \log n + \log{p} - \log e )} & < h
\end{alignat*}
\end{proof}

For fixed constants $p$ and $c$ and large enough $n$ we have that $h = \frac{n}{cp} + \Omega(1)$.

\begin{theorem}
Let $\epsilon > 0$, $n \in \mathbb{N}$ be large enough. If $n^{1 + \epsilon} < h$, then there is a system of functions $H$, which contains mappings from $[n^2]$ to $[n]$, of size $h$ such that 
\[
\expects{f \in_U H}{\lbin{f}{S}} = O(\log n / \log \log n).
\]

The constant in the big-O notation depends on $\epsilon$.
\end{theorem}
\begin{proof}
Let $c \in [2, n \log \log n / \log n]$ and set $p := c^{-(1 + \epsilon)}$.

From Theorem~\ref{theorem-fixed-p-c} it follows that
\begin{alignat*}{1}
& \probs{H \in_U \mathcal{H}^h}{\exists S \in \binom{[n^2]}{n} \colon \probs{f \in_U H}{\lbin{f}{S} \geq (c + 2) \log n / \log \log n} > c^{-(1 + \epsilon)}} \\ 
& \quad < 2^{n(\log n + \log e) + c^{-(1 + \epsilon)}h(\log e + (1 + \epsilon)\log{c} - c \log n)} \\
& \quad < 2^{2n\log n + c^{-(1 + \epsilon)}h((1 + \epsilon)\log c - \frac{2c}{3} \log n)}.
\end{alignat*}

The above expression is maximized for $c = n \log \log n / \log n$ and hence may be bounded as
\begin{alignat*}{1}
& 2^{2n\log n + (n \log \log n / \log n)^{-(1 + \epsilon)}h((1 + \epsilon)\log n - \frac{2n \log \log n}{3})} \\ 
	& \quad < 2^{2n \log n - h(n\log \log n / \log n)^{-(1 + \epsilon)} \cdot \frac{n \log \log n}{2}} \\
	& \quad < 2^{2n \log n - \frac{h(\log n)^{1 + \epsilon}}{2n^\epsilon (\log \log n)^\epsilon}} \\
	& \quad < 2^{2n \log n - \frac{3h\log n}{n^\epsilon}}
\end{alignat*}

As in the previous case we union bound not only over all choices of $S$ but also over at most $n \log \log n / \log n$ choices of $c$. If the probability is less then one, i.e. the exponent is negative, we get the existence of a system $H$ having the expected value of $\lbin{f}{S}$ bounded. In such case the following has to be true:
\begin{alignat*}{1}
3n \log n - \frac{3h \log n}{n^\epsilon} & < 0 \\
n^{1 + \epsilon} & < h.
\end{alignat*}
\end{proof}

A similar minimization can be done in order to estimate the size of systems which will have $\expects{f \in_U H}{\lbin{f}{S} \leq c \log n}$.
We use Corollary~\ref{corollary-single-function-polylog} instead of Corollary~\ref{corollary-single-function} and prove the following corollary of Theorem~\ref{theorem-fixed-p-c}.

\begin{theorem}
\label{theorem-fixed-p-c-polylog}
Let $p \in (0, 1)$, $c > 1$ and $n \in \mathbb{N}$ be large enough. If $\frac{n(\log n + \log e)}{p(c \log n \log \log n - \log e - \log{p})} < h$, then there is a system of functions $H$, which contains mappings from $[n^2]$ to $[n]$, of size $h$ such that 
\[
\forall S \in \binom{[n^2]}{n} \colon \probs{f \in_U H}{\lbin{f}{S} \geq (c + 1) \log n} \leq p.
\]
\end{theorem}
\begin{proof}
The same as in the case of Theorem~\ref{theorem-fixed-p-c}. The additional factor $\log \log n$ in the term with $c$ comes from Corollary~\ref{corollary-single-function-polylog}.
\end{proof}

\begin{corollary}
There exists a system $H$ of size $\Omega(n/\log \log n)$ satisfying \[\expects{f \in_U H}{\lbin{f}{S}} = O(\log n).\]
\end{corollary}
\begin{proof}
First fix $\epsilon > 0$. We put $p:=c^-{1+\epsilon}$ for each $c \in [2, n/\log n]$ and compute the estimate on $h$ using Theorem~\ref{theorem-fixed-p-c-polylog}.

We estimate the bound on $h$ from above as follows.
\begin{align*}
    \frac{n(\log n + \log e)}{c^{-(1 + \epsilon)}(c \log n \log \log n - \log e - (1 + \epsilon) \log c)} < \frac{2n\log n}{c^{-(1+\epsilon)}c/2\log n \log \log n} < \frac{4n}{2^{-\epsilon}\log \log n}.
\end{align*}

Hence if $h > \frac{4n}{2^{-\epsilon}\log \log n} = \Omega(n/\log \log n)$, then there exists a system $H$ having $O(\log n)$ expected largest bin.
\end{proof}

Let us note that if we multiply the size of $H$ by a constant, then we may satisfy all of the required properties with high probability.
\end{document}


