\section{Slovníky}
\begin{frame}
	\frametitle{Slovníky}
	
	\begin{block}{Statické}
		\begin{itemize}
			\item Napríklad perfektné hashovanie, typicky FKS schéma.
			\item Garantovaný čas na vyhľadávanie.
			\item Možná dynamizácia spolu so zachovaním $O(1)$ času na vyhľadanie prvku.
		\end{itemize}
	\end{block}

	\begin{block}{Dynamické}
		\begin{itemize}
			\item Napríklad stromy (binárne, vyvažované, van Emde Boasove stromy), ale zaoberáme sa ...
			\item ... hashovaním -- veľa schém, hlavne univerzálne hashovanie.
			\item Najlepšie metódy dávajú worst-case garanciu na vyhľadanie, problémom je update či zložitý univerzálny systém.
		\end{itemize}
	\end{block}
\end{frame}

\section{Hashovanie}
\begin{frame}
	\frametitle{Schémy}
	
	\begin{block}{Cuckoo hashing}
		\begin{itemize}
			\item Použijeme dve tabuľky a dve funkcie -- prvok $x$ je na pozícii $T_1[h_1(x)]$ alebo $T_2[h_2(x)]$.
			\item Teda $O(1)$ vyhľadanie a mazanie. Vkladanie môže zlyhať, ale je to nepravdepodobné.
			\item Vyžadujú veľmi kvalitný uniformný hashovací systém -- $O(1)$ teoreticky ale pri použití už spôsobuje mierny problém.
		\end{itemize}
	\end{block}
	
	\begin{block}{Two choices}
		\begin{itemize}
			\item S vysokou pravdepobodnosťou dosahuje $O(\log \log n)$ čas na vyhľadanie.
			\item Používa aspoň dve hashovacie funkcie a dá sa kombinovať aj s jednoduchými univerzálnymi systémami.
		\end{itemize}
	\end{block}
\end{frame}

\begin{frame}	
	\frametitle{Systémy funkcií}
	
	\begin{itemize}
		\item \emph{Polynómy a ich modifikácie} -- jednoduché a rýchle, ale nefungujú s two-choices, otvoreným hashovaním (ak použijeme malý stupeň).
		\item \emph{Lineárne funkcie} -- experimenty s two-choices sú ok, ale nemáme dôkaz.
		\item \emph{Tabulačné hashovanie} -- klúč rozdelíme na podreťazce a tie hashujeme nezávisle, t.j. jednoduché a rýchle. Majú výhodné teoretické vlastnosti, fungujú s two-choices, s linear probing, ale poskytujú len insertion-only kukučkové hashovanie.
		\item \emph{Uniformné systémy} -- výhodné teoretické vlastnosti, ale pomalé, použité s kukučkovým hashovaním.
	\end{itemize}
\end{frame}

\section{Výsledky a pokračovanie}
\begin{frame}
	\frametitle{Limit function}

	\begin{itemize}
		\item \emph{Limit function} je maximálna dĺžka reťazca, ktorú povolíme. 
		\item Ak po vložení prvku nie je dodržaná, tabuľku prehashujeme.
		\item Dá sa použiť pri separovaných reťazcoch, ako aj s otvoreným hashovaním.
		\item Extrakcia z rôznych univerzálnych systémov:
		\begin{itemize}
			\item uniformné systémy -- $\log n / \log \log n$, implementačne nevýhodné, 
			\item lineárne funkcie -- $\log n \log \log n$, jednoduchý, ale nie úplne výhodný systém,
			\item two choices -- $\log \log n$, funguje aj s otvoreným hashovaním.
		\end{itemize}
	\end{itemize}
\end{frame}

\begin{frame}
	\frametitle{Čo ďalej?}

	\begin{itemize}
		\item univerzálne hashovanie a nové limit funkcie pre linear probing
		\item prakticky rýchle hashovanie založené na linear probing schéme alebo rýchle kukučkové hashovanie
		\item tabulačné hashovanie + kukučka + updaty -- máme len statickú verziu
		\item prečo sú lineárne zobrazenia kvalitné spolu s two-choices
		\item ďalšie vlastnosti či vylepšenie už známych výsledkov o lineárnych funkciách
	\end{itemize}
\end{frame}
