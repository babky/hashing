\input amstex
\input amsppt.sty

\magnification=\magstep1

\proclaim{Theorem}Let $u$ and $t$ be non-zero natural numbers and 
let $p$ be a prime.  Then the family of all linear mappings from 
$\Bbb Z^u_p$ into $\Bbb Z_p^t$ is $1$-universal.  
\endproclaim

\demo{Proof}Let $A=\{\vec {a}_1,\vec {a}_2,\dots,\vec {a}_u\}$ be a base of $
\Bbb Z_p^u$. It is well 
known that for every mapping $f:A@>>>\Bbb Z_p^t$ there exists a 
unique linear mapping extended $f$ and $|\Bbb Z^t_p|=p^t$. Thus the 
number of linear mapping from $\Bbb Z_p^u$ into $\Bbb Z^t_u$ is $
(p^t)^u=p^{tu}$. Let 
$\vec {x}$ and $\vec {y}$ be distinct vectors of $\Bbb Z_p^u$ and let $
j\in \{1,2,\dots,u\}$ such 
that if $\vec {x}-\vec {y}=\sum_{i=1}^nb_i\vec {a}_i$ then $b_j\ne 
0$ (since $\vec {0}\ne\vec {x}-\vec {y}$ such $j$ 
exists). Then $f:\Bbb Z^u_p@>>>\Bbb Z_p^t$ is a 
linear mapping with $f(x)=f(y)$ if and only if 
$f(\vec {x}-\vec {y})=f(\vec {x})-f(\vec {y})=\vec {0}$. Thus for every mapping 
$g:A\setminus \{a_j\}@>>>\Bbb Z_p^t$ there exists exactly one linear mapping 
$f:\Bbb Z^u_p@>>>\Bbb Z^t_p$ with $f(x)=f(y)$ because necessarily
$$f(\vec {a}_j)=\sum_{i=1,i\ne j}^ib_ig(\vec {a}_i).$$
Hence the number of linear mappings $f:\Bbb Z^u_p@>>>\Bbb Z_p^t$ with 
$f(\vec {x})=f(\vec {y})$ is equal to $(p^t)^{u-1}=p^{t(u-1)}$. From $
p^{t(u-1)}=\frac {p^{tu}}{p^t}$ it 
follows that the family of all linear mappings from $\Bbb Z^u_p$ into 
$\Bbb Z^T_p$ is $1$-universal. \qed
\enddemo

\proclaim{Theorem 3}For every $\varepsilon$ with $0<\varepsilon <
1$ there exists a 
constant $c_{\varepsilon}>0$ such that for all natural numbers $w$ and $
t$ 
and for every set $A\subseteq \Bbb Z_2^w$ with $|A|\ge c_{\varepsilon}
t2^t$ we have 
$$\bold P(T(A)=\Bbb Z^t_2)\ge 1-\varepsilon$$
for every random uniformly chosen linear mapping 
$T:\Bbb Z_2^w@>>>\Bbb Z^t_2$.
\endproclaim

\demo{Proof}Set $u=\lceil\log(\frac {2|A|}{\varepsilon})\rceil$. Let $
T_1:\Bbb Z_2^u@>>>\Bbb Z_2^t$ be a random 
uniformly chosen surjective linear mapping (since $u\ge t$ such 
mapping exists). Fix $T_1$. Then 
for every random uniformly chosen linear mapping $T:\Bbb Z_2^w@>>>
\Bbb Z_2^t$ 
there exists a linear mapping $T_0:\Bbb Z_2^w@>>>\Bbb Z_2^u$ with $
T=T_0\circ T_1$ 
and $T_0$ is a random linear mapping with uniform distribution. 
Since the family of all linear mappings from $\Bbb Z_2^w$ into $\Bbb Z^
u_2$ is 
$1$-universal we conclude that 
$$\bold P(T_0(\vec {x})=T_0(\vec {y}))=2^{-u}$$
for all distinct vectors $\vec {x}$ and $\vec {y}$ from $\Bbb Z^w_
2$. If $d_A$ is the number of 
all pairs of distinct vectors $\vec {x},\vec {y}\in A$ with $T_0(
\vec {x})=T_0(\vec {y})$ then 
the expected value of a random variable $d_A$ is
$$\bold E(d_A)=\binom {|A|}22^{-u}.$$
If $|T_0(A)|\le\frac {|A|}2$ then there exist at least $\frac {|A
|}2$ pairs of distinct 
vectors $\vec {x},\vec {y}\in A$ with $T_0(\vec {x})=T_0(\vec {y}
)$. By Markov inequality 
$$\bold P(c_A\ge k\binom {|A|}22^{-u})\le\frac 1k.$$
Thus if we set $k=\frac {|A|2^u}{2\binom {|A|}2}$ then we obtain  
$$\bold P(|T_0(A)|\ge\frac {|A|}2)\le \bold P(d_A\ge\frac {|A|}2)
\le\frac {2\binom {|A|}2}{|A|2^u}=\frac {|A|-1}{2^u}<\frac {|A|}{
2^u}\le\frac {\varepsilon |A|}{2|A|}=\frac {\varepsilon}2$$
We can summarize that
$$\bold P(T(A)\ne \Bbb Z^t_2\wedge |T_0(A)|\le\frac {|A|}2)\le\frac {
\varepsilon}2.$$

Secondly we compute $\bold P(T(A)\ne \Bbb Z_2^t\wedge |T_0(A)|\ge\frac {
|A|}2)$. By Theorem 6 
for $T_1:\Bbb Z^u_2@>>>\Bbb Z^t_2$ and $T_0(A)\subseteq \Bbb Z^u_
2$, we have
$$\bold P(T(A)=T_1(T_0(A))\ne \Bbb Z^t_2\wedge |T_0(A)|\ge\frac {
|A|}2)\le\alpha^{u-t-\log t+\log\log(\frac 1{\alpha})}$$
where $\alpha =1-\frac {|T_0(A)|}{2^u}$. Clearly
$$\alpha <1-\frac {|A|}22^{-\log(\frac {2|A|}{\varepsilon})-1}=1-\frac {
|A|}{4\frac {2|A|}{\varepsilon}}=1-\frac {\varepsilon}8\le e^{-\frac {
\varepsilon}8}.$$
Set $c_{\varepsilon}=4(\frac 2{\varepsilon})^{\frac 8{\varepsilon}}$. Then we can estimate 
$$\align-\frac {\varepsilon}8(u-t-\log t+&\log\log(\frac 1{\alpha}
))=-\frac {\varepsilon}8(\lceil\log(\frac {2|A|}{\varepsilon})\rceil 
-t-\log t+\log\log(\frac 1{\alpha}))=\\
&-\frac {\varepsilon}8(\lceil\log(\frac {8(\frac 2{\varepsilon})^{\frac 
8{\varepsilon}}t2^t}{\varepsilon})\rceil -t-\log t+\log\log(\frac 
1{\alpha}))\le\\
&-\frac {\varepsilon}8(3+\frac 8{\varepsilon}\log\frac 2{\varepsilon}
-\log\varepsilon +\log t+t-t-\log t+\log(\frac {\varepsilon}8\log 
e))=\\
&-\frac {\varepsilon}8(3-\log\varepsilon +\frac 8{\varepsilon}\log
(\frac 2{\varepsilon})+\log\varepsilon -3+\log\log e)=\\
&-\frac {\varepsilon}8(\frac 8{\varepsilon}\log(\frac 2{\varepsilon}
)+\log\log e)=\\
&\log\frac {\varepsilon}2-\frac {\varepsilon}8\log\log e\le\log\frac {
\varepsilon}2.\endalign$$
Hence we infer that
$$\align \bold P(T(A)=T_1(T_0(A))\ne \Bbb Z^t_2\wedge |T_0(A)|\ge\frac {
|A|}2)\le&\alpha^{u-t-\log t+\log\log(\frac 1{\alpha})}\le\\
&e^{-\frac {\varepsilon}8(u-t-\log t+\log\log(\frac 1{\alpha}))}\le\\
&e^{\log(\frac {\varepsilon}2)}\le e^{\ln(\frac {\varepsilon}2)}=\frac {
\varepsilon}2.\endalign$$
If we connect both alternatives we deduce that
$$\bold P(T(A)=T_1(T_0(A))\ne \Bbb Z^t_2)\le\frac {\varepsilon}2+\frac {
\varepsilon}2=\varepsilon$$
and form this it follows that $\bold P(T(A)=\Bbb Z^t_2)\ge 1-\varepsilon$. \qed
\enddemo

\end

