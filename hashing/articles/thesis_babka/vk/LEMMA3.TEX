\input amstex
\input amsppt.sty


\magnification=\magstep1


The proof of Lemma 3.

\demo{Proof}We prove the statement by induction over $k$. 
The initial step $k=0$. Since $\alpha_0$ is constant then 
$$\bold P(\alpha_0\ge t)=\cases 0&\text{ if }\alpha_0<t,\\
1&\text{ if }\alpha_0\ge t.\endcases $$
On the other hand, in any case 
$$\alpha_0^{0-\log\log(\frac 1t)+\log\log(\frac 1{\alpha_0)}}\ge 
0$$
because $\alpha_0>0$ and if $t\ge\alpha_0$ then $-\log\log(\frac 
1t)+\log\log(\frac 1{\alpha_0})\le 0$ and 
hence 
$$\alpha_0^{0-\log\log(\frac 1t)+\log\log(\frac 1{\alpha_0)}}\ge 
1.$$
Hence the statement is true.

Induction step. Let $k\ge 0$ be a natural number and assume that the 
statement holds for $k$ and we prove it for $k+1$. For simplicity, 
let us denote $c=k-\log\log(\frac 1t)$. Then we have to prove
$$\bold P(\alpha_{k+1}\ge t)\le\alpha_0^{c+1+\log\log(\frac 1{\alpha_
0})}.$$
If $c+1+\log\log(\frac 1{\alpha_0})\le 0$ then the rigth side is at least $
1$ and the 
statement holds. Thus we can restrict ourselves on the case 
$c+1+\log\log(\frac 1{\alpha_0})>0$. The idea of the proof is based on the 
following fact which enable us to fix a value $\alpha_1$. For $a\in 
<0,\alpha_0>$ 
let us define $g(a)=\bold P(\alpha_{k+1}\ge t|\alpha_1=a)$. Then 
$$\bold P(\alpha_{k+1}\ge t)=\sum_{a\in <0,\alpha_0>}\bold P(\alpha_{
k+1}\ge t|\alpha_1=a)\bold P(\alpha_1=a)=\bold E(g).$$
For $0<x<1$, let $f_0(x)=x^{c+\log\log(\frac 1x)}$ and $f(x)=\min
\{1,f_0(x)\}$, 
$f(0)=1$. Observe that if $\beta_i$ for $i=1,2,\dots,k$ are random 
variables and $\beta_0=a\in <0,\alpha_0>$ is constant such that $
0\le\beta_i\le\beta_{i-1}$ and 
$\bold E(\beta_i|\beta_{i-1},\beta_{i-2},\dots,\beta_0)=\beta_{i-
1}^2$ for all $i=1,2,\dots,k$ then 
$\bold P(\beta_k\ge t)=g(a)$ and, by induction hypothesis, $g(a)\le 
f(a)$ for all 
$a\in <0,\alpha_0>$. Observe that $g(0)\le f(0)=1$ because $t\ge 
0$.

Next we investigate a behaviour of the function $\frac {f_0(x)}x$ on the 
interval (0,1).  The derivation of the function $\frac {f_0(x)}x$ on the 
interval (0,1) is 
$$(\frac {f_0(x)}x)'=(c-1+\log\log(\frac 1x)+\log e)\frac {f_0(x)}{
x^2}.$$
Hence if $x<2^{-2^{-c+1-\log e}}$ then $\frac {f_0(x)}x$ is increasing in $
x$ and if 
$x>2^{-2^{-c+1-\log e}}$ then $\frac {f_0(x)}x$ is decreasing in $
x$.  Let us define 
$x_1=2^{-2^{-c}}$.  Since $-c>-c+1-\log e$ we conclude that $\frac {
f_0(x)}x$ is 
increasing in $x_1$.  For every $x\in <x_1,1)$ we have $f(x)=1$ because 
$f_0(x_1)=x_1^{c+\log\log(2^{2^{-c}})}=x_2^{c-c}=1$ and $f_0$ is an increasing 
function.  

The proof is divided into two cases.  Set $x_2=2^{-2^{-c-1}}$then 
$x_1=x_2^2$.  Since $-2<-\log e$ we obtain that $-c-1<-c+1-\log e$ and 
hence $x_2=2^{-2^{-c-1}}>2^{-2^{-c+1-\log e}}$.  

First assume that $\alpha_0\le x_2$. We prove
$$f(x)=\frac {f_0(\alpha_0)x}{\alpha_0}$$
for all $x\in (0,\alpha_0>$. If $\alpha_0\le 2^{-2^{-c+1-\log e}}$ then $\frac {
f_0(x)}x\le\frac {f_0(\alpha_0)}{\alpha_0}$ for 
all $x\in (0,\alpha_0>$ because $\frac {f_0(x)}x$ is an increasing in the interval 
$(0,\alpha_0>$. Hence 
$$f(x)=\frac {f(x)x}x\le\frac {f_0(x)x}x\le\frac {f_0(\alpha_0)x}{
\alpha_0}$$
because $f(x)\le f_0(x)$ for all $x\in (0,1)$.  

If $2^{-2^{-c+1-\log e}}<\alpha_0<x_2$  then for 
$x\in (0,x_1>$ we have 
$$\frac {f(x)}x\le\frac {f(x_1)}{x_1}=\frac 1{x_1}$$
and thus $f(x)=\frac {f(x)}xx\le\frac x{x_1}$. If $x\in <x_1,\alpha_
0>$ then $f(x)=1=\frac xx\le\frac x{x_1}$.
Since 
$$\align\frac {f_0(x_2)}{x_2}=&\frac {(2^{-2^{-c-1}})^{c+\log(-\log
(2^{-2^{-c-1}}))}}{2^{-2^{-c-1}}}=\\
&\frac {(2^{-2^{-c-1}})^{c+\log(2^{-c-1})}}{2^{-2^{-c-1}}}=\\
&\frac {(2^{-2^{-c-1}})^{-1}}{2^{-2^{-c-1}}}=\frac 1{x_2^2}=\frac 
1{x_1}\endalign$$
we infer that $f(x)\le\frac x{x_1}=\frac {f_0(x_2)x}{x_2}\le\frac {
f_0(\alpha_0)x}{\alpha_0}$, because $\frac {f_0(x)}x$ is 
decreasing on the interval $<2^{-2^{-c+1-\log e}},1)$. 

Hence we obtain that 
$$\align \bold P(\alpha_{k+1}\ge t)\le&\bold E(g)\le \bold E(f|x\le
\alpha_0)\le \bold E(\frac {f_0(\alpha_0)}{\alpha_0}x|x\le\alpha_
0)=\\
&\frac {f_0(\alpha_0)}{\alpha_0}\bold E(\alpha_1|\alpha_0)=\alpha_
0f_0(\alpha_0)=\alpha_0^{c+1+\log\log(\frac 1{\alpha_0})}\endalign$$
here the expected value of $g$ and of $f$ are computed 
through $\alpha_1$ and, by the the assumption, $\bold E(\alpha_1|
\alpha_0)=\alpha_0^2$. Thus the 
statement is proved.

Secondly assume that $\alpha_0>x_2$. We prove that then 
$c+1+\log\log(\frac 1{\alpha_0})<0$ and the proof follows. Indeed
$c+1+\log\log(\frac 1{\alpha_0})<c+1+\log\log(\frac 1{x_2})=c+1+\log
(-\log(2^{-2^{-c-1}}))=c+1+\log(2^{-c-1})=c+1-c-1=0$.
\qed
\enddemo

Let $u\ge t$.  We describe as we can random uniformly choose a 
surjective linear mapping $T:Z_2^u@>>>Z^t_2$.  Choose a base 
$\{v_1,v_2,\dots,v_t\}$ of $Z_2^t$.  Observe that if $T:Z_2^u@>>>
Z^t_2$ is a 
surjective linear mapping then there exist a base 
$\{w_1,w_2,\dots,w_u\}$ of $Z_2^u$ and a set $A\subseteq \{w_1,w_
2,\dots,w_u\}$ with 
$|A|=u-t$ such that $T(A)=\vec {0}$ and 
$T(\{w_1,w_2,\dots,w_u\}\setminus A)=\{v_1,v_2,\dots,v_t\}$.  Conversely if 
$T:Z_2^u@>>>Z^t_2$ is a linear mapping such that there exist a base 
$\{w_1,w_2,\dots,w_u\}$ of $Z_2^u$ and a set $A\subseteq \{w_1,w_
2,\dots,w_u\}$ with 
$|A|=u-t$, $T(A)=\vec {0}$ and $T(\{w_1,w_2,\dots,w_u\}\setminus 
A)=\{v_1,v_2,\dots,v_t\}$ 
then $T$ is surjective.  Because a linear mapping is uniquelly 
determined by the image of a base we can proceed as follows:  
we fix a base $\{v_1,v_2,\dots,v_t\}$ of $Z_2^t$, then uniformly choose a 
random base $\{w_1,w_2,\dots,w_u\}$ of $Z_2^u$ and a set 
$A\subseteq \{w_1,w_2,\dots,w_u\}$ with $|A|=u-t$ and finally we uniformly 
choose a random permutation 
$\tau :\{w_1,w_2,\dots,w_u\}@>>>\{v_1,v_2,\dots,v_t\}$.  Then define 
$$T(w_i)=\cases \vec {0}&\text{ if }w_i\in A,\\
\tau (w_i)&\text{ if }w_i\notin A.\endcases $$
Then a linear extension of $T$ is a random uniformly choosen 
surjective linear mapping. 

\end
