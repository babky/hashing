\normalsize
\setcounter{page}{2}
\ \vspace{10mm} 

\noindent At this place, I would like to thank my supervisor doc.~RNDr.~Václav Koubek,~DrSc. for the invaluable advice he gave to me, for the great amount of knowledge he shared and time he generously spent helping me when working on the thesis. I would also like to say thank you to my friends and family.

% Nastavuje dynamické umístění následujícího textu do spodní části stránky
\vspace{\fill}
\noindent I hereby proclaim, that I worked out this master thesis myself, using only the cited resources. I agree that the thesis may be publicly available.

\bigskip
\noindent In Prague on 16\textsuperscript{th} April, 2010 \hspace{\fill}Martin Babka

% Vkládá automaticky generovaný obsah dokumentu
\tableofcontents 

% Přechod na novou stránku
\newpage 

% Následuje strana s abstrakty.
\noindent
\selectlanguage{slovak}
\begin{flushleft}
Názov práce: Vlastnosti univerzálneho hashovania\\
Autor: Martin Babka\\
Katedra (ústav): Katedra teoretické informatiky a matematické logiky\\
Vedúci bakalárskej práce: doc. RNDr. Václav Koubek, DrSc.\\
E-mail vedúceho: vaclav.koubek@mff.cuni.cz\\

\end{flushleft}
\noindent Abstrakt: Cieľom tejto práce je navrhnúť model hashovania s akceptovateľnou ča\-so\-vou zložitosťou aj v najhoršom prípade. Vytvorený model je založený na princípe univerzálneho hashovania a výsledná očakávaná časová zložitosť je konštantná. Ako systém univerzálnych funkcií sme zvolili množinu všetkých lineárnych zobrazení medzi dvomi vektorovými priestormi. Tento systém už bol študovaný s výsledkom predpovedajúcim dobrý asymptotický odhad. Táto práca nadväzuje na predchádzajúci výsledok a rozširuje ho. Tiež ukazuje, že očakávaný amortizovaný čas navrhnutej schémy je konštantný.
\begin{flushleft}
\noindent Kľúčové slová: univerzálne hashovanie, lineárne zobrazenie, amortizovaná časová zložitosť

\selectlanguage{english}
\vspace{10mm}

\noindent
Title: Properties of Universal Hashing\\
Author: Martin Babka\\
Department: Department of Theoretical Computer Science and Mathematical Logic\\
Supervisor: doc. RNDr. Václav Koubek, DrSc.\\
Supervisor's e-mail address: vaclav.koubek@mff.cuni.cz\\

\end{flushleft}
\noindent Abstract: The aim of this work is to create a model of hashing with an acceptable worst case time complexity. The model is based on the idea of universal hashing and the expected time of every operation is constant. The used universal class of functions consists of linear transformations between vector spaces. This class has already been studied with a good asymptotic result. This work improves the result and proves that the expected amortised time of the scheme is constant.
\begin{flushleft}
\noindent Keywords: universal hashing, linear transformation, amortised time complexity

\end{flushleft}
\newpage
