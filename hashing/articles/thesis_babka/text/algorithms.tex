\begin{algorithm}[H]
\caption{Initialisation of the hash table}
\label{algorithm-initialisation}
\floatname{algorithm}{Initialisation of the hash table}
\begin{algorithmic}
\REQUIRE $p \in (0, 1)$, default 0.5
\REQUIRE $m \in \mathbb{N}$, default 4 096
\STATE
\ENSURE the random uniform choice of a linear transformation $Hash$
\STATE
\STATE initialise the value $\alpha'$ according to the delete operation
\STATE use Algorithm \ref{algorithm-scheme-3} to compute the bound $Limit(m)$ for the prescribed $p$ and $\alpha'$
\STATE initialise the variables $\alpha_{min}$ and $\alpha_{max}$
\STATE create the table $T$ of size $m$
\STATE $Size \leftarrow m$
\STATE $Count \leftarrow 0$
\STATE choose the hash function -- random binary matrix $Hash$
\end{algorithmic}
\end{algorithm}

\begin{algorithm}[H]
\caption{Rehash operation}
\label{algorithm-rehash}
\floatname{algorithm}{Rehash operation}
\begin{algorithmic}
\REQUIRE $m \in \mathbb{N}$, default $Size$ \COMMENT{New size of the hash table.}
\STATE
\REPEAT
	\STATE $T'.Initialise(m)$
	\STATE turn off the chain limit rule in $T'$
	\STATE turn off the load factor rule in $T'$
	\FORALL{element $x$ stored in the hash table}
		\STATE $T'.Insert(x)$
	\ENDFOR
\UNTIL{chain limit rule is satisfied in $T'$}
\STATE
\STATE swap $T$ and $T'$
\STATE turn on the chain limit rule in $T'$
\STATE turn on the load factor rule in $T'$
\STATE free $T$
\end{algorithmic}
\end{algorithm}

\begin{algorithm}[H]
\caption{Find operation}
\label{algorithm-find}
\floatname{algorithm}{Find operation}
\begin{algorithmic}
\REQUIRE $x \in U$
\STATE
$i \leftarrow h(x)$
\IF{T[i].Chain.Contains(x)}
	\RETURN \textbf{true} \COMMENT{Successful find.}
\ELSE
	\RETURN \textbf{false} \COMMENT{Unsuccessful find.}
\ENDIF
\end{algorithmic}
\end{algorithm}

\begin{algorithm}[ph!]
\caption{Insert operation}
\label{algorithm-insert}
\floatname{algorithm}{Insert operation}
\begin{algorithmic}
\REQUIRE $x \in U$
\STATE $i \leftarrow h(x)$
\IF{T[i].Chain.Insert(x)}
	\STATE \COMMENT{Set the number of stored elements.}
	\STATE $Count \leftarrow Count + 1$
	\STATE $T[i].Size \leftarrow T[i].Size + 1$
	\STATE
	\STATE $Rehash \leftarrow \mathbf{false}$
	\STATE \COMMENT{The load factor rule may be violated.}
	\IF{$Count > Size * \alpha_{max}$} 
		\STATE $NewSize \leftarrow 2 * Size$
		\STATE $Rehash \leftarrow \mathbf{true}$
	\ELSE
		\STATE $NewSize \leftarrow Size$	
	\ENDIF
	\STATE
	\STATE \COMMENT{The chain limit rule may be violated.}
	\IF{$T[i].Size > Limit(Size)$}
		\STATE $Rehash \leftarrow \mathbf{true}$
	\ENDIF
	\STATE
	\STATE \COMMENT{Rehash the table if needed, new function is chosen.}
	\IF{$Rehash$}
		\STATE $Rehash(NewSize)$
	\ENDIF
	\STATE
	\RETURN \textbf{true} \COMMENT{Successful insert.}
\ELSE
	\RETURN \textbf{false} \COMMENT{Unsuccessful insert.}
\ENDIF
\end{algorithmic}
\end{algorithm}

\begin{algorithm}[H]
\caption{Delete operation}
\label{algorithm-delete}
\floatname{algorithm}{Delete operation}
\begin{algorithmic}
\REQUIRE $x \in U$
\STATE
$i \leftarrow h(x)$
\IF{T[i].Chain.Remove(x)}
%	\STATE \COMMENT{Set the number of stored elements.}
	\STATE $Count \leftarrow Count - 1$
	\STATE $T[i].Size \leftarrow T[i].Size - 1$
	\STATE
%	\STATE \COMMENT{The load factor rule may be violated.}
	\IF{$Count < Size * \alpha_{min}$}
		\STATE Rehash($Size / 2$)
	\ENDIF
	\STATE
	\RETURN \textbf{true} \COMMENT{Successful delete.}
\ELSE
	\RETURN \textbf{false} \COMMENT{Unsuccessful delete.}
\ENDIF
\end{algorithmic}
\end{algorithm}