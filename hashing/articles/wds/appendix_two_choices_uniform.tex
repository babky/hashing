\section*{Appendix -- proof of Theorem \ref{theorem-universal-hashing-two-choices}}
\setcounter{theorem}{3}

\begin{theorem}
Let $H$ be an almost uniform universal system with a constant $c$ and $d \in \bbbn$, $d > 1$. Assume that each stored element $x$ is placed into a least loaded chain of chains $f_1(x), \dots, f_d(x)$ where hash functions $f_1, \dots, f_d$ are chosen uniformly and independently from $H$. If $n \leq m$, then $$\Prob{\lpsl > \frac{\ln \ln n}{\ln d} + 3c^2 + 4} \in \littleo\left(1\right).$$
\end{theorem}

We state that the original result \cite{DBLP:journals/jacm/Vocking03} of Vöcking holds in case of $c > 1$.
A more detailed proof can be found in the original paper. 
First we state the result when the size of the table is the same as the number of elements, $m = n$.
Here we briefly describe the construction but emphasize all the changes.
It is obvious that in case of universal system with $c = 1$ the statement holds since we may assume full independence up to $n$ elements and the probability bounds used in the proof are automatically correct.
When $c > 1$ we have to compensate higher collision probability by slightly increasing the height of the witness tree.

First, we show that if there is a chain containing $L + 3c^2$ elements, then we are able to construct an activated pruned witness tree of order $L$.
The next step is to estimate from above the probability of existence of an activated pruned witness tree of order $L$.
Hence we obtain the bound on the probability of having a chain consisting of at least $L + 3c^2$ elements.

\subsection{Witness tree}
\begin{definition}[Witness tree]
\emph{Witness tree} of order $L$ is a complete oriented $d$-ary tree with edges oriented in the direction from root to the leaves. Each node of the tree is assigned an element of the stored set $S$.
\end{definition}

We freely exchange the roles of a node and the element assigned to it because this simplification does not cause any mistake.
We refer to the element assigned to a node $v$ as to the variable $v$ without any further notification.
The value $f(x)$ denotes the chain where the element $x \in S$ is placed.

\begin{definition}[Edge event, leaf event, activated witness tree]
We associate two types of events with the tree:
\begin{itemize}
	\item \textbf{Edge event} Consider en edge $(u, v)$. The \emph{edge event associated with the edge $(u, v)$} occurs if $\exists i \in \{1, \dots, d \} \colon f(u) = f_i(v)$.
	\item \textbf{Leaf event} Consider a leaf $v$. The \emph{leaf event associated with the leaf $v$} occurs if the chain $f(v)$ at the insertion time of $v$ contains at least $3c^2$ elements.
\end{itemize}

We say that \emph{witness tree is activated} if all the edge and leaf events occur.
\end{definition}

\subsection{Construction of the activated witness tree}
We just repeat the original construction. 
Let $y$ be the chain having at least $L + 3c^2$ elements. 
We create an activated witness tree of order $L$ witnessing the event of having a chain of length $L + 3c^2$ elements.
We begin by assigning the topmost element $x$ of the chain $y$ to the root. 
Then we look at the chains $f_1(x), \dots, f_d(x)$ and they must contain at least $L + 3c^2 - 1$ elements. 
The topmost elements in these chains are assigned to the children of the root and the process continues recursively. 

Notice that leaves are assigned elements placed into the chains containing at least $3c^2$ elements at their insertion times.
Hence all the leaf events occur.
Realize that all the edge events, too, because on each edge $(u, v)$ the element $v$ is chosen to be from the chains $f_1(u), \dots, f_d(u)$.

\subsection{Probability of the activation of a witness tree}
We estimate from above the probability that there is an activate witness tree of order $L$.
First we make a simplifying assumption that elements assigned to the nodes of the tree are different.
In the following we will get rid of the assumption and obtain a complete proof of the statement.
Let $N$ be the number of all nodes in the tree of order $L$. The number of edges in the tree is thus $N - 1$ and the number of leaves is $d^L$.
\begin{itemize}
	\item The number of all assignments of elements to the nodes is at most $n^N$.
	\item The probability that en edge event occurs is at most ${cd}/{n}$ because we use an almost uniform system with the constant $c$.
	\item 
		Since we have $n$ elements, there are at most ${n}/{3c^2}$ chains consisting of at least $3c^2$ elements.
		The probability that an leaf event occurs is hence at most $({3c})^{-d}$ because each of $d$ function must place the element inside a chain having at least $3c^2$ elements.
\end{itemize}

We use the following estimates in the computation $N \leq 2d^L$ and $2d^2 \leq 3^d$.
Since we assume hashing with a uniform system and that the nodes are assigned different elements the leaf and edge events are independent.
Thus we can conclude that the probability of existence of an activated witness tree of order $L$ is at most:
\[
\begin{split}
n^N \cdot \left(\frac{cd}{n}\right)^{N - 1} \cdot \left({3c})^{-d}\right)^{d^L} 
	& \leq n \cdot \left(\frac{c^2 d^2}{(3c)^{d}}\right)^{d^L} \\
	& \leq n \cdot \left(\frac{c^2 3^d}{2(3c)^{d}}\right)^{d^L} \\
	& \leq n \cdot 2^{-d^L}.
\end{split}
\]

To make the probability estimate a function from $\littleo(1)$ it is sufficient to choose $L = \log_d \log_2 n + \log_d(1 + \alpha)$ for any constant $\alpha > 0$.

\subsection{Pruned witness tree}
Our witness tree of order $L$ witness the event of having a chain consisting of $L + 3c^2$ elements, which is slightly more than in the original result when $c > 1$.
This is the way how we managed to get the same upper bound on the probability of existence of a activated witness tree.
Now we remove the assumption that the nodes of the tree are assigned different elements.

For more comprehensive definitions consult the original proof, however these definitions are satisfactory in our situation.

\begin{definition}[Full witness tree, edge events]
Let $\mathcal{K} \geq 2$ be an integer. Full witness tree of order $L$ is a tree which root has exactly $\mathcal{K}$ children. Each of these children has exactly one child that is the parent of a witness tree of order $L$. Labeling of nodes is done in this way:
\begin{itemize}
	\item Root is assigned one of the $n$ chains. 
	\item Each child of the root is assigned an element $x \in S$ and a number $i \in \{1, \dots, d\}$.
	\item Elements assigned to the children of the root are different.
	\item Grandchildren of the root conform to the assignment defined for witness trees.
\end{itemize}

Let $r$ be the root of the full witness tree.
The edge events for the added edges need to be defined. 
\begin{itemize}
	\item Edge event on the edge $(r, u)$ occurs if $\exists i: f_i(u) = r$.
	\item Assume that $u$ is a child of $r$ and $i$ is the number assigned to the node $u$. Edge event on the edge $(u, v)$ occurs if $\exists j: f_j(v) = f_i(u)$.
\end{itemize}
\end{definition}

We prune full witness tree so that the nodes in the remaining tree are assigned different elements.
The tree which is obtained after pruning is called \emph{pruned witness tree}.
The algorithm for pruning the full witness tree is a BFS procedure which cuts off each edge $(u, v)$ if the element assigned to the node $v$ is has already been encountered before visiting the node $v$.
The edge $(u, v)$ is called \emph{cutoff}.
The pruning procedure performs at most $\mathcal{K}$ cutoffs and then stops.
The visited nodes create the pruned witness tree.
With each witness tree we also remember the cutoff edges.

First observe that if there is a chain $y$ having $L + \mathcal{K} + 3c^2$ elements, then we can create an activated full witness tree.
Root is assigned the chain $y$.
Children of the root are assigned the $\mathcal{K}$ topmost balls of the chain $y$.
Each child $u$ also needs to have a value $i$ -- it is assigned $i \in \{1, \dots, d\}$ such that $f_i{u} = y$. 
The child $v$ of the node $u$ is assigned the topmost element in the chain $f_{(i + 1) \bmod d + 1}(u)$ at the insertion time of $u$.
Also observe that the constructed full witness tree is activated.

Now we estimate the probability of existence of an activated pruned witness tree.

First, assume that there are less than $\mathcal{K}$ cutoff edges in the pruned witness tree.
Then at least one witness tree rooted at a grandchild of the root has not been pruned at all.
The probability of its activation is $o(1)$ and this completes the proof for this case.

Now assume that there are exactly $\mathcal{K}$ cutoff edges, the pruned witness tree consists of $q$ leaves and $N$ nodes and full witness tree from which we started contains $M$ nodes.

We estimate, as in the previous case, the number of assignments and the probability of activation of each event.
\begin{itemize}
	\item There are at most $M^{\mathcal{K}}$ ways how to created a pruned witness tree from the full one.
	\item The number of all assignments to the nodes is bounded from above by $n^N \cdot d^\mathcal{K}$.
	\item The probability that a leaf event occurs is at most $(3c)^{-d}$.
	\item The probability of each type of edge event is ${cd}/{n}$.
	\item The probability of existence of a cutoff edge $(u, v)$ is at most ${Mcd}{n}$.
\end{itemize}

We choose $L = \log_d \log_2 n + \log_d(1 + \alpha)$, as in the case of plain witness trees.
From that we obtain that $M \leq 1 + \mathcal{K} + \mathcal{K} \cdot d^L \leq 2\mathcal{K}(1+\alpha)\log_2 n$.
In the following we also need that $N \leq 2q$ and $d^2 \leq 3^d$. 

Hence the probability of existence of an activated pruned witness tree is at most
\[
\begin{split}
M^{\mathcal{K}} \cdot n^N & \cdot d^\mathcal{K} \cdot \left(\frac{cd}{n}\right)^{N - 1} \cdot (3c)^{-dq} \cdot \left(\frac{Mcd}{n}\right)^{\mathcal{K}} \\
	& \leq n \cdot (cd)^N \cdot (3c)^{-dq} \cdot \left(\frac{M^2 c d^2}{n}\right)^{\mathcal{K}} \\
	& \leq n \cdot \left((cd)^{2} \cdot (3c)^{-d}\right)^{q} \cdot \left(\frac{M^2 c d^2}{n}\right)^{\mathcal{K}} \\
	& \leq n \cdot \left(\frac{M^2 c d^2}{n}\right)^{\mathcal{K}} = n^{-\mathcal{K} + 1 + \littleo(1)} = \littleo(1). \\
\end{split}
\]

The theorem is the statement for $\alpha = 3$ and $\mathcal{K} = 2$.

\subsection{Hashing more than $m$ elements}

So far we have considered hashing of $n$ elements into a table of size $n$. For storing $n$, $n < m$, elements the result clearly holds.
In case of having more than $m$ elements, $\alpha > 1$, we can repeat the same argument as in \cite{DBLP:journals/jacm/Vocking03}.

\qed
