\section{Improvements and conclusion}
\label{section-conclusion}
Our approach works well not only with separate chaining but with linear probing or double hashing as well.
Malalla in his dissertation \cite{Malalla:2004:THS:1124034} showed how to use the two choices paradigm with linear probing in case of plain hashing.
Now we have two possibilities how to apply the result.
If the input data is random enough, we can assume full randomness of the hash function and get the limit function $\bigo(\log \log n)$ as described in \cite{DBLP:conf/soda/MitzenmacherV08} . 
Naturally, without any further assumption we can use an almost uniform system and get the same limit.
Again we do not need to store any further information with the stored keys.
When inserting an element we rehash only if the length of the tried probe sequence exceeds the limit.

%TODO: Ktore mnoziny.
Another solution, which works without any further assumptions on the input data and without uniform systems, is to study behavior of universal hashing with efficient and natural classes. 
An interesting problem with the worst case aware hashing is a way how to derive bounds for two choices paradigm connected with a reasonable universal system.
Our simulations show that using simple systems, e.g. functions of the form $((ax + b) \bmod |U|) \bmod |V|$ or higher degree polynomials for $|U|$ being a prime are not satisfying. 
Hashing $2^{29}$ elements with these universal systems results in longest chains containing about $250$ elements for certain types of sets. 
However, with system of all linear transformations we were not able to find a set having more than $15$ elements for the same size of the stored set.

As showed in \cite{DBLP:conf/alenex/ThorupZ10} linear probing is also possible with universal hashing provided that at least $5$-wise independent family of functions is used.
However, bounds on the length of the longest chain are not known so far.
Because of the cache behavior currently fastest hash tables are based on linear probing.
Therefore obtaining bounds on $\lpsl$ for universal hashing could bring an interesting improvement.

Also in case of separate chaining representing chains by another hash table significantly improves the provided warranty.
Since updates in both hash tables take expected amortized constant time, operations in the combined table are fast.
Better cache behavior and thus greater improvement should be obtained if chains are represented by a linear probing hash table.
