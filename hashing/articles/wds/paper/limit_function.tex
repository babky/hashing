\section{Obtaining the limit function}
\label{section-limit}

So far we have studied the possibility of providing a worst case guarantee for Find if we are given a limit function. 
In this section we show various examples of sublinear limit functions.

\subsection{Linear hash functions}
Representing the universe $U$ and the addresses of the hash table $V$ as vector spaces is natural when keys and addresses are interpreted as zero-padded  bitstrings of a fixed length. Alon, Dietzfelbinger, Bro Miltersen, Petrank and Tardos\cite{DBLP:journals/jacm/AlonDMPT99} found an interesting upper bound on $\Expect \lpsl$ with the system of all linear functions between two vector spaces over the field $\mathbb{Z}_2$.

\begin{theorem}
\label{theorem-linear-hash-functions-dietzefelbinger}
Suppose universal hashing with the system of linear functions from $U$ to $V$. 
If $m \log m$ elements are stored in a table of size $m$, then $\Expect{\lpsl} = \bigo(\log m \log \log m)$. 
\end{theorem}

The major problem of Theorem \ref{theorem-linear-hash-functions-dietzefelbinger} is a high multiplicative constant. However, it can be significantly reduced by a refinement of the original proof. The improved result is stated in Theorem \ref{theorem-linear-refined}.

\begin{theorem}
\label{theorem-linear-refined}
Assume universal hashing with the system of all linear transformations between vector spaces over $\mathbb{Z}_2$. 
Let $p \in (0, 1)$ be the trimming rate and $\alpha > 0$. 
If $n = \alpha m$ elements are stored inside the hash table of size $m$, then $$\Expect{\lpsl} \leq 538 \alpha \log n \log \log n + 44\mbox{ and}$$ $$\Prob{\lpsl \geq a_{\alpha, p} \log m \log \log m + b_{\alpha, p}\log m} < p.$$ where the values $a$ and $b$ depend only on the choice of $\alpha$ and $p$.
\end{theorem}

Let us note that the exact constants for different trimming rates and load factors can be found by a simple computer program.
For example the choice of $\alpha = 1.5$ and $p = 0.5$ yields $a = 57.29$ and $b = 0$.
In addition, constant $a$ can get arbitrarily close to $1$ but such estimates hold only for large values of $n$.

\subsection{Two choices paradigm}
A recent study\cite{DBLP:conf/stoc/AzarBKU94} of balls and bins systems discovered that hashing with at least two independent fully random hash functions brings remarkable results if it is done properly. 
If each stored element is put inside a least loaded chain of $d$ ones, then with a high probability the most loaded bin contains ${\ln \ln n}/\ln d + \bige(1)$ balls.
Further improvements of the states result may be derived using witness tree analysis\cite{DBLP:journals/jacm/Vocking03}. 

\begin{theorem}
\label{theorem-universal-hashing-two-choices}
Let $H$ be an $\omega$-universal system and $d \in \mathbb{N}$, $d \geq 2$. Assume that each stored element $x$ is placed into a least loaded chain of chains $f_1(x), \dots, f_d(x)$ where hash functions $f_1, \dots, f_d$ are chosen uniformly and independently from $H$. If $n \leq m$, then $\Prob{\lpsl > \frac{\ln \ln n}{\ln d} + 5} \in \littleo\left(1\right).$
\end{theorem}

Bounds for $k$-wise independence with constant number of functions and for general $S \subset U$ do not follow from the stated result and are not trivial.
On the other hand the stated result holds in case of uniform or almost uniform systems.
In addition a convenient example of a uniform system appeared in\cite{DBLP:journals/siamcomp/PaghP08}.
The system consists of functions which can be computed in a constant time. 
Let us note that a slight increase of the trimming rate occurs when using the uniform system since its uniformity is probabilistic.

The problem of using $k$-wise independent functions is addressed by Woelfel\cite{InProc-Woe2006a}. 
This work shows how to use simple universal classes with the two choices paradigm.
So we can conclude that there is a way how obtain an impressive doubly logarithmic worst case warranty with the two choices paradigm.
