\section{Obtaining the limit function}
\label{section-limit}

So far we have studied the possibility of providing a worst case guarantee for Find if we are given a limit function. 
In this section we show various limit functions.

\subsection{Linear hash functions}
Alon, Dietzfelbinger, Bro Miltersen, Petrank and Tardos \cite{DBLP:journals/jacm/AlonDMPT99} found an interesting upper bound on $\Expect \lpsl$ with the system of all linear functions between two vector spaces over the field $\mathbb{Z}_2$. 
Representing the universe $U$ and the addresses of the hash table $V$ as vector spaces is natural and causes no problem. 
If $m_0$ is a power of two at the end of $\alpha$-cycle we just double or halve the address space and thus $V$ is kept to be a vector space over $\mathbb{Z}_2$.

\begin{theorem}[\cite{DBLP:journals/jacm/AlonDMPT99}]
\label{theorem-linear-hash-functions-dietzefelbinger}
Suppose universal hashing with the system of linear functions from $U$ to $V$. 
When storing $m \log m$ elements, $\Expect{\lpsl} = \bigo(\log m \log \log m)$. 
\end{theorem}

The major problem of Theorem \ref{theorem-linear-hash-functions-dietzefelbinger} is the high multiplicative constant but it can be significantly reduced by a refinement of the original proof. The improved result is stated in Theorem \ref{theorem-linear-refined}.

\begin{theorem}
\label{theorem-linear-refined}
Assume universal hashing with the system of all linear transformations between vector spaces over $\mathbb{Z}_2$. 
Let $p \in (0, 1)$ be the trimming rate and $\alpha > 0$. 
If $n = \alpha m$ elements are stored inside the hash table of size $m$, then $$\Expect{\lpsl} \leq 538 \alpha \log n \log \log n + 44\mbox{ and}$$ $$\Prob{\lpsl \geq a_{\alpha, p} \log m \log \log m + b_{\alpha, p}\log m} < p.$$ where constants $a$ and $b$ depend on the choice of $\alpha$ and $p$.
\end{theorem}

Let us note that the exact constants for different trimming rates and load factors can be found by a simple computer program.
For example the choice of $\alpha = 1.5$ and $p = 0.5$ yields $a = 57.29$ and $b = 0$.
In addition, constant $a$ can get arbitrarily close to $1$ but such estimates hold only for large $n$.

\subsection{Two choices paradigm}
A study \cite{DBLP:conf/stoc/AzarBKU94} of balls and bins systems discovered that hashing with at least two independent fully random hash functions brings remarkable results. 
If each stored element is put inside a least loaded chain of $d$ ones, then with a high probability the most loaded bin contains ${\ln \ln n}/\ln d + \bige(1)$ balls.
Further improvements were brought by witness tree analysis by Vöcking \cite{DBLP:journals/jacm/Vocking03}. 
Let us note that the result of the following theorem may be improved by asymmetric tie breaking.

\begin{theorem}
\label{theorem-universal-hashing-two-choices}
Let $H$ be an $\omega$-universal system and $d \in \mathbb{N}$, $d \geq 2$. Assume that each stored element $x$ is placed into a least loaded chain of chains $f_1(x), \dots, f_d(x)$ where hash functions $f_1, \dots, f_d$ are chosen uniformly and independently from $H$. If $n \leq m$, then $\Prob{\lpsl > \frac{\ln \ln n}{\ln d} + 5} \in \littleo\left(1\right).$
\end{theorem}

Bounds for $k$-wise independence with constant number of functions and general $S \subset U$ are quite complicated.
On the other hand in case of uniform or almost uniform systems it is easy to derive a similar result.
A convenient example of a uniform system appeared in \cite{DBLP:journals/siamcomp/PaghP08}.
However a slight increase of the trimming rate occurs when using the uniform system since its uniformity is only probabilistic.
Although the functions of the uniform system are computed in a constant time, they are not efficient enough. 
This problem is addressed in a work by Woelfel \cite{InProc-Woe2006a} which shows how to use simple universal classes with the two choices paradigm.
Finally we can conclude that there are several ways how to get an impressive doubly logarithmic worst case warranty with the two choices paradigm.
