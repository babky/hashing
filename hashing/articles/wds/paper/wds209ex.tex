%to be compiled with LATEX 2.09 

%use this settings with 11pt fonts 
\documentstyle[wds11,epsf]{article}

%use this settings with 10pt fonts
%\documentstyle[wds10,epsf]{article}



% Preamble Information

\lefthead{SAFRANKOVA ET AL.}
\righthead{MAGNETOSHEATH RESPONSE}

\setcounter{secnumdepth}{0}

\begin{document}

\title{Example of WDS style article: IMF tangential discontinuity}

\author{J. \v Safr\'ankov\'a, L. P\v rech, and Z. N\v eme\v cek}

\affil{Charles University, Faculty  of  Mathematics  and  Physics,
     Prague, Czech Republic.}

\author{D. G. Sibeck}

\affil{JHU, APL, Laurel, Maryland.}

\author{T. Mukai}

\affil{Institute of Space and Astronautical Science, Sagamihara, Japan.}

\begin{abstract}
This is an example of the WDS style article in LATEX. The
original article has been heavily truncated.
We present a multipoint observational study of the magnetosheath
response to the interplanetary magnetic field tangential
discontinuities which form  hot flow anomaly-like structures.
We identify these structures (HFAs) in the vicinity of the bow shock
as well as deeper in the magnetosheath. Two or
more points of simultaneous observations allow us to describe the
gradual evolution and propagation of these HFAs through the
magnetosheath. From tens of events recorded by INTERBALL-1, we present
two cases. In the first, GEOTAIL identified HFAs in the solar  wind
near the bow shock, and INTERBALL-1 and MAGION-4
observed related events in the magnetosheath. During the second
interval, all three spacecraft observed the HFA
features in the magnetosheath. Our analysis of reported events
suggests the negligible evolution of these structures in the
magnetosheath. A survey of the INTERBALL-1 data has shown that
magnetosheath HFAs are observed predominantly during periods of
fast solar wind.
\end{abstract}

\begin{article}

\section{Introduction}

The interaction of the solar wind with the Earth's magnetosphere
generates a population of backstreaming ions directed from the
bow shock into the solar wind. This high-energy population was
invoked to explain Hot Flow Anomalies (HFAs) [\markcite{{\it
e.g., Schwartz et al.}, 1985; {\it Thomsen et al.}, 1986}]
identified as heated regions of solar wind plasma
flowing nearly perpendicular to the Earth-Sun line.
The main observational features of HFAs include \markcite{{\it
Schwartz} [1995]}: (1) Central regions with hot plasma flowing
significantly slower than that in the ambient solar wind in a
direction highly deflected (nearly $90^o$) from the Sun-Earth line.
The flow velocities are often roughly tangential to the nominal
bow shock shape [\markcite{{\it Schwartz et al.}, 1988}].
(2) HFAs are bounded by regions of enhanced
magnetic field strength, density, and temperature. The outer
edges of these enhancements are fast shocks generated by
pressure enhancements within the core region. The inner
edges of the enhancements are probably  tangential
discontinuities [\markcite{{\it Paschmann et al.}, 1988}].
Published examples indicated that  many HFAs are bounded by only
one enhancement.
(3) HFAs occur in conjunction with significant changes in the IMF
direction. The angle between pre- and post event orientations is
typically $\sim 70^o$.

\markcite{{\it Thomsen et al.} [1986, 1988]} have shown that many
HFAs occur when the geometry of the bow shock is changing and large fraction
of specularly reflected ions are present. These and others
observations led to many theoretical/numerical studies.
One-dimensional simulations with a finite length backstreaming
ion beam suggested
that the interaction of the beam and background plasma can produce hot,
low-density regions from which the solar wind plasma is largely
excluded [\markcite{{\it Onsager et al.}, 1990b}]. However, an
examination of the ion temperature inside HFAs indicated that
complete thermalization of the ion beam and subsequent adiabatic
expansion of the heated plasma leads to final ion temperatures that
are generally below those observed [\markcite{{\it Onsager et al.},
1990a}]. \markcite{{\it Thomas and Brecht} [1988]} presented
a 2-D hybrid simulation for a beam of backstreaming ions
relative to the ambient solar wind. The authors demonstrated that
the thermalized backstreaming ions create a diamagnetic cavity of
depressed magnetic field  strengths and densities.
According to numerical simulations of \markcite{{\it  Thomas et
al.} [1991]} and \markcite{{\it
Lin} [1997]}, kinetic effects at the intersection of the magnetic
discontinuities with the bow shock can create very deflected
flows of heated plasma surrounded by enhanced densities and
magnetic field strengths.

\section{Observations}
To demonstrate HFA properties, we have chosen two representative
examples with favourable positions for all mentioned spacecraft.
The first case study shows an interval when two of three IMF tangential
discontinuities identified in WIND solar wind data resulted in HFAs.
These HFAs are observed in the bow shock region by GEOTAIL,
and simultaneously by
INTERBALL-1 in the magnetosheath. The second case is related to the
propagation of HFAs through the magnetosheath. We have found an
interval when the same HFA is observed by GEOTAIL in the dayside
magnetosheath and by INTERBALL-1 and MAGION-4 in the dusk
magnetosheath flank, about $\sim 20\> R_E$ downstream.

In order to
show that the solar wind velocity can be an important factor for the
HFA creation, we have reanalyzed data published in
[\markcite{{\it Onsager et al.}, 1990a}] and
complemented their survey of ISEE and AMPTE observations by our
analysis of INTERBALL-1 measurements.

\subsection{Case 1}
On August 31, 1996, the INTERBALL-1 satellite
registered a series of HFA-like events in the dawn magnetosheath
at (-0.2; -17.2; -5.1)$_{GSE}\> R_E$. Figure~1 shows observations during two of
these events.  This figure presents 15~s
time resolution VDP ion flux [\markcite{{\it Safrankova et
al.}, 1997}] and 1~s time resolution MIF-M magnetic field
[\markcite{{\it Klimov et al.}, 1997}] observations for the interval
from 0740 to 0820 UT.
%The events at $\sim 0745$ and 0805~UT
The first HFA at 0740-0750 UT as well as the second one at
0805-0815 UT can be identified by rapid decreases of the ion flux
bounded by transient enhancements (top panel in Figure~1). The time
resolution of the measurements does not allow us to see the full
amplitude of the enhancements in this case but our analysis of
similar events has shown that they can exceed the mean magnetosheath flux
by a factor of 3.
The cone angle (second panel in Figure~1) of the ion flux shows
that the flow is highly deflected. Whereas the undisturbed
magnetosheath flow is deflected by $\sim 20^o$ from the
Sun--Earth's line, consistent with INTERBALL-1's location near
the terminator, the ions inside the structures are highly
deflected. They flow nearly perpendicularly to this line during our
second event.

\begin{figure}[htb]
\begin{center}
\begin{tabular}{c}
  \epsfxsize=102mm
  \epsfysize=100mm
%\epsfbox{iii60831.ps}
\epsfbox{iii.ps}
\end{tabular}
\end{center}
\caption{An observation of HFAs in the magnetosheath:
INTERBALL-1 data. From top to bottom: $f_{\rm VDP}$ (ion flux), alpha
(cone angle between ion bulk velocity and the $X_{\rm GSE}$
coordinate), $f_{\rm DOK}$ (flux of electron in the range
28-31~keV), $B$ (magnetic field magnitude), $B_X, B_Y, B_Z$ (components of
the magnetic field), $E_{\rm e0}$ (electron energy spectrogram).}
\end{figure}


We can compute its normal from $({\bf \vec B_1\times \vec
B_2})/({\bf |\vec B_1| \cdot |\vec B_2|})$ where the subscripts 1
and 2 refer to the region upstream and downstream of a
discontinuity, respectively. We tested the normal direction by
minimum variance analysis of the WIND high-resolution (3~s) data.
Both approaches yield the same direction of the discontinuity
normal (${\bf n} = 0.22, 0.67, 0.70$).
The normal defines the discontinuity plane and we can compute
the time lag for observations of the discontinuity at
WIND, IMP-8 and GEOTAIL (GEOTAIL observations will be discussed
in detail later).

Thus, we can conclude that both necessary conditions ($B_n = 0$ and
$u_n = 0$ where $B_n$ and $u_n$ are normal components of the magnetic field
and velocity, respectively) are fulfilled and that all
discontinuities in this time interval are tangential
discontinuities. Such discontinuities can create HFAs, when the
motional electric field ${\bf \vec v_{sw} \times \vec B}$ is oriented
toward the discontinuity
on one or both sides of the current sheet [\markcite{{\it Lin et
al.}, 1997}; \markcite{{\it Thomsen et al.}, 1988}].
We tested this criterion and found that it is obeyed for the first
and third discontinuities but not for the second one.
%has oppositely oriented electric field and
Thus only the first and third discontinuities can create HFAs
when interacting with the bow shock.
This is consistent with magnetosheath observations presented in
Figure~1 where only two HFA-like events are seen during the time
interval under question.

\subsection{Case 2}
A similar situation occurred on January 31, 1997. The IMF
tangential discontinuity observed by WIND at $\sim 1953$~UT resulted
in the creation of an HFA observed in the dusk
magnetosheath by three spacecraft: GEOTAIL, INTERBALL-1, and
MAGION-4. For the sake of simplicity, Figure~2 shows only the ion
flux and magnetic field magnitudes observed by these
spacecraft but we have carefully examined all parameters to ensure
that the magnetosheath event has  HFA characteristics. It should be
noted that the characteristics of the event differ a little at
different spacecraft positions.
\begin{figure}[htb]
\begin{center}
\begin{tabular}{c}
  \epsfxsize=102mm
  \epsfysize=100mm
  \epsfbox{gim70131.ps}
\end{tabular}
\end{center}
\caption{Propagation and evolution of HFA through the
magnetosheath. $B$ and $f$ stand for the magnetic field magnitude
and ion flux, respectively, and subscripts are abbreviations of
the spacecraft names.}
\end{figure}
The locations of the spacecraft projected onto the ecliptic plane are
schematically shown in the left part of Figure~3. GEOTAIL was
located in the magnetosheath near the magnetopause and registered
the leading enhancement of HFA at 2034~UT. Due to the decreased
density in the core region of HFA, GEOTAIL crossed magnetopause
and entered the plasma sheet. It exited into HFA at 2041~UT to
observe the trailing enhancement of the ion flux.

\begin{figure}[htb]
\begin{center}
\begin{tabular}{c}
  \epsfxsize=82mm
%  \epsfysize=30mm
  \epsfbox{dvepara2.eps}
\end{tabular}
\end{center}
\caption{ A projection of the satellite positions into ecliptic
plane during discussed events. The right part of the figure
belongs to August 31, 1996 event, the left part refers to January
31, 1996 event. The intersection of discontinuities with the
ecliptic plane is shown by double lines.}
\end{figure}


\subsection{A survey of INTERBALL HFA observations}
Both cases analyzed above occurred during a period of relatively
high solar wind speed (505$\>km/s$ for the first case and
495$\>km/s$ for the
second case). In order to determine whether or not this is a random
occurrence, we surveyed  3 months of the INTERBALL-1
observations (January - March, 1997).
For our survey, we have taken into account only those events
which had all basic HFA characteristics, i.e., a region of tenuous
highly deflected flow and low magnetic field bounded by short
regions of significantly enhanced ion density and magnetic field
magnitude. We exclude events connected with bow shock or
magnetopause crossings because the plasma and magnetic field
behaviour are more complicated in such cases and a complex
analysis of all parameters is needed to decide whether the particular
event is connected with the HFA.

\section{Discussion}
We have presented two multipoint case studies of the
magnetosheath response to the arrival of  IMF tangential
discontinuity at the bow shock.
In both of them, an HFA is created at the bow shock and then
observed in the magnetosheath. The INTERBALL-1 observation of HFA
in the magnetosheath is supported by simultaneous GEOTAIL
observation at the bow shock region in our first case. The analysis
of this event shows that the characteristics of HFA in the
solar wind and in the magnetosheath are basically the same. HFAs
can be distinguished as regions of the hot tenuous plasma bounded
by density enhancements in both regions.

In the second case, a single HFA was observed in the dusk magnetosheath
near the magnetopause by three spacecraft.
We suggested that the differing profiles of the
events at different points of the magnetosheath could be attributed
to the shape of a HFA cavity rather than to a temporal evolution of
its dimensions on a time scale of several minutes.
We suggest that the durations of HFAs depend on the locations
where IMF discontinuities intersect the bow shock.
Thus, the different durations of the events observed by
INTERBALL-1 and GEOTAIL  in our first case  may result from
GEOTAIL's separation from the subsolar region where the plasma
observed by INTERBALL-1 enter the magnetosheath.
Simulations indicate that HFAs need about 10
ion gyroperiods to develop into notable disturbances [\markcite{{\it
Thomas et al.}, 1991]}.

\section{Conclusion}
HFAs are an important magnetosheath phenomenon.
They can be encountered throughout the
magnetosheath from the bow shock to the magnetopause. Our
observations are limited by the INTERBALL-1 satellite's orbit to the
range $|X_{GSE}| < 10\> R_E$ but the negligible evolution of reported
events suggests that HFAs should be observed further tailward, if the
duration of the event ($\sim$ several minutes) and the temporal
resolution of observations allows their identification.

We have pointed out a {\it double structure} of magnetosheath
HFAs. A complete description of this feature requires a
multidevice analysis of high-time resolution data.

The magnetosheath HFAs exhibit a clear tendency to occur
predominantly during periods of enhanced solar wind speed.
This needs more careful examination because high speed streams
have different characteristics than slow speed streams,
% one and probably only a good
perhaps computer simulation can determine which factors are
most important in HFA formation.

\acknowledgments %\footnotesize
{The authors thank the WIND, IMP-8, GEOTAIL, and INTERBALL working
teams for the
magnetic field and plasma data. The  present  work  was  supported by
the Czech  Grant  Agency  under  Contracts  205/99/1712, and
205/00/1686 and by the Charles University Grant Agency under
Contract 181.}

\begin{references}

%\footnotesize
\reference
Klimov, S. {\it et al.}, ASPI experiment: Measurements of fields and
waves onboard the Interball-1 spacecraft, {\it Ann. Geophys.}, 15,
514-527, 1997.

\reference
Kudela, K., M. Slivka, J. Rojko, and V. N. Lutsenko, The
apparatus DOK-2 (project INTERBALL): Output data structure and
modes of operation, preprint of Inst. Exp. Phys. UEF-01-95,
Kosice, 20, 1995.

\reference
Lin Y., Generation of anomalous flows near the bow shock by
its interaction with interplanetary discontinuities, {\it J. Geophys.
Res.}, 102, 24265-24281, 1997.

\reference
Lutsenko, V. N., J. Rojko, K. Kudela, T. V. Gretchko, J. Balaz,
J. Matisin, E. T. Sarris, K. Kalaitzides, and N. Paschalidis,
Energetic particle experiment DOK-2 (Interball project), in:
{it INTERBALL Mission and Payload,} ed. by Yu. Galperin, IKI-CNES,
249, 1995.

%\reference
%Nemecek, Z., J. Safrankova, L. Prech, S. Kokubun, T. Mukai,
%and D. G. Sibeck, Transient flux events in the magnetosheath,
%{\it Geophys. Res. Lett.}, 25, 1273-1276, 1998.

\reference
Onsager, T. G., M. F. Thomsen, J. G. Gosling, and S. J.
Bame, Observational test of a hot flow anomaly formation
mechanism, {\it J. Geohys. Res.}, 95, 11.967-11.974, 1990a.

\reference
Onsager, T. G., M. F. Thomsen, and D. Winske,
Hot flow anomaly formation by magnetic deflection,
{\it Geophys. Res. Lett.}, 17, 1621-1624, 1990b.

\reference
Schwartz, S. J., G. E. Paschmann, N. Sckopke, T. M. Bauer, M. W.
Dunlop, A. N. Fazakerley, and M. F. Thomsen,  Hot flow anomalies
revisited, {\it Int. J. Geomag. and Aeronomy}, 1, 1999, in print.

\end{references}







\end{article}


\end{document}
