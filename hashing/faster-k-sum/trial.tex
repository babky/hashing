\documentclass{article}

\usepackage{amsmath}
\usepackage{amsfonts}
\usepackage{amssymb}
\usepackage{amsthm}
\usepackage{geometry}
\geometry{
	b5paper,
	margin=0.5cm,
	top=1.5cm,
	bottom=1.5cm
}

\newcommand{\llinnm}[2]{\operatorname{L}_{\operatorname{lin}}({#1}, {#2})}
\newcommand{\llinn}[1]{\llinnm{#1}{#1}}
\newcommand{\hlinr}[1]{\operatorname{H}_{\operatorname{lin}}^{#1}}
\newcommand{\hlin}{\operatorname{H}_{\operatorname{lin}}}
\newcommand{\hfact}[2]{\operatorname{H}_{\operatorname{factor}}({#1}, {#2})}
\newcommand{\rot}[2]{\operatorname{H}_{\operatorname{rot}}^{{#1}, {#2}}}

\newcommand{\bin}[3]{\operatorname{\mathbf{bin}}({#1}, {#2}, {#3})}
\newcommand{\lbin}[2]{\operatorname{\mathbf{lbin}}({#1}, {#2})}
\newcommand{\vbin}[2]{\operatorname{\mathbf{bin}}({#1}, {#2})}
\newcommand{\vlbin}[1]{\operatorname{\mathbf{lbin}}({#1})}


\newcommand{\probs}[2]{\operatorname{\mathbf{Pr}}_{{#1}}\left[{#2}\right]}
\newcommand{\prob}[1]{\probs{}{#1}}
\newcommand{\expects}[2]{\operatorname{\mathbf{E}}_{{#1}}\left[{#2}\right]}
\newcommand{\expect}[1]{\expects{}{#1}}
\newcommand{\inu}{\in_U}

\newtheorem{lemma}{Lemma}
\newtheorem{theorem}{Theorem}
\newtheorem{claim}{Claim}
\newtheorem{corollary}{Corollary}

\title{Faster $k$-SUM algorithm improvement on randomness requirements}
\author{Martin Babka}

\begin{document}

\maketitle

By $[m]$ we understand the set $\{1, \dots, m\}$.
Let $z \in [m]^n$. 
We define $z(x) := z_x$ and $p(z) = \sum_{i = 1}^{n} |z^{-1}(i)|^2$.

Let $h$ be a 4-wise independent function. 
Let $K = (k_1, \dots, k_s)$ be chosen randomly and 4-wise independently from $[n]^s$.
Let $f = h \circ z$.
Let $W_L(K) := k_1, f(k_1), f^2(k_1), \dots, f^{l_1}(k_1), k_2, f(k_2), \dots, f^{l_2}(k_2), k_3, \dots$ be a sequence generated by the following algorithm:
\begin{itemize}
    \item We start each run by the next value from $K$.
    \item We generate values using function $f$ applied to the previous value.
    \item We stop whenever we encounter a collision with respect to $z$, i.e. for the current value $c$ there is a previous value $p$ such that $z(p) = z(c)$.
\end{itemize}

\begin{lemma}
Fix $z$ with $p(z) \leq p$ and a positive integral value of $s$ such that $s \leq n^2/p$.
Put $L := \frac{1}{2}n\sqrt{\frac{s}{p}}$.
Let $\beta$ be a string generated by the above algorithm. 
Then $\prob{\beta\mbox{ contains }i^{*}\mbox{ and }j^{*}} \geq \Omega\left(\left(\frac{L}{n}\right)^2\right)$.
\end{lemma}
\begin{proof}
From the following three claims and the next observation we immediately obtain the statement
\begin{align*}
    \prob{\beta\mbox{ contains }i^{*}\mbox{ and }j^{*}} 
        & \geq \prob{\beta\mbox{ contains }i^{*}\mbox{ and }j^{*} \wedge |\beta| = L} \\
        & = \prob{\beta\mbox{ contains }i^{*}\mbox{ and }j^{*} \mid |\beta| = L}\prob{|\beta| = L}.
\end{align*}

\begin{claim}
\label{claim-independence}
For each $\{i_1, i_2, i_3, i_4\} \in \binom{[|\beta|]}{4}$ we have that $\beta_{i_1}, \beta_{i_2}, \beta_{i_3}, \beta_{i_4}$ are independent. Moreover for arbitrary $i \in |\beta|$ and $y \in [n]$ it holds that $\prob{\beta_{i} = y} = n^{-1}$.
\end{claim}
\begin{proof}
Without loss of generality we assume that $i_1 < i_2 < i_3 < i_4$. If $i_1$ is the start of a run, then $\beta_{i_1}$ is chosen uniformly randomly from $[n]$ and independenly of all the values $\beta_{i_1}, \beta_{i_2}, \beta_{i_3}, \beta_{i_4}$.
Notice that the same reasoning may be used when any of $i_2, i_3, i_4$ is a start of the run.

If none of the four values is the start of a run, then $|z(\{\beta_{i_1 - 1}, \beta_{i_2 - 1}, \beta_{i_3 - 1}, \beta_{i_4 - 1}\})| = 4$. 
Since $h$ is 4-wise independent both parts of the claim hold.
We can use the same reasoning when there are less then four values not being the starts.
\end{proof}

\begin{claim}
\[ \prob{\beta\mbox{ contains }i^{*}\mbox{ and }j^{*} \mid |\beta| = L} \geq \Omega\left(\left(\frac{L}{n}\right)^2\right). \]
\end{claim}
\begin{proof}
For $\{i, j\} \in \binom{[L]}{2}$ we define event $E_{i, j}$ as $\beta_i = i^* \wedge \beta_j = j^*$.
In the proof we assume that $|\beta| = L$ and omit the conditioning from the probability.
Notice that $n \leq p \leq n^2$ and $1 \leq s \leq n^2/p$.
Observe that we may bound $L = 1/2 \cdot n\sqrt{s/p} \in [1/2, n/2]$.
\begin{align*}
    \prob{\beta\mbox{ contains }i^{*}\mbox{ and }j^{*}} 
        & \geq \sum_{\{i, j\} \in \binom{[L]}{2}} \prob{E_{i, j}} - \sum_{\substack{\{i, j\} \in \binom{[L]}{2} \\ \{k, l\} \in \binom{[L]}{2} \\ \{i, j\} \neq \{k, l\}}} \prob{E_{i, j} \wedge E_{k, l}} \\
        & \geq \frac{L(L-1)}{2} \cdot n^{-2} - L^3 \cdot n^{-3} - L^4 \cdot n^{-4} \\
        & \geq \Omega\left((L/n)^2\right).
\end{align*}
\end{proof}

\begin{claim}
\[ \prob{|\beta| = L} \geq \frac{3}{4}. \]
\end{claim}
\begin{proof}
Let $\beta \in [n]^L$ be a random string generated by the Collide algorithm.
It holds that whenever $|\beta| < L$, then $\beta$ has $s$ repetitions with respect to $z$.
We prove that $\prob{\beta\mbox{ has }s\mbox{ repetitions w.r.t. }z} \leq 1/4$.

We say that $i_1$, $i_2$ collide w.r.t. $z$ if $z(\beta_{i_1}) = z(\beta_{i_2})$.
Let $\operatorname{rep}$ be the number of repetitions in $\beta$ w.r.t. $z$ and $\operatorname{coll}$ be the number of collisions in $\beta$ w.r.t. $z$. 
It holds that $\operatorname{rep} \leq \operatorname{coll}$.
Let $\{i_1, i_2\} \in \binom{[L]}{2}$ from Claim~\ref{claim-independence} it follows that $\prob{i_1, i_2\mbox{ collide w.r.t. }z} \leq \frac{2p}{n^2}$.
Now we bound $\expect{\operatorname{rep}} \leq \expect{\operatorname{coll}} \leq \frac{L^2}{2} \cdot \frac{2p}{n^2} = \frac{s}{4}.$
We use Markov inequality to bound $\prob{\operatorname{rep} = s} \leq \prob{\operatorname{rep} \geq s} \leq \frac{1}{4}.$ 
\end{proof}

\end{proof}
\end{document}