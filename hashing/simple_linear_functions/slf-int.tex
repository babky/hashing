\documentclass{article}

\usepackage[utf8]{inputenc}

\usepackage{color}
\usepackage{amsmath}
\usepackage{amsfonts}
\usepackage{amssymb}
\usepackage{amsthm}
\usepackage{mathabx}
\usepackage{geometry}
\usepackage{graphicx}
\geometry{
	b5paper,
	margin=1.5cm,
	top=1.75cm,
	bottom=1.75cm
}

\newcommand{\llinnm}[2]{\operatorname{L}_{\operatorname{lin}}({#1}, {#2})}
\newcommand{\llinn}[1]{\llinnm{#1}{#1}}
\newcommand{\hlinr}[1]{\operatorname{H}_{\operatorname{lin}}^{#1}}
\newcommand{\hlin}{\operatorname{H}_{\operatorname{lin}}}
\newcommand{\leap}[3]{\operatorname{\mathbf{leap}}({#1}, {#2}, {#3})}
\newcommand{\hfact}[2]{\operatorname{H}_{\operatorname{factor}}({#1}, {#2})}
\newcommand{\rot}[2]{\operatorname{H}_{\operatorname{rot}}^{{#1}, {#2}}}

\newcommand{\bin}[3]{\operatorname{\mathbf{bin}}({#1}, {#2}, {#3})}
\newcommand{\lbin}[2]{\operatorname{\mathbf{lbin}}({#1}, {#2})}
\newcommand{\vbin}[2]{\operatorname{\mathbf{bin}}({#1}, {#2})}
\newcommand{\vlbin}[1]{\operatorname{\mathbf{lbin}}({#1})}


\newcommand{\probs}[2]{\operatorname{\mathbf{Pr}}_{{#1}}\left[{#2}\right]}
\newcommand{\prob}[1]{\probs{}{#1}}
\newcommand{\expects}[2]{\operatorname{\mathbf{E}}_{{#1}}\left[{#2}\right]}
\newcommand{\expect}[1]{\expects{}{#1}}
\newcommand{\inu}{\in_U}

\newtheorem{lemma}{Lemma}
\newtheorem{theorem}{Theorem}
\newtheorem{claim}{Claim}
\newtheorem{corollary}{Corollary}


\title{Expected number of uniformly distributed balls in a most loaded bin using placement with simple linear functions}
\author{Martin Babka\thanks{Research supported by the Czech Science Foundation grant GA14-10799S.}}

\begin{document}

\maketitle

\begin{abstract}
We estimate the size of a most loaded bin when the balls, which are chosen from a transformed interval, are placed into the bins using a random linear function where the operations are taken in a finite field.
We show that in such a setting the expected load of the most loaded bins is constant.

This is an interesting fact because using fully random hash functions with the same class of input sets leads to an expectation of $\Theta\left(\frac{\log m}{\log \log m}\right)$ balls in most loaded bins where $m$ is the number of balls and bins.

Although the family of the functions is quite common both in theory and practice, the size of largest bins was not known even in this simple case.
\end{abstract}

\section{Introduction}

We estimate the size of a largest bin in a special case of the balls and bins model. The balls and bins model simply means placement of each ball into a bin in some random fashion. The process of the placement is of a various study -- choice of balls, randomness of placement, independence of the placement and especially the study how these properties affect the bin sizes.

The simplest model of the placement of the balls into bins is formed by using complete randomness and independence of the elements.
It is a well-known fact that if $m$ balls are thrown completely at random to $m$ bins, then the expected size of the largest bin is $\Theta\left(\frac{\log m}{\log \log m}\right)$.

There are various similar statements dealing with the estimates of bin sizes for various placement processes.
For example Carter and Wegman \cite{cw} showed that the expected size of a bin is constant when they assumed the placement by the functions which we will refer to as simple linear functions. These functions are two-wise independent and thus they trivially achieve $O(\sqrt{m})$ expected size of a largest bin. We tighten this result to $O(1)$ when the input set is a linearly transformed interval.

It is also possible to use functions of higher degree of independence and thus obtain better bounds. There are also lower bounds shown by Siegel \cite{siegel} for various trade-offs between the speed of such functions, size needed to represent them on one side and independence they achieve on the other side.

The need to improve the size of the largest bins leads to the two-choice paradigm. For the placement we use two functions and the ball is placed into the smaller bin out of the two chosen bins. In this model the size of the largest bin is $O(\log \log m)$ where $m$ is the number of balls and bins. This was first shown shown by Azar et al \cite{azar} and later improved by V\"{o}cking \cite{vocking}.

Nowadays more complicated family of functions are studied in \cite{wieder}. The functions no longer rely on high degree of independence but are designed so that they achieve small largest bins with high probability.

\section{Notation and definitions}
\label{sec:notation}
We refer to the set $\{0, \dots, k - 1\}$ as to $[k]$. 
In the whole text we assume that $p$ is a fixed prime. 
The set of chosen balls is denoted by $S \subset [p]$.
The number of bins is the same as the number of balls and is denoted by $m$, i.e. $|S| = m$.

For each pair $(a, b) \in [p]^2$ we define the function $h'_{a, b}$ as $h'_{a, b}(x) = (ax + b) \bmod p$ and the function $h_{a, b}$ as $h_{a, b}(x) = h'_{a, b}(x) \bmod m$.

The multiset of simple linear functions mapping $[p]$ to the range $[m]$ is denoted by $\hlin$ and is defined as $\hlin = \{h_{a, b} \mid a, b \in [p] \}$.
For a function $h \in \hlin$ we define the size of $i$-th bin as $\bin{h}{S}{i} = |S \cap h^{-1}(i)|$ and the maximal size of the bin as $\lbin{h}{S} = \max_{i \in [m]} \bin{h}{S}{i}$.

In the following text we fix the probability space to be formed by a uniform choice of $h \in \hlin$.
The symbols $\vbin{S}{i}$ and $\vlbin{S}$ then refer to the random variables formed by the mentioned random uniform choice.

For an element $x$ we put the value $\leap{x}{a}{b} = \left\lfloor\frac{ax + b}{p}\right\rfloor$; that is how many \emph{leaps}, overflows over $p$, are caused by applying the function $h_{a, b}$ on the element $x$ in the field $\mathbb{Z}_p$. Notice that $(ax + b) \bmod  p = ax + b - \leap{x}{a}{b}p$.

Our model relies on the above random choice of the hash function $h_{a, b} \in \hlin$ and the fact that $S$ is a linear transformation of $[m]$ in $\mathbb{Z}_p$. In Theorem~\ref{thm:interval-constant} we show that the expected size of largest bin is $O(1)$.

\section{Collision probability for three elements}

We first study the probability of collision of three arbitrary elements.
By collision of the elements we understand the event when all of the elements are mapped to the same element in $[m]$ by the randomly chosen linear function.

We fix three different elements $x, y, z \in [p]$ and we count the number of pairs $(a, b) \in [p]^2$ such that $|h_{a, b}(\{x, y, z\})| = 1$.

We start by simplifying to the case when $x = 0, y = 1$ and the third element $z = d$ for a suitable $d \in [p]$ such that $d > 1$ depends on the choice of $x, y, z$.

\begin{lemma}[Transformation lemma]
\label{lemma:transformation}
Let $x, y, z \in [p]$ be arbitrary different elements. Moreover, assume that $i_x, i_y, i_z \in [m]$. Then there exist an element $d \in [p]$ such that
\[
\prob{h(x) = i_x, h(y) = i_y,  h(z) = i_z} = \prob{h(0) = i_x, h(1) = i_y, h(d) = i_z}.
\]
\end{lemma}
\begin{proof}
The idea of the proof is simple. We show that there is a one-to-one map between simple linear functions mapping $x, y, z$ to $i_x, i_y, i_z$ and simple linear functions transforming $0, 1, d$ to the same elements.

In the first part of the proof we observe that combining simple linear functions with a linear function in $\mathbb{Z}_p$ does not change the probability space.
There is a single linear function transforming $0, 1$ to $x, y$ in $\mathcal{Z}_p$ which we refer to as $h'_{\alpha,\beta}$.
Finally we choose $d$ so that $h'_{\alpha, \beta}(d) = z$ and the proof is finished.

We show that the elements $x, y, z$ can be transformed to the elements $0, 1, d$ so that the probability of the mappings from the statement of the lemma remains the same.

Choose $\alpha, \beta \in [p]$ so that $\alpha \neq 0$.
Observe that the mapping $(\gamma, \delta) \mapsto (\alpha \gamma, \beta \gamma + \delta)$ is a one-to-one map on $[p]^2$.
If there is another pair $(\epsilon, \phi)$ such that $(\alpha \epsilon, \beta \epsilon + \phi) = (\alpha \gamma, \beta \gamma + \delta)$, then $\gamma = \epsilon$ and $\delta = \phi$. Thus the mapping is injective.
Also for arbitrary $(r, s) \in [p]^2$ the element $(\alpha^{-1}r, s - \beta\alpha^{-1}r)$ is mapped to $(\alpha \alpha^{-1}r, \beta \alpha^{-1} r + s - \beta\alpha^{-1}r) = (r, s)$.

The compound function $h'_{a, b} \circ h'_{\alpha, \beta}$ is exactly equal to the function $h'_{\alpha a, \beta a + b}$; this also follows from the fact that the set of all linear functions in $\mathbb{Z}_p$ forms a group with the operation of compounding functions. 

Let $H' = \{h'_{a, b} \mid (a, b) \in [p]^2\}$.
From the previous we can conclude that the combination of a function $h' \in H'$ with a fixed function $h'_{\alpha, \beta}$ is a one-to-one map in the space of functions $H'$.
Also observe that the composition of a function $h_{a, b} \in \hlin$ with $h'_{\alpha, \beta}$ can not change the probability (count of the functions) of mapping arbitrary three elements to a their prescribed images.

There is also a single function $(\alpha, \beta) \in [p] ^ 2$, i.e. a single function $h'_{\alpha, \beta}$, transforming the elements $0$ and $1$ to $x$ and $y$ in the field $[p]$ without taking modulo $m$. It is the function $\beta = x$ and $\alpha = y - x$.
To prove the lemma we choose $d \in [p]$ such that $h'_{\alpha, \beta}(d) = z$, i.e. $d = \alpha ^ {-1}(z - \beta)$.
\end{proof}

Lemma \ref{lemma:transformation} shows that the probability properties, e.g. collision, mapping to the prescribed elements, for the elements $x, y, z$ are the same as for the elements $\{0, 1, d\}$ where $d$ comes from the previous lemma.

Next we estimate the collision probability for the elements $0, 1, d$.

\begin{lemma}[Probability of collision of three elements]
\label{lemma:probability-3-elements}
Let $d \in [p]$ be arbitrary element.
\[
\prob{|h(\{0, 1, d\})| = 1 } = \frac{\lceil \lceil\frac{p}{d}\rceil / m \rceil\left\lceil\frac{d}{m}\right\rceil}{p}.
\]
\end{lemma}
\begin{proof}
We count the number of functions $h \in \hlin$ such that $h(\{0, 1, d\}) = \{y\}$ for some $y \in [m]$.
For each $x \in [p]$ it holds that $\leap{x}{a}{b} \in [x]$ and $h(x) = (ax + b - \leap{x}{a}{b}p) \bmod m$.

Whenever the elements $0, 1$ and $d$ are mapped to the same element $y$ it must hold that $h(0) = h(1)$ and $h(0) = h(d)$. Hence
\begin{align*}
	b \bmod p \bmod m & = (a + b) \bmod p \bmod m \\
	b \bmod p \bmod m & = (da + b) \bmod p \bmod m. \\
\end{align*}
From which we obtain the following sequence of equivalent equations
\begin{align*}
	m & \mid a + b - \leap{1}{a}{b}p - b \\
	m & \mid da + b - \leap{d}{a}{b}p - b, \\
\end{align*}
\begin{align*}
	m & \mid a - \leap{1}{a}{b}p \\
	m & \mid da - \leap{d}{a}{b}p, \\
\end{align*}
\begin{align*}
	m & \mid da - d\leap{1}{a}{b}p \\
	m & \mid (d\leap{1}{a}{b} - \leap{d}{a}{b})p. \\
\end{align*}
Since $p$ is a prime from the last equation we can conclude that $m \mid d\leap{1}{a}{b} - \leap{d}{a}{b}$.
We estimate the collision probabilities from the first equation in the second step and from the last observation:
\begin{align}
	m & \mid a - \leap{1}{a}{b}p \label{3-prob-1-statement} \\
	m & \mid d\leap{1}{a}{b} - \leap{d}{a}{b}. \label{3-prob-2-statement}
\end{align}

The statement (\ref{3-prob-2-statement}) roughly means that out of $d$ possible values for $\leap{d}{a}{b}$ only the $1 / m$ fraction may generate the collision of the three elements. Notice that for a fixed $l \in [d]$ it holds that $\{a \in [p] \mid \leap{d}{a}{b} = l\}$ equals is a subinterval of $[p]$.
From (\ref{3-prob-1-statement}) we can observe that only the $1 / m$ fraction from the possible values of $a$ lying in the appropriate intervals allowed by valid values of $\leap{d}{a}{b}$ are causing collisions.

For the rest of the proof fix the value of $b$. 
First, we show that the values of $a$ such that $\leap{d}{a}{b} = l \in [d]$ form disjoint intervals in $[p]$ each of size at most $\lceil p/d \rceil$.
Then we count the number of values $a$ in an interval causing collisions -- using (\ref{3-prob-1-statement}).
And finally we count the number of the valid intervals, i.e. the intervals satisfying equation (\ref{3-prob-2-statement}).

Let $\leap{d}{a}{b} = l$, then it holds that $l \leq \frac{da + b}{p} < l + 1$. Immediately we get that $a \in \left[\frac{pl - b}{d}, \frac{p(l + 1) - b}{d}\right) \cap \mathbb{Z}$. The total number of values of $a$, i.e. integers, in each valid interval is at most $\lceil p / d \rceil$. The ceiling must be applied. For example assume an interval of length of 1.5 starting at point 0.8 -- it contains two integer points 1 and 2. This happens whenever $\frac{pl - b}{d}$ is an integer.

Now fix the value $l \in [d]$ such that $\leap{d}{a}{b} = l$.
In order to estimate the number of values of $a$ causing the collisions we split into two cases according to the value of $\leap{1}{a}{b}$.

\subparagraph{The first case, $\leap{1}{a}{b} = 0$.} 
From the two previous statements we conclude that
\begin{align*}
	m & \mid a \\
	m & \mid l. \\
\end{align*}

\subparagraph{The second case, $\leap{1}{a}{b} = 1$.}
As in the first case it must hold that
\begin{align*}
	m & \mid a - p \\
	m & \mid d - l. \\
\end{align*}

In both cases, there are at most $\lceil d/m \rceil$ values of $l$ satisfying the  condition (\ref{3-prob-2-statement}).
Also for each satisfying value of $l$ there are at most $\lceil \lceil\frac{p}{d}\rceil / m \rceil$ values of $a$ causing the collision.

\end{proof}

The probability of collision of the three elements can be bounded as
\[
\frac{\left(1 + \frac{1 + \frac{p}{d}}{m}\right)\left(1 + \frac{d}{m}\right)}{p} =
\begin{cases}
	\frac{O(1)}{m^2} + \frac{1}{dm} + O\left(p^{-1}\right) & \mbox{if } d \leq p/m \\
	\frac{O(1)}{m^2} + \frac{d}{pm} + O\left(p^{-1}\right) & \mbox{otherwise, i.e. } \frac{p}{m} \leq d < p.
\end{cases}
\]

The worst possible case is for $d = 2$ and the probability is roughly $1/2m$. 
When $d \geq p/m$, the formula is a great overestimate as shown in Figure~1.
% When $d \geq p/m$, the formula is a great overestimate as shown in Figure~\ref{fig:probability-3}.

\begin{figure}[h]
	\label{fig:probability-3}
	\centering
	\includegraphics[width=8cm]{coll-3-21787-512-scaled}
	\caption{The function of probability of collision of the elements $0, 1, d$ with respect to $d$. Notice that the probability is decreasing in the part when $d \leq p / m$ and is almost symmetric. In this figure $m = 512$ and $p = 21787$.}
\end{figure}

\begin{corollary}
\label{co:d-elements}
Let $d \leq p / m$. Then $\prob{|h([d])| = 1} \leq \frac{1}{(d - 1) m} + O(1)/m^2 + O(p^{-1})$.
\end{corollary}
\begin{proof}
When all the elements from $[d]$ collide, then the elements $\{0, 1, d - 1\}$ must collide as well. The probability of the collision of $\{0, 1, d - 1\}$ is hence a valid upper bound on the probability of the collision of the whole interval. The statement is then a direct application of Lemma~\ref{lemma:probability-3-elements}.
\end{proof}

For completeness we just show a simple fact that our probability estimate is tight when we have a stronger assumption, namely we assume $p > 3m^2$.

\begin{lemma}
\label{lm:0-d-prob-lower-bound}
If $d \leq m$ and $p > 3m^2$, then $\prob{|h(\{0, 1, \ldots, d - 1\})| = 1} = \Omega\left(\frac{1}{dm}\right).$
\end{lemma}
\begin{proof}
For a fixed $b$, if $a < (p - b)/d$ and $m \mid a$, then the elements $\{0, 1, \ldots, d - 1\}$ must collide.
For each $b$ there are at least $\lfloor (p - b)/dm \rfloor$ such values of $a$.

We conclude that the number of pairs $(a, b)$ making the elements collide is at least
\[
\sum_{b \in [p]} \left(\frac{p - b}{dm} - 1\right) = \frac{(p + 1)p}{2dm} - p \geq \frac{p ^ 2 - 2pdm}{2dm} \geq \frac{p^2}{6dm}.
\]

Thus the resulting probability is at least $\frac{1}{6dm}$.
\end{proof}

\section{The structure of elements in a single bin}

First realize that for each $i \in [m]$ it holds that $h^{-1}(i) = \{a^{-1} (i + j m - b) \mid j \in [p / m]\}$ where the operations are taken in $\mathbb{Z}_p$.
Thus for $\alpha = a^{-1}m$ and $\beta = i-a^{-1}b$ we see that $h^{-1}(i)$ is the image of a linear transformation of the interval $[p / m]$. Precisely $h^{-1}(i) = \alpha [p/m] + \beta$ where the operations are taken in $\mathbb{Z}_p$.

Notice that for a fixed $0 < m < p$, there is a one-to-one map among the pairs $(a, b)$ and $(\alpha, \beta)$.

From the almost two-wise independence of $\hlin$ we can derive the following property. {\color{red} Is it really from the two-wise independence? How does this property relates to the two-wise independence?}

\begin{lemma}
Assume that $g, f \in [p] \setminus \{0\}$ and $I = \left[\frac{p}{m}\right]$. Then \[\left\lfloor \frac{f^{-1}g}{m} \right\rfloor \left\lfloor \frac{1}{f^{-1}g} \left\lfloor\frac{p}{m}\right\rfloor \right\rfloor \leq |gI \cap fI| \leq \left\lfloor \frac{f^{-1}g}{m} \right\rfloor \left\lceil \frac{p}{f^{-1}gm} \right\rceil.\]
\end{lemma}
\begin{proof}
We begin the proof by simplifying the statement using the fact that $|gI \cap fI| = |f^{-1}gI \cap I|$. Hence without loss of generality we may assume that $f = 1$.

From now on we are estimating $|gI \cap I|$.
We count the number of $x \in I$ such that $gx \in I$.
First, trivially, for each $0 \leq x < \frac{p}{gm}$ we know that $gx \in I$.
Observe that no $x \in I \setminus \left\{ 0, \dots, \left\lfloor {p}/{(gm)} \right\rfloor \right\}$ such that $\leap{x}{g}{0} = 1$ satisfies the condition $gx \in I$.

Realize that the similar observations hold for each possible leap and there are $g$ leaps. More precisely -- each leap adds at least $\left\lfloor \frac{\lfloor p/m \rfloor}{m} \right\rfloor$ and at most $\left\lceil \frac{p}{gm} \right\rceil$ elements to the final intersection.

Since there are $\lfloor\frac{gp}{mp}\rfloor$\footnote{This may be less by 1 since the $\lfloor p/m \rfloor$ element is the last and the expression thus can be by one smaller.} leaps we get that $\left(\lfloor\frac{g}{m}\rfloor - 1\right) \left\lfloor \frac{p}{gm} \right\rfloor \leq |gI \cap I| \leq \lfloor \frac{g}{m}\rfloor\left\lceil \frac{p}{gm} \right\rceil$.
\end{proof}

In order to continue we have to perform a more detailed analysis of the leaps and possible overflows so that we get tighter bounds.
The previous lemma yields good results only for $1 < g \leq p/m$ which is just a $1/m$ fraction of the possible values of $g$.

\begin{lemma}
Assume that $g, f \in [p] \setminus \{0\}$ and $I = \left[\frac{p}{m}\right]$. Then $|gI \cap fI| \leq O(p/m^2)$.
\end{lemma}
\begin{proof}
First, assume that $f = 1$. And let us denote $\lfloor p/m \rfloor$ as $\phi$.
Now realize that there are two types of leaps long and short. The long leaps add $\lceil \phi / g \rceil$ to the result and the short ones add only $\lfloor \phi / g \rfloor$.

So we know that $|gI \cap I| = \#\operatorname{leaps} \cdot \lfloor \phi / g \rfloor + \#\operatorname{long-leaps} \leq (g/m) \left\lceil p/gm  \right\rceil.$ The number of leaps from the previous proof is at most $g/m$. If $g \leq p/m$, then we get that $|I \cap gI| \leq O(p/m^2)$ and we are finished. From now on assume that $g > p/m$. Also notice that the contribution of short leaps to the intersection is zero.

The number of long leaps can be estimated similarly as $|g'[g/m] \cap I|$ for $g' = (-p) \bmod g$ where the operations are taken in $\mathbb{Z}_g$.
Similarly $|g'[g/m] \cap I| \leq \lfloor g'/m \rfloor \lfloor p/mg' \rfloor + \#\operatorname{long-leaps}$. If $g' \leq p/m$, then we are finished otherwise we have to go down one level in the similar way.
\end{proof}

The following is a straightforward application of the previous lemma.
\begin{corollary}
Let $a, a' \in [p] \setminus {0}$ and let $i \in [m]$. Then it holds that $|h_{a, 0}^{-1}(i) \cap h_{a', 0}^{-1}(i)| = 1/m^2$.
\end{corollary}

\section{The expected size of most loaded bins}

First we study the role of the parameter $b$ in the hash function $h_{a, b}$.
% TODO: Mention that for universality it is not so much required but makes a good constant and independence.

The following lemma states that the effect of $b$ on $\vlbin{S}$ is not asymptotic since it more or less only shifts the largest bin.
\begin{lemma}
\label{lm:b-zero}
Assume that  $a, b \in [p]$ and $S \subseteq [p]$. Then \[ \frac{1}{2} \lbin{h_{a, b}}{S} \leq \lbin{h_{a, 0}}{S} \leq 2\lbin{h_{a, b}}{S} . \]
\end{lemma}
\begin{proof}
Let $L \subseteq S$ be elements of bin $y$, i.e. $h_{a, b}(L) = y$.
For each $x \in L$ we have that 
\begin{align*}
h_{a, 0}(x) 
	& = ax \bmod p \bmod m \\ 
	& = (ax + b - b) \bmod p \bmod m \\ 
	& = \begin{cases}
((ax + b) \bmod p - b \bmod p)\bmod m & \mbox{ if } (ax + b) \bmod p \geq b \\
(p + (ax + b) \bmod p - b \bmod p) \bmod m & \mbox{ otherwise.} \\
\end{cases}
\end{align*}

Notice that the two possible new bins are either $(y - b) \bmod m$ or $(p + y - b) \bmod m$.
The lemma now follows from the following two observation.
First each original bin is either shifted and keeps its size or is split into two possibly uneven shifted bins -- hence $\frac{1}{2} \lbin{h_{a, b}}{S} \leq \lbin{h_{a, 0}}{S}$.
And notice that each new bin can only contain elements from at most two different original bins and thus $\lbin{h_{a, 0}}{S} \leq 2\lbin{h_{a, b}}{S}$.
\end{proof}

For completeness let us mention that the change of the sign of $a$ has almost no effect on $\vlbin{S}$.
\begin{lemma}
\label{lemma:sign-a}
Assume that  $a \in [p]$ and $S \subseteq [p]$ such that $0 \not\in S$. Then \[ \lbin{h_{a, 0}}{S} = \lbin{h_{p - a, 0}}{S} . \]
\end{lemma}
\begin{proof}
Similarly as in the proof of the previous lemma. Let $L \subseteq S$ be elements of bin $y$, i.e. $h_{a, 0}(L) = y$.
Let $x \in L$, then $h_{p - a, 0}(x) = (p - a)x \bmod p \bmod m = (p - ((ax) \bmod p))\bmod m = (p - y) \bmod m$.
When $x \neq 0$, then $(p - a)x \bmod p = p - ((ax) \bmod p)$.
The bin $y$ is thus moved to the bin $(p - y) \bmod m$ and the lemma holds.
\end{proof}

Obviously allowing zero makes only a negligible change.
\begin{corollary}
Let $S \subseteq [p]$, then
\[
\lbin{h_{a, 0}}{S} -1 \leq \lbin{h_{p - a, 0}}{S} \leq \lbin{h_{a, 0}}{S} + 1.
\]
\end{corollary}

\section{The equation $ax \bmod p \bmod m = 0$}

Finding the structure of the solutions of the equation $ax \bmod p \bmod m = 0$ for $x \in [m]$ enables us to find a structure in the chains in the following manner. 
After showing proving that the solutions form an arithmetic progression starting by $0$, we can simply show that the chains which are formed when $S=[m]$ are arithmetic progressions. Such a structure of the chains allows us to proceed further and is vital in order to show our main result.

Let $G$ be a group and $a \in G$. By $\langle a \rangle_{G}$ we denote the subgroup of $G$ generated by $a$, i.e. $\langle a \rangle_{G} = \{0, a, 2a, \dots \}$ where the multiplication operation is taken in $G$.

The next lemma says that whenever the sequence $0, a \bmod p \bmod m, 2a \bmod p \bmod m, \cdots$ forms the whole subgroup $\langle a \rangle_{[m]}$ in the first leap, i.e. $a (|\langle a \rangle_{[m]}| < p$, then the solutions of the equation in $[m]$ form an arithmetic progression. In the lemma we use a slightly stronger assumption since the size of a leap can be by one shorter then the size of the first leap. It is easy to observe that each leap we just shift the $\langle a \rangle_{[m]}$ by $-p$ compared to the previous leap. In the leaps we obtain disjoint affine subgroups which generate at most $m$ elements. Hence the zero solutions thus may only occur in the first leap.

\begin{lemma}
Let $a \in [p]$ such that $a (|\langle a \rangle_{[m]}| + 1) < p$. Let $d$ be the minimal positive integer such that $ad \bmod p \bmod m = 0$.
Then all the solutions of the equation $ax \bmod p \bmod m = 0$ for $x \in [m]$ form the arithmetic progression starting in $0$ with the difference $d$.
\end{lemma}
\begin{proof}
Since $p$ is a prime, $-p$ (and also $p$) are generators of $\mathbb{Z}_{m}$.
Let $k \in [m]$ be the minimal integer such that $k(-p) \bmod m \in \langle a \rangle_{[m]}$.
Notice that $j(-p) \bmod m \in \langle a \rangle_{[m]}$ $(j - k)(-p) \bmod m \in \langle a \rangle_{[m]}$ and hence $|\langle a \rangle_{[m]}| k = m$.

Observe that after each leap there is a shift by $-p$ ($\bmod m$), i.e. after $l$ leaps we are generating the elements from the affine subgroup $l(-p) + \langle a \rangle_{[m]}$.

Assume that there is a value $e$ such that $(ae - \leap{e}{a}{0}p) \bmod m = 0$.
From the first assumption of the lemma we have that $e \geq |\langle a \rangle_{[m]}| \leap{e}{a}{0}$.
Also it must be true that $\leap{e}{a}{0} \geq k$.
Now we conclude that $e \geq |\langle a \rangle_{[m]}| k = m$ which proves the lemma.
\end{proof}

The following lemma shows an interesting property of the set of solutions which we only mention and do not use further.
\begin{lemma}
Let $d_{\operatorname{min}}$ be the minimal solution of the equation $ax \bmod p \bmod m = 0$.
The set of all solutions of the above equations from $[p / ({a d_{\operatorname{min}}} \bmod p)]$ forms a finite arithmetic progression starting at $0$ with the difference $d_{\operatorname{min}}$.
\end{lemma}
\begin{proof}
If $k d_{\operatorname{min}} \in [p / (a d_{\operatorname{min}}  bmod p)]$, then $a k d_{\operatorname{min}} \bmod p = akd_{\operatorname{min}}$ and hence $k d_{\operatorname{min}}$ is a solution of the equation.

Let $d \in [p / (a d_{\operatorname{min}} \bmod p)]$ be a solution such that $d > d_{\operatorname{min}}$. Let $k d_{\operatorname{min}}$ be the maximal solution smaller than $d$. Then $(a(d - kd_{\operatorname{min}})) \bmod p = a(d - kd_{\operatorname{min}})$ and $d - kd_{\operatorname{min}}$ is thus a solution. 
If $d$ is not from the above arithmetic progression, then $d - 
kd_{\operatorname{min}} < d_{\operatorname{min}}$ which is in the direct 
contradiction with the fact that $d_{\operatorname{min}}$ is the minimal 
solution.
\end{proof}

\begin{lemma}
\label{lemma-equation}
Let $a \in [p]$. The solutions of the equation $ax \bmod p \bmod m = 0$ for $x \in [m]$ form an arithmetic progression starting in $0$.
\end{lemma}
\begin{proof}
First realize that the statement can be rewritten as $[m] \cap a^{-1}m [p/m]$ is an arithmetic progression.

In the following claim the notation $x \cdot_G y$ means that the operation is taken in $\mathbb{Z}_{G}$, i.e. $x \cdot_G y = xy \bmod G$. 
\begin{claim}
\label{claim:arithmetic-progression}
Let $G$ be a positive integer and $g \in [G]$. We prove that $[m] \cap g \cdot_{G} [G/m]$ is an arithmetic progression.
\end{claim}
\begin{proof}
We proceed by induction on $G$.

Assume that $G < m$, then $[G/m] = \{0\}$ and the statement trivially holds for each $g \in G$.

Let $G \geq m$ be arbitrary and assume that $g \leq m$, then we obtain the wanted statement with the arithmetic progression with the difference exactly $g$.

Assume that $g > m$.
Observe that each leap over $G$ may intersect with $[m]$ only in the first element of the leap because the difference between the elements is $g > m$.
Notice that there are at most $\lfloor g/m \rfloor$ leaps and hence the elements of the leaps are of the form $(-G)k \bmod g$ where $k \in [g/m]$.
Thus our statement holds only and only if $((-G) \bmod g) \cdot_g [g/m] \cap [m]$ forms an arithmetic progression. Since $((-G) \bmod g) + g < g + G$ from the induction we get the wanted statement.
\end{proof}
The lemma follows from a simple application of the previous claim by using $G := p$ and $g := a^{-1}m \bmod p$.
\end{proof}

Lemma~\ref{lemma-equation} implies that all the chains are arithmetic progressions as precisely stated in its following corollary.
\begin{corollary}
For all $y \in [m]$ it holds that $|h_{a, 0}^{-1}(y) \cap [m]|$ forms an arithmetic progression.
\end{corollary}
\begin{proof}
Let $\{x_1 < \dots < x_l\} = h_{a, 0}^{-1}(y)$.
For each $i \in \{1, \dots, l\}$ it holds that \[0 = a x_i \bmod p \bmod m - a x_1 \bmod p \bmod m = a (x_i \bmod p - a x_1 \bmod p) \bmod m.\]

From the previous observation it follows that for each $i = 1, \dots, l$ the elements $x_i - x_1$ are solutions of the equation $ax \bmod p \bmod m$.
By Lemma~\ref{lemma-equation} they form an arithmetic progression.
\end{proof}

\section{Main theorem}

For the choice of $S = [m]$ we show that the expected size of a most loaded bin is within $O(1)$. 
This can be compactly formulated as follows.
\begin{theorem}
\label{thm:interval-constant}
Assume that $p \geq m^2$, then
\[
\expect{\vlbin{[m]}} = O(1).
\]
\end{theorem}
\begin{proof}
By Lemma~\ref{lm:b-zero} we may assume that the chosen function has $b = 0$ without asymptotically increasing the expected size of the largest bin. In the proof of the claims we thus assume that the chosen linear function is exactly the function $h_{a, 0}$. Moreover we assume that $a \neq 0$. Notice that this assumption adds exactly $m/p$ to the computed expected value which is $O(1)$.

Observe that each bin is formed by a single arithmetic progression. Notice that since $p$ is a prime it holds that $(-p) \bmod m$ is co-prime with $m$.
The reason can be stated as follows.
Let $x_1 < x_2$ be two elements in a single bin, then for $d = x_2 - x_1$ it holds that either $m \divides ad \bmod p$ or $m \divides p - (ad \bmod p)$.

All the solutions of the equation $ax \bmod p \bmod m = 0$ where $x \in [m]$ form a finite arithmetic progression.
For the proof of the previous statement notice that since $p$ is a prime it holds that $(-p) \bmod m$ is co-prime with $m$ and thus to return back to the bin $y$ we need at least $m$ leaps. 

In addition for each difference $d$ and each length $l$, $l \geq 3$, there is a canonical value $x \in [m]$ such that if there is a bin of size at least $l$, then there is another bin formed by an arithmetic progression of length at least $l$ with the same difference $d$ having $x$ as the minimal element. If $ad \bmod p < p/2$, we choose $x = \operatorname{argmin}_{x \in [m - ld]} ax \bmod p$. Otherwise we put $x = \operatorname{argmax}_{x \in [m - ld]} ax \bmod p$.

After establishing the previous facts we simply compute the expected value of $\vlbin{[m]}$ using the following idea. Now we allow $b$ to have arbitrary value.

Assume that $\vlbin{[m]} > l \geq 3$, then there is an arithmetic progression chosen from $[m]$ of size at least $l/2$ collapsing into a single bin, here we use Lemma~\ref{lm:b-zero}. Since for a fixed difference and length we have its canonical position, then there are at most $m/l$ possible arithmetic progressions from which we choose from. By Corollary~\ref{co:d-elements} we upper bound the probability of the collapse of the arithmetic progression as 
\[
\frac{m}{l} \left(\frac{1}{(l/2 - 1)m} + 1/m^2 + O(p^{-1})\right) \leq O(l^{-2}).
\]


Hence for $l \geq 3$ we have
\[
\prob{\vlbin{[m]} \geq l} = O(l^{-2}).
\]

Then we simply conclude that
\[
\expect{\vlbin{[m]}} \leq O(1) + \sum_{l = 1}^m O\left(\frac{1}{l^2}\right) = O(1).
\]

\end{proof}

We can conclude the main result, i.e. each set transformable to $[m]$ in $\mathbb{Z}_p$ has constant sized largest bins.
\begin{corollary}
Let $S \subseteq [p]$, $a, b \in [p]$. 
If $\forall x \in [m] \colon (ax + b) \bmod p \in S$, then $\expect{\vlbin{S}} = O(1)$.
\end{corollary}
\begin{proof}
Direct corollary of Theorem~\ref{thm:interval-constant} since by Lemma~\ref{lemma:transformation} (extended to all the elements of $S$) the probabilistic properties of $S$ do not change under the transformation $x \mapsto (ax + b) \bmod p$.
\end{proof}


\begin{thebibliography}{32}
\bibitem{cw}
J.L. Carter, and M.N. Wegman. 
\newblock Universal Classes of Hash Functions.
\newblock Journal of Computer and System Sciences, 18. pages 143--154, 1979.

\bibitem{siegel}
A. Siegel. 
\newblock On universal classes of extremely random constant-time hash functions.
\newblock SIAM Journal on Computing, 33(3). pages 505--543 (electronic), 2004

\bibitem{wieder}
L.E. Celis, O. Reingold, G. Segen, and U. Wieder
\newblock Balls and Bins: Smaller Hash Families and Faster Evaluation
\newblock Foundations of Computer Science (FOCS), 2011 IEEE 52nd Annual Symposium, pages 599 -- 608, 2011

\bibitem{azar}
Y. Azar, A. Broder, A. Karlin, and E. Upfal
\newblock Balanced allocations.
\newblock SIAM Journal on Computing, 29(1). pages 180--200, 1999.

\bibitem{vocking}
B. V\"{o}cking
\newblock How asymmetry helps load balancing.
\newblock In Proceedings of the Fortieth Annual Symposium on Foundations of Computer Science. pages 131--140, 1999.
\end{thebibliography}

\end{document}


