\documentclass{article}

\usepackage{amsmath}
\usepackage{amsfonts}
\usepackage{amssymb}
\usepackage{amsthm}
\usepackage{geometry}
\geometry{
	b5paper,
	margin=0.5cm,
	top=1.5cm,
	bottom=1.5cm
}

\newcommand{\llinnm}[2]{\operatorname{L}_{\operatorname{lin}}({#1}, {#2})}
\newcommand{\llinn}[1]{\llinnm{#1}{#1}}
\newcommand{\hlinr}[1]{\operatorname{H}_{\operatorname{lin}}^{#1}}
\newcommand{\hlin}{\operatorname{H}_{\operatorname{lin}}}
\newcommand{\hfact}[2]{\operatorname{H}_{\operatorname{factor}}({#1}, {#2})}
\newcommand{\rot}[2]{\operatorname{H}_{\operatorname{rot}}^{{#1}, {#2}}}

\newcommand{\bin}[3]{\operatorname{\mathbf{bin}}({#1}, {#2}, {#3})}
\newcommand{\lbin}[2]{\operatorname{\mathbf{lbin}}({#1}, {#2})}
\newcommand{\vbin}[2]{\operatorname{\mathbf{bin}}({#1}, {#2})}
\newcommand{\vlbin}[1]{\operatorname{\mathbf{lbin}}({#1})}


\newcommand{\probs}[2]{\operatorname{\mathbf{Pr}}_{{#1}}\left[{#2}\right]}
\newcommand{\prob}[1]{\probs{}{#1}}
\newcommand{\expects}[2]{\operatorname{\mathbf{E}}_{{#1}}\left[{#2}\right]}
\newcommand{\expect}[1]{\expects{}{#1}}
\newcommand{\inu}{\in_U}

\newtheorem{lemma}{Lemma}
\newtheorem{theorem}{Theorem}
\newtheorem{corollary}{Corollary}

\title{Expected load of a fullest bin with simple linear functions}
\author{Martin Babka}

\begin{document}

\maketitle

This is just a compilation of ideas which can possibly lead to some bounds on the quantity $\llinn{m}$ which is defined in Section~\ref{sec:notation}.

\section{Notation and definitions}
\label{sec:notation}
We refer to the set $\{0, \dots, k - 1\}$ as to $[k]$.
Let $A, B$ be sets of integers and $q$ be an integer, then $A + B = \{a + b \mid a \in A,\ b \in B\}$ and $qA = \{qa \mid a \in A \}$ and $A + q = A + \{q\}$.

We assume that $p$ is a fixed prime such that $n = m \ll p$.
Then the universe from which the balls are chosen is the set $[p]$.
All the operations are then taken in the field $\mathbb{Z}_p$.

In the following the number of bins is a positive integer $m$.
We further assume that $m$ is a power of two.
The number of balls is $n \leq m$, in this text we further assume that $n = m$.
The set of chosen balls is denoted by $S \subset [p]$ such that $|S| = m$.

For each pair $(a, b) \in [p]^2$ and an integer $r$, $0 < r < p$, we define the function $h_{a, b}^r$ as $h_{a, b}^r(x) = (ax + b) \bmod p \bmod r$.
In the following text we use simplify the notation by putting $h_{a, b} = h_{a, b}^m$.

The multiset of simple linear functions mapping $[p]$ to the range [r] is denoted by $\hlinr{r}$ and is defined as $\hlinr{r} = \{h_{a, b}^r \mid a, b \in [p] \}$.
As in the case of a single function we simplify the notation so that $\hlin = \hlinr{r}$.
We also define the rotation functions $\rot{f}{m} = \{h_f \mid h_f(x) = (x + b_f) \bmod m \text{ where } b_f \in [m] \}$.

By $\hfact{h}{f}$ we understand the factorization of $h \in \hlin$ through the factor space $[f]$.
Let $h_{a, b} \in \hlin, h_{a_u, b_u} \in \hlinr{f}, h_f \in \rot{f}{m}$.
We define that $h_{a, b} = h_u \odot h_f$ iff $a_u = a$.
The factorization of $h$ through the factor space $[f]$ is the set $\hfact{h}{f} = \{ (h_u, h_f) \mid h_u \in \hlinr{f}, h_f \in \rot{f}{m}, h = h_f \odot h_u \}.$

For a function $h \in \hlin$ we define $\bin{h}{S}{i} = |S \cap h^{-1}(i)|$ and $\lbin{h}{S} = \max_{i \in [m]} \bin{h}{S}{i}$.

In the following when speaking about probabilities we implicitly use the probability space of the uniform choice of $h \in \hlin$, referred to as $h \inu \hlin$.
The notation $\vbin{S}{i}$ and $\vlbin{S}$ then refer to the random variables formed over the mentioned uniform choice.

\section{Properties of the set $[m]$}

The set $[m]$ is a set for which it is natural to study its behaviour.
We are going to conclude two results about the set which are in contrast.

First the set $[m]$ (or its arbitrary linear transformation) has the maximal probability of being collided into a single element.
On the other hand this is the set with the exepected longest chain lenght $O(1)$.

What is remarkable is the fact that the chains of the colliding elements somehow correspond to a transformation of a subset of the set $[\frac{p}{m}]$.
We think that the subset with the highest probability of being collided is just a transformation of the set $[l]$ for some $l \leq m$.
So the fact that the set $[m]$ has small longest chains but the chains which are the longest should be some transformations of this set is just a remarkable fact.

\subsection{Collision probability for three elements}

First of all, assume that we have three different elements $x, y, z$ and we count the number of pairs $(a, b)$ such that $|h_{a, b}(\{x, y, z\})| = 1$.
As stated in the following lemma we may assume that $x = 0, y = 1$ and the third element $z = d$.

\begin{lemma}[Transformation lemma]
Let $x, y, z \in [p]$ be arbitrary different elements. Then there exist an element $d \in [p]$ such that
\[
\prob{h(x) = h(y) = h(z)} = \prob{h(0) = h(1) = h(d)}.
\]
\end{lemma}
\begin{proof}
There is a single pair $(a, b) \in [p] ^ 2$ determining the linear function transforming the elements $0$ and $1$ to $x$ and $y$ (in the field $[p]$ without taking modulo $m$); $b = x$ and $a = y - x$.
There is also a single element $d$ such that $ad + b = z$, i.e. $d = a ^ {-1}(z - b)$.
Fix the chosen pair $(a, b)$.

Observe that the mapping $(a', b') \mapsto (aa', b' + ab')$ is a one-to-one map on $[p]^2$.
A function determined by the pair $(a', b') \in [p]^2$ behaves on $\{x, y, z\}$ in the same way as the function $(aa', b' + ab')$ on $\{0, 1, d\}$.
So each function mapping $\{0, 1, d\}$ to a single element corresponds to a function mapping $\{x, y, z\}$ to a single element and vice versa.
\end{proof}

\begin{corollary}
The probability properties, e.g. collision, mapping to the prescribed elements, for the elements $x, y, z$ are the same as for the elements $\{0, 1, d\}$ where $d$ comes from the previous lemma with the appropriate transformations.
\end{corollary}

\begin{lemma}[Probability of collision of three elements]
Let $d \in [p]$ be arbitrary element. Then
\[
\prob{\mbox{elements }\{0, 1, d\}\mbox{ collide}} = 
\frac{1 + \max\left(1, \frac{p}{dm}\right)\left(1 + \frac{d}{m}\right)}{p}.
\]
\end{lemma}

We count the number of functions $h \in \hlin$ such that $h(\{0, 1, d\}) = \{y\}$ for some $y \in [m]$.
For an element $x$ we define the value $l(x, a, b) = \left\lfloor\frac{ax + b}{p}\right\rfloor$; that is how many ``leaps'' are created by applying the function $h_{a, b}$ on the element $x$ in the field $\mathbb{Z}_p$.
For each $x \in [p]$ it holds that $l(x, a, b) \in [x]$ and $h(x) = (ax + b - l(x, a, b)p) \bmod m$.

Whenever the elements $0, 1$ and $d$ are mapped to the same element $y$ it must hold that 
\begin{align*}
	b \bmod p \bmod m & = (a + b) \bmod p \bmod m \\
	b \bmod p \bmod m & = (da + b) \bmod p \bmod m. \\
\end{align*}
From that we conclude that
\begin{align*}
	m & \mid a + b - l(1, a, b)p - b \\
	m & \mid da + b - l(d, a, b)p - b. \\
\end{align*}
Hence
\begin{align*}
	m & \mid a - l(1, a, b)p \\
	m & \mid da - l(d, a, b)p. \\
\end{align*}
Therefore it holds that 
\begin{align*}
	m & \mid da - dl(1, a, b)p \\
	m & \mid (dl(1, a, b) - l(d, a, b))p. \\
\end{align*}
Since $p$ is a prime we have that \[m \mid dl(1, a, b) - l(d, a, b).\]
We estimate the collision probabilities from the two statements following from the previous formulas:
\begin{align}
	m & \mid a - l(1, a, b)p \label{3-prob-1-statement} \\
	m & \mid (dl(1, a, b) - l(d, a, b))p. \label{3-prob-2-statement}
\end{align}

From (\ref{3-prob-1-statement}) we can observe that only the $1 / m$ fraction from the possible values of $a$ lying in the appropriate intervals allowed by valid values of $l(d, a, b)$ are causing collisions.

The statement (\ref{3-prob-2-statement}) roughly means that out of $d$ possible values for $l(d, a, b)$ only the $1 / m$ fraction may generate the collision of the three elements.

For the rest of the proof fix the value of $b$. 
First, we show that the values of $a$ such that $l(d, a, b) = l \in [d]$ form disjoint intervals in $[p]$ each of size at most $\lceil p/d \rceil$.
Then we count the number of values $a$ in an interval causing collisions -- using (\ref{3-prob-1-statement}).
And finally we count the number of the valid intervals.

Let $l(d, a, b) = l$, then it holds that $l \leq \frac{da + b}{p} < l + 1$. Immediately we get that $a \in \left[\frac{pl - b}{d}, \frac{p(l + 1) - b}{d}\right) \cap \mathcal{Z}$. The ceiling must be applied assume a interval length of 1.5 starting at point 1 -- it contains the two points 1 and 2. So the total number of values of $a$ in each valid interval is at most $\lceil p / d \rceil$. This happens whenever $\frac{pl - b}{d}$ is an integer.

Now fix the value $l \in [d]$ such that $l(d, a, b) = l$.
In order to estimate the number of values of $a$ causing the collisions we split into two cases according to the value of $l(1, a, b)$.

\subparagraph{The first case, $l(1, a, b) = 0$.} 
From the two previous statements we conclude that
\begin{align*}
	m & \mid a \\
	m & \mid l. \\
\end{align*}

\subparagraph{The second case, $l(1, a, b) = 1$.}
As in the first case it must hold that
\begin{align*}
	m & \mid a - p \\
	m & \mid d - l. \\
\end{align*}

In both cases, there are at most $\lceil d/m \rceil$ values of $l$ satisfying the second condition.
Also for each satisfying value of $l$ there are at most $\lceil \lceil\frac{p}{d}\rceil / m \rceil$ values of $a$ causing the collision.

In both cases and for each $b$ it holds that the probability of collision of the three elements is bounded by
\[
\frac{\left(1 + \frac{1 + \frac{p}{d}}{m}\right)\left(1 + \frac{d}{m}\right)}{p} =
\begin{cases}
	\frac{1}{m^2} + \frac{1}{dm} + O\left(p^{-1}\right) & \mbox{if } p/dm \geq 1 \\
	\frac{1}{m^2} + \frac{d}{pm} + O\left(p^{-1}\right) & \mbox{otherwise, i.e. } \frac{p}{m} \leq d < p.
\end{cases}
\]

The worst possible case is when $d = 2$ and the probability is roughly $1/2m$.

In the case when $d \geq p/m$ another proof is possible showing the bound $\frac{1}{2m(p - d)}$. The proof is based on a completely different idea.

\begin{lemma}
\label{lm:three-elements}
For arbitrary three elements the probability of their collision is at most $1/2m + 2/m^2$.
\end{lemma}

\begin{corollary}
\label{co:d-elements}
For arbitrary $d$ elements the probability of their collision is at most $\frac{2}{(d - 1) m} + 2/m^2$.
\end{corollary}
\begin{proof}
These $d$ elements are normalized to a set of size $d$ and we assume that $e$ is its maximal element.
Assume that $e < p / m$. Then if all $d$ elements of this set collide, then the elements $\{0, 1, e\}$ must collide as well.
Since $e \geq d - 1$ the following $\frac{1}{e} \leq \frac{1}{d - 1}$ holds and from this we can conclude that the corollary holds as well.
The worst case is if the elements are distributed uniformly at the beginning and at the end of $\mathbb{Z}_p$.
Then we can get just the mentioned probability using the same technique as in the first case.
\end{proof}

The above statement is tight for the set $\{0, 1, \ldots, d - 1\}$.
For this set we have $\frac{p}{(d - 1)m}$ functions making it collide.

\subsection{Collision probability for the set $\{0, 1, \ldots, d\}$}

The goal is to prove for $d \ll m \ll p$ that this probability is more or less the same as the probability of colliding the elements $\{0, 1, d\}$.

Now assume that the three elements $\{0, 1, d\}$ collide.
Then it holds
\begin{align*}
m & \mid a - l(1, a, b)p \\
m & \mid 2a - l(2, a, b)p \\
\vdots & \\
m & \mid da - l(d, a, b)p.
\end{align*}

From which it follows that
\begin{align*}
m & \mid 2l(1, a, b) - l(2, a, b) \\
m & \mid 3l(1, a, b) - l(3, a, b) \\
\vdots & \\
m & \mid dl(1, a, b) - l(d, a, b).
\end{align*}

But I do not know what to do next :(.
The experiments indicate that something like this should hold (at least up to a constant factor, of course).

\begin{lemma}
\label{lm:0-d-prob-lower-bound}
For $d < m << p$ the following statement is true;
\[
\prob{|h(\{0, 1, \ldots, d\})| = 1} = \Omega\left(\frac{1}{dm}\right).
\]
\end{lemma}
\begin{proof}
For a fixed $b$, if $a < (p - b)/d$ and $m \mid a$, then the elements $\{0, 1, \ldots, d\}$ collide.
For each $b$ there are at least $\lfloor (p - b)/dm \rfloor$ such values of $a$.

We conclude that the number of pairs $(a, b)$ making the elements collide is at least
\[
\sum_{b \in [p]} \frac{p - b}{dm} - 1 = \frac{2p ^ 2 - (p - 1)p}{2dm} - p \geq \frac{p ^ 2}{2dm} - p.
\]

Thus the resulting probability is at least $\frac{1}{3dm}$.
\end{proof}

\subsection{Bounds on $\vlbin{[m]}$}

The effect of $b$ on $\vlbin{S}$ is not asymptotic as stated in the following lemma.
\begin{lemma}
\label{lm:b-zero}
Assume that  $a, b \in [p]$ and $S \subseteq [p]$. Then \[ \frac{1}{2} \lbin{h_{a, b}}{S} \leq \lbin{h_{a, 0}}{S} \leq 2\lbin{h_{a, b}}{S} . \]
\end{lemma}
\begin{proof}
% TODO.
To be done.
\end{proof}

\begin{lemma}
Assume that  $a \in [p]$ and $S \subseteq [p]$ such that $0 \not\in S$. Then \[ \lbin{h_{a, 0}}{S} = \lbin{h_{p - a, 0}}{S} . \]
\end{lemma}
\begin{proof}
% TODO.
To be done.
\end{proof}

\begin{corollary}
If $0 \in S$, then we have the inequality
\[
\lbin{h_{p - a, 0}}{S} -1 \leq  \lbin{h_{a, 0}}{S} \leq \lbin{h_{p - a, 0}}{S} + 1.
\]
\end{corollary}

\begin{theorem}
\[
\expect{\vlbin{[m]}} = O(1).
\]
\end{theorem}
\begin{proof}
Notice that by Lemma~\ref{lm:b-zero} we may assume that the chosen function has $b = 0$ and $a \neq 0$.

We now observe that for each chain $L$ it holds that the set is an arithmetic progression chosen from the set $[m]$.
And the set $aL$ also forms an arithmetic progression.

Depending on the choice of $a$ we either have
\begin{itemize}
  \item $1$ belongs to a chain which is by one element shorter than the longest chain and not containing zero or
  \item $1$ belongs to a chain which is the longest one and containing zero.
\end{itemize}

The reason for the previous observation is simple. If we hash the set $[m]$ then we are subsequently adding $a$ to the previous result.
Since $p$ is a prime it is true that $(-p) \bmod m$ is coprime with $m$.
Fix the smallest element $x$ in a chain $L$ and let $y = x + d$ be the second smallest one.
Then each element $x + id$ must be in the same chain unless $x + id > p$.
The first time the sequence $x + id$ goes back to the same chain has $i > m$ which is not possible for the set $[m]$.

If the chain has the longest chain of $l$ elements, then the difference of the arithmetic progression ranges from $1$ to $\lfloor \frac{m}{l} \rfloor$.
After this observation we use the statement of Lemma~\ref{co:d-elements} and the union bound.
\[
\prob{\vlbin{[m]} \geq l} \leq 2 \frac{m}{l} \frac{O(1)}{(l - 1)m} = O\left(\frac{1}{(l - 1)l}\right).
\]

Then we simply conclude that
\[
\expect{\vlbin{[m]}} \leq \sum_{l = 1}^m O\left(\frac{1}{l^2}\right) = O(1).
\]

\end{proof}


\section{Lemmas about factorization}
\begin{lemma}
Let $f \geq m$ be an integer such that $m \mid f$.

If $h \in \hlin$, then $|\hfact{h}{f}| = m p$.

Conversely; if $(h_u, h_f) \in \hlinr{f} \times \rot{f}{m}$, then $|\{h \in \hlinr{m} \mid h = h_u \odot h_f \}| = p$.
\end{lemma}

\begin{corollary}
\[
    \prob{\vlbin{S} = x} = \probs{\substack{h_u \inu \hlinr{f} \\ h_f \inu \rot{f}{m}}}{\lbin{h_u \circ h_f}{S} = x}
\]
\end{corollary}
\begin{proof}
Fix $h \in \hlin$ and $(h_u, h_f) \in \hfact{h}{f}$.
It is clear that there exists a unique rotation $r \in \rot{m}{m}$ such that $r \circ h = h_f \circ h_u$ and thus $\lbin{h}{S} = \lbin{r \circ h}{S} = \lbin{h_f \circ h_u}{S}$.
\begin{align*}
    \prob{\vlbin{S} = x} & = \probs{\substack{h \inu \hlin \\ r \inu \rot{m}{m}}}{\lbin{r \circ h}{S} = x} \\
        & = \probs{\substack{h_u \inu \hlinr{f} \\ h_f \inu \rot{f}{m}}}{\lbin{h_u \circ h_f}{S} = x}.
\end{align*}
\end{proof}

\section{Other ideas}
\subsection{A failed lower bound}
Let us split the interval $[m]$ into $m/d - 1$ subintervals of the same length $d$; omit the last possibly incomplete one.

If one of these subintervals collides into a single element, we have that $\vlbin{[m]} \geq d$.
By Bonferroni inequalities we may conclude that
\[
\begin{split}
\prob{\vlbin{[m]} \geq d} & \geq \left(\frac{m}{d} - 1\right) \cdot \prob{|h([d])| = 1} \\
	& \qquad - \left(\frac{m}{d} - 1\right)^2\prob{|h([d])| = 1 \wedge |h(i + [d])| = 1} \\
	& \geq \frac{m}{2d} \cdot \frac{1}{3dm} - \frac{m^2}{d^2} \cdot \frac{1}{dm} \\
	& \geq \frac{1}{6d^2} - \frac{m}{d^3}.
\end{split}
\]

When $\frac{1}{6d^2} \geq \frac{m}{d^3}$, i.e. $d \geq \frac{m}{6}$ we can take an integral only with an uninteresting result.

This is not the right way, obviously.
The number of events in the minus part is quadratic wrt. the number of events in the positive part.
The asymptotic nature of the estimates is the same, too.
Even if the upper bound was even zero, the summing $\sum_{x \in \mathbb{N}}\frac{1}{x^2}$ would be $\Theta(1)$.

With these asymptotically tight bounds finding a lower or upper bound is the same task as finding the upper bound (more or less).


\end{document}


