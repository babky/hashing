\section*{Appendix -- proof of Theorem \ref{theorem-amortised-expected-time}}

Let us describe how we have chosen the potential function. In the proof we partition the sequence of performed operations into two types of cycles. We distinguish between the work required to enforce Load Factor Rule and the work needed by keeping Chain Limit Rule. During so called $\alpha$-cycles we gather potential needed to rehash the table to enforce Load Factor Rule. From this potential we pay the needed \emph{Rehash} operation at the end of the cycle. The second type of cycles, l-cycle, is essential for analysis of Chain Limit Rule. Every l-cycle has its potential charged at the beginning and from this potential we are able to pay the expected time spent by keeping Chain Limit Rule.

We deal with the amortised time of Find and unsuccessful Insert or Delete operations in advance. Their expected running time is proportional to the expected chain length. From Theorem \ref{theorem-expected-chain-length-universal} it follows that this value is constant. Since chains are bounded by $l(m, \alpha', p)$ we have that the worst case time of Find operation is $O(1 + l(m, \alpha', p))$. We require that these operations do not change the potential and with our potential this is true. Our analysis is thus simplified by omitting \emph{Find} and unsuccessful \emph{Delete} and \emph{Insert} operations from the sequence of operations. So let the sequence $o = \{o_i\}_{i=1}^{k}$ denote the successful \emph{Insert} and \emph{Delete} operations, $o_i \in \{Insert, Delete\}$ for $i = 1, \dots, k$.

\begin{definition}[$\alpha$-cycle]
Every \emph{$\alpha$-cycle} ends just after the operation causing violation of Load Factor Rule.
\end{definition}
Notice that it is not important if load factor violates the upper or the lower bound.

\begin{definition}[l-cycle]
The \emph{l-cycles} are the partitioning of the sequence $\{o\}_{i = 1}^{k}$ such that every l-cycle ends after the operation satisfying either of the following conditions is satisfied.
\begin{enumerate}
\item The operation causes the violation of the load factor rule.
\item The operation is the $(\alpha' - \alpha_u) m$\textsuperscript{th} successful insertion from the beginning of the l-cycle.
\end{enumerate}
\end{definition}
Notice that if an $\alpha$-cycle ends after an operation, the corresponding l-cycle also ends after the same operation, too. 

The potential $p$ consists is the sum of two parts $p_1$ and $p_2$ so $p = p_1 + p_2$. The first part of potential is used to distribute the time needed for rehashing the table at the end of an $\alpha$-cycle across operations inside it. The second parts deals with the expected time needed to enforce Chain Limit Rule.

Let $e$ denote the expected number of trails when finding a suitable function for a set, $i_{\alpha}$ be the number of insertions and $d_{\alpha}$ be the number of deletions performed successfuly so far in the current $\alpha$-cycle. The value $i_l$ denotes the number of insertions performed so far in the current l-cycle. The variable $r$ denotes the number of performed \emph{Rehash} operations, which are caused by Chain Limit Rule violation counted from the initial state. The variable $c$ denotes the number of current l-cycle counted from the beginning starting at one. We define the parts $p_1$ and $p_2$ as
\[
\begin{split}
p_1 & = \frac{2ei_{\alpha}}{\alpha_u - \alpha_k} + \frac{2ed_{\alpha}}{\alpha_m - \alpha_l}, \\
p_2 & = \frac{ei_{l}}{\alpha' - \alpha_u} + (ce - r) m.
\end{split}
\]

Remark that the initial potential $p_0$ equals $em_0$ where $m_0$ is the initial size of the hash table and hence $p_0 \in O(1)$. Execution of a single operation, without possible subsequent rehash, is expected to take $O(1)$ time because we iterate through a chain with a constant expected length and obtaining the hash value takes only constant time. Possible rehash seeks for a suitable function every try requires $O(m)$ time and by Lemma \ref{lemma-linear-transformations-trials} we expect $e$ trials. In the analysis we just compute the potential difference and assume that rehash takes $O(em)$ time and the operation itself runs in $O(1)$ time. In the proof we use the notation that values of variables $c, r, i_\alpha, d_\alpha, i_l$ refer to the state just before the execution of the analysed operation.

The analysis of \emph{Delete} operation is simpler and is shown first. When a deletion is performed we have to discuss the following two cases.
\begin{itemize}
\item \textbf{\emph{Delete} operation is not the last one in its $\alpha$-cycle.} The potential difference is constant since $\Delta p = \Delta p_1 + \Delta p_2 = \frac{2e}{\alpha_m - \alpha_l} + 0 \in O(1)$.

\item \textbf{\emph{Delete} operation is the last one in its $\alpha$-cycle.} Notice that at the end of the cycle $d_\alpha = (\alpha_m - \alpha_l)m$ and after the operation values of $i_\alpha$ and $d_\alpha$ are zeroed. The expected amortised complexity of the operation is constant since
\[
\begin{split}
a
	& = O(1) + O(em) + \Delta p_1 + \Delta p_2 \\
	& \leq O(em) -2em + ((c + 1)e - r)m - (ce - r)m \\
	& = O(em) - em.
\end{split}
\]

After rescaling the potential the claim holds.
\end{itemize}

The analysis of \emph{Insert} now follows.
\begin{itemize}
\item \textbf{The operation is not last in neither of its $\alpha$-cycle or l-cycle and Chain Limit Rule is not violated.}
We have already shown that the expected running time is constant and the potential change is constant, too, since the potential change is constant, 
$\Delta p = \Delta p_1 + \Delta p_2 = \frac{2e}{\alpha_u - \alpha_k} + \frac{e}{\alpha' - \alpha_u} \in O(1)$.

\item \textbf{The operation is last in its $\alpha$-cycle.} 
Since at the end of the $\alpha$-cycle $i_\alpha = (\alpha_u - \alpha_m)m$ the expected amortised time required to execute the whole operation may be bounded from above as
\[
\begin{split}
a
	& = O(1) + O(em) + \Delta p  \\
	& \leq O(em) - 2em + ((c + 1)e - r)m - (ce - r)m \\
	& = O(em) - em.
\end{split}
\]

Scaling of the potential from the analysis of \emph{Delete} operation is sufficient for this case and the claim thus holds. 

\item \textbf{Operation is the last one in the l-cycle and Chain Limit Rule is not violated.} Under these assupmtions it follows that $i_l = (\alpha' - \alpha_u)m$ hence $\Delta p_2 = ((c + 1)e + r)m - em - rm = 0$. Since $\Delta p_1 = \frac{2e}{\alpha_u - \alpha_m}$ the expected amortised time of the operation is constant.

\item \textbf{Chain Limit Rule was violated during the performed insertion.}
The operation took $O(1) + O(\Delta r m)$ time. Whole potential change is equal to \[ \frac{2e}{\alpha_u - \alpha_m} + \frac{e}{\alpha' - \alpha_u} - \Delta r m .\] The already performed rescaling of the potential deals with the time needed to rehash the table. The expected amortised complexity of the operation is constant.
\end{itemize}

In order to be properly able to estimate the expected running time of the sequence of operations we have to show that $\Expect{p_k} \geq 0$. If it holds, then from Remark \ref{remark-expected-sequence} it follows that $\Expect{T} = \Expect{A} + O(1)$. In our analysis we have already shown that $\Expect{A} + O(1) = O(k) + O(1) = O(k)$. 

At first notice that the variable $c$ is incremented by one at the beginning of every $l$-cycle. The part $p_2$ of potential is thus increased by $em$; the potential is ``charged''. This ``charge'' is paid by the operations from the previous $l$-cycle or from $p_0$ if we are in the initial state. Now consider the sequence of sets $S_1, S_2, \dots$. Let $S_1$ be equal to the set stored at the beginning of the l-cycle. $S_2$ is the union of $S_1$ and the set stored immediately after the first violation of Chain Limit Rule. $S_3$ is the union of $S_2$ and the set stored after the second violation and so on. Realise that there are at most $(\alpha' - \alpha_u)m$ successful insertions in an l-cycle and $|S_1| \leq \alpha_u m$. So the last set of the sequence contains at most $\alpha'm$ elements. We can use Lemma \ref{lemma-sets} for the sequence and immediately obtain that the expected number of trials in an l-cycle equals $e$. From this fact it is clear that during an l-cycle $\Expect{\Delta r} = e$ and at the end of every l-cycle $\Expect{ce - r} = 0$. Realising the obvious fact that during an l-cycle the value of $r$ may only grow we conclude that
\[
\Expect{p_k} = \Expect p_1 + \Expect p_2 \geq m\Expect{ce - r} \geq 0.
\]
\qed