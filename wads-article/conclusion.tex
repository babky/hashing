\section{Conclusion}
\label{section-conclusion}
Our approach would work well not only with separate chaining but also with linear probing or double hashing if the distribution or at least expected value of length of the longest probe sequence is known. Mallalla in his dissertation \cite{Malalla:2004:THS:1124034} showed how to use two choices paradigm with linear probing. Again we have two possibilities if the data is distributed well we could assume full randomness of the hash function or use a uniform universal system. Then we can apply his solution and design a limit function bounding the length of the longest probe sequence which in this case is again $\bigo(\log \log n)$.

However the best solution working without further assumptions on data and uniform systems is to study behaviour of universal hashing with efficient and natural classes of hash functions. An interesting  problem for this worst case aware hashing is to derive bounds for two choices paradigm connected with a reasonable universal system. Our simulations show that using simple systems, e.g. functions of the form $((ax + b) \bmod |U|) \bmod |V|$ or higher degree polynomials for $|U|$ being a prime are not impressive. Hashing $2^{29}$ elements with these universal systems results in longest chains containing about $250$ elements for certain types of sets. However, with system of all linear transformations we were not able to find a sets having more than $15$ elements for the same size of the stored set.

As showed in \cite{DBLP:conf/alenex/ThorupZ10} linear probing is also possible with universal hashing provided that at least $5$-wise independent family of functions is used. However bounds on the length of the longest chain are not known so far. Because of cache behaviour currently fastest hash tables use linear probing and thus obtaining bounds on $\lpsl$ for universal hashing is an interesting problem which could result in practically fast hash table giving warranties for worst case look up time. In addition two choices paradigm may be helpful in a significant improvement of such warranties.