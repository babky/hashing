% This is the LaTeX source for the article dealing with Worst Case 
% Aware Universal Hashing. The source code is a modification of the 
% file typeinst.tex which may be found at http://www.springer.com/lncs
% and ftp://ftp.springer.de/data/pubftp/pub/tex/latex/llncs/latex2e

\documentclass[runningheads,a4paper]{llncs}
\newtheorem{definition}{Definition}
\newtheorem{theorem}{Theorem}
\newtheorem{lemma}{Lemma}

\newcommand{\setdelim}{\mid}
\newcommand{\Prob}[1]{\mathbf{Pr}\left(#1\right)}
\newcommand{\Expect}[1]{\mathbf{E}\left(#1\right)}
\newcommand{\Variance}[1]{\mathbf{Var}\left(#1\right)}
\newcommand{\lpsl}{\mathbf{lpsl}}
\newcommand{\psl}{\mathbf{psl}}
\newcommand{\Indicator}{\mathbf{I}}
\newcommand{\rank}[1]{\mathrm{rank}\left({#1}\right)}
\newcommand{\dimension}[1]{\mathrm{dim}\left({#1}\right)}
\newcommand{\nullspace}[1]{\mathcal{N}\left({#1}\right)}
\newcommand{\matrixkernel}[1]{\mathrm{Ker}\left({#1}\right)}
\newcommand{\vecspan}[1]{\mathrm{span}\left({#1}\right)}
\newcommand{\vecspace}[1]{\mathbb{Z}_{2}^{#1}}
\newcommand{\bigo}{\mathcal{O}}
\newcommand{\littleo}{o}
\newcommand{\bige}{\mathrm{\Theta}}


\usepackage{amsmath}
\usepackage{amsfonts}
\usepackage{amssymb}
\setcounter{tocdepth}{3}
\usepackage{graphicx}
\usepackage{url}

\urldef{\mail}\path|babkys@gmail.com,vcunat@matfyz.cz|
\newcommand{\keywords}[1]{\par\addvspace\baselineskip
\noindent\keywordname\enspace\ignorespaces#1}
\usepackage[utf8]{inputenc}

\begin{document}

\mainmatter

\author{Martin Babka, Vladimír Čunát}
\authorrunning{Martin Babka, Vladimír Čunát}

\title{Worst Case Aware Universal Hashing\thanks{This research was supported by SVV project number 263 314.}}

\titlerunning{Worst Case Aware Universal Hashing}

\institute{Faculty of Mathematics and Physics, Charles University,\\ Prague, Czech Republic \\
\mail\\
\url{http://www.ktiml.mff.cuni.cz/~babka/hashing}}

\toctitle{Worst Case Aware Universal Hashing}
\tocauthor{Martin Babka, Vladimír Čunát}
\maketitle

\begin{abstract}
Hash tables are one of the simplest and most often used solutions to the dictionary problem. 
We study possibilities of designing a simple hash table giving a worst case time warranty for Find operation.
We exploit properties of various universal families of functions so that a hash table is able to guarantee the worst case running time of Find operation.
Since the mentioned time warranty strongly depends on the used universal system, we present various systems capable of interesting bounds.
To obtain such a limit we also combine the "two choices paradigm" with universal hashing which gives the $\bigo(\log \log n)$ bound with certain reasonable requirements.
The proposed hash table achieves constant amortized running times of all operations in the expected case.

\keywords{universal hashing, data structures, two choices}
\end{abstract}

\section{Introduction}
Dictionary is a data structure which allows storing and querying data associated with a given key. If there are no assumptions on the set of the keys, e.g. it is not ordered, then we often represent such dictionaries by \emph{hash tables}. We just need a suitable function that maps keys to positions in a table and we can easily perform operations \emph{Find}, \emph{Insert} and \emph{Delete}. 

We always assume that the hash function behaves randomly in some way, which ensures each operation has expected constant running time. There are two major ways how to deal with the randomness. The first approach is to assume uniformly and independently distributed input data. The formal description of the requirements may be found in \cite{DBLP:books/sp/Mehlhorn84}. In some situations they are not necessarily satisfied but weaker conditions may hold. In this case we switch to randomization provided by uniform selection of the hash function from a universal family. This method is known as \emph{universal hashing} and was pioneered by Carter and Wegman in \cite{DBLP:journals/jcss/CarterW79}. 

The main difference between universal and plain hashing is that we switch to a different probability space. In the area of plain hashing the probability space is formed by uniform and independent selection of the input data. On the other hand in the case of universal hashing a random hash function is chosen uniformly from a finite family of functions. Unfortunately dealing with common universal systems brings a lot of dependencies when considering probabilities of collisions.

Although universal hashing guarantees expected constant time, this is not enough when we need a worst case warranty. Perfect hashing \cite{Fredman:1984:SST:828.1884} is an extension of universal hashing which addresses this problem for static sets. In order to add update operations various dynamizations of perfect hashing \cite{DBLP:journals/siamcomp/DietzfelbingerKMHRT94} and a real time hash table \cite{DBLP:conf/icalp/DietzfelbingerH90} were proposed. Perfect hashing, real time hash tables together with cuckoo hashing \cite{DBLP:conf/esa/PaghR01} guarantee a constant look up and the other operations run in expected amortized constant time.

Our motivation is to close the gap between perfect hashing and plain universal hashing. We propose a simple modification of universal hashing giving a worst case warranty to Find operation. A natural way how to do it is to bound lengths of chains by a convenient limit function and rehash the table when this rule is violated. This is the case of our model.

Main advantage of the model is its simplicity when compared to perfect hashing. Since we do not need a perfect hash function, the update operations should run faster. Even greater speedup may be achieved when just a simple universal class is required. Unlike cuckoo hashing, which resembles open addressing, we have chosen a different approach. On the other hand the amortization technique that we present may be used with double hashing, too. Compared to perfect hashing our method does not use second level hash tables. However, utilization of this technique is possible and significantly improves the warranty.

\subsection{Notation}
$U$ denotes the universe, the set of possible key values. $V$ denotes the addresses of the hash table. We refer to $S \subset U$ as to the set of stored keys. The size of the hash table is denoted by $m = |V|$ and the number of stored elements is $n = |S|$, usually $n \ll |U|$. The load factor of the table is denoted by $\alpha = \frac{n}{m}$. It is kept in a predefined interval so that the space is not wasted and its maximal value is usually less than one.

Mapping $f\colon U \rightarrow V$ denotes the current hash function. We will discuss probabilistic properties of hash functions and their families in more detail later.

We often work with the following two random variables for a fixed stored set~$S$. The length of the chain at address $y \in V$ is denoted by $\psl(y, S)$. The length of the longest chain, $\lpsl(S)$, is defined as $\lpsl = \max_{y \in V} \psl(y, S).$ We omit the parameter $S$ when we talk about bounds that hold for any $S \subseteq U$.

\subsection{Universal hashing}
Universal hashing solves the dictionary problem using explicit randomization. Hash functions are picked from a universal system. These classes of hash functions ensure that for any stored set there are many functions  behaving "properly".

This proper behavior may be understood differently and leads to various definitions of universal systems. However, from each of the definitions it follows that there is only a constant number of elements colliding with a given element. Hence the expected length of a chain is constant and thus the expected time complexity of Find operation is constant, too.

\begin{definition}[$c$-universal system \cite{DBLP:journals/jcss/CarterW79}]
\label{definition-c-universal-system}
Let $H$ be a multiset of hash functions from $U$ to $V$. We say that $H$ is a \emph{$c$-universal system} where $c \in \bbbr_0^+$ if for arbitrary different $x, y \in U$
\[
\left|\lbrace f \in H \setdelim f(x) = f(y) \rbrace\right| \leq \frac{c|H|}{m}.
\]
\end{definition}

To be precise, the set on the left side of the above expression is also a multiset. Since functions are chosen from $H$ uniformly, we may equivalently restate our definition in terms of probability as
\[
\Prob{f(x) = f(y)} \leq \frac{c}{m}.
\]

From now on we assume the uniform choice of the current hash function $f$ from a~universal system $H$ without any further reference.

More powerful definitions include strong \emph{$k$-universality}, which is also called \emph{$k$-wise independence}, \emph{strongly $\omega$-universal} systems and \emph{uniform} systems.
\begin{definition}
Let $k > 0$ be an integer. System of functions $H$ is
\begin{itemize}
	\item \emph{almost strongly $k$-universal} with constant $c > 0$ if for any sequence of pairwise different elements $x_1, \dots, x_k \in U$ and arbitrary $y_1, \dots, y_k \in V$ \[\Prob{f(x_1) = y_1, \dots, f(x_k) = y_k} \leq \frac{c}{m^k}\mbox{,}\]
	\item \emph{strongly $k$-universal} \cite{DBLP:conf/focs/WegmanC79} if it is almost strongly $k$-universal with $c = 1$,
	\item \emph{strongly $\omega$-universal} \cite{DBLP:conf/focs/WegmanC79} if it is strongly $k$-universal for each $k \in \bbbn$,
	\item \emph{(almost) uniform} \cite{DBLP:journals/siamcomp/PaghP08} if it is (almost) strongly $n$-universal.
\end{itemize}
\end{definition}

Notice that strongly $k$-universal systems are fully random for up to $k$ different elements and provide only \emph{limited randomness} for more than $k$ elements. On the other hand strongly $\omega$-universal systems behave fully randomly. When we need estimates only for the $n$ stored elements, then the concept of uniform systems is as powerful as full randomness.

\subsection{Our approach}
We aim to design a model of universal hashing which guarantees the worst case running time of Find operation independently of the stored set. In addition, we want the operations to have constant amortized running times in the expected case. One obvious reason for amortization is the fact that we have to keep the load factor in some predefined bounds. Reasonable space consumption determines a~lower bound and the expected chain length give an upper bound.

The achieved warranty strongly depends on the underlying universal system. Our first step is to point out some systems for which estimates on the expected length of the longest chain are known. The second step is to connect these classes with our model.

An obvious way of giving such worst case warranty is to rehash the whole table whenever the length of the longest chain exceeds the prescribed limit. However, we need an extra requirement on the bound. The probability that a random function \emph{creates a long chain} is less than a prescribed probability rate. The requirement means that a randomly chosen function is likely to work well with the stored set.
%^TODO unclear: what is the requirement?

The concept of limit functions is formalized in Sect.~\ref{section-limit} and in order to find them we deal with various universal systems. In Sect. \ref{section-model} we describe our model and connect it with the discussed systems. Section~\ref{section-conclusion} points out further improvements and alternatives of the model. Possibilities of finding other limit functions are discussed, too. The newly proposed approaches should be competitive with current plain hashing and they should have better behavior than chained hashing in practice.

\section{Obtaining the limit function}
\label{section-limit}
Limit functions play a crucial role in our model. The lower the values of a~limit function are the better warranty is obtained. Apparently each limit function depends on the size of the table and the load factor. However, our concept also estimates the probability that a selected function creates a chain longer than the predefined limit. When thinking about this probability, there are two choices standing against each other. Naturally the probability of a random function being unsuitable is higher for small limit functions.

\begin{definition}[Limit function, trimming rate, suitable function]
\label{definition-limit-function}
Let $l\colon \bbbn \times \mathbb{R}_0^+ \times (0, 1) \rightarrow \bbbn$, $m$ be the size of the hash table and $p \in (0, 1)$.  We say that $l$ is a \emph{limit function} with \emph{trimming rate} $p$ if $\Prob{\lpsl > l(m, \alpha, p)} \leq p$.

Let $l$ be a limit function with a~trimming rate $p$. We say that a function $f \in H$ \emph{does not create a long chain}, or equivalently is \emph{suitable}, if $\lpsl \le l(m, \alpha, p)$ where $\lpsl$ is the value of the random variable $\lpsl$ for the the function $f$.
\end{definition}

When we know $\Expect{\lpsl}$, then a limit function may be found easily using Markov inequality. For the trimming rate $1/k$ the limit function $k \Expect{\lpsl}$ is valid since from Markov inequality it follows that $\Prob{\lpsl \geq k \Expect{\lpsl}} \leq {1}/{k}$. Better limit functions may be obtained when we have an upper bound on probability distribution function of the random variable $\lpsl$.

\subsection{Linear hash functions}
Alon, Dietzfelbinger, Bro Miltersen, Petrank and Tardos \cite{DBLP:journals/jacm/AlonDMPT99} found an interesting upper bound on $\Expect \lpsl$. They used the system of all linear functions between two vector spaces over the field $\bbbz_2$. Representing $U$ and $V$ as vector spaces does not cause any problem. Each key -- binary number may be represented as a vector from a sufficiently large vector space over $\bbbz_2$. If the initial size of the hash table is a power of two and rehash may only double or halve it, then $V$ is always a vector space over $\bbbz_2$.

\begin{theorem}[\cite{DBLP:journals/jacm/AlonDMPT99}]
\label{theorem-linear-hash-functions-dietzefelbinger}
Suppose universal hashing with the system of linear functions from $U$ to $V$. When storing $m \log m$ elements, $\Expect{\lpsl} = \bigo(\log m \log \log m)$. 
\end{theorem}

The major problem of Theorem \ref{theorem-linear-hash-functions-dietzefelbinger} is the high multiplicative constant but it can be significantly reduced by a refinement of the original proof. First, in order to obtain a valid limit function, a dependence of $\lpsl$ on $\alpha$ needs to be discovered. Further improvement is obtained by storing $n$ elements for $n = \bige(m)$ instead of $n = m \log m$. In our computations we also introduced some variables and optimized their values in order to reduce the multiplicative constant.

\begin{theorem}
\label{theorem-linear-refined}
Assume universal hashing with the system of all linear transformations between vector spaces over $\bbbz_2$. When storing $n = \bige(m)$ elements, then $$\Expect{\lpsl} \leq 538 \alpha \log n \log \log n + 44,$$ $$\Prob{\lpsl \geq 57.29 \log m \log \log m} < 0.5.$$
\end{theorem}

Let us note that the exact constants for different probabilities and load factors can be found by a simple computer program. Now we need only the stated results.

\subsection{Two choices paradigm}
The two choices paradigm comes from the area of balls and bins systems. For separate chaining with a single perfectly random function it is known that $\Expect{\lpsl} = \bigo\left({\log n}/{\log \log n}\right)$. A proof of this bound may be found in \cite{DBLP:books/sp/Mehlhorn84}. The same bound also holds with a single function chosen from an almost uniform system. On the other hand for $k$-wise independence this problem becomes harder and general bounds for universal hashing are not available.

A further study \cite{DBLP:conf/stoc/AzarBKU94} of balls and bins systems discovered that hashing with at least two independent fully random hash functions brings far better results. If each stored element is put inside a least loaded bin of $d$ ones, then with a high probability the most loaded bin contains less than ${\ln \ln n}/{\ln d} + \bigo(1)$ elements. It is not so hard to realize that these results hold in case of universal hashing with almost uniform systems. Bounds that are possible to achieve with a limited independence for a general $S \subset U$, are not known so far.

Because of the following definition we have to extend the notation. For $k \in \bbbn$, $x_1, \dots, x_k \in U$ and $f \in H$ the function $f$ applied to a vector $(x_1, \dots, x_k) \in U^k$ is defined as $f((x_1, \dots, x_k)) = (f(x_1), \dots, f(x_k))$.
\begin{definition}[Independent hash functions]
\label{definition-independent-hash-functions}
Let $H$ be a uniform system of hash functions and $f, g \in H$ be two functions chosen uniformly at random from the system $H$. We say that the functions $f, g$ are \emph{independent} if for each $\vec{x} \in U^n$ with different elements and arbitrary $y, z \in V^n$ $$\Prob{f(\vec{x}) = \vec{y}, g(\vec{x}) = \vec{z}} = \Prob{f(\vec{x}) = \vec{y}} \Prob{g(\vec{x}) = \vec{z}}.$$
%TODO revise
\end{definition}

Notice that the independent choice of $f_1, \dots, f_d$ from $H$ is sufficient for the functions to be independent in terms of Definition \ref{definition-independent-hash-functions}. Theorem \ref{theorem-universal-hashing-two-choices} is a straightforward restatement of results pioneered in \cite{DBLP:conf/stoc/AzarBKU94} which originally hold only with fully random functions.

\begin{theorem}
\label{theorem-universal-hashing-two-choices}
Let $H$ be an almost uniform universal system with a constant $c$ such that $c > 0$ and $d \in \bbbn$. Assume that each stored element $x$ is placed into a least loaded bucket from $f_1(x), \dots, f_d(x)$. The addresses are determined by $d$ independent hash functions $f_1, \dots, f_d \in H$. If $n \leq m$ and $c \alpha \leq 1$, then $$\Prob{\lpsl > \frac{\ln \ln n}{\ln d} + 7} \in \littleo\left(\frac{1}{n}\right).$$
\end{theorem}
\begin{proof}
We only discuss differences from the proof for perfectly random functions found in \cite{Mitzenmacher:2005:PCR:1076315} or \cite{DBLP:conf/stoc/AzarBKU94}. Observe that the probability of collisions of $k$, $k \leq n$, different elements is at most ${c}/{m^k}$. The probability of collisions in the same sequence of elements with a truly random function equals ${1}/{m^k}$. Thus we have to calculate with the multiplicative constant $c$.
From the independence of functions $f_1, \dots, f_d$ it is clear that $\Prob{f_i(x) \in R,\, \forall i \in \{0, \dots, d\}} = \prod_{i = 1}^{d}\Prob{f_i(x) \in R}$ for each $x \in U$ and $R \subseteq V$. Hence for $c = 1$ no change has to be done. However, for $c > 1$ our assumption $\alpha c \leq 1$ is sufficient to prove the result.
\qed
\end{proof}

To be precise we have to discuss existence and efficiency of uniform systems. It is obvious that the system of all functions is strongly $\omega$-universal and hence uniform. The main problem here are space requirements -- $\bigo(|U| \log |V|)$ bits. There are other $\omega$-universal classes known but they are quite complex. Although their functions may be computed in a constant time, they are not efficient enough. 

A convenient example of a uniform system appeared in \cite{DBLP:journals/siamcomp/PaghP08}. Its construction is randomized and the system is uniform with a high probability.
\begin{theorem}[\cite{DBLP:journals/siamcomp/PaghP08}]
\label{theorem-uniform-system}
For any constant $c > 0$ there is a RAM algorithm constructing a random family of functions from $U$ to $V$ in expected $\littleo(n) + (\log \log|U|)^{\bigo(1)}$ time and $\littleo(n)$ words of space, such that:
\begin{itemize}
\item With probability $1 - \bigo(1 / n^c)$ the family is uniform on $S$.
\item There is a RAM data structure of $\bigo(n \log|V| + \log \log|U|)$ bits, which is optimal, representing its functions such that function values can be computed in constant time. Choosing a function from the system takes $\bigo(n)$ time.
\end{itemize}
\end{theorem}

The underlying RAM has words of size $\bige(\log |U| + \log |V|)$ bits. Since the construction of the system is probabilistic and it does not have to be uniform, the limit function is valid only with a certain probability. Because the limit function is valid when $H$ is uniform, the overall trimming rate may be computed as
\[
\begin{split} 
& \Prob{\lpsl \leq \frac{\ln \ln n}{\ln d} + 7,\, H \mbox{ is uniform}} \\
	& \qquad = \Prob{\lpsl \leq \frac{\ln \ln n}{\ln d} + 7 \mid H \mbox{ is uniform}} \Prob{H \mbox{ is uniform}} \\ 
	& \qquad = \left(1 - \littleo\left(\frac{1}{n}\right)\right)\left(1 - \bigo\left(\frac{1}{n^c}\right)\right). \\
\end{split}
\]
The positive fact is that this probability tends to one as $n$ increases. The unpleasant thing is that choosing a new function means regenerating the system and takes $\bigo(n)$ time.

\subsection{Truly random functions}
\label{subsection-truly-random-functions}
In plain hashing we get full randomness because of the probabilistic properties of input. Although these assumptions are seldom correct, we often have in some way randomly distributed data. As observed in \cite{DBLP:conf/soda/MitzenmacherV08}, strong $2$-universality may behave like a truly random function when input contains certain amount of entropy. If the requirements are satisfied, then we can use the same limit functions as for truly random functions.

In case of the two choices paradigm the limit function is $\bigo(\log \log n)$ and with a~single hash function $\bigo(\log n / \log \log n)$. These bounds are tight.

Notice that bounds discussed in previous sections hold for any stored set without any other assumption. They rely only on the randomness inherited by using a high-quality universal system. In a situation when the input data are random enough it is correct and much more efficient to use a simpler universal system with the assumption of full randomness.

\section{Model of Universal Hashing}
\label{section-model}
The hashing scheme we propose is a dictionary allowing operations Find, Insert and Delete. Moreover, the model guarantees a worst case time for Find operation determined by the underlying universal system. Of course constant expected time complexity of universal hashing is inherited. 

\subsection{Algorithms}
With separate chaining we are able to provide a worst case warranty if there is not a chain longer than the value of the limit function. If there is a chain violating this bound the whole table is rehashed. Naturally when rehashing we always seek for a function which does not create such a long chain. Similarly in case of linear probing we ask if each probe sequence is shorter than the prescribed limit. Following rules exactly describe the implementation of the model.
\begin{itemize}
\item \textbf{Universal class.} We suppose using at least $c$-universal system. We also assume that we have a valid limit function $l$ with a trimming rate $p$.

\item \textbf{Load Factor Rule.} The load factor of the table is kept in a predefined interval $[\alpha_l, \alpha_u] \subset (0, 1)$. If the load factor is outside the interval, then the table is rehashed into a greater or smaller table respectively. Moreover, new $m$ is chosen so that the load factor is as near as possible to a prescribed value $\alpha_m$ such that $\alpha_l < \alpha_m < \alpha_u$. 

\item \textbf{Chain Length Limit Rule.} Assume we are given a parameter $\alpha'$ such that $\alpha_u < \alpha'$. This parameter specifies the load factor in respect to which the limit function is computed. If there is a chain longer than $l(m, \alpha', p)$, then the table is rehashed.
\end{itemize}

First we determine the expected length of a chain. Then we estimate number of trials required to find a suitable function during a sequence of update operations. The state of the hash table changes after each update and therefore we must carefully estimate the probability of violation of Chain Length Limit Rule. Finally such estimate for the load factor equal to $\alpha'$ allows us to finish the amortized analysis of the model.

Limit functions based on dictionaries allowing only Find and Insert should be treated carefully. Although deletion of an element can not prolong chains such static bound may be invalid since the assumed static model is no longer preserved. Let us note that dynamic extensions of Theorem \ref{theorem-universal-hashing-two-choices} are available and discussed in \cite{DBLP:journals/jacm/Vocking03}. However, it is easy to see that allowing deletions with the static estimates does not cause any problem in our model .

\subsection{Consequences of Trimming Long Chains}
In order to analyze the expected length of a chain we define the system of suitable functions for arbitrary stored set $S$.
\begin{definition}[$(l, p)$ - trimmed system]
\label{definition-trimmed-system}
Let $l$ be a limit function with a trimming rate $p$. The system of functions \[ H(p, l, S) = \{ f \in H \setdelim f \textit{ does not create a long chain for l with p} \} \] is called \emph{$(p, l)$-trimmed system}.
\end{definition}

The restriction of $H$ to $H(p, l, S)$ is informed about the stored set since it has a feedback if a function is suitable or not. Moreover, the uniform choice the original system $H$ with refusal of the unsuitable functions is equivalent to the uniform choice from the restricted system $H(p, l, S)$.

We show that a $(l, p)$-trimmed system has the constant of universality at most $1 / (1 - p)$ times higher than the original system. So these systems are universal only with a higher constant and thus $\Expect{\psl} \leq c\alpha/(1-p)$ where $c$ is the constant of universality of the original system $H$.

\begin{lemma}
\label{lemma-trimmed-system}
If $H$ is a universal system with a constant $c$ and $l$ is a limit function with a trimming rate $p$, then $|H(p, l, S)| \geq (1 - p)|H|$ and  \[ \Prob{f(x) = f(y) \mbox{ for } f \in H(p, l, S)} \leq \frac{1}{1 - p} \cdot \Prob{f(x) = f(y) \mbox{ for } f \in H}. \]
\end{lemma}
\begin{proof}
From Definitions \ref{definition-trimmed-system} and \ref{definition-limit-function} it is clear that $\Prob{f \in H(p, l, S)} \geq 1 - p$ and as a result $|H(p, l, S)| \geq (1 - p)|H|$. In addition for different $x, y \in U$ 
\[
\begin{split}
& \Prob{f(x) = f(y) \mid f \in H(p, l, S)} 
	= \frac{\Prob{f(x) = f(y),\, f \in H(p, l, S)}}{\Prob{f \in H(p, l, S)}} \\
	& \qquad \leq \frac{\Prob{f(x) = f(y),\, f \in H}}{1 - p} = \frac{\Prob{f(x) = f(y) \mbox{ for } f \in H}}{1 - p}. \\
\end{split}
\]
\qed
\end{proof}

For a $k$-wise independent class $H$ the probability of a collision with $H(p, l, S)$ is at most $1 / (1 - p)$ times the probability of the same collision with $H$.

Lemma \ref{lemma-no-trials} gives the estimate on the expected number of trials needed to find a function $f \in H(p, l, S)$ if $f$ is chosen uniformly from $H$.

\begin{lemma}
\label{lemma-no-trials}
If $l$ is a limit function with a trimming rate $p$, then the expected number of trials needed to find a function from $H(p, l, S)$ is at most ${1}/{(1 - p)}$.
\end{lemma}
\begin{proof}
Because subsequent choices are independent we know that \[\Prob{\mbox{at least }k\mbox{ trials are required}} \leq p^{k - 1}.\]
Hence the expected value is at most $\sum_{k = 1}^{\infty} p^{k - 1} = \sum_{k = 0}^{\infty} p^k = {1}/{(1 - p)}.$
\qed
\end{proof}

\subsection{Amortized Analysis}
The expected amortized complexity of the model is analyzed using the potential method. Let $p_i$ be the potential of the dictionary after performing $i$\textsuperscript{th} operation, $t_i$ be the time consumed by the operation and we define its amortized complexity as $a_i = t_i + p_i - p_{i - 1}$. The expected amortized complexity of the sequence, $A$, is
\[
\Expect{A} = \sum_{i=1}^{k} \Expect{a_i} = \sum_{i = 1}^{k} \left(\Expect{t_i} + \Expect{p_i} - \Expect{p_{i - 1}}\right) = \Expect T + \Expect{p_k} - \Expect{p_0}.
\]
The expected time of any sequence of operations may be estimated using the potential method as $\Expect T = \Expect A - \Expect{p_k} + \Expect{p_0}$.

Lemma \ref{lemma-sets} states an upper bound on the expected number of trials needed to enforce Chain Length Limit Rule for the sequence of sets $S_1, \dots, S_k$.
\begin{lemma}
\label{lemma-sets}
Let $S_1 \subset \dots \subset S_k$ be a sequence of sets such that $|S_k| \leq \alpha' m$ and $h_0 \in H$ be an initial function. Let $h_1, \dots, h_l \in H$ be a sequence of independently and uniformly chosen functions tried to enforce Chain Length Limit Rule. Assume that $0 = i_0 < \dots < i_k = l$ is a sequence of integers such that for each $j \in \{0, \dots, k - 1\}$
\begin{itemize}
\item the functions $h_{i_{j}}, h_{i_{j} + 1}, \dots, h_{i_{j + 1} - 1}$ are not suitable for the set $S_{j + 1}$ and 
\item the function $h_{i_{j}}$ is suitable for the set $S_j$.
\end{itemize}
Then $\Expect{l} \leq {1}/{(1 - p)}$.
\end{lemma}
\begin{proof}
A function $h$ is suitable for the set $S_k$ if only if it is suitable for the sets $S_1, \dots, S_k$. From Lemma \ref{lemma-no-trials} used for the set $S_k$ the lemma follows.
\qed
\end{proof}

\begin{theorem}
\label{theorem-amortised-expected-time}
Suppose a model of universal hashing with Load Factor and Chain Length Limit Rules with initially empty stored set. If $l$ is a limit function with a trimming rate $p$, then the expected amortized time complexity of each operation is constant and the worst case running time of Find operation is $\bigo(1 + l(m, \alpha', p))$.
\end{theorem}
\begin{proof}
To prove the theorem assume that we are given a sequence of operations. Since Find, unsuccessful Delete and unsuccessful Insert operations do not change the stored set and have no effect on the dictionary we may omit them from the sequence. Because Chain Length Limit Rule is enforced the worst case complexity of Find is clear. The expected running time follows from Lemma \ref{lemma-trimmed-system}. Now we have to prove the statement for the remaining real update operations.

To do so we partition the sequence into two types of cycles and for both of them we define a potential. The first cycle type is used to amortize violations of Load Factor Rule. With the second one we are able to distribute the time required to repair violations of Chain Length Limit Rule.

\begin{definition}[$\alpha$-cycle, l-cycle]
Cycles are partitioning of the analyzed sequence operations.
Each \emph{$\alpha$-cycle} ends just after the operation causing violation of Load Factor Rule.
Each l-cycle ends just after the operation satisfying either of the following conditions:
\begin{enumerate}
\item The operation causes the violation of Load Factor Rule.
\item The operation is $(\alpha' - \alpha_u) m$\textsuperscript{th} successful insertion counted from the beginning of the l-cycle.
\end{enumerate}
\end{definition}

Let $e$ denote the expected number of trials when finding a suitable function for a given set $S$ such that $|S| \leq \alpha'm$. From Lemma \ref{lemma-no-trials} it follows that $e \leq 1 / (1 - p)$. Let $i_{\alpha}$, ($d_\alpha$, $i_l$) be the number of all insertions (deletions, insertions) successfully performed in the current $\alpha$, ($\alpha$, l)-cycle. Let $r$ be the number of performed trials of a hash function caused by Chain Length Limit Rule violation and $c$ be the number of started l-cycles. Both $r$ and $c$ are counted from the initial state. We define the potential $p_1 = {2ei_{\alpha}}/{(\alpha_u - \alpha_k)} + {2ed_{\alpha}}/{(\alpha_m - \alpha_l)}$ and the potential $p_2 = {ei_{l}}/{(\alpha' - \alpha_u)} + (ce - r) m$.  The overall potential $p = p_1 + p_2$.

We decompose each update operation into an actual update and a possible rehashing part. Since the expected length of a chain is constant the expected running time of an update is constant as well. Observe that the overall potential is increased by a constant if the analyzed operation is not the last one in a cycle. Hence the amortized complexity of an update is constant. 

Each trial of a function from $H$ is amortized by an increase of $r$. Notice that after finishing an $\alpha$-cycle $p_1$ is decreased by at least $em$ and $p_2$ is increased by at most $em$, hence $\Delta p \leq 0$. Similarly when an l-cycle is ended $i_l = (\alpha' - \alpha_u)m$ and hence $\Delta p \leq 0$. 

Now we argue that $\Expect{ce - r} = 0$. If we omitted deletes from each l-cycle, we would get a sequence of sets created only be insertions. Since deletes may not prolong chains and from Lemma \ref{lemma-sets} applied to the obtained sequence we know that in each l-cycle there are at most $e$ trials in the expected case. Since $c$ is incremented at the beginning of each l-cycle $\Expect{p_k} \geq 0$. The theorem now follows because our potential is expected to be non-negative, $p_0 = \bigo(1)$ and as we have already seen $\Expect{T} \leq \Expect{A} - \Expect{p_k} + \Expect{p_0}$.
\qed
\end{proof}

In case of "insertion only" limit functions Delete operation may be allowed as well without any further change. If each l-cycle started with a rehash, then we would obtain an insertion only hash table. Observe that one additional rehash in each l-cycle can be amortized to a constant by introducing another potential $p_3 = {ei_{l}}/{(\alpha' - \alpha_u)}$. Moreover, performing the mentioned rehash is not necessary and as a result the first rehash in each l-cycle may be caused by invalidity of the limit function.

With trimming rates which tend to zero with growing $n$, e.g. in case of the two choices paradigm, the amortization overhead caused by Chain Length Limit Rule gradually disappears. It is caused by the fact that for a sufficiently large $n$ almost each function is suitable for every $S$.

\chapter{Conclusion}
We present a model of universal hashing which preserved $O(1)$ expected running time of the find operation. We were also able to compute the expected amortised running time for the insert and delete operations. In addition our model bounded the worst case running time by $O(\log n \log \log n)$. 

Since we based it on the system of linear transformations the time to compute the hash value of an element is worsened. The solution that we propose is to store once computed hash values within the object. This optimisation exploits warranties of the model if the find operation is dominant. 

\section{Future work}
There are many ways how the model can be improved. First it may be interesting to mathematically describe behaviour of double hashing when used with a universal class of functions. Combined with the system of linear transformations it may be possible to obtain a similar worst case bound without violating the expected running times. 

Similarly to perfect hashing the chain may be represented by a hash table allowed to have load factor with a small value. But if the elements in the bucket can not be accessed in a constant time it might be possible to rehash the small table instead of the large one. This approach may bring another optimisation.

Another brilliant idea that may play a role for example in the area of load balancing is hash by two functions simultaneously. A newly stored element is placed into a smaller bucket as stated in \cite{1076315}. When finding we have to search in both buckets associated with an element. However the expected worst case time in classic hashing is substantially better and the expected complexity is preserved.

And, of course, what is not shown in the thesis are the experimental results. A high quality benchmark of the model is required. The benchmark should be done with and without the optimisation to show the influence of the linear system on running times. Also the benchmark has to show when the warranty is needed in dependence on the operation composition and the input distribution. When we use  inputs created by random number generators they are uniformly distributed. The obtained chains are then short even for classic hashing. Such a good input are seldom present and the real cases of inconvenient inputs should be pointed out. Also when the chains are longer and the find operation is the most frequent one then it is convenient to have a worst case warranty. The question is how frequent the find operation has to be when compared to the modifying ones.

Of course, using simpler classes may bring faster times of computing the hash codes. Linear classes are quite similar and maybe we can find correspondence allowing the results found for the class of linear transformations to be brought to another faster class.

Models providing a reasonable worst case warranty with a good expected complexity may be a suitable choice for various set representation problems. Current models of hashing may provide such an warranty when enriched by a simple rule. Also approach relaxing the models providing warranties may be in help of achieving similar bounds.


\subsubsection*{Acknowledgments.}
We would like to thank Václav Koubek for many helpful comments.

\bibliographystyle{plain}
\bibliography{bibliography}


\clearpage
% \section*{Appendix -- proof of Theorem \ref{theorem-amortised-expected-time}}

Let us describe how we have chosen the potential function. In the proof we partition the sequence of performed operations into two types of cycles. We distinguish between the work required to enforce Load Factor Rule and the work needed by keeping Chain Limit Rule. During so called $\alpha$-cycles we gather potential needed to rehash the table to enforce Load Factor Rule. From this potential we pay the needed \emph{Rehash} operation at the end of the cycle. The second type of cycles, l-cycle, is essential for analysis of Chain Limit Rule. Every l-cycle has its potential charged at the beginning and from this potential we are able to pay the expected time spent by keeping Chain Limit Rule.

We deal with the amortised time of Find and unsuccessful Insert or Delete operations in advance. Their expected running time is proportional to the expected chain length. From Theorem \ref{theorem-expected-chain-length-universal} it follows that this value is constant. Since chains are bounded by $l(m, \alpha', p)$ we have that the worst case time of Find operation is $O(1 + l(m, \alpha', p))$. We require that these operations do not change the potential and with our potential this is true. Our analysis is thus simplified by omitting \emph{Find} and unsuccessful \emph{Delete} and \emph{Insert} operations from the sequence of operations. So let the sequence $o = \{o_i\}_{i=1}^{k}$ denote the successful \emph{Insert} and \emph{Delete} operations, $o_i \in \{Insert, Delete\}$ for $i = 1, \dots, k$.

\begin{definition}[$\alpha$-cycle]
Every \emph{$\alpha$-cycle} ends just after the operation causing violation of Load Factor Rule.
\end{definition}
Notice that it is not important if load factor violates the upper or the lower bound.

\begin{definition}[l-cycle]
The \emph{l-cycles} are the partitioning of the sequence $\{o\}_{i = 1}^{k}$ such that every l-cycle ends after the operation satisfying either of the following conditions is satisfied.
\begin{enumerate}
\item The operation causes the violation of the load factor rule.
\item The operation is the $(\alpha' - \alpha_u) m$\textsuperscript{th} successful insertion from the beginning of the l-cycle.
\end{enumerate}
\end{definition}
Notice that if an $\alpha$-cycle ends after an operation, the corresponding l-cycle also ends after the same operation, too. 

The potential $p$ consists is the sum of two parts $p_1$ and $p_2$ so $p = p_1 + p_2$. The first part of potential is used to distribute the time needed for rehashing the table at the end of an $\alpha$-cycle across operations inside it. The second parts deals with the expected time needed to enforce Chain Limit Rule.

Let $e$ denote the expected number of trails when finding a suitable function for a set, $i_{\alpha}$ be the number of insertions and $d_{\alpha}$ be the number of deletions performed successfuly so far in the current $\alpha$-cycle. The value $i_l$ denotes the number of insertions performed so far in the current l-cycle. The variable $r$ denotes the number of performed \emph{Rehash} operations, which are caused by Chain Limit Rule violation counted from the initial state. The variable $c$ denotes the number of current l-cycle counted from the beginning starting at one. We define the parts $p_1$ and $p_2$ as
\[
\begin{split}
p_1 & = \frac{2ei_{\alpha}}{\alpha_u - \alpha_k} + \frac{2ed_{\alpha}}{\alpha_m - \alpha_l}, \\
p_2 & = \frac{ei_{l}}{\alpha' - \alpha_u} + (ce - r) m.
\end{split}
\]

Remark that the initial potential $p_0$ equals $em_0$ where $m_0$ is the initial size of the hash table and hence $p_0 \in O(1)$. Execution of a single operation, without possible subsequent rehash, is expected to take $O(1)$ time because we iterate through a chain with a constant expected length and obtaining the hash value takes only constant time. Possible rehash seeks for a suitable function every try requires $O(m)$ time and by Lemma \ref{lemma-linear-transformations-trials} we expect $e$ trials. In the analysis we just compute the potential difference and assume that rehash takes $O(em)$ time and the operation itself runs in $O(1)$ time. In the proof we use the notation that values of variables $c, r, i_\alpha, d_\alpha, i_l$ refer to the state just before the execution of the analysed operation.

The analysis of \emph{Delete} operation is simpler and is shown first. When a deletion is performed we have to discuss the following two cases.
\begin{itemize}
\item \textbf{\emph{Delete} operation is not the last one in its $\alpha$-cycle.} The potential difference is constant since $\Delta p = \Delta p_1 + \Delta p_2 = \frac{2e}{\alpha_m - \alpha_l} + 0 \in O(1)$.

\item \textbf{\emph{Delete} operation is the last one in its $\alpha$-cycle.} Notice that at the end of the cycle $d_\alpha = (\alpha_m - \alpha_l)m$ and after the operation values of $i_\alpha$ and $d_\alpha$ are zeroed. The expected amortised complexity of the operation is constant since
\[
\begin{split}
a
	& = O(1) + O(em) + \Delta p_1 + \Delta p_2 \\
	& \leq O(em) -2em + ((c + 1)e - r)m - (ce - r)m \\
	& = O(em) - em.
\end{split}
\]

After rescaling the potential the claim holds.
\end{itemize}

The analysis of \emph{Insert} now follows.
\begin{itemize}
\item \textbf{The operation is not last in neither of its $\alpha$-cycle or l-cycle and Chain Limit Rule is not violated.}
We have already shown that the expected running time is constant and the potential change is constant, too, since the potential change is constant, 
$\Delta p = \Delta p_1 + \Delta p_2 = \frac{2e}{\alpha_u - \alpha_k} + \frac{e}{\alpha' - \alpha_u} \in O(1)$.

\item \textbf{The operation is last in its $\alpha$-cycle.} 
Since at the end of the $\alpha$-cycle $i_\alpha = (\alpha_u - \alpha_m)m$ the expected amortised time required to execute the whole operation may be bounded from above as
\[
\begin{split}
a
	& = O(1) + O(em) + \Delta p  \\
	& \leq O(em) - 2em + ((c + 1)e - r)m - (ce - r)m \\
	& = O(em) - em.
\end{split}
\]

Scaling of the potential from the analysis of \emph{Delete} operation is sufficient for this case and the claim thus holds. 

\item \textbf{Operation is the last one in the l-cycle and Chain Limit Rule is not violated.} Under these assupmtions it follows that $i_l = (\alpha' - \alpha_u)m$ hence $\Delta p_2 = ((c + 1)e + r)m - em - rm = 0$. Since $\Delta p_1 = \frac{2e}{\alpha_u - \alpha_m}$ the expected amortised time of the operation is constant.

\item \textbf{Chain Limit Rule was violated during the performed insertion.}
The operation took $O(1) + O(\Delta r m)$ time. Whole potential change is equal to \[ \frac{2e}{\alpha_u - \alpha_m} + \frac{e}{\alpha' - \alpha_u} - \Delta r m .\] The already performed rescaling of the potential deals with the time needed to rehash the table. The expected amortised complexity of the operation is constant.
\end{itemize}

In order to be properly able to estimate the expected running time of the sequence of operations we have to show that $\Expect{p_k} \geq 0$. If it holds, then from Remark \ref{remark-expected-sequence} it follows that $\Expect{T} = \Expect{A} + O(1)$. In our analysis we have already shown that $\Expect{A} + O(1) = O(k) + O(1) = O(k)$. 

At first notice that the variable $c$ is incremented by one at the beginning of every $l$-cycle. The part $p_2$ of potential is thus increased by $em$; the potential is ``charged''. This ``charge'' is paid by the operations from the previous $l$-cycle or from $p_0$ if we are in the initial state. Now consider the sequence of sets $S_1, S_2, \dots$. Let $S_1$ be equal to the set stored at the beginning of the l-cycle. $S_2$ is the union of $S_1$ and the set stored immediately after the first violation of Chain Limit Rule. $S_3$ is the union of $S_2$ and the set stored after the second violation and so on. Realise that there are at most $(\alpha' - \alpha_u)m$ successful insertions in an l-cycle and $|S_1| \leq \alpha_u m$. So the last set of the sequence contains at most $\alpha'm$ elements. We can use Lemma \ref{lemma-sets} for the sequence and immediately obtain that the expected number of trials in an l-cycle equals $e$. From this fact it is clear that during an l-cycle $\Expect{\Delta r} = e$ and at the end of every l-cycle $\Expect{ce - r} = 0$. Realising the obvious fact that during an l-cycle the value of $r$ may only grow we conclude that
\[
\Expect{p_k} = \Expect p_1 + \Expect p_2 \geq m\Expect{ce - r} \geq 0.
\]
\qed
\section*{Appendix -- proof of Theorem \ref{theorem-linear-refined}}
\setcounter{theorem}{2}

\begin{theorem}
Assume universal hashing with the system of all linear transformations between vector spaces over $\mathbb{Z}_2$. 
Let $p \in (0, 1)$ be the trimming rate and $\alpha > 0$. 
If $n = \alpha m$ elements are stored inside the hash table of size $m$, then $$\Expect{\lpsl} \leq 538 \alpha \log n \log \log n + 44\mbox{ and}$$ $$\Prob{\lpsl \geq a_{\alpha, p} \log m \log \log m + b_{\alpha, p}\log m} < p.$$ where constants $a$ and $b$ depend on the choice of $\alpha$ and $p$.
\end{theorem}

The proof of Theorem \ref{theorem-linear-refined} is rather technical we shall enclose so that verification of the results is possible.
It may be found in Chapters 5 and 6.3 at \url{http://ktiml.mff.cuni.cz/~babka/hashing/thesis.pdf}.
However, there is one change in the notation, the set of all addresses of the hash table is denoted by $B$ instead of $V$.

\section*{Appendix -- proof of Theorem \ref{theorem-universal-hashing-two-choices}}
\subsection{Notation}
For the proof of Theorem \ref{theorem-universal-hashing-two-choices} we need further notation which comes from the original proof found in \cite{Mitzenmacher:2005:PCR:1076315}. By $\mathrm{Bi}(n, p)$ we understand a binomial random variable with parameters $n$ and $p$. Elements of the stored set $S$ are placed inside the table in an arbitrary but fixed order. Let $x_1$ be the first inserted element, $x_2$ the second and $x_n$ be the last one. By the state of the system at the time $t$ we understand the placement of all elements in the table immediately after the insertion of the element $x_t$. 

The position of element $x_t$ in the chain, where it is placed, is referred to as $h(t)$. Let $\mu_i(t)$ be the number of already stored elements at the time $t$ that are at least at $i$\textsuperscript{th} position in their chains. The number of chains that contain at least $i$ elements at the time $t$ is denoted by $\nu_i(t)$. Remark that $\nu_i(t) \leq \mu_i(t)$ for every $t \in \{1, \dots, n \}$. Let $\mu_i = \mu_i(n)$ and $\nu_i = \nu_i(n)$.

\subsection{The Actual Proof}
First we create a sequence of values $\beta_i$ such that $\beta_i \geq \nu_i$ with a high probability. To estimate the probabilities we use Chernoff bound stated in Lemma \ref{lemma-chernoff-bound}. Low probabilities of the event $\beta_i < \nu_i$ are obtained using a technique called layered induction. The induction is stopped when $i = i^* > \ln \ln n/\ln d + 4$ since for $i > i^*$ Chernoff bound is no longer able to provide a good estimate. Finally the probability that $\nu_{i^*+3} \geq 1$ is obtained using a few raw estimates. The result then immediately follows from the fact that $\lpsl \geq i^*+3 \Leftrightarrow \nu_{i^* + 3} \geq 1$.

\begin{lemma}[Chernoff bound]
\label{lemma-chernoff-bound}
Let $n \in \bbbn$ and $p \in (0, 1)$. Then $$\Prob{\mathrm{Bi}(n, p) > 2np} \leq e ^ {- \frac{np}{3}} \textit{.}$$
\end{lemma}

\begin{lemma}
\label{lemma-height-of-inserted-ball}
If $\beta_i > \nu_i(t)$, then $\Prob{h(t + 1) > i} \leq \left(\frac{c\beta_i}{m}\right) ^ d$.
\end{lemma}
\begin{proof}
First we estimate the probability of placing an element into a set of bins $R$, which contains bins with $i$ or more elements, by a single function $f_1$. So let $x \in U$, $R \subseteq V$ such that $|R| \leq \beta_i$, then $$\Prob{f_1(x) \in R} = \displaystyle\sum_{r \in R} \Prob{f_1(x) = r} \leq \frac{c\beta_i}{m}.$$
The element $x_{t + 1}$ may get behind $i$\textsuperscript{th} position in the chain only when each of $d$ functions places it inside $R$. Since the functions are mutually independent we obtain the desired bound.
\qed
\end{proof}

Now we define the sequence of values $\beta_i$. To make an insight into the way how it is chosen let $p_i = \left({c\beta_i}/{m}\right) ^ d$. If $\nu_i \leq \beta_i$, then $p_i$ is greater than the probability of placing an element behind the $i$\textsuperscript{th} position as observed in Lemma \ref{lemma-height-of-inserted-ball}. From Lemma \ref{lemma-chernoff-bound} we have that $\Prob{\mathrm{Bi}(n, p_i) \geq 2np_i} \leq \exp\left({-np_i}/{3}\right)$. Random variable $\Prob{\mathrm{Bi}(n, p_i) > k}$ majorizes $\Prob{\mu_{i + 1} > k}$ and with a high probability we have that $2np_i \geq \mu_{i + 1} \geq \nu_{i + 1}$. So if we need a sequence of values $\beta_i$ such that $\beta_i \geq \nu_{i}$ with a high probability, then we have a good reason to put $\beta_{i + 1} = 2np_i = 2n\left({c\beta_i}/{m}\right) ^ d$.

Let the event $\epsilon_i$ occur if and only if $\beta_i \geq \nu_i$. Now put $\beta_4 = \frac{n}{4}$ and remark that $\Prob{\epsilon_4} = 1$. Assume that $\epsilon_4$ does not hold. Then $n \leq 4 \nu_4 < 4 \frac{n}{4} = n$ which is not possible. And finally, let us define a random binary variable $Y_t^i$ as $$Y_t^i = 1 \Leftrightarrow h(t) \geq i + 1 \mbox{ and } \nu_i(t - 1) \leq \beta_i \textit{.}$$

If the event $\epsilon_i$ occurs for every $i \in \{1, \dots, n\}$, then $\sum_{t = 1}^{n} Y_t^i = \mu_{i + 1}$. This statement holds because $\beta_i \geq \nu_i \geq \nu_i(t)$ for every $t \in \{1, \dots, n\}$ and thus the second condition for $Y_t^i$ is satisfied. Also we add one for each ball placed behind $i$\textsuperscript{th} position. 
From the previous observation if follows that
\[
\begin{split}
\Prob{\nu_{i + 1} > k | \epsilon_i} 
	& \leq \Prob{\mu_{i + 1} > k | \epsilon_i} \\
	& = \Prob{\displaystyle\sum_{t = 1}^{n} Y_t^{i} > k | \epsilon_i} \\
	& \leq \frac{\Prob{\sum_{t = 1}^{n} Y_t^{i} > k}}{\Prob{\epsilon_i}}. \\
\end{split}
\]

Now because of uniformity of the system $H$ it is true that $$\Prob{\sum_{t = 1}^{n} Y_t^i > k} \leq \Prob{\mathrm{Bi}(n, p) > k}.$$ If the system was not uniform, then we would have to deal with the problem of numerous dependencies. However, in the case of uniformity of the universal system from Lemma \ref{lemma-height-of-inserted-ball} we know that if $\beta_i \geq \nu_i$, then $Y_t^i$ holds with probability at most $p_i$ independently on $Y_{t'}^i$ for $t' < t$. Now if $\beta_i < \nu_i$, then $Y_t^i$ can not be 1 and thus $\Prob{Y_t^i = 1} \leq p_i$ independently on $\epsilon_i$.

If we substitute $2np_i$ into $k$, then we get
\[
\Prob{\nu_{i + 1} > 2np_i | \epsilon_i} \leq \frac{1}{\exp(\frac{np_i}{3})\Prob{\epsilon_i}}.
\]
Moreover, if we assume that $np_i \geq 6 \ln n$, we have that 
\[
\Prob{\nu_{i + 1} > \beta_{i + 1}} = \Prob{\nu_{i + 1} > 2np_i | \epsilon_i} = \Prob{\neg \epsilon_{i + 1}} \leq \frac{1}{n ^ 2\Prob{\epsilon_i}} \textit{.}
\]
So our desired bound works whenever $np_i \geq 6 \ln n$. Let $i^*$ be the least $i \in \bbbn$ such that $np_i < 6 \ln n$. We later show that $i^*$ is sufficiently low. Now we perform one step of the layered induction assuming that $i<i^*$. As a result we obtain the bound for $\epsilon_{i + 1}$.
\[
\begin{split}
\Prob{\neg \epsilon_{i + 1}} 
	& = \Prob{\neg \epsilon_{i + 1} | \epsilon_i}\Prob{\epsilon_i} + \Prob{\neg \epsilon_{i + 1} | \neg \epsilon_i}\Prob{\neg \epsilon_i} \\
	& = \Prob{\nu_{i + 1} > \beta_{i + 1} | \epsilon_i}\Prob{\epsilon_i} + \Prob{\neg \epsilon_i} \\
	& \leq \frac{1}{n ^ 2} + \Prob{\neg \epsilon_i} \leq \frac{i + 1}{n ^ 2} \textit{.}
\end{split} 
\]
From the layered induction we get that $\Prob{\neg \epsilon_{i^*}} \leq i^* / n^2$. 

Since $np_{i^* + 1} < 6 \ln n$ the following computation estimates the probability of the event $\neg \epsilon_{i^* + 1}$. 
\[
\begin{split}
\Prob{\neg \epsilon_{i^* + 1}}
	\leq \frac{\Prob{\mathrm{Bi}\left(n, \frac{6 \ln n}{n}\right) > 12 \ln n}}{\Prob{\epsilon_{i^*}}} 
	& \leq \frac{1}{\exp\left(2 \ln n\right)\Prob{\epsilon_{i^*}}} = \frac{1}{n ^ 2 \Prob{\epsilon_{i^*}}} \textit{.}
\end{split}
\]
By using the conditioning similar to the one used in the induction we get
\[
\Prob{\neg \epsilon_{i^* + 1}} \leq \Prob{\nu_{i^* + 1} > 12 \ln n} \leq \Prob{\neg \epsilon_{i^*}} + \frac{1}{n ^ 2} \leq \frac{i^* + 1}{n ^ 2} \textit{.}
\]

The probability of having a chain with at least $i^* + 3$ elements is found in a similar fashion.
\[
\begin{split}
& \Prob{\nu_{i^* + 3} \geq 1}
	\leq \Prob{\mu_{i^* + 3} \geq 1} \leq \Prob{\mu_{i^* + 2} \geq 2} \\
	& \quad \leq \Prob{\mu_{i^* + 2} \geq 2 | \epsilon_{i^* + 1}}\Prob{\epsilon_{i^* + 1}} + \Prob{\neg \epsilon_{i^* + 1}} \\
	& \quad \leq \frac{\Prob{\mathrm{Bi}\left(n, \left(\frac{12 c \ln n}{m}\right) ^ d\right) \geq 2}}{\Prob{\epsilon_{i^* + 1}}} + \Prob{\neg \epsilon_{i^* + 1}} \\
	& \quad \leq \binom{n}{2} \left(\frac{12 c \ln n}{m}\right) ^ {2d} + \frac{i^* + 1}{n ^ 2} \in o\left(\frac{1}{n}\right) \textit{.}
\end{split}
\]

The promised estimate for $i^* = \displaystyle\min \{i \in \bbbn \setdelim np_i < 6 \ln n\} \leq \ln \ln n / \ln d + 4$ follows from the explicit formula for $\beta_i$. By induction on $i$ we prove that $$\beta_{i + 4} = \frac{n \left(\alpha c\right) ^ {\sum_{j = 1}^{i}d ^ j}}{2 ^ {2 d ^ i - \sum_{j = 0}^{i - 1}{d ^ j}}} \textit{.}$$ For $i = 0$ we have that $\beta_4 = \frac{n}{4} = \frac{n\left(\alpha c\right) ^ 0}{2 ^ {2}}$. For a general $i$ we have that
\[
\begin{split}
\beta_{i + 5} 
	& = 2np_i = 2n \left(\frac{c\beta_{i + 4}}{m}\right) ^ d = 2n \left(\frac{cn\left(\alpha c\right) ^ {\sum_{j = 1}^{i}d ^ j}}{m 2 ^ {2 d ^ i - \sum_{j = 1}^{i}d^j}}\right) ^ d \\
	& = \frac{n\left(\alpha c\right) ^ {d + \sum_{j = 1}^{i} d ^ {j + 1}}}{2 ^ {2d ^ {i + 1} - \sum_{j = 0}^{i  -1} d ^ {j + 1} - 1}} = \frac{n \left(\alpha c\right) ^ {\sum_{j = 1}^{i + 1} d ^ j}}{2 ^ {2d ^ {i + 1} - \sum_{j = 0}^{i} d ^ j}} \textit{.}
\end{split}
\]

Because of the assumption that $\alpha c \leq 1$ we are able to state that $\beta_{i + 4} \leq \frac{n}{2 ^ {d ^ i}}$. The fact that $i^* \leq \frac{\ln \ln n}{\ln d} + 4$ follows from
\[
p_{i^*} = \left(\frac{c\beta_{i^*}}{m}\right) ^ d \leq \left(\frac{\alpha c}{2 ^ {d ^ {i^* - 4}}}\right) ^ d \leq \frac{1}{2 ^ {de^{\ln \ln n}}} = \frac{1}{2 ^ {d \ln n}} = \frac{1}{n ^ {d \ln 2}} < \frac{6 \ln n}{n}.
\]
\qed


\end{document}
