\section{Estimating the multiplicative constant}

\subsection{Integral}
When we were proving the upper bond on the length of the longest chain in a hash table we defined the constant $I$ as:
\begin{displaymath}
I = \displaystyle\int\limits_4^\infty 2 \left(\frac{r}{\log r}\right)^{-\log \left(\frac{r}{\log r}\right) - \log \log \left(\frac{r}{\log r}\right)}\textit{.}
\end{displaymath}

Now we will extend the definition of $I$ to $I_{\epsilon}$. This integral is one part of the multiplicative constant.
\begin{displaymath}
I_{\epsilon} = \frac{1}{1 - \epsilon} \displaystyle\int\limits_4^\infty \left(\frac{r}{\log r}\right)^{-\log \left(\frac{r}{\log r}\right) - \log \log \left(\frac{r}{\log r}\right)}\textit{.}
\end{displaymath}
Originally we used $I_{\frac{1}{2}} = I$. The extension is motivated by the fact that we were not forced to use $\epsilon$ equal to $0.5$. When we choose other value we can obtain even a smaller integral value. If we select an arbitrary $\epsilon$ the conditional probability of event $E_2 | E_1$ is constrained as $P(E_2 | E_1) \geq 1 - \epsilon$. And the bound on the probability of event $E_1$ becomes $P(E_1) = \frac{1}{1-\epsilon} P(E_2)$.

Evaluation of the integral $I$ may be split into two parts. We try to compare it to a function which has a convergent improper integral. The function chosen here is $r^{1.5}$. But the chosen function becomes greater after the value $r = 16$. In the interval $[4, 16]$ the integral $I$ is bounded by its upper Riemann sum.

For $r = 16$:
\begin{displaymath}
\frac{1}{16 ^ {1.5}} = \frac{1}{64}
\end{displaymath}

\begin{displaymath}
\left(\frac{16}{\log 16}\right)^{-\log \left(\frac{16}{\log 16}\right) - \log \log \left(\frac{16}{\log 16}\right)} = 4^{-2 - 1} = \frac{1}{64}
\end{displaymath}

For $r > 16$ we can use our estimates.
\begin{displaymath}
\begin{split}
I_{\epsilon} 
	& \leq \frac{1}{1 - \epsilon} \left( \displaystyle \sum_{r = 4}^{15} \left(\frac{r}{\log r}\right)^{-\log \left(\frac{r}{\log r}\right) - \log \log \left(\frac{r}{\log r}\right)} + \int\limits_{16}^\infty \frac{1}{r^{1.5}} dr \right) \\
	& \leq \frac{1}{1 - \epsilon} \left(4.3 + \frac{1}{2}\right) = \frac{4.8}{1-\epsilon}
\end{split}
\end{displaymath}

\subsection{Choosing $\epsilon$}
The most important step is the optimization of the value $4 c_\epsilon (4 + I_{\epsilon})$. This is the explicit formula for the multiplicative constant. This value is less then $22 568$ for $\epsilon = 0.91$, $k = 3.28$ and $l = 0.5$. These values come from a slight modification of the theorem \ref{theorem-set-onto-by-linear-transform} and we will explain them later. Though the asymptotic rate of the growth is $O(\log n \log \log n)$ with this large multiplicative constant it becomes less than linear for $n$ approximately equal to 1 000 000.

The first approach is to modify the proof of the theorem \ref{theorem-set-onto-by-linear-transform}. We try to parameterize every constant in it. Then we will optimize these parameters so that we get the least $c_{\epsilon}$. The optimization itself has not been performed analytically because of the complexity of the constraints. We created a straightforward program that assigns each parameter a value from a predefined interval. Then it computes the multiplicative constant and the best achieved value is remembered and used.

We actually created two parameter $k$ and $l$. The limit of the size of the set $T_0(A)$ is changed to $\frac{|A|}{k}$, for $k > 2$. The second parameter is obtained by modifying the dimension of the factor vector space $Z_2^u$. We change the definition of $u$ to:
\begin{displaymath}
u = \left\lceil \log \left(\frac{2^l |A|}{\epsilon}\right) \right\rceil \textit{.}
\end{displaymath}

The probability of the event $T(A) \neq Z_2^t$ can be expressed by using law of total probability as:
\begin{displaymath}
\begin{split}
P(T(A) \neq Z_2^t) 
    & = P(T(A) \neq Z_2^t \wedge |T_0(A)| \leq \frac{|A|}{k}) + P(T(A) \neq Z_2^t \wedge |T_0(A)| > \frac{|A|}{k}) \\ 
    & \leq P(|T_0(A)| \leq \frac{|A|}{k}) + P(T_1(T_0(A)) \neq Z_2^t \wedge |T_0(A)| > \frac{|A|}{k}) \\
    & \leq \epsilon \\
\end{split}
\end{displaymath}
The right side, $\epsilon$, is the needed result which we must obtain by choosing the convenient $c_{\epsilon}$. Also the estimate of $c_{\epsilon}$ was modified comparing to the original proof. By putting $c_{\epsilon}$ equal to $4\left(\frac{2}{\epsilon}\right)^{\frac{8}{\epsilon}}$ we could not get a good result. So we compute the constant $c_{\epsilon}$ directly without any estimations.

\begin{lemma}
When the size of the set $|T_0(A)|$ is less than $\frac{|A|}{k}$ for $1 \leq k$ then there are at least $\frac{|A|(k - 1)}{2}$ collisions.
\end{lemma} 
\begin{proof}
Define the sequence $b_i$ for $i \in T_0(A)$ where $b_i = \left|A \cap T_0^{-1}(i)\right|$. Also note that $\sum_{i \in T_0(A)} b_i = |A|$.
The number of all colliding pairs can be computed as follows.
\begin{displaymath}
\frac{1}{2} \sum_{i \in T_0(A)} b_i (b_i - 1) \geq \frac{|A|}{2}\left(\frac{|A|}{|T_0(A)|} - 1\right) \geq \frac{|A|(k - 1)}{2}
\end{displaymath}
The last inequality can be obtained from Cauchy–Bunyakovsky–Schwarz inequality.
\end{proof}

\begin{displaymath}
\begin{split}
P(|T_0(A)| \leq \frac{|A|}{k}) & + P(T_1(T_0(A)) \neq Z_2^t \wedge |T_0(A)| > \frac{|A|}{k}) \\ 
& \leq \frac{|A| - 1}{(k - 1)|W|} + \left(1 - \frac{|A|}{k|W|}\right)^{u - t - \log t + \log \log \left(\frac{1}{1 - \frac{|A|}{k|W|}}\right)} \\
& < \frac{\epsilon}{(k - 1) 2 ^ l} + \left(1 - \frac{\epsilon}{2 k 2^l}\right)^{\log c_\epsilon + l - \log \epsilon + \log \log \left(\frac{1}{1 - \frac{\epsilon}{2 k 2^l}}\right)} \\
& \leq \epsilon \\
\end{split}
\end{displaymath}

For convenience we define a new variable $\alpha'$ equal to $1 - \frac{\epsilon}{2 k 2 ^l}$. From the above inequality we need to find the minimal value of $c_\epsilon$:
\begin{displaymath}
\begin{split}
\frac{\epsilon}{(k - 1) 2 ^ l} + {\alpha'}^{\log c_\epsilon + l - \log \epsilon + \log \log \left(\frac{1}{\alpha'}\right)} & \leq \epsilon \\
{\alpha'}^{\log c_\epsilon}{\alpha'}^{l - \log \epsilon + \log \log \left(\frac{1}{\alpha'}\right)} & \leq \epsilon - \frac{\epsilon}{(k - 1) 2 ^ l} \\
{\alpha'}^{\log c_\epsilon} & \leq \left(\epsilon - \frac{\epsilon}{(k - 1) 2 ^ l}\right) {\alpha'}^{\log \epsilon - l - \log \log \left(\frac{1}{\alpha'}\right)} \\
{\log c_\epsilon} & \geq \frac{\log \left( \left( \epsilon - \frac{\epsilon}{(k - 1) 2 ^ l}\right) {\alpha'}^{\log \epsilon - l - \log \log \left(\frac{1}{\alpha'}\right)}\right)}{\log \alpha'}  \\
\end{split}
\end{displaymath}


\section{Hashing a linear amount of elements}
A common use of a hash table is with load factor lower than one. We already showed if there is a super-linear amount of elements hashed we can expect a reasonable result in the worst case. However the most important is the expected size of a bucket. The upper bound of the expected size is equal to $1 + c \alpha$. So using a load factor lower than one has a significant impact on the average case. We need to examine a scheme of hashing $\alpha_f n$ elements into a hash table of size $n$, so $alpha_f$ is the table's load factor. 

We need to change the previous claims to fit our new scheme: 
\begin{theorem}
\label{theorem-n-to-n}
When hashing $\alpha_f n$ elements into a table of size $n$ the expected length of the longest chain is bounded by $O(\alpha_f \log n \log \log n)$.
\end{theorem}
\begin{proof}
We have to modify the previous lemmas and their proofs from. Apparently we must lower the value of a chain length when we get a convenient probability estimate proportionally to $\alpha_f$. 

\begin{remark}
There is a constant $C$ so that for all $r > 4$, $S$ is the hashed set ($S \subset D$, $|S| = \alpha_f n$) and $B = Z_2^{\log n}$ is the hash table the probability of existence of a long chain is:
\begin{displaymath}
P(lpsl > r \alpha_f C \log n \log \log n) \leq 2 \left(\frac{r}{\log r}\right)^{-\log \left(\frac{r}{\log r}\right) - \log \log \left(\frac{r}{\log r}\right)}\textit{.}
\end{displaymath}
\end{remark}
\begin{proof}
We show changes that need to be made. 
\begin{displaymath}
\begin{split}
l & = \left\lfloor \log n + \log \log n + \log r - \log \log r + \log \alpha_f \right\rfloor \\
t & = 4\alpha_f c_{\frac{1}{2}} \log n \log \log n \\
d & = \frac{2^l}{\alpha_f n \log n} \\
\end{split}
\end{displaymath}

The assumptions of the \ref{remark-e2-probability}\textsuperscript{th} remark are met. There is the factor $\alpha_f$ in the denominator because the size of the hashed set is certainly limited by $\alpha_f n \leq \alpha_f n \log n$:
\begin{displaymath}
d = \frac{2^l}{\alpha_f n \log n} \geq \frac{\alpha_f n \log n \log r}{\alpha_f n \log n \log \log r} > 1 \textit{.}
\end{displaymath}

In the proof of the \ref{remark-e2-probability} we also used a relation among $h_1(S)$, $A = Z_2^l$, $\alpha$ and $d$. We present here that simalar holds and this makes it valid for this scheme, too.
\begin{displaymath}
\alpha = \frac{|h_1(S)|}{|A|}\leq \frac{|S|}{|A|} = \frac{\alpha_f n}{2^l} \leq \frac{\alpha_f n \log n}{2^l} = \frac{1}{d} \textit{.}
\end{displaymath}

For using the other lemma the value of the variable $t$ has to be large enough:
\begin{displaymath}
\begin{split}
c_{\frac{1}{2}} \frac{2^l}{n} \log \left(\frac{2^l}{n}\right) 
	& < c_{\frac{1}{2}} 2 \alpha_f \log n \left( \frac{r}{\log r} \right) \left(2 \log \log n \log r \right) \\
	& \leq 4 c_{\frac{1}{2}} \alpha_f \log n \log \log n \log r \\
	& = t
\end{split}
\end{displaymath}

The conditions of the remark \ref{remark-prob-t-length-chain} are fulfilled. Its proof is still right, since we use it with no further assumptions. The same vector space $A = Z_2^l$ is constructed; we never referenced the value of $l$. The vector space $B$, the hash table, is unchanged.

We continue identically to the previous case:
\begin{displaymath}
P(E1) \leq \frac{1}{P(E2|E1)}P(E2) \leq 2 d ^ {-\log d - \log \log d}\textit{.}
\end{displaymath}
\end{proof}

In order to achieve the desired length we modify the calculation by taking $K = C \alpha_f \log n \log \log n$.
\begin{displaymath}
\begin{split}
E lpsl 
	& = \int\limits_0^\infty P(lpspl > t) dt \\
	& \leq 4K + \int\limits_{4K}^\infty P(lpspl > t) dt \\
	& = 4K + K\int\limits_4^\infty P(lpspl > tK) dt \\
	& \leq K(4 + I) = O(K) = O(\alpha_f \log n \log \log n)
\end{split}
\end{displaymath}

Expression $I$ has not been modified.
\end{proof}