\section{Minimising the Integrals}
\label{section-integral-estimate}
\begin{lemma}
\label{lemma-convergent-I-e}
The integral $I_\epsilon$ from Theorem \ref{theorem-n-logn-to-n} converges for every $\epsilon \in (0, 1)$ and
\[
I_\epsilon \leq \frac{4.8}{1 - \epsilon} \text{.}
\]
\end{lemma}
\begin{proof}
We recall that the integral $I_\epsilon$ defined in Equality \ref{equality-i-e} equals
\[
I_{\epsilon} = \frac{1}{1 - \epsilon} \displaystyle\int\limits_4^\infty \left(\frac{r}{\log r}\right)^{-\log \left(\frac{r}{\log r}\right) - \log \log \left(\frac{r}{\log r}\right)} \text{.}
\]

This integral is a part of the multiplicative constant $16c_\epsilon(4 + I_\epsilon)$. This lemma states a a way how to estimate $I_\epsilon$ and compute the value of the multiplicative constant so that it is minimal.

In the original proof of Theorem \ref{theorem-n-logn-to-n} in \cite{DBLP:journals/jacm/AlonDMPT99} only the special case $I_{\frac{1}{2}}$ is considered. We moved to $I_\epsilon$ because choosing other the values of $\epsilon \in (0, 1)$ may bring an improvement over the multiplicative constant. Indeed, if we select different value of $\epsilon$ and use better estimates for the constant $c_\epsilon$, then we really obtain a smaller value of the multiplicative constant and a tighter bound on $\Expect{\lpsl}$.

Evaluation of the integral $I_\epsilon$ is split into two parts. We compare the integrand of $I_\epsilon$ to a function $f: \mathbb{R} \rightarrow \mathbb{R}$ which has a convergent improper integral $\int_{4}^{\infty} f(x) dx$. The function $f$ chosen here is $x ^ {-1.5}$. 

The function $f$ majors the integrand only for $r \geq 16$. Therefore we bound the value of the integral $I_\epsilon$ in the interval $[4, 16]$ by its upper Riemann sum.

For $r = 16$, the function $f$ equals $f(r) = r ^ {-1.5} = \frac{1}{64}$.
The value of the integrand equals
\[
	\left(\frac{16}{\log 16}\right)^{-\log \left(\frac{16}{\log 16}\right) - \log \log \left(\frac{16}{\log 16}\right)} = 4^{-2 - 1} = \frac{1}{64} \text{.}
\]

By combining our estimates we obtain that
\[
\begin{split}
I_{\epsilon} 
	& \leq \frac{1}{1 - \epsilon} \left( \displaystyle \sum_{r = 4}^{15} \left(\frac{r}{\log r}\right)^{-\log \left(\frac{r}{\log r}\right) - \log \log \left(\frac{r}{\log r}\right)} + \int\limits_{16}^\infty \frac{1}{r^{1.5}} dr \right) \\
	& \leq \frac{1}{1 - \epsilon} \left(4.3 + \frac{1}{2}\right) = \frac{4.8}{1-\epsilon} \text{.}
\end{split}
\]

The estimate of the Riemann sum can be computed numerically.
\end{proof}

\begin{lemma}
\label{lemma-estimate-j-a}
Let $b \in \mathbb{N}$, $b \geq 4$, $\alpha \in [0.5, 1]$, $J'_{\alpha, b}$ be the integral defined by Equality \ref{equality-j-prime-a-b} and $J_{\alpha, b}$ be the integral defined by Equality \ref{equality-j-a-b}. Then 
\[
	J'_{\alpha, b} \leq 2b \log b - 4 + J_{\alpha, b} \text{ and}
\]
\[
	J_{\alpha, b} \leq 3.36 \text{.}
\]
\end{lemma}
\begin{proof}
We estimate the integral \[ J'_{\alpha, b} = \int\limits_{4}^\infty \min \left(1, \left(\frac{r}{\log r}\right)^{\log b - \log \alpha - \log \left(\frac{r}{\log r}\right) - \log \log \left(\frac{r}{\log r}\right)}\right) dr \] in a similar way as the previous one. 

However the situation is slightly complicated. First we must realise that if the exponent is negative the integrand is certainly less than one. We state that whenever $r > 2 b \log b$, then the exponent is negative.

First we show that $r > 2 b \log b$ implies $\log \left( \frac{r}{\log r} \right) \geq \log b$ since
\[
\begin{split}
\frac{r}{\log r} 
	& \geq \frac{2 b \log b}{1 + \log b + \log \log b} \\
	& = \frac{2 b}{1 + \frac{1}{\log b} + \frac{\log \log b}{\log b}} \\
	& \geq \frac{2 b}{2} = b \text{.}
\end{split}
\]

Since we assume $b \geq 4$, we have that $r > 2 b \log b \geq 16$. So if $r \geq 2b \log b$, then for the remaining part of the exponent we have that \[ -\log \alpha - \log \log \left( \frac{r}{\log r} \right) < -\log 0.5 - \log \log \frac{16}{4} = 0 \text{.} \]

Indeed, if $r > 2 b \log b$, then the exponent is negative.

Now the integral $J'_{\alpha, b}$ is split into two parts according to the value of the variable $r$. If $r \in [4, 2b \log b]$ then we estimate the integrand by one. In the interval $[2b \log b, \infty)$ we use the value $J_{\alpha, b}$. This value is determined in the similar fashion as the value of the integral $I_\epsilon$ in Lemma \ref{lemma-convergent-I-e}. So
\[
J'_{\alpha, b} \leq 2b \log b - 4 + J_{\alpha, b} \text{.}
\]

The integrand is majored by the function $x ^ {-1.3}$ for $x \geq 2048$. In the interval $[16, 2048]$ we estimate the integral by its upper Riemann sum. We partition the interval uniformly with the norm equal to $0.1$. For simplicity put $d(r) = \frac{r}{\log r}$ and estimate the integral $J_{\alpha, b}$ as
\[
\begin{split}
J_{\alpha, b} 
	& = \int_{2 b \log b}^{\infty} d(r) ^ {\log b - \log \alpha - \log d(r) - \log \log d(r)} dr \\
	& \leq \int\limits_{16}^{\infty} d(r) ^ {1 - \log \log d(r)} dr \\
	& \leq \sum\displaylimits_{r \in \{16, 16.1, \dots, 2048 \}} \frac{d(r) ^ {1 - \log \log d(r)}}{10} + 
		\int\limits_{2048}^{\infty} r ^ {-1.3} dr \\
	& \leq 3.01 + \frac{2048 ^ {-0.3}}{0.3} \leq 3.36 \text{.}
\end{split}
\]
\end{proof}