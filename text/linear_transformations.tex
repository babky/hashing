\chapter{The system of linear maps between vector spaces}
Finally we concentrate on the systems of linear transformations between vector spaces over the field $\mathbb{Z}_2$. The reason is simple, they allow construction of a model of universal hashing in which the expected amortised time of all operations can be guaranteed without any assumptions on the hashed set $S$. 

Main result regarding systems of linear maps is the~upper bound on the~length of the~longest probe sequence. Proofs of this property does not come directly from the~idea showed in the previous chapter. The~approach used does not try to compute the~probability of collision of $k$-elements directly and does not exploit attributes of strongly universal systems. We concentrate and work with special properties of this unique system instead.

In the following sections we restrict ourselves only to vector spaces over the field $\mathbb{Z}_2$. The shown theorems also suppose hashing $n \log n$ elements into a~hash table of size $n$ slots which is quite novel. However the results may be used for hashing only $n$ elements they are later improved for hashing such small sets. Under these conditions one very important property of this model can be shown; length of the~longest chain is bounded by a function contained in class $O(\log n \log \log n)$. If vector spaces used are over other finite fields we can not expect such good properties as shown in \cite{DBLP:journals/jacm/AlonDMPT99}.

\begin{section}{Models of uniform choice of a linear map}
Technical preparations also include some models which show how uniform choice of a random linear map can be performed. These models are later used to simplify some proofs assuming uniform choice of linear transformation. Instead of simply choosing linear map we select other objects that uniquely define a linear function and work with them. When uniformly performing mentioned selections we obtain uniform selection of suitable linear function.

Definition $\ref{definition-system-of-linear-transformations}$ of systems of linear maps can be extended to denote not only the sets of all linear maps but all surjective linear maps as well. This notation is especially useful when considering various models of choice of (surjective) linear transformation.
\begin{definition}
Let $A$ and $B$ be two vector spaces. Define the set of all linear transformations
\[
LT(A, B) = \{ T: A \rightarrow B \setdelim T \text{ linear transformation} \}
\]
and
\[
LTS(A, B) = \{ T: A \rightarrow B \setdelim T \text{ surjective linear transformation} \}
\] set of all surjective linear transformations between vector spaces $A$ and $B$.
\end{definition} 

\begin{definition}[Uniform selection model]
Let $A$ be a set such that $A = \{ a_1, \dots, a_n \}$. By random uniform selection of element $a \in A$ we understand a model of selection where 
\[
	\Prob{a \in A \text{ was selected}} = \frac{1}{n} \text{.}
\]
\end{definition}

The first model depicts correspondence between uniform choice of a basis of the source space that is mapped onto a fixed basis of the target space and uniform choice of surjective linear transformation.
\begin{remark}[Surjective linear map selection]
\label{remark-model-surjective-linear-map-selection}
Let $\mathcal{B}$ be a set of all bases of vector space $\mathbb{Z}_2^u$ and $\{y_1, \dots, y_t\}$ be a basis of vector space $\mathbb{Z}_2^t$ where $t \leq u$ and let vectors in bases be lexicographically ordered. Define \[\mathcal{S} = \{\{a_1, \dots, a_{u - t}\} \setdelim 1 \leq a_1 < \dots < a_{u-t} \leq u \} \text{.} \] For a random uniform choice of basis $b = \{b_1, \dots, b_u\} \in \mathcal{B}$, $s \in \mathcal{S}$ and permutation $\pi \in \Pi_t$ define linear map $T_{b, s, \pi}$ as
\[
T_{b, s, \pi}(b_i) =  
  \begin{cases} 
    y_{\pi(i)} & \text{if } i \notin s \\
    0 & \text{if } i \in s
  \end{cases} \text{.}
\]
If we perform these random uniform choices of $b, s$ and $\pi$ we obtain randomly and uniformly selected surjective linear map $T_{b, s, \pi}$.
\end{remark}
\begin{proof}
First remark that $T_{b, s, \pi}$ is a surjective linear map for every choice of $b \in \mathcal{B}$ and $s \in \mathcal{S}$. From the fact that $T_{b, s, \pi}$ is defined for every vector of the basis $b$ we known it may be uniquely extended to a linear map. It is surjective since for every vector $y_i$, $1 \leq i \leq t$ there is a vector $b_j$, $1 \leq j \leq u$ such that $T_{b, s, \pi}(b_j) = y_i$. 

Now we show that for every surjective linear transformation $T$ there is a choice $b, s$ and $\pi$ such that $T_{b, s, \pi} = T$. Consider set $T^{-1}(0)$, it certainly contains $t$ linearly independent vectors, denote them as $B_0$. Now consider set $\mathbb{Z}_2^u - T^{-1}(0)$, it must contain $u - t$ linearly independent vectors $c_1, \dots, c_{u-t}$ such that $T(c_i) \neq T(c_j)$ for $1 \leq i \neq j \leq u - t$. Let $B_1 = \{c_1, \dots, c_{u - t}\}$ denote them. A basis $b$ can be formed from $B_0 \cup B_1$. We can select vectors of $b$ by using $s$ so that we obtain $B_1$. They must permuted by $\pi$ to get $T$. 

It must be proved that for each pair of functions $T_{b_1, s_1, \pi_1}$ and $T_{b_2, s_2, \pi_2}$ there is the same number of choices generating it. Consider the identity isomorphism $id_{b_1, b_2}$ of vector space $\mathbb{Z}_2^u$ mapping $i$-th vector of base $b_1$ onto $i$-th vector of base $b_2$. Every choice generating $T_{b_1, s_1, \pi_1}$ can be uniquely transformed by $id_{b_1, b_2}$ to a choice generating $T_{b_2, s_2, \pi_2}$.
\end{proof}

\begin{remark}
\label{remark-model-uniform-linear-map-selection}
Let $t, u, w$ be natural natural numbers such that $t \leq l$. For a random uniform choice of a linear transformation $T_0: \mathbb{Z}_2^w \rightarrow \mathbb{Z}_2^u$ among $LT(\mathbb{Z}_2^w, \mathbb{Z}_2^u)$ and $T_1: \mathbb{Z}_2^u \rightarrow \mathbb{Z}_2^t$ among $LTS(\mathbb{Z}_2^u, \mathbb{Z}_2^t)$ we obtain random uniform choice of linear transformation $T = T_1 \circ T_0$ among $LT(\mathbb{Z}_2^w, \mathbb{Z}_2^t)$.
\end{remark}
\begin{proof}
The idea of the proof is generally the same as in the previous remark. We show that every linear map $T$ can be generated by a constant number of choices $T_0, T_1$ and every map $T$ can be generated.

First notice that for every $T$ and $T_1$ there is the same number of linear maps $T_0$ such that $T_1 \circ T_0 = T$. Let $M, M_1, M_2$ denote matrices of linear functions $T, T_0, T_1$. Since $T_1$ is surjective $\rank(M_2) = t$ therefore solution $x$ of system of linear equations $M_2 x = y$ certainly exists. Moreover from the standard linear algebra we know that the number of all such solutions remains the same for every vector $y$ and matrix $M_2$ such that $\rank(M_2) = t$ and is equal to $|\matrixkernel{M_2}|$. Thus the number of matrix solutions $M_1$ of system $M_2 M_1 = M$ where $M_2$ and $M$ are given remains the same. Also note that all the solutions are different.

Now we need to estimate how many surjective pairs $T_1 \in LTS(\mathbb{Z}_2^u, \mathbb{Z}_2^t)$ can generate given mapping $T: \mathbb{Z}_2^w \rightarrow \mathbb{Z}_2^t$. Let $M, M'$ denote matrices of two different linear transformations $T, T' \in LT(\vecspace{w}, \vecspace{t})$. Matrix $M$ can be uniquely transformed to matrix $M'$ by an automorphism $id_{M, M'}$ defined by matrix $L$.
\[
	M' = LM
\]
If $T_0, T_1$ generates $T$ and $T_0', T_1'$ generates $T'$ we can substitute into previous.
\[
	M_2' M_1' = L M_2 M_1
\]

Every pair $T_0, T_1$ generating $T$ can be uniquely mapped by $id_{M, M'}$ onto choice $T_0', T_1'$ generating $T'$. Bijection $id_{M, M'}$ transforms $T_1$ to $T_1'$ and vice-versa. From the previous notice for each function $T_1$ there is exactly one transformation $T_0$ leading to $T$.

Finally for every linear map $T$ there is constant number of pairs $T_0, T_1$ leading to $T$. Because of uniformity of selection the probability of choosing pair $T_0, T_1$ leading to $T = T_1 \circ T_0$ is constant for every map $T$.
\end{proof}
\end{section}

\begin{section}{Probabilistic properties of the system of linear maps}

Most of the~following claims in this section are taken from \cite{DBLP:journals/jacm/AlonDMPT99}. It is convenient to show the~original proofs and then modify them according to our future needs. Some technical definitions and statements follow that are useful in order to show our goal.

\begin{definition}
Let $V$ be a vector space and $A$ be a subset of $V$. For a vector $v \in V$ define set $v + A$ as
\[ v + A = \{ v + a \setdelim a \in A \} \text{.} \] 
\end{definition}

\begin{definition}
Let $V$ be a vector space and $A, B$ be a subsets of $V$. Define set $A + B$ as
\[ A + B = \{ a + b \setdelim a \in A, b \in B \} \text{.} \] 
\end{definition}

\begin{lemma}
\label{lemma-choose-random-vector}
Let $V$ be a~finite vector space and $A$ be its subset. Define $\mu = 1 - \frac{|A|}{|V|}$ as inverse density of set $A$ in vector space $V$. Let $v \in V$ be a random uniformly chosen vector independent of $A$. Then
\begin{displaymath}
\Expect{1 - \frac{|A \cup (v + A)|}{|V|}} = \mu^2
\end{displaymath}
where the~expectation is taken throughout all possible choices of $v \in V$.

\begin{proof}
To simplify further computations define $X_v = |A \cup (v + A)|$ as a random variable taken throughout random uniform choice of vector $v \in V$. The most difficult part of the proof is to compute $\Expect{X_v}$.
\[
\Expect{X_v} = \displaystyle\sum_{v \in V} |A \cup (v + A)| . \Prob{v \text{ is chosen}} = \displaystyle\sum_{v \in V} \frac{|A \cup (v + A)|}{|V|}
\]

Size of set $|A \cup (v + A)|$ can be expressed using indicator function.
\[
|A \cup (v + A)| = \displaystyle\sum_{u \in V} I(u \in A \vee u \in (v + A)) \\
\]

Now notice if $u \in A$ there are exactly $|V|$ vectors that satisfy the above condition. For $u \notin A$ and $a \in A$ there is exactly one vector $v \in V$ such that $a + v = u$ and there are $|A|(|V| - |A|)$ such possibilities and the remaining choices are refused. Thus
\[ 
\begin{split}
|\{(u, v) \setdelim u \in A \vee u \in (v + A), u, v \in V \}| 
	& = |A|(|V| - |A|) + |V||A| \\
	& = 2|V||A| - |A|^2 \text{.} \\
\end{split}
\]

Substituting into the definition of $\Expect{X_v}$ and rewriting sums into the just computed size of a suitable set.
\[
\begin{split}
\Expect{X_v} 
	& = \frac{\sum_{v \in V} \sum_{u \in V} I(u \in A \vee u \in (v + A))}{|V|}  \\
	& = \frac{|\{(u, v) \setdelim u \in A \vee u \in (v + A), u, v \in V \}|}{|V|} \\ 
	& = \frac{2|V||A| - |A|^2}{|V|} \\
\end{split}
\]

Now finally compute the wanted expected value.
\[
\begin{split}
\Expect{1 - \frac{|A \cup (v + A)|}{|V|}} 
	& = 1 - \frac{\Expect{|A \cup (v + A)|}}{|V|}  \\
	& = 1 - \frac{\Expect{X_v}}{|V|} \\
	& = 1 - \frac{2|V||A| - |A|^2}{|V|^2} \\
	& = \frac{|V|^2 + 2|V||A| - |A|^2}{|V|^2} \\
	& = \left(1 - \frac{|A|}{|V|}\right)^2 = \mu^2 \\
\end{split}
\]
\end{proof}
\end{lemma}

The following lemma is a~very technical one and is used to estimate the~probabilities of some events related to linear maps.
\begin{lemma}
\label{lemma-random-variable}
For $1 \leq i \leq k$ the~$\mu_i$ are random variables and $0 < \mu_0 < 1$ is a~constant. For the~random variables, $1 \leq i \leq k$, we assume the~following:
\begin{gather*}
0 \leq \mu_i \leq \mu_{i - 1} \\
E[ \mu_i | \mu_{i-1} \dots \mu_1 ] = \mu_{i-1}^{2} \\
\end{gather*}
Then for every constant $0 < t < 1$ we can estimate the~probability:
\begin{displaymath}
P(\mu_k \geq t) \leq \mu_0^{k - \log \log (\frac{1}{t}) + \log \log \left(\frac{1}{\mu_0}\right)}
\end{displaymath}
\end{lemma}
\begin{proof}
We prove the statement by induction over $k$. 

\paragraph*{The initial step, $k = 0$.}
Since $\mu_0$ is constant we have
\[
	\Prob{\mu_0 \geq t} = \begin{cases}
		0 & \text{ if } \mu_0 < t \\
		1 & \text{ if } \mu_0 \geq t \text{.} \\
	\end{cases}
\]

From $\mu_0 > 0$ it follows
\[
	\mu_0^{0 - \log \log \left(\frac{1}{t}\right) + \log \log \left(\frac{1}{\mu_0}\right)} \geq 0
\]
and thus the estimate holds for $\mu_0 < t$.

If $1 > \mu_0 \geq t$ then $-\log \log \left(\frac{1}{t}\right) + \log \log \left(\frac{1}{\mu_0}\right) \leq 0$ and hence
\[
	\mu_0^{0 - \log \log \left(\frac{1}{t}\right) + \log \log \left(\frac{1}{\mu_0}\right)} \geq 1 \text{.}
\]

Thus the statement holds for $k = 0$.

\paragraph*{The induction step.} We prove the statement holds for $k \geq 0$ then it holds for $k + 1$. Let $t \in (0, 1)$ be fixed. For simplicity, let us denote $c = k - \log \log \left(\frac{1}{t}\right)$. Then we have to prove
\[
	\Prob{\mu_{k + 1} \geq t} \leq \mu_0 ^ {c + 1 + \log \log \left(\frac{1}{\mu_0}\right)} \text{.}
\]
Whenever exponent $c + 1 + \log \log \left(\frac{1}{\mu_0}\right) \leq 0$ the estimate holds, because $\mu_0 < 1$. We can restrict ourselves to case when $c + 1 + \log \log \left(\frac{1}{\mu_0}\right) > 0$. To prove our statement we fix $\mu_1$ and we the induction hypothesis for $k$. For value $a \in \left[0, \mu_0\right]$ define $g(a) = \Prob{\mu_{k + 1} \geq t | \mu_1 = a}$. Then
\[
	\Prob{\mu_{k + 1} \geq t} = \int\limits_{0}^{\mu_0}\Prob{\mu_{k + 1} \geq t | \mu_1 = a}\Prob{\mu_1 = a} \, da = \Expect{g(\mu_1)} \text{.}
\]

Functions $f$ and $f_0$ are defined as
\[ 
	f_0(x) = \begin{cases}
		x ^ {c + \log \log \left(\frac{1}{x}\right)} & \text{ if } 0 < x < 1 \\ 
		0 & \text{ if } x = 0 \\
	\end{cases}
\]
and $f(x) = \min \{1, f_0(x) \}$ for $0 \leq x < 1$.

If $\beta_0 = a \in \left[0, \mu_0 \right]$ is constant and $\beta_i$ for $i = 1, \dots, k$ are random variables satisfying conditions of this lemma then $\Prob{\beta_k \geq t} = g(a)$. From the induction hypothesis for $k$ we have $g(a) \leq f(a) = a^{k - \log \log \left(\frac{1}{t}\right) + \log \log \left(\frac{1}{a}\right)}$ for every $a \in \left(0, \mu_0 \right]$.

Next we investigate behaviour of $\frac{f_0}{x}$ on the interval $(0, 1)$. We start to compute the first derivation for every $x \in (0, 1)$.
\[
\begin{split}
\left(\frac{f_0(x)}{x}\right)' 
	& = \frac{xf_0(x)\left[\ln(x)\left(c + \log \log \left(\frac{1}{x}\right)\right)\right]' - f_0(x)}{x^2} \\
	& = \frac{f_0(x)\left[c + \log \log \left(\frac{1}{x}\right) + x \cdot \frac{\ln x}{\log\left(\frac{1}{x}\right) \ln 2}\cdot\frac{x}{\ln 2}\cdot\frac{-1}{x ^ 2} \right] - f_0(x)}{x^2} \\
	& = \left(c - 1 + \log \log \left( \frac{1}{x} \right) + \log e \right)\frac{f_0(x)}{x^2} \\
\end{split}
\]

Define the stationary point $x_s = 2 ^ {-2 ^ {-c + 1 - \log e}}$. Function $\frac{f_0(x)}{x}$ is then increasing in the interval $(0, x_s)$ and decreasing in the interval $(x_s, 1)$. 

Let us also define $x_1 = 2 ^ {-2 ^ {-c}}$, the point where $f_0(x)$ first reaches 1.
\[
f_0(x_1) = {x_1} ^ {c + \log \log \left(\frac{1}{x_1}\right)} = {x_1} ^ {c - c} = 1
\]
Since $-c > -c + 1 -\log e$ inequality $x_1 < x_s$ holds and the investigated function $\frac{f_0(x)}{x}$ is still increasing in the point $x_1$.
Function $f_0(x)$ is increasing for every $x \in \left[x_1, 1\right)$. Let $x' < x$.
\[
\begin{split}
f_0(x') 
	& = {x'} ^ {c + \log \log \left(\frac{1}{x'}\right)} \\
	& < {x} ^ {c + \log \log \left(\frac{1}{x'}\right)} \\
	& < {x} ^ {c + \log \log \left(\frac{1}{x}\right)} \\
\end{split}
\]

Thus we have $f_0(x) \geq f_0(x_1) = 1$ and $f(x) = \min \{1, f_0(x)\} = 1$.

At last define the third point $x_2 = x_1 ^ 2 = 2 ^ {-2 ^ {-c - 1}}$. This point lies in the decreasing phase because $-c - 1 < -c + 1 - \log e$ and then $x_2 > x_s$. In the just defined point $x_2$ the following statement holds.
\[
\begin{split}
\frac{f_0(x_2)}{x_2} 
	& = \frac{\left(2 ^ {-2 ^ {-c - 1}}\right) ^ {c + \log \left(- \log \left(2 ^ {-2 ^ {-c - 1}}\right)\right)}}{2 ^ {-2 ^ {-c - 1}}} \\
	& = \frac{\left(2 ^ {-2 ^ {-c - 1}}\right) ^ {c + \log \left(2 ^ {-c - 1}\right)}}{2 ^ {-2 ^ {-c - 1}}} \\
	& = \frac{\left(2 ^ {-2 ^ {-c - 1}}\right) ^ {-1}}{2 ^ {-2 ^ {-c - 1}}} \\
	& = \frac{1}{x_2^2} = \frac{1}{x_1}
\end{split}
\]

The proof is divided into three cases. The first and the second case take care about the situation when the exponent is non-negative, $c + 1 + \log \log \left(\frac{1}{\mu_0}\right) \geq 0$. In both cases it is proved $f(x) \leq \frac{f_0(\mu_0)x}{\mu_0}$ for every $x \in (0, \mu_0]$.
\paragraph{Constant $\mu_0$ is in the increasing phase, $\mu_0 \leq x_s$.}
Function $\frac{f(x)}{x}$ is increasing in this phase hence
\[
f(x) = \frac{f(x)x}{x} \leq \frac{f_0(x)x}{x} \leq \frac{f_0(\mu_0)x}{\mu_0} \text{.}
\]

\paragraph{Constant $\mu_0$ is in the decreasing phase, $x_s \leq \mu_0 \leq x_2$.}
For $x \in (0, x_1]$ following holds because we are still in the increasing phase.
\[
\frac{f(x)}{x} \leq \frac{f_0(x)}{x} \leq \frac{f_0(x_1)}{x_1} = \frac{1}{x_1}
\]

Function $f(x)$ is then in the interval $(0, x_1]$ bounded by \[f(x) = \frac{f(x)x}{x} \leq \frac{x}{x_1} \text{.} \]

For every $x \in [x_1, \mu_0]$ 
\[ 
	f(x) = 1 = \frac{x}{x} \leq \frac{x}{x_1}
\]

Using the fact $\frac{1}{x_1} = \frac{f_0(x_2)}{x_2}$ it is clear that 
\[
	f(x) \leq \frac{x}{x_1} = \frac{f_0(x_2)x}{x_2} \leq \frac{f_0(\mu_0)x}{\mu_0}
\]
because both $x_2 > \mu_0$ are already in the decreasing phase.

In both first two cases we showed $f(x) \leq \frac{f_0(\mu_0)x}{\mu_0}$ for every $x \in (0, \mu_0]$. Now using this statement we prove the lemma.
\[
\begin{split}
\Prob{\mu_{k + 1} \geq t}
	& = \Expect{g(\mu_1)} \leq \Expect{f(\mu_1)} \leq \Expect{\frac{f_0(\mu_0)\mu_1}{\mu_0}} = \frac{f_0(\mu_0)}{\mu_0}\Expect{\mu_1|\mu_0} \\
	& = \frac{f_0(\mu_0)}{\mu_0}\mu_0 ^ 2 = \mu_0 f_0(\mu_0) = {\mu_0}^{c + 1 + \log \log \left(\frac{1}{\mu_0}\right)}
\end{split}
\]

\paragraph{Constant $\mu_0$ is set so that exponent was negative, $\mu_0 \geq x_2$.}
To prove the lemma in this case as observed before it is sufficient to show that the exponent is not positive and the estimate is then at least 1.
\[
\begin{split}
	c + 1 + \log \log \left( \frac{1}{\mu_0} \right) 
		& \leq c + 1 + \log \log \left( \frac{1}{x_2} \right) \\ 
		& = c + 1 + \log \left(- \log \left(2 ^ {-2 ^ {-c - 1}}\right)\right) \\ 
		& = c + 1 + \log \left(2 ^ {-c - 1}\right) \\ 
		& = c + 1 - c - 1 = 0
\end{split}
\]

The inequality holds for every of the three cases and the induction step is complete.
\end{proof}

\begin{theorem}
\label{theorem-linear-function-set-onto}
Let $u$ and $t$ be natural numbers such that $0 < t \leq u$. Let $S$ be a~proper and non-empty subset of the~vector space $\vecspace{u}$. Set $\mu = 1 - \frac{|S|}{2^u}$ as inverse density of $S$ in $\vecspace{u}$. Then for a random uniformly chosen surjective linear map $T: \vecspace{u} \rightarrow \vecspace{t}$ we have
\[
	\Prob{T(S) \neq \vecspace{t}} \leq \mu^{u - t - \log t + \log \log \frac{1}{\mu}} \textit{.}
\]
\end{theorem}
\begin{proof}
Set $s = u - t$. Choose vectors $v_1, \dots, v_s \in \vecspace{u}$ independently and randomly using the uniform distribution. Note that vectors $v_1, \dots, v_s \in \vecspace{u}$ need not to be linearly independent.

We perform random and uniform choice of linear transformation $T$ according to Model \ref{remark-model-surjective-linear-map-selection}. Maximal linearly independent subset of vectors $v_1, \dots, v_s$ may be extended to a random basis $b$ of vector space $\vecspace{u}$. Now choose a random permutation $\pi \in \Pi_t$ as stated in the just mentioned model. Since selection of vectors $v_1, \dots, v_s$ is random and uniform, moreover vectors are independent of each other the basis $b$ is chosen uniformly indeed. Moreover the vectors $v_1, \dots, v_s$ define a part $s'$ of the needed set $s \in S$ since we need to place them in the kernel of the constructed mapping $T$. Set $s'$ must be uniformly extended to any set $s \in S$ such that $s' \subseteq s$. The uniform choice of function $T$ is done by this. Just note that $T(v_i) = 0$ for all $i = 1, \dots, s$.	

Let us define $S_0 = S$ and $S_i = S_{i - 1} \cup (S_{i - 1} + v_i)$ and set $\mu_i = 1 - \frac{|S_i|}{2 ^ u}$. When considering $\mu_i$ as random variables by using Lemma \ref{lemma-choose-random-vector} we can derive the fact that $\Expect{\mu_i} = \mu_{i - 1} ^ 2$ for every $i \in \{1, \dots, s \}$. Because every set $S_i$ is an extension of the previous set $S_{i - 1}$ it is clear that $0 < \mu = \mu_0 < 1$ and $\mu_i \leq \mu_{i - 1}$ for all $i = 1, \dots, s$. The assumptions of Lemma \ref{lemma-random-variable} are satisfied and we obtain
\[
\begin{split}
\Prob{\mu_s \geq 2 ^ {-t}} 
	& \leq \mu ^ {s - \log \log \left(\frac{1}{2 ^ {-t}}\right) + \log \log \left( \frac{1}{\mu} \right)} \\
	& = \mu ^ {s - \log t + \log \log \left(\frac{1}{\mu}\right)} \\
	& = \mu ^ {u - t - \log t + \log \log \left(\frac{1}{\mu}\right)}
\end{split}
\]

We want to show that whenever $\mu_s < 2^{-t}$ the event $T(S_s) = \vecspace{t}$ occurs. Since $\mu_s = 1 - \frac{|S_s|}{2 ^ u}$ the size of set $S_s$ equals ${2 ^ u}(1 - \mu_s)$. Using the assumption $\mu_s < 2 ^ {-t}$ it follows that $|S_s| > 2^u - 2^{u - t}$. To get the contradiction we assume that there is a vector $x \in \vecspace{t} - T(S_s)$ or equivalently $T(S_s) \neq \vecspace{t}$. Under these conditions it is clear that $T ^ {-1}(x)$ and $S_s$ are disjoint. From Lemma \ref{lemma-linear-transformation-domain-distribution} we have $|T ^ {-1}(x)| = 2 ^ {u - t}$. This would mean that
\[
2 ^ u = |\vecspace{u}| \geq |S_s \cup T^{-1}(x)| > 2 ^ u - 2 ^ {u - t} + 2 ^ {u - t} = 2 ^u
\] which is impossible. The fact that if $\mu_s < 2^{-t}$ then $T(S_s) = \vecspace{t}$ can be rewritten in terms of probability as
\[
	\Prob{\mu_s < 2^{-t}} \leq \Prob{T(S_s) = \vecspace{t}}
\]

Because $\vecspace{u}$ is a vector space over field $\mathbb{Z}_2$, by induction over $i$, we obtain that $S_i = S_{i - 1} \cup (v_i + S_{i - 1}) = S_0 + \vecspan{v_1, \dots, v_i}$. Note that $T(v_i) = 0$ because mapping $T$ was chosen so that every vector $v_i$ was placed in the kernel of $T$. This simply implies $T(S_s) = T(S)$.

The proof of the theorem is finished by combining the previous notes.

\[
\begin{split}
\Prob{T(S) \neq \vecspace{t}} 
	& = \Prob{T(S_s) \neq \vecspace{t}}  \\
	& = 1 - \Prob{T(S_s) = \vecspace{t}} \\
	& \leq 1 - \Prob{\mu_s < 2 ^ {-t}} \\
	& = \Prob{\mu_s \geq 2 ^ {-t}} \\
	& \leq \mu ^ {u - t - \log t + \log \log \left(\frac{1}{\mu}\right)} \\
\end{split}
\]
\end{proof}

The next theorem shows the probability of the complementary event, $T(S) \neq \vecspace{t}$, if the set $S$ is large enough.
\begin{theorem}
\label{theorem-set-onto-by-linear-transform}
If $T: \vecspace{w} \rightarrow \vecspace{t}$ is a random uniformly chosen linear map. Then for every $0 < \epsilon < 1$ there is a constant $c_\epsilon > 0$ such that for every subset $S$ of the domain $\vecspace{w}$, $|S| \geq c_\epsilon t 2^t$, the probability of mapping $S$ onto the whole space is
\[
	\Prob{T(S) = \vecspace{t}} \geq 1 - \epsilon \text{.}
\]
\end{theorem}
\begin{proof}
First set $u = \lceil\log(\frac {2|S|}{\epsilon})\rceil$. Let $T_1: \vecspace{u} \rightarrow \vecspace{t}$ be a random uniformly chosen surjective linear mapping. Since $c_\epsilon$ gets chosen large enough we have that $u \geq t$ and thus such mapping exists.  Fix $T_1$. Then for every random uniformly chosen linear mapping $T_0: \vecspace{w} \rightarrow \vecspace{u}$ linear mapping $T$ defined as $T = T_1 \circ T_0$  is a random linear mapping with uniform distribution by using Model \ref{remark-model-uniform-linear-map-selection}. 

Since the family of all linear mappings from $\vecspace{w}$ into $\vecspace{u}$ is $1$-universal we conclude that \[ \Prob{T_0(\vec{x}) = T_0(\vec{y})} = 2 ^ {-u} \] for all distinct vectors $\vec {x}$ and $\vec {y}$ from $\vecspace{w}$. If $d_S$ is the number of  all pairs of distinct vectors $\vec {x},\vec {y}\in S$ with $T_0(\vec {x}) = T_0(\vec {y})$ then the expected value of the random variable $d_S$ is \[ \Expect{d_S}= \binom{|S|}{2} 2 ^ {-u} \text{.} \]

If $|T_0(S)| \leq \frac {|S|}{2}$ then there exist at least $\frac {|S|}{2}$ pairs of distinct vectors $\vec {x},\vec {y} \in S$ with $T_0(\vec {x}) = T_0(\vec {y}
)$. By Markov inequality \[ \Prob{d_S \geq k\binom {|S|}{2} 2^{-u}} \leq \frac{1}{k} \text{.} \]
Thus if we set $k = \frac {|S|2^u}{2\binom {|S|}{2}}$ then we obtain 
\[ 
	\Prob{|T_0(S)| \leq \frac {|S|}{2}} 
		\leq \Prob{d_S \geq \frac {|S|}{2}} 
		\leq \frac{2 \binom{|S|}{2}}{|S|2^u} = \frac{|S| - 1}{2^u} < \frac{|S|}{2^u} 
		\leq \frac{\epsilon |S|}{2|S|} = \frac{\epsilon}{2}
\]

We can summarize that
\[ 
\Prob{T(S) \neq \vecspace{t} \wedge |T_0(S)| \leq \frac{|S|}{2}} \leq \frac{\epsilon}{2} \text{.}
\]

Secondly we compute $\Prob{T(S) \neq \vecspace{t} \wedge |T_0(S)| > \frac{|S|}{2}}$. By Theorem \ref{theorem-linear-function-set-onto} for $T_1: \vecspace{u} \rightarrow \vecspace{t}$ and $T_0(S) \subseteq \vecspace{u}$, we have
\[
	\Prob{T(S) = T_1(T_0(S)) \neq \vecspace{T} \wedge |T_0(S)| > \frac{|S|}{2}} \leq \mu ^ {u - t - \log t  +\log\log \left(\frac{1}{\mu}\right) } 
\]
where $\mu = 1- \frac{|T_0(S)|}{2^u}$. Clearly
\[
\mu = 1 - \frac{|T_0(S)|}{2 ^ u} < 1 - \frac{|S|}{2 . 2 ^ u} \leq 1 - \frac{\epsilon |S|}{8|S|} \leq e^{-\frac{\epsilon}{8}}
\text{.}
\]

In the following constant $c_{\epsilon}$ is chosen as $4\left(\frac{2}{\epsilon}\right) ^ {\frac{8}{\epsilon}}$. Then we can estimate:
\[
\begin{split}
& -\frac{\epsilon}{8} \left(u - t - \log t + \log \log \left( \frac{1}{\mu} \right) \right) \\
& \qquad = -\frac{\epsilon}{8}\left(\left\lceil\log\left(\frac{2|S|}{\epsilon}\right)\right\rceil - t - \log t + \log\log\left(\frac{1}{\mu}\right)\right) \\
& \qquad \leq -\frac{\epsilon}{8} \left( \left\lceil \log\left( \frac{8 \left(\frac{2}{\epsilon}\right) ^ {\frac{8}{\epsilon}}t2^t}{\epsilon} \right) \right\rceil - t - \log t + \log\log\left(\frac{1}{\mu}\right)\right) \\
& \qquad \leq -\frac{\epsilon}{8} \left(3 + \frac{8}{\epsilon}\log\frac{2}{\epsilon} - \log\epsilon + \log t + t - t - \log t + \log \left( \left(\frac{\epsilon}{8}\right)\log e \right) \right) \\
& \qquad = -\frac{\epsilon}{8} \left(3 - \log\epsilon + \frac{8}{\epsilon} \log \frac{2}{\epsilon} + \log\epsilon - 3 + \log\log e\right) \\
& \qquad = -\frac{\epsilon}{8}\left(\frac{8}{\epsilon}\log \frac{2}{\epsilon} + \log\log e\right) \\
& \qquad = \log \frac{\epsilon}{2} - \frac{\epsilon}{8} \log\log e \\
& \qquad \leq \log\frac{\epsilon}{2} \\
\end{split}
\]

And for the calculated probability we derive:
\[
\begin{split}
\Prob{T(S) \neq \vecspace{t} \wedge |T(S)| > \frac{|S|}{2}} 
	& \leq \mu ^ {u - t - \log t + \log\log\left(\frac{1}{\mu}\right)} \\
	& \leq e ^ {-\frac{\epsilon\left(u - t - \log t + \log\log\left(\frac{1}{\mu}\right)\right)}{8}} \\
	& \leq e ^ {\log \left( \frac{\epsilon}{2} \right)} \leq e ^ {\ln \left(\frac{\epsilon}{2}\right)} = \frac{\epsilon}{2} \text{.}
\end{split}
\]

If we connect both alternatives we deduce that 
\[ 
	\Prob{T(S) = T_1(T_0(S)) \neq \vecspace{t}} \leq \frac{\epsilon}{2} + \frac{
\epsilon}{2} = \epsilon
\]
and form this it follows that $\Prob{T(S) = \vecspace{t}} \geq 1 - \epsilon$.

\end{proof}

The biggest disadvantage of the current estimate of $c_\epsilon$ is high inaccuracy. For a practical use of the asymptotic growth proved in the following theorem we would need the smallest value possible. In the following sections we will try to lower the value of $c_\epsilon$ so that we are able to create a rule that allows a reasonable time warranty for the find operation.
\end{section}

\begin{section}{The original result}
The last two theorems of the previous chapter give us enough power to achieve our goal - asymptotic restriction of the worst case chain length. Once again we achieve this limit when hashing even super-linear amount of $n \log n$ elements into $n$ slots. By hashing a linear amounts this expected length can not grow. Every hashed set can be extended into $n \log n$ elements and the estimate must still remain valid. 

The most common models use load factors lower than one. Because of these models we generalise theorem so that it is parametrised by the table's load factor. Important relation of the expected length of the longest chain on the table's load factor is found. The result is that the multiplicative constant of the asymptotic growth is relative to the load factor of a hash table. New results are achieved by modifications of the given proofs with some new ideas.

\begin{theorem}
\label{theorem-n-logn-to-n}
Let $U = \vecspace{w}$ be the hashed universe, $B = \vecspace{t}$ be the hash table and $S \subset U$ be the hashed set such that $|S| = m \log m$ where $t \in \mathbb{N}, m = 2 ^ t$. Let $H = LT(\vecspace{w}, \vecspace{t})$ be the universal system used. Then 
\[
	\Expect{\lpsl} \in O(\log m \log \log m) \text{.}
\]
\end{theorem}
The original proof is showed in the following few pages. It contains some remarks and definitions that are pointed out on their own and later refined. It uses two basic ideas; factorisation and probability estimates of two correlated events.

Let the target space, representation of the hash table, be denoted by $B = \vecspace{t}$. As in the previous proof we will factor a linear transformation $T \in H$, $T: \vecspace{w} \rightarrow \vecspace{t}$. We define factor vector space $A = \vecspace{u}$ for $u \geq \log n$ and two functions $T_0: \vecspace{w} \rightarrow \vecspace{u}$ and $T_1: \vecspace{u} \rightarrow \vecspace{t}$ which is surjective. Both linear functions $T_0$ and $T_1$ are selected uniformly among $LT(U, A)$ and $LTS(A, B)$ respectively. By direct use of model \ref{remark-model-uniform-linear-map-selection} we obtain uniform choice of the transformation $T = T_1 \circ T_0$ .

The second idea of the proof is to estimate the probability of event called $E1$, $\lpsl > l$, for any natural number $l$. The probability bound on $E_1$ is found by inspecting another quite unnatural event $E_2$. Probability of event $E_1$ is then simply determined by discovering the probabilities $\Prob{E_2 | E_1}$ and $\Prob{E_2}$.

\begin{definition}
Let $l$ be a natural number, $T: U \rightarrow B$ be a linear transformation and $S \subset U$ be the hashed set. Event $E_1(S, T, l)$ denotes existence of a chain of length at least $l$ elements when using function $T$. 
\[ 
	E_1(S, T, l) \equiv \exists \vec{y} \in B: | T^{-1}(\vec{y}) \cap S | > l
\]
\end{definition}

\begin{definition}
Let $T_0: U \rightarrow A, T_1: A \rightarrow B$ be linear transformations where $T_2$ is surjective and $S \subset U$ be the hashed set. Event $E_2(S, T_0, T_1)$ is then defined as:
\[
	E_2(S, T_0, T_1) \equiv \exists \vec{y} \in B: T_1^{-1}(\vec{y}) \subseteq T_0(S) \text{.}
\]
\end{definition}

When it is clear what we mean by $S, T_0, T_1$ and $t$ we omit the parametrisation of the events and just use $E_1$ or $E_2$.

Now we will point out another definition of the event $E_2$ that fits better to the scheme of remark $\ref{theorem-linear-function-set-onto}$.
\begin{remark}
\label{remark-e2-equivalency}
Let $T_0: U \rightarrow A$ and $T_1: A \rightarrow B$ be linear transformations and moreover $T_2$ be surjective. Let $S \subset U$ be the hashed set. Appearance of event $E_2(S, T_0, T_1)$ is then equivalent to
\[
	E_2(S, T_0, T_1) \equiv \exists \vec{y}: T_1^{-1}(\vec{y}) \subseteq T_0(S) \Leftrightarrow T_1(A - T_0(S)) \neq B \text{.}
\]
\end{remark}
\begin{proof}
To proof direction from the left to the right assume that event $E_2$ is present. Event $E_2$ occurs if and only if there is a vector $\vec{y} \in B$ such that $T_1^{-1}(\vec{y}) \subseteq T_1(S)$. Under these conditions transformation $T_1$ may not display set $A - T_0(S)$ onto $B$ since $\vec{y} \notin T_1(A - T^{-1}(y)) \supseteq T_1(A - T_0(S))$ or equivalently $B \neq T_1(A - T_0(S))$.

If $B \neq T_1(A - T_0(S))$ then there exists a vector $\vec{y} \in B$ such that $\vec{y} \notin T_1(A - T_0(S))$. Since $T_1$ is surjective we have $T_1(A) = B$. Because no point from $A - T_0(S)$ is displayed on $\vec{y}$ the whole preimage of $\vec{y}$ must be contained in $T_0(S)$, $T_1^{-1}(\vec{y}) \subseteq T_0(S)$.
\end{proof}

As mentioned before we can use the previous equivalency to estimate probability of event $E_2$ as stated in the following remark. 
\begin{remark}
\label{remark-e2-probability}
Let $U = \vecspace{w}$, $A = \vecspace{u}$ and $B = \vecspace{t}$. Let $T_0: U \rightarrow A$ and $T_1: A \rightarrow B$ be linear transformations where $T_1$ is surjective and $S \subset U$ denote the hashed set such that $|S| \leq |B| \log |B|$. Define $d = \frac{|A|}{|S|}$. If $d > 1$ then 
\[
	\Prob{E_2(S, T_0, T_1))} \leq d^{-\log d - \log \log d} \text{.}
\]
\end{remark}
\begin{proof}
Theorem \ref{theorem-linear-function-set-onto} for surjective transformation $T_1$, set $A - T_0(S) \subset A$, target space $B = \vecspace{t}$ and $\mu = \frac{|T_0(S)|}{|A|}$ states that
\[
	\Prob{T_1(A - T_0(S)) \neq B} \leq \mu ^ {u - t - \log t + \log \log \frac{1}{\mu}} \text{.}
\]

Corresponding inverse density $\mu$ can be computed as 
\[
	\mu = 1 - \frac{|A - T_0(S)|}{|A|} = 1 - \frac{|A| - |T_0(S)|}{|A|} = \frac{|T_0(S)|}{|A|} \leq \frac{|S|}{|A|} = \frac{1}{d} < 1 \text{.}
\] 

Now we can rewrite logarithm of variable $d$ as 
\[
	\log d = \log \frac{|A|}{|S|} = \log |A| - \log |S| \geq \log |A| - \log (|B| \log |B|) = u - t - \log t \text{.}
\]

Since $E_2 \equiv T_1(A - T_0(S)) \neq B$ we can conclude
\[
\begin{split}
\Prob{E_2}
	& \leq \mu^{u - t - \log t + \log \log \left(\frac{1}{\mu}\right)} \\
	& \leq \mu ^ {\log d + \log \log \left(\frac{1}{\mu}\right)} \\
	& \leq \left(\frac{1}{d}\right) ^ {\log d + \log \log d} \\
	& = d ^ {-\log d - \log \log d} \text{.} \\
\end{split}
\]

Because of the assumptions of theorem \ref{theorem-linear-function-set-onto} the proof of this remark holds only when $\emptyset \neq A - T_0(S) \neq A$. Since $S$ is non-empty set we can see that $A - T_0(S) \neq A$. Because $d = \frac{|A|}{|S|} > 1$ it must be true that $|A| > |S| \geq |T_0(S)|$ and set $A - T_0(S)$ can not be empty.
\end{proof}

A similar lemma for estimating the conditional probability of event $E_2 | E_1$ follows.
\begin{remark}
\label{remark-prob-l-length-chain}
Let $U = \vecspace{w}$, $A = \vecspace{u}$ and $B = \vecspace{t}$. Let $T_0: U \rightarrow A$ and $T_1: A \rightarrow B$ be random uniformly chosen linear transformations where $T_1$ is surjective. Then for every $0 < \epsilon < 1$ there is a constant $c_{\epsilon} > 0$ such that for every $S \subset U$ denoting the hashed  elements and for every $l \in \mathbb{N}$, $l \geq c_{\epsilon}{\frac{|A|}{|B|}}\log\frac{|A|}{|B|}$ the upper bound for the probability of event $E_2 | E_1$ is
\[
	\Prob{E_2(S, T_0, T_1) | E_1(S, T, l)} \geq 1 - \epsilon \text{.}
\]
\end{remark}
\begin{proof}
Suppose that we are given a mapping $T$ and the event $E_1$ appears. There must be a subset $S' \subseteq S$ consisting of at least $l$ elements mapped by $T$ to a single element $\vec{y} \in B$. We fix this element and define $U' = T^{-1}(\vec{y})$ and $A' = T_1^{-1}(\vec{y})$. Notice that $S' = U' \cap S$.

From lemma \ref{lemma-linear-transformation-domain-distribution} it follows that size of the set $A'$ is exactly $\frac{|A|}{|B|}$. And from lemma \ref{lemma-linear-transformation-domain-distribution} we have that sets $A'$ and $U'$ are affine subspaces. Since $T_0$ is random uniformly chosen linear mapping its restriction $T_0|_{U'}$ is also a random and uniformly chosen linear transformation as stated in model \ref{remark-model-uniform-linear-map-selection-affine}.

Since we assumed presence of $E_1$ cardinality of $S' = U' \cap S$ must be at least $l \geq c_{\epsilon}\frac{|A|}{|B|} \log\frac{|A|}{|B|} = c_{\epsilon}{A'}\log{A'}$. Now we can use theorem \ref{theorem-set-onto-by-linear-transform} for the source space $U'$, set $S' \subseteq U'$, target space $A'$ and mapping $T_0|_{U'}$ and we obtain that 
\[
	\Prob{T_0|_{U'}(S') = A' | E_1} \geq 1 - \epsilon \text{.}
\]

Remark that we used theorem \ref{theorem-set-onto-by-linear-transform} for affine mapping $T_0|_{U'}$. However, we can use the generating mapping of $T_0|_{U'}$, corresponding non-affine subspaces and transform $S'$ to the original non-affine subspace instead.

Event $E_2$ is certainly present whenever $A' \subseteq T_0(S)$. In the language of probability:
\[
	\Prob{A' \subseteq T_0(S) | E_1} \leq \Prob{E_2 | E_1} \text{.}
\]

Finally we can finish the remark's proof
\[
	\Prob{E_2 | E_1} \geq \Prob{A' \subseteq T_0(S) | E_1} \geq \Prob{A' = T_0|_{U'}(S') | E_1} \geq 1 - \epsilon \text{.}
\]
\end{proof}

\begin{corollary}
\label{corollary-prob-e2-e1}
Let $U = \vecspace{w}$, $A = \vecspace{u}$ and $B = \vecspace{t}$. Let $T_0: U \rightarrow A$ and $T_1: A \rightarrow B$ be random uniformly chosen linear transformations where $T_1$ is surjective. Then for every $0 < \epsilon < 1$ there is a constant $c_{\epsilon} > 0$ such that for every $S \subset U$ denoting the hashed  elements and for every $l \in \mathbb{N}$, $l \geq c_{\epsilon}{\frac{|A|}{|B|}}\log\frac{|A|}{|B|}$ for the probability of event $E_1$ we have
\[
	\Prob{E_1(S, T, l)} \leq \frac{1}{1 - \epsilon} \Prob{E_2(S, T_0, T_1)} \text{.}
\]
\end{corollary}
\begin{proof}
The proof is a straightforward use of the previous remark and definition of conditional probability.
\[
	\frac{\Prob{E_2, E_1}}{\Prob{E_1}} = \Prob{E_2 | E_1} \geq 1 - \epsilon
\]

Using the fact $\Prob{E_2} \geq \Prob{E_2, E_1}$ we conclude:
\[
	\Prob{E_1} \leq \frac{1}{1 - \epsilon}\Prob{E_2, E_1} \leq \frac{1}{1 - \epsilon}\Prob{E_2} \text{.}
\]
\end{proof}

The probability bound of existence of a long chain is estimated by the following theorem.
\begin{remark}
\label{remark-probability-long-chain}
Let $U = \vecspace{w}$, $B = \vecspace{t}$ be vector spaces representing the domain ($U$) and the hash table ($B$). Let $T: U \rightarrow B$ be random uniformly chosen linear map and $m = 2 ^ t = |B|$. For every $0 < \epsilon < 1$ and for every $r > 4$ when hashing $S \subset U$, $|S| \leq t 2 ^ t$ following estimate on the length of the longest chain holds.
\[
	\Prob{\lpsl > 4 c_\epsilon r \log m \log \log m} \leq \frac{1}{1 - \epsilon} \left(\frac{r}{\log r}\right)^{-\log \left(\frac{r}{\log r}\right) - \log \log \left(\frac{r}{\log r}\right)} \text{.}
\]
\end{remark}
\begin{proof}
Proof of this remark is a straightforward use of previous remarks, we only have to choose the values of their parameters. As mentioned before we create the factorisation space $A = \vecspace{u}$. By validity of model \ref{remark-model-uniform-linear-map-selection-affine} uniform and independent choice of two random linear transformations $T_0: U \rightarrow A$ and surjective $T_1: A \rightarrow B$ corresponds to the uniform choice $T: U \rightarrow B$ such that $T = T_1 \circ T_0$.

Since $|S| \leq t 2 ^ t$ we have also that $|S| \leq n \log n$. Our choice of $u$ must confirm to $|A| \geq |B|$ since we need a surjective function $T_1$. The choices also satisfy assumptions of the remark \ref{remark-e2-probability} and corollary \ref{corollary-prob-e2-e1}.
\[
\begin{split}
	u & = \left\lfloor \log m + \log \log m + \log r - \log \log r + 1 \right\rfloor \\
	l & = 4c_{\epsilon}r\log m \log \log m \\
\end{split}
\]

Since $r > 4$ we have that $\frac{r}{\log r} > 1$:
\[
	2 ^ u \geq \frac{rm \log m}{\log r} > m \log m \geq 2 ^ t = |B| \text{.}
\]

Hence the condition $d = \frac{|A|}{|S|} > 1$ of remark \ref{remark-e2-probability} is satisfied because
\[
	d = \frac{|A|}{|S|} = \frac{2^u}{m \log m} \geq \frac{m \log m}{m \log m}\frac{r}{\log r} = \frac{r}{\log r} > 1 \text{.}
\]

Following inequality helps us to prove the assumption $l \geq c_\epsilon \frac{|A|}{|B|} \log \frac{|A|}{|B|}$ of the corollary \ref{corollary-prob-e2-e1}.
\[
	\frac{2^u}{m} \leq \frac{2 ^{\log m + \log \log m + \log r - \log \log r + 1}}{m} = \frac{2 r\log m}{\log r}
\]

Because $\log \left(\frac{2 r\log m}{\log r}\right) \leq 2 \log \log m \log r$ the assumption holds:
\[
\begin{split}
c_{\epsilon}\frac{2^u}{m}\log\left(\frac{2^u}{m}\right)
	& \leq 2 c_{\epsilon} \log m \frac{r}{\log r} \log \left(\frac{2 r\log m}{\log r}\right) \\
	& \leq 4 c_{\epsilon} \log m \frac{r}{\log r} \log \log m \log r \\
	& = 4 c_{\epsilon} r \log m \log \log m \\
	& = l \text{.}
\end{split}
\]

Now using remark \ref{remark-e2-probability} and corollary \ref{corollary-prob-e2-e1} we obtain:
\[
\begin{split}
\Prob{E_1}
	& \leq \frac{1}{1 - \epsilon} \Prob{E_2} \\
	& \leq \frac{1}{1 - \epsilon} d ^ {-\log d - \log \log d} \\ 
	& \leq \frac{1}{1 - \epsilon} \left(\frac{r}{\log r}\right)^{-\log \left(\frac{r}{\log r}\right) - \log \log \left(\frac{r}{\log r}\right)} \text{.}
\end{split}
\]

The event $E_1$ denotes the existence of a chain of size at least $l$ elements. A chain longer than $l$ in a hash table exists if and only if the longest chain is longer than $l$, $\lpsl > l \Leftrightarrow E_1(S, T, l)$. This proof completed by writing down the facts observed so far.

\[
\begin{split}
\Prob{\lpsl > l} 
	& = \Prob{\lpsl > 4c_{\epsilon} r \log m \log \log m} \\
	& = \Prob{E_1(S, T, 4c_{\epsilon} r \log m \log \log m)} \\
	& \leq \frac{1}{1 - \epsilon} \left(\frac{r}{\log r}\right)^{-\log \left(\frac{r}{\log r}\right) - \log \log \left(\frac{r}{\log r}\right)} \\
\end{split}
\]
\end{proof}

Previous claim is very convenient in order to successfully prove theorem \ref{theorem-n-logn-to-n}. For every fixed $0 < \epsilon < 1$ set $K_\epsilon = 4 c_\epsilon \log m \log \log m$.
\[
\begin{split}
\Expect{\lpsl}
	& = \int\limits_0^{\infty} \Prob{\lpsl > r} dr \\
	& \leq 4K_\epsilon + \int\limits_{4K_\epsilon}^\infty \Prob{\lpsl > r} dr \\
	& = 4K_\epsilon + K_\epsilon \int\limits_4^\infty \Prob{\lpsl > rK_\epsilon} dr \\
	& \leq 4K_\epsilon + K_\epsilon \int\limits_4^\infty \frac{1}{1 - \epsilon} \left(\frac{r}{\log r}\right)^{-\log \left(\frac{r}{\log r}\right) - \log \log \left(\frac{r}{\log r}\right)} dr \\
	& = K_\epsilon(4 + I_\epsilon) = O(K_\epsilon) = O(\log m \log \log m)
\end{split}
\]

We substituted $I_\epsilon$ for
\[
I_\epsilon = \int\limits_4^\infty \frac{1}{1 - \epsilon} \left(\frac{r}{\log r}\right)^{-\log \left(\frac{r}{\log r}\right) - \log \log \left(\frac{r}{\log r}\right)} dr \text{.}
\]
The fact that this integral is convergent for every $0 < \epsilon < 1$ is shown later.

The proof of the theorem \ref{theorem-n-logn-to-n} is completed even in more general form. We made no special assumptions on the size of the hashed set $S$ except $|S| \leq |B| \log |B|$. \qed

\mbox{\qedhere}

Multiplicative constant $4 c_\epsilon(4 + I_\epsilon)$ plays an important role for a practical use of this result. In our proofs performed so far we neglected its estimation. We only obtained a good asymptotic result that is negated by the constant's great value. For example when choosing $\epsilon$ equal to $\frac{1}{2}$ only constant $c_\epsilon$ equals $4 ^ {17}$ by usage of the original estimate. Our next goal is to show a better constant's estimate, explore the dependency of the longest chain on load factor of the hash table and find an even better constant when using load factors lower than one.
\end{section}

\begin{section}{Hashing a linear amount of elements with respect to the table's size}
Usage of the hash table with load factors lower than one brings us even lower expected lengths of the longest chains. This situation lowers the multiplicative constant and this is the reason why we examine this case. We already showed if a super-linear amount of elements is hashed then we can expect reasonable worst case behaviour. However the most important is the real expected size of a bucket. Upper bound on the expected operation's running time is equal to $1 + c \alpha$ in every model of universal hashing. So using load factors lower than one has significant impact on the expected case. We must examine a scheme of hashing $n = \alpha m$ elements into a hash table of size $m$. 

\begin{theorem}
\label{theorem-n-to-n}
Assume that the table's load factor $\alpha$ is bounded in $\left[0.5, 1\right]$. Then whenever hashing $\alpha m$ elements into a table of size $m$ the expected length of the longest chain is bounded by $O(\alpha \log m \log \log m)$.
\end{theorem}
\begin{proof}

In the case of hashing $m \log m$ elements we prepared some useful and remarks in advance and then finished main proof. We have to modify them, especially the choices for the size of the factor space differ. Apparently we must lower the value of a chain length when we get a convenient probability estimate proportionally to $\alpha_f$. 

\begin{remark}
Suppose a model of hashing domain $U = \vecspace{w}$ to a hash table $B = \vecspace{t}$ and define $m = |B|$. Assume we use $LT(U, B)$ as universal class and let $T \in LT(U, B)$ be a random uniformly chosen linear map used as a hash function. Moreover let $S \subset U$ be hashed set such that $|S| = \alpha m$ for load factor $0.5 \leq \alpha \leq 1$. Then for every $0 < \epsilon < 1$ there and $r > 4$,  probability of existence of a long chain is bounded as:
\[
\Prob{\lpsl > 4 c_\epsilon \alpha r \log m \log \log m} \leq \frac{1}{1 - \epsilon} \left(\frac{r}{\log r}\right)^{-\log \left(\frac{r}{\log r}\right) - \log \log \left(\frac{r}{\log r}\right)} \text{.}
\]
\end{remark}
\begin{proof}
We just show how the choices have to be made to prove the remark. We use the same approach and follow the proof of remark \ref{remark-probability-long-chain}. Just remember that we factored through the vector space $A = \vecspace{u}$ and the choice of its dimension has been made. We used a uniform model of selection of two linear functions $T_0: U \rightarrow A$ and surjective $T_1: A \rightarrow B$ such that $T = T_1 \circ T_0$. This gave us uniform choice of $T$.

\[
\begin{split}
	u & = \left\lfloor \log m + \log \log m + \log r - \log \log r + \log \alpha + 1 \right\rfloor \\
	l & = 4 c_\epsilon \alpha r \log m \log \log m \\
	d & = \frac{|A|}{\alpha m \log m}
\end{split}
\]

In order to ensure existence of surjective mapping $T_1$ we have to verify that $|A| \geq |B|$.
\[
	|A| = 2 ^ u \geq \frac{\alpha m \log m r}{\log r} \geq m \log m \geq |B|
\]

We slightly changed the choice of variable $d$. In the original remark \ref{remark-e2-probability} we defined $d$ as $\frac{|A|}{|S|}$ but in its proof three inequalities concerning $d$ were needed.
\[
\begin{split}
	d & > 1 \\
	\mu & = 1 - \frac{|A - T_0(S)|}{|A|} \leq \frac{1}{d} < 1 \\
	\log d & \geq u - t - \log t \\
\end{split}
\]

To keep remark \ref{remark-e2-probability} in validity we have to verify each of them. For the first one we can use just observed fact $|A| > |B|$.
\[
	d = \frac{|A|}{\alpha m \log m} \geq \frac{m \log m}{\alpha m \log m} = \frac{1}{\alpha} \geq 1
\]

The second one.
\[
	\mu = 1 - \frac{|A - T_0(S)|}{|A|} = \frac{|T_0(S)|}{|A|} \leq \frac{|S|}{|A|} = \frac{\alpha |B|}{|A|} \leq \frac{\alpha m \log m}{|A|} = \frac{1}{d} < 1
\]

And the third one follows. Just remember the bound on $\alpha$, $\alpha \in \left[0.5, 1\right]$ and this implies that $\log \alpha \leq 0$.
\[
	\log d = u - \log \alpha - \log |B| - \log \log |B| = u - t - \log t - \log \alpha \geq u - t - \log t
\]

The assumptions of modified remark \ref{remark-e2-probability} are met. For using the other remark, \ref{remark-prob-l-length-chain}, the value of the variable $l$ has to be large enough:
\[
\begin{split}
c_{\epsilon} \frac{|A|}{|B|} \log \left(\frac{|A|}{|B|}\right) 
	& = c_{\epsilon} \frac{2^u}{m} \log \left(\frac{2^u}{m}\right) \\
	& < 2 c_{\epsilon} \alpha \log m \left( \frac{r}{\log r} \right) \left(2 \log \log m \log r \right) \\
	& \leq 4 c_{\epsilon} \alpha r \log m \log \log m \\
	& = l
\end{split}
\]

The conditions of both remarks \ref{remark-e2-probability} and \ref{remark-prob-l-length-chain} are satisfied. We can carry on identically as in the case without the load factor. In order to express probability density function of the variable $\lpsl$ we have to estimate $\Prob{E2(S, T_0, T_1)|E1(S, T, l)}$. Now use theorem \ref{theorem-set-onto-by-linear-transform} for sets $U' = T^{-1}(\vec{y})$, $A' = T_1^{-1}(\vec{y})$ and restricted $T_0|_U'$ as in the original proof. Vector $\vec{y}$ is taken from appearance of the event $E_2(S, T, l)$ and define $S' = U' \cup S$.
\[
	\Prob{T_0|_{U'}(S') = A' | E1} \geq 1 - \epsilon
\]

As in the previous case whenever $T_0|_{U'}(S') = A'$ event $E2$ appears as well and we conclude.

\[
\begin{split}
\Prob{E1} 
	& \leq \frac{1}{\Prob{E2|E1}}{\Prob{E2}} \\
	& \leq \frac{1}{1 - \epsilon} d ^ {-\log d - \log \log d} \\
	& = \frac{1}{1 - \epsilon} \left(\frac{r}{\log r}\right)^{-\log \left(\frac{r}{\log r}\right) - \log \log \left(\frac{r}{\log r}\right)} \text{.}
\end{split}
\]

For the last inequality we used the fact $d \geq \frac{r}{\log r} > 1$.
\[
	d = \frac{|A|}{\alpha m \log m} \geq \frac{2 \alpha m \log m}{\alpha m \log m} \frac{r}{\log r} \geq \frac{r}{\log r}
\]

Now we have since $\frac{1}{d} \leq \frac{\log r}{r} < 1$:
\[
\begin{split}
d ^ {-\log d - \log \log d} 
	& = \left(\frac{1}{d}\right) ^ {\log d + \log \log d} \\
	& \leq \left(\frac{1}{d}\right) ^ {\log \left(\frac{r}{\log r}\right) + \log \log
 \left(\frac{r}{\log r}\right)} \\
	& \leq \left(\frac{\log r}{r}\right) ^ {\log \left(\frac{r}{\log r}\right) + \log \log
 \left(\frac{r}{\log r}\right)} \\
	& = \left(\frac{r}{\log r}\right)^{-\log \left(\frac{r}{\log r}\right) - \log \log \left(\frac{r}{\log r}\right)} \text{.}
\end{split}
\]
\end{proof}

In order to achieve the desired expected longest chain length we perform similar computation as in the original theorem. Now we set $K = 4 \alpha c_\epsilon \log m \log \log m$.
\[
\begin{split}
\Expect{\lpsl}
	& = \int\limits_0^\infty \Prob{\lpsl > r} dr \\
	& \leq 4K + \int\limits_{4K}^\infty \Prob{\lpsl > r} dr \\
	& = 4K + K\int\limits_4^\infty \Prob{\lpsl > rK} dr \\
	& \leq K(4 + I_\epsilon) = O(K) = O(\alpha \log m \log \log m)
\end{split}
\]
\end{proof}
\end{section}

