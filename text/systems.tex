\chapter{Universal classes of functions}

Universal classes of functions play an important role in hashing since they improve independence of the associative memory on its input. Their greatest advantage is removal of the two assumptions of classic hashing on uniformity of input. The first one is the uniform distribution of hashed sets having the same size. The second one says that every element of the universe has the same probability of being hashed. 

In addition models of standard hashing take probability of collision relative to the choice of the hashed set. In models of universal hashing probability space corresponds to the choice of a random function from the universal system. Notice that there is no direct relation between a collision and the input - the represented set. Collision is caused by an inconvenient choice of a universal function instead by an unsuitable input. Once again probabilities of collision events are taken relative to the uniform selection of a hash function. The greatest advantage of this approach is that whatever bound is obtained it holds for any hashed set. 

Universal systems of functions are constructed in a way that the probability for a universal function not being suitable for a hashed set is low. Functions in such universal systems are convenient for many hashed sets. If the selected function is not suitable for any hashed set we are free to pick one. Rehashing the whole represented set using the new one is not likely to fail. Moreover probability of failing many times in a row decreases exponentially.

\section{Universal classes of functions}
% TODO: Odkazy na jednotlive systemy
% TODO: Citacia na clanok s definiciami.

\begin{definition}[$c$-universal class of hash functions]
\label{c_universal_system}
% TODO: Oprava cez indexovy system alebo multimnozina.
Let $c \in R$ be a positive number and $H$ be a set of functions $h: U \rightarrow B$ such that for each pair of elements $x, y \in U$, $x \neq y$ we have \[ \left| \lbrace h \in H \setdelim h(x) = h(y) \rbrace \right| \leq \frac {c |H|}{|B|} \text{.} \] The set $H$ is then called $c$-universal class of hash functions.
\end{definition}

At first notice that the bound on the number of colliding functions for each domain pair is proportional to the size of the class $H$. More interesting is the inverse proportionality to the size of the hash table, set $B$ is formalisation of a hash table. A generalisation of the above definition for more than two elements provides us with a better estimate on the number of colliding functions. This count becomes inversely proportional to the power of the table size.

\begin{definition}[Nearly strongly $n$-universal class of hash functions]
\label{nearly_strong_universal_n_system}
Let $n \in \mathbb{N}$, $c \in \mathbb{R}, c > 1$ and $H$ be a set of functions $h: U \rightarrow B$ such that for every choice of $n$ different elements $x_1, x_2, \dots, x_n \in U$ and their images $y_1, y_2, \dots, y_n \in B$ the following inequality holds
\[ 
	\left|\lbrace h \in H \setdelim h(x_1) = y_1, h(x_2) = y_2, \dots, h(x_n) = y_n \rbrace \right| \leq \frac{c|H|}{|B|^n} \textit{.} 
\] 
The system of functions $H$ is called nearly strongly $n$-universal class of hash functions with constant $c$.
\end{definition}

\begin{definition}[Strongly $n$-universal class of hash functions]
\label{strong_universal_n_system}
Let $n \in \mathbb{N}$ and $H$ be a set of functions $h: U \rightarrow B$ such that for every choice of $n$ different elements $x_1, x_2, \dots, x_n \in U$ and their images $y_1, y_2, \dots, y_n \in A$ the following holds \[ \left|\lbrace h \in H \setdelim h(x_1) = y_1, h(x_2) = y_2, \dots, h(x_n) = y_n \rbrace \right| \leq \frac{|H|}{|B|^n} \textit{.} \] Then the system of functions $H$ is called strongly $n$-universal class of hash functions.
\end{definition}

Every strongly $n$-universal class of hash functions is nearly strongly $n$-universal class with constant 1. For most of the presented strongly universal classes not only the inequality is satisfied but the given bound is exactly the count of the colliding functions.

\begin{definition}[Strongly $\omega$-universal class of hash functions]
\label{strong_universal_omega_system}
Family of functions $H$ is strongly $\omega$-universal if and only if it is strongly $n$-universal for every $n \in \mathbb{N}$.
\end{definition}

% TODO: Citacia na clanok o systeme omega univerzalnych funkcii.
Such systems provide us with and estimate of the length of the longest chain in the hashing table. This bound can be proved directly from the above property without regarding any other special attributes of a system. So the above property is difficult to be satisfied by a small class of functions. A straightforward example is the set of all function from the domain $U$ to a hash table $B$. As mentioned this system is not very convenient because of its size.

The above definitions of universal classes can be rewritten in terms of probability as well. The probability space corresponds to the uniform choice of function from the system of functions. For example in definition \ref{c_universal_system} of $c$-universal class $H$ of hash functions probability of collision of two different elements $x$ and $y$ can be rewritten as
\[
	\Prob{h(x) = h(y)} = \frac{|\{h \in H \setdelim h(x) = h(y)\}|}{|H|} = \frac{c}{m} \text{.}
\]
In the same manner we can reformulate and use definitions for the other universal systems.

In the remainder of this chapter we sum up some basic facts about universal systems, show remarks about their combinations and derive some theoretic bounds of strongly $k$-universal systems.
\begin{remark}
Every $2$-strongly universal class of hash functions is also 1-universal.
\end{remark}
\begin{proof}
Let $H$ be $2$-strongly universal class of hash functions containing functions mapping universe $U$ to hash table $B$. We have to find the number of functions in $H$ that make a given pair of elements $x, y \in U$, $x \neq y$ collide. 

A collision for a function $h \in H$ and given pair $x \neq y \in U$ can be written in the form $h(x) = h(y) = t$ for any image $t \in B$. Define the set of functions that display both $x$ and $y$ on the image $t \in B$ \[ H_t = \lbrace h \in H \setdelim h(x) = h(y) = t \rbrace \textit{.} \] These sets are disjoint and moreover $|H_t| \leq \frac{|H|}{{|B|}^{2}}$. By summing throughout all $t \in B$ we have
\begin{displaymath}
|\lbrace h \in H \setdelim h(x) = h(y) \rbrace | = \displaystyle \sum_{t \in B} |H_t| \leq |B| \frac{|H|}{{|B|}^{2}} = \frac{H}{|B|}\textit{.}
\end{displaymath}
The number of all colliding functions is then less or equal to $\frac{H}{|B|}$ and the system $H$ is clearly $1$-universal.
\end{proof}

\begin{remark}
Every strongly $n$-universal class of functions is strongly $k$-universal for every $1 \leq k \leq n$.
\end{remark}
\begin{proof}
Let $H$ be a $n$-universal class of functions mapping a universe $U$ to a hash table $B$. Similarly to the previous proof there are $k$ different elements $x_1, \dots, x_k \in U$ that should be mapped to the prescribed buckets $y_1, \dots, y_k \in B$. For every choice of their images $y_1, \dots, y_k \in B$ we need to find the number of functions that display the element $x_i$ to its image $y_i$, $1 \leq i \leq k$. 

The only estimate we can use comes from the strong $n$-universality of $H$ and holds for $n$ elements only. We extend the set of images by other $n - k$ elements $x_{k + 1}, \dots, x_n$. They are let to be mapped on arbitrary element of $B$. Such extension must be done carefully so that the final sequence consists of different elements $x_1, \dots, x_n$, too. From now on fix one such extension, $x_{k + 1}, \dots, x_n$.

The set of functions mapping given $k$ elements onto their prescribed images, $\bar H$, may be seen as:
\[
\begin{split}
\bar{H}	& = \lbrace h \in H \setdelim h(x_i) = y_i, i \in \lbrace 1, \dots, k \rbrace \rbrace \\
	& = \displaystyle \bigcup_{y_{k+1}, \dots, y_n \in B} \lbrace h \in H \setdelim h(x_i) = y_i, i \in \lbrace 1, \dots, n \rbrace \rbrace \textit{.}
\end{split}
\]

Notice that sets of functions that display elements $x_{k + 1}, \dots, x_n$ onto different images $y_{k + 1}, \dots, y_n$ are disjoint. The sum of their sizes is then equal to $|\bar H|$.
\[
\begin{split}
|\bar H|& = \left| \displaystyle \bigcup_{y_{k+1}, \dots, y_n \in B} \lbrace h \in H \setdelim h(x_i) = y_i, i \in \lbrace 1, \dots, n \rbrace \rbrace \right| \\
	& = \displaystyle \sum_{y_{k+1}, \dots, y_n \in B} \left| \lbrace h \in H \setdelim h(x_i) = y_i, i \in \lbrace 1, \dots, n \rbrace \rbrace \right| \\
	& \leq \displaystyle \sum_{y_{k+1}, \dots, y_n \in B} \frac{H}{{|B|}^{n}} = {|B|}^{n - k} \frac{H}{{|B|}^{n}} = \frac{H}{{|B|}^{k}}
\end{split}
\]

Now we can see that the class of functions $H$ is strongly $k$-universal, too.
\end{proof}

\section{Examples of universal classes of functions}
In this section we show some examples of common universal classes of hash functions. With each system we provide a simple proof of its universality or strong universality. Presented systems not only differ in contained functions but also in their domains and codomains. However every system can be thought of as a mapping from a subset of natural numbers onto another subset of natural numbers.

% Linear systems
\paragraph{Linear system}
System of linear functions was among the first systems of universal hash functions. They were introduced by Carter and Wegman in \cite{DBLP:journals/jcss/CarterW79} where they showed basic properties of universal hashing. Above all they mentioned the expected constant time of the find operation. Nowadays various modifications of linear systems are known and are presented in the following text. 

\begin{definition}[Linear system]
Let $N$ be a prime and $m \in \mathbb{N}$. Define the sets $U = \{0, \dots, N - 1 \}$ and $B = \{0, \dots, m - 1\}$. Then the class of linear functions 
\[ LS = \{h_{a, b}(x): U \rightarrow B \setdelim a, b \in U \textit{ where } h_{a, b}(x) = ((ax + b) \bmod N) \bmod m \} \]
is called the linear system.
\end{definition}

\begin{remark}
Linear system is $\frac{\left\lceil \frac{N}{m} \right\rceil ^ 2}{\left(\frac{N}{m}\right) ^ 2}$-universal.
\end{remark}
\begin{proof}
Consider two different elements $x, y \in U$. We need to find the number of functions $h_{a, b} \in H$ such that 
\[ (ax + b \bmod N) \bmod m = (ay + b \bmod N) \bmod m \textit{.} \]

This is equivalent to existence of numbers $r, s, t$ that satisfy the following constraints:
\begin{gather*}
r \in \{0, \dots, m - 1 \} \\
t, s \in \left\{ 0, \dots, \left \lceil \frac{N}{m} \right \rceil - 1 \right\} \\
(ax + b) \bmod N = s m + r \\
(ay + b) \bmod N = t m + r \\
\end{gather*}

Since $N$ is a prime number and $x \neq y$ for every choice of parameters $r, s, t$ there is an exactly unique solution; parameters $a, b$ that satisfy the equalities.

Number of all possible choices of parameters $r, s, t$ is $m \left \lceil \frac{N}{m} \right \rceil ^ 2$. So there are at most $m \left \lceil \frac{N}{m} \right \rceil ^ 2$ functions $h_{a, b}$ that display the elements $x, y$ to a same value.

We compute the system's $c$-universality constant.
\[
m \left \lceil \frac{N}{m} \right \rceil ^ 2 = 
m \left \lceil \frac{N}{m} \right \rceil ^ 2 \frac{\frac{N ^ 2}{m}}{\frac{N ^ 2}{m}} = 
\frac{\left \lceil \frac{N}{m} \right \rceil ^ 2}{\frac{N ^ 2}{m ^ 2}} \frac{N ^ 2}{m} = 
\frac{\left \lceil \frac{N}{m} \right \rceil ^ 2}{\frac{N ^ 2}{m ^ 2}} \frac{|LS|}{|B|}
\]

Hence the linear system is $\frac{\left \lceil \frac{N}{m} \right \rceil ^ 2}{\left(\frac{N}{m}\right) ^ 2}$-universal.
\end{proof}
% End linear systems

% Multiplicative system
For examining the properties of the original linear system simpler and smaller multiplicative system may be used. They have many common properties and share some same drawbacks. This multiplicative system may provide us with some insights for which choices of hash function and hashed set they both fail.

\begin{definition}[Multiplicative system]
Let $p$ be a prime and $m$, $m < p$ be the size of the hash table. Then the system \[ \mathrm{Mult}_{m, p} = \lbrace h:\mathbb{Z}_p \rightarrow \mathbb{Z}_m \setdelim h(x) = (kx \bmod p) \bmod m \text{ for } k \in \mathbb{Z}_p - \{0\} \rbrace \] is called the multiplicative system.
\end{definition}

\begin{theorem}
For a prime $p$ and $m < p$ the multiplicative system is $2$-universal.
\end{theorem}
\begin{proof}
Let $x$ and $y$ be different elements of the universe. We have to find the number of functions in the multiplicative system that make their images the same.
\begin{displaymath}
\begin{split}
kx \bmod p \bmod m & = ky \bmod p \bmod m \\
kx - ky \bmod p \bmod m & = 0 \\
\end{split}
\end{displaymath}

This may be rewritten as:
\begin{displaymath}
k(x - y) \bmod p = l m \qquad \text{ for  } l \in \left\lbrace -\left\lfloor\frac{p}{m}\right\rfloor, \dots, \left\lfloor\frac{p}{m}\right\rfloor \right\rbrace - \{0\} \text{.}
\end{displaymath}

Notice the parameter $l \neq 0$ since $k \neq 0$. Thus for every choice of $x$ and $y$ there are at most $2\left\lfloor\frac{p}{m}\right\rfloor$ functions. For every value of $l$ there is exactly one $k$ satisfying the equality because parameter $p$ is a prime. For the number of colliding functions we have \[ | \lbrace h \in \mathrm{Mult}_{m,p} \setdelim h(x) = h(y) \rbrace | \leq \frac{2p}{m} \] and the system is then $2$-universal.
\end{proof}
% End of multiplicative system

% Linear transformations
Previous systems contain only simple linear functions from natural numbers to natural numbers. An analogous class can be constructed by using linear transformations between vector spaces. The systems have various similar properties but the latter one gives us a suitable bound on the size of the largest bucket.

\paragraph{System of linear transformations}
\begin{definition}[System of linear transformations]
\label{definition-system-of-linear-transformations}
Let $U$ and $B$ be two vector spaces over $\mathbb{Z}_2$ and $dim(U) = m$ and $dim(B) = n$. System of functions \[ LT(U, B) = \{L : U \rightarrow B \setdelim L \text{ is a linear transformation}\} \] is called system of linear transformations between $U$ and $B$.
\end{definition}
\begin{remark}
\label{remark-system-of-linear-transformations}
System of linear transformations between vector spaces $U$ and $B$ is $1$-universal.
\end{remark}
\begin{proof}
Let $m$ and $n$ denote the dimensions of vector spaces $U$ and $B$ respectively. Then every linear transformation $L: U \rightarrow B$ is associated with a unique $n \times m$ binary matrix $T$. 

Let $x, y \in U$ be two distinct vectors mapped on a same image. We need to find the number of mappings confirming this collision. Let $k$ be a position such that $x_k \neq y_k$. Such $k$ exists since $x \neq y$. We show the following statement. If every element, except those in the $k$-th column, of the matrix $T$ is fixed then the remaining ones are determined uniquely. We just lost $n$ degrees of freedom when creating matrix $T$ causing the collision of $x$ and $y$. This lost implies $1$-universality of the system which is proved later as well.

In order to create a collision of $x$ and $y$ their images $Tx$, $Ty$ must be the same. The system of equalities must then be true for every $i$, $1 \leq i \leq n$:
\[
\displaystyle\sum_{j = 1}^{m}t_{i, j}x_j = \displaystyle\sum_{j = 1}^{m}t_{i, j}y_j \text{.}
\]

For $t_{i, k}$ we have:
\[
t_{i, k} = (x_k - y_k)^{-1}\displaystyle\sum_{j = 1, j \neq k}^{m}t_{i, j}(y_j - x_j) \text{.}
\]

The elements in the $k$-th column are uniquely determined by the previous equality. 

The number of all functions, equivalently the number of all binary matrices, is $2^{mn}$. Notice that the size of the target space $B$ is $2^n$. The number of all functions mapping $x$ and $y$ to a same value is $2^{mn - n}$ since we are allowed to arbitrarily choose every element of the matrix $T$ except the $n$ ones in the $k$-th column. 

System of linear transformations is $1$-universal because:
\[
2^{mn - n} = \frac{2^{mn}}{2^n} = \frac{|LT|}{|B|} \text{.}
\]
\end{proof}

We are not limited only to vector spaces over field $\mathbb{Z}_2$. Systems of all linear transformations between vector spaces over any finite field $\mathbb{Z}_p$ are $1$-universal, too. However the most interesting results are achieved for the choice of $\mathbb{Z}_2$.
% End linear transformations

% Bit string sum
\paragraph{Bit sum of a string over the alphabet \{0, 1\}}
Presented family is another linear system that is $1$-universal, too. It is clear that every natural number can be written as a string over alphabet consisting from two characters only, $0$ and $1$, it is the number's binary form. Weighted digit sum is considered the number's hash value. This reminds us of the system of linear transformations.

If we are hashing strings that are $k + 1$-bit long we also need $k + 1$ coefficients. Also assume hashing into a table of size $p \in \mathbb{N}$ where $p$ is a prime. Coefficients may be chosen arbitrarily but are not greater than a parameter $l$, $0 < l \leq p$. To transform a digit sum into an address of the hash table simple modulo $p$ operation is used. Parameter $l$ may seem quite artificial. However it sets the range for coefficients and therefore determines the size of the whole system.

\begin{definition}[Bit string system]
Let $p$ be a prime, $k \in \mathbb{N}$, $B = \{0, \dots, p - 1 \}$ and $l \in \mathbb{N}$, $l \leq p$. System of functions
\begin{displaymath}
BSS_{p, l} = \left\{ h: \{0, 1\}^{k + 1} \rightarrow B \setdelim h(x) = \left(\displaystyle \sum_{i=0}^{k} c_i x_i\right) \bmod p, c_i \in \{0, 1, \dots, l - 1\} \right\}
\end{displaymath} 
is called bit string system with parameters $p$ and $l$.
\end{definition}

\begin{remark}
Bit string system for binary numbers of length $k + 1$ modulo prime $p$ and constant $l \in \mathbb{N}$, $l \leq p$ is $\frac{p}{l}$-universal.
\end{remark}
\begin{proof}
Let $x$ and $y$ be two different bit string $x, y \in \{0, 1\} ^ {k + 1}$. We must estimate the number of all sequences of coefficients $0 \leq c_0, \dots, c_k < l$ that make the two elements collide. Every collision sequence must satisfy:
\[
\left( \displaystyle \sum_{i = 0}^{k} c_i x_i \right) \bmod p = \left( \displaystyle \sum_{i = 0}^{k} c_i y_i \right) \bmod p \text{.}
\]

Since $x \neq y$ there is an index $j$, $0 \leq j \leq k$ such that $x_j - y_j \neq 0$. This allows us to exactly determine $j$\textsuperscript{th} coefficient $c_j$ from the others as stated below.

\begin{displaymath}
\begin{split}
\left(\displaystyle \sum_{i=0}^{k-1} c_i x_i\right) \bmod p & = \left(\displaystyle \sum_{i=0}^{k-1} c_i y_i\right) \bmod p \\
c_j(x_j - y_j) \bmod p & = \left(\displaystyle \sum_{i=0, i \neq j}^{k-1} c_i (x_i - y_i)\right) \bmod p \\
c_j & = (x_j - y_j) ^ {-1}\left(\displaystyle \sum_{i=0, i \neq j}^{k-1} c_i (x_i - y_i)\right) \bmod p
\end{split}
\end{displaymath}

Last equality holds since $p$ is a prime number and the number $x_j - y_j$ is invertible in the field $\mathbb{Z}_p$. There is one degree of freedom lost when choosing a function that makes $x$ and $y$ collide. The number of all functions is $l ^ {k + 1}$ and the size of the table is $p$. The number of colliding functions can be computed as:

\begin{displaymath}
|\{h \in BSS_{p, l} | h(x) = h(y) \}| = l^{k} = \frac{p}{l}\frac{l^{k + 1}}{p} = \frac{p}{l}\frac{|BSS_{p, l}|}{|B|} \textit{.}
\end{displaymath}

The bit string system is thus $\frac{p}{l}$-universal.
\end{proof}
% End of bit string sum.

% Polynomials
The following system is a generalisation of so far discussed linear functions and linear transformations to polynomials. The advantage delivered by the system of polynomials is its strong universality. Unfortunately it is traded for the bigger size of the system.

\paragraph{Polynomials over finite fields}
\begin{definition}[Polynomials over finite fields]
Let $N$ be a prime number and $n \in \mathbb{N}$. Define $U = \{0, \dots, N - 1 \}$ and $B = \{0, \dots, m - 1\}$ The system of functions \[ P_n = \left\lbrace h_{c_0, \dots, c_n}(x): U \rightarrow B \setdelim c_i \in U, 0 \leq i \leq n \right\rbrace \] where \[ h_{c_0, \dots, c_n}(x) = \left( \left(\displaystyle \sum_{i=0}^{n} c_i x^i \right) \bmod N \right) \bmod m \] is called the system of polynomial hash functions of the $n$-th degree.
\end{definition}

\begin{remark}
Let $N$ be a prime number and $n \in \mathbb{N}$. If $B = U$ then the system of polynomial hash functions of the $n$-th degree is strongly $n + 1$-universal. When $B \neq U$ the system is nearly $n + 1$-strongly universal.
\end{remark}
\begin{proof}
Let $x_1, x_2, \dots, x_{n+1}$ be different elements of $U$ and $y_1, y_2, \dots, y_{n+1}$ are their prescribed images. We can write down a system of linear equations that can be used to find the coeficients $c_0, c_1, \dots, c_n$.
\[ 
h(x_i) = y_i, \quad 0 \leq i \leq n 
\]

If $U = B$ the function $h(x)$ is reduced to the form: \[ h(x) = \left( \displaystyle \sum_{i=0}^{n} c_i x^i \right) \bmod N \textit{.} \] Since $N$ is a prime number and the elements $x_i$ are different there is exactly one solution of the above system, let $c_0, \dots, c_n$ denote the solution. Size of the system is ${|U|}^{n+1}$ and $1 = \frac{|U| ^ {n + 1}}{|B|^{n + 1}} = \frac{|U| ^ {n + 1}}{|U|^{n + 1}}$. Now it is clear that the system of polynomial hash functions is strongly $n + 1$-universal.

Let $U \neq B$. We follow the proof of $c$-universality of the linear system. At first we write down the equations $h(x_i) = y_i$ in the field $\mathbb{Z}_N$. Instead using the last modulo operation new variables $r_i$ are introduces as in the case of the linear system.

\begin{displaymath}
\begin{split}
\left(\displaystyle \sum_{j=0}^{n} c_j x_{i}^{j} \right) \bmod N = {y}_i + {r_i}{m} \qquad 
 & y_i \in \{0, \dots, m - 1 \} \\
 & r_i \in \left\{0, m, 2m, \dots, \left\lceil \frac{N}{m} \right\rceil - 1 \right\} \\
\end{split}
\end{displaymath}
For every choice of all $r_i$, $0 \leq i \leq n$, we obtain an unique solution of the above system of equations since we are in the field $\mathbb{Z}_N$. The number of all choices of $r_i$ is ${\left\lceil \frac{N}{m} \right\rceil}^{n + 1}$. The count of all hash functions mapping $x_i$ onto $y_i$ complies the following:
\begin{displaymath}
|\{h \in H \setdelim h(x_i) = y_i, i \in \{1, \dots, n + 1\}\}| \leq {\left\lceil \frac{N}{m} \right\rceil}^{n + 1} = \frac{{\left\lceil \frac{N}{m} \right\rceil}^{n + 1}}{(\frac{N}{m})^{n+1}}\left(\frac{N}{m}\right)^{n+1}
\end{displaymath}

This system is nearly strongly $n+1$-universal with the constant $\frac{{\left\lceil \frac{N}{m} \right\rceil}^{n + 1}}{(\frac{N}{m})^{n+1}}$.
\end{proof}

Although the choice of $U = B$ is not very practical one however it gives us a clue that the system can be nearly strongly universal.
% End of polynomials

% All functions
Definition of strongly $\omega$-universality gives us a powerful property but is there a system that satisfies the definition. One example, although not very useful, can be immediately constructed - it is a set of all functions between two sets.

\paragraph{System of all functions}
\begin{definition}[System of all functions]
Let $U$ and $B$ be sets such that $|B| < |U|$. The family of functions
\[
H = \{h: U \rightarrow B \}
\]
is called the system of all functions between $U$ and $B$.
\end{definition}

\begin{remark}
System of all functions between sets $U$ and $B$ is strongly $\omega$-universal.
\end{remark}
\begin{proof}
We are given $n$ different elements $x_1, \dots, x_n$ and their images $y_1, \dots, y_n$. We must compute size of the set $H_n = \lbrace h \in H \setdelim h(x_i) = y_i, i \in \lbrace 1, \dots, n\rbrace \rbrace$. Now note that the system's size is $|H| = {|B|}^{|U|}$. Since the values for elements $x_1, \dots, x_n$ are fixed we have \[ |H_n| = {|B|}^{|U| - n} = \frac{{|B|}^{|U|}}{{|B|}^{n}} = \frac{|H|}{|B|^n} \] and thus the system is strongly $\omega$-universal.
\end{proof}

% TODO: Citation.
Let us show some considerations when we would like to use this system. The main problem when using this system is that fact that to encode a whole function $h \in H$ is highly inconvenient. To store one we need $|U| \log |B|$ bits. When using this simple approach we need to encode the values of elements that with great probability will never be stored in the hash table. An alternative is to construct random partial function from the set of all functions from $B$ to $U$. 

These ideas come from the existence of fast associative memory having expected time of find and insert operations $O(1)$ and existence of a random number generator. Whenever we have to store an element we must determine its hash value. At first by using the associative memory we find out if the stored element already has a hash value associated. If not we use the random number generator to obtain an element from $B$ and remember this connection, element - its hash value, to the associative memory. This sequentially constructs a random mapping from the system of all functions. Because the random number generator used chooses the elements of $B$ uniformly created system is certainly strongly $\omega$-universal. In addition we do not store the hash values for irrelevant elements. 

By using this idea to construct a fast hash table we are trapped in a cycle since we already supposed existence of a fast associative memory. But $\omega$-universal class can not only be used to construct a fast solution to the dictionary problem but to solve other problems like the set equality. %TODO: \cite{nieco}
% End of All functions

% TODO: Maly univerzalny system.
% \paragraph{Malý univerzálny systém}

% TODO: Maticovy system
% \paragraph{Systém definovaný pomocou matíc}

\section{Properties of systems of universal functions}
In this section we sum up some properties of $k$-strongly universal classes of functions. We concentrate on the generalisation of the known properties of the $c$-universal classes to the strongly universal classes. These results may inspire us to create more powerful or better behaving systems without losing strong universality.

The first theorems show the results of combinations of two strongly universal systems.
\begin{theorem}
Let $H$ and $I$ be $k$-strongly universal systems both mapping universe $U$ to a table of size $m$. The system of pair functions 
\begin{displaymath}
J = \lbrace f_{(h, i)}: U \rightarrow \lbrace 0, \dots, m - 1 \rbrace \times \lbrace 0, \dots, m - 1 \rbrace \setdelim h \in H, i \in I \rbrace
\end{displaymath} 
is $k$-strongly universal, too. Where the result of pair function $f_{(h, i)}$ is defined as $f_{(h, i)}(x) = (h(x), i(x))$.
\end{theorem}
\begin{proof}
Let $x_1, \dots, x_k \in U$ be given different elements that are supposed to be mapped onto selected buckets $(y_1, z_1), \dots, (y_k, z_k)$. We compute the number of pair functions $f_{(h, i)}$ such that for both functions $h$ and $i$ we have $h(x_1) = y_1, \dots, h(x_k) = y_k$, $i(x_1) = z_1, \dots, i(x_k) = z_k$. From strong $k$-universality of both systems we know that there are at most $\frac{|H|}{m^k}$ functions $h$ and $\frac{|I|}{m^k}$ functions $i$ satisfying mentioned criteria. So the number of all functions $f_{(h, i)}$ such that $f_{(h, i)}(x_1) = (y_1, z_1), \dots, f_{(h, i)}(x_k) = (y_k, z_k)$ is at most $\frac{|H||I|}{m^{2k}}$. The size of the table is $m^2$ and thus the constructed system is $k$-strongly universal.
\end{proof}

Combining two universal systems does not have to be limited to increasing the table's size. From the previous result we obtain a smaller probability of collision by paying the expensive price of the table expansion. When using single class of hash functions the same effect on probability decrease can be achieved by simple squaring the table's size. We could also combine two strongly universal functions and use a suitable operation to merge the given results into a smaller table. The system created remains strongly universal as it is stated in the next theorem and the combination does not bring us any obvious advantage.

\begin{theorem}
Let $B$ denote the set of buckets of a hash table of size $m$, $|B| = m$. Let both $H$ and $I$ be strongly $k$-universal classes of functions over the table $B$. Let the operation $o: B \times B \rightarrow B$ be given. The operation satisfies that for every $c, a \in B$ there is exactly one $b \in B$ such that $o(a, b) = c$. Then the system of functions \[F = \lbrace f_{(h, i)} | h \in H, i \in I \rbrace\] where \[f_{(h, i)}(x) = o(h(x), i(x))\] is $k$-strongly universal.
\end{theorem}
\begin{proof}
To prove the theorem we are given different elements $x_1, \dots, x_k$ that are supposed to be mapped on prescribed images $y_1, \dots, y_k$. For every $y_i$ there are exactly $m$ pairs $(a_i, b_i)$ such that $o(a_i, b_i) = y_i$ because we can choose $m$ different values for $a_i \in B$. A pair for a single value $y_i$ may be chosen arbitrarily for every $1 \le i \le k$. There are $m^k$ pairs that fit the values $y_i$.

From the previous theorem we have that probability of the event \[ h(x_1) = a_1, \dots, h(x_k) = a_k \text{ and } i(x_1) = b_1, \dots, i(x_k) = b_k \] is less than $\frac{1}{m^{2k}}$. Now we see that $\Prob{f(x_i} = y_i, 1 \le i \le k) = \frac{m^k}{m^{2k}} = \frac{1}{m^k}$. Now we see the system $F$ is strongly $k$-universal.
\end{proof}

\begin{corollary}
Combining the results of pair functions created from $k$-strongly universal classes by operations
\begin{itemize}
\item $xor$ where the table size is a power of 2
\item $+$, $-$ over an additive group $Z_l$ where $l$ is the table size
\end{itemize}
gives $k$-strongly universal classes of functions.
\end{corollary}

Next step is to discover the dependence of the parameter $k$ of strongly $k$-universal classes on their size. This will show us the best possible probability bound for the collision of $k$ different elements depending on the size of the system.
\begin{theorem}
Let $H$ be a $k$-strongly universal system of functions mapping universe $U$ to a table $B$. Size of the system is then bounded by:
\begin{displaymath}
|H| > m^{k - 1} \log_m \left( \frac{N}{km} \right) \textit{.}
\end{displaymath}
\end{theorem}
\begin{proof}
At first assign numbers to every function in H, $H = \{h_1, \dots, h_{|H|}\}$ and define $m = |B|$. Now we create the sequence of sets $U_0$, $U_1$, $\dots$, where $U_0 = U$. Inductively define set $U_i$ as the greatest subset of $U_{i - 1}$ such that $h_i$ is a constant on $U_i$. 

From the Dirichlet's principle size of every set $U_i$ may be estimated as $|U_i| \geq \frac{|U_{i - 1}|}{m}$. If for every $x \in B$ we had $|\displaystyle h_{i}^{-1}(x)| < \frac{|U_{i - 1}|}{m}$ for the size of $U_{i -1}$ we would have \[|U_{i -1}| = \displaystyle\sum_{x \in B} |\displaystyle h_{i}^{-1}(x)| < m\frac{|U_{i - 1}|}{m} = |U_{i -1}|\] which is impossible. By using a bounded induction one can obtain $|U_i| \geq \frac{|U|}{m^i}$.

%TODO: veta, ked mam silne k univerzalny, tak prechod na pravdepodobnost kolizie jednej fcie.
Now let the index $i$ denote the last set $U_i$ such that $|U_i| \geq k$. Functions $h_1, h_2,\dots, h_i$ map at least $k$ elements of set $U_i$ to a single image. The probability of this event is less or equal to $\frac{1}{m ^ {k - 1}}$ by using theorem 1.16. From this argument we know that $i \leq \frac{|H|}{m ^ {k - 1}}$. Because index $i$ is last with the property $U_i \geq k$ we also have that $\frac{|U|}{m ^ {i + 1}} \leq |U_{i + 1}| < |U_i| \leq k$. 

Putting it all together:
\begin{gather*}
\frac{N}{m^{i + 1}} < k \\
\frac{N}{k} < m ^ {i + 1} \\
\frac{\log \left( \frac{N}{km} \right)}{\log m} < i \\
\log_m \left( \frac{N}{km} \right) < i \\
\end{gather*}

Obtaining the bound on the size of system of functions:
\begin{gather*}
\frac{|H|}{m^{k - 1}} \geq i > \log_m \left( \frac{N}{km} \right) \\
|H| > m^{k - 1} \log_m \left( \frac{N}{km} \right) \\
\end{gather*}
\end{proof}

\begin{corollary}
If $H$ is a strongly $k$-universal class of functions then
\begin{displaymath}
k < 1 + \frac{\log |H| - \log \log_m \left( \frac{N}{km} \right)}{\log m} \textit{.}
\end{displaymath}
\end{corollary}
\begin{proof}
The result of the previous theorem can be rewritten to get bound on $k$ as:
\begin{displaymath}
\begin{split}
m^{k - 1} & < \frac{|H|}{\log_m \left( \frac{N}{km} \right)} \\
k & < 1 + \frac{\log |H| - \log \log_m \left( \frac{N}{km} \right)}{\log m} \\
\end{split}
\end{displaymath}
\end{proof}
