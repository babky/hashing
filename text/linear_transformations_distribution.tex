\begin{section}{Probability Distribution of the Variable \texorpdfstring{$\lpsl$}{lpsl}}
\label{section-linear-transformations-distribution}
The last theorems of the previous chapter give us enough power to achieve our first goal -- asymptotic restriction of the expected length of the chain length. We achieve this goal when hashing even super-linear amount of $m \log m$ elements into a table consisting of $m$ slots. When storing sets of size equal to $\alpha m$ for a bounded load factor $\alpha$ the expected length can not grow. This fact comes from the idea that every stored set can be further extended into $m \log m$ elements and the estimate must still remain valid.

However common models use load factors lower than one. Therefore we generalise and refine our statements so that they are parametrised by the table's load factor. Then we find an important dependence of the expected length of the longest chain on the table's load factor. We get that the multiplicative constant of the asymptotic growth is relative to the load factor of the hash table.

First let us summarise what happens in the following pages. We use two basic ideas~--~factorisation and probability estimates of two highly correlated events. First we define the two events that enable us to estimate the probability of existence of a chain of length at least $l \in \mathbb{N}$. After the definitions we show facts that come from the original work \cite{DBLP:journals/jacm/AlonDMPT99}. They get immediately improved and we later use them for hashing. The theorems we state are not said in the notation of hashing directly. Rather we say them in general terms of vector spaces and then use them in the model proposed by us. 

The source space, represents later the universe, is the vector space $\vecspace{u}$. The target space, becomes a representation of the hash table, is denoted by $\vecspace{b}$. As already performed, we factor a random uniformly chosen linear transformation $T \in LT(\vecspace{u}, \vecspace{b})$ through the factor space $\vecspace{f}$. Model \ref{remark-model-uniform-linear-map-selection} implies existence of two linear functions $T_0: \vecspace{u} \rightarrow \vecspace{f}$ and $T_1: \vecspace{f} \rightarrow \vecspace{b}$. In addition the last one is surjective and they may be chosen uniformly among $LT(\vecspace{u}, \vecspace{f})$ and $LTS(\vecspace{f}, \vecspace{b})$.

\begin{definition}
Let $l \in \mathbb{N}$, $T: \vecspace{u} \rightarrow \vecspace{b}$ be a linear transformation and $S \subset \vecspace{u}$. \emph{Event $E_1(S, T, l)$} occurs if there is a subset of $S$ containing at least $l$ elements wich are mapped by the function $T$ on a singleton,

\[ 
	E_1(S, T, l) \equiv \exists \vec{y} \in \vecspace{b}: | T^{-1}(\vec{y}) \cap S | > l \text{.}
\]
\end{definition}

The second event is defined only to simplify the estimate of probability of the event $E_1$. Note it may seem quite unnatural but it fits the scheme of Theorem \ref{theorem-linear-function-set-onto} as shown later.
\begin{definition}
Let $T_0: \vecspace{u} \rightarrow \vecspace{f}, T_1: \vecspace{f} \rightarrow \vecspace{b}$ be linear transformations with $T_2$ being surjective and $S \subset \vecspace{u}$. \emph{Event $E_2(S, T_0, T_1)$} occurs if
\[
	E_2(S, T_0, T_1) \equiv \exists \vec{y} \in B: T_1^{-1}(\vec{y}) \subseteq T_0(S) \text{.}
\]
\end{definition}

Remember that when it is clear what we mean by $S, T_0, T_1$ and $l$ we omit the parametrisation of the events and just use $E_1$ or $E_2$.

Now we will point an equivalent definition of the event $E_2$ showing why it may be used with Corollary $\ref{corollary-affine-e2}$.
\begin{remark}
\label{remark-e2-equivalency}
Let $T_0: \vecspace{u} \rightarrow \vecspace{f}$, $T_1: \vecspace{f} \rightarrow \vecspace{b}$ be linear transformations with $T_2$ being surjective and $S \subset \vecspace{u}$. Then the event $E_2(S, T_0, T_1)$ occurs if and only if $T_1(\vecspace{f} - T_0(S)) \neq \vecspace{b}$. Formally said
\[
	(E_2(S, T_0, T_1) \equiv \exists \vec{y}: T_1^{-1}(\vec{y}) \subseteq T_0(S)) \Leftrightarrow (T_1(\vecspace{f} - T_0(S)) \neq \vecspace{b}) \text{.}
\]
\end{remark}
\begin{proof}
To prove the direction from the left to the right assume that the event $E_2$ occurs. This happens if there is a vector $\vec{y} \in \vecspace{b}$ such that $T_1^{-1}(\vec{y}) \subseteq T_1(S)$. Hence transformation $T_1$ can not map the set $\vecspace{f} - T_0(S)$ onto the vector space $\vecspace{b}$ since $\vec{y} \notin T_1(\vecspace{f} - T^{-1}(\vec{y})) \supseteq T_1(\vecspace{f} - T_0(S))$ or equivalently $\vecspace{b} \neq T_1(\vecspace{f} - T_0(S))$.

Now we show the reverse direction. If $\vecspace{b} \neq T_1(\vecspace{f} - T_0(S))$, then there is a vector $\vec{y} \in \vecspace{b}$ such that $\vec{y} \notin T_1(\vecspace{f} - T_0(S))$. Since $T_1$ is surjective we have that $T_1(\vecspace{f}) = \vecspace{b}$. Since there is no point in $\vecspace{f} - T_0(S)$ displayed onto $\vec{y}$ it follows that the whole preimage of $\vec{y}$ must be contained in $T_0(S)$. Thus $T_1^{-1}(\vec{y}) \subseteq T_0(S)$.

\begin{figure}
  \centering
    \includegraphics[width=0.7\textwidth]{images/e2}

  \caption{Occurrence of the event $E_2$.}
\end{figure}

\end{proof}

Now we use the previous equivalency to estimate the probability of the event $E_2$ as stated in the following remark. 
\begin{remark}
\label{remark-e2-probability}
Let $T_0: \vecspace{u} \rightarrow \vecspace{f}$, $T_1: \vecspace{f} \rightarrow \vecspace{b}$ be linear transformations with $T_1$ being surjective and $S \subset U$. Set $d = \frac{|\vecspace{f}|}{|S|}$. If $|S| \leq b2 ^ b$ and $d > 1$, then 
\[
	\Prob{E_2(S, T_0, T_1))} \leq d^{-\log d - \log \log d} \text{.}
\]
\end{remark}
\begin{proof}
We apply Theorem \ref{theorem-linear-function-set-onto} for the transformation $T_1$, the set $\vecspace{f} - T_0(S)$, the target space $\vecspace{b}$ and the inverse density $\mu = 1 - \frac{|\vecspace{f} - T_0(S)|}{|\vecspace{f}|}$ and obtain
\[
	\Prob{T_1(\vecspace{f} - T_0(S)) \neq \vecspace{b}} \leq \mu ^ {f - b - \log b + \log \log \frac{1}{\mu}} \text{.}
\]

We can estimate $\mu$ using only the value $d$ as
\[
	\mu = 1 - \frac{|\vecspace{f} - T_0(S)|}{|\vecspace{f}|} = 1 - \frac{|\vecspace{f}| - |T_0(S)|}{|\vecspace{f}|} = \frac{|T_0(S)|}{|\vecspace{f}|} \leq \frac{|S|}{|\vecspace{f}|} = \frac{1}{d} < 1 \text{.}
\]
To obtain the bound claimed by the remark rewrite the logarithm of the value $d$,
\[
	\log d = \log \frac{|\vecspace{f}|}{|S|} = \log |\vecspace{f}| - \log |S| \geq \log |\vecspace{f}| - \log (|B| \log |B|) = f - b - \log b \text{.}
\]

In the following computation we use Corollary \ref{corollary-f-decreasing} to remove the inverse density $\mu$. Since $E_2 \equiv T_1(\vecspace{f} - T_0(S)) \neq \vecspace{b}$ and $\mu \leq \frac{1}{d}$ we have the needed bound,
\[
\begin{split}
\Prob{E_2}
	& \leq \mu^{f - b - \log b + \log \log \left(\frac{1}{\mu}\right)} \\
	& \leq \mu ^ {\log d + \log \log \left(\frac{1}{\mu}\right)} \\
	& \leq \left(\frac{1}{d}\right) ^ {\log d + \log \log d} \\
	& = d ^ {-\log d - \log \log d} \text{.} \\
\end{split}
\]

To meet the assumptions of Theorem \ref{theorem-linear-function-set-onto}, we must have that $\emptyset \neq \vecspace{f} - T_0(S)$ and $\vecspace{f} - T_0(S) \neq \vecspace{f}$. Since the set $S$ is non-empty it certainly holds that $\vecspace{f} - T_0(S) \neq \vecspace{f}$. Because $d = \frac{|\vecspace{f}|}{|S|} > 1$ it follows that $|\vecspace{f}| > |S| \geq |T_0(S)|$ and the set $\vecspace{f} - T_0(S)$ then can not be empty.
\end{proof}

We show similar remarks used when having the size of the set $S$ proportional to the size of the target space $\vecspace{b}$. This corresponds to the situation when hashing only $\alpha m$ elements.

\begin{statement}
\label{statement-e2-probability-linear-good}
Let $T_0: \vecspace{u} \rightarrow \vecspace{f}$, $T_1: \vecspace{f} \rightarrow \vecspace{b}$ be linear transformations with $T_1$ being surjective and $S \subset U$. Let $\alpha \in \mathbb{R}$, $\alpha > 0$ and assume that $|S| = \alpha 2 ^ b$.
\begin{enumerate}
\item If $d = \frac{2 ^ f}{\alpha b 2 ^ b}$ and $d > 1$, then $\Prob{E_2(S, T_0, T_1))} \leq d ^ {-\log \alpha - \log d - \log \log d}$.
\item If $d = \frac{2 ^ f}{\alpha 2 ^ b}$ and $d > 1$, then $\Prob{E_2(S, T_0, T_1))} \leq d ^ {\log b - \log \alpha - \log d - \log \log d}$.
\end{enumerate}
\end{statement}
\begin{proof}
We slightly changed the choice of variable $d$ and naturally switched to the different size of $|S|$. In Remark \ref{remark-e2-probability} we have $d = \frac{|A|}{|S|}$. In fact we needed just three inequalities concerning $d$ in its proof
\[
\begin{split}
	d & > 1 \\
	\mu & = 1 - \frac{|A - T_0(S)|}{|A|} \leq \frac{1}{d} < 1 \\
	\log d & \geq f - b - \log b \text{.}
\end{split}
\]
First we verify each of them. The first is an assumption of the statement. The second one follows from $d > 1$
\[
	\mu = 1 - \frac{|\vecspace{f} - T_0(S)|}{|\vecspace{f}|} = \frac{|T_0(S)|}{|\vecspace{f}|} \leq \frac{|S|}{|\vecspace{f}|} = \frac{\alpha |\vecspace{b}|}{|\vecspace{f}|} \leq 
\begin{cases}
\frac{\alpha b 2 ^ b}{2 ^ f} = \frac{1}{d} < 1 & \text{if } d = \frac{2 ^ f}{\alpha b 2 ^ b} \\
\frac{\alpha 2 ^ b}{2 ^ f} = \frac{1}{d} < 1 & \text{if } d = \frac{2 ^ f}{\alpha 2 ^ b} \text{.}
\end{cases}
\]

If we do not place any additional bound on the value $\alpha$ we can not get the third inequality exactly. In the first case if $d = \frac{2 ^ f}{\alpha b 2 ^ b}$, we have 
\[
	\log d = f - \log \alpha - \log |B| - \log \log |B| = f - b - \log b - \log \alpha \text{.}
\]
In the second case if $d = \frac{2 ^ f}{\alpha 2 ^ b}$, then
\[
	\log d = f - \log \alpha - \log |B| = f - b - \log \alpha \text{.}
\]

Recall that from the first two inequalities if follows that the assumptions of Theorem \ref{theorem-linear-function-set-onto} are met
\begin{gather*}
\emptyset \neq \vecspace{f} - T_0(S) \neq \vecspace{f} \\
\mu < 1 \text{.}
\end{gather*}
Now we use the theorem again for the transformation $T_1$, the set $\vecspace{f} - T_0(S)$ and the inverse density $\mu$ and obtain
\[
	\Prob{E_2} \leq \mu ^ {f - b - \log b - \log \log \left( \frac{1}{\mu} \right)} \text{.}
\]

To prove the statement perform a similar estimation as in the original proof. For next the estimates use Corollary \ref{corollary-f-decreasing} and the fact that $\mu \leq \frac{1}{d}$. For the first case it follows that
\[
\begin{split}
\Prob{E_2} 
	& \leq \mu ^ {f - b - \log b + \log \log \left( \frac{1}{\mu} \right)} \\
	& \leq \mu ^ {\log d + \log \alpha + \log \log \left( \frac{1}{\mu} \right)} \\
	& \leq \left(\frac{1}{d}\right) ^ {\log d + \log \alpha + \log \log d} \\
	& = d ^ {-\log \alpha - \log d - \log \log d} \text{.}
\end{split}
\]
In the second case we have that 
\[
\begin{split}
\Prob{E_2} 
	& \leq \mu ^ {f - b - \log b + \log \log \left( \frac{1}{\mu} \right)} \\
	& \leq \mu ^ {\log d + \log \alpha - \log b  + \log \log \left( \frac{1}{\mu} \right)} \\
	& \leq \left(\frac{1}{d}\right) ^ {\log d + \log \alpha - \log b + \log \log d} \\
	& = d ^ {- \log b -\log \alpha - \log d - \log \log d} \text{.}
\end{split}
\]
\end{proof}

A similar remark for an estimate of the conditional probability of the event $E_2 \mid E_1$ now follows.
\begin{remark}
\label{remark-prob-l-length-chain}
Let $T_0: \vecspace{u} \rightarrow \vecspace{f}$ and $T_1: \vecspace{f} \rightarrow \vecspace{b}$ be random uniformly chosen linear transformations with $T_1$ being surjective and $T = T_1 \circ T_0$. Let $S \subset \vecspace{u}$, $\epsilon \in (0, 1)$ and $l \in \mathbb{N}$, $l \geq c_{\epsilon}(f - b)2 ^ {f - b}$ where constant $c_\epsilon$ is from Theorem \ref{theorem-linear-function-set-onto}. Then
\[
	\Prob{E_2(S, T_0, T_1) \mid E_1(S, T, l)} \geq 1 - \epsilon \text{.}
\]
\end{remark}
\begin{proof}
According to Model \ref{remark-model-uniform-linear-map-selection} we have that the linear transformation $T = T_1 \circ T_0$ is chosen uniformly.

Assume that the event $E_1$ occurs. So there is a maximal subset $S_A \subseteq S$, $|S_A| > l$ mapped to a singleton $\vec{y} \in \vecspace{b}$. We fix the vector and define $U_A = T^{-1}(\vec{y})$ and $F_A = T_1^{-1}(\vec{y})$. Notice that $S_A = U_A \cap S$. According to Lemma \ref{lemma-linear-transformation-domain-distribution} the sets $U_A$ and $F_A$ are affine vector subspaces and $|F_A| = 2 ^ {f - b}$.

Since $T_0$ is a random uniformly chosen linear mapping from Model \ref{remark-model-uniform-linear-map-selection-affine} it follows that its restriction $T_0|_{U_A}$ is also a random uniformly chosen affine linear transformation.

Now we use Corollary \ref{corollary-affine-e2} for the source space $U_A$, the set $S_A \subseteq U_A$, the target space $F_A$ and mapping $T_0|_{U_A}$ and obtain that 
\[
	\Prob{T_0|_{U_A}(S_A) = F_A \mid E_1} \geq 1 - \epsilon \text{.}
\]

If $F_A \subseteq T_0(S)$, then $T_1^{-1}(\vec{y}) \subseteq T_0(S)$ and the event $E_2$ certainly occurs. Stated in the language of probability:
\[
	\Prob{F_A \subseteq T_0(S) \mid E_1} \leq \Prob{E_2 \mid E_1} \text{.}
\]

By using the fact $F_A = T_0|_{U_A}(S) \Rightarrow F_A \subseteq T_0(S)$ the remark's proof is finally finished by claiming
\[
	\Prob{E_2 \mid E_1} \geq \Prob{F_A \subseteq T_0(S) \mid E_1} \geq \Prob{F_A = T_0|_{U_A}(S_A) \mid E_1} \geq 1 - \epsilon \text{.}
\]

\begin{figure}
  \centering
    \includegraphics[width=0.9\textwidth]{images/elpsl_proof}
  \caption{Image depicting the situation in the proof.}
\end{figure}

\end{proof}

The next corollary puts the previous claims together.
\begin{corollary}
\label{corollary-prob-e2-e1}
Let $T_0: \vecspace{u} \rightarrow \vecspace{f}$ and $T_1: \vecspace{f} \rightarrow \vecspace{b}$ be a random uniformly chosen linear transformations with $T_1$ being surjective. Let $\epsilon \in (0, 1)$, $S \subset \vecspace{u}$ and $l \in \mathbb{N}$, $l \geq c_{\epsilon}(f - b)2 ^ {f - b}$. Then
\[
	\Prob{E_1(S, T, l)} \leq \frac{1}{1 - \epsilon} \Prob{E_2(S, T_0, T_1)} \text{.}
\]
\end{corollary}
\begin{proof}
The proof is a straightforward use of the previous remarks and Lemma \ref{lemma-conditional-probability-event-estimate} stating that
\[
	\Prob{E_1} \leq \frac{\Prob{E_2}}{\Prob{E_2 \mid E_1}} \leq \frac{1}{1 - \epsilon}\Prob{E_2} \text{.}
\]
\end{proof}

Definition \ref{definition-lpsl} of the variable $\lpsl$ comes from the area of hashing and uses the notation of chains and their lengths. However we can refer to it now, too. Assume that the universe is equal to the vector space $\vecspace{u}$ and the hash table is represented by the vector space $\vecspace{b}$ and $S \subset U$. The randomness is brought by the uniform choice of linear transformation $T: \vecspace{u} \rightarrow \vecspace{b}$ among $LT(\vecspace{u}, \vecspace{b})$. Realise that the random variable $\psl(\vec{b})~=~|T^{-1}(\vec{b}) \cap S|$ for a vector $\vec{b} \in \vecspace{b}$. By setting $\lpsl = \displaystyle\max_{\vec{b} \in \vecspace{b}} \psl(\vec{b})$ their meanings remain the same. 

\begin{lemma}
\label{lemma-e1-lpsl-equivalence}
If $T: \vecspace{u} \rightarrow \vecspace{b}$ is a random uniformly chosen linear transformation, $S \subset \vecspace{u}$ and $l \in \mathbb{N}$, then $E_1(S, T, l) \Leftrightarrow \lpsl > l$.
\end{lemma}
\begin{proof}
The event $E_1(S, T, l)$ denotes the existence of a vector $\vec{y} \in \vecspace{b}$ such that $|T^{-1}(\vec{y}) \cap S| > l$. Observe that such vector $\vec{y}$ exists if and only if the variable $\lpsl > l$ because
\[
(\exists \vec{y} \in \vecspace{b}: |T^{-1}(\vec{y}) \cap S| > l) \Leftrightarrow (\exists \vec{y} \in \vecspace{b}: \psl(\vec{y}) > l) \Leftrightarrow (\lpsl > l) \text{.}
\] 
\end{proof}

We are able to bound the probability density function of the random variable $\lpsl$ when regarding its above definition.
\begin{remark}
\label{remark-probability-long-chain}
Let $T: \vecspace{u} \rightarrow \vecspace{b}$ be a random uniformly chosen linear mapping, $\epsilon \in (0, 1)$ and $S \subset \vecspace{u}$, $|S| \leq b 2 ^ b$. Then for every $r \geq 4$
\[
	\Prob{\lpsl > 4 c_\epsilon r b \log b} \leq \frac{1}{1 - \epsilon} \left(\frac{r}{\log r}\right)^{-\log \left(\frac{r}{\log r}\right) - \log \log \left(\frac{r}{\log r}\right)} \text{.}
\]
\end{remark}
\begin{proof}
The proof itself is a straightforward use of the previous remarks. We only have to choose their parameters. First we create a factor space $\vecspace{f}$, its dimension is specified later. From Model \ref{remark-model-uniform-linear-map-selection-affine} and the random uniform and independent selection of two linear transformations $T_0: \vecspace{u} \rightarrow \vecspace{f}$ and surjective $T_1: \vecspace{f} \rightarrow \vecspace{b}$ it follows that the linear mapping $T: \vecspace{u} \rightarrow \vecspace{b}$ such that $T = T_1 \circ T_0$ is chosen uniformly as well. Fix the mappings $T_0$, $T_1$ and $T$.

Now set 
\[
\begin{split}
	f & = \left\lfloor b + \log b + \log r - \log \log r + 1 \right\rfloor \text{,} \\
	l & = 4c_{\epsilon}r b \log b \text{,} \\
	d & = \frac{2 ^ f}{|S|} \geq \frac{2 ^ f}{b 2 ^ b} \geq \frac{b 2 ^ b}{b 2 ^ b} \cdot \frac{r}{\log r} = \frac{r}{\log r} \geq 2 \text{.}
\end{split}
\]
The choice of $f$ implies that $|\vecspace{f}| \geq \vecspace{b}|$ because 
\[ 
f \geq b + \log b + \log r - \log \log r \geq b \text{.}
\]
Hence a surjective function $T_1$ exists and may be fixed. Notice that the choices meet all the assumptions, $d < 1$, of Remark \ref{remark-e2-probability}. To verify the condition $l \geq c_\epsilon (f - b) 2 ^ {f - b}$ of Remark \ref{remark-probability-long-chain} we first show that
\[
	2 ^ {f - b} \leq 2 ^ {b + \log b + \log r - \log \log r + 1 - b} = \frac{2 r b}{\log r} \text{.}
\]
From the fact, $f - b \leq \log \left(\frac{2 r b}{\log r}\right) \leq 2 \log b \log r$, we have that the assumption holds since
\[
\begin{split}
c_{\epsilon}(f - b) 2 ^ {f - b}
	& \leq c_{\epsilon} \frac{2 r b}{\log r} \log \left(\frac{2 r b}{\log r}\right) \\
	& \leq 4 c_{\epsilon} b \frac{r}{\log r} \log b \log r \\
	& = 4 c_{\epsilon} r b \log b \\
	& = l \text{.}
\end{split}
\]

From Corollary \ref{corollary-prob-e2-e1}, Remark \ref{remark-e2-probability} and Lemma \ref{lemma-f-increasing} it follows that
\[
\begin{split}
\Prob{E_1}
	& \leq \frac{1}{1 - \epsilon} \Prob{E_2} \\
	& \leq \frac{1}{1 - \epsilon} d ^ {-\log d - \log \log d} \\ 
	& \leq \frac{1}{1 - \epsilon} \left(\frac{r}{\log r}\right)^{-\log \left(\frac{r}{\log r}\right) - \log \log \left(\frac{r}{\log r}\right)} \text{.}
\end{split}
\]

According to Lemma \ref{lemma-e1-lpsl-equivalence} the event $E_1$ occurs if and only if $\lpsl > l$. The proof is completed by writing down the facts obtained so far
\[
\begin{split}
\Prob{\lpsl > l} 
	& = \Prob{\lpsl > 4c_{\epsilon} r b \log b} \\
	& = \Prob{E_1(S, T, 4c_{\epsilon} r b \log b)} \\
	& \leq \frac{1}{1 - \epsilon} \left(\frac{r}{\log r}\right)^{-\log \left(\frac{r}{\log r}\right) - \log \log \left(\frac{r}{\log r}\right)} \text{.}
\end{split}
\]
\end{proof}

We modify the previous remark for the set $S$ having $|S| = \alpha 2 ^ b$.
\begin{remark}
\label{remark-lpsl-pdf-linear-amount}
Let $T: \vecspace{u} \rightarrow \vecspace{b}$ be a random uniformly chosen linear mapping, $\epsilon \in (0, 1)$, $r \geq 4$ and $\alpha \in (0.5, \frac{\log r}{2})$. If $S$ is a subset of $\vecspace{u}$ such that $|S| = \alpha 2 ^ b$, then
\[
\begin{split}
& \Prob{\lpsl > 4 c_\epsilon \alpha r b \log b} \leq \frac{1}{1 - \epsilon} \left(\frac{r}{\log r}\right)^{-\log \alpha - \log \left(\frac{r}{\log r}\right) - \log \log \left(\frac{r}{\log r}\right)} \text{and} \\
& \Prob{\lpsl > 2 \alpha c_\epsilon r} \leq \frac{1}{1 - \epsilon}\left(\frac{r}{\log r}\right)^{\log \log m - \log \alpha - \log \left(\frac{r}{\log r}\right) - \log \log \left(\frac{r}{\log r}\right)} \text{.}
\end{split}
\]
\end{remark}
\begin{proof}
We do not repeat the whole proof of Remark \ref{remark-probability-long-chain} since the approach remains the same. Like in the previous proof, fix the linear mappings $T_0$, surjective $T_1$ with $T = T_1 \circ T_0$ and let $\vecspace{f}$ be the factor space. We just show the choices that prove the remark. The choices for the first claim are indexed with $1$ and in the second case we use the index $2$. When we refer to a chosen variable without an index, then the statement, in which it appears, must be valid for either choice.
Now perform the choices by setting
\[
\begin{split}
	f_1 & = \left\lfloor b + \log b + \log r - \log \log r + \log \alpha + 1 \right\rfloor \\
	f_2 & = \left\lfloor b + \log r - \log \log r + \log \alpha + 1 \right\rfloor \\
	d_1 & = \frac{2 ^ {f_1}}{\alpha b 2 ^ b} \\
	d_2 & = \frac{2 ^ {f_2}}{\alpha 2 ^ b} \\
	l_1 & = 4 c_\epsilon \alpha r b \log b \\
	l_2 & = 2 \alpha c_\epsilon r \text{.}
\end{split}
\]

So that $T_1$ exists, we have to verify that $|\vecspace{f}| \geq |\vecspace{b}|$. It is enough to have $f \geq b$. The inequality $f \geq b$ follows from the fact $\log r - \log \log r + \log \alpha \geq 0$ and from our choice of $f$ since
\[
\begin{split}
	f_1 & \geq b + \log b + \log r - \log \log r + \log \alpha \geq b \\
	f_2 & \geq b + \log r - \log \log r + \log \alpha \geq b \text{.}
\end{split}
\]

Secondly we show that the assumptions of Statement \ref{statement-e2-probability-linear-good} are met for both choices of $d$. Precisely we need $d > 1$ and this is satisfied because
\[
\begin{split}
& d_1 = \frac{2 ^ {f_1}}{\alpha b 2 ^ b} \geq \frac{r}{\log r} \geq 2 \\
& d_2 = \frac{2 ^ {f_2}}{\alpha 2 ^ b} \geq \frac{r}{\log r} \geq 2 \text{.}
\end{split}
\]

Naturally we want to meet the condition placed on the variable $l$ of Corollary \ref{corollary-prob-e2-e1}. Recall the assumption $\alpha < \frac{\log r}{2}$ that is used in either case. Let us discuss the first case
\[ 
	f_1 - b \leq \log b + \log r - \log \log r + \log \alpha + 1 \leq \log b + \log r \leq 2 \log b \log r \text{.} 
\] Thus for the value of the variable $l_1$ we have that 
\[
\begin{split}
c_{\epsilon} (f_1 - b) 2 ^ {f_1 - b}
	& < c_{\epsilon} \left(\frac{2 \alpha b r}{\log r} \right) \left(2 \log b \log r \right) \\
	& \leq 4 c_{\epsilon} \alpha r b \log b \\
	& = l_1 \text{.}
\end{split}
\]
The validity in the second case follows from
\[
\begin{split}
c_\epsilon (f_2 - b) 2 ^ {f_2 - b}
	& \leq c_\epsilon \left(\frac{2 \alpha r}{\log r}\right) \log \left( \frac{2 \alpha r}{\log r} \right) \\
	& \leq 2 c_\epsilon \alpha \frac{r}{\log r} \log r \\
	& = 2 c_\epsilon \alpha r = l_2 \text{.}
\end{split}
\]


Thus we are able to refer to Corollary \ref{corollary-prob-e2-e1}. Now use Lemma \ref{lemma-e1-lpsl-equivalence}, Corollary \ref{corollary-prob-e2-e1}, Remark \ref{remark-e2-probability} and Lemma \ref{lemma-f-increasing} in either case. In the first one we have
\[
\begin{split}
\Prob{\lpsl > 4 c_{\epsilon} \alpha r b \log b} 
	& = \Prob{\lpsl > l_1} \\
	& = \Prob{E_1(S, T, l_1)} \\
	& \leq \frac{1}{1 - \epsilon}{\Prob{E_2}} \\
	& \leq \frac{1}{1 - \epsilon} d ^ {-\log \alpha - \log d - \log \log d} \\
	& = \frac{1}{1 - \epsilon} \left(\frac{r}{\log r}\right)^{-\log \alpha - \log \left(\frac{r}{\log r}\right) - \log \log \left(\frac{r}{\log r}\right)} \text{.}
\end{split}
\]
And in the second case we have
\[ 
\begin{split}
\Prob{\lpsl > 2 \alpha c_\epsilon r} 
	& = \Prob{\lpsl > l_2} \\
	& = \Prob{E_1(S, T, l_2)} \\
	& \leq \frac{1}{1 - \epsilon}{\Prob{E_2}} \\
	& \leq \frac{1}{1 - \epsilon} d ^ {\log b - \log \alpha - \log d - \log \log d} \\
	& \leq \frac{1}{1 - \epsilon} \left(\frac{r}{\log r}\right)^{\log b - \log \alpha - \log \left(\frac{r}{\log r}\right) - \log \log \left(\frac{r}{\log r}\right)} \text{.}
\end{split}
\]
\end{proof}
