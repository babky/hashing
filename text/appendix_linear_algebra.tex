\begin{chapter}{Facts from linear algebra}
\begin{definition}[Affine subspace]
Let $V$ be a vector space, $U \subseteq V$ be its subspace and $\vec{x} \in V$. Set $A = \vec{x} + V$ is called an affine subspace of $V$.
\end{definition}

\begin{definition}[Null-space of matrix]
Let $A$ be a matrix from $T ^ {m \times n}$ where $T$ is a field and $n, m \in \mathbb{N}$. Set $\{ \vec{x} \in T^{n} \setdelim A\vec{x} = \vec{0} \}$ is called the null-space of matrix $A$.
\end{definition}

\begin{definition}[Orthogonal complement]
Let $V$ be a vector subspace of a vector space $U$. Orthogonal complement of subspace $V ^ {\bot}$ is defined by:
\[
	V ^ {\bot} = \{ \vec{u} \in U \setdelim \langle \vec{u} | \vec{v} \rangle = 0 \text{ for all } \vec{v} \in V \} \text{.}
\]
\end{definition}

\begin{remark}
Null-space of every matrix $A \in T ^ {m \times n}$ is a subspace of vector space $T^{n}$.
\end{remark}
\begin{proof}
Proof is a straightforward verification of the three properties of vector subspaces.

Zero vector is contained in $\nullspace{A}$ because $A\vec{0} = \vec{0}$.

For every $\vec{x}, \vec{y} \in \nullspace{A}$ their sum $\vec{x} + \vec{y} \in \nullspace{A}$.
\[
	A(\vec{x} + \vec{y}) = A\vec{x} + A\vec{y} = \vec{0} + \vec{0} = \vec{0}
\]

For every $\vec{x} \in \nullspace{A}$ and $t \in T$ we have $t \vec{x} \in \nullspace{A}$.
\[
	At\vec{x} = tA\vec{x} = t\vec{0} = \vec{0}
\]
\end{proof}

\begin{definition}[Affine linear map]
Let $V, U$ be vector spaces and $V_0 \leq V$, $U_0 \leq U$ be their subspaces. Let $V_A = \vec{v} + V_0$ and $U_A = \vec{u} + U_0$ be affine subspaces of $V$ and $U$ respectively for some vectors $\vec{v} \in V$, $\vec{u} \in U$. Function $T_A: V_A \rightarrow U_A$ is called affine linear map if there is a linear map $T_0: V_0 \rightarrow U_0$ such that $T_A(x) = \vec{u} + T_0(x - \vec{v})$.
\end{definition}

\begin{definition}[Set of all affine linear map]
Let $V, U$ be vector spaces and $V_0 \leq V$, $U_0 \leq U$ be their subspaces. Let $V_A = \vec{v} + V_0$ and $U_A = \vec{u} + U_0$ be affine subspaces of $V$ and $U$ respectively for $\vec{v} \in V$, $\vec{u} \in U$. Set of all affine linear mapping between affine spaces $V_A$ and $U_A$, $LT_A(V_A, U_A)$, is defined as:
\[
	LT_A(V_A, U_A) = \{ T: V_A \rightarrow U_A \setdelim T \text{ is an affine map} \} \text{.}
\]
\end{definition}

\begin{lemma}
\label{lemma-linear-transformation-domain-distribution}
Let $T: \vecspace{u} \rightarrow \vecspace{t}$, $u \geq t$ be an~onto linear map and vector $\vec{x} \in \vecspace{t}$. Then $|T^{-1}(\vec{x})| = 2 ^ {u - t}$ and moreover set $T^{-1}(\vec{x})$ is an affine subspace of $\vecspace{u}$.
\end{lemma}
\begin{proof}
At first remark that $T^{-1}(0)$ is a~subspace of~space $\vecspace{u}$.

Set $T^{-1}(0)$ contains zero.
\[ T(0) = 0 \]

It is closed under addition.
\[
\begin{split}
\vec{u}, \vec{v} \in T^{-1}(0) 
	& \Rightarrow T(\vec{u}) = 0, T(\vec{v}) = 0 \\ 
	& \Rightarrow T(\vec{u} + \vec{v}) = T(\vec{u}) + T(\vec{v}) = 0 \\ 
	& \Rightarrow \vec{u} + \vec{v} \in T^{-1}(0) \\
\end{split}
\]

It is closed under scalar multiplication.
\[
\begin{split}
\delta \in \{0, 1\} \wedge \vec{u} \in T^{-1}(0) 
	& \Rightarrow T(\delta \vec{u}) = \delta T(\vec{u}) = \delta 0 = 0 \\
	& \Rightarrow \delta \vec{u} \in T^{-1}(0) \\
\end{split}
\]

$T^{-1}(\vec{x})$ is an~affine subspace since $T^{-1}(\vec{x}) = u + T^{-1}(0)$. The~consequence is that all sets $T^{-1}(\vec{x})$ have the~same size, exactly $|T^{-1}(\vec{x})| = 2^{u-t}$. For every vector $\vec{u} \in T^{-1}(\vec{x})$ the~set $T^{-1}(\vec{x})$ is the~same as $\vec{u} + T^{-1}(0)$. This is also the one to one map between $T^{-1}(0)$ and $T^{-1}(\vec{x})$.
\begin{displaymath}
\begin{split}
\vec{v} \in T^{-1}(0) 
	& \Rightarrow T(\vec{u} + \vec{v}) = T(\vec{u}) + T(\vec{v}) = \vec{x} + 0 = \vec{x}  \\
	& \Rightarrow \vec{u} + T^{-1}(0) \subseteq T^{-1}(\vec{x})
\end{split}
\end{displaymath}
\begin{displaymath}
\begin{split}
\vec{v} \in T^{-1}(\vec{x}) 
	& \Rightarrow T(\vec{v} - \vec{u}) = T(\vec{v}) - T(\vec{u}) = \vec{x} - \vec{x} = 0 \\
	& \Rightarrow \vec{v} - \vec{u} \in T^{-1}(0) \\
	& \Rightarrow T^{-1}(\vec{x}) \subseteq \vec{u} + T^{-1}(0)
\end{split}
\end{displaymath}
\end{proof}

\begin{lemma}
\label{lemma-system-of-linear-equations-solution-count}
Let $A \in \mathbb{Z}_2^{m \times n}$ be a matrix such that $\rank{A} = m$ and $m \leq n$. Then system of linear equations $A\vec{x} = \vec{y}$ has $2 ^ {n - m}$ solutions for every $\vec{y} \in \vecspace{m}$.
\end{lemma}
\begin{proof}
Consider homogeneous system $A\vec{x} = \vec{0}$. It is a well-known fact that the dimension of the right null-space of the matrix $A$, denoted by $\nullspace{A}$, is equal to \[ \dimension{\nullspace{A}} =n - \rank{A} = n - m = \text{.} \] This fact implies that $|\nullspace{A}| = 2 ^ {n - m}$.

Because $\rank{A} = m$ then for every vector $\vec{y} \in \vecspace{m}$ there is at least one solution $\vec{x} \in \vecspace{n}$ of the non-homogeneous system $A\vec{x} = \vec{y}$. It is clear that the set of all solutions of the non-homogeneous system is an affine subspace of $V$ equal to $\vec{x} + \nullspace{A}$. The number of all solutions of non-homogeneous system for every vector $\vec{y}$ is then $2 ^ {n - m}$, too.
\end{proof}
\end{chapter}
