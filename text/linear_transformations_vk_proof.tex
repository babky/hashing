The idea of the proof is to factorise the transformation $T$ through the factor space $F$ into two linear functions $T_0$ and $T_1$ such that $T = T_1 \circ T_0$. To successfully use the estimates following from the previous theorems we need a bound on the input $\mu$. The bound may be obtained by using the Law of Total Probability, Theorem \ref{theorem-law-of-total-probability},
\[
\begin{split}
& \Prob{T(S) = \vecspace{b}} \\
& \qquad = \Prob{T(S) = \vecspace{b} \wedge |T_0(S)| \leq \frac{|S|}{2}} + \Prob{T(S) = \vecspace{b} \wedge |T_0(S)| > \frac{|S|}{2}} \text{.}
\end{split}
\]
Technically we estimate the probability of the complementary event and use the Markov's inequality to estimate the expression $\Prob{|T_0(S)| \leq \frac{|S|}{2}}$.

First set $f = \lceil\log(\frac {2|S|}{\epsilon})\rceil$. Let $T_1: \vecspace{f} \rightarrow \vecspace{b}$ be a random uniformly chosen surjective linear mapping. Since $c_\epsilon$ is later chosen large enough, we have that $f \geq b$ and thus such onto mapping $T_1$ exists. Fix it. Secondly, perform the random uniform choice of a linear mapping $T_0: \vecspace{u} \rightarrow \vecspace{f}$. Fix the function $T_1$. From Model \ref{remark-model-uniform-linear-map-selection} it follows that the linear mapping $T = T_1 \circ T_0$ is chosen randomly and uniformly, too.

Since the family of all linear mappings from $\vecspace{u}$ into $\vecspace{f}$ is $1$-universal we conclude that \[ \Prob{T_0(\vec{x}) = T_0(\vec{y})} = 2 ^ {-f} \] for all distinct vectors $\vec {x}$ and $\vec {y}$ from the vector space $\vecspace{u}$. If $d_S$ is the number of  all pairs of distinct vectors $\vec {x},\vec {y}\in S$ with $T_0(\vec {x}) = T_0(\vec {y})$, then the expected value of the random variable $d_S$ is \[ \Expect{d_S}= \binom{|S|}{2} 2 ^ {-f} \text{.} \]

If $|T_0(S)| \leq \frac {|S|}{2}$ then there exist at least $\frac {|S|}{2}$ pairs of distinct vectors $\vec{x}, \vec{y} \in S$ with $T_0(\vec {x}) = T_0(\vec {y})$. By Markov's inequality, Theorem \ref{theorem-markov-inequality}, \[ \Prob{d_S \geq k\binom {|S|}{2} 2^{-f}} \leq \frac{1}{k} \text{.} \]
Thus if we set $k = \frac {|S|2 ^ f}{2\binom {|S|}{2}}$, then we obtain 
\[ 
	\Prob{|T_0(S)| \leq \frac {|S|}{2}} 
		\leq \Prob{d_S \geq \frac {|S|}{2}} 
		\leq \frac{2 \binom{|S|}{2}}{|S|2 ^ f} = \frac{|S| - 1}{2 ^ f} < \frac{|S|}{2 ^ f} 
		\leq \frac{\epsilon |S|}{2|S|} = \frac{\epsilon}{2}
\]
We can summarise that
\[ 
\Prob{T(S) \neq \vecspace{b} \wedge |T_0(S)| \leq \frac{|S|}{2}} \leq \frac{\epsilon}{2} \text{.}
\]

Secondly we compute $\Prob{T(S) \neq \vecspace{t} \wedge |T_0(S)| > \frac{|S|}{2}}$. By Theorem \ref{theorem-linear-function-set-onto} used for the mapping $T_1:~\vecspace{f}~\rightarrow~\vecspace{b}$ and the set $T_0(S) \subseteq \vecspace{f}$ we have
\[
	\Prob{T(S) = T_1(T_0(S)) \neq \vecspace{T} \wedge |T_0(S)| > \frac{|S|}{2}} \leq \mu ^ {u - t - \log t  +\log\log \left(\frac{1}{\mu}\right) } 
\]
where $\mu = 1- \frac{|T_0(S)|}{2^u}$. Clearly
\[
\mu = 1 - \frac{|T_0(S)|}{2 ^ u} < 1 - \frac{|S|}{2 . 2 ^ u} \leq 1 - \frac{\epsilon |S|}{8|S|} \leq e^{-\frac{\epsilon}{8}}
\text{.}
\]
In the following constant $c_{\epsilon}$ is chosen as $4\left(\frac{2}{\epsilon}\right) ^ {\frac{8}{\epsilon}}$. Then we can estimate:
\[
\begin{split}
& -\frac{\epsilon}{8} \left(u - t - \log t + \log \log \left( \frac{1}{\mu} \right) \right) \\
& \qquad = -\frac{\epsilon}{8}\left(\left\lceil\log\left(\frac{2|S|}{\epsilon}\right)\right\rceil - t - \log t + \log\log\left(\frac{1}{\mu}\right)\right) \\
& \qquad \leq -\frac{\epsilon}{8} \left( \left\lceil \log\left( \frac{8 \left(\frac{2}{\epsilon}\right) ^ {\frac{8}{\epsilon}}t2^t}{\epsilon} \right) \right\rceil - t - \log t + \log\log\left(\frac{1}{\mu}\right)\right) \\
& \qquad \leq -\frac{\epsilon}{8} \left(3 + \frac{8}{\epsilon}\log\frac{2}{\epsilon} - \log\epsilon + \log t + t - t - \log t + \log \left( \left(\frac{\epsilon}{8}\right)\log e \right) \right) \\
& \qquad = -\frac{\epsilon}{8} \left(3 - \log\epsilon + \frac{8}{\epsilon} \log \frac{2}{\epsilon} + \log\epsilon - 3 + \log\log e\right) \\
& \qquad = -\frac{\epsilon}{8}\left(\frac{8}{\epsilon}\log \frac{2}{\epsilon} + \log\log e\right) \\
& \qquad = \log \frac{\epsilon}{2} - \frac{\epsilon}{8} \log\log e \\
& \qquad \leq \log\frac{\epsilon}{2} \text{.}
\end{split}
\]
And for the calculated probability we derive:
\[
\begin{split}
\Prob{T(S) \neq \vecspace{t} \wedge |T(S)| > \frac{|S|}{2}} 
	& \leq \mu ^ {u - t - \log t + \log\log\left(\frac{1}{\mu}\right)} \\
	& \leq e ^ {-\frac{\epsilon\left(u - t - \log t + \log\log\left(\frac{1}{\mu}\right)\right)}{8}} \\
	& \leq e ^ {\log \left( \frac{\epsilon}{2} \right)} \leq e ^ {\ln \left(\frac{\epsilon}{2}\right)} = \frac{\epsilon}{2} \text{.}
\end{split}
\]

If we put both alternatives together, we deduce that 
\[ 
	\Prob{T(S) = T_1(T_0(S)) \neq \vecspace{t}} \leq \frac{\epsilon}{2} + \frac{
\epsilon}{2} = \epsilon \text{.}
\]


From this observation follows the required estimate: \[ \Prob{T(S) = \vecspace{t}} \geq 1 - \epsilon \text{.} \]
