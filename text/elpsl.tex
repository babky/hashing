\chapter{Očakávaná dĺžka najdlhšieho reťazca}

Pokúsime sa odhadnúť očakávanú dĺžku najdlhšieho reťazca pri univerzálnom hashovaní. Vychádzame z postupu použitého pri štandardnom hashovaní, ktorý neskôr upravíme. Pri tomto prístupe sme očakávanú dĺžku odhadovali pomocou pravdepodobnosti, že v jedinom reťazci koliduje $i$ a viac prvkov. Vychádzali sme z nasledujúcich dvoch faktov:
\begin{displaymath}
P(lpsl \geq i) \leq m P(psl \geq i)
\end{displaymath}

\begin{displaymath}
\begin{split}
E lpsl	& = \displaystyle \sum_{i=0}^{\infty} i P(lpsl = i) \\
		& = \displaystyle \sum_{i = 0}^{\infty} i (P(lpsl \geq i) - P(lpsl \geq i + 1)) \\ 
		& = \displaystyle \sum_{i = 0}^{\infty} P(lpsl \geq i)
\end{split}
\end{displaymath}

Teda platí:
\begin{displaymath}
E lpsl \leq \displaystyle \sum_{i = 0}^{\infty} m P(psl \geq i)
\end{displaymath}

Na základe predpokladu nezávislosti výberu hashovaných prvkov z univerza odhadujeme pravdepodobnosť kolízie ako:
\begin{displaymath}
P(psl \geq i) \leq \dbinom{n}{i}\left(\frac{1}{m}\right)^i\left(1 - \frac{1}{m}\right)^{n-i}
\end{displaymath}

Dopočítame odhad očakávanej dĺžky najdlhšieho reťazca:
\begin{displaymath}
\begin{split}
E lpsl	& \leq \displaystyle \sum_{i = 0}^{\infty} m P(psl \geq i) \\
		& \leq \displaystyle \sum_{i = 0}^{\infty} m \dbinom{n}{i}\left(\frac{1}{m}\right)^i\left(1 - \frac{1}{m}\right)^{n-i} \\
		& = O(\frac{\log n}{\log \log n})
\end{split}
\end{displaymath}

Teraz upravíme výpočet pre univerzálne hashovanie. Zásadný rozdiel je v odhade pravdepodobnosti kolízie $i$ prvkov. Tu sa oprieme o rôzne definície univerzálnych systémov hashovacích funkcií a o špecifické vlastnosti jednotlivých sád funkcií.

\begin{definition}[Indikátor]
\begin{displaymath}
I: P(U) \rightarrow \lbrace 0, 1 \rbrace
\end{displaymath}
\begin{displaymath}
I(M) = \left\{ 
\begin{array}{l l}
  0 & \quad M = \emptyset \\
  1 & \quad M \neq \emptyset \\
\end{array} \right.
\end{displaymath}
\end{definition}

Z predchádzajúcej definície je vidieť, že platí $I(M) \leq |M|$.

\paragraph{}
Ďalej sa pokúsime načrtnúť výpočet očakávanej dĺžky najdlhšieho reťazca pre univerzálne hashovanie. Nech $h \in H$, $k \in N_0$ a $\tilde{x} \in A$, potom pre
\begin{displaymath}
\begin{split}
M_{\geq k} & = \{Y \subseteq S | |Y| \geq k, h(Y) = \{\tilde{x}\}\} \\
M_{= k} & = \{Y \subseteq S | |Y| = k, h(Y) = \{\tilde{x}\}\} 
\end{split}
\end{displaymath}
platí:
\begin{displaymath}
I(M_{\geq k}) = I(M_{= k})
\end{displaymath}
Chceme ukázať, že neprázdnosť množiny $M_{\geq k}$ je ekvivalentná s neprázdnosťou $M_{= k}$. Nech $M_{\geq k}$ je neprázdna:
\begin{displaymath}
Y \in M_{\geq k} \Rightarrow \forall Y' \subseteq Y, |Y'| = k: h(Y') = \{\tilde{x}\} \Leftrightarrow \forall Y' \subseteq Y, |Y'| = k: Y' \in M_{=k}
\end{displaymath}
Nech $M_{=k}$ je neprázdna:
\begin{displaymath}
Y' \in M_{=k} \Rightarrow Y' \in M_{\geq k}
\end{displaymath}
Máme:
\begin{displaymath}
\begin{split}
M_{\geq k} \neq \emptyset & \Leftrightarrow  M_{= k} \neq \emptyset \\
I(M_{\geq k}) & = I(M_{= k})
\end{split}
\end{displaymath}

Zvolíme reťazec obsahujúci prvok $\tilde{x}$ a jeho dĺžku označíme $l_{\tilde{x}}$. Ak má príslušný reťazec dĺžku aspoň $k$, potom musí existovať podmnožina $Y$ hashovanej množiny $S \subset U$ taká, že použitá funkcia $h$ je na nej konštantne rovná $\tilde{x}$. Aby bola dĺžka zvoleného reťazca presne $k$, množina $Y$ musí byť maximálna čo do počtu prvkov.
\begin{displaymath}
\begin{split}
P(l^{\tilde{x}} \geq k) & = \frac{\sum\displaylimits_{h \in H} I(\{ Y \subseteq S | |Y| \geq k, h(Y) = \{ \tilde{x} \} \})}{|H|} \\
P(l^{\tilde{x}} = k) & = \frac{\sum\displaylimits_{h \in H} I(\{ Y \subseteq S | |Y| = k, h(Y) = \{ \tilde{x} \}, Y \text{ maximálna} \})}{|H|} \\
\end{split}
\end{displaymath}

K tomuto výpočtu je nutné dodať niekoľko užitočných poznámok:
\begin{itemize}
\item Reťazec máme vždy určený jediným prvkom $\tilde{x}$, ktorý do neho patrí.
\item Pravdepodobnosť počítame pre jedinú fixovanú hashovanú množinu $S$ a pravdepodobnostný priestor odpovedá voľbe funkcie $h$.
\item Získame odhad $P(l^{\tilde{x}} \geq k) \leq p(A, U)$. Tento odhad použijeme obdobne ako pri výpočte v klasickom hashovaní na odhad pravdepodobnosti pre dĺžku najdlhšieho reťazca: $P(lpsl \geq k) \leq |A| p(A, U)$.
\end{itemize}
\begin{displaymath}
\begin{split}
P(l^{\tilde{x}} \geq k) 
	& = \frac{\sum\displaylimits_{h \in H} I(\{ Y \subseteq S | |Y| \geq k, h(Y) = \{ \tilde{x} \} \})}{|H|} \\
	& = \frac{\sum\displaylimits_{h \in H} I(\{ Y \subseteq S | |Y| = k, h(Y) = \{ \tilde{x} \} \})}{|H|} \\
	& \leq \frac{\sum\displaylimits_{h \in H} |\{ Y \subseteq S | |Y| = k, h(Y) = \{ \tilde{x} \} \}|}{|H|} \\
	& = \frac{|\{ (h, Y) | |Y| = k, h(Y) = \{ \tilde{x} \} \}|}{|H|} \\
	& = \frac{\sum\displaylimits_{Y \subseteq S, |Y| = k} | \{ h \in H | h(Y) = \{\tilde{x}\} \}|}{|H|} \\
\end{split}
\end{displaymath}
Tento odhad silne závisí na vlastnostiach množiny funkcií $H$.
\begin{theorem}
Pokiaľ $H$ je silne $\omega$-univerzálny systém, očakávaná dĺžka najdlhšieho reťazca pri univerzálnom hashovaní je $O(\frac{\log n}{\log \log n})$.
\end{theorem}
\begin{proof}
\begin{displaymath}
\begin{split}
P(l^{\tilde{x}} \geq k) 
	& \leq \frac{\sum\displaylimits_{Y \subseteq S, |Y| = k} | \{ h \in H | h(Y) = \{\tilde{x}\} \}|}{|H|} \\
	& = \frac{\sum\displaylimits_{Y \subseteq S, |Y| = k} \frac{|H|}{{|A|}^{k}}}{|H|} \\
	& = \dbinom{|S|}{k}\frac{1}{{|A|}^k}
\end{split}
\end{displaymath}
Tu môžeme pokračovať presne ako v prípade klasického hashovania, pretože odhad pravdepodobnosti kolízie $k$ prvkov vyšiel úplne rovnako. Dopracujeme sa k rovnakému výsledku $O(\frac{\log n}{\log \log n})$.
\end{proof}
Získať malý silne $\omega$-univerzálny systém je problém. Musíme použiť iné systémy a pri odhade sa opierať o ich špecifické vlastnosti.
