\section{The Expected Value of the Variable \texorpdfstring{$\lpsl$}{lpsl}}
In this section we estimate the expected value of the variable $\lpsl$, $\Expect{\lpsl}$. We exploit the bounds, we prove in Section \ref{section-linear-transformations-distribution}, on the cumulative probability density of the random variable $\lpsl$. We are interested in this expected value because it corresponds to the length of the longest chain when using the system of linear transformations as a universal class.

\begin{theorem}
\label{theorem-n-logn-to-n}
Let $T: \vecspace{u} \rightarrow \vecspace{b}$ be a random uniformly chosen linear transformation and $\epsilon \in (0, 1)$. If $S$ is a subset of the vector space $\vecspace{u}$ such that $|S| \leq b 2 ^ b$, then \[ \Expect{\lpsl} \leq 4 c_\epsilon(4 + I_\epsilon) b \log b \text{.} \] The value $I_\epsilon$ is defined as:
\stepcounter{definition}
\begin{equation}
\label{equality-i-e}
I_\epsilon = \int\limits_4^\infty \frac{1}{1 - \epsilon} \left(\frac{r}{\log r}\right)^{-\log \left(\frac{r}{\log r}\right) - \log \log \left(\frac{r}{\log r}\right)} dr \text{.}
\end{equation}
\end{theorem}
\begin{proof}
First set $K_\epsilon = 4 c_\epsilon b \log b$. In order to prove the theorem we use the probability distribution of the variable $\lpsl$ from Remark \ref{remark-probability-long-chain}. For every $r \geq 4$ it states that 
\[ 
\Prob{\lpsl \geq r K_\epsilon} \leq \frac{1}{1 - \epsilon}\left(\frac{1}{\log r}\right) ^ {-\log\left(\frac{r}{\log r}\right) - \log \log \left(\frac{r}{\log r}\right)} \text{.} 
\]

Lemma \ref{lemma-expected-value-cumulative} implies 
\[
\begin{split}
\Expect{\lpsl}
	& = \int\limits_0^{\infty} \Prob{\lpsl > l} dl \\
	& \leq 4K_\epsilon + \int\limits_{4K_\epsilon}^\infty \Prob{\lpsl > l} dl \\
	& = 4K_\epsilon + K_\epsilon \int\limits_4^\infty \Prob{\lpsl > rK_\epsilon} dr \\
	& \leq 4K_\epsilon + K_\epsilon \int\limits_4^\infty \frac{1}{1 - \epsilon} \left(\frac{r}{\log r}\right)^{-\log \left(\frac{r}{\log r}\right) - \log \log \left(\frac{r}{\log r}\right)} dr \\
	& = K_\epsilon(4 + I_\epsilon) \in O(K_\epsilon) = O(b \log b) \text{.}
\end{split}
\]

The fact that the integral $I_\epsilon$ is convergent for every $\epsilon \in (0, 1)$ is shown later in Lemma \ref{lemma-convergent-I-e}.
\end{proof}

Multiplicative constant $4 c_\epsilon(4 + I_\epsilon)$ plays an important role for a practical use of the result. For example when choosing $\epsilon$ equal to $\frac{1}{2}$ the value of the constant $c_\epsilon$ equals $4 ^ {17}$ when using the original estimate. Our next goal is to show a better estimate when assuming that $\alpha \leq 1$ and using our results.

\begin{theorem}
\label{theorem-n-to-n}
Let $T: \vecspace{u} \rightarrow \vecspace{b}$ be a random uniformly chosen linear transformation, $\epsilon \in (0, 1)$ and $\alpha \in \left[0.5, 1\right]$. If $S$ is a subset of the vector space $\vecspace{u}$ such that $|S| = \alpha 2 ^ b$, then \[ \Expect{\lpsl} \leq 4 \alpha c_\epsilon(4 + I_\epsilon) b \log b \text{.} \] The value $I_\epsilon$ is defined in Equality \ref{equality-i-e}.
\end{theorem}
\begin{proof}
Set $K_\epsilon = 4 c_\epsilon b \log b$. When estimating $\Prob{\lpsl > r \alpha K_\epsilon}$ we use Remark \ref{remark-lpsl-pdf-linear-amount}. Its assumptions are satisfied since we use it for $r \geq 4$ and $\alpha \leq 1 \leq \frac{\log r}{2}$. It states that
\[
	\Prob{\lpsl > r \alpha K_\epsilon} \leq \frac{1}{1 - \epsilon} \left(\frac{r}{\log r}\right)^{-\log \alpha - \log \left(\frac{r}{\log r}\right) - \log \log \left(\frac{r}{\log r}\right)} \text{.}
\]
Since $\alpha \leq 1$, then $\log \alpha \leq 0$, and it follows that
\[
	\Prob{\lpsl > r \alpha K_\epsilon} \leq \frac{1}{1 - \epsilon} \left(\frac{r}{\log r}\right)^{- \log \left(\frac{r}{\log r}\right) - \log \log \left(\frac{r}{\log r}\right)} \text{.}
\]

Now we compute the expected value using Lemma \ref{lemma-expected-value-cumulative} as
\[
\begin{split}
\Expect{\lpsl}
	& = \int\limits_0^\infty \Prob{\lpsl > l} dl \\
	& \leq 4 \alpha K_\epsilon + \int\limits_{4 \alpha K_\epsilon}^\infty \Prob{\lpsl > l} dl \\
	& = 4 \alpha K_\epsilon + \alpha K_\epsilon \int\limits_4^\infty \Prob{\lpsl > r \alpha K_\epsilon} dr \\
	& \leq 4 \alpha K_\epsilon + \alpha K_\epsilon \int\limits_4^\infty \frac{1}{1 - \epsilon} \left(\frac{r}{\log r}\right)^{- \log \left(\frac{r}{\log r}\right) - \log \log \left(\frac{r}{\log r}\right)} dr \\
	& = \alpha K_\epsilon(4 + I_\epsilon) \in O(\alpha K_\epsilon) = O(\alpha b \log b) \text{.}
\end{split}
\]
\end{proof}
\end{section}

\section{The Achieved Bound}

The improved estimate of $c_\epsilon$ combined with the last claim of Remark \ref{remark-lpsl-pdf-linear-amount} are used together to show a tighter bound on $\Expect{\lpsl}$.

\begin{theorem}
\label{theorem-hashing-linear-amount}
Let $b \in \mathbb{N}$, $b \geq 4$, $T: \vecspace{u} \rightarrow \vecspace{b}$ be a random uniformly chosen linear transformation, $\epsilon \in (0, 1)$ and $\alpha \in \left[0.5, 1\right]$. If $S$ is a subset of the vector space $\vecspace{u}$ such that $|S| = \alpha 2 ^ b$, then \[ \Expect{\lpsl} \leq \frac{4 c_\epsilon \alpha}{1 - \epsilon} b \log b + 2 c_\epsilon \alpha \left( \frac{J_{\alpha, b} - 4}{1 - \epsilon} + 4 \right) \text{.} \] The value $J_{\alpha, b}$ is defined as
\stepcounter{definition}
\begin{equation}
\label{equality-j-a-b}
J_{\alpha, b} = \int\limits_{2 b \log b}^\infty \left(\frac{r}{\log r}\right)^{\log b - \log \alpha - \log \left(\frac{r}{\log r}\right) - \log \log \left(\frac{r}{\log r}\right)} dr \text{.}
\end{equation}
\end{theorem}
\begin{proof}
From Remark \ref{remark-lpsl-pdf-linear-amount}, Lemma \ref{lemma-expected-value-cumulative} and Lemma \ref{lemma-estimate-j-a} it follows that
\[
\begin{split}
& \Expect{\lpsl}
	= \int\limits_0^\infty \Prob{\lpsl \geq l} dl \\ 
	& \qquad \leq 8 c_\epsilon \alpha + \int\limits_{8 c_\epsilon \alpha}^\infty \Prob{\lpsl \geq l} dl \\ 
	& \qquad = 8c_\epsilon \alpha + 2 c_\epsilon \alpha \int\limits_{4}^\infty \Prob{\lpsl \geq 2c_\epsilon \alpha r} dr \\
	& \qquad = 2c_\epsilon \alpha \left(4 + \frac{1}{1 - \epsilon} \int\limits_{4}^\infty \min\left(1, \left(\frac{r}{\log r}\right)^{\log b - \log \alpha - \log \left(\frac{r}{\log r}\right) - \log \log \left(\frac{r}{\log r}\right)}\right) dr \right) \\ 
	& \qquad \leq 2c_\epsilon \alpha \left( 4 + \frac{1}{1-\epsilon}\left( 2 b \log b - 4 + J_{\alpha, b} \right) \right) \text{.}
\end{split}
\]

Let the value $J'_{\alpha, b}$ denote the integral,
\stepcounter{definition}
\begin{equation}
\label{equality-j-prime-a-b}
J'_{\alpha, b} = \int\limits_{4}^\infty \min\left(1, \left(\frac{r}{\log r}\right)^{\log b - \log \alpha - \log \left(\frac{r}{\log r}\right) - \log \log \left(\frac{r}{\log r}\right)}\right) dr \text{.}
\end{equation}

Later, in Lemma \ref{lemma-estimate-j-a} we estimate the integral $J'_{\alpha, b}$ by $2 b \log b - 4 +  J_{\alpha, b}$, where

The whole bound on the expected value, $\Expect{\lpsl}$, then looks like:
\[
\Expect{\lpsl} \leq \frac{4 c_\epsilon \alpha}{1 - \epsilon} b \log b + 2 c_\epsilon \alpha \left( \frac{J_{\alpha, b} - 4}{1 - \epsilon} + 4 \right) \text{.}
\]

In advance, let us note that from Lemma \ref{lemma-estimate-j-a} it follows that $J_{\alpha, b} \leq 3.36$. The bound thus exists for every $\alpha \in [0.5, 1]$.
\end{proof}

\begin{corollary}
\label{corollary-best-elpsl}
Let $b \in \mathbb{N}$, $b \geq 4$, $T: \vecspace{u} \rightarrow \vecspace{b}$ be a random uniformly chosen linear transformation and $\alpha \in \left[0.5, 1\right]$. If $S$ is a subset of the vector space $\vecspace{u}$ such that $|S| = \alpha 2 ^ b$, then \[ \Expect{\lpsl} \leq 538 \alpha b \log b + 44 \text{.} \] 
\end{corollary}
\begin{proof}
The best estimate on the expected value $\Expect{\lpsl}$ is achieved by the parametrisation technique using the estimate of $c_\epsilon$ from Statement \ref{statement-better-c-e}. For the choice of $\epsilon = 0.8$ and by settings parameters $k = 2.26$, $l = 2$ of Statement \ref{statement-better-c-e} we get the required bound.
\end{proof}
