%%AmSTeX
%\documentstyle{amsppt}
\input amstex
\input amsppt.sty

\magnification=\magstep1

\input lhead


\def \emph#1{\underbar{#1}}
\def \Prob{\operatorname{Prob}}
\def \count{\operatorname{count}}
\def \NIL{\operatorname{NIL}}
\def \otec{\operatorname{otec}}
\def \list{\operatorname{list}}
\def \Prv{\operatorname{Prv}}
\def \Nasl{\operatorname{Nasl}}
\def \levy{\operatorname{levy}}
\def \pravy{\operatorname{pravy}}
\def \key{\operatorname{key}}
\def \Prst{\operatorname{Prst}}
\def \hloubka{\operatorname{hloubka}}
\def \bratr{\operatorname{bratr}}
\def \npl{\operatorname{npl}}
\def \Cont{\operatorname{Cont}}

\flushpar V praxi se \v casto setk\'av\'ame s n\'asleduj\'\i c\'\i m 
probl\'emem, kter\'y je modifikac\'\i\ slovn\'\i kov\'eho 
probl\'emu.\newline 
Zad\'an\'\i\ probl\'emu: $U$ je univerzum. Je d\'ana mno\v zina 
$S\subseteq U$ a funkce $f:S@>>>\Bbb R$, kde $\Bbb R$ jsou re\'aln\'a \v c\'\i sla.
M\'ame navrhnout reprezentaci $S$ a $f$, kter\'a umo\v z\v nuje 
operace:
\roster
\item"{}"
{\bf INSERT$\left(s,a\right)$} -- p\v rid\'a k mno\v zin\v e $S$ prvek $s$ tak, \v ze 
$f\left(s\right)=a$;
\item"{}" 
{\bf MIN} -- nalezne prvek $s\in S$ s nejmen\v s\'\i\ hodnotou 
$f\left(s\right)$;
\item"{}" 
{\bf DELETEMIN} -- odstran\'\i\ prvek $s\in S$ s nejmen\v s\'\i\ hodnotou 
$f\left(s\right)$;
\item"{}" 
{\bf DELETE$\left(s\right)$} -- odstran\'\i\ prvek $s\in S$;
\item"{}"
{\bf DECREASE$\left(s,a\right)$} -- zmen\v s\'\i\ hodnotu $f\left(s\right)$ o $a$ (tj. 
$f\left(s\right):=f\left(s\right)-a$);
\item"{}"
{\bf INCREASE$\left(s,a\right)$} -- zv\v et\v s\'\i\ hodnotu $f\left(s\right)$ o $a$ (tj. 
$f\left(s\right):=f\left(s\right)+a$).
\endroster 
P\v ri operaci {\bf INSERT$\left(s,a\right)$} se p\v redpokl\'ad\'a, \v ze $
s\notin S$, tento 
p\v redpoklad operace {\bf INSERT} neov\v e\v ruje. P\v ri operac\'\i ch {\bf DE\-LE\-TE$
\left(s\right)$},
{\bf DECREASE$\left(s,a\right)$ a INCREASE$\left(s,a\right)$} se p\v red\-pokl\'ad\'a, \v ze $
s\in S$ a 
operace nav\'\i c dost\'av\'a informaci, jak nal\'ezt $s$ a s n\'\i m 
spojen\'a data v reprezentaci $S$ a $f$. Haldy jsou typ 
struktury, kter\'a se pou\v z\'\i v\'a pro \v re\v sen\'\i\ tohoto probl\'emu. 
\medskip

\flushpar Halda je stromov\'a struktura, kde vrcholy 
reprezentuj\'\i\ prvky z $S$ a spl\v nuj\'\i\ lok\'aln\'\i\ podm\'\i nku na 
$f$. Obvykle se pou\v z\'\i v\'a n\'asleduj\'\i c\'\i\ podm\'\i nka nebo 
jej\'\i\ symetrick\'a verze:
skip
\roster
\item"{(usp)}"
Pro ka\v zd\'y vrchol $v$, kdy\v z $v$ reprezentuje prvek 
$s\in S$ a $\otec\left(v\right)$ reprezentuje $t$, pak $f\left(t\right)\le f\left(s\right)$.
\endroster


\flushpar Probereme n\v ekolik verz\'\i\ hald a budeme 
p\v redpokl\'adat, \v ze v\v zdy spl\v nuj\'\i\ tuto podm\'\i nku a \v ze operace 
{\bf DELETE$\left(s\right)$}, {\bf DECREA\-SE$\left(s,a\right)$} a {\bf INCREASE$\left(s,a\right)$} zad\'avaj\'\i\ 
tak\'e ukazatel na vrchol repre\-zentuj\'\i c\'\i\ $s\in S$.  Nav\'\i c budeme 
uva\v zovat operace
\medskip

\flushpar {\bf MAKEHEAP$\left(S,f\right)$} -- operace vytvo\v r\'\i\ haldu 
reprezentuj\'\i c\'\i\ mno\v zinu $S$ a funkci $f$.\newline 
{\bf MERGE$\left(H_1,H_2\right)$} -- p\v redpokl\'ad\'ame, \v ze halda $H_i$ reprezentuje 
mno\v zinu $S_i$ a funkci $f_i$ pro $i=1,2$ a $S_1\cap S_2=\emptyset$. Operace 
vytvo\v r\'\i\ haldu $H$ reprezentuj\'\i c\'\i\ $S_1\cup S_2$ a $
f_1\cup f_2$. 
Operace neov\v e\v ruje disjunktnost $S_1$ a $S_2$.
\medskip

\heading
Regul\'arn\'\i\ haldy
\endheading

\flushpar P\v redpokl\'adejme, \v ze $d>1$ je p\v rirozen\'e \v c\'\i slo. 
$d$-\emph{regul\'arn\'\i} \emph{strom} je ko\v renov\'y strom 
$\left(T,r\right)$ takov\'y, \v ze existuje po\v rad\'\i\ syn\accent23u 
jednotliv\'ych vnit\v rn\'\i ch vrchol\accent23u takov\'e, \v ze 
o\v c\'\i slov\'an\'\i\ vrchol\accent23u pro\-hle\-d\'a\-v\'an\'\i m do \v s\'\i\v rky 
(ko\v ren $r$ je \v c\'\i slov\'an $1$) spl\v nuje 
\roster
\item
ka\v zd\'y vrchol m\'a nejv\'y\v se $d$ syn\accent23u;
\item
kdy\v z vrchol nen\'\i\ list, tak v\v sechny vrcholy s men\v s\'\i m 
\v c\'\i slem maj\'\i\ pr\'av\v e $d$ syn\accent23u;
\item
kdy\v z vrchol m\'a m\'en\v e ne\v z $d$ syn\accent23u, pak v\v sechny 
vrcholy s v\v et\v s\'\i m \v c\'\i slem jsou listy.
\endroster
Toto o\v c\'\i slov\'an\'\i\ se naz\'yv\'a \emph{p\v rirozen\'e} 
\emph{o\v c\'\i slov\'an\'\i} $d$-regul\'arn\'\i ho stromu. Toto o\v c\'\i slov\'an\'\i\ 
budeme ozna\v covat $o$, to znamen\'a, \v ze pro vrchol $v$ je $o
\left(v\right)$ jeho 
\v c\'\i slo/po\v rad\'\i\ v tomto o\v c\'\i slov\'an\'\i.


\proclaim{Tvrzen\'\i}$d$-regul\'arn\'\i\ strom m\'a nejv\'y\v se jeden 
vrchol, kter\'y m\'a m\'en\v e ne\v z $d$ syn\accent23u. Kdy\v z 
$d$-regul\'arn\'\i\ strom m\'a $n$ vrchol\accent23u, pak jeho 
v\'y\v ska je $\left\lceil\log_d\left(n\left(d-1\right)+1\right)\right\rceil-1$. Kdy\v z pro vrchol 
$v$ je $o\left(v\right)=k$, pak vrchol $w$ je syn vrcholu $v$, pr\'av\v e 
kdy\v z $o\left(w\right)\in \left\{\left(k-1\right)d+2,\left(k-1\right)d+3,\dots,kd+1\right\}$, a vrchol $u$ je 
otcem vrcholu $v$, pr\'av\v e kdy\v z $o\left(u\right)=1+\lfloor\frac {k-2}d
\rfloor$. 
\endproclaim

\flushpar\v Rekneme, \v ze mno\v zina $S$ s funkc\'\i\ $f$ je 
\emph{reprezentov\'ana} $d$-regul\'ar\-n\'\i\ haldou $H$ a bijekc\'\i\ $\key$, kdy\v z 
$H$ je $d$-regul\'arn\'\i\ strom $\left(T,r\right)$ a bijekce $\key$ z vrchol\accent23u 
stromu $T$ na mno\v zinu $S$ spl\v nuje podm\'\i nku \thetag{usp}.  
\medskip

\flushpar Implementace $d$-regul\'arn\'\i\ haldy $H$. Nech\v t $o$ je 
p\v rirozen\'e o\v c\'\i s\-lo\-v\'a\-n\'\i\ $d$-regul\'arn\'\i ho stromu $
\left(T,r\right)$, pak halda $H$ je 
reprezentovan\'a polem $H\left[1..|S|\right]$, kde $H\left(i\right)=\left(\key\left(v\right),f\left(\key\left(v\right)\right)\right)$ a 
$o\left(v\right)=i$. Algoritmy budeme popisovat pro stromy, proto\v ze 
je to n\'azor\-n\v ej\-\v s\'\i . P\v reformulovat je pro pole je 
snadn\'e. Pro jednoduchost z\'apisu $f\left(v\right)$ pro vrchol $v$ 
ozna\v cuje $f\left(s\right)$, kde $s$ je reprezentov\'an vrcholem $v$. U 
$d$-regul\'arn\'\i ho stromu p\v redpo\-kl\'a\-d\'ame, \v ze zn\'ame p\v rirozen\'e 
o\v c\'\i slov\'an\'\i\ a fr\'aze `posledn\'\i\ vrchol', `p\v red\-ch\'azej\'\i c\'\i\ 
vrchol'  atd. se vztahuj\'\i\ k tomuto o\v c\'\i slov\'a\-n\'\i .
\medskip

\flushpar Nen\'\i\ zn\'ama efektivn\'\i\ implementace operace 
{\bf MERGE} pro $d$-regul\'arn\'\i\ haldy.  Efektivn\'\i\ implementace 
ostatn\'\i ch operac\'\i\ jsou zalo\v zeny na dvou pomocn\'ych 
operac\'\i ch {\bf UP$\left(v\right)$} a {\bf DOWN$\left(v\right)$}.  Operace {\bf UP$\left(
v\right)$} posunuje 
prvek $s$ reprezentovan\'y vrcholem $v$ sm\v erem ke ko\v reni, 
dokud vrchol reprezentuj\'\i c\'\i\ prvek $s$ nespl\v nu\-je podm\'\i nku 
\thetag{usp}. Na druhou stranu operace {\bf DOWN$\left(v\right)$} posunuje prvek $s$ sm\v erem k 
list\accent23um dokud nen\'\i\ spln\v ena podm\'\i nka \thetag{usp}.  
skip

\subhead
Algoritmy
\endsubhead
\medskip

{\bf UP$\left(v\right)$}:\newline 
{\bf while} $v$ nen\'\i\ ko\v ren a $f\left(v\right)<f\left(\otec\left(v\right)\right)$ {\bf do}\newline 
\phantom{---}vym\v e\v n $\key\left(v\right)$ a $\key\left(\otec\left(v\right)\right)$, $v:=\otec\left(v\right)$\newline 
{\bf enddo
\medskip

DOWN$\left(v\right)$}:\newline 
{\bf if} $v$ nen\'\i\ list {\bf then\newline}
\phantom{---}$w:=$syn vrcholu $v$ reprezentuj\'\i c\'\i\ prvek s nejmen\v s\'\i\ 
hodnotou $f\left(w\right)$\newline 
\phantom{---}{\bf while} $f\left(w\right)<f\left(v\right)$ a $v$ nen\'\i\ list {\bf do\newline }
\phantom{------}vym\v e\v n $\key\left(v\right)$ a $\key\left(w\right)$, $v:=w$\newline 
\phantom{------}$w:=$syn vrcholu $v$ reprezentuj\'\i c\'\i\ prvek s nejmen\v s\'\i\ 
hodnotou $f\left(w\right)$\newline 
\phantom{---}{\bf enddo}\newline 
{\bf endif}
\medskip

{\bf INSERT$\left(s\right)$}:\newline 
$v:=$nov\'y posledn\'\i\ list, $\key\left(v\right):=s$, {\bf UP$\left(v\right)$
\medskip

MIN}:\newline 
{\bf V\'ystup} $\key\left(\text{\rm ko\v ren }T\right)$
\medskip

{\bf DELETEMIN}:\newline 
$v:=$posledn\'\i\ list, $r:=$ko\v ren\newline 
$\key\left(r\right):=\key\left(v\right)$, odstra\v n $v$\newline 
{\bf DOWN$\left(r\right)$
\medskip

DELETE$\left(s\right)$},\newline 
$v:=$vrchol reprezentuj\'\i c\'\i\ $s$, $w:=$posledn\'\i\ list\newline 
$t:=\key\left(w\right)$, $\key\left(v\right):=t$\newline 
odstra\v n $w$\newline 
{\bf if} $f\left(t\right)<f\left(s\right)$ {\bf then UP$\left(v\right)$ else DOWN$\left(v\right)$ endif
\medskip

DECREASE$\left(s,a\right)$}:\newline 
$v:=$vrchol reprezentuj\'\i c\'\i\ $s$\newline 
$f\left(s\right):=f\left(s\right)-a$, {\bf UP$\left(v\right)$
\medskip

INCREASE$\left(s,a\right)$}:\newline 
$v:=$vrchol reprezentuj\'\i c\'\i\ $s$\newline 
$f\left(s\right):=f\left(s\right)+a$, {\bf DOWN$\left(v\right)$
\medskip

MAKEHEAP$\left(S,f\right)$}:\newline 
$T:=d$-regul\'arn\'\i\ strom s $|S|$ vrcholy\newline 
zvolme libovolnou reprezentaci $S$ vrcholy stromu $T$\newline 
$v:=$posledn\'\i\ vrchol, kter\'y nen\'\i\ list\newline 
{\bf while} $v$ je vrchol $T$ {\bf do}\newline 
\phantom{---}{\bf DOWN}$\left(v\right)$, $v:=$vrchol p\v redch\'azej\'\i c\'\i\ vrcholu $v$\newline 
{\bf enddo}
\medskip

\subhead 
Korektnost algoritm\accent23u
\endsubhead

\flushpar U operace {\bf INSERT} je podm\'\i nka \thetag{usp} spln\v ena pro v\v sechny vrcholy s v\'yjimkou nov\v e vytvo\v ren\'eho listu a operace {\bf UP} zajist\'\i\ jej\'\i\ spln\v en\'\i .  P\v ri operaci {\bf DELETEMIN} je pod\-m\'\i n\-ka \thetag{usp} spln\v ena pro v\v sechny vrcholy s v\'yjimkou 
ko\v rene a operace {\bf DOWN} zajist\'\i\ jej\'\i\ spln\v en\'\i.  U 
operac\'\i\ {\bf DE\-LE\-TE$\left(s\right)$}, {\bf DECREASE$\left(s,a\right)$} a {\bf INCREASE$\left(s,a
\right)$} je pod\-m\'\i n\-ka \thetag{usp} spln\v ena pro v\v sechny vrcholy s 
v\'yjimkou vrcholu $v$ a proveden\'\i\ operace {\bf UP} nebo {\bf DOWN} 
zajist\'\i\ jej\'\i\ spln\v en\'\i .  Pro operaci {\bf MAKEHEAP} budeme 
uva\v zovat du\'aln\'\i\ formulaci podm\'\i nky \thetag{usp}:
\roster
\item"{(d-usp)}"
kdy\v z $s$ je prvek reprezentovan\'y vrcholem $v$, pak 
$f\left(s\right)\le f\left(t\right)$ pro v\v sechny prvky reprezentovan\'e syny $t$
vrcholu $v$. 
\endroster
Kdy\v z ka\v zd\'y vrchol spl\v nuje podm\'\i nku \thetag{d-usp}, pak 
spl\v nuje i 
pod\-m\'\i nku \thetag{usp}.  Ka\v zd\'y list spl\v nuje podm\'\i nku 
\thetag{d-usp}, a proto kdy\v z algoritmus {\bf MAKEHEAP} pracuje s 
vrcholem $v$, tak podm\'\i nku \thetag{d-usp} spl\v nuj\'\i\ v\v sechny 
vrcholy $w$ takov\'e, \v ze $o\left(w\right)>o\left(v\right)$.  Po proveden\'\i\ podprocedury 
{\bf DOWN$\left(v\right)$} je podm\'\i nka \thetag{d-usp} spln\v ena i pro vrchol $
v$ (tj.  
pro v\v sechny vrcholy $w$ takov\'e, \v ze $o\left(w\right)\ge o\left(v\right)$).  Odtud plyne 
korektnost algoritmu pro operaci {\bf MAKEHEAP}, proto\v ze 
algoritmus kon\v c\'\i\ proveden\'\i m operace {\bf DOWN} na ko\v ren.  
\medskip

\subhead 
Efektivita operac\'\i
\endsubhead
\smallskip 

\flushpar Jeden b\v eh cyklu v 
operaci {\bf UP} vy\v zaduje \v cas $O\left(1\right)$ a v operaci {\bf DOWN }\v cas $
O\left(d\right)$. Proto 
v nej\-hor\v s\'\i m p\v r\'\i pad\v e operace {\bf UP} vy\v zaduje \v cas $
O\left(\log_d|S|\right)$ 
a {\bf DOWN }\v cas 
$O\left(d\log_d|S|\right)$. Operace {\bf MIN} z\v rejm\v e vy\v zaduje \v cas $O\left(1\right)$, operace 
{\bf INSERT} a {\bf DECREASE} vy\v zadu\-j\'\i\ \v cas $O\left(\log_d|
S|\right)$ a operace 
{\bf DELETEMIN}, {\bf DELETE} a {\bf INCREASE }\v cas 
$O\left(d\log_d|S|\right)$. Haldu m\accent23u\v zeme vytvo\v rit i tak, \v ze 
opakujeme $|S|$-kr\'at operaci {\bf INSERT}, to vy\v zaduje \v cas 
$O\left(|S|\log_d\left(|S|\right)\right)$. Spo\v c\'\i t\'ame slo\v zi\-tost operace {\bf MAKEHEAP}. 
Operace {\bf DOWN$\left(v\right)$} na vrchol ve v\'y\v sce $h$ vy\v zaduje v 
nejhor\v s\'\i m p\v r\'\i pad\v e \v cas $O\left(hd\right)$. Vrchol\accent23u v hloubce $
i$ je 
nejv\'y\v se $d^i$. P\v red\-pok\-l\'a\-dejme, \v ze strom m\'a v\'y\v sku $
k$, pak 
operace {\bf MAKEHEAP} vy\v zaduje \v cas 
$O\left(\sum_{i=0}^{k-1}d^i\left(k-i\right)d\right)=O\left(\sum_{i=0}^{k-1}d^{i+1}\left(k-i\right)\right)$. Oz\-na\v c\-me 
$A=\sum_{i=0}^{k-1}d^{i+1}\left(k-i\right)$, pak
$$\align dA-A=&\sum_{i=0}^{k-1}d^{i+2}\left(k-i\right)-\sum_{i=0}^{k-1}d^{i+
1}\left(k-i\right)=\\
&\sum_{i=2}^{k+1}d^i\left(k-i+2\right)-\sum_{i=1}^kd^i\left(k-i+1\right)=\\
&d^{k+1}+\sum_{i=2}^kd^i\left(k-i+2-k+i-1\right)-dk=\\
&d^{k+1}+\sum_{i=2}^kd^i-dk=d^{k+1}+d^2\frac {d^{k-1}-1}{d-1}-dk.\endalign$$
Tedy $A=\frac {d^{k+1}}{d-1}+\frac {d^{k+1}-d^2}{\left(d-1\right)^2}-dk$. Proto\v ze 
$k=\left\lceil\log_d\left(|S|\left(d-1\right)+1\right)\right\rceil-1$, 
dost\'av\'ame, \v ze $d^{k+1}\le d^2\left(\left(d-1\right)|S|+1\right)$, a proto $A\le 
2d^2|S|$. Tedy 
{\bf MAKEHEAP} vy\v zaduje v nejhor\v s\'\i m p\v r\'\i pad\v e jen \v cas 
$O\left(d^2|S|\right)$. 
\medskip

\subhead 
Aplikace: T\v r\'\i d\v en\'\i\
\endsubhead
\smallskip

\flushpar Pro set\v r\'\i d\v en\'\i\ posloupnosti prvk\accent23u m\accent23u\v zeme  pou\v zit n\'asleduj\'\i c\'\i\ algoritmus.
skip

$d${\bf -HEAPSORT$\left(x_1,x_2,\dots,x_n\right)$}:\newline 
{\bf MAKEHEAP$\left(\left\{x_i\mid i=1,2,\dots,n\right\},f\right)$\newline 
$i=1$\newline 
while} $i\le n$ {\bf do}\newline 
\phantom{---}$y_i:=${\bf MIN}, {\bf DELETEMIN}, $i:=i+1$\newline 
{\bf enddo\newline 
V\'ystup}: $y_1,y_2,\dots,y_n$
skip

\flushpar P\v redpokl\'ad\'ame, \v ze $x_1,x_2,\dots,x_n$ je posloupnost 
\v c\'\i sel a $f$ v tomto p\v r\'\i pad\v e bude identick\'a funkce.
\medskip

\flushpar Teoreticky lze uk\'azat, \v ze $d=3$ a $d=4$ jsou lep\v s\'\i\ ne\v z 
$d=2$.  Experimenty uk\'azaly, \v ze optim\'aln\'\i\ algoritmus pro 
posloupnosti d\'elek do 1 000 000 by m\v el b\'yt $d=6$ nebo $d=7$.  
\medskip

\subhead
Nalezen\'\i\ nejkrat\v s\'\i ch cest v grafu z dan\'eho 
bodu.
\endsubhead
\smallskip
 
\flushpar Vstupem pro tuto aplikaci je orientovan\'y ohodnocen\'y graf $\left(X,R,c\right)$ a 
vrchol $z\in X$, $c$ je funkce z $R$ do mno\v ziny kladn\'ych 
re\'aln\'ych \v c\'\i sel.\newline 
\'Ukol: nal\'ezt pro ka\v zd\'y bod $x\in X$ d\'elku 
nejkrat\v s\'\i\ cesty ze $z$ do $x$ -- d\'elka cesty je sou\v cet 
ohodnocen\'\i\ hran na dan\'e cest\v e funkc\'\i\ $c$.\newline 
\v Re\v sen\'\i: 
skip

\flushpar{\bf Dijkstr\accent23uv algoritmus\newline }
$d\left(z\right):=0$, $U:=\left\{z\right\}$\newline
{\bf for every} $x\in X\setminus \left\{z\right\}$ {\bf do} $d\left(x\right):=+\infty$ {\bf enddo\newline
while} $U\ne\emptyset$ {\bf do\newline}
\phantom{---}najdi vrchol $u\in U$ s nejmen\v s\'\i\ hodnotou $d\left(u\right)$, odstra\v n ho z $
U$\newline
\phantom{---}{\bf for every} $\left(u,v\right)\in R$ {\bf do\newline}
\phantom{------}{\bf if} $d\left(u\right)+c\left(u,v\right)<d\left(v\right)$ {\bf then}\newline 
\phantom{---------}{\bf if} $d\left(v\right)=+\infty$ {\bf then} vlo\v z $v$ do $U$ {\bf endif}\newline 
\phantom{---------}$d\left(v\right):=d\left(u\right)+c\left(u,v\right)$\newline 
\phantom{------}{\bf endif}\newline 
\phantom{---}{\bf enddo\newline 
enddo}
\medskip

\flushpar Kdy\v z $U=\emptyset$, pak $d\left(x\right)$ jsou d\'elky nejkrat\v s\'\i ch cest ze $
z$ do $x$.
Kdy\v z $U$ reprezentujeme jako $d$-regul\'arn\'\i\ haldu, pak 
se provede nejv\'y\v se $|X|$ operac\'\i\ {\bf INSERT}, {\bf MIN} a {\bf DELETEMIN }
a $|R|$ operac\'\i\ {\bf DECREASE} a $|U|\le |X|$. Pro $d=2$ dost\'av\'ame, 
\v ze algoritmus vy\v zaduje \v cas $O\left(|X|\log\left(|X|\right)+|R|\log\left(|X|\right)\right)$. kdy\v z polo\v z\'\i me  
$d=\max\left\{2,\lfloor\frac {|R|}{|X|}\rfloor \right\}$, pak algoritmus vy\v zaduje \v cas $
O\left(|R|\log_d|X|\right)$.
Kdy\v z $|R|>|X|^{1+\varepsilon}$ pro $\varepsilon >0$, pak $\log_
d|X|=O\left(1\right)$ a algoritmus 
je line\'arn\'\i\ (tj. vy\v zaduje \v cas $O\left(|R|\right)$).


\heading
Leftist haldy
\endheading

\flushpar M\v ejme bin\'arn\'\i\ strom $\left(T,r\right)$, to znamen\'a, \v ze $
r$ je ko\v ren, 
ka\v zd\'y vrchol m\'a nejv\'y\v se dva syny a u ka\v zd\'eho syna 
v\'\i me, zda je to prav\'y nebo lev\'y syn. Pro vrchol $v$ 
ozna\v cme $\npl\left(v\right)$ d\'elku nejkrat\v s\'\i\ cesty z $v$ do vrcholu, kter\'y m\'a 
nejv\'y\v se jednoho syna. Tak\v ze nap\v r. list m\'a $\npl$ rovno $
0$.
\medskip

\flushpar M\v ejme $S\subseteq U$ a funkci $f:S@>>>\Bbb R$. Pak bin\'arn\'\i\ strom 
$\left(T,r\right)$ takov\'y, \v ze
\roster
\item
kdy\v z vrchol $v$ m\'a jen jednoho syna, pak je to lev\'y syn,
\item
kdy\v z vrchol $v$ m\'a dva syny, pak 
$$\npl\left(\pravy\left(v\right)\right)\le\npl\left(\levy\left(v\right)\right),$$
\item
existuje jednozna\v cn\'a korespondence mezi prvky z $S$ a 
vrcholy $T$, kter\'a spl\v nuje podm\'\i nku \thetag{usp}
\endroster
je \emph{leftist} \emph{halda} 
reprezentuj\'\i c\'\i\ mno\v zinu $S$ a funkci $f$. 
\medskip

\flushpar Struktura vrcholu $v$:\newline 
ukazatel\'e $\otec\left(v\right)$, $\levy\left(v\right)$ a $\pravy\left(v\right)$ na otce vrcholu $
v$, 
na lev\'eho syna $v$ a na prav\'eho syna $v$. Kdy\v z ukazatel 
nen\'\i\ definov\'an, pak p\'\i\v seme, \v ze jeho hodnota je 
$NIL$;\newline 
$\npl\left(v\right)$ -- prom\v enn\'a s hodnotou $\npl\left(v\right)$;\newline 
$\key\left(v\right)$ -- prvek reprezentovan\'y vrcholem $v$;\newline 
$f\left(v\right)$ -- prom\v enn\'a obsahuj\'\i c\'\i\ hodnotu $f\left(\key\left(v\right)\right)$.
\medskip

\flushpar Z\'akladn\'\i\ vlastnost leftist haldy.\newline 
Posloupnost vrchol\accent23u $v_0,v_1,\dots,v_k$ se naz\'yv\'a 
\emph{prav\'a} \emph{cesta} z vrcholu $v$, kdy\v z $v=v_0$, $v_{i
+1}$ 
je prav\'y syn $v_i$ pro ka\v zd\'e $i=0,1,\dots,k-1$ a $v_k$ nem\'a 
prav\'eho syna. Pak podstrom vrcholu $v$ do hloubky $k$ je 
\'upln\'y bin\'arn\'\i\ strom, a tedy podstrom vrcholu $v$ m\'a 
alespo\v n $2^{k+1}-1$ vrchol\accent23u. Tedy d\'elka prav\'e 
cesty z vrcholu $v$ je 
$$O\left(\log\left(\text{\rm velikost podstromu ur\v cen\'eho vrcholem }v\right)\right).$$
\medskip

\flushpar Z\'akladn\'\i\ operace pro leftist haldy je {\bf MERGE}.
skip

\subhead
Algoritmy
\endsubhead
\medskip

{\bf MERGE$\left(T_1,T_2\right)$}:\newline 
{\bf if} $T_1=\emptyset$ {\bf then V\'ystup$=T_2$} konec {\bf endif\newline 
if} $T_2=\emptyset$ {\bf then V\'ystup$=T_1$} konec {\bf endif\newline 
if} $\key\left(\text{\rm ko\v ren }T_1\right)>\key\left(\text{\rm ko\v ren }T_2\right)$ {\bf then\newline}
\phantom{---}zam\v e\v n $T_1$ a $T_2$\newline 
{\bf endif\newline 
$T':=$MERGE$\left(\text{{\rm podstrom prav\'eho syna ko\v rene }}T_1,T_
2\right)$\newline 
$\pravy\left(\text{{\rm {\rm ko\v ren }}}T_1\right):=\text{{\rm {\rm ko\v ren }}}
T'$}, 
$\otec\left(\text{\rm ko\v ren }T'\right):=\text{\rm ko\v ren }T_1$\newline 
{\bf if} $\npl\left(\pravy\left(\text{\rm ko\v ren }T_1\right)\right)>\npl\left(\levy\left(\text{\rm ko\v ren }
T_1\right)\right)$ {\bf then}\newline
\phantom{---}zam\v e\v n lev\'eho a prav\'eho syna ko\v rene $T_1$\newline 
{\bf endif\newline 
$\npl\left(\text{{\rm {\rm ko\v ren }}}T_1\right):=\npl\left(\pravy\left(\text{{\rm {\rm ko\v ren }}}
T_1\right)\right)+1$}
skip

{\bf INSERT$\left(x\right)$}:\newline 
Vytvo\v r haldu $T_1$ reprezentuj\'\i c\'\i\ $\left\{x\right\}$\newline 
$\bold M\bold E\bold R\bold G\bold E\left(T,T_1\right)$
skip

{\bf MIN}:\newline 
{\bf V\'ystup}: $\key\left(\text{\rm ko\v ren }T\right)$
skip

{\bf DELETEMIN}:\newline 
$T_1:=$podstrom lev\'eho syna ko\v rene $T$\newline 
$T_2:=$podstrom prav\'eho syna ko\v rene $T$\newline 
{\bf MERGE$\left(T_1,T_2\right)$
skip

MAKEHEAP$\left(S,f\right)$}:\newline 
$Q:=$pr\'azdn\'a fronta\newline 
{\bf for every} $s\in S$ {\bf do}\newline 
\phantom{---}vytvo\v r leftist haldu $T_s$ reprezentuj\'\i c\'\i\ $\left\{s\right\}$\newline 
\phantom{---}vlo\v z $T_s$ do $Q$\newline 
{\bf enddo\newline 
while} $|Q|>1$ {\bf do}\newline 
\phantom{---}vezmi haldy $T_1$ a $T_2$ z vrcholu $Q$ (odstra\v n je)\newline 
\phantom{---}{\bf MERGE$\left(T_1,T_2\right)$} vlo\v z do $Q$\newline
{\bf enddo}
skip

\subhead
Efektivnost algoritm\accent23u
\endsubhead 
\smallskip

\flushpar Ka\v zd\'y b\v eh algoritmu {\bf MERGE} (bez rekurzivn\'\i ho vol\'an\'\i ) 
vy\v zaduje \v cas $O\left(1\right)$. Po\v cet rekurzivn\'\i ch vol\'an\'\i\ je 
sou\v cet d\'elek prav\'ych cest, proto algoritmus {\bf MERGE} vy\v zaduje 
\v cas $O\left(\log\left(|S_1|+|S_2|\right)\right)$, 
kde $S_i$ je mno\v zina reprezentovan\'a haldou $T_i$ pro $i=1,2$.
Proto algoritmy {\bf INSERT} a {\bf DELETEMIN} vy\v zaduj\'\i\ v 
nejhor\v s\'\i m p\v r\'\i pad\v e \v cas $O\left(\log\left(|S|\right)\right)$. Ope\-race {\bf MIN} vy\v zaduje 
\v cas $O\left(1\right)$. Pro odhad slo\v zitosti {\bf MA\-KEHEAP} budeme 
uva\v zovat, \v ze na za\v c\'atku algoritmu je na vrcholu fronty 
speci\'aln\'\i\ znak, kter\'y se jen p\v renese na konec fronty. 
Odhadneme \v cas, kter\'y pot\v rebuj\'\i\ {\bf while-}cykly mezi dv\v e\-ma 
p\v renesen\'\i mi speci\'aln\'\i ho znaku. P\v redpokl\'adejme, \v ze se 
spe\-ci\'al\-n\'\i\ znak p\v renesl $k$-kr\'at. V tomto okam\v ziku, a\v z na 
jednu haldu, v\v sechny haldy ve front\v e maj\'\i\ velikost $2^k$. 
Proto ve front\v e $Q$ je $\left\lceil\frac {|S|}{2^k}\right\rceil$ hald a ka\v zd\'a operace 
{\bf MERGE} vy\v zadu\-je $O\left(k\right)$ \v casu. Tedy {\bf while}-cykly vy\v zaduj\'\i\ 
\v cas $O\left(k\frac {|S|}{2^k}\right)$. Dost\'a\-v\'a\-me, \v ze ope\-race {\bf MAKEHEAP} vy\v zaduje \v cas 
$$O\left(\sum_{k=1}^{\infty}k\frac {|S|}{2^k}\right)=O\left(|S|\sum_{k=1}^{\infty}\frac 
k{2^k}\right)=O\left(|S|\right).$$
Implementace operac\'\i\ {\bf DECREASE} a {\bf INCREASE} pomoc\'\i\ 
operac\'\i\ {\bf UP} a {\bf DOWN} jako v $d$-regul\'arn\'\i ch hald\'ach nen\'\i\ 
efektivn\'\i , proto\v ze d\'elka cesty z ko\v rene do listu v leftist 
hald\v e m\accent23u\v ze b\'yt a\v z $|S|$. Proto navrhneme 
slo\v zit\v ej\v s\'\i , ale efektivn\'\i\ algoritmus pro tyto operace.
\medskip

\flushpar Nejprve pop\'\i\v seme pomocnou operaci {\bf Oprav$\left(T,v
\right)$}, 
kter\'a vy\-tvo\-\v r\'\i\ lef\-tist haldu z bin\'arn\'\i ho stromu $
T'$ vznikl\'eho z 
leftist haldy $T$ odtrhnut\'\i m podstromu s ko\v renem ve vrcholu $
v$.
skip

{\bf Oprav$\left(T,v\right)$}:\newline 
$t:=\otec\left(v\right)$, $\npl\left(t\right):=0$\newline 
{\bf if} $\pravy\left(t\right)\ne v$ {\bf then} $\levy\left(t\right):=\pravy\left(t\right)$ {\bf endif\newline 
$\pravy\left(t\right):=NIL$\newline 
while} se zmen\v silo $\npl\left(t\right)$ a $t$ nen\'\i\ ko\v ren {\bf do}\newline 
\phantom{---}$t:=\otec\left(t\right)$\newline
\phantom{---}{\bf if} $\npl\left(\pravy\left(t\right)\right)>\npl\left(\levy\left(t\right)\right)$ {\bf then}\newline 
\phantom{------}vym\v e\v n $\levy\left(t\right)$ a $\pravy\left(t\right)$\newline 
\phantom{---}{\bf endif\newline 
\phantom{---}$\npl\left(t\right):=\npl\left(\pravy\left(t\right)\right)+1$\newline 
enddo
skip

}\flushpar Po proveden\'\i\ procedury {\bf Oprav} maj\'\i\ v\v sechny vrcholy 
spr\'avn\'e \v c\'\i slo $\npl$ a nav\'\i c jsou spln\v eny podm\'\i nky polo\v zen\'e na 
leftist haldu.  Tedy po proveden\'\i\ {\bf Oprav} je $T$ op\v et 
leftist halda.  Kdy\v z $t$ je posledn\'\i\ vrchol, u kter\'eho se zmen\v silo 
$\npl$, pak vrcholy, kde se zmen\v silo $\npl$ tvo\v r\'\i\ pravou cestu z 
vrcholu $t$.  To znamen\'a, \v ze {\bf while}-cyklus se prov\'ad\v el nejv\'y\v se 
$\log\left(|S|\right)$-kr\'at a ka\v zd\'y b\v eh {\bf while}-cyklu vy\v zadoval \v cas $
O\left(1\right)$.  
Proto  
algoritmus {\bf Oprav} vy\v zaduje \v cas $O\left(\log\left(|S|\right)\right)$.  
\medskip

\subhead
Ostatn\'\i\ algoritmy.
\endsubhead
skip

{\bf DECREASE$\left(s,a\right)$}:\newline
$v:=$ prvek reprezentuj\'\i c\'\i\ $s$\newline 
$T_1:=$ podstrom $T$ ur\v cen\'y vrcholem $v$, $f\left(v\right):=f\left(v\right)-a$\newline 
odtrhni podstrom $T_1$ od stromu $T$\newline 
$T_2:=${\bf Oprav$\left(T,v\right)$}, $T:=${\bf MERGE$\left(T_1,T_2\right)$}
skip

{\bf INCREASE$\left(s,a\right)$}:\newline 
$v:=$ prvek reprezentuj\'\i c\'\i\ $s$\newline 
$T_1:=$ podstrom $T$ ur\v cen\'y vrcholem $\levy\left(v\right)$\newline 
$T_2:=$ podstrom $T$ ur\v cen\'y  vrcholem $\pravy\left(v\right)$\newline 
$T_3:=$ leftist halda reprezentuj\'\i c\'\i\ prvek $s$\newline 
odtrhni od stromu $T$ podstrom ur\v cen\'y vrcholem $v$\newline 
$f\left(v\right):=f\left(v\right)+a$, $T_4:=${\bf Oprav$\left(T,v\right)$}, $T_1:=${\bf MERGE$\left(T_1
,T_3\right)$\newline 
$T_2:=$MERGE$\left(T_2,T_4\right)$}, $T:=${\bf MERGE$\left(T_1,T_2\right)$
skip

DELETE$\left(s,a\right)$}:\newline 
$v:=$ prvek reprezentuj\'\i c\'\i\ $s$\newline 
$T_1:=$ podstrom $T$ ur\v cen\'y vrcholem $\levy\left(v\right)$\newline 
$T_2:=$ podstrom $T$ ur\v cen\'y  vrcholem $\pravy\left(v\right)$\newline 
odtrhni od stromu $T$ podstrom ur\v cen\'y vrcholem $v$\newline 
$T_3:=${\bf MERGE$\left(T_1,T_2\right)$}, $T_4:=${\bf Oprav$\left(T,v\right)$\newline 
$T:=$MERGE$\left(T_3,T_4\right)$
skip

}\flushpar Odtr\v zen\'\i\ podstromu ur\v cen\'eho vrcholem $v$ od stromu $
T$ 
znamen\'a proveden\'\i\ akce\medskip
\flushpar {\bf if} $v=\levy\left(\otec\left(v\right)\right)$ {\bf then}\newline 
\phantom{---}$\levy\left(\otec\left(v\right)\right):=NIL$\newline 
{\bf else}\newline 
\phantom{---}$\pravy\left(\otec\left(v\right)\right):=NIL$\newline 
{\bf endif}\newline 
$\otec\left(v\right):=NIL$
\medskip

\subhead
Shrnut\'\i\ v\'ysledk\accent23u
\endsubhead

\proclaim{V\v eta}V leftist hald\'ach existuje implementace operace 
{\bf MIN}, kter\'a v nejhor\v s\'\i m p\v r\'\i pad\v e vy\v zaduje \v cas $
O\left(1\right)$, 
implementace ope\-rac\'\i\ {\bf INSERT}, {\bf DELETEMIN}, {\bf DELETE}, 
{\bf DEC\-REA\-SE}, {\bf INCREA\-SE} a {\bf MERGE}, kter\'e vy\v zaduj\'\i\ v nejhor\v s\'\i m 
p\v r\'\i\-pa\-d\v e \v cas $O\left(\log\left(|S|\right)\right)$, a implementace operace {\bf MAKE\-HEAP},
kter\'a vy\v zaduje \v cas $O\left(|S|\right)$, kde $S$ je reprezentovan\'a mno\v zina.
\endproclaim
\medskip


\heading 
Amortizovan\'a slo\v zitost
\endheading


\flushpar P\v redpokl\'adejme, \v ze $h$ je funkce, kter\'a ohodnucuje 
konfigurace. Kdy\v z na konfiguraci $D$ aplikujeme ope\-raci $o$ 
a dostaneme konfiguraci $D'$, pak amortizovan\'a slo\v zitost 
$am\left(o\right)$ ope\-race $o$ je $am\left(o\right)=t\left(o\right)+h\left(D'\right)-h\left(D\right)$, kde $t\left(o\right)$ je 
\v cas pot\v rebn\'y pro proveden\'\i\ ope\-race $o$. P\v redpokl\'adejme, 
\v ze
$$D_0@>{o_1}>>D_1@>{o_2}>>D_2@>{o_3}>>\dots@>{o_n}>>D_n.$$
D\'ale p\v redpokl\'adejme, \v ze zn\'ame odhady amortizovan\'e slo\v zitos\-ti 
ope\-rac\'\i : $am\left(o_i\right)\le c\left(o_i\right)$ pro v\v sechna $i=1,2,\dots,n$. Pak
$$\align\sum_{i=1}^nam\left(o_i\right)=&\sum_{i=1}^n\left(t\left(o_i\right)+h\left(D_i\right)-h\left(D_{
i-1}\right)\right)=\\
&h\left(D_n\right)-h\left(D_0\right)+\sum_{i=1}^nt\left(o_i\right)\le\sum_{i=1}^nc\left(o_i\right).\endalign$$
Z toho plyne, \v ze 
$$\sum_{i=1}^nt\left(o_i\right)\le\sum_{i=1}^nc\left(o_i\right)-h\left(D_n\right)+h\left(D_0\right).$$
\medskip

\flushpar Obvykle $h\left(D\right)\ge 0$ pro v\v sechny konfigurace $D$ nebo $
h\left(D\right)\le 0$ 
pro v\v sechny konfigurace $D$. Kdy\v z $h\left(D\right)\ge 0$ pro v\v sechny 
konfigurace $D$, pak m\accent23u\v zeme ps\'at
$$\sum_{i=1}^nt\left(o_i\right)\le\sum_{i=1}^nc\left(o_i\right)+h\left(D_0\right),$$
kdy\v z $h\left(D\right)\le 0$ pro v\v sechny konfigurace $D$, pak m\accent23u\v zeme 
ps\'at
$$\sum_{i=1}^nt\left(o_i\right)\le\sum_{i=1}^nc\left(o_i\right)-h\left(D_n\right).$$
To znamen\'a, \v ze odhad amortizovan\'e slo\v zitosti d\'av\'a tak\'e 
odhad na \v casovou slo\v zitost, kter\'y je vhodn\'y pro 
posloupnost operac\'\i . Odhad amortizovan\'e slo\v zitosti b\'yv\'a 
lep\v s\'\i\ ne\v z odhad slo\v zitosti v nejhor\v s\'\i m p\v r\'\i pad\v e, a tedy i odhad 
slo\v zitosti posloupnosti operac\'\i\ v nejhor\v s\'\i m p\v r\'\i pad\v e nalezen\'y 
pomoc\'\i\ amortizovan\'e slo\v zitosti je men\v s\'\i\ ne\v z sou\v cet slo\v zitost\'\i\ 
jednotliv\'ych operac\'\i\ v nejhor\v s\'\i m p\v r\'\i pad\v e. Zd\accent23uvodn\v en\'\i\ 
tohoto jevu je, \v ze je nepravd\v e\-podobn\'e, aby dv\v e operace za 
sebou vy\v zadovaly nejv\v et\v s\'\i\ \v cas. To hraje zvl\'a\v s\v t v\'yznamnou 
roli, kdy\v z nejhor\v s\'\i ch p\v r\'\i pad\accent23u je m\'alo (jin\'ymi slovy, 
to znamen\'a, \v ze jev, \v ze nastane nejhor\v s\'\i\ p\v r\'\i pad, je m\'alo 
pravd\v epodobn\'y).

\heading
Binomi\'aln\'\i\ haldy
\endheading 

\flushpar Definujme rekurentn\v e binomi\'aln\'\i\ stromy $H_i$ pro 
$i=0,1,\dots$.  Jsou to ko\v renov\'e stromy takov\'e, \v ze $H_0$ je 
jednoprvkov\'y strom a  strom $H_{i+1}$ vznikne ze dvou 
disjunktn\'\i ch strom\accent23u $H_i$, kde ko\v ren jednoho  
se stane dal\v s\'\i m synem ko\v rene druh\'eho z nich.  Viz 
Obr.~1 
\medskip

\midinsert
\centerline{\input fig10.tex}
\botcaption{Obr. 1}
\endcaption
\endinsert

\flushpar Nejprve uvedeme z\'akladn\'\i\ vlastnosti t\v echto 
strom\accent23u.

\proclaim{Tvrzen\'\i}Pro ka\v zd\'e p\v rirozen\'e \v c\'\i slo $
i=0,1,\dots$ 
plat\'\i :
\roster
\item
strom $H_i$ m\'a $2^i$ vrchol\accent23u;
\item
ko\v ren stromu $H_i$ m\'a $i$ syn\accent23u;
\item
d\'elka nejdel\v s\'\i\ cesty z ko\v rene do listu ve stromu $H_i$ je 
$i$;
\item
podstromy ur\v cen\'e syny ko\v rene stromu $H_i$ jsou izomorf\-n\'\i\ 
po \v rad\v e se stromy $$H_0,H_1,\dots,H_{i-1}.$$
\endroster
\endproclaim

\demo{D\accent23ukaz}Tvrzen\'\i\ plat\'\i\ pro strom $H_0$ a 
jednoduchou indukc\'\i\ dok\'a\v zeme toto tvrzen\'\i\  i pro ostatn\'\i\ stromy. \qed
\enddemo


\flushpar\emph{Binomi\'aln\'\i} \emph{halda} $\Cal H$ reprezentuj\'\i c\'\i\ 
mno\v zinu $S$ je soubor strom\accent23u $\left\{T_1,T_2,\dots,T_k\right\}$ 
takov\'y, \v ze 
\roster
\item"{}"
po\v cet vrchol\accent23u v t\v echto stromech je roven 
velikosti $S$ a 
je d\'ano jednozna\v cn\'e p\v ri\v razen\'\i\ prvk\accent23u 
z $S$ vrchol\accent23um strom\accent23u tako\-v\'e, \v ze plat\'\i\ 
podm\'\i nka \thetag{usp};
\item"{}"
ka\v zd\'y strom $T_i$ je izomorfn\'\i\ s n\v ejak\'ym stromem $H_
j$;
\item"{}"
$T_i$ nen\'\i\ izomorfn\'\i\ s \v z\'adn\'ym $T_j$ pro $i\ne j$.
\endroster
\medskip

\flushpar Z bin\'arn\'\i ho z\'apisu p\v rirozen\'ych \v c\'\i sel plyne, \v ze 
pro ka\v zd\'e p\v riro\-ze\-n\'e \v c\'\i slo $n>0$ existuje prost\'a 
posloupnost $i_1,i_2,\dots,i_k$ p\v riro\-ze\-n\'ych \v c\'\i sel takov\'a, \v ze 
$n=\sum_{j=1}^k2^{i_j}$. Z toho plyne, \v ze pro ka\v zdou nepr\'azd\-nou 
mno\v zinu $S$ existuje binomi\'aln\'\i\ halda repre\-zentuj\'\i c\'\i\ $
S$.
\medskip

\flushpar Operace pro binomi\'aln\'\i\ haldy jsou zalo\v zeny 
stejn\v e jako pro leftist haldy na operaci {\bf MER\-GE}. Operace 
{\bf MERGE} pro binomi\'aln\'\i\ haldy je analogi\'\i\ s\v c\'\i t\'an\'\i\ 
p\v rirozen\'ych \v c\'\i sel v bin\'ar\-n\'\i m z\'apise.
skip

{\bf MERGE$\left(\Cal H_1,\Cal H_2\right)$}:\newline 
(koment\'a\v r: $\Cal H_i$ reprezentuje mno\v zinu $S_i$ pro $i=1
,2$ a $S_1\cap S_2=\emptyset$)\newline 
$i:=0$, $T:=$pr\'azdn\'y strom, $\Cal H:=\emptyset$\newline 
{\bf while} $i<\log\left(|S_1|+|S_2|\right)$ {\bf do}\newline 
\phantom{---}{\bf if} existuje $U\in \Cal H_1$ izomorfn\'\i\ s $H_i$ {\bf then}\newline 
\phantom{------}$U_1:=U$\newline 
\phantom{---}{\bf else}\newline
\phantom{------}$U_1:=$pr\'azdn\'y strom\newline 
\phantom{---}{\bf endif}\newline 
\phantom{---}{\bf if} existuje $U\in \Cal H_2$ izomorfn\'\i\ s $H_i$ {\bf then}\newline 
\phantom{------}$U_2:=U$\newline 
\phantom{---}{\bf else}\newline 
\phantom{------}$U_2:=$pr\'azdn\'y strom\newline 
\phantom{---}{\bf endif}\newline 
\phantom{---}{\bf case}\newline
\phantom{------}(stromy $T$, $U_1$, $U_2$ jsou pr\'azdn\'e) {\bf do}\newline 
\phantom{------}nic\newline 
\phantom{------}(existuje pr\'av\v e jeden nepr\'azdn\'y strom $V$ mezi stromy $T$, $U_1$ a $U_2$) {\bf do}\newline 
\phantom{------}vlo\v zme $V$ do $\Cal H$, $T:=$pr\'azdn\'y strom;\newline 
\phantom{------}(existuj\'\i\ pr\'av\v e dva nepr\'azdn\'e stromy $V_1$ a $V_2$ mezi stromy $T$, $U_1$ a $U_2$) {\bf do}\newline 
\phantom{------}$T:=${\bf spoj$\left(V_1,V_2\right)$}\newline 
\phantom{------}(v\v sechny stromy $T$, $U_1$ a $U_2$ jsou nepr\'azdn\'e)
{\bf do}\newline 
\phantom{------}vlo\v zme $T$ do $\Cal H$, $T:=${\bf spoj$\left(U_1,U_2\right)$}\newline 
\phantom{---}{\bf endcase}\newline
\phantom{---}$i:=i+1$\newline 
{\bf enddo\newline 
if} $T\ne$pr\'azdn\'y strom {\bf then} vlo\v zme $T$ do $\Cal H$ {\bf endif\newline 
V\'ystup}:$\Cal H$
skip

{\bf spoj$\left(T_1,T_2\right)$}:\newline 
{\bf if} $f\left(\text{\rm ko\v ren }T_1\right)>f\left(\text{\rm ko\v ren }T_2\right)$ {\bf then}\newline 
\phantom{---}vym\v en\'\i me stromy $T_1$ a $T_2$\newline 
{\bf endif}\newline 
ko\v ren $T_2$ p\v ripoj\'\i me jako dal\v s\'\i ho syna ko\v rene $
T_1$
\medskip

\flushpar Je vid\v et, \v ze kdy\v z oba stromy $T_1$ a $T_2$ jsou 
izomorfn\'\i\ s $H_i$, pak v\'ysledn\'y strom operace {\bf spoj} je 
izomorfn\'\i\ se stromem $H_{i+1}$. Korektnost ope\-race {\bf MERGE} plyne z tohoto pozorov\'an\'\i\ a z faktu, \v ze $\Cal H_j$ obsahuje strom 
izomorfn\'\i\ s $H_i$, pr\'av\v e kdy\v z v bin\'arn\'\i m z\'apise 
\v c\'\i sla $|S_j|$ je na $i$-t\'em 
m\'\i st\v e zprava $1$, a \v ze $T$ je nepr\'azdn\'y strom, kdy\v z se 
prov\'ad\'\i\ posun \v r\'adu p\v ri s\v c\'\i t\'an\'\i . Proto\v ze ka\v zd\'y b\v eh 
cyklu vy\v zaduje \v cas $O\left(1\right)$, algoritmus {\bf MER\-GE} vy\v zaduje \v cas 
$O\left(\log\left(|S_1|+|S_2|\right)\right)$. Implementace dal\v s\'\i ch algoritm\accent23u 
je podobn\'a jako pro leftist haldy. 
skip

{\bf INSERT$\left(x\right)$}:\newline 
Vytvo\v r\'\i me haldu $\Cal H_1$ reprezentuj\'\i c\'\i\ $\left\{x\right\}$\newline 
{\bf MERGE}$Cal H,\Cal H_1)$
skip

{\bf MIN}:\newline 
Prohled\'ame prvky reprezentovan\'e ko\v reny strom\accent23u 
v $\Cal H$ a nalez\-ne\-me mezi nimi nejmen\v s\'\i\ prvek
skip

{\bf DELETEMIN}:\newline 
Prohled\'ame prvky reprezentovan\'e ko\v reny strom\accent23u 
v $\Cal H$ a nalez\-neme mezi nimi strom $T$, jeho\v z ko\v ren reprezentuje 
nejmen\v s\'\i\ prvek\newline
$\Cal H_1:=\Cal H\setminus \left\{T\right\}$, vytvo\v r\'\i me haldu $
\Cal H_2$ z podstrom\accent23u $T$ ur\v cen\'ych syny ko\v rene $T$\newline 
{\bf MERGE$\left(\Cal H_1,\Cal H_2\right)$}
skip

\flushpar Z podm\'\i nky \thetag{usp} je z\v rejm\'e, \v ze 
nejmen\v s\'\i\ prvek v $S$ je reprezentov\'an v ko\v reni n\v ejak\'eho  
stromu haldy. T\'\i m dost\'av\'ame korekt\-nost ope\-race {\bf MIN}.
Z tvrzen\'\i\ plyne, \v ze $\Cal H_2$ v ope\-raci {\bf DELE\-TE\-MIN} je 
binomi\'aln\'\i\ halda, a odtud plyne korekt\-nost ope\-race 
{\bf DE\-LE\-TEMIN}. Operace {\bf DECREASE} se implementuje pomoc\'\i\ 
ope\-race {\bf UP} a ope\-race {\bf INCREASE} pomoc\'\i\ 
ope\-race {\bf DOWN} stejn\v e jako pro regul\'arn\'\i\ haldy. Tato 
struktura nepodporuje p\v r\'\i mo operaci {\bf DELETE} (d\'a se ud\v elat 
jedin\v e jako {\bf DECREASE($\infty$)} a pak {\bf DELETEMIN)}) a ope\-race 
{\bf MAKEHEAP} se prov\'ad\'\i\ iterac\'\i\ operace {\bf INSERT}.
\medskip

\flushpar N\'asleduj\'\i c\'\i\ v\v eta popisuj\'\i c\'\i\ efektivitu operac\'\i\ v 
t\'eto struktu\v re vyu\v z\'\i v\'a faktu, \v ze binomi\'aln\'\i\ halda 
reprezentuj\'\i c\'\i\ mno\v zinu $S$ m\'a tolik strom\accent23u, 
kolik je jedni\v cek v bin\'arn\'\i m z\'apise $|S|$, co\v z je nejv\'y\v se 
$\log\left(|S|\right)$, d\'ale \v ze operace {\bf MERGE} simuluje s\v c\'\i t\'an\'\i\ 
\v c\'\i sel $|S_1|$ a $|S_2|$ v bin\'arn\'\i m z\'apise 
a m\'a tedy odpov\'\i daj\'\i c\'\i\ slo\v zitost, 
a kone\v cn\v e \v ze amortizovan\'a slo\v zitost p\v ri\v c\'\i t\'an\'\i\ $
1$ k 
bin\'arn\'\i mu \v c\'\i slu je $O\left(1\right)$. 
\medskip

\proclaim{V\v eta}V binomi\'aln\'\i ch hald\'ach algoritmy operac\'\i\ 
{\bf IN\-SERT}, {\bf MIN}, {\bf DELETEMIN}, {\bf DECREASE} a {\bf MERGE} vy\-\v za\-duj\'\i\ \v cas 
$O\left(\log\left(|S|\right)\right)$, algoritmus operace {\bf INCRE\-ASE} vy\-\v za\-duje \v cas 
$O\left(\log^2\left(|S|\right)\right)$ a algoritmus ope\-race {\bf MAKEHEAP} vy\v zaduje \v cas 
$O\left(|S|\right)$.  
\endproclaim

\flushpar Z tvrzen\'\i\ plyne, \v ze v\'y\v ska strom\accent23u v 
binomi\'aln\'\i\ hald\v e je $\le\log\left(|S|\right)$, ale po\v cet syn\accent23u je 
tak\'e $\le\log\left(|S|\right)$ a tento odhad se ned\'a zlep\v sit. Odtud 
dost\'av\'ame slo\v zitost operac\'\i\ {\bf DECREASE} a {\bf INCREASE} v nejhor\v s\'\i m p\v r\'\i pad\v e.
\medskip

\flushpar Z t\v echto v\'ysledk\accent23u je vid\v et, \v ze ostatn\'\i\ 
haldy maj\'\i\ efektivn\v ej\v s\'\i\ chov\'an\'\i\ ne\v z binomi\'aln\'\i\ haldy. 
V\'yznam binomi\'aln\'\i ch hald spo\-\v c\'\i\-v\'a v tom, \v ze Fibonacciho 
haldy jsou jejich zobecn\v en\'\i m. Na Fibonacciho hald\'ach lze kr\'asn\v e 
ilustrovat princip, \v ze pro \v radu akc\'\i\ je v\'yhodn\'e s nimi po\v ckat 
a neprov\'ad\v et je okam\v zit\v e. Na tomto principu pracuje i 
n\'asleduj\'\i c\'\i\ modifikace binomi\'aln\'\i ch hald.
skip

\subhead
L\'\i n\'a implentace operac\'\i\ binomi\'aln\'\i\ haldy
\endsubhead
\smallskip

\flushpar N\'asleduj\'\i c\'\i\ algoritmy jsou zalo\v zeny na ideji, \v ze 
`vyva\v zov\'an\'\i ' sta\v c\'\i\ prov\'ad\v et jen p\v ri operac\'\i ch {\bf MIN} a 
{\bf DELETEMIN}, kdy stejn\v e mus\'\i me prohledat v\v sechny 
stromy. Z tohoto d\accent23uvodu zeslab\'\i me podm\'\i nky na 
binomi\'aln\'\i\ haldy.
\medskip

\flushpar\emph{L\'\i n\'a} \emph{binomi\'aln\'\i} \emph{halda} $\Cal H$ 
reprezentuj\'\i c\'\i\ mno\v zinu $S$ je soubor strom\accent23u 
$\left\{T_1,T_2,\dots,T_k\right\}$ tako\-v\'y, \v ze  
\roster
\item"{}"
po\v cet vrchol\accent23u v t\v echto stromech je roven 
velikosti $S$ a 
je d\'ano jednozna\v cn\'e p\v ri\v razen\'\i\ prvk\accent23u 
z $S$ vrchol\accent23um strom\accent23u tako\-v\'e, \v ze plat\'\i\ 
podm\'\i nka \thetag{usp};
\item"{}"
ka\v zd\'y strom $T_i$ je izomorfn\'\i\ s n\v ejak\'ym stromem $H_
j$.
\endroster


\flushpar V l\'\i n\'e binomi\'aln\'\i\ hald\v e vynech\'av\'ame p\v redpoklad 
na neizo\-morf\-nost strom\accent23u. Tento fakt se 
projev\'\i\ ve velmi jednoduch\'em algoritmu pro operaci 
{\bf MERGE}.
\medskip

{\bf MERGE$\left(\Cal H_1,\Cal H_2\right)$}:\newline 
Provedeme konkatenaci seznam\accent23u $\Cal H_1$ a $\Cal H_2$.
skip

\flushpar Algoritmus pro operaci {\bf INSERT} se nezm\v en\'\i , jen 
provede tuto implementaci operace {\bf MERGE}.  Operace 
{\bf MIN} a {\bf DELETE\-MIN} pou\v zij\'\i\ n\'asleduj\'\i c\'\i\ pomocnou 
proceduru {\bf vyvaz}.  Vstupem pro tuto operaci je soubor 
seznam\accent23u $\left\{O_i\mid i=0,1,\dots,k\right\}$, kde seznam $O_i$ 
obsahuje jen stromy izomorfn\'\i\ s $H_i$.  
Procedura {\bf vyvaz} pak z t\v echto seznam\accent23u strom\accent23u 
vytvo\v r\'\i\ binomi\'aln\'\i\ hal\-du.  
skip

{\bf vyvaz$\left(\left\{O_i\mid i=0,1,\dots,k\right\}\right)$}:\newline 
$i:=0$, $\Cal H:=\emptyset$\newline 
{\bf while} $i\le k$ nebo $O_i\ne\emptyset$ {\bf do}\newline 
\phantom{---}{\bf while} $|O_i|>1$ {\bf do\newline}
\phantom{------}vezmeme dva r\accent23uzn\'e stromy $T_1$ a $T_2$ z $O_i$ a 
odstran\'\i me je z $O_i$\newline 
\phantom{------}{\bf spoj$\left(T_1,T_2\right)$} vlo\v z\'\i me do $O_{i+1}$\newline 
\phantom{---}{\bf enddo}\newline 
\phantom{---}{\bf if} $O_i\ne\emptyset$ {\bf then}\newline 
\phantom{------}strom $T\in O_i$ odstran\'\i me z $O_i$ a vlo\v z\'\i me do $\Cal H$\newline 
\phantom{---}{\bf endif}\newline 
\phantom{---}$i:=i+1$\newline 
{\bf enddo\newline 
V\'ystup}: $\Cal H$
skip

{\bf MIN}:\newline 
Prohled\'ame v\v sechny stromy v $\Cal H$, nalezneme nejmen\v s\'\i\ prvek 
repre\-zentovan\'y v ko\v reni n\v ejak\'eho stromu a stromy 
rozd\v el\'\i me do mno\v zin $O_i$ obsahuj\'\i c\'\i ch v\v sechny stromy v 
izomorfn\'\i\ s $H_i$.\newline 
{\bf vyvaz$\left(\left\{O_i\mid i=0,1,\dots,\lfloor\log\left(|S|\right)\rfloor \right\}\right)$
skip

DELETEMIN}:\newline 
Prohled\'ame v\v sechny stromy v $\Cal H$, nalezneme nejmen\v s\'\i\ prvek 
repre\-zentovan\'y v ko\v reni n\v ejak\'eho stromu $T\in \Cal H$, stromy 
rozd\v e\-l\'\i me do mno\v zin $O_i$ obsahuj\'\i c\'\i ch v\v sechny stromy  
izomorfn\'\i\ s $H_i$ r\accent23uzn\'e od $T$ a d\'ale d\'ame do 
$O_i$ podstrom $T$ ur\v cen\'y synem ko\v rene $T$, kter\'y je  
izomorfn\'\i\ s $H_i$.\newline 
{\bf vyvaz$\left(\left\{O_i\mid i=0,1,\dots,\lfloor\log\left(|S|\right)\rfloor \right\}\right)$}
skip

\flushpar Amortizovan\'a slo\v zitost operac\'\i\ 
{\bf INSERT} a {\bf MER\-GE} p\v ri l\'\i n\'e implementaci je $O
\left(1\right)$ a 
amortizovan\'a slo\v zitost operac\'\i\ {\bf MIN} a {\bf DELETEMIN  }
je $O\left(\log\left(|S|\right)\right)$. Ohodnocen\'\i m l\'\i n\'e 
binomi\'aln\'\i\ haldy bude po\v cet strom\accent23u v t\'eto 
hald\v e (p\v resn\v eji dvojn\'asobek po\v ctu strom\accent23u). 
Amortizovan\'a slo\v zitost je \v cas operace plus 
ohodnocen\'\i\ v\'ysledn\'e struktury minus ohodnocen\'\i\ 
po\v c\'ate\v cn\'\i\ struktury. L\'\i n\'a implementace operac\'\i\ {\bf MER\-GE} a {\bf INSERT  }
vy\v zaduje \v cas $O\left(1\right)$ a operace {\bf MERGE} nem\v e\-n\'\i\ 
po\v cet strom\accent23u, kde\v zto operace {\bf INSERT} p\v rid\'a jeden 
strom. Odtud amortizovan\'a slo\v zitost obou operac\'\i\ je 
omezena kons\-tantou, a tedy je $O\left(1\right)$. Proto\v ze ka\v zd\'y b\v eh 
vnit\v rn\'\i ho {\bf while}-cyklu v operaci {\bf vyvaz} vy\v zaduje \v cas $
O\left(1\right)$ 
a zmen\v s\'\i\ po\v cet stro\-m\accent23u v seznamech $O_i$ o $1$, 
dostaneme, \v ze ope\-ra\-ce {\bf vyvaz} vy\v zaduje \v cas $O\left(k+\sum_{i=0}^k|O_i|
\right)=O\left(k+|\Cal H|\right)$, kde $k$ je po\v cet 
seznam\accent23u. Z Tvrzen\'\i\ plyne, \v ze $k\le\log\left(|S|\right)$. Ope\-ra\-ce {\bf MIN}
bez podprocedury {\bf vyvaz} vy\v zaduje \v cas $O\left(|\Cal H|\right)$ a operace 
{\bf DE\-LE\-TEMIN} bez podprocedury {\bf vyvaz} vy\v zaduje \v cas $
O\left(\Cal H+i\right)$, 
kde $T$ je izomorfn\'\i\ s $H_i$. Podle tvrzen\'\i\ je $i\le\log\left(
|S|\right)$, a 
tedy ope\-ra\-ce {\bf MIN} vy\v zaduje \v cas $O\left(2|\Cal H|+\log\left(|S|\right)\right)$ a ope\-race 
{\bf DE\-LE\-TEMIN} vy\-\v zaduje \v cas $O\left(2|\Cal H|+2\log\left(|S|\right)\right)$. Proto\v ze  
ohodnocen\'\i\ binomi\'aln\'\i\ haldy je nejv\'y\v se $2\log\left(|S|\right)$, 
dostaneme, \v ze odhad  
amortizovan\'e slo\v zitosti operac\'\i\ {\bf MIN} a {\bf DELETEMIN} je
$$O\left(2|\Cal H|-2|\Cal H|+4\log\left(|S|\right)\right)=O\left(\log\left(|S|\right)\right).$$
\medskip

\flushpar Proto\v ze si funkci ohodnocen\'\i\ vol\'\i me, 
m\accent23u\v zeme pou\v z\'\i t takov\'e multiplikativn\'\i\ 
koeficienty, aby jednotka \v casu odpov\'\i dala jednotce v 
amortizovan\'e slo\v zitosti. Proto lze $|\Cal H|$ od sebe ode\v c\'\i st.

\heading
Fibonacciho haldy
\endheading

\flushpar Zhruba \v re\v ceno, Fibonacciho halda je mno\v zina 
strom\accent23u, kde n\v ekter\'e vrcholy r\accent23uzn\'e od 
ko\v ren\accent23u strom\accent23u jsou ozna\v cen\'e a existuje 
jednozna\v cn\'a korepondence mezi prvky $S$ a vrcholy 
strom\accent23u takov\'a, \v ze spl\v nuje podm\'\i nku 
\thetag{usp}. Bohu\v zel toto je jen p\v ribli\v z\-n\'e vyj\'ad\v ren\'\i . 
Existuj\'\i\ takov\'eto struktury, kter\'e nevznikly z pr\'azd\-n\'e 
Fibonacciho haldy pomoc\'\i\ posloupnosti ope\-rac\'\i\ 
implementovan\'ych navr\v zen\'ymi algoritmy. P\v ritom 
d\accent23ukaz efekti\-vity Fibonacciho hald se dost 
v\'yrazn\v e op\'\i r\'a o fakt, \v ze halda takto vznikla. Proto 
nejd\v r\'\i ve pop\'\i\v seme algoritmy pro operace a pak \v rekneme, 
\v ze \emph{Fibonacciho} \emph{halda} je struktura vznikl\'a 
z pr\'azdn\'e Fibonacciho haldy pomoc\'\i\ posloupnosti operac\'\i , 
kter\'e byly realizov\'any navr\v zen\'ymi algoritmy.
\medskip

\flushpar Budeme p\v redpokl\'adat, \v ze Fibonacciho halda je 
seznam strom\accent23u, kde n\v ekter\'e vrcholy 
r\accent23uzn\'e od ko\v rene jsou ozna\v ceny.  Vrchol je 
ozna\v cen, pr\'av\v e kdy\v z nen\'\i\ ko\v ren a byl mu n\v ekdy d\v r\'\i v v tomto 
strom\v e odtr\v zen n\v ejak\'y jeho syn.  \v Rekneme, \v ze strom m\'a rank $
i$, 
kdy\v z ko\v ren m\'a $i$ syn\accent23u.  
\medskip

\flushpar Algoritmy pro operace {\bf MERGE}, {\bf INSERT}, {\bf MIN} a 
{\bf DELE\-TEMIN} jsou analogick\'e jako pro l\'\i nou implementaci 
v binomi\'aln\'\i ch hald\'ach. Jen po\v zadavek, aby strom byl 
izomorfn\'\i\ s $H_i$, je nahrazen po\v zadavkem, \v ze m\'a rank $
i$. 
Algoritmy pro ope\-race {\bf DECREASE}, {\bf INCREASE} a {\bf DELETE }
jsou zalo\v zeny na algoritmech pro tyto operace v leftist 
hald\'ach. V algoritmech p\v redpokl\'ad\'ame, \v ze $a=\log^{-1}
\left(\frac 32\right)$.
\medskip

\subhead
Algoritmy
\endsubhead
\smallskip

{\bf MERGE$\left(\Cal H_1,\Cal H_2\right)$}:\newline 
Provedeme konkatenaci seznam\accent23u $\Cal H_1$ a $\Cal H_2$.
skip

{\bf INSERT$\left(x\right)$}:\newline 
Vytvo\v r\'\i me haldu $\Cal H_1$ reprezentuj\'\i c\'\i\ $\left\{x\right\}$\newline 
{\bf MERGE}$\left(\Cal H,\Cal H_1\right)$
skip

{\bf MIN}:\newline 
Prohled\'ame v\v sechny stromy v $\Cal H$, nalezneme nejmen\v s\'\i\ prvek 
repre\-zentovan\'y v ko\v reni n\v ejak\'eho stromu a stromy 
rozd\v el\'\i me do mno\v zin $O_i$ obsahuj\'\i c\'\i ch v\v sechny stromy  
s rankem $i$.\newline 
{\bf vyvaz1$\left(\left\{O_i\mid i=0,1,\dots,\lfloor a\log\left(\sqrt 5|S|+1\right)\rfloor 
\right\}\right)$
skip

DELETEMIN}:\newline 
Prohled\'ame v\v sechny stromy v $\Cal H$, nalezneme nejmen\v s\'\i\ prvek 
repre\-zentovan\'y v ko\v reni n\v ejak\'eho stromu $T\in \Cal H$, stromy 
rozd\v e\-l\'\i me do mno\v zin $O_i$ obsahuj\'\i c\'\i ch v\v sechny stromy  
s rankem $i$ r\accent23uzn\'e od $T$ a d\'ale do 
$O_i$ d\'ame v\v sechny podstromy stromu $T$ ur\v cen\'e n\v ejak\'ym synem ko\v rene $T$, kter\'e maj\'\i\ rank $i$.\newline 
{\bf vyvaz1$\left(\left\{O_i\mid i=0,1,\dots,\lfloor a\log\left(\sqrt 5|S|+1\right)\rfloor 
\right\}\right)$
skip

vyvaz1}$\left(\left\{O_i\mid i=0,1,\dots,k\right\}\right)$:\newline 
$i:=0$, $\Cal H:=\emptyset$\newline 
{\bf while} $i\le k$ nebo $O_i\ne\emptyset$ {\bf do}\newline 
\phantom{---}{\bf while} $|O_i|>1$ {\bf do\newline}
\phantom{------}vezmeme dva r\accent23uzn\'e stromy $T_1$ a $T_2$ z $O_i$ a 
odstran\'\i me je z $O_i$\newline 
\phantom{------}{\bf spoj$\left(T_1,T_2\right)$} vlo\v z\'\i me do $O_{i+1}$\newline 
\phantom{---}{\bf enddo}\newline 
\phantom{---}{\bf if} $O_i\ne\emptyset$ {\bf then}\newline 
\phantom{------}strom $T\in O_i$ odstran\'\i me z $O_i$ a vlo\v z\'\i me do $\Cal H$\newline 
\phantom{---}{\bf endif}\newline
\phantom{---}$i:=i+1$\newline 
{\bf enddo\newline 
V\'ystup}: $\Cal H$
skip

{\bf spoj$\left(T_1,T_2\right)$}:\newline 
{\bf if} $f\left(\text{\rm ko\v ren }T_1\right)>f\left(\text{\rm ko\v ren }T_2\right)$ {\bf then}\newline 
\phantom{---}vym\v en\'\i me stromy $T_1$ a $T_2$\newline 
{\bf endif}\newline 
ko\v ren $T_2$ p\v ripoj\'\i me jako dal\v s\'\i ho syna ko\v rene $
T_1$
skip

{\bf DECREASE$\left(s,z\right)$}:\newline 
$T:=$strom v $\Cal H$ reprezentuj\'\i c\'\i\ $s$\newline 
$v:=$vrchol ve stromu $T$ reprezentuj\'\i c\'\i\ $s$\newline 
{\bf if} $v$ je ko\v ren {\bf then}\newline 
\phantom{---}$f\left(v\right):=f\left(v\right)-z$\newline 
{\bf else\newline}
\phantom{---}odtrhneme podstrom $T'$ ur\v cen\'y vrcholem $v$\newline 
\phantom{---}{\bf vyvaz2$\left(T,v\right)$\newline}
\phantom{---}pokud $v$ byl ozna\v cen, zru\v s\'\i me ozna\v cen\'\i\ vrcholu $
v$\newline 
\phantom{---}$f\left(v\right):=f\left(v\right)-z$, $T'$ vlo\v z\'\i me do $\Cal H$\newline 
{\bf endif
skip

INCREASE$\left(s,z\right)$}:\newline 
$T:=$strom v $\Cal H$ reprezentuj\'\i c\'\i\ $s$\newline 
$v:=$vrchol ve stromu $T$ reprezentuj\'\i c\'\i\ $s$\newline 
{\bf if} $v$ nen\'\i\ list {\bf then\newline}
\phantom{---}odtrhneme podstrom $T'$ ur\v cen\'y vrcholem $v$\newline 
\phantom{---}{\bf if} $v$ nen\'\i\ ko\v ren {\bf then vyvaz2$\left(T,v\right)$ endif\newline}
\phantom{---}pokud $v$ byl ozna\v cen, zru\v s\'\i me ozna\v cen\'\i\ vrcholu $
v$\newline 
\phantom{---}$f\left(v\right):=f\left(v\right)+z$,\newline 
\phantom{---}{\bf for every} $u$ syn vrcholu $v$ {\bf do\newline}
\phantom{------}zru\v s\'\i me ozna\v cen\'\i\ vrcholu $u$ \newline 
\phantom{------}do $\Cal H$ vlo\v z\'\i me podstrom $T'$ ur\v cen\'y vrcholem $u$\newline 
\phantom{---}{\bf enddo\newline}
\phantom{---}do $\Cal H$ vlo\v z\'\i me strom maj\'\i c\'\i\ jen vrchol $v$\newline
{\bf else}\newline 
\phantom{---}$f\left(v\right):=f\left(v\right)+z$\newline 
{\bf endif
skip

DELETE$\left(s\right)$}:\newline 
$T:=$strom v $\Cal H$ reprezentuj\'\i c\'\i\ $s$\newline 
$v:=$vrchol ve stromu $T$ reprezentuj\'\i c\'\i\ $s$\newline 
odtrhneme podstrom $T'$ ur\v cen\'y vrcholem $v$\newline 
{\bf if} $v$ nen\'\i\ ko\v ren {\bf then vyvaz2$\left(T,v\right)$ endif}\newline 
zru\v s\'\i me ozna\v cen\'\i\ u v\v sech syn\accent23u vrcholu $
v$ a\newline 
do $\Cal H$ vlo\v z\'\i me v\v sechny podstromy $T'$ ur\v cen\'e syny $
v$
skip

{\bf vyvaz2$\left(T,v\right)$}:\newline 
$u:=\otec v$\newline 
{\bf while} $u$ je ozna\v cen {\bf do\newline 
\phantom{---}$u':=\otec\left(u\right)$}, zru\v s\'\i me ozna\v cen\'\i\ $u$\newline 
\phantom{---}odtrhneme podstrom $T$ ur\v cen\'y vrcholem $u$ a vlo\v z\'\i me ho 
do $\Cal H$, $u:=u'$\newline 
{\bf enddo\newline 
if} $u$ nen\'\i\ ko\v ren $T$ {\bf then} ozna\v c\'\i me $u$ {\bf endif
\medskip

}\flushpar V\v simn\v eme si, \v ze kdy\v z stromy $T_1$ a $T_2$ maj\'\i\ rank 
$i$, pak procedura {\bf spoj$\left(T_1,T_2\right)$} vytvo\v r\'\i\ strom s rankem 
$i+1$.  Aby algoritmy pro operace {\bf MIN} a {\bf DELETEMIN} byly 
korektn\'\i , mus\'\i me uk\'azat, \v ze v\v sechny stromy ve 
Fibonacciho hald\v e $\Cal H$ reprezentuj\'\i c\'\i\ mno\v zinu $
S$ maj\'\i\ rank 
nejv\'y\v se $a\log\left(\sqrt 5|S|+1\right)$. Jen tak zajist\'\i me, aby v\'ysledn\'a 
halda reprezentovala $S$, res\-pektive 
$S\setminus \left\{\text{\rm prvek s nejmen\v s\'\i\ hodnotou }f\right\}$.  Operace {\bf vyvaz2 }
zaji\v s\v tuje, \v ze od ka\v zd\'eho vrcholu stromu r\accent23uzn\'eho 
od ko\v rene byl v tomto strom\v e odtr\v zen podstrom nejv\'y\v se 
jednoho syna -- v tom p\v r\'\i pad\v e je tento prvek ozna\v cen a 
kdy\v z se mu odtrh\'av\'a podstrom dal\v s\'\i ho syna, bude odtr\v zen 
i cel\'y podstrom tohoto vrcholu (t\'\i m se stane ko\v renem 
stromu).  Kdy\v z se pozd\v eji stane tento vrchol zase 
vrcholem r\accent23uzn\'ym od ko\v rene, cel\'y proces se 
opakuje. 
\medskip

\subhead
Slo\v zitost operac\'\i
\endsubhead
\smallskip

\flushpar Nejd\v r\'\i ve spo\v c\'\i t\'ame \v casovou slo\v zitost 
jednotliv\'ych operac\'\i :
\roster
\item"{}"
{\bf MERGE }\v casov\'a slo\v zitost $O\left(1\right)$, nevznik\'a \v z\'adn\'y nov\'y strom, ozna\-\v ce\-n\'e vrcholy se nem\v en\'\i;
\item"{}" 
{\bf INSERT }\v casov\'a slo\v zitost $O\left(1\right)$, p\v ribyl jeden strom, 
ozna\v cen\'e vrcholy se nem\v en\'\i;
\item"{}"
{\bf MIN }\v casov\'a slo\v zitost $O\left(|\Cal H|\right)$, po proveden\'\i\ operace 
r\accent23uzn\'e stro\-my v hald\v e maj\'\i\ r\accent23uzn\'e 
ranky;  
\item"{}"
{\bf DELETEMIN }\v casov\'a slo\v zitost $O\left(|\Cal H|+\text{\rm po\v cet syn\accent23u }v\right)$,kde $v$ reprezentoval prvek s nejmen\v s\'\i\ hodnotou $f$, po proveden\'\i\ ope\-ra\-ce r\accent23uzn\'e stromy v hald\v e maj\'\i\ 
r\accent23uzn\'e ranky, \v z\'adn\'y nov\'y vrchol nebyl ozna\v cen; 
\item"{}"
{\bf DECREASE }\v casov\'a slo\v zitost $O\left(1+c\right)$, kde $c$ je po\v cet 
vrchol\accent23u, kter\'e p\v restaly b\'yt ozna\v cen\'e, 
bylo p\v rid\'ano $1+c$ nov\'ych strom\accent23u a byl ozna\v cen 
nejv\'y\v se jeden vrchol;
\item"{}"
{\bf INCREASE }\v casov\'a slo\v zitost $O\left(1+c+d\right)$, kde $c$ je po\v cet 
vrchol\accent23u, kter\'e p\v restaly b\'yt ozna\v cen\'e, $d$ je 
po\v cet syn\accent23u vrcholu $v$, bylo p\v rid\'ano $1+c+d$ 
nov\'ych strom\accent23u a byl ozna\v cen nejv\'y\v se jeden 
vrchol;
\item"{}"
{\bf DELETE }\v casov\'a slo\v zitost $O\left(1+c+d\right)$, kde $c$ je po\v cet 
vrchol\accent23u, kter\'e p\v restaly b\'yt ozna\v cen\'e, $d$ je 
po\v cet syn\accent23u vrcholu $v$, bylo p\v rid\'ano $c+d$ 
nov\'ych strom\accent23u a byl ozna\v cen nejv\'y\v se jeden 
vrchol.
\endroster

\flushpar Abychom spo\v c\'\i tali amortizovanou slo\v zitost, 
mus\'\i me nejd\v r\'\i ve navrhnout funkci ohodnocuj\'\i\-c\'\i\ 
konfigurace.  Nech\v t ohodnocen\'\i\ konfigurace je po\v cet 
strom\accent23u v konfiguraci plus dvojn\'aso\-bek po\v ctu 
ozna\v cen\'ych vrchol\accent23u.  Nech\v t $\rho \left(n\right)$ je maxim\'aln\'\i\ 
po\v cet syn\accent23u vrcholu ve Fibonacciho hald\v e 
reprezentuj\'\i c\'\i\ $n$-prvkovou mno\v zinu.  Pak amortizovan\'a 
slo\v zitost operac\'\i\ je:\newline 
\phantom{---}{\bf MERGE} amortizovan\'a slo\v zitost je $O\left(1\right)$;\newline 
\phantom{---}{\bf INSERT} amortizovan\'a slo\v zitost je $O\left(1\right)$;\newline 
\phantom{---}{\bf MIN} amortizovan\'a slo\v zitost je $O\left(\rho \left(n\right)\right)$;\newline 
\phantom{---}{\bf DELETEMIN} amortizovan\'a slo\v zitost je 
$O\left(\rho \left(n\right)\right)$;\newline 
\phantom{---}{\bf DECREASE} amortizovan\'a slo\v zitost je $O\left(1\right)$;\newline 
\phantom{---}{\bf INCREASE} amortizovan\'a slo\v zitost je 
$O\left(\rho \left(n\right)\right)$;\newline 
\phantom{---}{\bf DELETE} amortizovan\'a slo\v zitost je $O\left(\rho \left(n\right)\right)$.
\medskip

\flushpar Abychom spo\v c\'\i tali odhad $\rho \left(n\right)$, vyu\v zijeme toho, \v ze 
Fibonacciho halda vznikla z pr\'azdn\'e haldy pomoc\'\i\ 
popsan\'ych algoritm\accent23u. Nejprve jedno technick\'e 
lemma.
\medskip

\proclaim{Lemma}Nech\v t $v$ je vrchol stromu ve 
Fibonacciho hald\v e a nech\v t $u$ je $i$-t\'y nejstar\v s\'\i\ syn 
vrcholu $v$, pak $u$ m\'a aspo\v n $i-2$ syn\accent23u.
\endproclaim

\demo{D\accent23ukaz}Kdy\v z se $u$ st\'aval synem $v$, 
aplikovala se operace {\bf spoj} na stromy s ko\v reny $u$ a $v$. 
P\v ritom vrcholy $u$ a $v$ m\v ely stejn\'y po\v cet syn\accent23u. 
Podle p\v redpoklad\accent23u m\v el vrchol $v$ alespo\v n $i-1$ 
syn\accent23u (jinak by $u$ nebyl $i$-t\'y nejstar\v s\'\i\ syn), a 
proto\v ze od $u$ se mohl odtrhnout jen jeden syn, 
dost\'av\'ame, \v ze $u$ mus\'\i\ m\'\i t alespo\v n $i-2$ syn\accent23u. \qed
\enddemo


\proclaim{Tvrzen\'\i}Nech\v t $v$ je vrchol stromu ve 
Fibonacciho hald\v e, kte\-r\'y m\'a pr\'av\v e $i$ syn\accent23u, pak 
podstrom ur\v cen\'y vrcholem $v$ m\'a aspo\v n $F_{i+2}$ 
vrchol\accent23u.
\endproclaim

\demo{D\accent23ukaz}Tvrzen\'\i\ dok\'a\v zeme pomoc\'\i\ indukce 
podle maxim\'aln\'\i\ d\'elky cesty z vrcholu $v$ do n\v ekter\'eho 
listu.  Tato d\'elka je $0$, pr\'av\v e kdy\v z $v$ je list.  V tom 
p\v r\'\i pad\v e $v$ nem\'a syna a podstrom ur\v cen\'y vrcholem $
v$ m\'a 
jedin\'y vrchol.  Proto\v ze $1=F_2=F_{0+2}$, tak tvrzen\'\i\ plat\'\i .  
M\v ejme vrchol $v$, kter\'y m\'a $k$ 
syn\accent23u, a nech\v t maxim\'aln\'\i\ d\'elka cesty z vrcholu $
v$ do 
list\accent23u je $j$. P\v redpokl\'adej\-me, \v ze tvrzen\'\i\ plat\'\i\ pro 
v\v sechny vrcholy, pro n\v e\v z maxim\'aln\'\i\ d\'elka cesty z nich 
do list\accent23u je men\v s\'\i\ ne\v z $j$.  Tedy tvrzen\'\i\ plat\'\i\ pro 
v\v sechny syny vrcholu $v$.  Pak pro $i>1$ m\'a $i$-t\'y nejstar\v s\'\i\ 
syn vrcholu $v$ podle p\v redchoz\'\i ho lemmatu alespo\v n $i-2$ 
syn\accent23u a podle induk\v cn\'\i\ hypot\'ezy podstrom ur\v cen\'y 
t\'\i mto synem m\'a alespo\v n $F_i$ vrchol\accent23u.  Odtud 
dost\'av\'ame, \v ze podstrom ur\v cen\'y vrcholem $v$ m\'a alespo\v n 
$$1+F_2+\sum_{i=2}^kF_i=1+\sum_{i=1}^kF_i$$
vrchol\accent23u, proto\v ze $F_1=F_2$ (prvn\'\i\ $1$ je za vrchol $
v$, 
prvn\'\i\ $F_2$ je za nejstar\v s\'\i\ vrchol). Indukc\'\i\ dok\'a\v zeme, \v ze 
$$1+\sum_{i=1}^nF_i=F_{n+2}$$
pro v\v sechna $n\ge 0$. Skute\v cn\v e, pro $n=0$ plat\'\i\ 
$$1+\sum_{i=1}^0F_i=1=F_2=F_{0+2}$$
a pro $n=1$ m\'ame 
$$1+\sum_{i=1}^1F_i=1+F_1=2=F_3=F_{1+2}.$$
\medskip

\flushpar D\'ale indukc\'\i\ dost\'av\'ame, \v ze 
$$1+\sum_{i=1}^nF_i=1+\sum_{i=1}^{n-1}F_i+F_n=F_{n+1}+F_n=F_{n+2}
.$$
Kdy\v z shrneme tato fakta, dost\'av\'ame, \v ze podstrom 
ur\v cen\'y vrcholem $v$ m\'a alespo\v n $F_{k+2}$ vrchol\accent23u, a 
tvrzen\'\i\ je dok\'az\'ano. \qed
\enddemo

\flushpar Vezm\v eme nejmen\v s\'\i\ $i$ takov\'e, \v ze $n<F_i$. Proto\v ze 
posloupnost $\left\{F_i\right\}_{i=1}^{\infty}$ je rostouc\'\i , plyne z p\v redchoz\'\i ho 
tvrzen\'\i , \v ze ka\v zd\'y vrchol ve Fibonacciho hald\v e 
reprezentuj\'\i c\'\i\ $n$ prvkovou mno\v zinu m\'a m\'en\v e ne\v z $
i-2$ 
syn\accent23u (kdy\v z vrchol $v$ Fibonacciho haldy m\'a $i-2$ 
syn\accent23u, pak podstrom vrcholu $v$ reprezentuje 
mno\v zinu ales\-po\v n s $F_i$ prvky). Proto $\rho \left(n\right)<i-2$. K odhadu 
velikosti $i$ pou\v zijeme explicitn\'\i\ vzorec 
pro $i$-t\'e Fibonacciho \v c\'\i slo: 
$$F_i=\frac {\left(\frac {1+\sqrt 5}2\right)^i-\left(\frac {1-\sqrt 5}
2\right)^i}{\sqrt 5}=\frac 1{\sqrt 5}\left(\frac {1+\sqrt 5}2\right)^i
-\frac 1{\sqrt 5}\left(\frac {1-\sqrt 5}2\right)^i.$$
\medskip

\flushpar Proto\v ze $0>\frac {1-\sqrt 5}2>-\frac 34$ a proto\v ze $\sqrt 5
>2$, je $|\frac 1{\sqrt 5}\left(\frac {1-\sqrt 5}2\right)^i|<\frac 3
8$ pro v\v sechna 
$i=1,2,\dots$, a tedy 
$$\frac 1{\sqrt 5}\left(\frac {1+\sqrt 5}2\right)^i-\frac 38<F_i<\frac 
1{\sqrt 5}\left(\frac {1+\sqrt 5}2\right)^i+\frac 38.$$
Odtud dost\'av\'ame, \v ze kdy\v z $i$ spl\v nuje 
$$n\le\frac 1{\sqrt 5}\left(\frac {1+\sqrt 5}2\right)^i-\frac 38,$$
pak $n<F_i$. 
P\v reveden\'\i m $\frac 38$ na druhou stranu v\'yrazu, jeho 
vyn\'asoben\'\i m $\sqrt 5$ a zlogaritmov\'an\'\i m dostaneme n\'asleduj\'\i c\'\i\ 
ekvivalenci: 
$$\gather\log_2\left(\sqrt 5n+\frac {3\sqrt 5}8\right)\le i\log_2\left(\frac {1+\sqrt 
5}2\right)\quad\Leftrightarrow\\ n\le\frac 1{\sqrt 5}\left(\frac {
1+\sqrt 5}2\right)^i-\frac 38.\endgather$$
Z $\frac {3\sqrt 5}8<1$ a z $\frac 32<\frac {1+\sqrt 5}2$ plyne, \v ze 
$$\frac {\log_2\left(\sqrt 5n+\frac {3\sqrt 5}8\right)}{\log_2\frac {1+\sqrt 
5}2}<\frac {\log_2\left(\sqrt 5n+1\right)}{\log_2\frac 32}.$$
Tedy plat\'\i\ n\'asleduj\'\i c\'\i\ implikace:
$$\frac {\log_2\left(\sqrt 5n+1\right)}{\log_2\frac 32}<i\quad\implies\frac {\log_
2\left(\sqrt 5n+\frac {3\sqrt 5}8\right)}{\log_2\left(\frac {1+\sqrt 5}2\right)}
<i.$$
Proto kdy\v z $\frac {\log_2\left(\sqrt 5n+1\right)}{\left(\log_23\right)-1}<i$, pak $n<F_
i$, a tedy $\rho \left(n\right)<i-2$.
\medskip

\flushpar V\'ysledky shrneme do n\'asleduj\'\i c\'\i\ v\v ety: 

\proclaim{V\v eta}Ve Fibonacciho hald\v e, kter\'a   
reprezentuje $n$ prvkovou mno\v zinu, m\'a ka\v zd\'y vrchol 
stupe\v n men\v s\'\i\ ne\v z 
$$\frac {\log_2\left(\sqrt 5n+1\right)}{\left(\log_23\right)-1}-2=O\left(\log n\right).$$
Amortizovan\'a slo\v zitost operac\'\i\ {\bf INSERT}, {\bf MERGE} a {\bf DECREASE }
je $O\left(1\right)$ a amortizovan\'a slo\v zitost ope\-rac\'\i\ {\bf MIN}, 
{\bf DE\-LE\-TE\-MIN}, {\bf INCREASE} a {\bf DELETE} je $O\left(\log n
\right)$. Ope\-race {\bf MIN }
a {\bf DELETEMIN} jsou korektn\'\i.
\endproclaim


\flushpar Pro \'uplnost dok\'a\v zeme, \v ze $F_i=\frac {\left(\frac {
1+\sqrt 5}2\right)^i-\left(\frac {1-\sqrt 5}2\right)^i}{\sqrt 5}$ .\newline 
Pro $i=1$ plat\'\i\ 
$$\frac {\left(\frac {1+\sqrt 5}2\right)^1-\left(\frac {1-\sqrt 5}2
\right)^1}{\sqrt 5}=\frac {1+\sqrt 5-1+\sqrt 5}{2\sqrt 5}=\frac {2\sqrt 
5}{2\sqrt 5}=1=F_1.$$
Pro $i=2$ plat\'\i\ 
$$\align\frac {\left(\frac {1+\sqrt 5}2\right)^2-\left(\frac {1-\sqrt 
5}2\right)^2}{\sqrt 5}=&\frac {1+2\sqrt 5+5-1+2\sqrt 5-5}{4\sqrt 5}
=\\
&\frac {4\sqrt 5}{4\sqrt 5}=1=F_2.\endalign$$
\medskip

\flushpar Induk\v cn\'\i\ krok:
$$\align&\frac {\left(\frac {1+\sqrt 5}2\right)^i-\left(\frac {1-\sqrt 
5}2\right)^i}{\sqrt 5}=\\
&\frac {\left(\frac {1+\sqrt 5}2\right)^{i-2}\left(\frac {1+\sqrt 5}2
\right)^2-\left(\frac {1-\sqrt 5}2\right)^{i-2}\left(\frac {1-\sqrt 5}2
\right)^2}{\sqrt 5}=\\
&\frac {\left(\frac {1+\sqrt 5}2\right)^{i-2}\left(\frac {3+\sqrt 5}2
\right)-\left(\frac {1-\sqrt 5}2\right)^{i-2}\left(\frac {3-\sqrt 5}2\right)}{\sqrt 
5}=\\
&\frac {\left(\frac {1+\sqrt 5}2\right)^{i-2}\left(1+\frac {1+\sqrt 5}
2\right)-\left(\frac {1-\sqrt 5}2\right)^{i-2}\left(1+\frac {1-\sqrt 5}
2\right)}{\sqrt 5}=\\
&\frac {\left(\frac {1+\sqrt 5}2\right)^{i-2}+\left(\frac {1+\sqrt 5}
2\right)^{i-1}-\left(\frac {1-\sqrt 5}2\right)^{i-2}-\left(\frac {1-\sqrt 
5}2\right)^{i-1}}{\sqrt 5}=\\
&\frac {\left(\frac {1+\sqrt 5}2\right)^{i-2}-\left(\frac {1-\sqrt 5}
2\right)^{i-2}}{\sqrt 5}+\frac {\left(\frac {1+\sqrt 5}2\right)^{i-1}-
\left(\frac {1-\sqrt 5}2\right)^{i-1}}{\sqrt 5}=\\
&F_{i-2}+F_{i-1}=F_i.\endalign$$
Tedy indukc\'\i\ dost\'av\'ame po\v zadovan\'y vztah.
\medskip

\subhead
Hled\'an\'\i\ nejkrat\v s\'\i ch cest
\endsubhead
\smallskip

Vr\'at\'\i me se k Dijkstrov\v e algoritmu. 
Mno\v zinu $U$ bude\-me reprezentovat pomoc\'\i\ Fibonacciho 
haldy. Proto\v ze ohodnocen\'\i\ hald je nez\'aporn\'e a ohodnocen\'\i\ 
po\v c\'ate\v cn\'\i\ haldy je $0$, d\'av\'a odhad amortizovan\'e slo\v zitosti  
tak\'e odhad \v casov\'e slo\v zitos\-ti. Proto \v casov\'a slo\v zitost Dijkstrova 
algoritmu v nejhor\v s\'\i m p\v r\'\i pad\v e je  
$O\left(|X|\left(1+\log|X|\right)+|R|\right)=O\left(|R|+|X|\log|X|\right)$. Stejn\'y v\'ysledek 
dostane\-me i pro konstrukci nejmen\v s\'\i\ napnut\'e kostry 
grafu. 
\medskip

\flushpar Ot\'azka je, kdy pou\v z\'\i t Fibonacciho haldu a kdy 
pou\v z\'\i t $d$-regu\-l\'ar\-n\'\i\ haldy v Dijkstrov\v e algoritmu nebo v 
algoritmu konstruuj\'\i\-c\'\i m nejmen\v s\'\i\ napnutou kostru. Lze 
\v r\'\i ci, \v ze Fibonacciho halda by m\v ela b\'yt v\'yrazn\v e lep\v s\'\i\ pro 
v\v et\v s\'\i , ale \v r\'\i dk\'e grafy (tj. grafy s mal\'ym po\v ctem hran). 
D\'a se p\v redpokl\'adat, \v ze $d$-regul\'arn\'\i\ haldy budou lep\v s\'\i\ 
(d\'\i ky sv\'ym jednodu\v s\v s\'\i m algoritm\accent23um) pro hust\'e 
grafy (tj. grafy, kde po\v cet hran je $|X|^{1+\varepsilon}$ pro vhodn\'e 
$\varepsilon >0$). Probl\'em je, pro kter\'e hodnoty nast\'av\'a zlom. 
Nev\'\i m o \v z\'adn\'ych experiment\'aln\'\i ch nebo teoretick\'ych 
v\'ysledc\'\i ch tohoto typu.
\medskip

\flushpar Historick\'y p\v rehled: Bin\'arn\'\i\ neboli $2$-regul\'arn\'\i\ 
haldy zavedl Williams 1964. Jejich zobecn\v e\-n\'\i\ na $d$-regul\'arn\'\i\ 
haldy poch\'az\'\i\ od Johnsona 1975. Leftist haldy definoval Crane 1972 
a detailn\v e popsal Knuth 1975. Binomi\'aln\'\i\ haldy navrhl Vuillemin 1978, 
Brown 1978 je implementoval a prok\'azal jejich praktickou pou\v zitelnost. 
Fibonacciho haldy byly zavedeny Fredmanem a Tarjanem 1987.
  
\heading
T\v r\'\i dic\'\i\ algoritmy
\endheading

\flushpar Jeden ze z\'akladn\'\i ch probl\'em\accent23u datov\'ych 
struktur je n\'asleduj\'\i c\'\i :\newline 
$U$ je tot\'aln\v e uspo\v r\'adan\'e univerzum.\newline 
Vstup: Prost\'a posloupnost $\left\{a_1,a_2,\dots,a_n\right\}$ prvk\accent23u z 
univerza $U$.\newline 
V\'ystup: Rostouc\'\i\ posloupnost $\left\{b_1,b_2,\dots,b_n\right\}$ takov\'a, \v ze 
$\left\{a_i\mid i=1,2,\dots,n\right\}=\left\{b_i\mid i=1,2,\dots,n\right\}$.
Tento probl\'em se naz\'yv\'a \emph{t\v r\'\i d\v en\'\i}. V mnoha 
aplikac\'\i ch datov\'ych struktur je nutn\'e ho \v re\v sit. 
\medskip

\flushpar Jsou t\v ri z\'akladn\'\i\ algoritmy, kter\'e  \v re\v s\'\i\ t\v r\'\i dic\'\i\ 
probl\'em: {\bf HEAP\-SORT}, {\bf MERGESORT}, {\bf QUICKSORT}. 
{\bf HEAPSORT} byl prvn\'\i\ algoritmus pou\v z\'\i vaj\'\i c\'\i\ 
haldy (bin\'arn\'\i\ regul\'ar\-n\'\i\ haldy by\-ly definov\'any p\v ri 
n\'avrhu {\bf HEAP\-SORTu}). Byl pops\'an jako jedna z 
aplikac\'\i\ regul\'arn\'\i ch hald. Je mu  
st\'ale v\v enov\'ana velk\'a pozornost a bylo navr\v ze\-no n\v ekolik 
jeho modifikac\'\i . \v Rekneme si v\'\i ce o implementaci 
t\v r\'\i d\v en\'\i\ na m\'\i st\v e.
\medskip

\flushpar T\v r\'\i dic\'\i\ algoritmy se \v casto pou\v z\'\i vaj\'\i\ jako 
podprocedura p\v ri \v re\-\v se\-n\'\i\ jin\'ych \'uloh. V takov\'em p\v r\'\i pad\v e je 
obvykle vstupn\'\i\ posloupnost ulo\v zena v poli v pracovn\'\i\ 
pam\v eti programu a po\v za\-dav\-kem  je set\v r\'\i dit ji 
bez pou\v zit\'\i\ dal\v s\'\i\ 
pam\v eti pouze s v\'yjimkou omezen\'eho (mal\'eho) po\v ctu pomocn\'ych 
prom\v enn\'ych. Pro \v re\v se\-n\'\i\ tohoto probl\'emu se hod\'\i\ 
{\bf HEAPSORT} implementovan\'y pomoc\'\i\ 
$d$-regul\'arn\'\i ch hald, kter\'e jsou reprezentov\'any  
polem, v n\v em\v z je ulo\v zena vstupn\'\i\ posloupnost. Pou\v zijeme 
algoritmus s jedinou zm\v enou -- budeme po\v zadovat du\'aln\'\i\ 
podm\'\i nku na uspo\v r\'ad\'an\'\i\ 
a nahrad\'\i me operace {\bf MIN} a {\bf DELETEMIN }
ope\-racemi {\bf MAX} a {\bf DELETEMAX}. V algoritmu v\v zdy um\'\i st\'\i\-me 
odebran\'e maximum na m\'\i sto prvku v posled\-n\'\i m listu 
haldy (tj. prvku, kter\'y ho p\v ri operaci {\bf DELETEMAX }
nahradil) m\'\i sto toho, abychom ho vlo\v zili 
do v\'ystupn\'\i\ posloupnosti. 
\medskip

\flushpar Nejstar\v s\'\i\ z uveden\'ych algoritm\accent23u je 
{\bf MERGESORT} a je star\v s\'\i\ ne\v z je po\v c\'\i ta\v cov\'a \'era. Jeho 
verze se pou\v z\'\i valy u\v z p\v ri mecha\-nick\'em t\v r\'\i d\v en\'\i . Pop\'\i\v seme 
jednu jeho itera\v cn\'\i\ verzi. 
skip

{\bf MERGESORT$\left(a_1,a_2,\dots,a_n\right)$}:\newline 
$Q:=$ pr\'azdn\'a fronta, $i:=1$\newline 
{\bf while} $i\le n$ {\bf do}\newline 
\phantom{---}$j:=i$\newline 
\phantom{---}{\bf while} $i<n$ a $a_{i+1}>a_i$ {\bf do} $i:=i+1$ {\bf enddo} \newline 
\phantom{---}posloupnost $P=\left(a_j,a_{j+1},\dots,a_i\right)$ vlo\v z\'\i me do $Q$\newline 
\phantom{---}$i:=i+1$\newline 
{\bf enddo\newline 
while} $|Q|>1$ {\bf do}\newline 
\phantom{---}vezmeme $P_1$ a $P_2$ dv\v e posloupnosti z vrcholu $Q$\newline 
\phantom{---}odstran\'\i me $P_1$ a $P_2$ z $Q$\newline 
\phantom{---}{\bf MERGE$\left(P_1,P_2\right)$} vlo\v z\'\i me na konec $Q$\newline 
{\bf enddo\newline 
V\'ystup}: posloupnost z $Q$
skip

{\bf MERGE$\left(P_1=\left(a_1,a_2,\dots,a_n\right),P_2=\left(b_1,b_2,\dots,b_m\right)\right)$}:\newline 
$P:=$ pr\'azdn\'a posloupnost, $i:=1$, $j:=1$, $k:=1$\newline 
{\bf while} $i\le n$ a $j\le m$ {\bf do}\newline 
\phantom{---}{\bf if} $a_i<b_j$ {\bf then}\newline 
\phantom{------}$c_k:=a_i$, $i:=i+1$, $k:=k+1$\newline 
\phantom{---}{\bf else}\newline 
\phantom{------}$c_k:=b_j$, $j:=j+1$, $k:=k+1$\newline 
\phantom{---}{\bf endif\newline 
enddo}\newline 
{\bf while} $i\le n$ {\bf do}\newline 
\phantom{---}$c_k:=a_i$, $i:=i+1$, $k:=k+1$\newline 
{\bf enddo\newline 
while} $j\le m$ {\bf do}\newline 
\phantom{---}$c_k:=b_j$, $j:=j+1$, $k:=k+1$\newline 
{\bf enddo\newline 
V\'ystup:} $P=\left(c_1,c_2,\dots,c_{n+m}\right)$
skip

\flushpar V\v simn\v eme si, \v ze v\v sechny posloupnosti v $Q$ jsou 
rostouc\'\i\ a \v ze mno\v zina $\left\{a_i\mid i=1,2,\dots,n\right\}$ je sjednocen\'\i m 
v\v sech prvk\accent23u z posloupnost\'\i\ v $Q$ v\v zdy na za\v c\'atku 
b\v ehu cyklu {\bf while} $|Q|>1$. Ka\v zd\'y pr\accent23ub\v eh tohoto 
cyklu zmen\v s\'\i\ po\v cet posloupnost\'\i\ v $Q$ o $1$. Proto\v ze po\v cet 
posloupnost\'\i\ ve front\v e $Q$ je nejv\'y\v se d\'elka vstupn\'\i\ 
posloupnosti, je algoritmus 
{\bf MERGE\-SORT} korektn\'\i .
\medskip

\flushpar Slo\v zitost podprocedury {\bf MERGE}. Ur\v cen\'\i\ prvku $c_k$ 
vy\v za\-du\-je \v cas $O\left(1\right)$ (nejv\'y\v se jedno porovn\'an\'\i ) a 
maxim\'aln\'\i\ hodnota $k$ je $n+m$. Tedy 
podprocedura {\bf MERGE} vy\v zaduje \v cas $O\left(n+m\right)$ (a provede 
nejv\'y\v se $n+m$ 
porovn\'an\'\i ), kde $n$ a $m$ jsou d\'elky vstupn\'\i ch posloupnost\'\i .
\medskip

\flushpar Slo\v zitost procedury {\bf MERGESORT}. 
Prvn\'\i\ cyklus vy\v zaduje \v cas $O\left(n\right)$, kde $n$ je d\'elka vstupn\'\i\ posloupnosti. 
P\v red prvn\'\i\ b\v eh cyklu {\bf while} polo\v z\'\i me na vrchol $Q$ speci\'aln\'\i\ 
znak $\natural$, kter\'y se v\v zdy jen 
p\v renese z vrcholu $Q$ na jej\'\i\ konec. Proto\v ze mezi dv\v ema 
p\v renosy znaku $\natural$ projde ka\v zd\'y prvek 
podprocedurou {\bf MERGE} pr\'av\v e jednou, vy\v zaduj\'\i\ b\v ehy cyklu 
{\bf while} mezi dv\v ema p\v renosy $\natural$ \v cas $
O\left(n\right)$. V\v sech\-ny posloupnosti 
na po\v c\'atku maj\'\i\ d\'elku $\ge 1$, a proto po $i$-t\'em p\v renosu 
$\natural$ maj\'\i\ d\'elku $\ge 2^{i-1}$ a po\v cet p\v renos\accent23u 
$\natural$ je 
nejv\'y\v se $\left\lceil\log_2n\right\rceil$. Tedy algoritmus {\bf MERGESORT }
vy\v zaduje \v cas $O\left(n\log n\right)$.
\medskip

\flushpar Nyn\'\i\ pop\'\i\v seme algoritmus {\bf QUICKSORT}. Je to 
nejv\'\i ce pou\-\v z\'\i van\'y algoritmus, proto\v ze pro obecn\v e danou 
posloupnost m\'a nejlep\v s\'\i\ o\v cek\'avan\'y \v cas. 
skip

{\bf Quick$\left(a_i,a_{i+1},\dots,a_j\right)$}:\newline 
{\bf if} $i=j$ {\bf then}\newline 
\phantom{---}{\bf V\'ystup}: $\left(a_i\right)$\newline 
{\bf else}\newline 
\phantom{---}zvol\'\i me $k$ takov\'e, \v ze $i\le k\le j$, $a:=a_k$,\newline 
\phantom{---}vym\v en\'\i me $a_i$ a $a_k$, $l:=i+1$, $q:=j$\newline 
\phantom{---}{\bf while} true {\bf do}\newline 
\phantom{------}{\bf while} $a_l<a$ {\bf do} $l:=l+1$ {\bf enddo}\newline 
\phantom{------}{\bf while} $a_q>a$ {\bf do} $q:=q-1$ {\bf enddo}\newline
\phantom{------}{\bf if} $l\ge q$ {\bf then}\newline 
\phantom{---------}exit\newline 
\phantom{------}{\bf else}\newline 
\phantom{---------}vym\v en\'\i me $a_l$ a $a_q$, $l:=l+1$, $q:=q-1$\newline 
\phantom{------}{\bf endif}\newline
\phantom{---}{\bf enddo}\newline 
\phantom{---}{\bf if} $i+1=l$ {\bf then}\newline 
\phantom{------}{\bf V\'ystup}$\left(a,\text{Quick}\left(a_{q+1},a_{q+2},\dots,a_j\right)\right)$\newline
\phantom{---}{\bf else}\newline 
\phantom{------}{\bf if} $j=q$ {\bf then}\newline 
\phantom{---------}{\bf V\'ystup}$\left(\text{Quick}\left(a_{i+1},\dots,a_{l-1}\right),a\right)$\newline \phantom{------}{\bf else}\newline 
\phantom{---------}{\bf V\'ystup}$\left(\text{Quick}\left(a_{i+1},\dots,a_{l-1}\right),a,\text{Quick}\left(a_{q+1},\dots,a_j\right)\right)$\newline 
\phantom{------}{\bf endif}\newline 
\phantom{---}{\bf endif\newline 
endif
skip

QUICKSORT$\left(a_1,a_2,\dots,a_n\right)$}:\newline 
{\bf V\'ystup}$\left(\text{Quick}\left(a_1,a_2,\dots,a_n\right)\right)$
skip

\flushpar Algoritmus {\bf Quick} t\v r\'\i d\'\i\ posloupnost 
$\left(a_i,a_{i+1},\dots,a_j\right)$ tak, \v ze posloupnost $\left(a_i,a_{i+1}
,a_{l-1}\right)$ obsahuje v\v sechny prvky vstupn\'\i\ posloupnosti $<a=a_k$ a  
posloupnost $\left(a_{q+1},a_{q+2},\dots,a_j\right)$ obsahuje v\v sechny prvky 
vstupn\'\i\ posloupnosti $>a=a_k$. Na tyto posloupnosti pak 
zavol\'a s\'am sebe a do v\'ysledn\'e posloupnosti ulo\v z\'\i\ 
set\v r\'\i d\v enou prvn\'\i\ posloupnost, pak prvek $a$ a nakonec 
set\v r\'\i d\v enou druhou posloupnost. Korektnost procedury 
{\bf Quick}  i algoritmu {\bf QUICKSORT} je tedy z\v rejm\'a, proto\v ze 
$l\le j$ a $i\le q$. 
\medskip

\flushpar Procedura {\bf Quick} bez rekurzivn\'\i ho vol\'an\'\i\ 
vy\v zaduje \v cas $O\left(j-i\right)$. Tedy kdyby $a_k$ byl medi\'an 
posloupnosti $\left(a_i,a_{i+1},\dots,a_j\right)$ (tj. prost\v redn\'\i\ prvek), pak 
by algoritmus {\bf QUICKSORT} vy\-\v zadoval \v cas $O\left(n\log n\right)$. Jak 
uvid\'\i me pozd\v eji, medi\'an lze nal\'ezt v line\'ar\-n\'\i m \v case, ale 
pou\v z\'\i t jakoukoliv proceduru pro jeho nalezen\'\i\ m\'a za 
d\accent23usledek, \v ze {\bf MERGESORT} a {\bf HEAPSORT }
budou rychlej\v s\'\i\ (nikoliv asymptoticky). Proto je t\v reba 
volit prvek $a_k$ (tento prvek se naz\'yv\'a \emph{pivot}) co 
nejrychleji. P\accent23uvodn\v e se bral prvn\'\i\ nebo 
posledn\'\i\ prvek. P\v ri rovnom\v ern\'em rozd\v elen\'\i\ vstupu je 
pak o\v cek\'avan\'y \v cas algoritmu $O\left(n\log n\right)$ a algoritmus je 
obvykle rychlej\v s\'\i\ ne\v z algoritmy {\bf MERGESORT} a 
{\bf HEAPSORT}. Nev\'yhoda je, \v ze pro ur\v cit\'e rozd\v elen\'\i\ dat se 
takov\'y algoritmus chov\'a \v spatn\v e (to znamen\'a, \v ze 
vy\v zaduje kvadra\-tick\'y \v cas). Proto tuto 
verzi algoritmu nen\'\i\ vhodn\'e pou\v z\'\i t pro \'ulohy, kdy 
rozd\v elen\'\i\ dat bude pro takovou volbu nev\'yhodn\'e. 
Lze to napravit tak, \v ze budeme volit $k$ n\'ahodn\v e. Bohu\v zel, 
pou\v zit\'\i\ pseudon\'ahodn\'eho gener\'atoru vy\v zaduje \v cas, a pak 
u\v z algoritmus zase nemus\'\i\ b\'yt rychlej\v s\'\i\ ne\v z algoritmy 
{\bf MERGESORT} a {\bf HEAPSORT} (a nav\'\i c n\'ahodn\v e zvolen\'y 
prvek nen\'\i\ skute\v cn\v e n\'ahodn\'y, ale to v tomto p\v r\'\i pad\v e 
nevad\'\i ). D\accent23usledkem je n\'avrh br\'at pivota jako 
medi\'an t\v r\'\i\ nebo p\v eti pevn\v e zvolen\'ych prvk\accent23u 
posloupnosti. Praxe uk\'azala, \v ze tento v\'yb\v er pivota je 
nej\-prakti\v ct\v ej\v s\'\i , d\'a se prov\'est rychle a zaji\v s\v tuje 
dostate\v cnou n\'a\-hod\-nost.
\medskip

\flushpar Proto\v ze p\v ri ka\v zd\'em vol\'an\'\i\ m\'a {\bf Quick} krat\v s\'\i\ 
vstupn\'\i\ posloupnost, lze uk\'azat, \v ze p\v ri ka\v zd\'e volb\v e 
pivota je nejhor\v s\'\i\ \v cas algoritmu {\bf QUICKSORT} $O\left(n^2
\right)$, a 
pokud je pivot vybr\'an jedno\-du\-ch\'ym a rychl\'ym 
zp\accent23usobem (to plat\'\i , i kdy\v z se vol\'\i\ n\'ahodn\v e), pak 
exi\-stuje konfigurace, kter\'a vy\v zaduje \v cas $O\left(n^2\right)$. Nyn\'\i\ 
uk\'a\-\v zeme, \v ze o\v cek\'avan\'y \v cas je $O\left(n\log n\right)$. N\'asledn\'a anal\'y\-za 
je pro n\'a\-hod\-n\v e zvolen\'eho pivota (bez dal\v s\'\i ho p\v redpokladu na 
vstupn\'\i\ data) nebo pro p\v r\'\i pad, kdy pivot je pevn\v e zvolen a data 
jsou rovno\-m\v ern\v e rozd\v e\-lena.
\medskip

\flushpar Uk\'a\v zeme dva v\'ypo\v cty o\v cek\'avan\'eho \v casu.  Jeden 
je zalo\v zen na n\v ekolika jednoduch\'ych pozorov\'an\'\i ch a druh\'y 
na rekurzivn\'\i m po\v c\'\i t\'an\'\i .  Hlavn\'\i\ idea v obou 
v\'ypo\v ctech je zalo\v zena na pozorov\'an\'\i\ 
\smallskip
\flushpar O\v cek\'avan\'y \v cas algoritmu {\bf QUICKSORT} je \newline 
$O\left(\text{\rm o\v cek\'avan\'y po\v cet porovn\'an\'\i\ v algoritmu {\bf QUICKSORT}}
\right).$
\smallskip
\flushpar Tento fakt plyne p\v r\'\i mo z popisu algoritmu.  Spo\v c\'\i t\'ame 
o\v cek\'a\-van\'y po\v cet porovn\'an\'\i\ pro algoritmus {\bf QUICKSORT}.  
\medskip

\subhead
Prvn\'\i\ v\'ypo\v cet
\endsubhead
\smallskip

\flushpar Prvky $a_i$ a $a_j$ algoritmus {\bf QUICKSORT} porovn\'a 
p\v ri t\v r\'\i d\v en\'\i\ posloupnosti $\left(a_1,a_2,\dots,a_n\right)$ nejv\'y\v se jednou.  
Kdy\v z algoritmus {\bf QUICKSORT} porovn\'av\'a prvky $a_i$ a $a_
j$, 
pak pro n\v ejak\'y b\v eh podprocedury {\bf Quick} je $a_i$ nebo $
a_j$ 
pivot.  P\v ritom v p\v redchoz\'\i ch b\v ez\'\i ch {\bf Quick} $
a_i$ ani $a_j$ 
nebylo pivotem, proto\v ze pivot se v\v zdy vy\v rad\'\i\ z n\'asleduj\'\i c\'\i ch 
vol\'an\'\i\ t\'eto podprocedury.
\medskip

\flushpar Nech\v t $\left(b_1,b_2,\dots,b_n\right)$ je v\'ysledn\'a posloupnost.  
Ozna\v cme $X_{i,j}$ boolskou prom\v enou, kter\'a m\'a hodnotu $
1$, 
kdy\v z {\bf QUICKSORT} provedl porovn\'an\'\i\ mezi prvky $b_i$ a $
b_j$, 
jinak m\'a hodnotu $0$.  P\v redpokl\'adejme, \v ze je to n\'ahodn\'a 
veli\v cina. Kdy\v z $p_{i,j}$ je prav\-d\v e\-podobnost, \v ze $
X_{i,j}=1$, 
pak o\v cek\'avan\'a hodnota $X_{i,j}$ je 
$$\bold E\left(X_{i,j}\right)=0\left(1-p_{i,j}\right)+1p_{i,j}=p_{i,j}.$$
Proto\v ze po\v cet porovn\'an\'\i\ p\v ri b\v ehu algoritmu {\bf QUICKSORT} je
$$\sum_{i=1}^n\sum_{j=i+1}^nX_{i,j}$$
a proto\v ze o\v cek\'avan\'a hodnota sou\v ctu n\'ahodn\'ych 
prom\v enn\'ych je sou\v cet o\v cek\'avan\'ych hodnot, dost\'av\'ame, \v ze 
o\v cek\'avan\'y po\v cet porovn\'an\'\i\ v algoritmu {\bf QUICKSORT} je
$$\sum_{i=1}^n\sum_{j=i+1}^n\bold E\left(X_{i,j}\right)=\sum_{i=1}^n\sum_{j=
i+1}^np_{i,j}.$$
\medskip

\flushpar Abychom spo\v c\'\i tali $p_{i,j}$, pop\'\i\v seme chov\'an\'\i\ algoritmu 
{\bf QUICKSORT} pomoc\'\i\ modifikace stromu v\'ypo\v ctu. Bude to bin\'arn\'\i\ 
strom, v n\v em\v z ka\v zd\'y vrchol odpov\'\i d\'a jednomu 
b\v ehu podprocedury {\bf Quick} a vrchol $v$ bude vnit\v rn\'\i\ vrchol, 
kdy\v z odpov\'\i da\-j\'\i\-c\'\i\ podprocedura volila pivota, kter\'y  
ohodnot\'\i\ vrchol $v$. V podstromu lev\'eho syna vrcholu $v$
budou pr\'av\v e v\v sechna n\'asleduj\'\i c\'\i\ rekurzivn\'\i\ vol\'an\'\i\ podprocedury 
{\bf Quick} nad posloupnost\'\i , kter\'a p\v redch\'az\'\i\ pi\-vo\-tu. 
Analogicky v podstromu 
prav\'eho syna vrcholu $v$ budou pr\'av\v e v\v sechna n\'asleduj\'\i c\'\i\ 
rekurzivn\'\i\ vol\'an\'\i\ procedury {\bf Quick} nad posloupnost\'\i , kter\'a n\'asleduje po pivotu. 
Listy stromu jsou ozna\v ceny prvky, kter\'e jsou v 
posloupnosti, s n\'\i\v z je vol\'ano odpov\'\i daj\'\i c\'\i\ {\bf Quick}.
Kdy\v z vrchol $v$ odpov\'\i d\'a vol\'an\'\i\ {\bf Quick} s posloupnost\'\i\ 
$\left(a_i,a_{i+1},\dots,a_j\right)$, pak vrcholy v podstromu lev\'eho syna $
v$ jsou 
ohodnoceny prvky z posloupnosti $\left(a_i,a_{i+1},\dots,a_{l-1}\right)$ a vrcholy v 
podstromu prav\'eho syna vrcholu $v$ jsou ohodnoceny prvky z posloupnosti 
$\left(a_{q+1},\dots,a_j\right)$ (po p\v rerovn\'an\'\i\ posloupnosti).
D\'ale plat\'\i\ $\left\{a_l\mid i\le l\le j\right\}=\left\{b_l\mid i\le l\le j\right\}$.
\medskip

\flushpar O\v c\'\i slujeme vrcholy tohoto stromu prohled\'av\'an\'\i m do 
\v s\'\i\v rky, za p\v redpokladu, \v ze lev\'y syn vrcholu p\v redch\'az\'\i\ 
prav\'emu synu. Nech\v t $\left(c_1,c_2,\dots,c_n\right)$ je posloupnost 
prvk\accent23u $\left\{a_i\mid 1\le i\le n\right\}$ v po\v rad\'\i\ dan\'em t\'\i mto 
o\v c\'\i slov\'an\'\i m. Pak plat\'\i , \v ze $X_{i,j}=1$, pr\'av\v e kdy\v z prvn\'\i\ 
prvek v posloupnosti $\left(c_1,c_2,\dots,c_n\right)$ z mno\v zi\-ny 
$\left\{b_l\mid i\le l\le j\right\}$ je bu\v d $b_i$ nebo $b_j$. Pravd\v epodobnost tohoto 
jevu je $\frac 2{j-i+1}$, tedy $p_{i,j}=\frac 2{j-i+1}$ pro $1\le 
i<j\le n$. Odtud 
o\v cek\'avan\'y po\v cet porovn\'an\'\i\ algoritmu {\bf QUICKSORT} je
$$\align\sum_{i=1}^n\sum_{j=i+1}^np_{i,j}=&\sum_{i=1}^n\sum_{j=i+
1}^n\frac 2{j-i+1}=\sum_{i=1}^n\sum_{k=2}^{n-i+1}\frac 2k\le\\
&2\sum_{i=1}^n\sum_{k=2}^n\frac 1k=2n\left(\sum_{k=2}^n\frac 1k\right)\le 2n
\int_1^n\frac 1xdx=\\
&2n\ln n.\endalign$$
\medskip

\subhead
Druh\'y v\'ypo\v cet
\endsubhead
\smallskip

Ozna\v cme $QS\left(n\right)$ o\v cek\'avan\'y po\v cet 
porovn\'an\'\i\ proveden\'y algoritmem {\bf QUICKSORT} p\v ri 
t\v r\'\i d\v en\'\i\ $n$-\v clenn\'e po\-sloupnosti. Pak plat\'\i
$$\gather QS\left(0\right)=QS\left(1\right)=0\text{\rm\ a }\\
{\align QS\left(n\right)=&\frac 1n\left(\sum_{k=0}^{n-1}n-1+QS\left(k\right)+QS\left(n-k-1\right)
\right)=\\
&n-1+\frac 2n\left(\sum_{k=0}^{n-1}QS\left(k\right)\right).\endalign}\endgather$$
\medskip

\flushpar Z toho dost\'av\'ame, \v ze 
$$\gather nQS\left(n\right)=n\left(n-1\right)+2\sum_{k=0}^{n-1}QS\left(k\right)\text{\rm\ a }\\
\left(n+1\right)QS\left(n+1\right)=\left(n+1\right)n+2\sum_{k=0}^nQS\left(k\right)\endgather$$
a tedy
$$QS\left(n+1\right)=\frac {2n}{n+1}+\frac {n+2}{n+1}QS\left(n\right).$$
\medskip

\flushpar Proto\v ze $\frac {i-1}i\le 1$ pro ka\v zd\'e $i\ge 1$, dostaneme, \v ze 
$$\align QS\left(n\right)=&\sum_{i=2}^n\frac {n+1}{i+1}\frac {2\left(i-1\right)}i\le 2\left(
n+1\right)\left(\sum_{i=2}^n\frac 1{i+1}\right)=\\
&2\left(n+1\right)\left(\sum_{i=3}^{n+1}\frac 1i\right)\le 2\left(n+1\right)\left(\left(\int_{i=
1}^{n+1}\frac 1xdx\right)-\frac 12\right)=\\
&2n\ln\left(n+1\right)+2\ln\left(n+1\right)-n-1.\endalign$$
Pro dostate\v cn\v e velk\'a $n$ plat\'\i\ 
$$2n\ln\left(n+1\right)+2\ln\left(n+1\right)-n\le 2n\ln n.$$
\medskip

\subhead
Porovn\'an\'\i\ algoritm\accent23u
\endsubhead
\smallskip

\flushpar Nyn\'\i\ porovn\'ame slo\v zitost algoritm\accent23u 
{\bf HEAPSORT}, {\bf MER\-GE\-SORT}, {\bf QUICKSORT}, {\bf A-sort}, 
{\bf SELECTIONSORT}, {\bf INSERTIONSORT}.  P\v ripome\v nme, \v ze 
{\bf SELECTION\-SORT} t\v r\'\i\-d\'\i\ posloupnost tak, \v ze jedn\'\i m
pr\accent23uchodem nalezne jej\'\i\ nej\-men\v s\'\i\ prvek, 
kter\'y vy\v rad\'\i\ a vlo\v z\'\i\ do v\'ysledn\'e 
posloupnosti. Tento proces pak opakuje se zbytkem 
p\accent23uvodn\'\i\ posloupnosti (tato idea byla 
z\'akla\-dem algoritmu {\bf HEAPSORT}).  {\bf INSERTIONSORT} t\v r\'\i d\'\i\ 
tak, \v ze do ji\v z set\v r\'\i d\v en\'e 
\v c\'asti posloupnosti vkl\'ad\'a dal\v s\'\i\ prvek, kter\'y pomoc\'\i\ v\'ym\v en 
za\v rad\'\i\ na spr\'avn\'e m\'\i sto, a tento proces opakuje.  
\medskip

\flushpar {\bf QUICKSORT} v nejhor\v s\'\i m p\v r\'\i pad\v e vy\v zaduje \v cas 
$\Theta \left(n^2\right)$, o\v ce\-k\'avan\'y \v cas je $9n\log n$, v 
nejhor\v s\'\i m p\v r\'\i pad\v e vy\v zaduje $\frac {n^2}2$ porov\-n\'an\'\i ,  
o\v cek\'avan\'y po\v cet porov\-n\'an\'\i\ je $1.44n\log n$, 
pou\v z\'\i v\'a $n+\log n+\text{\rm konstanta}$ pam\v eti. Pou\v z\'\i v\'a p\v r\'\i m\'y 
p\v r\'\i stup k pam\v eti a nen\'\i\ adaptivn\'\i\ na p\v redt\v r\'\i d\v e\-n\'e 
posloupnosti.\newline 
{\bf HEAPSORT} v nejhor\v s\'\i m p\v r\'\i pad\v e vy\v zaduje \v cas $
20n\log n$, 
o\v ce\-k\'avan\'y \v cas je $\le 20n\log n$, v nejhor\v s\'\i m i v 
o\v cek\'avan\'em p\v r\'\i pad\v e vy\v zaduje $2n\log n$ porovn\'an\'\i , pou\v z\'\i v\'a 
$n+\text{\rm konstanta}$ pam\v eti. Pou\v z\'\i v\'a p\v r\'\i m\'y p\v r\'\i stup k pam\v eti a 
nen\'\i\ adaptivn\'\i\ na p\v redt\v r\'\i d\v e\-n\'e posloupnosti.\newline 
{\bf MERGESORT} v nejhor\v s\'\i m p\v r\'\i pad\v e vy\v zaduje \v cas 
$12n\log n$, o\v cek\'avan\'y \v cas je $\le 12n\log n$, v nejhor\v s\'\i m 
i v o\v cek\'ava\-n\'em p\v r\'\i pad\v e vy\v zaduje $n\log n$ 
porovn\'an\'\i , pou\v z\'\i v\'a $2n+\text{\rm konstanta}$ pam\v eti. Pou\v z\'\i v\'a 
sekven\v cn\'\i\ p\v r\'\i stup k pam\v eti a verze, kterou jsme uvedli, 
je adaptivn\'\i\ na p\v redt\v r\'\i d\v e\-n\'e posloupnosti, kter\'e se 
skl\'adaj\'\i\ z mal\'eho po\v ctu dlouh\'ych set\v r\'\i d\v en\'ych 
\'usek\accent23u.\newline 
{\bf A-sort} vy\v zaduje \v cas $O\left(n\log\frac Fn\right)$ v nejhor\v s\'\i m 
i v o\v cek\'avan\'em p\v r\'\i pad\v e, kde $F$ je po\v cet inverz\'\i\ ve vstupn\'\i\ 
posloupnosti, stejn\v e tak po\v cet porovn\'an\'\i\ je $O\left(n\log\frac Fn\right)$ 
v nejhor\v s\'\i m i v o\v cek\'avan\'em p\v r\'\i pad\v e, pou\v z\'\i v\'a 
$5n+\text{\rm konstanta}$ pam\v e\-ti. Pou\v z\'\i v\'a p\v r\'\i m\'y p\v r\'\i stup k pa\-m\v e\-ti 
a je adaptivn\'\i\ na p\v redt\v r\'\i d\v en\'e posloupnosti s mal\'ym 
po\v ctem inverz\'\i .\newline
{\bf SELECTIONSORT} v nejhor\v s\'\i m i v o\v cek\'avan\'em p\v r\'\i pad\v e 
po\-t\v rebuje \v cas $2n^2$, po\v cet porov\-n\'an\'\i\ v nejhor\v s\'\i m 
i v o\v cek\'avan\'em p\v r\'\i pad\v e je $\frac {n^2}2$, pou\v z\'\i v\'a 
$n+\text{\rm konstanta}$ pam\v eti. Pou\v z\'\i v\'a p\v r\'\i m\'y p\v r\'\i stup k pam\v eti a 
nen\'\i\ adaptivn\'\i\ na p\v redt\v r\'\i d\v en\'e posloupnosti.\newline 
{\bf INSERTIONSORT} v nejhor\v s\'\i m i v o\v cek\'avan\'em p\v r\'\i pad\v e  
vy\-\v zaduje \v cas $O\left(n^2\right)$, po\v cet porovn\'an\'\i\ v nejhor\v s\'\i m 
p\v r\'\i pad\v e je $\frac {n^2}2$, v o\v ce\-k\'a\-van\'em p\v r\'\i pad\v e 
$\frac {n^2}4$, pou\v z\'\i\-v\'a $n+\text{\rm konstanta}$ pam\v eti. Pou\v z\'\i\-v\'a 
sekven\v cn\'\i\ p\v r\'\i stup k pam\v eti a m\'a verzi, kter\'a je 
adaptivn\'\i\ na p\v redt\v r\'\i d\v en\'e posloupnosti s mal\'ym po\v ctem 
inverz\'\i .
\medskip

\flushpar\v Cas prezentovan\'y ve v\'ysledc\'\i ch byl spo\v c\'\i t\'an 
pro model RAM (viz Mehlhorn).
\medskip

\flushpar O\v cek\'avan\'y \v cas pro {\bf HEAPSORT} je prakticky stejn\'y jako 
nej\-hor\v s\'\i\ \v cas.  Byly navr\v zeny verze, kter\'e optimalizuj\'\i\ 
po\v cet porov\-n\'an\'\i , ale v\v et\v sinou maj\'\i\ v\v et\v s\'\i\ n\'aroky na 
\v cas, a proto a\v z na v\'yjimky ne\-jsou vhodn\'e.  
Situace pro {\bf MERGESORT} je komplikovan\v ej\v s\'\i , hod\-n\v e z\'avis\'\i\ 
na konkr\'etn\'\i\ verzi algoritmu.  Algoritmus 
{\bf MER\-GE\-SORT} je nejvhodn\v ej\v s\'\i\ pro extern\'\i\ pam\v eti, proto\v ze 
pou\v z\'\i v\'a sekven\v cn\'\i\ p\v r\'\i stup, pro intern\'\i\ pam\v e\v t 
kv\accent23uli velk\'e prostorov\'e n\'aro\v cnosti 
nen\'\i\ doporu\v cov\'an.  Tak\'e se hod\'\i\ pro 
n\'avrh pa\-ra\-leln\'\i ch algoritm\accent23u. Pro t\v r\'\i d\v en\'\i\ 
kr\'atk\'ych posloupnost\'\i\ je doporu\v ceno m\'\i sto {\bf QUICKSORTU }
pro posloupnosti d\'elky $\le 22$ pou\v z\'\i t {\bf SELECTIONSORT} a pro 
posloupnosti d\'elky $\,\le 15$ pou\v z\'\i t {\bf INSERTIONSORT.}
To vede k n\'avrhu algoritmu, kter\'y 
pro dlouh\'e posloupnosti pracuje jako {\bf QUICKSORT,} a kdy\v z 
vol\'a rekurzivn\v e s\'am sebe na kr\'atkou posloupnost, pak 
pou\v zije {\bf SELECTIONSORT} ne\-bo {\bf INSERTIONSORT}. V algoritmu {\bf A-sort }
se doporu\-\v cu\-je pou\v z\'\i t $\left(2,3\right)$-strom.  
Pom\v er \v cas\accent23u v klasick\'ych po\v c\'\i ta\v c\'\i ch vy\v zadovan\'ych algoritmy {\bf QUICKSORT}, 
{\bf MERGESORT} a {\bf HEAPSORT} je 1 : 1.33 : 2.22 (viz Mehlhorn). To v\v sak nemus\'\i\ 
b\'yt pravda pro RISK-architekturu ani pro cache-pam\v eti 
apod. 
\medskip

\flushpar V algoritmu {\bf MERGESORT} jsme pou\v zili frontu, 
kter\'a \v r\'\i dila slu\v cov\'an\'\i\ posloupnost\'\i . Tato metoda je 
uspokojuj\'\i c\'\i\ a d\'av\'a optim\'aln\'\i\ v\'ysledek, pokud 
posloupnosti ve front\v e jsou stejn\v e dlouh\'e. Pokud se 
jejich d\'elky hodn\v e li\v s\'\i , nedos\'ahneme 
optim\'aln\'\i ho v\'ysledku. Budeme \v re\v sit n\'asleduj\'\i c\'\i\ 
probl\'em, kter\'y se poprvn\'e vyskytl p\v ri n\'avrhu Huffmanova k\'odu. 
skip

\flushpar Vstup: Mno\v zina rostouc\'\i ch navz\'ajem 
disjunktn\'\i ch posloupnost\'\i .\newline 
\'Ukol:  Pomoc\'\i\ operace {\bf MERGE} co nejrychleji spojit 
v\v sechny tyto posloupnosti do jedin\'e rostouc\'\i\ posloupnosti.  
\medskip

\flushpar P\v redpokl\'adejme, \v ze m\'ame postup, kter\'y vytvo\v r\'\i\ 
jedinou po\-sloupnost. Tento postup ur\v cuje \'upln\'y 
bin\'arn\'\i\ strom $T$ (tj. strom, kde ka\v zd\'y vnit\v rn\'\i\ 
vrchol m\'a dva syny) tako\-v\'y, \v ze vstupn\'\i\ posloupnosti 
ohodnocuj\'\i\ listy a ka\v zd\'a posloupnost vznik\-l\'a slu\v cov\'an\'\i m 
ohodnocuje n\v ekter\'y vnit\v rn\'\i\ vrchol tak, \v ze plat\'\i :
\roster
\item"{}"
kdy\v z $P\left(v\right)$ je posloupnost ohodnocuj\'\i c\'\i\ vrchol $v$ 
a $v_1$ a $v_2$ jsou synov\'e $v$, pak 
$P\left(v\right)=${\bf MERGE$\left(P\left(v_1\right),P\left(v_2\right)\right)$}.
\endroster
Pro posloupnost $P$ ozna\v cme $l\left(P\right)$ jej\'\i\ d\'elku. Pak sou\v cet 
\v cas\accent23u, kter\'e v tomto procesu vy\v zaduje 
podprocedura {\bf MERGE},
je $O\left(\sum \left\{l\left(P\left(v\right)\right)\mid v\text{\rm \ je vnit\v rn\'\i\ vrchol stromu }
T\right\}\right)$. Indukc\'\i\ 
lehce dostaneme, \v ze 
$$\sum \left\{l\left(P\left(v\right)\right)\mid v\text{\rm\ vnit\v rn\'\i\ vrchol stromu }T\right\}
=\sum_{t\text{\rm\ list }T}d\left(t\right)l\left(P\left(t\right)\right),$$
kde $d\left(t\right)$ je hloubka listu $t$.
\medskip

\flushpar Kdy\v z $T$ je \'upln\'y bin\'arn\'\i\ strom takov\'y, \v ze listy
jsou ohodnoceny rostouc\'\i mi navz\'ajem disjunktn\'\i mi 
posloupnostmi, pak algoritmus {\bf Slevani} spoj\'\i\ tyto  
posloupnosti do jedin\'e rostouc\'\i\ posloupnosti a procedury 
{\bf MERGE}  vy\v zaduj\'\i\ \v cas 
$$O\left(\sum_{t\text{\rm\ list }T}d\left(t\right)l\left(P\left(t\right)\right)\right).$$
skip

{\bf Slevani$\left(T,\left\{P\left(l\right)\mid\;l\text{\rm \ {list }}T\right\}\right)$\newline }
{\bf while} $P\left(\text{\rm ko\v ren }T\right)$ nen\'\i\ definov\'ano {\bf do}\newline 
vezm\v eme vrchol $v$ takov\'y, \v ze $P\left(v\right)$ nen\'\i\ definov\'ano 
a pro oba syny $v_1$ a $v_2$ vrcholu $v$ jsou $P\left(v_1\right)$ a $P\left(v_2\right)$ 
definov\'any, polo\v zme $P\left(v\right):=${\bf MERGE$\left(P\left(v_1\right),P\left(v_2\right)\right)$\newline 
enddo
skip

}\flushpar Nyn\'\i\ m\accent23u\v zeme p\v reformulovat p\accent23uvodn\'\i\ 
probl\'em:\newline 
Vstup: $n$ \v c\'\i sel $x_1,x_2,\dots,x_n$\newline 
V\'ystup: \'upln\'y bin\'arn\'\i\ strom $T$ s $n$ listy a bijekce $
\phi$ z 
mno\v ziny $\left\{1,2,\dots,n\right\}$ do list\accent23u $T$ takov\'a, \v ze 
$\sum_{i=1}^nd\left(\phi \left(i\right)\right)x_i$ je minim\'aln\'\i , kde $d\left(\phi \left(i\right)\right)$ je hloubka 
listu $\phi \left(i\right)$. 
\medskip

\flushpar\v Rekneme, \v ze dvojice $\left(T,\phi \right)$ je \emph{optim\'aln\'\i} 
\emph{strom} vzhledem k $x_1,x_2,\dots,x_n$, kdy\v z  
sou\v cet $\sum_{i=1}^nd\left(\phi \left(i\right)\right)x_i$ je nejmen\v s\'\i\ mo\v zn\'y. 
\medskip

\flushpar V p\v reformulovan\'e \'uloze u\v z nepracujeme s 
posloupnostmi, ale jen s jejich d\'elkami. 
To znamen\'a, kdy\v z pro p\accent23uvodn\'\i\ \'ulohu 
byly vstupem posloupnosti $P_1,P_2,\dots,P_n$, pak pro 
p\v reformulovanou \'ulohu jsou vstupem jen jejich d\'elky  
$l\left(P_1\right),l\left(P_2\right),\dots,l\left(P_n\right)$. Vy\-tvo\v ren\'y strom je pak pou\-\v zit v algoritmu {\bf Slevani }
(kde posloupnost $P_i$ nahrad\'\i\ p\v ri ohodnocen\'\i\ svoji d\'elku).
\medskip

\flushpar Pro \'upln\'y bin\'arn\'\i\ strom $T$ s $n$ listy a bijekci $
\phi$ 
z mno\v ziny $\left\{1,2,\dots,n\right\}$ do list\accent23u stromu $T$ 
definujme 
$$\Cont\left(T,\phi \right)=\sum_{i=1}^nd\left(\phi \left(i\right)\right)x_i,$$
kde $d\left(\phi \left(i\right)\right)$ je hloubka listu $\phi \left(i\right)$, tj.  d\'elka cesty z 
ko\v rene do listu $\phi \left(i\right)$ pro $i=1,2,\dots,n$.  
Chceme zkonstruovat \'upln\'y bin\'arn\'\i\ strom s $n$ listy, 
kter\'y minimalizuje hodnotu $\Cont$. Zavedeme proto pojem 
\emph{ko\v renov\'y} \emph{les} -- to je disjunktn\'\i\ sjednocen\'\i\ 
ko\v re\-no\-v\'ych strom\accent23u. Vrchol se naz\'yv\'a \emph{list} 
\emph{lesa}, kdy\v z je listem n\v ekter\'eho stromu, jejich\v z 
sjednocen\'\i m les vznikl. \emph{Velikost} \emph{lesa} $V$, zna\v c\'\i me 
ji $|V|$, je po\v cet strom\accent23u, jejich\v z sjednocen\'\i m $
V$ vznikl.
Uva\v zujme n\'asleduj\'\i c\'\i\ algoritmus.  
skip

{\bf Optim$\left(x_1,x_2,\dots x_n\right)$}:\newline 
$V$ je ko\v renov\'y les vznikl\'y disjunktn\'\i m sjednocen\'\i m $
n$ 
jednoprvkov\'ych strom\accent23u,\newline 
$\phi$ je bijekce mezi $\left\{1,2,\dots,n\right\}$ a listy lesa $V$\newline 
{\bf for every} $v\in V$ {\bf do} $c\left(v\right):=x_{\phi^{-1}\left(v\right)}$ {\bf enddo\newline 
while} $|V|>1$ {\bf do}\newline 
\phantom{---}vezmeme z $V$ dva stromy $T_1$ a $T_2$ s nejmen\v s\'\i m 
ohodnocen\'\i m, odstran\'\i me je z $V$,\newline 
\phantom{---}vytvo\v r\'\i me strom $T$ jako disjunktn\'\i\ sjednocen\'\i\  
$T_1$, $T_2$ a nov\'eho vrcholu $v$, \newline 
\phantom{---}$v$ je ko\v ren $T$ a jeho dva synov\'e jsou ko\v reny strom\accent23u $
T_1$ a 
$T_2$, \newline 
\phantom{---}$c\left(T\right)=c\left(T_1\right)+c\left(T_2\right)$, $T$ vlo\v z\'\i me do $V$\newline 
{\bf enddo\newline 
V\'ystup}: $\left(V,\phi \right)$.
skip

\proclaim{V\v eta}Pro danou posloupnost \v c\'\i sel 
$\left(x_1,x_2,\dots,x_n\right)$ algoritmus {\bf Optim} nalezne optim\'aln\'\i\ strom 
pro mno\v zinu $x_1,x_2,\dots,x_n$ a pokud je posloupnost 
$\left(x_1,x_2,\dots,x_n\right)$ neklesaj\'\i c\'\i , pak vy\v zaduje \v cas 
$O\left(n\right)$.  
\endproclaim


\demo{D\accent23ukaz}Nejprve si v\v simn\v eme, \v ze v ka\v zd\'em 
okam\v ziku je $\phi \left(i\right)$ list lesa $V$ pro ka\v zd\'e $i\in \left\{1,
2,\dots,n\right\}$.  
Algoritmus kon\v c\'\i , kdy\v z $|V|=1$, tedy, kdy\v z $V$ je strom.  Kdy\v z v 
dan\'em okam\v ziku $V$ vznikl disjunktn\'\i m sjednocen\'\i m 
strom\accent23u $T_1,T_2,\dots,T_k$ a na stromy $T_1$ a $T_2$ pou\v zijeme 
popsan\'y proces, pak dostaneme les $V'$ vznikl\'y ze 
strom\accent23u $T,T_3,\dots,T_k$, tedy $|V'|+1=|V|$.  Proto\v ze na 
za\v c\'atku $V$ obsahoval jen $n$ jednoprvko\-v\'ych strom\accent23u, 
tak algoritmus po kone\v cn\'em po\v ctu krok\accent23u skon\v c\'\i .  Uk\'azali jsme, \v ze ka\v zd\'y b\v eh cyklu {\bf while 
do} zmen\v s\'\i\ po\v cet strom\accent23u o jeden, ale nezm\v en\'\i\ mno\v zinu 
list\accent23u.  Proto v\'ysledn\'y les $V$ je strom s $n$ listy a $
\phi$ je 
bijekce z $\left\{1,2,\dots,n\right\}$ do mno\v ziny list\accent23u $V$.  Zb\'yv\'a 
uk\'azat, $\left(V,\phi \right)$ je optim\'aln\'\i\ strom vzhledem k 
$x_1,x_2,\dots,x_n$. Dok\'a\v zeme to indukc\'\i\ podle $n$.  Kdy\v z 
$n=2$, tak tvrzen\'\i\ z\v rejm\v e plat\'\i .  P\v redpokl\'adejme, \v ze tvrzen\'\i\ 
plat\'\i\ pro ka\v zdou posloupnost \v c\'\i sel $\left(y_1,y_2,\dots,
y_{n-1}\right)$ a nech\v t 
$x_1\le x_2\le\dots\le x_n$ je posloupnost \v c\'\i sel.  Bez \'ujmy na obecnosti 
m\accent23u\v zeme p\v redpokl\'adat, \v ze v prvn\'\i m kroku algoritmus 
{\bf Optim} zvolil stro\-my $\phi \left(1\right)$ a $\phi \left(2\right)$.  Uva\v zujme mno\v zinu 
$\left(y_1,y_2,\dots,y_{n-1}\right)$, kde $y_i=x_{i+2}$ pro $i=1,2,\dots,n-2$, 
$y_{n-1}=x_1+x_2$. Nech\v t $V'$ je strom z\'\i skan\'y ze stromu $
V$  
odstran\v en\'\i m list\accent23u $\phi \left(1\right)$ a $\phi \left(2\right)$ a nech\v t $
\psi$ je bijekce z 
mno\v ziny $\left\{1,2,\dots,n-1\right\}$ takov\'a, \v ze $\psi \left(i\right)=\phi \left(i+
2\right)$ pro 
$i=1,2,\dots,n-2$ a $\psi \left(n-1\right)=$otec listu $\phi \left(1\right)$.  Pak 
m\accent23u\v zeme p\v redpokl\'adat, \v ze algoritmus 
{\bf Optim$\left(y_1,y_2,\dots,y_{n-1}\right)$} zkonstruoval $\left(T',\psi \right)$. Podle induk\v c\-n\'\i ho 
p\v redpokladu je to optim\'aln\'\i\ strom pro $\left(y_1,y_2,\dots,y_{
n-1}\right)$.  Nech\v t 
$\left(U,\theta \right)$ je optim\'aln\'\i\ strom vzhledem k $\left(x_1,x_2,\dots
,x_n\right)$.  Zvolme 
vnit\v rn\'\i\ vrchol $u$ ve strom\v e $U$ s nejv\v et\v s\'\i\ hloubkou.  Pak synov\'e 
$u_1$ a $u_2$ vrcholu $u$ jsou listy. Nech\v t $i,j\in \left\{1,2,\dots
,n\right\}$ takov\'e, \v ze 
$\theta \left(i\right)=u_1$, $\theta \left(j\right)=u_2$.  M\accent23u\v zeme p\v redpokl\'adat, \v ze kdy\v z 
$i,j\in \left\{1,2\right\}$, pak $i=1$ a $j=2$.  Definujme $\eta$ z 
$\left\{1,2,\dots,n\right\}$ do list\accent23u $U$ tak, \v ze $\eta \left(1\right)=u_1$, $
\eta \left(2\right)=u_2$, 
$\eta \left(i\right)=\theta \left(1\right)$, $\eta \left(j\right)=\theta \left(2\right)$ a $\eta \left(k\right)=\theta \left(
k\right)$ pro v\v sechna 
$k\in \left\{3,4,\dots,n\right\}\setminus \left\{i,j\right\}$.  Pak $\eta$ je bijekce a 
$$\multline \Cont\left(U,\eta \right)-\Cont\left(U,\theta \right)=\\
\left(d\left(u_1\right)-d\left(\theta \left(1\right)\right)\right)\left(x_1-x_i\right)+\left(d\left(u_2\right)-d\left(\theta \left(2\right)\right)\right)\left(x_2-x_j\right).\endmultline$$
Z volby $u$ plyne, \v ze $d\left(u_1\right)\ge d\left(\theta \left(1\right)\right)$, $d\left(
u_2\right)\ge d\left(\theta \left(2\right)\right)$, 
$x_1\le x_i$ a $x_2\le x_j$. Odtud  
$$\left(d\left(u_1\right)-d\left(\theta \left(1\right)\right)\right)\left(x_1-x_i\right)+\left(d\left(u_2\right)-d\left(\theta \left(2\right)\right)\right)\left(x_2-x_j\right)\le 
0$$
a proto\v ze $\left(U,\theta \right)$ je optim\'aln\'\i\ strom pro $\left(x_1,x_2
,\dots,x_n\right)$, 
dost\'a\-v\'a\-me, \v ze $\left(U,\eta \right)$ je tak\'e optim\'aln\'\i\ strom pro 
$\left(x_1,x_2,\dots,x_n\right)$.  Od\-stran\v en\'\i m list\accent23u $u_1$ a $
u_2$ ze 
stromu $U$ dostaneme strom $U'$.  Definujme $\tau$ z 
$\left\{1,2,\dots,n-1\right\}$ p\v redpisem $\tau \left(i\right)=\eta \left(i+2\right)$ pro 
$i=1,2,\dots,n-2$ a $\tau \left(n-1\right)=u$.  Pak $\tau$ je bijekce z 
$\left\{1,2,\dots,n-1\right\}$ do mno\v ziny list\accent23u $U'$ a proto\v ze 
$\left(T',\psi \right)$ je optim\'aln\'\i\ strom pro $\left(y_1,y_2,\dots,y_{n-1}
\right)$, 
dost\'av\'ame, \v ze 
$$\Cont\left(T',\psi \right)\le\Cont\left(U',\tau \right).$$
Proto\v ze plat\'\i\ 
$$\gather\Cont\left(T,\phi \right)=\Cont\left(T,\psi \right)+x_1+x_2\text{\rm\ a }\\
\Cont\left(U,\eta \right)=\Cont\left(U',\tau \right)+x_1+x_2\endgather$$
dost\'av\'ame, \v ze $\left(T,\phi \right)$ je optim\'aln\'\i\ strom pro 
$\left(x_1,x_2,\dots,x_n\right)$.
P\v redpokl\'adejme, \v ze $x_1\le x_2\le\dots\le x_n$ a \v ze algoritmus 
po\-stupn\v e vytv\'a\v r\'\i\ v\'\i ceprvkov\'e stromy $T_1,T_2,\dots,T_k$. Pak indukc\'\i\ okam\v zit\v e 
dost\'av\'ame, \v ze $c\left(T_1\right)\le c\left(T_2\right)\le\dots\le c\left(T_k\right)$.  Tedy sta\v c\'\i, kdy\v z 
pou\v zijeme n\'asleduj\'\i c\'\i\ strukturu:  rostouc\'\i\ posloupnost 
prvk\accent23u $x_1,x_2,\dots,x_n$, ukazatel, 
kter\'y v ka\v zd\'em okam\v ziku ukazuje na nejmen\v s\'\i\ prvek 
t\'eto posloupnosti, 
kter\'y je reprezentov\'an listem v jednoprvkov\'em 
strom\v e (pak p\v red ukazatelem jsou prvky, kter\'e jsou 
reprezentov\'any listem ve v\'\i\-ce\-prvkov\'em strom\v e a za 
ukazatelem jsou prvky reprezentovan\'e listem v jednoprvkov\'em 
stro\-m\v e) a frontu v\'\i\-ce\-prvkov\'ych strom\accent23u (to 
znamen\'a, \v ze stromy odeb\'\i r\'ame zep\v redu a ukl\'ad\'ame je dozadu).  
Udr\v zovat tyto struktury vy\v za\-duje \v cas $O\left(1\right)$ stejn\v e jako 
nalezen\'\i\ dvou strom\accent23u s nej\-men\v s\'\i m ohodnocen\'\i m.  
Tedy algoritmus {\bf Optim} zkonstruuje optim\'aln\'\i\ strom v \v case $
O\left(n\right)$. \qed
\enddemo

\flushpar P\v ri aplikaci na na\v si p\accent23uvodn\'\i\ \'ulohu mus\'\i me 
je\v st\v e set\v r\'\i dit vstupn\'\i\ posloupnost (d\'elek) pro p\v reformu\-lo\-vanou \'ulohu.
Tato posloupnost je tvo\v rena p\v rirozen\'ymi \v c\'\i sly a d\'elku 
maxim\'aln\'\i\ posloupnosti nalezneme v \v case \'um\v ern\'em sou\v ctu 
d\'elek posloupnost\'\i . Na jej\'\i\ set\v r\'\i d\v en\'\i\ pak 
m\accent23u\v zeme pou\v z\'\i t algoritmus {\bf BUCKETSORT} (pop\'\i \v seme si ho v 
n\'asleduj\'\i c\'\i\ p\v redn\'a\v sce), kter\'y 
vy\v zaduje \v cas $O\left(n+m\right)$, kde $n$ je po\v cet posloupnost\'\i\ a $
m$ je 
maxim\'aln\'\i\ d\'elka posloupnosti.
\medskip

\proclaim{V\v eta}Uveden\'y algoritmus spoj\'\i\ 
disjunktn\'\i\ rostou\-c\'\i\ 
po\-sloup\-nosti $P_1,P_2,\dots,P_n$ o d\'elk\'ach $l\left(P_1\right),l\left(P_2\right)
,\dots,l\left(P_n\right)$ 
do jedin\'e rostouc\'\i\ posloupnosti v \v case 
$O\left(\sum_{i=1}^nl\left(P_i\right)\right)$.
\endproclaim

\heading
Rozhodovac\'\i\ stromy
\endheading


\flushpar V\v et\v sina obecn\'ych t\v r\'\i dic\'\i ch algoritm\accent23u 
pou\v z\'\i v\'a jedinou primitivn\'\i\ operaci mezi prvky vstupn\'\i\ 
posloupnosti, a to jejich vz\'a\-jemn\'e porov\-n\'an\'\i , jin\'e 
operace nejsou pou\v z\'\i v\'any. To znamen\'a, \v ze pr\'aci 
takov\'eho algoritmu pro $n$-prvkov\'e posloupnosti lze po\-psat 
bin\'arn\'\i m stro\-mem, jeho\v z vnit\v rn\'\i\ vrcholy jsou ohodnoceny 
porovn\'an\'\i m dvou prvk\accent23u vstupn\'\i\ posloupnosti 
(nap\v r. $a_i<a_j$). Bez \'ujmy na obecnosti p\v redpokl\'ad\'ame, \v ze 
vstupn\'\i\ posloupnost je permutace $\pi$
mno\v ziny $\left\{1,2,\dots,n\right\}$, a tato permutace proch\'az\'\i\ 
stromem takto:
\roster
\item"{}"
Za\v c\'\i n\'a v ko\v reni stromu. Kdy\v z je ve vnit\v rn\'\i m vrcholu $
v$ 
ohodnocen\'em porovn\'an\'\i m $a_i\le a_j$, pak $\pi \left(i\right)<\pi \left(j\right)$ znamen\'a, 
\v ze pokra\v cuje v lev\'em synu vrcholu $v$, a $\pi \left(j\right)<\pi \left(i\right)$ 
znamen\'a pokra\v cov\'an\'\i\ v prav\'em synu vrcholu $v$. Proces 
t\v r\'\i d\v en\'\i\ kon\v c\'\i , kdy\v z se dostane do listu.
\endroster
Aby byl algoritmus korektn\'\i , mus\'\i\ platit, \v ze dv\v e 
r\accent23uzn\'e permutace skon\v c\'\i\ v r\accent23uzn\'ych 
listech.  To znamen\'a, \v ze definovan\'y strom pak mus\'\i\ m\'\i t 
alespo\v n $n!$ list\accent23u.  D\'elka cesty z ko\v rene do listu, 
kde skon\v cila permutace $\pi$, d\'av\'a doln\'\i\ odhad na \v cas 
pot\v rebn\'y k set\v r\'\i d\v en\'\i\ posloupnosti $\pi$. To n\'am umo\v z\v nuje 
z\'\i skat doln\'\i\ odhad \v casu pot\v rebn\'eho k set\v r\'\i d\v en\'\i\ 
posloupnosti. Korektnost t\v echto \'uvah plyne z pozorov\'an\'\i , 
\v ze kdy\v z porovn\'an\'\i\ je jedin\'a primi\-tivn\'\i\ operace, pak 
algoritmus nen\'\i\ z\'avisl\'y na prvc\'\i ch vstup\-n\'\i\ 
posloupnosti, ale jen na jejich vz\'ajemn\'em vztahu. 
Proto sta\v c\'\i\ uva\v zovat pouze permutace $n$-prvkov\'e mno\v ziny, 
proto\v ze za\-chycuj\'\i\ v\v sechny mo\v zn\'e vztahy v $n$-prvkov\'e 
posloupnosti. D\'a\-le je t\v reba si uv\v edomit, \v ze vztah mezi 
stromem pro $n$-prvkov\'e posloupnosti a stromem pro 
$\left(n+1\right)$-prvkov\'e posloupnosti je d\'an konkr\'etn\'\i m algoritmem a 
ned\'a se popsat obecn\v e. 
\medskip

\flushpar V nevhodn\'em algoritmu se m\accent23u\v ze st\'at, \v ze 
v n\v ekter\'em listu neskon\v c\'\i\ \v z\'adn\'a permutace. To se 
stane, kdy\v z strom pro $n$-prvkov\'e posloupnosti m\'a v\'\i ce 
ne\v z $n!$ list\accent23u.
\medskip

\flushpar N\'asleduj\'\i c\'\i\ obr\'azek ilustruje na\v se \'uvahy na 
{\bf SELECTIONSORT}u pro $3$-prvkov\'e posloupnosti. Listy jsou 
ohodnoceny permutacemi mno\v ziny $\left\{1,2,3\right\}$, kter\'e v nich 
skon\v c\'\i\ nebo jsou pr\'azdn\'e. 
\midinsert
\centerline{\input fig11.tex}
\botcaption{Obr. 1}
\endcaption
\endinsert
\newpage

\definition{Definice}M\v ejme t\v r\'\i dic\'\i\ algoritmus {\bf A}, kter\'y 
jako jedinou pri\-mitivn\'\i\ operaci pro prvky vstupn\'\i\ 
posloupnosti pou\v z\'\i v\'a porov\-n\'an\'\i . \v Rekneme, \v ze 
bin\'arn\'\i\ strom $T$, jeho\v z vnit\v rn\'\i\ vrcholy jsou ohodnoceny 
porovn\'an\'\i mi  $a_i\le a_j$ pro $i,j=1,2,\dots,n$, $i\ne j$, je 
\emph{rozhodovac\'\i m} \emph{stromem} algoritmu {\bf A} pro 
$n$-prvkov\'e posloupnosti, kdy\v z pro ka\v zdou permutaci $\pi$
$n$-prvkov\'e mno\v ziny plat\'\i\ 
\roster
\item"{}"
posloupnost porovn\'an\'\i\ p\v ri t\v r\'\i d\v en\'\i\ posloupnosti $
\pi$ 
algoritmem {\bf A} je stejn\'a jako posloupnost porovn\'an\'\i\ p\v ri 
pr\accent23uchodu posloupnosti $\pi$ stromem $T$.
\endroster
\enddefinition
\medskip

\flushpar Pak korektnost algoritmu zaji\v s\v tuje, \v ze dv\v e r\accent23uzn\'e 
permutace mno\v ziny $\left\{1,2,\dots,n\right\}$ skon\v c\'\i\ v 
r\accent23uzn\'ych listech stromu $T$. Dol\-n\'\i m odhadem pro 
\v cas algoritmu {\bf A} v nej\-hor\v s\'\i m p\v r\'\i pad\v e je d\'elka nejdel\v s\'\i\ 
cesty z ko\v rene do listu, proto\v ze algoritmus {\bf A} p\v ri 
t\v r\'\i d\v en\'\i\ permutace pou\v zije tolik porovn\'an\'\i , jako je d\'elka 
jej\'\i\ cesty stromem $T$. Proto p\v ri rovnom\v ern\'em 
rozd\v elen\'\i\ vstupn\'\i ch po\-sloupnost\'\i\ je o\v cek\'avan\'y \v cas 
algoritmu {\bf A} pr\accent23um\v ern\'a d\'elka cesty z ko\v rene do 
listu. Tato fakta motivuj\'\i\ hled\'an\'\i\ n\'asleduj\'\i c\'\i ch veli\-\v cin. 
\medskip

\flushpar Definujme\newline 
$S\left(n\right)$ jako minimum p\v res v\v sechny stromy $T$ s alespo\v n $n
!$ listy z 
d\'elek nejdel\v s\'\i ch cest z ko\v rene 
do listu $T$ a\newline 
$A\left(n\right)$ jako minimum p\v res v\v sechny stromy $T$ s alespo\v n $n
!$ listy z
pr\accent23um\v ern\'ych d\'elek cest z ko\v rene do listu v $T$.\newline 
Na\v s\'\i m c\'\i lem je spo\v c\'\i tat doln\'\i\ odhad t\v echto veli\v cin.
\medskip

\flushpar Kdy\v z nejdel\v s\'\i\ cesta z ko\v rene do listu v 
bin\'arn\'\i m strom\v e $T$ m\'a d\'elku $k$, pak $T$ m\'a nejv\'y\v se $
2^k$ 
list\accent23u. Proto $n!\le 2^{S\left(n\right)}$. Odtud plyne $S\left(n\right)\ge\log_
2n!$.
P\v ripome\v nme si Stirling\accent23uv vzorec pro faktori\'al: 
$$n!=\sqrt {2\pi n}\left(\frac ne\right)^n\left(1+\frac 1{12n}+O\left(\frac 1{
n^2}\right)\right).$$
Proto\v ze pro $n\ge 1$ je $\frac 1{12n},\frac 1{n^2}\ge 0$, m\accent23u\v zeme 
p\v redpokl\'adat, \v ze $\left(1+\frac 1{12n}+O\left(\frac 1{n^2}\right)\right)\ge 1$ pro v\v sechna $
n\ge 1$. Po 
zlogaritmov\'an\'\i\ vzorce dost\'av\'ame 
$$\log_2n!\ge\frac 12\log_2n+n\left(\log_2n-\log_2e\right)+\log_2\sqrt {
2\pi}\ge\left(n+\frac 12\right)\log_2n-n\log_2e.$$
Proto\v ze plat\'\i\ 
$$e^1=e=2^{\log_2e}=\left(e^{\ln2}\right)^{\log_2e}=e^{\ln2\log_2e},$$
dost\'av\'ame, \v ze $\frac 1{\ln2}=\log_2e$, a tedy  
$$S\left(n\right)\ge\log_2n!\ge \left(n+\frac 12\right)\log_2n-\frac n{\ln2}.$$
\medskip

\flushpar Pro bin\'arn\'\i\ strom $T$ ozna\v cme $B\left(T\right)$ sou\v cet v\v sech 
d\'elek cest z ko\v rene do n\v ejak\'eho listu a polo\v zme 
$$B\left(k\right)=\min\left\{B\left(T\right)\mid T\text{\rm\ je bin\'arn\'\i\ strom s $k$ listy}
\right\}.$$
Kdy\v z uk\'a\v zeme, \v ze $B\left(k\right)\ge k\log_2k$, pak bude 
$$A\left(n\right)\ge\frac {B\left(n!\right)}{n!}\ge\frac {n!\log_2n!}{n!}=\log_2n!\ge \left(
n+\frac 12\right)\log_2n-\frac n{\ln2}.$$
Tak\v ze doka\v zme, \v ze $B\left(T\right)\ge k\log_2k$ pro ka\v zd\'y bin\'arn\'\i\ 
strom $T$ s $k$ listy. Kdy\v z ve strom\v e $T$ vynech\'ame ka\v zd\'y 
vrchol, kter\'y m\'a jen jednoho syna a tohoto syna spoj\'\i me 
s jeho p\v redch\accent23udcem, dostaneme \'upln\'y bin\'arn\'\i\ 
strom $T'$ s $k$ listy tako\-v\'y, \v ze $B\left(T'\right)\le B\left(T\right)$. Proto se 
sta\v c\'\i\  
omezit na \'upln\'e bin\'arn\'\i\ stromy. Kdy\v z $T$ je \'upln\'y 
bin\'arn\'\i\ strom s jedn\'\i m vrcholem, pak $B\left(T\right)=0=1\log_21$, 
kdy\v z $T$ je \'upln\'y bin\'arn\'\i\ strom s dv\v ema listy, pak 
$B\left(T\right)=2=2\log_22$. Tedy plat\'\i\ $B\left(1\right)\ge 1\log_21$ a $B\left(2\right)\ge 2\log_
22$. 
P\v redpokl\'adejme, \v ze $B\left(i\right)\ge i\log_2i$ pro $i<k$, a nech\v t $
T$ je 
\'upln\'y bin\'arn\'\i\ strom s $k$ listy. Nech\v t $T_1$ a $T_2$ jsou 
podstromy ur\v cen\'e syny ko\v rene $T$ a nech\v t $T_i$ m\'a $k_
i$ 
list\accent23u pro $i=1,2$. Pak $1\le k_1,k_2$ a $k_1+k_2=k$, tedy 
$k_1,k_2<k$, a podle induk\v cn\'\i ho p\v redpokladu $B\left(k_i\right)\ge 
k_i\log_2k_i$. 
Odtud 
$$\align B\left(T\right)=&k_1+B\left(T_1\right)+k_2+B\left(T_2\right)\ge k+B\left(k_1\right)+B\left(k_2\right)\ge\\
&k+k_1\log_2k_1+k_2\log_2k_2.\endalign$$
\medskip

\flushpar Tedy sta\v c\'\i\ uk\'azat, \v ze 
$$k+k_1\log_2k_1+k_2\log_2k_2\ge k\log_2k$$
pro v\v sechna $k_1,k_2>0$ takov\'a, \v ze $k=k_1+k_2$. Toto je 
ekvivalentn\'\i\ s tvrzen\'\i m, \v ze pro $k>0$ plat\'\i\ 
$$f\left(x\right)=x\log_2x+\left(k-x\right)\log_2\left(k-x\right)+k-k\log_2k\ge 0,$$
kde $x\in \left(0,k\right)$. 
Abychom to dok\'azali, v\v simn\v eme si, \v ze $f\left(\frac k2\right)=0$. Nyn\'\i\ 
spo\v c\'\i\-t\'ame derivaci $f$.
$$f'\left(x\right)=\log_2x+\log_2e-\log_2\left(k-x\right)-\log_2e=\log_2\frac x{k-x}.$$
Kdy\v z $x\in \left(0,\frac k2\right)$, pak $f'\left(x\right)<0$ a $f$ je na tomto intervalu 
klesaj\'\i c\'\i , kdy\v z $x\in \left(\frac k2,k\right)$, pak $f'\left(x\right)>0$ a $
f$ je na tomto 
intervalu rostouc\'\i . Odtud plyne, \v ze $f\left(x\right)\ge 0$ pro $x\in 
\left(0,k\right)$.
T\'\i m jsme dok\'azali, \v ze $A\left(n\right)\ge \left(n+\frac 12\right)\log_2n-\frac 
n{\ln2}$. Shrneme 
na\v se v\'ysledky.
\medskip

\proclaim{V\v eta}Ka\v zd\'y t\v r\'\i dic\'\i\ algoritmus, jeho\v z jedinou 
primitivn\'\i\ ope\-rac\'\i\ s prvky vstupn\'\i\ posloupnosti je 
porovn\'an\'\i , vy\v zaduje v nej\-hor\v s\'\i m i v o\v cek\'avan\'em p\v r\'\i pad\v e 
alespo\v n $cn\log n$ \v casu pro n\v eja\-kou konstantu $c>0$.  V 
nejhor\v s\'\i m p\v r\'\i pad\v e pou\v zije alespo\v n $\left\lceil 
\left(n+\frac 12\right)\log_2n-\frac n{\ln2}\right\rceil$ 
porovn\'an\'\i\ a o\v cek\'avan\'y po\v cet porovn\'an\'\i\ p\v ri rovnom\v ern\'em 
rozd\v elen\'\i\ vstupn\'\i ch posloupnost\'\i\ je alespo\v n 
$\left(n+\frac 12\right)\log_2n-\frac n{\ln2}$.  
\endproclaim

\flushpar V tomto v\'ysledku lze oslabit p\v redpoklady. V\v eta 
plat\'\i\ i za 
p\v red\-pokladu, \v ze t\v r\'\i dic\'\i\ algoritmus nepou\v z\'\i v\'a nep\v r\'\i m\'e 
adreso\-v\'an\'\i\ a celo\-\v c\'\i seln\'e d\v elen\'\i . Tato metoda pro 
nalezen\'\i\ doln\'\i ho odhadu se pou\v z\'\i v\'a i pro vy\v c\'\i slov\'an\'\i\ 
algebraick\'ych funkc\'\i\ a p\v ri algoritmick\'em \v re\v sen\'\i\ 
geometrick\'ych \'uloh. Na druh\'e stran\v e klasick\'y algoritmus 
{\bf BUCKETSORT} ukazuje, \v ze se nelze p\v red\-poklad\accent23u 
ve v\v et\v e \'upln\v e zbavit. V n\'asleduj\'\i c\'\i ch algoritmech 
p\v red\-po\-kl\'a\-d\'ame, \v ze $Q_i$ jsou spojov\'e seznamy, nov\'y 
prvek se vkl\'ad\'a na konec seznamu a konkatenace 
seznam\accent23u z\'avis\'\i\ na jejich po\v rad\'\i . V 
seznamech m\'ame okam\v zit\'y p\v r\'\i stup k prvn\'\i mu a posledn\'\i mu 
prvku (pomoc\'\i\ ukazatel\accent23u na tyto prvky).
skip

{\bf BUCKETSORT$\left(a_1,a_2,\dots,a_n,m\right)$}:\newline 
Koment\'a\v r: Vstup je p\v rirozen\'e \v c\'\i slo $m$ a posloupnost 
p\v riroze\-n\'ych \v c\'\i sel $a_1,a_2,\dots,a_n$ z intervalu $
<0,m>$. 
C\'\i lem je set\v r\'\i dit posloupnost $a_1,a_2,\dots,a_n$.\newline 
{\bf for every} $i=0,1,\dots,m$ {\bf do} $Q_i=\emptyset$ {\bf enddo\newline 
for every} $i=1,2,\dots,n$ {\bf do}\newline 
\phantom{---}$a_i$ vlo\v z na konec seznamu $Q_{a_i}$\newline 
{\bf enddo\newline 
$i:=0$}, $P:=\emptyset$\newline
{\bf while} $i\le m$ {\bf do}\newline 
\phantom{---}$P:=$konkatenace $P$ a $Q_i$, $i:=i+1$\newline 
{\bf enddo\newline 
$P$} je neklesaj\'\i c\'\i\ posloupnost prvk\accent23u 
$a_1,a_2,\dots,a_n$.
\medskip

\flushpar Algoritmus nevy\v zaduje, aby prvky ve vstupn\'\i\ 
posloupnosti by\-ly r\accent23uzn\'e.  Ve v\'ystupn\'\i\ posloupnosti se 
dan\'y prvek opakuje tolikr\'at, kolikr\'at se opakoval ve vstupn\'\i\ 
posloupnosti, se zachov\'an\'\i m po\v rad\'\i\ (algoritmus je 
stabiln\'\i ).  Konkatenace dvou 
seznam\accent23u a vlo\v zen\'\i\ prvku do seznamu vy\v zaduj\'\i\ \v cas $
O\left(1\right)$.  
Proto prvn\'\i\ a t\v ret\'\i\ cyklus vy\v zaduj\'\i\ \v cas $O\left(m
\right)$ a druh\'y cyklus 
\v cas $O\left(n\right)$.  Tedy algoritmus vy\v zaduje $O\left(n+m\right)$ \v casu a pam\v eti.  
Kdy\v z $m=O\left(n\right)$, tak pro algoritmus neplat\'\i\ tvrzen\'\i\ v\v ety.  
D\accent23uvod je, \v ze nejsou spln\v eny p\v redpoklady, proto\v ze 
druh\'y cyklus pou\v z\'\i v\'a nep\v r\'\i m\'e adresov\'an\'\i .  
\medskip

\flushpar Nyn\'\i\ uvedeme dv\v e sofistikovan\v ej\v s\'\i\ verze tohoto 
algoritmu. V prvn\'\i\ p\v redpokl\'ad\'ame, \v ze 
$a_1,a_2,\dots,a_n$ je posloupnost re\'aln\'ych 
\v c\'\i sel z intervalu $<0,1>$ a $\alpha$ je pevn\v e zvolen\'e kladn\'e 
re\'aln\'e \v c\'\i slo.
\medskip

{\bf HYBRIDSORT$\left(a_1,a_2,\dots,a_n\right)$}:\newline 
$k:=\alpha n$\newline 
{\bf for every} $i=0,1,\dots,k$ {\bf do} $Q_i:=\emptyset$ {\bf enddo\newline 
for every} $i=1,2,\dots,n$ {\bf do}\newline 
\phantom{---}$a_i$ vlo\v z na konec seznamu $Q_{\left\lceil ka_i\right\rceil}$\newline 
{\bf enddo\newline 
$i:=0$}, $P:=\emptyset$ \newline 
{\bf while} $i\le k$ {\bf do}\newline 
\phantom{---}{\bf HEAPSORT}$\left(Q_i\right)$, 
\phantom{---}$P:=$konkatenace $P$ a $Q_i$, $i:=i+1$\newline 
{\bf enddo\newline 
$P$} je rostouc\'\i\ posloupnost prvk\accent23u 
$a_1,a_2,\dots,a_n$.
skip

\proclaim{V\v eta}Algoritmus {\bf HYBRIDSORT} set\v r\'\i d\'\i\ 
posloupnost re\'al\-n\'ych \v c\'\i sel z intervalu $<0,1>$ v 
nejhor\v s\'\i m p\v r\'\i pad\v e v \v case $O\left(n\log n\right)$. Kdy\v z prvky $
a_i$ 
maj\'\i\ rovnom\v ern\'e rozlo\v zen\'\i\ a jsou na sob\v e nez\'a\-visl\'e, 
pak o\v cek\'avan\'y \v cas algoritmu {\bf HYBRIDSORT} je $O\left(n\right)$.
\endproclaim

\demo{D\accent23ukaz}Prvn\'\i\ dva cykly v algoritmu 
vy\v zaduj\'\i\ \v cas $O\left(n\right)$, $i$-t\'y b\v eh t\v ret\'\i ho cyklu vy\v zaduje 
nejv\'y\v se \v cas $O\left(1+|Q_i|\log|Q_i|\right)$. Proto t\v ret\'\i\ cyklus vy\v zaduje 
\v cas 
$$\align O\left(\sum_{i=0}^k\left(1+|Q_i|\log|Q_i|\right)\right)=&O\left(\sum_{i=0}^k\left(1+|Q_i|\log 
n\right)\right)=\\
&O\left(k+\left(\sum_{i=0}^k|Q_i|\right)\log n\right)=O\left(n\log n\right)\endalign$$
a celkov\'y \v cas {\bf HYBRIDSORT}u je nejv\'y\v se $O\left(n\log n\right)$.
\medskip

\flushpar Ozna\v cme $X_i=|Q_i|$, pak m\accent23u\v zeme 
p\v redpokl\'adat, \v ze $X_i$ je n\'a\-hodn\'a prom\v enn\'a. Proto\v ze 
pravd\v epodobnost, \v ze $x\in Q_i$, je $\frac 1k$, dost\'a\-v\'ame, \v ze 
$$\Prob\left(X_i=q\right)=\binom nq\left(\frac 1k\right)^q\left(1-\frac 1k\right)^{n-q}.$$
O\v cek\'avan\'y \v cas vy\v zadovan\'y t\v ret\'\i m cyklem se pak rovn\'a 
$$\align E\left(\sum_{i=0}^k1+X_i\log X_i\right)\le&k+k\sum_{q=2}^nq^2\binom 
nq\left(\frac 1k\right)^q\left(1-\frac 1k\right)^{n-q}=\\
&k+k\left(\frac {n\left(n-1\right)}{k^2}+\frac nk\right)=O\left(n\right),\endalign$$
proto\v ze $k=\alpha n$ a 
$$q^2\binom nq=\left(q\left(q-1\right)+q\right)\binom nq=n\left(n-1\right)\binom {n-2}{q-2}+n\binom {
n-1}{q-1}.$$
(Jedn\'a se vlastn\v e o zn\'am\'y v\'ypo\v cet druh\'eho momentu binomic\-k\'eho 
rozd\v elen\'\i .)\qed
\enddemo

\flushpar Nyn\'\i\ pou\v zijeme modifikaci {\bf BUCKETSORT}u k 
set\v r\'\i d\v en\'\i\ slov.  M\'ame tot\'aln\v e 
uspo\v r\'ada\-nou abecedu a chceme lexikograficky uspo\v r\'adat slova 
$a_1,a_2,\dots,a_n$ nad touto abecedou. P\v ri\-po\-me\v n\-me, \v ze kdy\v z $
a=x_1x_2\dots x_n$ a 
$b=y_1y_2\dots y_m$ jsou dv\v e slova nad tot\'aln\v e uspo\v r\'a\-danou 
abecedou $\Sigma$, pak $a<b$ v lexikografick\'em uspo\v r\'ad\'an\'\i , pr\'av\v e 
kdy\v z existuje $i=0,1,\dots,\min\left\{n,m\right\}$ takov\'e, \v ze $x_j=y_
j$ 
pro ka\v zd\'e $j=1,2,\dots,i$ a bu\v d $n=i<m$ nebo $i<\min\left\{n,m
\right\}$ 
a $x_{i+1}<y_{i+1}$. P\v redpokl\'adejme, \v ze $a_i=a_i^1a_i^2\dots 
a_i^{l\left(i\right)}$, kde 
$a_i^j\in\Sigma$ a $l\left(i\right)$ je d\'elka $i$-t\'eho slova $a_i$.
skip

{\bf WORDSORT$\left(a_1,a_2,\dots,a_n\right)$}:\newline 
{\bf for every} $i=1,2,\dots,n$ {\bf do} $l\left(i\right):=$d\'elka slova $a_
i$ {\bf enddo\newline 
$l:=\max\left\{l\left(i\right)\mid i=1,2,\dots,n\right\}$\newline 
for every} $i=1,2,\dots,l$ {\bf do} $L_i:=\emptyset$ {\bf enddo\newline 
for every} $i=1,2,\dots,n$ {\bf do}\newline 
\phantom{---}$a_i$ vlo\v z\'\i me do $L_{l\left(i\right)}$\newline 
{\bf enddo}\newline 
Koment\'a\v r: Pro ka\v zd\'e $i$ mno\v zina $L_i$ obsahuje v\v sechna slova z mno\v ziny 
$\left\{a_1,a_2,\dots,a_n\right\}$ o d\'elce $i$.\newline 
$P:=\left\{\left(j,a_i^j\right)\mid 1\le i\le n,\,1\le j\le l\left(i\right)\right\}$\newline 
$P_1:=${\bf BUCKETSORT$\left(P\right)$} podle druh\'e komponenty\newline 
$P_2:=${\bf BUCKETSORT$\left(P_1\right)$} podle prvn\'\i\ komponenty\newline
{\bf for every} $i=1,2,\dots,l$ {\bf do} $S_i:=\emptyset$ {\bf enddo\newline 
$\left(i,x\right):=$}prvn\'\i\ prvek $P_2$\newline 
{\bf while} $\left(i,x\right)\ne NIL$ {\bf do}\newline 
\phantom{---}$\left(i,x\right)$ vlo\v z\'\i me do $S_i$\newline 
\phantom{---}$\left(i,x\right):=$n\'asledn\'\i k $\left(i,x\right)$ v $P_2$\newline 
{\bf enddo\newline}
Koment\'a\v r: V $S_i$ jsou v\v sechny dvojice $\left(i,x\right)$ takov\'e, \v ze $x$ 
je $i$-t\'ym p\'\i smenem abecedy a kdy\v z $x<y$, pak $\left(i,x\right)$ je 
p\v red $\left(i,y\right)$.\newline 
{\bf for every} $s\in\Sigma$ {\bf do} $T_s:=\emptyset$ {\bf enddo}\newline 
$T:=\emptyset$, $i:=l$\newline 
{\bf while} $i>0$ {\bf do\newline 
\phantom{---}$T:=L_i$} konkatenace s $T$, $a:=$prvn\'\i\ slovo v $T$\newline 
\phantom{---}{\bf while} $a\ne NIL$ {\bf do}\newline 
\phantom{------}$s:=i$-t\'e p\'\i smeno $a$, vlo\v zme $a$ do $T_s$\newline 
\phantom{------}$a:=$n\'asledn\'\i k $a$ v $T$\newline 
\phantom{---}{\bf enddo\newline 
\phantom{---}$\left(i,x\right):=$}prvn\'\i\ prvek v $S_i$, $T:=\emptyset$\newline 
\phantom{---}{\bf while} $\left(i,x\right)\ne NIL$ {\bf do}\newline 
\phantom{------}$T:=T$ konkatenace s $T_x$, $T_x:=\emptyset$\newline 
\phantom{------}$\left(i,x\right):=$n\'asledn\'\i k $\left(i,x\right)$ v $S_i$\newline 
\phantom{---}{\bf enddo}\newline
\phantom{---}$i:=i-1$\newline 
{\bf enddo\newline 
$T$} je set\v r\'\i d\v en\'a posloupnost slov $a_1,a_2,\dots,a_n$.
skip

\flushpar Uva\v zujme b\v eh posledn\'\i ho cyklu pro dan\'e $i$. Po 
jeho skon\v cen\'\i\ jsou v $T$ v\v sechna slova z mno\v ziny 
$a_1,a_2,\dots,a_n$, kter\'a maj\'\i\ d\'elku alespo\v n $i$, a kdy\v z slovo 
$a_q$ je p\v red $a_r$ v seznamu $T$, pak existuje $j=i-1,i,\dots
,l$
takov\'e, \v ze $a^k_r=a^k_q$ pro ka\v zd\'e $k=i,i+1,\dots,j$ a bu\v d 
$l\left(r\right)=j\le l\left(q\right)$ nebo $j<\min\left\{l\left(r\right),l\left(q\right)\right\}$ a $a_r^{j+1}<a_q^{j+1}$. To 
dost\'av\'ame indukc\'\i\ podle $i$ (viz {\bf BUCKETSORT}). Jedin\'y a 
hlavn\'\i\ rozd\'\i l proti {\bf BUCKETSORT}u je, \v ze neproch\'az\'\i me 
v\v sechny p\v rihr\'adky $T_x$, ale pouze nepr\'azdn\'e p\v rihr\'adky. 
To n\'am zaji\v s\v tuje mno\v zina $S_i$, viz Koment\'a\v r. 
\medskip

\flushpar Ozna\v cme $L=\sum_{i=1}^nl\left(i\right)$. Prvn\'\i\ cyklus (spo\v c\'\i t\'an\'\i\ 
d\'elek slov) vy\-\v zaduje \v cas $O\left(L\right)$. Druh\'y cyklus vy\v zaduje 
\v cas $O\left(l\right)=O\left(L\right)$ a t\v ret\'\i\ cyklus \v cas $O\left(n\right)=O\left(L\right)$. Vytvo\v ren\'\i\ 
seznamu $P_1$ vy\v zaduje \v cas $O\left(L\right)$ a jeho set\v r\'\i d\v en\'\i\ 
\v cas $O\left(L+l\right)=O\left(L\right)$, proto\v ze $P_1$ i $P_2$ maj\'\i\ nejv\'y\v se $
L$ 
prvk\accent23u. Dal\v s\'\i\ cyklus (zalo\v zen\'\i\ seznam\accent23u 
$S_i$) vy\v zaduje \v cas $O\left(l\right)$ a n\'asleduj\'\i c\'\i\ cyklus vytv\'a\v rej\'\i c\'\i\ 
seznamy $S_i$ \v cas $O\left(L\right)$. Cyklus zakl\'adaj\'\i c\'\i\ 
seznamy $T_x$ vy\v zaduje \v cas $O\left(|\Sigma |\right)$. B\v ehy dal\v s\'\i ho cyklu jsou 
indexov\'any $i=1,2,\dots,l$. Pro ka\v zd\'e $i$ ozna\v cme $m_i$ po\v cet slov z 
mno\v ziny $\left\{a_1,a_2,\dots,a_n\right\}$, kter\'a maj\'\i\ d\'elku alespo\v n $
i$. Pak 
$L=\sum_{i=1}^lm_i$ a prvn\'\i\ vnit\v rn\'\i\ cyklus v $i$-t\'em b\v ehu vn\v ej\v s\'\i ho 
cyklu vy\v zaduje \v cas $O\left(m_i\right)$ a druh\'y vnit\v rn\'\i\ cyklus v $
i$-t\'em 
b\v ehu vn\v ej\v s\'\i ho cyklu vy\v zaduje \v cas $O\left(|S_i|\right)=O\left(m_
i\right)$. Tedy 
celkov\'y \v cas algoritmu je $O\left(L+m\right)$, kde $m=|\Sigma |$ a $L$ 
je sou\v cet d\'elek v\v sech slov z mno\v ziny $a_1,a_2,\dots,a_
n$.
\medskip

\subhead
Hled\'an\'\i\ $k$-t\'eho prvku
\endsubhead
\smallskip

\flushpar Na z\'av\v er pop\'\i\v seme dva algoritmy pro hled\'an\'\i\ $
k$-t\'eho 
nejmen\v s\'\i\-ho prvku v dan\'e podmno\v zin\v e tot\'aln\v e 
uspo\v r\'adan\'e\-ho univerza. Prv\-n\'\i\ z nich vyu\v z\'\i v\'a  
stejn\'y princip jako {\bf QUICKSORT}. P\v res\-n\'e zn\v e\-n\'\i\ 
probl\'emu:
\medskip

\flushpar Vstup: mno\v zina prvk\accent23u $M=\left\{a_1,a_2,\dots,a_n
\right\}$ a \v c\'\i slo $i$ takov\'e, 
\v ze $1\le i\le n$.\newline 
V\'ystup: prvek $a_k$ takov\'y, \v ze 
$|\left\{j\mid 1\le j\le n,\,a_j\le a_k\right\}|=i$.\newline 
Kdy\v z $i=\frac n2$, pak $a_k$ se naz\'yv\'a \emph{medi\'an}.
\medskip

\subhead
Praktick\'y algoritmus
\endsubhead
\medskip

{\bf FIND}$\left(M=\left(a_1,a_2,\dots,a_n\right),i\right)$:\newline 
zvolme $a\in M$, $M_1:=\left\{b\in M\mid b<a\right\}$, 
$M_2:=\left\{b\in M\mid b>a\right\}$\newline 
{\bf if} $|M_1|>i-1$ {\bf then}\newline 
\phantom{---}{\bf FIND}$\left(M_1,i\right)$\newline 
{\bf else}\newline 
\phantom{---}{\bf if} $|M_1|<i-1$ {\bf then}\newline 
\phantom{------}{\bf FIND}$\left(M_2,i-|M_1|-1\right)$\newline 
\phantom{---}{\bf else}\newline 
\phantom{------}$a$ je hledan\'y prvek\newline 
\phantom{---}{\bf endif\newline 
endif}
skip

\flushpar Korektnost algoritmu je z\v rejm\'a. V nejhor\v s\'\i m 
p\v r\'\i pad\v e vol\'ame {\bf FIND} $i$-kr\'at a jedno vol\'an\'\i\ vy\v zaduje \v cas 
$O\left(|M|\right)$. Tedy {\bf FIND} v nejhor\v s\'\i m p\v r\'\i pad\v e vy\v zaduje \v cas 
$O\left(n^2\right)$. Dobr\'e volby prvku $a$ mohou algoritmus zna\v cn\v e 
zrychlit. Zde plat\'\i\ stejn\'a diskuse jako pro {\bf QUICKSORT}.  
Spo\v c\'\i t\'ame o\v cek\'avan\'y \v cas, kdy\v z pr\-vek $a$ byl vybr\'an 
n\'ahodn\v e. Pak pravd\v epodob\-nost, \v ze je $k$-t\'ym nejmen\v s\'\i m 
prvkem, je $\frac 1n$, 
kde $n=|M|$. Ozna\v cme $T\left(n,i\right)$ o\v cek\'avan\'y \v cas algoritmu 
{\bf FIND} pro nalezen\'\i\ $i$-t\'eho nejmen\v s\'\i ho prvku v $
n$-prvkov\'e 
mno\v zin\v e $M$. Plat\'\i
$$T\left(n,i\right)=n+\frac 1n\left(\sum_{k=1}^{i-1}T\left(n-k,i-k\right)+\sum_{k=i+1}^nT\left(k,
i\right)\right),$$
proto\v ze procedura {\bf FIND} bez rekurzivn\'\i ho vol\'an\'\i\ sama 
sebe vy\v zaduje \v cas $O\left(n\right)$. P\v redpok\-l\'a\-dej\-me, \v ze $T\left(m,i
\right)\le 4m$ 
pro ka\v zd\'e $m<n$ a ka\v zd\'e $i$ takov\'e, \v ze $1\le i\le 
m$. Pak 
$$\align T\left(n,i\right)=&n+\frac 1n\left(\sum_{k=1}^{i-1}T\left(n-k,i-k\right)+\sum_{k=i+
1}^nT\left(k,i\right)\right)\le\\
&n+\frac 1n\left(\sum_{k=1}^{i-1}4\left(n-k\right)+\sum_{k=i+1}^n4k\right)=\\
&n+\frac 4n\left(\frac {\left(2n-i\right)\left(i-1\right)}2+\frac {\left(n+i+1\right)\left(n-i\right)}2\right)=\\
&n+\frac 4n\left(\frac {n^2+2ni-n-2i^2}2\right).\endalign$$
V\'yraz v \v citateli zlomku nab\'yv\'a sv\'eho maxima pro 
$i=\frac n2$ a jeho maximaln\'\i\ hodnota je 
$\frac 32n^2-n=\frac {3n^2-2n}2$. Tedy
$$T\left(n,i\right)\le n+\frac 4n\left(\frac {3n^2-2n}4\right)=n+3n-2=4n-2<4n$$
Proto\v ze tento odhad plat\'\i\ tak\'e pro $n=1$ a $n=2$, uk\'azali 
jsme, \v ze $T\left(n,i\right)\le 4n$ pro v\v sechna $n$ a v\v sechna $i$ 
takov\'a, \v ze $1\le i\le n$. Shrneme z\'\i skan\'e v\'ysledky o 
algoritmu {\bf FIND}.
\medskip

\proclaim{V\v eta}Algoritmus {\bf FIND} nalezne $i$-t\'y nejmen\v s\'\i\ 
prvek v $n$ prv\-kov\'e tot\'aln\v e uspo\v r\'adan\'e mno\v zin\v e. V nejhor\v s\'\i m p\v r\'\i pad\v e 
vy\v za\-du\-je \v cas $O\left(n^2\right)$, ale kdy\v z se pivot vol\'\i\ 
n\'ahodn\v e nebo kdy\v z v\v sech\-ny vstupn\'\i\ mno\v ziny maj\'\i\ stejnou 
pravd\v epodobnost, pak o\v cek\'a\-va\-n\'y \v cas je $O\left(n\right)$.
\endproclaim


\flushpar Pro velmi mal\'a $i$ nebo pro $i$ velmi bl\'\i zk\'a $n$ 
pracuje rychleji p\v r\'\i m\'y p\v rirozen\'y algoritmus (udr\v zuje si 
posloupnost $i$ nejmen\-\v s\'\i ch nebo $n-i$ nejv\v et\v s\'\i ch 
prvk\accent23u a k n\'\i\ p\v rid\'av\'a dal\v s\'\i\ tak, \v ze ten 
prvek, kter\'y p\v re\-kro\v cil danou hranici, je zapomenut). Tento algoritmus v\v sak nen\'\i\ efektivn\'\i\ pro obecn\'a $i$. 
\medskip

\subhead
Teoretick\'y algoritmus
\endsubhead
\smallskip

\flushpar N\'asleduj\'\i c\'\i\ algoritmus nalezne $i$-t\'y nejmen\v s\'\i\ 
prvek v line\'arn\'\i m \v case.  Vstup je podmno\v zina $M$ tot\'aln\v e 
uspo\v r\'adan\'eho univerza $U$ a p\v rirozen\'e \v c\'\i slo $i$ takov\'e, \v ze 
$1\le i\le |M|$.  
\medskip

{\bf SELECT$\left(M,i\right)$}:\newline 
$n:=|M|$\newline 
{\bf if} $n\le 100$ {\bf then}\newline 
\phantom{---}set\v r\'\i d\'\i me mno\v zinu $M$ a najdeme $i$-t\'y 
nejmen\v s\'\i\ prvek $m$\newline 
{\bf else}\newline
\phantom{---}rozd\v el\'\i me $M$ do $\left\lceil\frac n5\right\rceil$ navz\'ajem disjunktn\'\i ch p\v etiprvko\-v\'ych podmno\v zin\newline 
\phantom{---}$A_1,A_2,\dots,A_{\left\lceil\frac n5\right\rceil}$ (posledn\'\i\ z 
podmno\v zin m\accent23u\v ze m\'\i t m\'en\v e ne\v z 5 prvk\accent23u).\newline 
\phantom{---}{\bf for every} $j=1,2,\dots,\left\lceil\frac n5\right\rceil$ {\bf do}\newline 
\phantom{------}najdeme medi\'an $m_j$ mno\v ziny $A_j$\newline 
\phantom{---}{\bf enddo\newline 
\phantom{---}$\bar {m}:=$SELECT$\left(\left\{m_j\mid j=1,2,\dots,\left\lceil\frac n5\right\rceil \right\}
,\left\lceil\frac n{10}\right\rceil \right)$\newline 
\phantom{---}$M_1:=\left\{m\in M\mid m<\bar {m}\right\}$}, $M_2:=\left\{m\in M\mid\bar {m}<m\right\}$\newline 
\phantom{---}{\bf if} $|M_1|>i-1$ {\bf then}\newline 
\phantom{------}$m:=${\bf SELECT}$\left(M_1,i\right)$\newline 
\phantom{---}{\bf else}\newline 
\phantom{------}{\bf if} $|M_1|<i-1$ {\bf then}\newline 
\phantom{---------}$m:=${\bf SELECT}$\left(M_2,i-|M_1|-1\right)$\newline 
\phantom{------}{\bf else}\newline 
\phantom{---------}$m:=\bar {m}$\newline 
\phantom{------}{\bf endif}\newline 
\phantom{---}{\bf endif}\newline 
\phantom{---}{\bf V\'ystup}: $m$\newline 
{\bf endif}
\medskip

\flushpar Korektnost algoritmu je z\v rejm\'a, zb\'yv\'a vy\v set\v rit 
slo\v zitost. Nej\-prve uk\'a\v zeme, \v ze

\proclaim{Lemma}Kdy\v z $n\ge 100$, pak $|M_1|,|M_2|\le\frac {8n}{
11}$.
\endproclaim

\demo{D\accent23ukaz} Pro $j\le\lfloor\frac n5\rfloor$, kdy\v z $
m_j<\bar {m}$, pak 
$|A_j\cap M_1|\ge 3$, kdy\v z $m_j>\bar {m}$, pak $|A_j\cap M_2|\ge 
3$, kdy\v z $m_j=\bar {m}$, 
pak $|A_j\cap M_1|=|A_j\cap M_2|=2$.  Proto\v ze 
$|\left\{j=0,1,\dots,\lfloor\frac n5\rfloor\mid m_j<\bar {m}\right\}|,|\left\{j=0
,1,\dots,\lfloor\frac n5\rfloor\mid m_j>\bar {m}\right\}|\ge\lfloor\frac 
n{10}\rfloor$, 
dost\'av\'ame $|M_1|,|M_2|\ge\lfloor\frac {3n}{10}\rfloor -1$.  Proto\v ze plat\'\i\ $
M_1\cap M_2=\emptyset$ 
a $M_1\cup M_2=M\setminus \left\{\bar {m}\right\}$ a proto\v ze $\frac {8n}{
11}+\lfloor\frac {3n}{10}\rfloor -1\ge\frac {113n}{110}-2\ge n$ 
kdy\v z $n>100$, dost\'av\'ame po\v zadovan\'y odhad. \qed
\enddemo

\flushpar Maxim\'aln\'\i\ \v cas vy\v zadovan\'y algoritmem  
{\bf SELECT$\left(M,i\right)$} pro $|M|=n$ ozna\v cme $T\left(n\right)$.  Kdy\v z $n\le 
100$, pak z\v rejm\v e 
exis\-tuje konstanta $a$ takov\'a, \v ze $T\left(n\right)\le an$.  Kdy\v z $
n>100$, 
pak $\left\lceil\frac n5\right\rceil\le\frac {21n}{100}$, a proto\v ze {\bf SELECT$
\left(M,i\right)$} pro $|M|>100$ bez 
rekurentn\'\i ch vol\'an\'\i\ vy\v zaduje \v cas $O\left(|M|\right)$, dost\'av\'ame pro 
$n>100$, \v ze $T\left(n\right)\le T\left(\frac {21n}{100}\right)+T\left(\frac {8n}{11}\right)+bn$ pro n\v ejakou konstantu 
$b$.  Zvolme $c\ge\max\left\{a,\frac {1100b}{69}\right\}$. Uk\'a\v zeme, \v ze $
T\left(n\right)\le cn$.  
Kdy\v z $n\le 100$, tak tvrzen\'\i\ plat\'\i , proto\v ze $a\le c$.  Kdy\v z 
$n>100$, pak $\left\lceil\frac {21n}{100}\right\rceil ,\left\lceil\frac {8n}{11}\right\rceil 
<n$, a proto 
$$T\left(n\right)\le c\frac {21n}{100}+c\frac {8n}{11}+bn=\left(\frac {1031c}{110
0}+b\right)n\le cn.$$
Tedy
\medskip

\proclaim{V\v eta}Algoritmus {\bf SELECT} nalezne $i$-t\'y nejmen\v s\'\i\ 
prvek v line\'arn\'\i m \v case.
\endproclaim

\flushpar Algoritmus {\bf FIND} je ve velk\'e v\v et\v sin\v e 
p\v r\'\i pad\accent23u 
rychlej\v s\'\i\ ne\v z algoritmus {\bf SELECT}, proto se v praxi 
doporu\v cuje pou\v z\'\i vat {\bf FIND}, i kdy\v z existuj\'\i\ 
p\v r\'\i pady (velmi \v r\'\i dk\'e), kdy pot\v rebuje kvadratick\'y \v cas.
Je zn\'amo, \v ze lze nal\'ezt medi\'an $n$-prvkov\'e mno\v ziny s 
m\'en\v e ne\v z $3n$ porov\-n\'an\'\i mi, a \v ze ka\v zd\'y algoritmus hledaj\'\i c\'\i\ 
medi\'an a pou\v z\'\i vaj\'\i c\'\i\ porov\-n\'an\'\i\ jako jedinou primi\-tivn\'\i\ 
operaci mezi prvky mno\v ziny vy\v zaduje v\'\i ce ne\v z $2n$ 
porov\-n\'an\'\i .
\medskip

\flushpar Historick\'e resume:  Algoritmus {\bf HEAPSORT }
navrhl v roce 1964 Williams a vylep\v sil Floyd (rovn\v e\v z 1964).  N\'avrh 
na pou\v zit\'\i\ $d$-regul\'ar\-n\'\i ch hald je folklor stejn\v e tak jako  
algoritmus {\bf MER\-GESORT}.  Algoritmy {\bf QUICKSORT} a {\bf FIND }
zavedl Hoare (1962).  Anal\'yza operace {\bf MERGE} a 
hled\'an\'\i\ optim\'aln\'\i ho stro\-mu poch\'az\'\i\ od Huffmana 
(1952) a line\'arn\'\i\ implementaci algoritmu navrhl van 
Leeuwen (1976).  Anal\'yza rozhodovac\'\i ch strom\accent23u je 
folklor. Algoritmus {\bf HYBRIDSORT} navrhli Meijer a Akl (1980), 
vylep\v se\-n\'a verze {\bf BUCKETSORT}u (nazvan\'a {\bf WORDSORT}) 
poch\'az\'\i\ od Aho, Hopcrofta a Ullmana (1974), algoritmus {\bf SELECT} byl 
navr\v zen Blumem, Floydem, Prattem, Rivestem a 
Tarjanem (1972).  


\end
