%%AmSTeX
\input amstex
\input amsppt.sty

%\documentstyle{amsppt}

\magnification=\magstep1
\NoPageNumbers
\nopagenumbers
\pagewidth{5.4in}


\def \emph#1{\underbar{#1}}
\def \Prob{\operatorname{Prob}}
\def \count{\operatorname{count}}
\def \NIL{\operatorname{NIL}}
\def \otec{\operatorname{otec}}
\def \list{\operatorname{list}}
\def \Prv{\operatorname{Prv}}
\def \Nasl{\operatorname{Nasl}}
\def \levy{\operatorname{levy}}
\def \pravy{\operatorname{pravy}}
\def \bratr{\operatorname{bratr}}
\def \Prst{\operatorname{Prst}}
\def \key{\operatorname{key}}
\def \hloubka{\operatorname{hloubka}}
\input lhead

\heading
Uspo\v r\'adan\'y slovn\'\i kov\'y probl\'em
\endheading

\flushpar Jedn\'a se o roz\v s\'\i\v ren\'\i\ z\'akladn\'\i ho slovn\'\i kov\'eho probl\'emu. Je 
d\'ano tot\'aln\v e uspo\v r\'adan\'e univerzum $U$ (tj. pro ka\v zd\'e dva 
r\accent23uzn\'e prvky $u,v\in U$ plat\'\i\ bu\v d $u<v$ nebo $v<
u$). C\'\i lem 
je reprezentovat 
mno\v zinu $S\subseteq U$ a navrhnout algoritmy pro tyto operace:\newline 
{\bf MEMBER}, {\bf INSERT}, {\bf DELETE\newline 
MIN} -- nalezne nejmen\v s\'\i\ prvek v $S$,\newline 
{\bf MAX} -- nalezne nejv\v et\v s\'\i\ prvek v $S$,\newline 
$\bold S\bold P\bold L\bold I\bold T\left(x\right)$ -- zkonstruuje reprezentace dvou mno\v zin 
$S_1=\left\{s\in S\mid s<x\right\}$ a $S_2=\left\{s\in S\mid s>x\right\}$ a ozn\'am\'\i , zda 
$x\in S$,\newline 
{\bf JOIN} -- pou\v z\'\i vaj\'\i\ se dv\v e verze t\'eto operace:\newline 
$\bold J\bold O\bold I\bold N\bold 2\left(S_1,S_2\right)$ -- jsou d\'any reprezentace mno\v zin $
S_1$ a $S_2$, kter\'e 
spl\v nuj\'\i\ $\max S_1<\min S_2$, vytvo\v r\'\i\ se reprezentace mno\v ziny 
$S=S_1\cup S_2$,\newline 
$\bold J\bold O\bold I\bold N\bold 3\left(S_1,x,S_2\right)$ -- jsou d\'any reprezentace mno\v zin $
S_1$ a $S_2$ a 
prvek $x\in U$ tak, \v ze je spln\v eno $\max S_1<x<\min S_2$, vytvo\v r\'\i\ 
se reprezentace mno\v ziny $S=S_1\cup \left\{x\right\}\cup S_2$.
\smallskip

\flushpar Je vid\v et, \v ze operace {\bf JOIN2} a {\bf JOIN3} lze pomoc\'\i\ ope\-rac\'\i\ 
{\bf INSERT} a {\bf DELETE} p\v rev\'est jednu na druhou. Proto \v casto budeme 
popisovat pro danou strukturu jen jednu z nich. 
Ob\v cas se tak\'e pou\v z\'\i v\'a operace\newline 
$\bold o\bold r\bold d\left(k\right)$ -- p\v redpok\'ad\'ame, \v ze $k\le |S
|$, a operace nalezne $k$-t\'y 
nejmen\v s\'\i\ prvek v $S$.\newline 
Z\v rejm\v e operace {\bf MIN} a {\bf MAX} jsou speci\'aln\'\i m p\v r\'\i padem 
ope\-race {\bf ord$\left(k\right)$}, 
p\v resn\v e {\bf MIN} je operace $\bold o\bold r\bold d\left(1\right)$ a {\bf MAX} je operace $
\bold o\bold r\bold d\left(|S|\right)$.


\heading
$\left(a,b\right)$-stromy
\endheading

\flushpar D\accent23ule\v zitou datovou strukturou vhodnou pro 
\v re\v sen\'\i\ uspo\v r\'a\-dan\'e\-ho slovn\'\i\-ko\-v\'e\-ho probl\'emu jsou 
$\left(a,b\right)$-stromy. Tuto datovou strukturu lze pou\v z\'\i t pro intern\'\i\ i 
pro extern\'\i\ pam\v e\v t. Je to struktura zalo\v zen\'a na stromech. 
Nejobecn\v ej\v s\'\i\ grafov\'a defini\-ce $\left(a,b\right)$-stromu je:\newline 
Nech\v t $1\le a<b$ jsou kladn\'a p\v rirozen\'a \v c\'\i sla. Pak ko\v renov\'y 
strom $\left(T,t\right)$ se naz\'yv\'a $\left(a,b\right)$-\emph{strom}, kdy\v z 
\roster
\item
kdy\v z $v$ je vnit\v rn\'\i\ vrchol stromu $T$ r\accent23uzn\'y od 
ko\v rene $t$, pak m\'a alespo\v n $a$ a nejv\'y\v se $b$ syn\accent23u;
\item
v\v sechny cesty z ko\v rene do libovoln\'eho listu maj\'\i\ 
stejnou d\'elku.
\endroster
Tato definice je p\v r\'\i li\v s obecn\'a a pro datov\'e struktury se 
nehod\'\i . Proto pou\v z\'\i v\'ame jej\'\i\ speci\'aln\'\i\ p\v r\'\i pad. Datov\'a 
struktura $\left(a,b\right)$-strom je definov\'ana jen na t\v echto stromech: 
Nech\v t $a$ a $b$ jsou p\v rirozen\'a \v c\'\i sla takov\'a, \v ze $
2\le a$ a 
$2a-1\le b$. Pak ko\v renov\'y 
strom $\left(T,t\right)$ nazveme $\left(a,b\right)$-\emph{strom}, kdy\v z plat\'\i\ 
\roster
\item
ka\v zd\'y vnit\v rn\'\i\ vrchol $v$ stromu $T$ r\accent23uzn\'y od 
ko\v rene $t$ m\'a alespo\v n $a$ a nejv\'y\v se $b$ syn\accent23u;
\item
ko\v ren je bu\v d list nebo m\'a alespo\v n dva syny a nejv\'y\v se $
b$ 
syn\accent23u;
\item
v\v sechny cesty z ko\v rene do libovoln\'eho listu maj\'\i\ 
stejnou d\'elku.
\endroster
\medskip

\flushpar V\'yhody na\v sich $\left(a,b\right)$-strom\accent23u:\newline 
Kdy\v z m\'a $\left(a,b\right)$-strom \emph{v\'y\v sku} $h>0$ (tj. d\'elka ka\v zd\'e 
cesty z ko\v rene do libovoln\'eho listu je $h$), pak 
strom m\'a alespo\v n $2a^{h-1}$ list\accent23u a nejv\'y\v se $b^
h$ 
list\accent23u.


\proclaim{Tvrzen\'\i}M\v ejme p\v rirozen\'a \v c\'\i sla $a$ a $
b$ takov\'a, 
\v ze $a\ge 2$ a $b\ge 2a-1$. Pak pro ka\v zd\'e kladn\'e p\v rirozen\'e 
\v c\'\i slo $n$ existuje $\left(a,b\right)$-strom, kter\'y m\'a p\v resn\v e $
n$ 
list\accent23u. Kdy\v z $\left(a,b\right)$-strom m\'a p\v resn\v e $n$ 
list\accent23u, pak v\'y\v ska stromu je nejv\'y\v se $1+\log_a\left(\frac 
n2\right)$ a 
je alespo\v n $\log_bn$. Tedy v\'y\v ska stromu je $O\left(\log n\right)$.
\endproclaim


\flushpar M\v ejme ko\v renov\'y strom $\left(T,t\right)$ takov\'y, \v ze pro ka\v zd\'y 
vnit\v rn\'\i\ vrchol $v$ plat\'\i :\newline 
kdy\v z $v$ m\'a $\rho \left(v\right)$ syn\accent23u, pak jsou o\v c\'\i slov\'any od $
1$ do 
$\rho \left(v\right)$. \v Rekne\-me, \v ze vrchol $v$ je v \emph{hloubce} $h$, kdy\v z 
cesta z ko\v rene $t$ do $v$ m\'a d\'elku $h$. Mno\v zina v\v sech 
vrchol\accent23u v hloubce $h$ se naz\'yv\'a $h$-\emph{t\'a} 
\emph{hladina}. \emph{Lexikografick\'e} \emph{uspo\v r\'ad\'an\'\i} 
na $h$-t\'e hladin\v e je definov\'ano rekurzivn\v e:\newline 
\newline 
$v\le w$, pr\'av\v e kdy\v z bu\v d $\otec\left(v\right)<\otec\left(w\right)$ nebo 
$\otec\left(v\right)=\otec\left(w\right)$ a kdy\v z $v$ je $i$-t\'y syn $\otec\left(v\right)$ a $w$ je 
$j$-t\'y syn $\otec\left(v\right),$ pak $i\le j$.  
\medskip

\flushpar P\v redpokl\'ad\'ame, \v ze v $\left(a,b\right)$-stromu  
synov\'e ka\v zd\'eho vnit\v rn\'\i ho vrcholu jsou uspo\v r\'ad\'any. Listy 
tvo\v r\'\i\ hladinu $h$, kde $h$ je hloub\-ka $\left(a,b\right)$-stromu, a je na nich 
definov\'ano lexikografick\'e uspo\v r\'a\-d\'a\-n\'\i .
\medskip

\flushpar M\v ejme line\'arn\v e uspo\v r\'adan\'e univerzum $U$ a 
mno\v zinu $S\subseteq U$.  Pak $\left(a,b\right)$-strom $\left(T,t\right)$ 
\emph{reprezentuje} mno\v zinu $S$, kdy\v z m\'a p\v resn\v e $|S
|$ 
list\accent23u a je d\'an izomorfismus mezi lexikografick\'ym uspo\v r\'a\-d\'a\-n\'\i m 
list\accent23u stromu $T$ a uspo\v r\'adanou mno\v zinou $S$ 
(tj.  bijekce $\key:\list\left(T\right)@>>>S$, kter\'a pro $s,t\in S$ 
spl\v nuje $s\le t$ v $U$, pr\'av\v e kdy\v z $\key^{-1}\left(s\right)\le\key^{
-1}\left(t\right)$ v 
lexikografick\'em uspo\v r\'ad\'an\'\i\ na mno\v zin\v e list\accent23u 
stromu $T$).  
\medskip

\flushpar Struktura vnit\v rn\'\i ch vrchol\accent23u 
$\left(a,b\right)$-stromu $\left(T,t\right)$ reprezentuj\'\i c\'\i\-ho mno\v zinu 
$S\subseteq U$:\newline 
$\rho \left(v\right)$ -- po\v cet syn\accent23u vrcholu $v$,\newline 
$S_v\left(1..\rho \left(v\right)\right)$ -- pole ukazatel\accent23u na syny vrcholu $v$ takov\'e, 
\v ze $S_v\left(i\right)$ je $i$-t\'y syn vrcholu $v$ pro 
$i=1,2\dots,\rho \left(v\right)$,\newline 
$H_v\left(1..\rho \left(v\right)-1\right)$ -- pole prvk\accent23u z $U$ takov\'e, \v ze $
H_v\left(i\right)$ 
je nejv\v et\v s\'\i\ prvek z $S$ reprezentovan\'y v podstromu 
$i$-t\'eho syna vrcholu $v$ (alternativa: $H_v\left(i\right)$ je prvek z $U$ 
takov\'y, \v ze nejv\v et\v s\'\i\ prvek reprezentovan\'y v podstromu $
i$-t\'eho 
syna vrcholu $v$ je men\v s\'\i\ nebo roven $H_v\left(i\right)$ a to je men\v s\'\i\  
ne\v z nejmen\v s\'\i\ prvek reprezentovan\'y v podstromu $\left(i+1\right)$-n\'\i ho 
syna vrcholu $v$).
\medskip

\flushpar Struktura list\accent23u:\newline 
listu $v$ je p\v ri\v razen prvek $\key\left(v\right)\in S$.
\medskip

\flushpar N\v ekdy je ve struktu\v re ka\v zd\'eho vrcholu $v$ 
$\left(a,b\right)$-stromu r\accent23uzn\'eho od ko\v rene je\v st\v e ukazatel 
$\otec\left(v\right)$ na otce vrcholu $v$.
\medskip

\flushpar Kdy\v z $H_v\left(i\right)$ jsou prvky z reprezentovan\'e 
mno\v ziny, pak pro ka\v zd\'y prvek $s\in S$ krom\v e nejv\v et\v s\'\i ho 
existuje pr\'av\v e jeden vnit\v r\-n\'\i\ vrchol $v$ $\left(a,b\right)$-stromu 
a jedno $i$, \v ze $H_v\left(i\right)=s$, a nejv\v et\v s\'\i\ prvek v $S$ nen\'\i\ prvek 
$H_v$ pro \v z\'adn\'y vrchol $v$. Tento fakt se pou\v z\'\i v\'a p\v ri 
implementaci, kde se vynech\'avaj\'\i\ listy. Prvky z $S$ jsou 
reprezentov\'any v pol\'\i ch $H_v$ vnit\v r\-n\'\i ch vrchol\accent23u 
stromu 
a nejv\v et\v s\'\i\ prvek je ulo\v zen zvl\'a\v s\v t nebo je k mno\v zin\v e $S$ p\v rid\'an 
form\'aln\'\i\ nejv\v et\v s\'\i\ prvek (a ten je pak "ulo\v zen" zvl\'a\v s\v t). Je to prostorov\v e 
efektivn\v ej\v s\'\i\ reprezentace mno\v ziny $S$, ale je technicky 
nep\v rehledn\'a. Proto p\v ri pr\'aci s $\left(a,b\right)$-stromy pou\v z\'\i v\'am 
verzi s listy.
\medskip

\flushpar Nyn\'\i\ uvedeme algoritmy pro $\left(a,b\right)$-stromy.

\subhead 
Algoritmy
\endsubhead

\flushpar Pomocn\'y algoritmus
skip

{\bf Vyhledej$\left(x\right)$\newline 
$t:=$}ko\v ren stromu $T$, $w:=NIL$\newline 
{\bf while} $t$ nen\'\i\ list {\bf do}\newline 
\phantom{---}$i:=1$\newline 
\phantom{---}{\bf while} $H_t\left(i\right)<x$ a $i<\rho \left(t\right)$ {\bf do} $i:=i+1$ {\bf enddo}\newline 
\phantom{---}{\bf if} $H_t\left(i\right)=x$ {\bf then} $w:=t$ {\bf endif}\newline 
$t:=S_t\left(i\right)$ {\bf enddo V\'ystup}: $t$ a $w$.
\medskip

{\bf MEMBER$\left(x\right)$\newline 
Vyhledej$\left(x\right)$\newline 
if} $\key\left(t\right)=x$ {\bf then V\'ystup}: $x\in S$ {\bf else V\'ystup}: $x\notin S$ {\bf endif
\medskip

INSERT$\left(x\right)$\newline
Vyhledej$\left(x\right)$\newline 
if} $\key\left(t\right)\ne x$ {\bf then}\newline 
\phantom{---}vytvo\v r nov\'y list $t'$, $\key\left(t'\right):=x$, $u:=\otec\left(t\right)$\newline 
\phantom{---}{\bf if} $\key\left(t\right)<x$ {\bf then}\newline 
(koment\'a\v r: $x>\max S$)\newline 
\phantom{------}$S_u\left(\rho \left(u\right)+1\right):=t'$, $H_u\left(\rho \left(u\right)\right):=\key\left(t\right)$, $\rho \left(u\right):=\rho 
\left(u\right)+1$\newline 
\phantom{---}{\bf else}\newline 
\phantom{------}najdi $i$, \v ze $S_u\left(i\right)=t$\newline 
\phantom{------}$S_u\left(\rho \left(u\right)+1\right):=S\left(\rho \left(u\right)\right)$, $j:=\rho \left(u\right)-1$\newline 
\phantom{------}{\bf while} $j\ge i$ {\bf do\newline 
\phantom{---------}$S_u\left(j+1\right):=S_u\left(j\right)$}, $H_u\left(j+1\right):=H_u\left(j\right)$, $j:=j-1$\newline 
\phantom{------}{\bf enddo\newline 
\phantom{------}$S_u\left(i\right):=t'$}, $H_u\left(i\right):=x$, $\rho \left(u\right):=\rho \left(u\right)+1$\newline 
\phantom{---}{\bf endif}\newline 
\phantom{---}$t:=u$\newline 
\phantom{---}{\bf while} $\rho \left(t\right)>b$ {\bf do \v St\v epen\'\i$\left(t\right)$ enddo\newline 
endif}
\medskip

{\bf \v St\v epen\'\i$\left(t\right)$\newline 
if} $t$ je ko\v ren stromu {\bf then}\newline 
\phantom{---}vytvo\v r nov\'y ko\v ren $u$ s jedin\'ym synem $t$\newline 
{\bf endif}\newline 
$u:=\otec\left(t\right)$, najdi $i$, \v ze $S_u\left(i\right)=t$,\newline 
vytvo\v r nov\'y vnit\v rn\'\i\ vrchol $t'$, $j:=1$\newline 
{\bf while} $j<\lfloor\frac {b+1}2\rfloor$ {\bf do}\newline 
\phantom{---}$S_{t'}\left(j\right):=S_t\left(j+\left\lceil\frac {b+1}2\right\rceil \right)$, $H_{t'}\left(j\right):=H_t\left(j+\left\lceil\frac {b+1}2\right\rceil\right)$, $j:=j+1$\newline 
{\bf enddo\newline 
$S_{t'}\left(\lfloor\frac {b+1}2\rfloor \right):=S_t\left(b+1\right)$}, $\rho \left(t\right):=\left\lceil\frac {
b+1}2\right\rceil$, $\rho \left(t'\right):=\lfloor\frac {b+1}2\rfloor$,\newline 
{\bf if} $i<\rho \left(u\right)$ {\bf then} $S_u\left(\rho \left(u\right)+1\right):=S_u\left(\rho \left(u\right)\right)$ {\bf endif\newline}
$j:=\rho \left(u\right)-1$, $\rho \left(u\right):=\rho \left(u\right)+1$,\newline 
{\bf while} $j>i$ {\bf do}\newline 
\phantom{---}$S_u\left(j+1\right):=S_u\left(j\right)$, $H_u\left(j+1\right):=H_u\left(j\right)$, $j:=j-1$\newline 
{\bf enddo\newline 
$S_u\left(i+1\right):=t'$}, $H_u\left(i+1\right):=H_u\left(i\right)$, $H_u\left(i\right):=H_t\left(\rho \left(t\right)\right)$, 
$t:=u$
\medskip

{\bf DELETE$\left(x\right)$\newline 
Vyhledej$\left(x\right)$\newline 
if} $\key\left(t\right)=x$ {\bf then}\newline 
\phantom{---}$u:=\otec\left(t\right)$, najdi $i$, \v ze $S_u\left(i\right)=t$, a $j$, \v ze $H_w\left(j\right)=x$, $k:=i$\newline 
\phantom{---}{\bf if} $w\ne u$ a $w\ne NIL$ {\bf then} $H_w\left(j\right):=H_u\left(\rho \left(u\right)-1\right)$ {\bf endif}\newline 
\phantom{---}{\bf while} $k<\rho \left(u\right)-1$ {\bf do}\newline 
\phantom{------}$H_u\left(k\right):=H_u\left(k+1\right)$, $S_u\left(k\right):=S_u\left(k+1\right)$, $k:=k+1$ \newline 
\phantom{---}{\bf enddo}\newline 
\phantom{---}{\bf if} $i\ne\rho\left(u\right)$ {\bf then} $S_u\left(\rho \left(u\right)-1\right):=S_u\left(\rho \left(u\right)\right)$ {\bf endif}\newline
\phantom{---}$\rho \left(u\right):=\rho \left(u\right)-1$, odstra\v n $t$, $t:=u$ \newline 
\phantom{---}{\bf while} $\rho \left(t\right)<a$ a $t$ nen\'\i\ ko\v ren {\bf do}\newline 
\phantom{------}$y$ je bezprost\v redn\'\i\ bratr $t$\newline 
\phantom{------}{\bf if} $\rho \left(y\right)=a$ {\bf then Spojen\'\i$\left(t,y\right)$ else P\v resun$
\left(t,y\right)$ endif}\newline 
\phantom{---}{\bf enddo\newline 
endif}
\medskip

{\bf Spojen\'\i$\left(t,y\right)$\newline 
$u:=\otec\left(t\right)$}, najdi $i$, \v ze $S_u\left(i\right)=t$, $j:=1$\newline 
{\bf if} $S_u\left(i-1\right)=y$ {\bf then} vym\v e\v n $t$ a $y$, $i:=i-1$ {\bf endif\newline 
while} $j<\rho \left(y\right)$ {\bf do}\newline 
\phantom{---}$S_t\left(\rho \left(t\right)+j\right):=S_y\left(j\right)$, $H_t\left(\rho \left(t\right)+j\right):=H_y\left(j\right)$, $j:=j+1$\newline 
{\bf enddo}\newline 
$H_t\left(\rho \left(t\right)\right):=H_u\left(i\right)$, $S_t\left(\rho \left(t\right)+\rho \left(y\right)\right):=S_y\left(\rho \left(y\right)\right)$, $
\rho \left(t\right):=\rho \left(t\right)+\rho \left(y\right)$, odstra\v n $y$\newline 
{\bf while} $i<\rho \left(u\right)-1$ {\bf do}\newline 
\phantom{---}$S_u\left(i+1\right):=S_u\left(i+2\right)$, $H_u\left(i\right):=H_u\left(i+1\right)$, $i:=i+1$\newline 
{\bf enddo}\newline 
$\rho \left(u\right):=\rho \left(u\right)-1$\newline 
{\bf if} $u$ je ko\v ren a $\rho \left(u\right)=1$ {\bf then}\newline 
\phantom{---}odstra\v n $u$\newline 
{\bf else}\newline 
\phantom{---}$t:=u$\newline 
{\bf endif}
\medskip

{\bf P\v resun$\left(t,y\right)$\newline 
$u:=\otec\left(t\right)$}, najdi $i$ takov\'e, \v ze $S_u\left(i\right)=t$\newline 
{\bf if} $S_u\left(i+1\right)=y$ {\bf then}\newline 
\phantom{---}$S_t\left(\rho \left(t\right)+1\right):=S_y\left(1\right)$, $H_t\left(\rho \left(t\right)\right):=H_u\left(i\right)$,\newline 
\phantom{---}$H_u\left(i\right):=H_y\left(1\right)$, $j:=1$\newline 
\phantom{---}{\bf while} $j<\rho \left(y\right)-1$ {\bf do}\newline 
\phantom{------}$S_y\left(j\right):=S_y\left(j+1\right)$, $H_y\left(j\right):=H_y\left(j+1\right)$, $j:=j+1$\newline 
\phantom{---}{\bf enddo}\newline 
\phantom{---}$S_y\left(\rho \left(y\right)-1\right):=S_y\left(\rho \left(y\right)\right)$, $\rho \left(t\right):=\rho \left(t\right)+1$, $\rho 
\left(y\right):=\rho \left(y\right)-1$\newline 
{\bf else}\newline
\phantom{---}$S_t\left(\rho \left(t\right)+1\right):=S_t\left(\rho \left(t\right)\right)$, $j:=\rho \left(t\right)-1$\newline 
\phantom{---}{\bf while} $j>0$ {\bf do}\newline 
\phantom{------}$S_t\left(j+1\right):=S_t\left(j\right)$, $H_t\left(j+1\right):=H_t\left(j\right)$, $j:=j-1$\newline 
\phantom{---}{\bf enddo}\newline 
\phantom{---}$\rho \left(t\right):=\rho \left(t\right)+1$, $S_t\left(1\right):=S_y\left(\rho \left(y\right)\right)$, $H_t\left(1\right):=H_u\left(i-
1\right)$,\newline 
\phantom{---}$H_u\left(i-1\right):=H_y\left(\rho \left(y\right)-1\right)$, $\rho \left(y\right):=\rho \left(y\right)-1$\newline 
{\bf endif}
\medskip

{\bf MIN\newline 
$t:=$}ko\v ren stromu\newline 
{\bf while} $t$ nen\'\i\ list {\bf do} $t:=S_t\left(1\right)$ {\bf enddo}\newline 
$\key\left(t\right)$ je nejmen\v s\'\i\ prvek $S$
\medskip

{\bf MAX\newline 
$t:=$}ko\v ren stromu\newline 
{\bf while} $t$ nen\'\i\ list {\bf do} $t:=S_t\left(\rho \left(t\right)\right)$ {\bf enddo}\newline 
$\key\left(t\right)$ je nejv\v et\v s\'\i\ prvek $S$
\medskip

{\bf JOIN2$\left(T_1,T_2\right)$}\newline
P\v redpoklad $T_i$ je $\left(a,b\right)$-strom reprezentuj\'\i c\'\i\ 
mno\v zinu $S_i$ pro $i=1,2$, kter\'e spl\v nuj\'\i\ $\max S_1<\min 
S_2$ 
(tento p\v redpoklad je siln\v ej\v s\'\i\ ne\v z po\v zadavek, \v ze $
S_1$ a 
$S_2$ jsou disjunktn\'\i , ale algoritmus nekontroluje jeho spln\v en\'\i )\newline 
{\bf if} v\'y\v ska $T_1$ je v\v et\v s\'\i\ nebo rovna v\'y\v sce $T_2$ {\bf then}\newline 
\phantom{---}$t:=$ko\v ren $T_1$, $k:=v\left(T_1\right)-v\left(T_2\right)$\newline 
\phantom{---}{\bf while} $k>0$ {\bf do} $t:=S_t\left(\rho \left(t\right)\right)$, $k:=k-1$ {\bf enddo}\newline \phantom{---}{\bf Spojen\'\i}$\left(t,\text{ ko\v ren }T_2\right)$, $t:=\otec\left(t\right)$\newline 
\phantom{---}{\bf while} $\rho \left(t\right)>b$ {\bf do \v St\v epen\'\i$\left(t\right)$ enddo\newline 
else}\newline 
\phantom{---}$t:=$ko\v ren $T_2$, $k:=v\left(T_2\right)-v\left(T_1\right)$ \newline 
\phantom{---}{\bf while} $k>0$ {\bf do} $t:=S_t\left(1\right)$, $k:=k-1$ {\bf enddo}\newline \phantom{---}{\bf Spojen\'\i}$\left(t,\text{ ko\v ren }T_1\right)$, $t:=\otec\left(t\right)$\newline 
\phantom{---}{\bf while} $\rho \left(t\right)>b$ {\bf do \v St\v epen\'\i$\left(t\right)$ enddo\newline 
endif}
\medskip

{\bf SPLIT$\left(T,x\right)$\newline 
$Z_1$}, $Z_2$ pr\'azdn\'e z\'asobn\'\i ky, $t:=$ko\v ren $T$\newline 
{\bf while} $t$ nen\'\i\ list {\bf do}\newline
\phantom{---}$i:=1$\newline 
\phantom{---}{\bf while} $H_t\left(i\right)<x$ a $i<\rho \left(t\right)$ {\bf do} $i:=i+1$ {\bf enddo}\newline \phantom{---}$t:=S_t\left(i\right)$\newline 
\phantom{---}{\bf if} $i=2$ {\bf then} vlo\v z podstrom vrcholu $S_t\left(1\right)$ do $Z_1$ 
{\bf endif}\newline 
\phantom{---}{\bf if} $i>2$ {\bf then}\newline 
\phantom{------}vytvo\v r nov\'y vrchol $t_1$, $\rho \left(t_1\right)=i-1$,\newline 
\phantom{------}{\bf for every} $j=1,2,\dots,i-2$ {\bf do}\newline 
\phantom{---------}$S_{t_1}\left(j\right):=S_t\left(j\right)$, $H_{t_1}\left(j\right):=H_t\left(j\right)$\newline 
\phantom{------}{\bf enddo}\newline 
\phantom{------}$S_{t_1}\left(i-1\right):=S_t\left(i-1\right)$, vlo\v z podstrom vrcholu $t_1$ do $Z_1$\newline 
\phantom{---}{\bf endif}\newline 
\phantom{---}{\bf if} $i=\rho \left(t\right)-1$ {\bf then}\newline 
\phantom{------}vlo\v z podstrom $S_t\left(\rho \left(t\right)\right)$ do $Z_2$ 
\phantom{---}{\bf endif}\newline 
\phantom{---}{\bf if} $i<\rho \left(t\right)-1$ {\bf then}\newline 
\phantom{------}vytvo\v r nov\'y vrchol $t_2$, $\rho \left(t_2\right):=\rho \left(t\right)-i$\newline 
\phantom{------}{\bf for every} $j=1,2,\dots,\rho \left(t\right)-i-1$ {\bf do}\newline 
\phantom{---------}$S_{t_2}\left(j\right):=S_t\left(i+j\right)$, $H_{t_2}\left(j\right):=H_t\left(i+j\right)$\newline 
\phantom{------}{\bf enddo}\newline 
\phantom{------}$S_{t_2}\left(\rho \left(t\right)-i\right):=S_t\left(\rho \left(t\right)\right)$, vlo\v z podstrom $t_2$ do $Z_2$ \newline 
\phantom{---}{\bf endif\newline 
enddo\newline 
if} $\key\left(t\right)=x$ {\bf then}\newline 
\phantom{---}{\bf V\'ystup}: $x\in S$\newline 
{\bf else}\newline 
\phantom{---}{\bf V\'ystup}: $x\notin S$\newline 
\phantom{---}{\bf if} $\key\left(t\right)<x$ {\bf then}\newline 
\phantom{------}vlo\v z podstrom vrcholu $t$ do $Z_1$\newline 
\phantom{---}{\bf else}\newline 
\phantom{------}vlo\v z podstrom vrcholu $t$ do $Z_2$\newline 
\phantom{---}{\bf endif\newline 
endif}\newline 
$T_1:=\text{ vrchol }Z_1$, odstra\v n $T_1$ ze $Z_1$\newline 
{\bf while} $Z_1\ne\emptyset$ {\bf do}\newline 
\phantom{---}$T':=$vrchol $Z_1$, odstra\v n $T'$ ze $Z_1$, $T_1:=${\bf JOIN}$\left(T',T_1\right)$\newline enddo\newline 
$T_2:=\text{ vrchol }Z_2$, odstra\v n $T_2$ ze $Z_2$\newline 
{\bf while} $Z_2\ne\emptyset$ {\bf do}\newline 
$T':=\text{ vrchol }Z_2$, odstra\v n $T'$ ze $Z_2$, $T_2:=${\bf JOIN$\left(T_2,T'\right)$\newline 
enddo}
\medskip

\flushpar Pozn\'amky k algoritm\accent23um. \newline 
Odkaz na otce vrcholu: bu\v d je v ka\v zd\'em vrcholu $v$ 
stromu $T$ p\v r\'\i mo odkaz na $\otec\left(v\right)$, nebo se v procedu\v re 
{\bf Vyhledej} vkl\'ada\-j\'\i\ vrcholy do z\'asobn\'\i ku a $\otec
\left(v\right)$ 
je vrchol v z\'asobn\'\i ku p\v red vrcholem $v$.
\smallskip

\flushpar P\v ri operaci {\bf SPLIT} se z\'asobn\'\i ky pou\v z\'\i vaj\'\i\ 
jednopr\accent23uchodov\v e -- nejprve se napln\'\i\ a v t\'eto \v c\'asti 
algoritmu se nepou\v zije operace {\bf pop}, pak se vypr\'azdn\'\i\ a v t\'eto 
f\'azi se nepou\v z\'\i v\'a operace {\bf push}. V okam\v ziku, kdy\v z jsou 
z\'asobn\'\i ky napln\v en\'e, plat\'\i :
\roster
\item"{}"
v z\'asobn\'\i c\'\i ch jsou ulo\v zeny $\left(a,b\right)$-stromy reprezentuj\'\i c\'\i\ 
podmno\v ziny $S$;
\item"{}" 
kdy\v z $\left(a,b\right)$-stromy $T_i$ a $T_{i+1}$ reprezentuj\'\i\ mno\v ziny $S_
i$ a $S_{i+1}$ a jsou v z\'a\-sobn\'\i ku $Z_1$ (nebo $Z_2$) a strom $T_{i+1}$ n\'asleduje 
po stromu $T_{i}$, pak plat\'\i\ $\max S_i<\min S_{i+1}<x$ (nebo 
$\min S_i>\max S_{i+1}>x$) a v\'y\v ska $T_i$ je v\v et\v s\'\i\ nebo 
rovna v\'y\v sce $T_{i+1}$;
\item"{}"
kdy\v z $T_i$ a $T_{i+1}$ jsou dva po sob\v e n\'asleduj\'\i c\'\i\ $
\left(a,b\right)$-stromy v 
z\'asobn\'\i ku $Z_j$ pro $j=1,2$, kter\'e maj\'\i\ stejnou v\'y\v sku, pak 
n\'asleduj\'\i c\'\i\ strom v z\'asobn\'\i ku $Z_j$ m\'a ost\v re men\v s\'\i\ v\'y\v sku.
\endroster
\flushpar Toto plyne z prvn\'\i\ f\'aze algoritmu operace {\bf SPLIT} a zaji\v s\v tuje 
korektnost druh\'e f\'aze algoritmu. 
\medskip

\flushpar D\'ale si v\v simn\v eme, \v ze podprocedury {\bf \v St\v epen\'\i }, {\bf Spojen\'\i} a {\bf P\v re\-sun }
vy\v zaduj\'\i\ \v cas $O\left(1\right)$, a proto algoritmy pro operace {\bf MEMBER}, 
{\bf INSERT}, {\bf DELETE}, {\bf MIN}, {\bf MAX}, {\bf JOIN2} a pro prvn\'\i\ f\'azi algoritmu 
{\bf SPLIT} vy\v zaduj\'\i\ \v cas $O\left(1\right)$ pro pr\'aci v dan\'e hladin\v e.  Proto\v ze 
hladin je nejv\'y\v se $\log_a|S|$, m\accent23u\v zeme shrnout: 


\proclaim{V\v eta}Algoritmy pro operace {\bf MEMBER}, {\bf INSERT}, 
{\bf DE\-LETE}, {\bf MIN}, {\bf MAX}, {\bf JOIN2} a {\bf SPLIT} v $
\left(a,b\right)$-stromech vy\-\v zaduj\'\i\ 
v nejhor\v s\'\i m p\v r\'\i pad\v e \v cas $O\left(\log_a|S|\right)$, kde $
S$ je 
re\-pre\-zen\-tovan\'a mno\v zina.  
\endproclaim


\flushpar Je t\v reba je\v st\v e odhadnout spot\v rebovan\'y \v cas ve druh\'e f\'azi 
algoritmu pro operaci {\bf SPLIT}. Nejprve si v\v simn\v eme, \v ze 
algoritmus {\bf JOIN2$\left(T_1,T_2\right)$} vy\v zaduje ve skute\v cnosti jen \v cas 
rovn\'y 
$O\left(\text{\rm rozd\'\i l v\'y\v sek strom\accent23u }T_1\text{\rm \ a }
T_2\right)$. Kdy\v z po napln\v en\'\i\ 
z\'asobn\'\i k $Z_j$ pro $j=1,2$ obsahuje stromy $U_1,U_2,\dots,U_
k$ v tomto 
po\v rad\'\i , pak $k\le 2\log_a|S|$ a vypr\'azdn\v en\'\i\ z\'asobn\'\i ku $
Z_j$ 
vy\v zaduje 
\v cas $O\left(\sum_{i=1}^{k-1}\left(u_i-u_{i+1}+1\right)\right)=O\left(u_1+k\right)$, kde $u_i$ je v\'y\v ska stromu $
U_i$ pro 
$i=1,2,\dots,k$. Proto\v ze v\'y\v ska stromu $U_1$ je nejv\'y\v se rovna v\'y\v sce 
stromu $T$, dost\'av\'ame, \v ze druh\'a f\'aze algoritmu {\bf SPLIT} vy\v zaduje 
\v cas $O\left(\log_a|S|\right)$ a d\accent23ukaz je kompletn\'\i . 
\medskip

\flushpar Nyn\'\i\ pop\'\i\v seme algoritmus pro operaci {\bf ord$
\left(k\right)$}. Tato 
operace se \v casto naz\'yv\'a $k$-t\'a po\v r\'adko\-v\'a statistika
Tato operace nen\'\i\ podporov\'ana navr\v zenou 
strukturou, pro jej\'\i\ efektivn\'\i\ implementaci mus\'\i me 
roz\v s\'\i\v rit strukturu vnit\v rn\'\i ho vrcholu $v$ o pole\newline 
$P_v\left(1..\rho \left(v\right)-1\right)$, kde $P_v\left(i\right)$ je po\v cet prvk\accent23u $S$ reprezentovan\'ych 
v podstromu $i$-t\'eho syna vrcholu $v$.
\smallskip
 
\flushpar Udr\v zovat pole $P_v$ v aktu\'aln\'\i m stavu znamen\'a p\v ri \'usp\v e\v sn\'em 
pro\-veden\'\i\ aktualiza\v cn\'\i\ operace proj\'\i t cestu z vrcholu do ko\v re\-ne 
a aktualizovat pole $P$. Uvedeme algoritmus pro 
naleze\-n\'\i\ $k$-t\'e po\v r\'adkov\'e statistiky.
\medskip

{\bf ord$\left(k\right)$\newline 
if} $k>|S|$ {\bf then} neexistuje $k$-t\'y nejmen\v s\'\i\ prvek, konec 
{\bf endif\newline 
$t:=$}ko\v ren stromu\newline 
{\bf while} $t$ nen\'\i\ list {\bf do}\newline 
\phantom{---}$i:=1$\newline 
\phantom{---}{\bf while} $k>P_t\left(i\right)$ a $i<\rho \left(t\right)$ {\bf do}\newline 
\phantom{------}$k:=k-P_t\left(i\right)$, $i:=i+1$\newline 
\phantom{---}{\bf enddo}\newline 
\phantom{---}$t:=S_t\left(i\right)$\newline 
{\bf enddo}\newline 
$\key\left(t\right)$ je hledan\'y $k$-t\'y nejmen\v s\'\i\ prvek
skip

\flushpar Invariant algoritmu: V ka\v zd\'em okam\v ziku plat\'\i , 
\v ze p\accent23uvodn\'\i\ $k$ se rovn\'a aktu\'aln\'\i\ $k+$po\v cet prvk\accent23u z 
$S$, kter\'e jsou v podstromech vrchol\accent23u stromu, kter\'e v 
lexikografick\'em uspo\v r\'ad\'an\'\i\ p\v redch\'azej\'\i\ $i$-t\'emu synu 
vrcholu $t$. Korektnost algoritmu plyne z tohoto 
invariantu.


\proclaim{V\v eta}Algoritmy pro operace {\bf MEMBER}, {\bf INSERT}, 
{\bf DE\-LETE}, {\bf MIN}, {\bf MAX}, {\bf SPLIT}, {\bf JOIN2} a {\bf ord$
\left(k\right)$} pro v\v sechna $k$ v 
roz\v s\'\i\v ren\'e struktu\v re $\left(a,b\right)$-stromu vy\-\v za\-duj\'\i\ v 
nejhor\v s\'\i m p\v r\'\i\-pa\-d\v e \v cas $O\left(\log|S|\right)$, kde $S$ je reprezentovan\'a 
mno\v zina.  
\endproclaim


\flushpar $\left(a,b\right)$-stromy se pou\v z\'\i vaj\'\i\ jak v intern\'\i\ tak v 
extern\'\i\ pam\v eti. Jak\'e hodnoty $a$ a $b$ je vhodn\'e 
pou\v z\'\i vat?\newline 
Pro intern\'\i\ pam\v e\v t jsou doporu\v cen\'e hodnoty $a=2$, $
b=4$ 
nebo $a=3$ a $b=6$.\newline 
Pro extern\'\i\ pam\v e\v t jsou doporu\v cen\'e hodnoty $a\approx 
100$, 
$b=2a$.
\medskip

\flushpar Kdy\v z je mno\v zina reprezentovan\'a $\left(a,b\right)$-stromem 
ulo\v zena na serveru a m\'a k n\'\i\ p\v r\'\i stup v\'\i ce 
u\v zivatel\accent23u, vznik\'a probl\'em s aktualiza\v cn\'\i mi 
operacemi. Tyto operace m\v en\'\i\ strukturu $\left(a,b\right)$-stromu a 
v d\accent23usledku toho se v n\v em jin\'y u\v zivatel 
m\accent23u\v ze ztratit. Tento probl\'em se d\'a  
\v re\v sit tak, \v ze p\v ri aktu\-aliza\v cn\'\i ch operac\'\i ch 
se uzav\v re cel\'y strom.
\medskip 

\flushpar Nev\'yhoda: ostatn\'\i\ u\v zivatel\'e  
do n\v eho nemaj\'\i\ p\v r\'\i stup a nemohou pracovat. Tzv. paraleln\'\i\ 
implementace operac\'\i\ {\bf INSERT} a {\bf DELETE} nab\'\i z\'\i\ jin\'e, 
efektivn\v ej\v s\'\i\ \v re\v sen\'\i .
\medskip

\flushpar P\v redpoklad: $b\ge 2a$. \newline 
P\v ri operaci {\bf INSERT} jsou ve vyhled\'avac\'\i\ f\'azi v\v zdy 
uzav\v reny vrcholy $t$, $\otec\left(t\right)$ a synov\'e vrcholu $t$. 
Algoritmus zjist\'\i , ve kter\'em synu vrcholu $t$ m\'a 
pokra\v covat, a pak, kdy\v z $\rho \left(t\right)=b$, provede {\bf \v St\v epen\'\i\ }
(proto je nutn\v e $b\ge 2a$, abychom po t\'eto ope\-raci m\v eli 
zase $\left(a,b\right)$-strom). V algoritmu pak odpadne vyva\-\v zo\-vac\'\i\ 
\v c\'ast (tj. {\bf \v St\v epen\'\i} p\v ri cest\v e vzh\accent23uru ke 
ko\v reni).
\medskip

\flushpar P\v ri operaci {\bf DELETE} jsou ve vyhled\'avac\'\i\ f\'azi uzav\v reny 
vrcholy $t$, $\otec\left(t\right)$, bezprost\v redn\'\i\ bratr $y$ vrcholu $
t$ a 
jejich synov\'e. 
Kdy\v z $\rho \left(t\right)=a$, pak po najiti vrcholu, kde se bude 
pokra\v covat, se provede bu\v d {\bf P\v resun} (kdy\v z $\rho \left(
y\right)>a$) nebo 
{\bf Spojen\'\i} (kdy\v z $\rho \left(y\right)=a$). Stejn\v e jako p\v ri operaci 
{\bf INSERT} se vynech\'a vyva\-\v zo\-vac\'\i\ \v c\'ast 
uzav\'\i raj\'\i c\'\i\ p\accent23uvodn\'\i\ algoritmus.
\medskip

\flushpar Tato \'uprava vy\v zaduje sice v\'\i ce {\bf \v St\v epen\'\i }, {\bf Spojen\'\i} a
{\bf P\v resun\accent23u}, ale asymptoticky vych\'az\'\i\ \v cas stejn\'y 
(jen je v\v et\v s\'\i\ multiplikativn\'\i\ konstanta). Doporu\v cen\'e 
hodnoty $a$ a $b$ jsou $a\approx 100$ a $b=2a+2$ p\v ri ulo\v zen\'\i\ na serveru 
v extern\'\i\ pam\v eti, ve vnit\v rn\'\i\ pam\v eti 
se doporu\v cuje $a=2$, $b=6$.
\medskip

\flushpar Operace {\bf JOIN2} lze tak\'e paralelizovat, ale operaci {\bf SPLIT  }
paralelizovat nelze.
\medskip

\flushpar$\left(a,b\right)$-stromy d\'avaj\'\i\ tak\'e zaj\'\i mav\'e aplikace pro 
t\v r\'\i dic\'\i\ 
algoritmy. Pou\v zit\'\i\ $\left(a,b\right)$-strom\accent23u pro set\v r\'\i d\v en\'\i\ 
n\'ahodn\'e posloupnosti nen\'\i\ vhodn\'e, re\v zie na udr\v zov\'an\'\i\ 
struktury $\left(a,b\right)$-stromu vede k tomu, \v ze multiplikativn\'\i\ 
konstanta by byla o hodn\v e v\v et\v s\'\i\ ne\v z u klasick\'ych 
t\v r\'\i dic\'\i ch algoritm\accent23u. Tak\'e ulo\v zen\'\i\ $\left(
a,b\right)$-stromu 
vy\v zaduje v\'\i ce pam\v eti ne\v z je pot\v reba pro klasick\'e 
algoritmy. Situace se podstatn\v e zm\v en\'\i , kdy\v z vstupn\'\i\ 
posloupnost je p\v red\-t\v r\'\i\-d\v en\'a a je ji t\v reba jen dot\v r\'\i dit. 
Klasick\'e algoritmy v\v et\v sinou nejsou schopn\'e vyu\v z\'\i t faktu, \v ze 
posloupnost je p\v redt\v r\'\i d\v en\'a, a jejich \v casov\'a n\'aro\v cnost je 
prakticky stejn\'a (n\v ekdy i hor\v s\'\i ) jako u n\'ahodn\'e 
posloupnosti. Na rozd\'\i l od nich algoritmus $A$-{\bf sort }
zalo\v zen\'y na $\left(a,b\right)$-stromech je schopen p\v redt\v r\'\i d\v enost 
vyu\v z\'\i t a m\'a na p\v redt\v r\'\i d\v en\'ych posloupnostech lep\v s\'\i\ 
v\'ysledky ne\v z klasick\'e algoritmy.
\medskip

\flushpar Modifikace $\left(a,b\right)$-strom\accent23u pro algoritmus 
$A$-{\bf sort}. M\'ame $\left(a,b\right)$-strom reprezentuj\'\i c\'\i\ vstupn\'\i\ 
posloupnost, je d\'an ukazatel $\Prv$ na prvn\'\i\ list, listy 
$\left(a,b\right)$-stromu jsou propojeny do seznamu v rostouc\'\i m 
lexikografick\'em po\v rad\'\i\ (ukazatel na n\'asleduj\'\i c\'\i\ prvek je 
$\Nasl$) a je d\'ana cesta z prvn\'\i ho listu do ko\v rene (to 
znamen\'a, \v ze na cest\v e z prvn\'\i ho listu do ko\v rene zn\'ame 
pro ka\v zd\'y vrchol $v$ jeho otce). Nyn\'\i\ uvedeme algoritmus 
$A$-{\bf sort}. 
skip

$A$-{\bf sort$\left(x_1,x_2,\dots,x_n\right)$\newline 
$i:=n-1$}, vytvo\v r jednoprvkovy strom s vrcholem $t$\newline 
$\key\left(t\right):=x_n$, $\Prv:=t$\newline 
{\bf while} $i\ge 1$ {\bf do $A$-Insert$\left(x_i\right)$}, $i:=i-1$ {\bf enddo}\newline 
$y_1:=\key\left(\Prv\right)$\newline 
{\bf while} $i\le n$ {\bf do\newline 
\phantom{---}$y_i:=\key\left(t\right)$}, $i:=i+1$, $t:=\Nasl\left(t\right)$\newline 
{\bf enddo\newline 
V\'ystup}: $\left(y_1,y_2,\dots,y_n\right)$ set\v r\'\i d\v en\'a posloupnost $
\left(x_1,x_2,\dots,x_n\right)$
\medskip

$A$-{\bf Insert$\left(x\right)$\newline 
$t:=\Prv$\newline
while} $t\ne\text{\rm ko\v ren }T$ a $H_t\left(1\right)<x$ {\bf do $t:=\otec\left(t\right)$ enddo\newline 
while} $t\ne\text{\rm list}$ {\bf do}\newline 
\phantom{---}$i:=1$\newline 
\phantom{---}{\bf while} $H_t\left(i\right)<x$ a $i<\rho \left(t\right)$ {\bf do} $i:=i+1$ {\bf enddo}\newline 
\phantom{---}{\bf if} $i>1$ {\bf then} $v:=S_t\left(i-1\right)$ {\bf else} $v:=S_t\left(\rho \left(t\right)\right)$ {\bf endif}\newline 
\phantom{---}$t:=S_t\left(i\right)$\newline 
{\bf enddo\newline 
if} $\key\left(t\right)\ne x$ {\bf then\newline }
\phantom{---}vytvo\v r nov\'y list $t'$, $\key\left(t'\right)=x$,\newline
\phantom{---}{\bf if} $t$ je ko\v ren {\bf then}\newline 
\phantom{------}vytvo\v r nov\'y ko\v ren $u$, $\rho \left(u\right):=2$\newline 
\phantom{------}{\bf if} $\key\left(t\right)>x$ {\bf then}\newline 
\phantom{---------}$H_u\left(1\right):=x,$ $S_u\left(1\right):=t'$, $S_u\left(2\right):=t$,\newline
\phantom{---------}$\Prv:=t'$, $\Nasl\left(t'\right):=t$, $\Nasl\left(t\right):=NIL$\newline 
\phantom{------}{\bf else}\newline 
\phantom{---------}$H_u\left(1\right):=\key\left(t\right)$, $S_u\left(1\right):=t$, $S_u\left(2\right):=t'$\newline
\phantom{---------}$\Prv:=t$, $\Nasl\left(t\right):=t'$, $\Nasl\left(t'\right):=NIL$\newline 
\phantom{------}{\bf endif}\newline 
\phantom{---}{\bf else}\newline 
\phantom{------}$u:=\otec\left(t\right)$\newline 
\phantom{------}{\bf if} $\key\left(t\right)<x$ {\bf then}\newline  
(koment\'a\v r: $x>\max S$)\newline 
\phantom{---------}$S_u\left(\rho \left(u\right)+1\right):=t'$, $H_u\left(\rho\left(u\right)\right):=\key\left(t\right)$, $\rho\left(u\right):=\rho\left(u\right)+1$\newline 
\phantom{---------}$\Nasl\left(t\right):=t'$, $\Nasl\left(t'\right):=NIL$\newline 
\phantom{------}{\bf else}\newline 
\phantom{---------}najdi $i$, \v ze $S_u\left(i\right)=t$, $S_u\left(\rho\left(u\right)+1\right):=S\left(\rho\left(u\right)\right)$,\newline 
\phantom{---------}$j:=\rho \left(u\right)-1$, $\Nasl\left(v\right):=t'$, $\Nasl\left(t'\right):=t$\newline 
\phantom{---------}{\bf while} $j\ge i$ {\bf do\newline 
\phantom{------------}$S_u\left(j+1\right):=S_u\left(j\right)$}, $H_u\left(j+1\right):=H_u\left(j\right)$, $j:=j-1$\newline 
\phantom{---------}{\bf enddo}\newline 
\phantom{---------}$S_u\left(i\right):=t'$, $H_u\left(i\right):=x$, $\rho \left(u\right):=\rho \left(u\right)+1$,\newline 
\phantom{---------}{\bf if} $t=\Prv$ {\bf then} $\Prv:=t'$ {\bf endif}\newline 
\phantom{------}{\bf endif}\newline 
\phantom{------}$t:=u$\newline 
\phantom{------}{\bf while} $\rho \left(t\right)>b$ {\bf do \v St\v epen\'\i$\left(t\right)$ enddo}\newline 
\phantom{---}{\bf endif\newline 
endif}
\medskip

\flushpar Korektnost algoritmu plyne z faktu, \v ze $\key$ je 
izomorfismus uspo\-\v r\'a\-d\'an\'\i\ a seznam list\accent23u je v 
rostouc\'\i m po\v rad\'\i . Proto\v ze $v$ je v\v zdy bez\-prost\-\v redn\'\i\ 
p\v redch\accent23udce $t$, je seznam korektn\v e definov\'an. 
U\-ka\-zatel $\otec\left(t\right)$ je d\'an na cest\v e z vrcholu $\Prv$ do 
ko\v rene, pro ostatn\'\i\ vrcholy se \v re\v s\'\i\ stejn\'ym 
zp\accent23usobem jako pro $\left(a,b\right)$-stromy.
\medskip

\flushpar Slo\v zitost algoritmu: Algoritmus $A$-{\bf sort} vy\v zaduje v\'\i ce 
\v casu i v\'\i ce pam\v eti ne\v z klasick\'e t\v r\'\i dic\'\i\ algoritmy, ale jejich 
asymptotick\'a slo\v zitost je stejn\'a. Jeho v\'yhoda je v pou\v zit\'\i\ 
na p\v red\-t\v r\'\i\-d\v en\'e posloupnosti. M\v ejme posloupnost $
\left(x_1,x_2,\dots,x_n\right)$ 
prvk\accent23u z to\-t\'aln\v e uspo\v r\'adan\'eho univerza $U$ a definujme 
$$F=|\left\{\left(i,j\right)\mid i<j,\,x_j<x_i\right\}|.$$
Z\v rejm\v e $F=0$, pr\'av\v e kdy\v z posloupnost 
$\left(x_1,x_2,\dots,x_n\right)$ je set\v r\'\i\-d\v en\'a. D\'ale $0\le F\le\binom 
n2$ a $F=\binom n2$, pr\'av\v e 
kdy\v z je posloupnost $\left(x_1,x_2,\dots,x_n\right)$ klesaj\'\i c\'\i . To vede k tomu 
br\'at $F$ jako m\'\i ru p\v redt\v r\'\i d\v enosti posloupnosti. Spo\v c\'\i t\'ame 
slo\v zitost algoritmu $A$-{\bf sort} v z\'avislosti na $n$ a $F$
\medskip

\flushpar Z\v rejm\v e algoritmus $A$-{\bf sort v} nejhor\v s\'\i m p\v r\'\i pad\v e vy\v zaduje \v cas, 
kter\'y pot\v rebuje $A$-{\bf Insert}, plus $O\left(n\right)$.  Algoritmus $
A$-{\bf Insert$\left(x\right)$ }
vy\v zaduje \v cas pot\v rebn\'y na nalezen\'\i\ m\'\i sta, kam vlo\v zit $
x$, plus 
$O\left(\text{\rm po\v cet vol\'an\'\i\ {\bf \v St\v epen\'\i}}\right)$.  Proto\v ze ka\v zd\'y b\v eh procedury 
{\bf \v St\v e\-pen\'\i} vytvo\v ril jeden vnit\v rn\'\i\ vrchol $
\left(a,b\right)$-stromu a proto\v ze $a\ge 2$ 
a $\left(a,b\right)$-strom po skon\v cen\'\i\ vol\'an\'\i\ $A$-{\bf Insert} m\'a $
n$ list\accent23u, 
je vnit\v rn\'\i ch vrchol\accent23u $\left(a,b\right)$-stromu $<n$.  Proto v\v sechny 
b\v ehy procedury $A$-{\bf Insert} vy\v zaduj\'\i\ \v cas na nalezen\'\i\ m\'\i st 
jednotliv\'ych prvk\accent23u plus $O\left(n\right)$.  Kdy\v z procedura 
$A$-{\bf Insert$\left(x\right)$} p\v ri hled\'an\'\i\ m\'\i sta pro prvek $x$ skon\v cila ve v\'y\v sce $
h$ 
(tj.  prvn\'\i\ cyklus se $h$-kr\'at opakoval), pak nalezen\'\i\ m\'\i sta pro 
prvek $x$ vy\v zadovalo \v cas $O\left(h\right)$.  V\v sechny prvky reprezentovan\'e 
$\left(a,b\right)$-stromem pod prvn\'\i m vrcholem ve v\'y\v sce $h-1$ jsou men\v s\'\i\ 
ne\v z $x$ a je jich alespo\v n $a^{h-1}$.  Kdy\v z $x=x_i$, pak po\v cet 
prvk\accent23u reprezentovan\'ych $\left(a,b\right)$-stromem p\v ri b\v ehu 
procedury $A$-{\bf Insert$\left(x\right)$}, kter\'e jsou men\v s\'\i\ ne\v z $
x$, je po\v cet $j$ takov\'ych, \v ze $i<j$ a $x_j<x_i$.
Ozna\v cme $f_i$ tento po\v cet. Pak plat\'\i\ 
$$a^{h-1}\le f_i\implies h-1\le\log_af_i\implies h\in O\left(\log f_i\right).$$
\medskip

\flushpar Proto v nejhor\v s\'\i m p\v r\'\i pad\v e \v cas pot\v rebn\'y pro nalezen\'\i\ 
pozice $x_i$ je $O\left(\log f_i\right)$. Odtud plyne, \v ze \v cas algoritmu pot\v rebn\'y k 
b\v ehu algoritmu $A$-{\bf sort} je 
$$O\left(\left(\sum_{i=1}^n\log f_i\right)+n\right).$$
\medskip

\flushpar Z\v rejm\v e $\sum_{i=1}^nf_i=F$ a nyn\'\i\ vyu\v zijeme toho, \v ze geometrick\'y 
pr\accent23um\v er je v\v zdy men\v s\'\i\ nebo roven aritmetick\'emu 
pr\accent23um\v eru, a odtud dost\'av\'ame 
$$\align\sum_{i=1}^n\log f_i=&\log\prod_{i=1}^nf_i=n\log\left(\prod_{i
=1}^nf_i\right)^{\frac 1n}\le\\
&n\log\frac {\sum_{i=1}^nf_i}n=n\log\frac Fn.\endalign$$


\proclaim{V\v eta}Algoritmus $A$-{\bf sort} na set\v r\'\i d\v en\'\i\ $
n$-\v clenn\'e 
posloupnosti vy\v zaduje v nejhor\v s\'\i m p\v r\'\i pad\v e \v cas 
$O\left(n+n\log\frac Fn\right)$, kde $F$ je m\'\i ra set\v r\'\i d\v enosti vstupn\'\i\ 
posloupnosti.
\endproclaim


\flushpar Zhodnocen\'\i : Proto\v ze $A$-{\bf sort} nepou\v z\'\i v\'a operaci 
{\bf DELETE}, doporu\v cuje se pou\v z\'\i t $\left(2,3\right)$-stromy. Kdy\v z se 
budou t\v r\'\i dit posloupnosti s m\'\i rou $F\le n\log n$, pak 
algoritmus $A$-{\bf sort} bude pot\v re\-bo\-vat v nejhor\v s\'\i m p\v r\'\i pad\v e 
\v cas $O\left(n\log\log n\right)$. Mehlhorn a Tsa\-kalidis dok\'azali, \v ze kdy\v z 
$F\le 0.02n^{1.57}$, pak algoritmus $A$-{\bf sort} je rychlej\v s\'\i\ ne\v z 
algoritmus {\bf Quicksort}.

\subhead
Propojen\'e stromy s prstem
\endsubhead

\flushpar\emph{Hladinov\v e} \emph{propojen\'y} $\left(a,b\right)$-\emph{strom} \emph{s} 
\emph{prstem} je $\left(a,b\right)$-strom, kde struktura vnit\v rn\'\i ho 
vrcholu r\accent23uzn\'eho od ko\v rene je 
roz\v s\'\i\v rena (proti klasick\'emu $\left(a,b\right)$-stromu) o 
ukazatele:\newline 
$\otec\left(v\right)$, $\levy\left(v\right)$, $\pravy\left(v\right)$, kde\newline 
$\levy\left(v\right)$ ukazuje na nejv\v et\v s\'\i\ vrchol (v lexikografic\-k\'em 
uspo\v r\'a\-d\'a\-n\'\i ) ve stejn\'e hladin\v e jako $v$, kter\'y je men\v s\'\i\ ne\v z 
$v$ (kdy\v z neexistuje, tak je to $NIL$),\newline 
$\pravy\left(v\right)$ ukazuje na nejmen\v s\'\i\ vrchol (v lexikografic\-k\'em 
uspo\v r\'a\-d\'a\-n\'\i ) ve stejn\'e hladin\v e jako $v$, kter\'y je v\v et\v s\'\i\ ne\v z 
$v$ (kdy\v z ne\-existuje, tak je to $NIL$).\newline 
Nav\'\i c je d\'an ukazatel 
$\Prst$ na n\v ekter\'y list. 
\medskip

\flushpar Zde se li\v s\'\i\ hlavn\v e vyhled\'av\'an\'\i , kter\'e je zobecn\v en\'\i m 
postupu $A$-{\bf sortu}. Za\v c\'\i n\'ame od listu $p$, na kter\'y ukazuje 
$\Prst$. Kdy\v z $x$ je men\v s\'\i\ ne\v z prvek reprezentovan\'y t\'\i mto 
listem, pak se pokra\v cuje v jeho otci $v$, a kdy\v z $p$ 
byl $i$-t\'y syn $v$, tak se pomoc\'\i\ pole $H_v$ zji\v s\v tuje, zda $
x$ 
nem\'a b\'yt reprezentov\'an v podstromu jeho $j$-t\'eho syna pro 
$j<i$. Kdy\v z ne, pokra\v cuje se  ukazatelem $\levy\left(v\right)$. 
Kdy\v z $x$ nen\'\i\ reprezentov\'an ani v jeho podstromu, 
tak se cel\'y postup opakuje o hladinu v\'y\v s (zkoum\'a se otec vrcholu). 
Kdy\v z $x$ je v\v et\v s\'\i\ ne\v z prvek reprezentovan\'y listem $
p$, 
je postup zrcadlov\v e obr\'acen\'y. Kdy\v z se nalezne 
vrchol, v jeho\v z podstromu m\'a $x$ le\v zet, pak se aplikuje od 
tohoto vrcholu (m\'\i sto od ko\v rene) procedura {\bf Vyhledej}.
\newpage

\flushpar Struktura krom\v e operac\'\i\ uspo\v r\'adan\'eho 
slovn\'\i kov\'eho prob\-l\'e\-mu 
je\v st\v e pou\v z\'\i v\'a p\v ridanou ope\-raci {\bf PRST$\left(x\right)$}, kter\'a nastav\'\i\ ukazatel 
$\Prst$ na list, kter\'y reprezentuje nejmen\v s\'\i\ prvek v\v et\v s\'\i\ 
nebo rovn\'y $x$ (pokud $x>\max S$, tak ukazatel $\Prst$ bude 
ukazovat na nejv\v et\v s\'\i\ list). Operace provedou vyhled\'an\'\i\ a 
pak pokra\v cuj\'\i\ klasick\'ym zp\accent23usobem.
\medskip

\flushpar Pou\v zit\'\i : Tato struktura je velmi v\'yhodn\'a pro 
\'ulohy, kde v\v zdy skupina po sob\v e jdouc\'\i ch ope\-rac\'\i\  
pracuje v bl\'\i zk\'em okol\'\i\ n\v eja\-k\'e\-ho $x\in U$. Pak 
vyhled\'an\'\i\ prvku je rychlej\v s\'\i\ ne\v z v klasick\'em 
$\left(a,b\right)$-stromu, viz $A$-{\bf sort}.
\medskip

\flushpar Vyva\v zovac\'\i\ operace {\bf \v St\v epen\'\i , Spojov\'an\'\i , P\v resun }
vy\v zaduj\'\i\ \v cas $O\left(1\right)$, ale ve skute\v cnosti jsou nejpomalej\v s\'\i\ 
\v c\'ast\'\i\ 
algoritm\accent23u pro operace {\bf INSERT} a {\bf DELETE}.  Omezen\'\i\ 
jejich po\v ctu vedlo k men\v s\'\i\ slo\v zitosti algoritmu $A$-{\bf sort}.  To motivovalo anal\'yzu 
jejich pou\v zit\'\i .\newline 
Libovoln\'y b\v eh algoritmu {\bf INSERT }
vol\'a podproceduru {\bf \v St\v epe\-n\'\i} nejv\'y\v se $\log\left(|
S|\right)$-kr\'at a libovoln\'y b\v eh 
algoritmu  {\bf DELETE} m\accent23u\-\v ze nejv\'y\v se $\log\left(|S|
\right)$-kr\'at zavolat podproceduru 
{\bf Spojen\'\i\ }
a nejv\'y\v se jednou podproceduru {\bf P\v resun}.  V obecn\'em 
p\v r\'\i pad\v e tyto od\-ha\-dy nejdou zlep\v sit.  Pro vhodn\'y typ 
$\left(a,b\right)$-stromu v\v sak amortizovan\'y po\v cet vyva\v zovac\'\i ch 
ope\-rac\'\i\ (za\v c\'\i n\'ame-li s  
p\accent23u\-vod\-n\v e pr\'az\-dn\'ym stromem) je konstantn\'\i .
\medskip

\flushpar Pro pevn\'e $a$ a $b$ ozna\v cme 
$$c=\min\left\{\min\left\{2a-1,\left\lceil\frac {b+1}2\right\rceil \right\}-a,b-\max\left\{2a-1,\lfloor\frac {
b+1}2\rfloor \right\}\right\}.$$
P\v ripom\'\i n\'ame, \v ze \emph{v\'y\v ska} \emph{vrcholu} v ko\v renov\'em strom\v e je 
maxim\'aln\'\i\ d\'elka cesty z n\v eho do n\v ekter\'eho listu.
\medskip

\proclaim{V\v eta}Nech\v t $b\ge 2a$ a $a\ge 2$. Nech\v t $\Cal P$ je 
posloupnost $n$ operac\'\i\ {\bf INSERT} a {\bf DELETE}, aplikujme ji na 
pr\'azdn\'y $\left(a,b\right)$-strom. Ozna\v cme\newline 
$St_h$ -- po\v cet {\bf \v St\v epen\'\i} ve v\'y\v sce $h$ p\v ri aplikaci $
\Cal P$, $St=\sum_hSt_h$;\newline 
$Sp_h$-- po\v cet {\bf Spojen\'\i} ve v\'y\v sce $h$ p\v ri aplikaci $
\Cal P$, 
$Sp=\sum_hSp_h$;\newline 
$P_h$ -- po\v cet {\bf P\v resun\accent23u} ve v\'y\v sce $h$ p\v ri aplikaci $
\Cal P$, 
$P=\sum_hP_h$.\newline 
Pak plat\'\i
\roster
\item
$$P\le n\quad \text{\rm a}\quad \left(2c-1\right)St+cSp\le n+c+\frac {c\left(n-2\right)}{a+c-1};$$
\item
$$St_h+Sp_h+P_h\le\frac {2\left(c+2\right)n}{\left(c+1\right)^h}.$$
\endroster
\endproclaim


\flushpar Z definice plyne, \v ze $c\ge 1$, a proto\v ze $a\ge 2$, z 1) dostaneme
$$St+Sp\le\frac nc+1+\frac {n-2}a\le n+1+\frac {n-2}2\le\frac {3n}
2.$$
Amortizovan\'y po\v cet vyva\v zovac\'\i ch operac\'\i\ spl\v nuje tedy 
$$\frac {P+St+Sp}n\le\frac 52.$$
\medskip

\flushpar D\accent23ukaz je zalo\v zen na bankovn\'\i m principu -- 
navrhneme kvantitativn\'\i\ ohodnocen\'\i\ $\left(a,b\right)$-stromu, 
nalezneme jeho horn\'\i\ odhad a pop\'\i\v seme, jak 
toto ohodnocen\'\i\ mohou zm\v enit vyva\v zo\-vac\'\i\ ope\-race. Srovn\'an\'\i\ 
t\v echto odhad\accent23u d\'a po\v zadovan\'y v\'ysledek.
\medskip

\flushpar M\v ejme $\left(a,b\right)$-strom $T$, pro vnit\v rn\'\i\ vrchol 
$v$ r\accent23uzn\'y od ko\v rene definuj\-me 
$b\left(v\right)=\min\left\{\rho \left(v\right)-a,b-\rho \left(v\right),c\right\}$, pro 
ko\v ren $r$ definuj\-me $b\left(r\right)=\min\left\{\rho \left(r\right)-2,b-\rho \left(r\right),c\right\}$. 
\medskip

\proclaim{Pozorov\'an\'\i}Pro vnit\v rn\'\i\ vrchol stromu $v$  
r\accent23uzn\'y od ko\v rene plat\'\i
\roster
\item
$b\left(v\right)\le c$;
\item
kdy\v z $\rho \left(v\right)=a$ nebo $\rho \left(v\right)=b$, pak $b\left(v\right)=0$;
\item
kdy\v z $\rho \left(v\right)=a-1$ nebo $\rho \left(v\right)=b+1$, pak $b\left(v\right)=-1$;
\item
kdy\v z $\rho \left(v\right)=2a-1$, pak $b\left(v\right)=c$;
\item
Kdy\v z $v'$ a $v^{\prime\prime}$ jsou dva r\accent23uzn\'e vrcholy stromu 
r\accent23uzn\'e od ko\v rene takov\'e, \v ze $\rho \left(v'\right)=\left\lceil\frac {
b+1}2\right\rceil$ a 
$\rho \left(v^{\prime\prime}\right)=\lfloor\frac {b+1}2\rfloor$, pak $b\left(v'\right)+
b\left(v^{\prime\prime}\right)\ge 2c-1$;
\item
pro ko\v ren $r$ plat\'\i\ $b\left(r\right)\le c$.
\endroster
\endproclaim

\flushpar Strom $\left(T,r\right)$ ohodnot\'\i me 
$$\align b_h\left(T\right)=&\sum \left\{b\left(v\right)\mid v\ne r\text{\rm\ vnit\v rn\'\i\ vrchol stromu }\text{\rm\ ve v\'y\v sce }
h\right\}\\
b\left(T\right)=&\sum_{h=1}^{\infty}b_h\left(T\right)+b\left(r\right).\endalign$$
\v Rekneme, \v ze $\left(T,r,v\right)$ je \emph{parci\'aln\'\i} 
$\left(a,b\right)$-\emph{strom}, kdy\v z $r$ je ko\v ren stromu, $v$ je vnit\v rn\'\i\ 
vrchol $T$ a plat\'\i :
\roster
\item"{}"
kdy\v z $v\ne r$, pak $a-1\le\rho \left(v\right)\le b+1$ a $2\le\rho \left(r\right)\le 
b$;
\item"{}"
kdy\v z $v=r$, pak $2\le\rho \left(r\right)\le b+1$;
\item"{}"
kdy\v z $t$ je vnit\v rn\'\i\ vrchol $T$ r\accent23uzn\'y od $v$ a $
r$, pak 
$$a\le\rho \left(t\right)\le b;$$
\item"{}"
v\v sechny cesty z ko\v rene $r$ do n\v ejak\'eho listu maj\'\i\ stejnou 
d\'elku.
\endroster
\medskip

\flushpar Nyn\'\i\ rozlo\v z\'\i me operace {\bf INSERT} a {\bf DELETE} do 
jednotliv\'ych akc\'\i\ se stromem a vy\v set\v r\'\i\-me vliv t\v echto 
akc\'\i\ na jeho ohodnocen\'\i . D\accent23ukazy lemmat jsou 
zalo\v zen\'e na n\'asleduj\'\i c\'\i m pozorov\'an\'\i
\medskip

\proclaim{Pozorov\'an\'\i}M\v ejme dva stromy $T$ a $T'$, kter\'e maj\'\i\ 
stejnou mno\v zinu vrchol\accent23u ve v\'y\v sce $h$. Pak plat\'\i :
\roster
\item
kdy\v z ka\v zd\'y vrchol ve v\'y\v sce $h$ m\'a stejn\'y po\v cet syn\accent23u 
v obou stromech, pak $b_h\left(T\right)=b_h\left(T'\right)$;
\item
kdy\v z v\v sechny vrcholy ve v\'y\v sce $h$ a\v z na jeden vrchol maj\'\i\ stejn\'y 
po\v cet syn\accent23u v obou stromech a po\v cet syn\accent23u u zby\-l\'e\-ho vrcholu se ve stromech $
T$ a $T'$ li\v s\'\i\ nejv\'y\v se o $1$, pak $b_h\left(T\right)\ge b_h\left(T'\right)-
1$.
\endroster
\endproclaim

\proclaim{Lemma 1}Kdy\v z $\left(T,r\right)$ je $\left(a,b\right)$-strom a kdy\v z strom $
T'$ 
vznikne z $T$ p\v rid\'an\'\i m/ubr\'an\'\i m jednoho syna vrcholu $
v$ 
ve v\'y\v sce $1$ (tj. p\v rid\'avan\'y/ub\'\i ran\'y syn je list), pak 
$\left(T',r,v\right)$ je parci\'aln\'\i\ $\left(a,b\right)$-strom a plat\'\i\ 
$$\align b_1\left(T'\right)\ge&b_1\left(T\right)-1\quad\text{\rm a }b_h\left(T'\right)=b_h\left(T\right)\text{\rm\ pro }
h>1;\\
b\left(T'\right)\ge&b\left(T\right)-1.\endalign$$
\endproclaim


\proclaim{Lemma 2}Nech\v t $\left(T,r,v\right)$ je parci\'aln\'\i\ $\left(a,b\right)$-strom, 
$\rho \left(v\right)=b+1$ a $v$ je ve v\'y\v sce $l\ge 1$. Kdy\v z $T'$ vznikne z $
T$ 
operac\'\i\ {\bf \v St\v epen\'\i$\left(v\right)$}, pak $\left(T',r,\otec\left(v\right)\right)$ je parci\'aln\'\i\ 
$\left(a,b\right)$-strom a plat\'\i :  
$$\align b_l\left(T'\right)\ge&b_l\left(T\right)+2c,\text{\rm\ }b_{l+1}\left(T'\right)\ge b_{l+1}\left(
T\right)-1\\
b_h\left(T'\right)=&b_h\left(T\right)\text{\rm\ pro }h\ne l,l+1;\quad b\left(T'\right)\ge b\left(T\right)+2c-
1.\endalign$$
\endproclaim

\proclaim{Lemma 3}Nech\v t $\left(T,r,v\right)$ je parci\'aln\'\i\ $\left(a,b\right)$-strom, 
$\rho \left(v\right)=a-1$, $v$ je ve v\'y\v sce $l\ge 1$ a $y$ je bezprost\v redn\'\i\ bratr $v$ takov\'y \v ze $\rho\left(y\right)=a$. Kdy\v z $T'$ vznikne z $T$ operac\'\i\ {\bf Spojen\'\i$\left(v,y\right)$}, pak $\left(T',r,\otec\left(v\right)\right)$ je parci\'aln\'\i\ $\left(a,b\right)$-strom a 
plat\'\i :  
$$\align b_l\left(T'\right)\ge&b_l\left(T\right)+c+1,\text{\rm\ }b_{l+1}\left(T'\right)\ge b_{l+1}
\left(T\right)-1\\
b_h\left(T'\right)=&b_h\left(T\right)\text{\rm\ pro }h\ne l,l+1;\quad b\left(T'\right)\ge b\left(T\right)+c.\endalign$$
\endproclaim


\proclaim{Lemma 4} Nech\v t $\left(T,r,v\right)$ je parci\'aln\'\i\ $\left(a,b\right)$-strom, 
$\rho \left(v\right)=a-1$, $v$ je v\'y\v sce $l\ge 1$ a $y$ je bezprost\v redn\'\i\ bratr $v$ takov\'y, \v ze $\rho\left(y\right)>a$. 
Kdy\v z $T'$ vznikne z $T$ operac\'\i\ {\bf P\v resun$\left(v,y\right)$}, 
pak $\left(T',r\right)$ je $\left(a,b\right)$-strom a plat\'\i :  
$$b_l\left(T'\right)\ge b_l\left(T\right)\text{\rm\ a }b_h\left(T'\right)=b_h\left(T\right)\text{\rm\ pro }h\ne 
l;\quad b\left(T'\right)\ge b\left(T\right).$$
\endproclaim


\flushpar Ozna\v cme $T_k$ $\left(a,b\right)$-strom vznikl\'y proveden\'\i m posloupnosti 
$\Cal P$ na pr\'azdn\'y $\left(a,b\right)$-strom. Se\v cten\'\i m p\v redchoz\'\i ch 
v\'ysledk\accent23u dost\'a\-v\'ame
\medskip

\proclaim{D\accent23usledek 5}Kdy\v z polo\v z\'\i me 
$$\gather St_0+Sp_0=\text{\rm po\v cet list\accent23u v }T_k\le n
,\quad\text{\rm pak}\\
b_h\left(T_k\right)\ge 2cSt_h+\left(c+1\right)Sp_h-St_{h-1}-Sp_{h-1}\text{\rm\ pro }h\ge 
1.\endgather$$
D\'ale $b\left(T_k\right)\ge \left(2c-1\right)St+cSp-n$, kde $n$ je d\'elka posloupnosti $
\Cal P$.
\endproclaim

\flushpar Prvn\'\i\ v\'yraz uprav\'\i me (vyu\v z\'\i v\'ame, \v ze $
c\ge 1$):
$$\align St_h+Sp_h\le&\frac {b_h\left(T_k\right)}{c+1}+\frac {St_{h-1}+Sp_{h
-1}}{c+1}\le\\
&\frac {b_h\left(T_k\right)}{c+1}+\frac {b_{h-1}\left(T_k\right)}{\left(c+1\right)^2}+\frac {St_{h
-2}+Sp_{h-2}}{\left(c+1\right)^2}\le\dots\le\\
&\sum_{i=0}^{h-1}\frac {b_{h-i}\left(T_k\right)}{\left(c+1\right)^{i+1}}+\frac n{\left(c+1\right)^
h}=\\
&\frac n{\left(c+1\right)^h}+\sum_{l=1}^hb_l\left(T_k\right)\frac {\left(c+1\right)^l}{\left(c+1\right)^{h+1}}
.\endalign$$
\medskip

\flushpar Nyn\'\i\ odhadneme shora $b\left(T_k\right)$.
\medskip

\proclaim{Lemma 6}Kdy\v z $T$ je $\left(a,b\right)$-strom s $m$ listy, pak 
$0\le b\left(T\right)\le c+\left(m-2\right)\frac c{a+c-1}$.
\endproclaim

\demo{D\accent23ukaz}Pro $0\le j<c$ ozna\v cme $m_j$ po\v cet 
vnit\v rn\'\i ch vrchol\accent23u r\accent23uzn\'ych od ko\v rene, 
kter\'e maj\'\i\ p\v resn\v e $a+j$ syn\accent23u, a $m_c$ ozna\v cme 
po\v cet vnit\v rn\'\i ch vrchol\accent23u r\accent23uzn\'ych od 
ko\v rene, kter\'e maj\'\i\ alespo\v n $a+c$ syn\accent23u. Kdy\v z 
vrchol $v$ m\'a $a+j$ syn\accent23u, pak $b_T\left(v\right)\le j$ a pro 
ka\v zd\'y vnit\v rn\'\i\ vrchol $v$ plat\'\i\ $b_T\left(v\right)\le c$. Tedy 
$b\left(T\right)\le c+\sum_{j=0}^cjm_j$. Z vlastnost\'\i\ strom\accent23u plyne 
$$\align2+\sum_{j=0}^c\left(a+j\right)m_j\le&\sum \left\{\rho \left(v\right)\mid v\text{\rm\ je vnit\v rn\'\i\ vrchol }
T\right\}=\\
&m+\sum_{j=0}^cm_j.\endalign$$
Odtud plyne
$$\sum_{j=0}^c\left(a+j-1\right)m_j\le m-2.$$
Proto\v ze $\frac j{a+j-1}\le\frac c{a+c-1}$ pro ka\v zd\'e $j$ takov\'e, \v ze $
0\le j\le c$, 
dost\'av\'ame
$$\align b\left(T\right)\le&c+\sum_{j=0}^cjm_j=c+\sum_{j=0}^c\frac j{a+j-1}\left(
a+j-1\right)m_j\le\\
&c+\frac c{a+c-1}\left(m-2\right)\endalign$$
a lemma je dok\'az\'ano. \qed
\enddemo


\flushpar Nyn\'\i\ dok\'a\v zeme tvrzen\'\i\ (1) V\v ety. Proto\v ze ka\v zd\'a operace 
{\bf DE\-LETE} pou\v zije nejv\'y\v se jednu operaci {\bf P\v resun} (a operace 
{\bf INSERT} operaci {\bf P\v resun} nepou\v z\'\i v\'a) dost\'av\'ame, \v ze 
$$P\le\text{\rm po\v cet operac\'\i\ {\bf DELETE}}\le n$$
a prvn\'\i\ nerovnost plat\'\i . 
Abychom dok\'azali druhou nerovnost, spoj\'\i me druh\'e 
tvrzen\'\i\ v D\accent23usledku 5 a Lemma 6 ($T_k$ m\'a nejv\'y\v se 
$n$ list\accent23u) 
$$\left(2c-1\right)St+cSp-n\le b\left(T_k\right)\le c+\left(n-2\right)\frac c{a+c-1}$$
Odtud plyne po\v zadovan\'a nerovnost a (1) je dok\'az\'ano.
\medskip

\flushpar D\accent23ukaz (2) vyu\v zije n\'asleduj\'\i c\'\i\ odhad.
\medskip

\proclaim{Lemma 7}Pro ka\v zd\'e $h\ge 1$ a pro ka\v zd\'y 
$\left(a,b\right)$-strom $T$ s $m$ listy plat\'\i\ 
$$\sum_{l=1}^hb_l\left(T\right)\left(c+1\right)^l\le \left(c+1\right)m.$$
\endproclaim

\demo{D\accent23ukaz}Pro $0\le j<c$ a pro libovoln\'e $h$ 
ozna\v cme $m_j\left(h\right)$ po\v cet vrchol\accent23u ve v\'y\v sce $h$ 
r\accent23uzn\'ych od ko\v rene, kter\'e maj\'\i\ p\v resn\v e $a
+j$ 
syn\accent23u, a $m_c\left(h\right)$ po\v cet vrchol\accent23u ve v\'y\v sce $
h$ 
r\accent23uzn\'ych od ko\v rene, kter\'e maj\'\i\ alespo\v n $a+c$ 
syn\accent23u. Pak m\'ame
$$\align b_h\left(T\right)\le&\sum_{j=0}^cjm_j\left(h\right)\text{\rm},\\
\sum_{j=0}^c\left(a+j\right)m_j\left(h\right)\le&\sum_{j=0}^cm_j\left(h-1\right)\text{\rm\ pro ka\v zd\'e }
h\ge 1,\endalign$$
kde dodefinov\'av\'ame $\sum_{j=0}^cm_j\left(0\right)=m$. Tyto vztahy 
pou\v zijeme v n\'asleduj\'\i c\'\i m odhadu. Plat\'\i\ 
$$\align\sum_{l=1}^hb_l\left(T\right)&\left(c+1\right)^l\le\sum_{l=1}^h\left[\left(c+1\right)^l
\left(\sum_{j=0}^cjm_j\left(l\right)\right)\right]\le\\
&\sum_{l=1}^h\left[\left(c+1\right)^l\left(\sum_{j=0}^cm_j\left(l-1\right)-a\sum_{j=0}^cm_
j\left(l\right)\right)\right]=\\
&\left(c+1\right)\sum_{j=0}^cm_j\left(0\right)-\left(c+1\right)^ha\sum_{j=0}^cm_j\left(h\right)+\\
&\quad\sum_{l=1}^{h-1}\left(c+1\right)^{l+1}\left(\sum_{j=0}^cm_j\left(l\right)-\frac a{
c+1}\sum_{j=0}^cm_j\left(l\right)\right)\le\\
&\left(c+1\right)m,\endalign$$
kde rovnost jsme z\'\i skali p\v rerovn\'an\'\i m s\v c\'\i tanc\accent23u 
tak, aby v\'yrazy $\sum_{j=0}^cm_j\left(l\right)$ byly u sebe, a posledn\'\i\ 
nerovnost plyne z toho, \v ze $\frac a{c+1}\ge 1$, a tedy druh\'y s\v c\'\i tanec 
v p\v redchoz\'\i m v\'yrazu nen\'\i\ kladn\'y. \qed
\enddemo
 
\flushpar Zkombinujeme odhad $St_h+Sp_h$ s Lemmatem 7 a 
dostaneme
$$\align St_h+Sp_h\le&\frac n{\left(c+1\right)^h}+\sum_{l=1}^hb_l\left(T_k\right)\frac {
\left(c+1\right)^l}{\left(c+1\right)^{h+1}}\le\\
&\frac n{\left(c+1\right)^h}+\frac {n\left(c+1\right)}{\left(c+1\right)^{h+1}}=\frac {2n}{\left(c+1\right)^h}
.\endalign$$
Proto\v ze $P_h\le Sp_{h-1}-Sp_h\le St_{h-1}+Sp_{h-1}\le\frac {2n}{
\left(c+1\right)^{h-1}}$ 
dost\'av\'ame, \v ze 
$$\align St_h+Sp_h+P_h\le&\frac {2n}{\left(c+1\right)^h}+\frac {2n}{\left(c+1\right)^{h
-1}}=\frac {2n+2n\left(c+1\right)}{\left(c+1\right)^h}=\\
&\frac {2n\left(c+2\right)}{\left(c+1\right)^h}\endalign$$
a d\accent23ukaz (2) ve V\v et\v e je hotov.
\medskip

\flushpar V\v eta vysv\v etluje, pro\v c jsou doporu\v cen\'e hodnoty $
b\ge 2a$ -- 
pak je po\v cet vyva\v zovac\'\i ch ope\-rac\'\i\ b\v ehem posloupnosti 
operac\'\i\ {\bf INSERT} a {\bf DELETE} line\'arn\'\i\ vzhledem k d\'elce 
t\'eto posloupnosti. Pro $b=2a-1$ lze lehce nal\'ezt 
posloupnost operac\'\i\ {\bf INSERT} a {\bf DELETE} o d\'elce $n$ takovou, 
\v ze jej\'\i\ aplikace na pr\'azdn\'y $\left(a,b\right)$-strom vy\v zaduje po\v cet 
vyva\v zovac\'\i ch operac\'\i\ \'u\-m\v ern\'y $n\log n$ (pro ka\v zd\'e 
dostate\v cn\v e velk\'e $n$). Podobn\'a v\v eta plat\'\i\ i pro paraleln\'\i\ 
implementaci $\left(a,b\right)$-strom\accent23u, ale plat\'\i\ za 
p\v redpokladu $b\ge 2a+2$. Pro $b=2a$ nebo $b=2a+1$ lze 
nal\'ezt posloupnost, kter\'a je protip\v r\'\i kladem. Proto se 
doporu\v cuje hodnota $b=2a+2$ pro paraleln\'\i\ implementaci 
$\left(a,b\right)$-stromu. Pro propojen\'e $\left(a,b\right)$-stromy plat\'\i\ siln\v ej\v s\'\i\ 
verze.

\proclaim{V\v eta}P\v redpokl\'adejme, \v ze $b\ge 2a$ a $a\ge 2$. M\v ejme 
hladinov\v e propojen\'y $\left(a,b\right)$-strom s prstem $T$, kter\'y reprezentuje 
$n$-prvkovou mno\v zinu. Pak posloupnost $\Cal P$ operac\'\i\ {\bf MEMBER},
{\bf INSERT}, {\bf DELETE} a {\bf PRST} aplikovan\'a na $T$ vy\v zaduje \v cas 
$$O\left(\log\left(n\right)+\text{\rm\ \v cas na vyhled\'an\'\i\ prvk\accent23u}\right)
.$$
\endproclaim


\flushpar Vysv\v etlen\'\i : Za\v c\'\i n\'ame v libovoln\'em 
propojen\'em $\left(a,b\right)$-stro\-m\v e $T$, proto jeho struktura 
m\accent23u\v ze b\'yt nev\'yhodn\'a pro danou pos\-loupnost operac\'\i\ $
\Cal P$. 
Abychom se dostali do vhodn\'eho re\v zimu, m\accent23u\v ze 
b\'yt t\v reba a\v z $\log\left(n\right)$ vyva\v zovac\'\i ch operac\'\i . \v Cas na 
vy\-hled\'av\'a\-n\'\i\ nem\accent23u\v zeme ovlivnit, ten mus\'\i\ 
ovlivnit u\v zivatel.
\medskip

\flushpar Aplikace: anal\'yza hladinov\v e propojen\'ych strom\accent23u s prstem 
u\-mo\v z\-nila n\'avrh algoritmu, kter\'y pro dv\v e mno\v ziny $
S_1$ a $S_2$ 
repre\-zentovan\'e propojen\'ymi $\left(a,b\right)$-stromy, kde $b\ge 2a$ a 
$a\ge 2$, 
zkonstruuje propojen\'y $\left(a,b\right)$-strom reprezentuj\'\i c\'\i\ mno\v zinu 
$S_1\cup S_2$ (nebo mno\v zinu $\Delta \left(S_1,S_2\right)=\left(S_1\setminus S_
2\right)\cup \left(S_2\setminus S_1\right)$ nebo 
$S_1\cap S_2$ nebo $S_1\setminus S_2$) v \v case $O\left(\log\binom {n
+m}m\right)$, kde 
$n=\max\left\{|S_1|,|S_2|\right\}$ a $m=\min\left\{|S_1|,|S_2|\right\}$. Detaily budou v letn\'\i m 
semestru.
\medskip

\flushpar Vyva\v zov\'an\'\i\ p\v ri operaci {\bf INSERT} lze prov\'ad\v et 
tak, \v ze operace {\bf \v St\v epen\'\i$\left(t\right)$} se provede, jen kdy\v z oba 
brat\v ri vrcholu $t$ maj\'\i\ $b$ syn\accent23u. Jinak se prov\'ad\'\i\ 
operace {\bf P\v resun}. Nev\'\i m o \v z\'adn\'em seri\'ozn\'\i m pokusu tyto 
alternativy porovnat.   
\medskip

\heading
Vyhled\'av\'an\'\i\ v uspo\v r\'adan\'em poli
\endheading

\flushpar Zad\'an\'\i\ \'ulohy: M\'ame podmno\v zinu $S$ line\'arn\v e 
uspo\v r\'adan\'eho univerza a $S$ je ulo\v zena v poli 
$A\left[1..|S|\right]$ tak, \v ze pro $i<j$ je $A\left(i\right)<A\left(j\right)$. Pro dan\'e $x\in 
U$ 
m\'ame zjistit, zda $x\in S$ (operace {\bf MEMBER$\left(x\right)$}).
\medskip

\flushpar\v Re\v sen\'\i :  Pokud $x<A\left(1\right)$ nebo $A\left(|S|\right)<x$, pak $
x$ 
nen\'\i\ prvkem $S$.  V opa\v cn\'em p\v r\'\i pad\v e bu\v d $x=
A\left(1\right)$ nebo 
$x=A\left(|S|\right)$ nebo m\'ame dv\v e hodnoty $d$ 
a $h$ takov\'e, \v ze $1\le d<d+1<h\le |S|$ a $A\left(d\right)<x<A\left(h\right)$.  Pak 
najdeme $n$ takov\'e, \v ze $d<n<h$, a dotazem 
zjist\'\i me, zda $x=A\left(n\right)$ (pak kon\v c\'\i me a $x\in S$) nebo 
$x<A\left(n\right)$ (pak polo\v z\'\i me $h=n$) nebo $x>A\left(n\right)$ (pak 
polo\v z\'\i me $d=n$) a proces opakujeme.  Kon\v c\'\i me, kdy\v z 
$d+1\ge h$, pak $x\notin S$. Na za\v c\'atku polo\v z\'\i me $d=1$ a $
h=|S|$.
Form\'aln\'\i\ z\'apis algoritmu:
skip

{\bf MEMBER$\left(x\right)$\newline 
if} $x=A\left(1\right)$ {\bf then}\newline 
\phantom{---}{\bf V\'ystup}: $x\in S$ stop\newline 
{\bf else}\newline 
\phantom{---}{\bf if} $x<A\left(1\right)$ {\bf then}\newline 
\phantom{------}{\bf V\'ystup}: $x\notin S$ stop\newline 
\phantom{---}{\bf else}\newline 
\phantom{------}$d=1$\newline 
\phantom{---}{\bf endif\newline
\bf endif\newline 
if} $x=A\left(|S|\right)$ {\bf then}\newline 
\phantom{---}{\bf V\'ystup}: $x\in S$ stop\newline 
{\bf else}\newline 
\phantom{---}{\bf if} $x>A\left(|S|\right)$ {\bf then}\newline 
\phantom{------}{\bf V\'ystup}: $x\notin S$ stop\newline 
\phantom{---}{\bf else}\newline 
\phantom{------}$h=|S|$\newline 
\phantom{---}{\bf endif\newline
endif\newline 
while} $d+1<h$ {\bf do}\newline 
\phantom{---}$n:=\text{{\bf next}}\left(d,h\right)$\newline 
\phantom{---}{\bf if} $\bold x=A\left(n\right)$ {\bf then}\newline 
\phantom{------}{\bf V\'ystup}: $\bold x\in S$ stop\newline 
\phantom{---}{\bf else}\newline 
\phantom{------}{\bf if} $x<A\left(n\right)$ {\bf then} $h:=n$ {\bf else} $d:=n$ {\bf endif}\newline 
\phantom{---}{\bf endif\newline 
enddo\newline 
V\'ystup}: $x\notin S$ stop
\medskip

\flushpar V tomto metaalgoritmu je {\bf next$\left(d,h\right)$} funkce, kter\'a 
nalezne hodnotu $n$ takovou, \v ze $d<n<h$. Korektnost plyne 
z pozorov\'an\'\i , \v ze kdy\v z $d+1=h$, pak $A\left(d\right)<x<A\left(h\right)$ 
implikuje, \v ze neexistuje $i$ takov\'e, \v ze $x=A\left(i\right)$, a tedy 
$x\notin S$. Efektivita algoritmu z\'ale\v z\'\i\ na fukci {\bf next}. 
Zpracov\'an\'\i\ dotazu vy\v zaduje \v cas $O\left(1\right)$ a po\v cet 
dotaz\accent23u je 
po\v cet vol\'an\'\i\ funkce {\bf next}. 
\medskip

\flushpar Un\'arn\'\i\ vyhled\'av\'an\'\i : {\bf next$\left(d,h\right)=d+1$}, pak ka\v zd\'y 
dotaz zv\v et\v s\'\i\ $d$ o $1$, a tedy nejv\v et\v s\'\i\ po\v cet 
dotaz\accent23u je $|S|$. Algoritmus v nej\-hor\v s\'\i m p\v r\'\i pad\v e 
vy\v zaduje \v cas $O\left(|S|\right)$ a o\v cek\'avan\'y po\v cet dotaz\accent23u p\v ri 
rovnom\v ern\'em rozlo\v zen\'\i\ mno\v ziny $S$ a prvku $x$ je $\frac {
|S|}2$ 
(tedy o\v cek\'avan\'y \v cas je $O\left(|S|\right)$). 
\medskip

\flushpar Pozn\'amka: Du\'aln\'\i\ p\v r\'\i stup je, kdy\v z 
{\bf next$\left(d,h\right)=h-1$}, v\'ysled\-ky se nezm\v en\'\i . P\v ri aplikac\'\i ch je 
n\v ekdy v\'yhodn\'e pou\v z\'\i t funkci {\bf next$\left(d,h\right)=\min\left\{d
+c,h-1\right\}$}, 
kde $c$ je n\v ejak\'a konstanta (pak krok nen\'\i\ $1$, ale $c$). Jak 
uvid\'\i me pozd\v eji, jsou situace, kdy je v\'yhodn\'e  
takov\'eto un\'arn\'\i\ vyhled\'av\'an\'\i\ pou\v z\'\i t.
\medskip

\flushpar Bin\'arn\'\i\ vyhled\'av\'an\'\i : {\bf next$\left(d,h\right)=\left\lceil\frac {
d+h}2\right\rceil$}, pak 
ka\v zd\'y dotaz zmen\v s\'\i\ rozd\'\i l $h-d$ p\v ribli\v zn\v e na polovinu. 
Po\v cet dotaz\accent23u je nejv\'y\v se $3+\log\left(|S|-2\right)$, 
algoritmus tedy v 
nejhor\v s\'\i m p\v r\'\i pad\v e vy\v zaduje \v cas $O\left(\log|S|\right)$ a  
o\v cek\'avan\'y \v cas p\v ri rovnom\v ern\'em roz\-lo\v zen\'\i\ mno\v ziny $
S$ a 
$x\in U$ je tak\'e $O\left(\log|S|\right)$.
\medskip

\flushpar Interpola\v cn\'\i\ vyhled\'av\'an\'\i : 
{\bf next$\left(d,h\right)=d+\left\lceil\frac {x-A\left(d\right)}{A\left(h\right)-A\left(d\right)}\left(h-d\right)\right\rceil$}. V nejhor\v s\'\i m p\v r\'\i pad\v e 
mus\'\i me polo\v zit v\'\i ce ne\v z $\frac {|S|}2$ dotaz\accent23u, a proto \v cas 
v nejhor\v s\'\i m p\v r\'\i pad\v e je $O\left(|S|\right)$, ale p\v ri rovnom\v ern\'em 
roz\-lo\v zen\'\i\ mno\v ziny $S$ a $x\in U$ je o\v cek\'avan\'y \v cas 
$O\left(\log\log|S|\right)$. To je zalo\v zeno na faktu, \v ze hodnota {\bf next }
z\'avis\'\i\ i na velikosti $x$. Kdy\v z $x$ je velk\'e, tak hodnota 
{\bf next} je posunuta do v\v et\v s\'\i ch hodnot, kdy\v z $x$ je mal\'e, 
pak je posunuta do men\v s\'\i ch hodnot.
\medskip

\flushpar Pozn\'amka: Kdy\v z rozlo\v zen\'\i\ prvk\accent23u nen\'\i\ 
rovnom\v ern\'e, ale je zn\'a\-m\'e, pak podle toho m\accent23u\v zeme  
upravit funkci {\bf next} a o\v cek\'avan\'y \v cas algoritmu se 
nezm\v en\'\i .
\medskip

\flushpar Pro n\'asleduj\'\i c\'\i\ funkci {\bf next} bude 
jednodu\v s\v s\'\i\ spo\v c\'\i tat o\v cek\'ava\-n\'y po\v cet dotaz\accent23u ne\v z 
pro interpola\v cn\'\i\ vyhled\'av\'an\'\i , ale v\'ysledek je 
asymptoticky stejn\'y.

\subhead
Zobecn\v en\'e kvadratick\'e vyhled\'av\'an\'\i
\endsubhead

\flushpar Funkce {\bf next }
je zde definov\'ana slo\v zit\v ej\v s\'\i\ procedurou, jej\'\i\v z v\'y\-sle\-dek 
z\'avis\'\i\ i na p\v redcho\-z\'\i ch situac\'\i ch a na v\'ysledku dotazu.
Procedura zad\'av\'a dotazy v bloc\'\i ch. Prvn\'\i\ dotaz v bloku je 
interpola\v cn\'\i\ a procedura p\v ritom zjist\'\i\ velikost kroku a 
zda $x$ je men\v s\'\i\ nebo v\v et\v s\'\i\ ne\v z prvn\'\i\ dotaz v bloku.  
Pak st\v r\'\i d\'a un\'arn\'\i\ a bin\'arn\'\i\ vy\-hled\'av\'an\'\i . Blok kon\v c\'\i , 
kdy\v z rozd\'\i l mezi $h$ a $d$ je nejv\'y\v se velikost 
kroku. Krok v n\'asleduj\'\i c\'\i m bloku klesne 
p\v ribli\v zn\v e na odmoc\-ninu velikosti kroku v tomto bloku. 
Procedura pou\v z\'\i v\'a boolsk\'e pro\-m\v enn\'e $blok$, $typ$, $
smer$. 
Prom\v enn\'a $blok$ je inicializov\'ana hodnotou $false$ a ur\v cuje, 
zda se dotaz zad\'av\'a v r\'amci stejn\'eho bloku nebo nikoliv. 
Prom\v enn\'a $typ$ ur\v cuje, zda p\v r\'\i\v st\'\i\ dotaz je un\'arn\'\i\ (kdy\v z 
$typ=true$) nebo bin\'arn\'\i . Prom\v enn\'a $smer$ ur\v cuje, zda 
dotazy jsou men\v s\'\i\ ne\v z prvn\'\i\ dotaz v bloku ($smer=tr
ue$) 
nebo v\v et\v s\'\i . D\'ale procedura pou\v z\'\i v\'a prom\v ennou $
krok$ 
typu integer, kter\'a obsahuje velikost kroku v r\'amci 
bloku. Hodnoty t\v echto pro\-m\v en\-n\'ych se p\v red\'avaj\'\i\ z jednoho 
vol\'an\'\i\ procedury do dal\v s\'\i ho vol\'an\'\i\ (tj. jsou to 
glob\'aln\'\i\ prom\v enn\'e, kter\'e se neinicializuj\'\i\ vol\'an\'\i m 
procedury {\bf next}). 
\medskip

{\bf next$\left(d,h\right)$\newline 
if} $blok$ {\bf then}\newline 
\phantom{---}{\bf if} $typ$ {\bf then}\newline 
\phantom{------}{\bf if} $smer$ {\bf then}\newline 
\phantom{---------}{\bf next}$\left(d,h\right):=h-krok$\newline 
\phantom{---------}{\bf if} $A\left(\text{{\bf next}}\left(d,h\right)\right)<x$ {\bf then}\newline
\phantom{------------}$blok:=false$\newline 
\phantom{---------}{\bf endif}\newline 
\phantom{------}{\bf else}\newline 
\phantom{---------}{\bf next}$\left(d,h\right):=d+krok$\newline 
\phantom{---------}{\bf if} $A\left(\text{{\bf next}}\left(d,h\right)\right)>x$ {\bf then}\newline \phantom{------------}$blok:=false$\newline 
\phantom{---------}{\bf endif}\newline 
\phantom{------}{\bf endif}\newline 
\phantom{------}$typ:=false$\newline 
\phantom{---}{\bf else}\newline 
\phantom{------}{\bf if} $\min\left\{h-\left\lceil\frac {d+h}2\right\rceil ,\left\lceil\frac {d+h}2\right\rceil-d\right\}<krok$ {\bf then}\newline 
\phantom{---------}$blok:=false$\newline 
\phantom{------}{\bf endif}\newline 
\phantom{------}{\bf next}$\left(d,h\right):=\left\lceil\frac {d+h}2\right\rceil$, $typ:=true$\newline 
\phantom{---}{\bf endif\newline 
else}\newline 
\phantom{---}$krok:=\lfloor\sqrt {h-d}\rfloor$, 
{\bf next$\left(d,h\right):=d+\left\lceil\frac {x-A\left(d\right)}{A\left(h\right)-A\left(d\right)}\left(h-d\right)\right\rceil$},\newline 
\phantom{---}{\bf if} $A\left(\text{{\bf next}}\left(d,h\right)\right)>x$ {\bf then}\newline 
\phantom{------}$smer:=true$\newline 
\phantom{---}{\bf else}\newline 
\phantom{------}$smer:=false$\newline 
\phantom{---}{\bf endif}\newline 
\phantom{---}$typ:=true$, $blok:=true$\newline 
{\bf endif}
skip

\flushpar Po dvou dotazech klesne $h-d$ bu\v d pod $\sqrt {h-d}$ nebo pod 
$\frac {h+d}2$. Proto procedura v nejhor\v s\'\i m p\v r\'\i pad\v e pou\v zije 
nejv\'y\v se $8+2\log\left(|S|-1\right)+2\log\log|S|$ dotaz\accent23u, a tedy v nejhor\v s\'\i m 
p\v r\'\i pad\v e vy\v zaduje \v cas $O\left(\log|S|\right)$.
\medskip

\flushpar Nyn\'\i\ spo\v c\'\i t\'ame o\v cek\'avan\'y po\v cet dotaz\accent23u 
b\v ehem jednoho bloku za p\v redpokladu rovnom\v er\-n\'eho rozd\v elen\'\i\ dat. Nech\v t $p_i$ je prav\-d\v e\-podobnost, \v ze 
v r\'amci bloku se polo\v z\'\i\ ales\-po\v n $i$ dotaz\accent23u. Pak 
o\v cek\'avan\'y po\v cet dotaz\accent23u v r\'amci bloku je 
$$E\left(C\right)=\sum_{i\ge 1}i\left(p_i-p_{i+1}\right)=\sum_{i\ge 1}p_i.$$
Nyn\'\i\ odhadneme $p_i$.  Ozna\v cme $n+d$ argument prvn\'\i ho dotazu 
(interpola\v cn\'\i\ vyhled\'avan\'\i ) v r\'amci bloku a nech\v t $
krok=k$ 
v r\'amci bloku.  Ozna\v cme $X=|\left\{i\mid i>d,\,A\left(i\right)\le x\right\}|$ na 
za\v c\'atku bloku, pak $X$ je n\'ahodn\'a prom\v enn\'a z\'avisl\'a na 
argumentu ope\-race a bloku.  Kdy\v z se v bloku polo\v z\'\i\ 
alespo\v n $i$ dotaz\accent23u pro $i>2$, pak $|X-n|\ge\lfloor\frac {
i-2}2\rfloor k$, 
proto\v ze ka\v zd\'y un\'arn\'\i\ dotaz, jeho\v z polo\v zen\'\i\  
nezm\v en\'\i\ blok, nalezne dal\v s\'\i ch $k$ hodnot $i$ v rozd\'\i lu $|X-n|$. 
Tedy 
$$p_i\le\Prob\left(|X-n|\ge\lfloor\frac {i-2}2\rfloor k\right).$$
Pou\v zijeme \v Ceby\v sevovu nerovnost pro n\'ahodnou 
prom\v en\-nou $X$. Kdy\v z $Y$ je n\'ahodn\'a prom\v enn\'a s o\v cek\'avanou 
(st\v redn\'\i ) hodnotou $\mu$ a rozptylem $\sigma^2$, pak \v Ceby\v sevova 
nerovnost \v r\'\i k\'a, \v ze 
$$\Prob\left(|Y-\mu |\ge t\right)\le\frac {\sigma^2}{t^2}\qquad\text{\rm pro ka\v zd\'e }
t>0.$$
\medskip

\flushpar Uva\v zujme okam\v zik, kdy jsme na za\v c\'atku n\v ejak\'eho 
bloku. Pro\-to\v ze $S$ je vybran\'a s rovno\-m\v er\-n\'ym rozd\v elen\'\i m, je 
pravd\v epodob\-nost, \v ze $A\left(i\right)<x$ pro $d<i<h$, rovna 
$p=\frac {x-A\left(d\right)}{A\left(h\right)-A\left(d\right)}$, a pak pravd\v epodobnost, \v ze 
$X=j$, je $\binom {h-d}jp^j\left(1-p\right)^{h-d-j}$. To znamen\'a, \v ze $X$ je n\'ahodn\'a 
veli\v cina s binomick\'ym rozd\v elen\'\i m s rozsahem $d-h$ a pravd\v epodobnost\'\i\ $p$, a tedy jej\'\i\ o\v cek\'avan\'a 
hodnota je
$$\mu =\sum_{j=0}^{h-d}j\binom {h-d}jp^j\left(1-p\right)^{h-d-j}=p\left(h-d\right)$$
a rozptyl m\'a hodnotu 
$$\sigma^2=\sum_{j=0}^{h-d}\left(j-\mu \right)^2\binom {h-d}jp^j\left(1-p\right)^{h-d-j}
=p\left(1-p\right)\left(h-d\right).$$
Kdy\v z si uv\v edom\'\i me, \v ze $k=\lfloor\sqrt {h-d}\rfloor$ a $
n=p\left(h-d\right)$, pak dost\'av\'ame
$$\align p_i,p_{i+1}\le&\Prob\left(|X-n|\ge\lfloor\frac {i-2}2\rfloor k 
\right)\le\frac {4p\left(1-p\right)\left(h-d\right)}{\left(i-2\right)^2k^2}\le\\
&\frac {4p\left(1-p\right)}{\left(i-2\right)^2}\le\frac 1{\left(i-2\right)^2},\endalign$$
proto\v ze pro $0\le p\le 1$ je $p\left(1-p\right)\le\frac 14$. Kdy\v z shrneme tato pozorov\'an\'\i , 
dost\'av\'a\-me, \v ze 
$$\align E\left(C\right)=&\sum_{i\ge 1}p_i\le 2+2\sum_{i\ge 3}\frac 1{\left(i-2\right)^2}=
2+2\sum_{i\ge 1}\frac 1{i^2}=\\
&2+2\frac {\pi^2}6=2+\frac {\pi^2}3\approx 5.3\endalign$$
Z\'av\v er: o\v cek\'avan\'y po\v cet dotaz\accent23u v bloku je men\v s\'\i\ ne\v z $
6$.
\medskip

\flushpar Kdy\v z $E\left(T\left(n\right)\right)$ je o\v cek\'avan\'y po\v cet dotaz\accent23u pro operaci 
{\bf MEMBER} a kdy\v z $|S|=n$, pak plat\'\i\ 
$$E\left(T\left(n\right)\right)\le E\left(C\right)+E\left(T\left(\sqrt n\right)\right).$$
Proto\v ze $E\left(T\left(1\right)\right)=1$ a $E\left(T\left(2\right)\right)\le 2$, dost\'av\'ame z rekurentn\'\i ho 
vzorce, \v ze 
$$E\left(T\left(n\right)\right)\le 2+E\left(C\right)\log\log n\qquad\text{\rm pro }n\ge 2.$$


\proclaim{V\v eta}\v Cas operace {\bf MEMBER} v uspo\v r\'adan\'em poli d\'elky $
n$ p\v ri 
zobec\-n\v en\'em kvadra\-tick\'em vyhled\'av\'an\'\i\ je v nejhor\v s\'\i m 
p\v r\'\i pa\-d\v e $O\left(\log n\right)$. Kdy\v z rozd\v elen\'\i\ vstupn\'\i ch dat je 
rovno\-m\v er\-n\'e, pak o\v cek\'avan\'y \v cas je $O\left(\log\log n\right)$. 
\endproclaim

\flushpar Nev\'yhoda t\'eto datov\'e struktury 
spo\v c\'\i v\'a v neexistenci p\v riroze\-n\'ych efektivn\'\i ch implementac\'\i\ 
ope\-rac\'\i\ {\bf INSERT}, {\bf DELETE}, {\bf SPLIT} a {\bf JOIN}. P\v rirozen\'e 
implementace t\v echto ope\-rac\'\i\ vy\v za\-duj\'\i\ \v cas 
$O\left(|S|\right)$, zhruba \v re\v ceno mus\'\i me pohybovat s t\'em\v e\v r ka\v z\-d\'ym prvkem. 
Pokusem o \v re\v sen\'\i\ 
tohoto probl\'emu byl n\'avrh bin\'ar\-n\'\i ch vyhled\'avac\'\i ch 
strom\accent23u.
\medskip

\heading 
Bin\'arn\'\i\ vyhled\'avac\'\i\ stromy
\endheading

\flushpar Bin\'arn\'\i\ vyhled\'avac\'\i\ strom je struktura pro bin\'ar\-n\'\i\ 
vyhled\'av\'an\'\i\ v uspo\v r\'ada\-n\'em poli rozta\-\v zen\'em do roviny a 
vyhled\'av\'an\'\i\ odpo\-v\'\i\-d\'a cest\v e ve stro\-m\v e. Form\'aln\'\i\ 
definice:

\flushpar P\v redpokl\'ad\'ame, \v ze $U$ je line\'arn\v e uspo\v r\'adan\'e 
univerzum a $S\subseteq U$.  \emph{Bin\'arn\'\i} \emph{vyhled\'avac\'\i} 
\emph{strom} $T$ \emph{reprezentuj\'\i c\'\i} mno\v zinu $S$ je \'upln\'y 
bin\'arn\'\i\ strom (tj. ka\v zd\'y vrchol je bu\v d listem nebo m\'a 
dva syny, lev\'eho a prav\'eho), kde existuje bijekce mezi 
mno\v zinou $S$ a vnit\v rn\'\i mi vrcholy stromu takov\'a, \v ze
\roster
\item"{}"
kdy\v z $v$ je vnit\v rn\'\i\ vrchol stromu $T$, kter\'emu je p\v ri\v razen 
prvek $s\in S$, pak ka\v zd\'emu vnit\v rn\'\i mu vrcholu $u$ v 
podstromu lev\'eho syna vrcholu $v$ je p\v ri\v razen prvek z $S$ 
men\v s\'\i\ ne\v z $s$ a ka\v zd\'emu vnit\v rn\'\i mu vrcholu $
w$ v 
podstromu prav\'eho syna vrcholu $v$ je p\v ri\v razen prvek z $S$ 
v\v et\v s\'\i\ ne\v z $s$.
\endroster
\medskip

\flushpar Strukura vnit\v rn\'\i ho vrcholu $v$:\newline 
ukazatel $\otec\left(v\right)$ na otce vrcholu $v$,\newline 
ukazatel $\levy\left(v\right)$ na lev\'eho syna vrcholu $v$,\newline 
ukazatel $\pravy\left(v\right)$ na prav\'eho syna vrcholu $v$,\newline 
atribut $\key\left(v\right)$ -- prvek z $S$ p\v ri\v razen\'y vrcholu $v$.
Kdy\v z $v$ je ko\v ren stromu, pak hodnota ukazatele $\otec\left(v\right)$ 
je $NIL$. List m\'a ukazatele pouze na otce.
\medskip

\flushpar Ka\v zd\'y list reprezentuje interval mezi dv\v ema 
sousedn\'\i mi prvky z $S$ -- p\v resn\v e, kdy\v z $u$ je list a je 
lev\'ym synem vrcholu $v$, nalezneme vrchol na cest\v e z $u$ 
do ko\v rene nejbl\'\i\v ze $u$ takov\'y, \v ze je prav\'ym synem 
vrcholu $w$. Pak $u$ repre\-zentuje interval $\left(\key\left(w\right),\key\left(v\right)\right)$ a 
kdy\v z vrchol $w$ neexistuje, pak $u$ repre\-zentuje 
interval $\left(-\infty ,\key\left(v\right)\right)$ a prvek $\key\left(v\right)$ je nej\-men\v s\'\i\ prvek v 
$S$.  Kdy\v z $u$ je list a je prav\'ym synem vrcholu $v$, 
nalezneme vrchol na cest\v e z $u$ do ko\v rene nejbl\'\i\v ze 
$u$ takov\'y, \v ze je lev\'ym synem vrcholu $w$. Pak $u$ repre\-zentuje 
interval $\left(\key\left(v\right),\key\left(w\right)\right)$ a kdy\v z takov\'y vrchol $w$ neexistuje, 
pak $u$ repre\-zentuje interval $\left(\key\left(v\right),+\infty \right)$ a prvek $\key
\left(v\right)$ 
je nejv\v et\v s\'\i\ prvek v $S$.  
\medskip

\flushpar P\v ri implementaci bin\'arn\'\i ch vyhled\'av\'ac\'\i ch 
strom\accent23u je v\'yhodn\'e vynechat listy (m\'\i sto nich bude 
ukazatel $NIL$). P\v ri n\'avrhu algoritm\accent23u je v\v sak 
naopak v\'yhodn\'e 
pracovat s listy (vyhl\'\i\v z\'\i\ to logi\v ct\v ej\v s\'\i ). Proto p\v ri n\'avrhu 
algoritm\accent23u budeme p\v redpokl\'adat, \v ze stromy maj\'\i\ 
listy reprezentuj\'\i c\'\i\ intervaly.
\medskip

\flushpar Navrhneme algoritmy pro bin\'arn\'\i\ vyhled\'avac\'\i\ stromy 
rea\-lizuj\'\i\-c\'\i\ operace z uspo\v r\'adan\'eho slovn\'\i kov\'eho probl\'emu.
\medskip

{\bf Vyhledej$\left(x\right)$\newline 
$t:=$}ko\v ren stromu\newline 
{\bf while} $t$ nen\'\i\ list a $\key\left(t\right)\ne x$ {\bf do}\newline 
\phantom{---}{\bf if} $\key\left(t\right)>x$ {\bf then} $t:=\levy\left(t\right)$ {\bf else} $t:=\pravy\left(t\right)$ {\bf endif\newline 
enddo
\medskip

MEMBER$\left(x\right)$\newline 
Vyhledej$\left(x\right)$\newline 
if} $t$ nen\'\i\ list {\bf then V\'ystup}: $x\in S$ {\bf else V\'ystup}: $
x\notin S$ {\bf endif
\medskip

INSERT$\left(x\right)$\newline 
Vyhledej$\left(x\right)$\newline 
if} $t$ je list {\bf then}\newline 
$t$ se zm\v en\'\i\ na vnit\v rn\'\i\ vrchol, $\key\left(t\right):=x$,\newline 
$\levy\left(t\right)$ a $\pravy\left(t\right)$ jsou nov\'e listy, jejich\v z otcem je $t$\newline 
{\bf endif
\medskip

DELETE$\left(x\right)$\newline 
Vyhledej$\left(x\right)$\newline 
if} $t$ nen\'\i\ list {\bf then}\newline 
\phantom{---}{\bf if} $\levy\left(t\right)$ je list {\bf then\newline}
\phantom{------}odstran\'\i me vrchol $\levy\left(t\right)$, $\otec\left(\pravy\left(t\right)\right):=\otec\left(t\right)$\newline 
\phantom{------}{\bf if} $t=\levy\left(\otec\left(t\right)\right)$ {\bf then}\newline
\phantom{---------}$\levy\left(\otec\left(t\right)\right):=\pravy\left(t\right)$\newline 
\phantom{------}{\bf else}\newline 
\phantom{---------}$\pravy\left(\otec\left(t\right)\right):=\pravy\left(t\right)$\newline 
\phantom{------}{\bf endif\newline}
\phantom{------}odstran\'\i me vrchol $t$\newline 
\phantom{---}{\bf else}\newline 
\phantom{------}$u:=\levy\left(t\right)$\newline 
\phantom{------}{\bf while} $\pravy\left(u\right)$ nen\'\i\ list {\bf do}\newline \phantom{---------}$u:=\pravy\left(u\right)$\newline 
\phantom{------}{\bf enddo}\newline 
\phantom{------}$\key\left(t\right):=\key\left(u\right)$, odstran\'\i me vrchol $\pravy\left(u\right)$,\newline 
\phantom{------}$\otec\left(\levy\left(u\right)\right):=\otec\left(u\right)$\newline 
\phantom{------}{\bf if} $u=\levy\left(\otec\left(u\right)\right)$ {\bf then}\newline 
\phantom{---------}$\levy\left(\otec\left(u\right)\right):=\levy\left(u\right)$\newline 
\phantom{------}{\bf else}\newline 
\phantom{---------}$\pravy\left(\otec\left(u\right)\right):=\levy\left(u\right)$\newline 
\phantom{------}{\bf endif\newline}
\phantom{------}odstran\'\i me vrchol $u$\newline 
\phantom{---}{\bf endif\newline 
endif}
\medskip

{\bf MIN\newline 
$t:=$}ko\v ren stromu\newline 
{\bf while} lev\'y syn $t$ nen\'\i\ list {\bf do} $t:=\levy\left(t\right)$ {\bf enddo\newline 
V\'ystup}: prvek reprezentovan\'y $t$ je nejmen\v s\'\i\ prvek v $
S$
\medskip

{\bf MAX\newline 
$t:=$}ko\v ren stromu\newline 
{\bf while} prav\'y syn $t$ nen\'\i\ list {\bf do} $t:=\pravy\left(t\right)$ {\bf enddo\newline 
V\'ystup}: prvek reprezentovan\'y $t$ je nejv\v et\v s\'\i\ prvek v $
S$
\medskip

{\bf SPLIT$\left(x\right)$}:\newline 
$T_1$ a $T_2$ jsou pr\'azdn\'e stromy\newline 
$u_1:=u_2:=NIL$\newline 
$t:=\text{\rm ko\v ren stromu }T$\newline 
{\bf while} $t$ nen\'\i\ list a $\key\left(t\right)\ne x$ {\bf do\newline}
\phantom{---}{\bf if} $\key\left(t\right)>x$ {\bf then\newline 
\phantom{------}$u:=\levy\left(t\right)$}, $\levy\left(t\right):=NIL$, $\otec\left(u\right):=NIL$\newline 
\phantom{------}{\bf if} $T_2$ je pr\'azdn\'y strom {\bf then}\newline \phantom{---------}$T_2:=\text{\rm podstrom vrcholu }t$\newline
\phantom{------}{\bf else}\newline 
\phantom{---------}$\levy\left(u_2\right):=t$, $\otec\left(t\right):=u_2$\newline 
\phantom{------}{\bf endif}\newline
\phantom{------}$u_2:=t$\newline 
\phantom{---}{\bf else}\newline 
\phantom{------}$u:=\pravy\left(t\right)$, $\pravy\left(t\right):=NIL$, $\otec\left(u\right):=NIL$\newline
\phantom{------}{\bf if} $T_1$ je pr\'azdn\'y strom {\bf then}\newline \phantom{---------}$T_1:=\text{\rm podstrom vrcholu }t$\newline
\phantom{------}{\bf else}\newline 
\phantom{---------}$\pravy\left(u_1\right):=t$, $\otec\left(t\right):=u_1$\newline 
\phantom{------}{\bf endif}\newline
\phantom{------}$u_1:=t$\newline 
\phantom{---}{\bf endif}\newline 
\phantom{---}$t:=u$\newline 
{\bf enddo\newline 
if} $\key\left(t\right)=x$ {\bf then}\newline 
\phantom{---}$\otec\left(\levy\left(t\right)\right):=u_1$, $\pravy\left(u_1\right):=\levy\left(t\right)$\newline 
\phantom{---}$\otec\left(\pravy\left(t\right)\right):=u_2$, $\levy\left(u_2\right):=\pravy\left(t\right)$\newline 
\phantom{---}$\otec\left(u_1\right):=NIL$, $\otec\left(u_2\right):=NIL$, {\bf V\'ystup}: $x\in S$\newline 
{\bf else}\newline 
\phantom{---}{\bf V\'ystup}: $x\notin S$\newline 
{\bf endif\newline }
Koment\'a\v r: $T_1$ je bin\'arn\'\i\ vyhled\'avac\'\i\ strom reprezentuj\'\i c\'\i\ 
mno\-\v zi\-nu 
$\left\{s\in S\mid s<x\right\}$ a $T_2$ je bin\'arn\'\i\ vyhled\'avac\'\i\ strom 
reprezentuj\'\i c\'\i\ mno\v zinu $\left\{s\in S\mid s>x\right\}$.
\medskip

{\bf JOIN3$\left(T_1,x,T_2\right)$} -- p\v redpokl\'ad\'ame, \v ze kdy\v z $
T_i$ reprezentuje 
mno\v zinu $S_i$ pro $i=1,2$, pak $\max S_1<x<\min S_2$\newline 
vytvo\v rme nov\'y vrchol $u$, $\key\left(u\right)=x$, $\otec\left(u\right):=NIL$, \newline 
$\otec\left(\text{ko\v rene }T_1\right):=u$, $\otec\left(\text{ko\v rene }T_2\right):=u$, \newline 
$\levy\left(u\right):=$ko\v ren $T_1$, $\pravy\left(u\right):=$ko\v ren $T_2$.
\medskip


\flushpar Abych dok\'azali korektnost algoritmu {\bf Vyhledej} -- 
jedn\'a se o modifikaci vyhled\'av\'an\'\i\ v uspo\v r\'adan\'em poli --
pop\'\i\v seme po\-drobn\v eji vlastnosti bin\'arn\'\i ho vyhled\'avac\'\i ho 
stromu. Nej\-prve roz\v s\'\i \v r\'\i me universum o dva nov\'e prvky, o nov\'y 
nejmen\v s\'\i\ prvek $-\infty$ a o nov\'y nejv\v et\v s\'\i\ prvek $+\infty$. 
M\v ejme bin\'arn\'\i\ vyhled\'avac\'\i\ strom $T$ reprezentuj\'\i c\'\i\ 
mno\v zinu $S$, pak pro vrchol $t$ stromu $T$ definujeme indukc\'\i\ hodnoty $\lambda\left(t\right)$ 
a $\pi\left(t\right)$. Kdy\v z $r$ je ko\v ren, pak $\lambda\left(r\right)=-\infty$ a $\pi\left(r\right)=+\infty$. 
Kdy\v z hodnoty $\lambda\left(t\right)$ a $\pi\left(t\right)$ jsou pro vrchol $t$ definov\'any, pak pro lev\'eho syna $u$ vrcholu $t$ definujeme $\lambda\left(u\right)=\lambda\left(t\right)$ a $\pi\left(u\right)=\key\left(y\right)$
a pro prav\'eho syna $w$ vrcholu $t$ definujeme $\lambda\left(w\right)=\key\left(t\right)$ a $\pi\left(w\right)=\pi\left(t\right)$.
Nyn\'\i\ dok\'a\v zeme


\proclaim{Lemma}Je-li $T'$ podstrom bin\'arn\'\i ho vyhled\'avac\'\i ho 
stromu $T$ ur\v cen\'y vrcholem $t$, pak $T'$ reprezentuje mno\v zinu 
$S\cap \left(\lambda \left(t\right),\pi \left(t\right)\right)$. Nav\'\i c interval $\left(\lambda \left(t\right),\pi 
\left(t\right)\right)$ je nejv\v et\v s\'\i\ interval, 
kter\'y obsahuje jenom prvky z $S$, kter\'e jsou reprezentov\'any 
vrcholy podstromu $T'$. Nav\'\i c, kdy\v z $t$ je list, pak $<\lambda\left(t\right),\pi\left(t\right)>$ je interval repreyentovan\'y listem $t$.
\endproclaim


\demo{D\accent23ukaz}Tvrzen\'\i\ dok\'a\v zeme indukc\'\i .  Z\v rejm\v e plat\'\i , 
kdy\v z $t$ je ko\v ren stromu $T$.  P\v redpokl\'adej\-me, \v ze plat\'\i\ pro vrchol 
$t$ a dok\'a\-\v ze\-me ho pro syny vrcholu $t$.  Ozna\v cme $t_l$ lev\'eho syna 
vrcholu $t$, $t_p$ prav\'eho syna vrcholu $t$.  Z definice bin\'arn\'\i ho 
vyhled\'avac\'\i ho stromu stromu plyne, \v ze kdy\v z $u$ je vnit\v rn\'\i\ 
vrchol v podstromu $T$ ur\v cen\'em vrcholem $t_l$ a kdy\v z $v$ je vnit\v rn\'\i\ 
vrchol v podstromu $T$ ur\v cen\'em vrcholem $t_p$, pak 
$\key\left(u\right)<\key\left(t\right)<\key\left(v\right)$.  Nyn\'\i\ platnost tvrzen\'\i\ pro $t$ 
implikuje platnost tvrzen\'\i\ i pro vrcholy $t_l$ a $t_p$.  \qed
\enddemo
\medskip

\flushpar Korektnost podprocedury {\bf Vyhledej} plyne z 
n\'asleduj\'\i c\'\i ho invariantu:
\roster
\item"{}"
Kdy\v z p\v ri vyhled\'av\'an\'\i\ $x$ vy\v set\v rujeme vrchol $
t$, pak 
$$\lambda \left(t\right)<x<\pi \left(t\right).$$
\endroster
Toto tvrzen\'\i\ se lehce dok\'a\v ze indukc\'\i\ z popisu algoritmu 
{\bf Vy\-hledej}.  Tedy operace {\bf Vy\-hledej} je korektn\'\i\ a korekt\-nost 
ope\-rac\'\i\ {\bf MEMBER} a {\bf INSERT} je te\v d z\v rejm\'a. V operaci {\bf DELE\-TE},  
kdy\v z $\levy\left(t\right)$ je list, pak korektnost je z\v rejm\'a. Kdy\v z $\levy
\left(t\right)$ 
nen\'\i\ list, pak algoritmus nalezne list $v$ takov\'y, \v ze $\pi\left(v\right)=x$. 
Pak pro $u=\otec\left(v\right)$ plat\'\i\ $v=\pravy\left(u\right)$ a $\lambda\left(v\right)=\key\left(u\right)$ a 
$\left(\lambda\left(v\right),\pi\left(v\right)\right)\cap S=\emptyset$. Kdy\v z $y=\key\left(u\right)$, pak 
odstran\v en\'\i\ vrchol\accent23u $u$ a $v$ d\'av\'a bin\'arn\'\i\ vyhled\'avac\'\i\ strom reprezentuj\'\i c\'\i\ $S\setminus \left\{y\right\}$. 
Proto\v ze $\left(y,x\right)\cap S=\emptyset$, tak p\v r\'\i kaz $\key\left(t\right):=y$ d\'av\'a bin\'arn\'\i\ 
vyhled\'avac\'\i\ strom reprezentuj\'\i c\'\i\ $S\setminus \left\{x\right\}$ a proto operace 
{\bf DELETE} je korektn\'\i . 
\medskip

\flushpar Korektnost operac\'\i\ {\bf MIN}, {\bf MAX} a {\bf JOIN3} plyne z definice 
bin\'arn\'\i ho vyhledavac\'\i ho stromu.  Korektnost operace {\bf SPLIT }
plyne z korektnosti algoritmu {\bf Vyhledej} a z faktu, \v ze $u_
1$ je otec 
nej\-prav\v ej\v s\'\i ho listu stromu $T_1$ a $u_2$ je otec nejlev\v ej\v s\'\i ho listu stromu $T_2$.  
Proto\v ze ke stromu $T_1$ se p\v rid\'av\'a \v c\'ast stromu $T$ reprezentuj\'\i c\'\i\ 
prvky, kter\'e jsou v\v et\v s\'\i\ ne\v z prvky reprezentovan\'e v $
T_1$, a ke 
stromu $T_2$ se p\v rid\'av\'a \v c\'ast stromu $T$ reprezentuj\'\i c\'\i\ prvky, 
kter\'e jsou men\v s\'\i\ ne\v z prvky reprezentovan\'e v $T_2$, korektnost 
algoritmu pro operaci {\bf SPLIT} je jasn\'a.  
\medskip

\flushpar Zpracov\'an\'\i\ jednoho vrcholu vy\v zaduje \v cas $O\left(
1\right)$ a 
algoritmus se pohybuje po jedn\'e cest\v e z ko\v rene do 
n\v ejak\'eho listu. Ozna\v cme $\hloubka\left(T\right)$ d\'elku nejdel\v s\'\i\ cesty 
z ko\v rene do n\v ejak\'eho listu. Pak dost\'av\'ame
\medskip

\proclaim{V\v eta}Algoritmy pro operace {\bf MEMBER}, {\bf INSERT}, {\bf DE\-LE\-TE}, {\bf MIN}, {\bf MAX}, 
{\bf JOIN3} a {\bf SPLIT} v bin\'ar\-n\'\i m vy\-hled\'a\-va\-c\'\i m stro\-m\v e $
T$ vy\v zaduj\'\i\ 
\v cas $O\left(\hloubka\left(T\right)\right)$. 
\endproclaim

\flushpar Bohu\v zel ani struktura bin\'arn\'\i ch vyhled\'avac\'\i ch 
strom\accent23u nepodporuje efektivn\'\i\ implementaci operace 
{\bf ord$\left(k\right)$}. Pro jej\'\i\ efektivn\'\i\ implentaci je vhodn\'e roz\v s\'\i\v rit 
datovou strukturu tak, \v ze u ka\v zd\'eho vrcholu $t$ je deklarov\'an 
tak\'e \'udaj $p\left(t\right)$ -- po\v cet list\accent23u v podstromu ur\v cen\'em 
vrcholem $t$. Po proveden\'\i\ operac\'\i\ {\bf INSERT}, {\bf DELETE}, {\bf JOIN3} a 
{\bf SPLIT} je pak nutn\'e aktualizovat tuto polo\v zku na cest\v e z 
vrcholu do ko\v rene. N\'asleduj\'\i c\'\i\ algoritmus pak 
realizuje operaci {\bf ord}$\left(k\right)$.
\medskip

{\bf ord$\left(k\right)$ \newline 
$t:=$}ko\v ren stromu \newline 
{\bf if} $k\ge p\left(t\right)$ {\bf then} $k$-t\'y prvek neexistuje, {\bf stop endif\newline 
while} true {\bf do}\newline 
\phantom{---}{\bf if} $k>p\left(\levy\left(t\right)\right)$ {\bf then}\newline 
\phantom{------}$k:=k-p\left(\levy\left(t\right)\right)$, $t:=\pravy\left(t\right)$\newline 
\phantom{---}{\bf else}\newline 
\phantom{------}{\bf if} $k<p\left(\levy\left(t\right)\right)$ {\bf then}\newline 
\phantom{---------}$t:=\levy\left(t\right)$\newline 
\phantom{------}{\bf else}\newline 
\phantom{---------}$\key\left(t\right)$ je $k$-t\'y prvek reprezentovan\'e mno\v ziny, stop \newline 
\phantom{------}{\bf endif}\newline 
\phantom{---}{\bf endif\newline 
enddo}
skip

\flushpar Korektnost algoritmu plyne z n\'asleduj\'\i c\'\i ho invariantu:  
Kdy\v z algoritmus m\'a v dan\'em okam\v ziku v prom\v enn\'e $t$ vrchol $
v$ 
a hodnota prom\v enn\'e $k$ je $k'$, pak $k$-t\'y prvek v $S$ se rovn\'a 
$k'$-t\'emu prvku v intervalu reprezentovan\'em v podstromu 
stromu $T$ ur\v cen\'em vrcholem $v$.  Proto\v ze na po\v c\'atku algoritmu 
je $v$ ko\v ren stromu a interval je $S$ (a $k'=k$), tak na po\v c\'atku 
b\v ehu algoritmu invariant plat\'\i.  P\v redpokl\'adejme, \v ze plat\'\i\ v 
n\v ekter\'em kroku.  Nech\v t $u$ je lev\'y syn $v$, $w$ je prav\'y syn $
v$ a 
$I_a$ je interval reprezovan\'y podstromem $T$ ur\v cen\'ym vrcholem $
a$.  
Pak $|I_u|=p\left(u\right)-1$, $\max I_u<\key\left(v\right)<\min I_w$ a 
$I_v=I_u\cup \left\{\key\left(v\right)\right\}\cup I_w$.  Odtud plyne, \v ze kdy\v z $k'
<p\left(u\right)$, pak 
$k'$-t\'y prvek v intervalu $I_v$ je $k'$-t\'y prvek v intervalu $
I_u$, kdy\v z 
$k'>p\left(u\right)$, pak $k'$-t\'y prvek v intervalu $I_v$ je $\left(k'-p\left(u\right)\right)$-t\'y prvek 
v intervalu $I_w$, a kdy\v z $k'=p\left(u\right)$, pak $k'$-t\'y prvek v intervalu $
I_v$ 
je $\key\left(v\right)$.  Odtud plyne invariant a korektnost algoritmu.  
Podle stejn\'ych argument\accent23u jako v p\v redchoz\'\i m p\v r\'\i pad\v e 
dostaneme, \v ze \v casov\'a slo\v zitost algoritmu je $O\left(\hloubka
\left(T\right)\right)$.  
Tedy m\accent23u\v zeme tato fakta shrnout.  

\proclaim{V\v eta}Algoritmy pro operace {\bf MEMBER}, {\bf INSERT}, {\bf DE\-LE\-TE}, {\bf MIN}, {\bf MAX}, 
{\bf JOIN3}, {\bf SPLIT} a $\bold o\bold r\bold d\left(k\right)$ pro v\v sechna $
k$ v roz\v s\'\i\v ren\'ych bin\'ar\-n\'\i ch 
vy\-hled\'avac\'\i ch stromech vy\v zaduj\'\i\ \v cas $O\left(\hloubka
\left(T\right)\right)$, kde $T$ je 
reprezentuj\'\i c\'\i\ strom. 
\endproclaim


\flushpar Tento v\'ysledek motivuje pou\v z\'\i v\'an\'\i\ bin\'arn\'\i ch 
vyhled\'avac\'\i ch strom\accent23u, kter\'e spl\v nuj\'\i\ dal\v s\'\i\ 
podm\'\i nku, kter\'a m\'a zajistit, \v ze 
$\hloubka\left(T\right)=O\left(\log|S|\right)$.  V takov\'emto p\v r\'\i pad\v e mluv\'\i me o 
\emph{vyv\'a\v zen\'ych} \emph{bin\'arn\'\i ch} 
\emph{vyhled\'avac\'\i ch} \emph{stromech}.  Je v\v sak nutn\'e 
p\v ridat k ope\-rac\'\i m {\bf INSERT}, {\bf DELETE}, {\bf JOIN3} a {\bf SPLIT} dal\v s\'\i\ 
kroky, kter\'e 
zaru\v c\'\i , \v ze po jejich proveden\'\i\ strom op\v et spl\v nuje 
po\v zadovan\'e podm\'\i nky.  To vede k po\v zadavku, aby 
vyva\v zovac\'\i\ operace byly rychl\'e a prov\'ad\v elo se jich m\'alo.  
\medskip

\flushpar P\v ri n\'ahodn\'e posloupnosti operac\'\i\ {\bf INSERT} a {\bf DELETE} je 
velk\'a pravd\v epodobnost, \v ze dostaneme n\'ahodn\'y bin\'arn\'\i\ 
vy\-hled\'avac\'\i\ strom.  Je zn\'amo, \v ze o\v cek\'avan\'a hodnota prom\v en\-n\'e 
$\hloubka\left(T\right)$ je $O\left(\log|S|\right)$.  Proto\v ze se nepou\v z\'\i vaj\'\i\ vyva\v zovac\'\i\ 
operace, m\accent23u\v zeme dostat lep\v s\'\i\ v\'ysledek (\v casov\v e) ne\v z pro 
vyv\'a\-\v zen\'e bin\'ar\-n\'\i\ vyhled\'avac\'\i\ stromy.  Tento probl\'em se te\v d 
intenzivn\v e studu\-je.  Velk\'a pozornost je v\v enov\'ana 
pravd\v epodobnost\-n\'\i m modifikac\'\i m bin\'arn\'\i ch vyhled\'avac\'\i ch 
strom\accent23u.  Hledaj\'\i\ se v\v sak i dal\v s\'\i\ mo\v znosti.  
\medskip

\flushpar Studuj\'\i\ se tzv.  samoupravuj\'\i c\'\i\ struktury.  Zde se 
pracuje s datovou strukturou bez dodate\v cn\'ych informac\'\i , ale 
operace nad touto strukturou prov\'ad\'\i\ vyva\v zov\'an\'\i\ v z\'avislosti 
na argumentu operace.  Dok\'azalo se, \v ze existuje strategie 
vyva\v zov\'an\'\i , kter\'a zaji\v s\v tuje dobr\'e chov\'an\'\i\ bez ohledu na 
vstupn\'\i\ data.  Dal\v s\'\i\ strategie je, \v ze se jen zji\v s\v tuje, zda datov\'a 
struktura nem\'a v\'yraz\-n\v e \v spatn\'e chov\'an\'\i , a pokud ho m\'a nebo 
po dlouh\'e \v rad\v e \'usp\v e\v s\-n\'ych aktualiza\v cn\'\i ch operac\'\i\ se vybuduje 
nov\'a datov\'a struktura (s optim\'aln\'\i m chov\'an\'\i m).  T\v ret\'\i , 
pom\v ern\v e star\'a, strategie je zalo\-\v ze\-na na p\v redpokladu, \v ze 
zn\'ame rozd\v elen\'\i\ vstupn\'\i ch dat.  Zde se datov\'a struktura 
p\v redem upravuje pro toto rozd\v elen\'\i .  Ukazuje se, \v ze tyto 
strategie maj\'\i\ \'usp\v ech.  Dal\v s\'\i\ podrobnosti v letn\'\i m semestru.  
\medskip

\flushpar Nyn\'\i\ si uk\'a\v zme dv\v e operace se stromy, na nich\v z jsou 
zalo\v zeny  vyva\v zovac\'\i\ operace pro bin\'arn\'\i\ vyhled\'avac\'\i\ 
stromy. Ob\v e ope\-race vy\v zaduj\'\i\ \v cas $O\left(1\right)$.
\medskip

\flushpar M\v ejme vrchol $v$ bin\'arn\'\i ho vyhled\'avac\'\i ho stromu $
T$ a 
jeho syna $u$, kter\'y je vnit\v rn\'\i\ vrchol.  Pak {\bf Rotace$
\left(v,u\right)$} je 
zn\'azorn\v ena na obr\'azku a prov\'ad\'\i\ ji n\'asleduj\'\i c\'\i\ algoritmus.  
skip

\midinsert
\centerline{\input fig1.tex}
\botcaption{Obr. 1}
\endcaption
\endinsert
\medskip

{\bf Rotace$\left(v,u\right)$\newline} 
$\otec\left(u\right):=\otec\left(v\right)$,\newline 
{\bf if} $v=\levy\left(\otec\left(v\right)\right)$ {\bf then}\newline 
\phantom{---}$\levy\left(\otec\left(v\right)\right):=u$\newline 
{\bf else}\newline 
\phantom{---}$\pravy\left(\otec\left(v\right)\right):=u$\newline 
{\bf endif\newline 
$\otec\left(v\right):=u$\newline 
if} $u=\levy\left(v\right)$ {\bf then}\newline 
\phantom{---}$\otec\left(\pravy\left(u\right)\right):=v$, $\levy\left(v\right):=\pravy\left(u\right)$, $\pravy\left(u\right):=v$\newline 
{\bf else}\newline 
\phantom{---}$\otec\left(\levy\left(u\right)\right):=v$, $\pravy\left(v\right):=\levy\left(u\right)$, $\levy\left(u\right):=v$\newline 
{\bf endif}
\medskip

\flushpar V\v simn\v eme si, \v ze p\v ri {\bf Rotace} m\accent23u\v zeme 
aktualizovat i funkci $p$. Pro vrchol $w\ne u,v$ se jej\'\i\ hodnota 
nem\v en\'\i , nov\'a hodnota $p\left(u\right)$ je rovn\'a p\accent23uvodn\'\i\ hodnot\v e 
$p\left(v\right)$ a novou hodnotu $p\left(v\right)$ dostaneme jako 
$p\left(\levy\left(v\right)\right)+p\left(\pravy\left(v\right)\right)$.
\medskip

\flushpar M\v ejme vrchol $w$ stromu $T$, jeho syna $v$ a jeho 
syna $u$ takov\'eho, \v ze $u$ nen\'\i\ list a $v$ je prav\'y syn 
vrcholu $w$, pr\'av\v e kdy\v z $u$ je lev\'y syn vrcholu $v$. Pak 
{\bf Dvojita}-{\bf rotace$\left(w,v,u\right)$} je zn\'azorn\v ena na obr\'azku a prov\'ad\'\i\ 
ji n\'asleduj\'\i c\'\i\ algoritmus.

\midinsert
\centerline{\input fig2.tex}
\botcaption{Obr. 2}
\endcaption
\endinsert
\medskip

{\bf Dvojita}-{\bf rotace$\left(w,v,u\right)$\newline 
$\otec\left(u\right):=\otec\left(w\right)$\newline 
if} $w=\levy\left(\otec\left(w\right)\right)$ {\bf then}\newline 
\phantom{---}$\levy\left(\otec\left(w\right)\right):=u$\newline 
{\bf else}\newline 
\phantom{---}$\pravy\left(\otec\left(w\right)\right):=u$\newline 
{\bf endif}\newline 
$\otec\left(v\right):=u$, $\otec\left(w\right):=u$\newline 
{\bf if} $v=\levy\left(w\right)$ {\bf then}\newline  
\phantom{---}$\levy\left(w\right):=\pravy\left(u\right)$, $\otec\left(\pravy\left(u\right)\right):=w$, $\pravy\left(v\right):=\levy\left(u\right)$\newline \phantom{---}$\otec\left(\levy\left(u\right)\right):=v$, $\levy\left(u\right):=v$, $\pravy\left(u\right):=w$\newline 
{\bf else}\newline 
\phantom{---}$\pravy\left(w\right):=\levy\left(u\right)$, $\otec\left(\levy\left(u\right)\right):=w$, $\levy\left(v\right):=\pravy\left(u\right)$\newline 
\phantom{---}$\otec\left(\pravy\left(u\right)\right):=v$, $\levy\left(u\right):=w$, $\pravy\left(u\right):=v$\newline 
{\bf endif}
\medskip

\flushpar Tak\'e zde m\accent23u\v zeme v \v case $O\left(1\right)$ spo\v c\'\i tat nov\'e hodnoty $
p$. 
Pro vrchol $x\ne u,v,w$ se hodnota nem\v en\'\i , nov\'a hodnota 
$p\left(u\right)$ je 
rovn\'a p\accent23uvodn\'\i\ hodnot\v e $p\left(w\right)$ a nov\'e hodnoty $
p\left(v\right)$ 
a $p\left(w\right)$ 
z\'\i sk\'ame pod\-le stejn\'eho vzorce  jako v {\bf Rotace}.

\heading
AVL-stromy
\endheading

\flushpar Bin\'arn\'\i\ vyhled\'avac\'\i\ strom je 
\emph{AVL}-\emph{strom}, kdy\v z pro ka\v zd\'y vnit\v r\-n\'\i\ vrchol 
$v$ se d\'elka nejdel\v s\'\i\ cesty z jeho lev\'eho syna do 
listu a d\'elka nejdel\v s\'\i\ cesty z jeho prav\'eho syna do listu 
li\v s\'\i\ nejv\'y\v se o $1$.  
\medskip

\flushpar Pro vnit\v rn\'\i\ vrchol $v$ stromu $T$ ozna\v cme $\eta 
\left(v\right)$ 
d\'elku nejdel\v s\'\i\ cesty z vrcholu $v$ do listu.
\medskip
 
\flushpar Struktura vnit\v rn\'\i ch vrchol\accent23u v 
AVL-stromech je roz\v s\'\i\v rena o hodnotu $\omega$:\newline 
$\omega \left(v\right)=-1$, kdy\v z 
$$\eta \left(\text{\rm lev\'y syn vrcholu }v\right)=\eta \left(\text{\rm prav\'y syn vrcholu }
v\right)+1;$$
$\omega \left(v\right)=0$, kdy\v z 
$$\eta \left(\text{\rm lev\'y syn vrcholu }v\right)=\eta \left(\text{\rm prav\'y syn vrcholu }
v\right);$$
$\omega \left(v\right)=+1$, kdy\v z
$$\eta \left(\text{\rm lev\'y syn vrcholu }v\right)+1=\eta \left(\text{\rm prav\'y syn vrcholu }
v\right).$$
\medskip

\flushpar V\v simn\v eme si, \v ze hodnota $\eta \left(v\right)$ pro vnit\v rn\'\i\ vrcholy 
$v$ stromu $T$ nen\'\i\ nikde ulo\v zena.  Hodnoty $\eta$ jsme schopni 
spo\v c\'\i tat z hodnot $\omega$, ale nen\'\i\ to t\v reba.  Sta\v c\'\i , kdy\v z po 
aktualiza\v cn\'\i ch operac\'\i ch budeme um\v et aktualizovat 
hodnoty $\omega$ a upravit bin\'arn\'\i\ vy\-hled\'avac\'\i\ strom tak, 
aby byl op\v et AVL-strom.
\medskip

\flushpar Odhad velikosti $\eta \left(\text{\rm ko\v ren }T\right)$ v z\'avislosti na 
velikosti reprezentovan\'e mno\v ziny $S$.\newline 
Kdy\v z $T$ je AVL-strom a $v$ je vnit\v rn\'\i\ vrchol $T$, pak 
podstrom $T$ ur\v cen\'y vrcholem $v$ je op\v et AVL-strom. 
Ozna\v cme\newline 
$mn\left(i\right)$ velikost nejmen\v s\'\i\ mno\v ziny reprezentovan\'e 
AVL-stromem $T$ takov\'ym, \v ze $$\eta \left(\text{\rm ko\v ren }T\right)=i,$$ 
$mx\left(i\right)$ velikost nejv\v et\v s\'\i\ mno\v ziny reprezentovan\'e 
AVL-stromem $T$ takov\'ym, \v ze $$\eta \left(\text{\rm ko\v ren }T\right)=i.$$ 
Z definice AVL-stromu plynou rekurze
$$\gather mn\left(i\right)=mn\left(i-1\right)+mn\left(i-2\right)+1,\,mx\left(i\right)=2mx\left(i-1\right)+1,\\
\text{a }mn\left(1\right)=mx\left(1\right)=1,\quad mn\left(2\right)=2,\quad mx\left(2\right)=3.\endgather$$
Nejprve spo\v c\'\i t\'ame $mx$.
\medskip

\flushpar Dok\'a\v zeme, \v ze $mx\left(i\right)=2^i-1$. Tento vzorec je 
spln\v en pro $i=1,2$. D\'ale 
$$mx\left(i+1\right)=2mx\left(i\right)+1=2\left(2^i-1\right)+1=2^{i+1}-1.$$
T\'\i m je vzorec dok\'az\'an. 
\medskip

\flushpar Abychom spo\v c\'\i tali $mn$, p\v ripomeneme si definici Fibonacciho \v c\'\i sel. \emph{Fibonacciho} 
\emph{\v c\'\i slo} $F_i$ je definov\'ano rekurenc\'\i
$$F_1=F_2=1\text{\rm\ a }F_{i+2}=F_i+F_{i+1}\text{\rm\ pro v\v sechna }
i\ge 3.$$
Pak plat\'\i\ vzorec $F_i=\frac {\left(\frac {1+\sqrt 5}2\right)^i-
\left(\frac {1-\sqrt 5}2\right)^i}{\sqrt 5}$ pro v\v sechna $i\ge 1$ 
(dok\'a\v zeme si ho v \v c\'asti o hald\'ach). 
\medskip

\flushpar Proto\v ze 
$-1<\frac {1-\sqrt 5}2<0$ a $\frac {1+\sqrt 5}2>1$, dost\'av\'ame, \v ze 
$$\lim_{n\mapsto\infty}F_n\sqrt 5\left(\frac {1+\sqrt 5}2\right)^{-n}=1.$$
Proto existuj\'\i\ konstanty $0<c_1<c_2$ takov\'e, \v ze 
$$c_1\left(\frac {1+\sqrt 5}2\right)^i<\sqrt 5F_i<c_2\left(\frac {1+\sqrt 5}2\right)^i.$$
\medskip

\flushpar Dok\'a\v zeme, \v ze $mn\left(i\right)=F_{i+2}-1$. Proto\v ze $F_3
=2$ a 
$F_4=3$, 
tvrzen\'\i\ plat\'\i\ pro $i=1$ a $i=2$. D\'ale 
$$\align mn\left(i+2\right)=&mn\left(i+1\right)+mn\left(i\right)+1=\\&F_{i+3}-1+F_{i+2}-1+1=F_{i+4}-1.\endalign$$
Z toho indukc\'\i\ plyne po\v zadovan\'y vztah. 
\medskip

\flushpar Kdy\v z AVL-strom $T$ o v\'y\v sce $i$ reprezentuje mno\v zinu $
S$ o 
velikosti $n$, pak plat\'\i\ 
$$\frac {c_1}{\sqrt 5}\left(\frac {1+\sqrt 5}2\right)^{i+2}-1<F_{i+2}-1\le n
\le 2^i-1.$$
Po zlogaritmov\'an\'\i\ z toho okam\v zit\v e dost\'av\'ame
$$\log\left(\frac {c_1}{\sqrt 5}\right)+\left(i+2\right)\log\left(\frac {1+\sqrt 5}2\right)<\log\left(n
+1\right)<i.$$
Proto\v ze $\log\left(\frac {1+\sqrt 5}2\right)\approx 0.69\approx\frac 1{1.
44}$ dost\'av\'ame, \v ze pro dostate\v cn\v e velk\'a $n$ 
plat\'\i , \v ze $0.69i<\log\left(n+1\right)\le i$. Odtud plyne, \v ze 
$\log\left(n+1\right)\le i\le 1.44\log\left(n\right)$, a tedy $i=\Theta \left(\log\left(n\right)\right)$.
\medskip

\flushpar Operace {\bf MEMBER$\left(x\right)$} pro AVL-stromy je stejn\'a jako 
ope\-race {\bf MEMBER$\left(x\right)$} pro nevy\-v\'a\-\v zen\'e bin\'arn\'\i\ vyhled\'avac\'\i\ 
stro\-my. Aktualiza\v cn\'\i\ operace pro AVL-stromy nejprve provedou 
p\v r\'\i slu\v snou ope\-raci pro nevyv\'a\v zen\'e bin\'arn\'\i\ vyhled\'avac\'\i\ stromy 
a pak n\'asle\-du\-je jejich vyva\v zovac\'\i\ \v c\'ast. P\v ri \'usp\v e\v sn\v e 
proveden\'e operaci {\bf IN\-SERT$\left(x\right)$} v nevyv\'a\v zen\'ych bin\'arn\'\i ch 
stromech zm\v en\'\i me vhod\-n\'y list $t$ na vnit\v rn\'\i\ vrchol 
stromu reprezentuj\'\i c\'\i\ $x$ a p\v rid\'ame k $t$ dva syny, kte\v r\'\i\ 
budou listy. D\accent23usledkem je, \v ze definujeme 
$\omega \left(t\right)=0$. Proto\v ze se v\v sak zv\v et\v sila hodnota $\eta 
\left(t\right)$ (bylo 
$\eta \left(t\right)=0$ a te\v d je $\eta \left(t\right)=1$), zavol\'ame proceduru 
{\bf Kontrola-INSERT$\left(t\right)$}, kter\'a zajist\'\i\ spr\'avnou hodnotu 
funkce $\omega$ pro otce $t$. Nav\'\i c, kdy\v z zjist\'\i , \v ze se zv\v et\v sila 
hodnota $\eta$ vrcholu otce $t$, pak zavol\'a sama sebe na vrchol 
otec $t$. Nejprve provedeme anal\'yzu situace. 
\medskip

\flushpar M\v ejme vrchol $t$, jeho $\eta \left(t\right)=a$ (ale $a$ nezn\'ame), na 
za\v c\'atku 
ope\-ra\-ce {\bf INSERT} bylo $\eta \left(t\right)=a-1$.  V podstromu ur\v cen\'em 
vrcholem $t$ m\'ame u\v z spr\'avn\'e hodnoty $\omega$.  Vrchol $
v$ je otcem $t$, 
$t=\levy\left(v\right)$ a $\omega \left(v\right)$ m\'a je\v st\v e p\accent23uvodn\'\i\ hodnotu.  
\medskip

\proclaim{Lemma}Kdy\v z se hodnota $\eta \left(t\right)$ p\v ri operaci {\bf INSERT }
zv\v et\v sila a $t$ nebyl listem p\v red operaci, pak po operaci 
neplat\'\i\ $\omega \left(t\right)=0$.
\endproclaim

\flushpar Ozna\v cme $u=\pravy\left(v\right)$ a uva\v zme p\v r\'\i pady:\newline 
A) $\omega \left(v\right)=1$, pak $\eta \left(u\right)=a$ a $\eta \left(v\right)=a+1$ se nezm\v enilo, 
tedy sta\v c\'\i\ polo\v zit $\omega \left(v\right)=0$.\newline 
B) $\omega \left(v\right)=0$, pak $\eta \left(u\right)=a-1$ a $\eta \left(v\right)=a+1$ se zm\v enilo, 
tedy mus\'\i me polo\v zit $\omega \left(v\right)=-1$ a zavolat proceduru 
{\bf Kontrola-INSERT} na vrchol $v$.\newline 
C) $\omega \left(v\right)=-1$, pak $\eta \left(u\right)=a-2$ a $\eta \left(v\right)=a+1$ se zm\v enilo. 
Nyn\'\i\ $\omega \left(v\right)=-2$ a to je zak\'azan\'e. Ozna\v cme $t_1=\levy
\left(t\right)$, 
$t_2=\pravy\left(t\right)$ ($\omega \left(t\right)=0$ nenastane, viz Lemma). 
\newline 
C1) $\omega \left(t\right)=-1$, pak $\eta \left(t_1\right)=a-1$, $\eta \left(t_2\right)=a-2$ a provedeme 
{\bf Rotace$\left(v,t\right)$}. Pak $t_2$ je druh\'y syn $v$ a sta\v c\'\i\ polo\v zit 
$\omega \left(v\right)=\omega \left(t\right)=0$.\newline 
C2) $\omega \left(t\right)=1$, pak $\eta \left(t_1\right)=a-2$, $\eta \left(t_2\right)=a-1$ a provedeme 
{\bf Dvojita-rotace$\left(v,t,t_2\right)$}. Pro $t_3=\levy\left(t_2\right)$ a $t_4=\pravy
\left(t_2\right)$ 
plat\'\i :\newline 
C2i) $\omega \left(t_2\right)=1\implies\eta \left(t_3\right)=a-3$ a $\eta \left(t_4\right)=a-2$ a sta\v c\'\i\ 
polo\v zit $\omega \left(t\right)=-1$, $\omega \left(v\right)=\omega \left(t_2\right)=0$, proto\v ze $
\eta \left(t_2\right)=a$.\newline 
C2ii) $\omega \left(t_2\right)=0\implies\eta \left(t_3\right)=\eta \left(t_4\right)=a-2$ a sta\v c\'\i\ polo\v zit 
$\omega \left(t_2\right)=\omega \left(v\right)=\omega \left(t\right)=0$, proto\v ze $\eta \left(t_2\right)=a$. \newline 
C2iii) $\omega \left(t_2\right)=-1\implies\eta \left(t_3\right)=a-2$ a $\eta \left(t_4\right)=a-3$ a 
sta\v c\'\i\ polo\v zit $\omega \left(v\right)=1$, $\omega \left(t_2\right)=\omega \left(t\right)=
0$, proto\v ze $\eta \left(t_2\right)=a$.\newline 
Kdy\v z $t$ je prav\'y syn $v$, pak situace je symetrick\'a.\newline 
Pop\'\i\v seme proceduru {\bf Kontrola-INSERT}.
\medskip

{\bf Kontrola-INSERT$\left(t\right)$\newline 
$v:=\otec\left(t\right)$\newline 
if} $t=\levy\left(v\right)$ {\bf then}\newline 
\phantom{---}{\bf if} $\omega \left(v\right)=1$ {\bf then}\newline 
\phantom{------}$\omega \left(v\right):=0$ \newline 
\phantom{---}{\bf else}\newline 
\phantom{------}{\bf if} $\omega \left(v\right)=0$ {\bf then}\newline 
\phantom{---------}$\omega \left(v\right):=-1$, $t:=v$, {\bf Kontrola-INSERT}$\left(t\right)$\newline 
\phantom{------}{\bf else}\newline
\phantom{---------}{\bf if} $\omega \left(t\right)=-1$ {\bf then}\newline 
\phantom{------------}{\bf Rotace$\left(v,t\right)$}, $\omega \left(v\right):=0$, $\omega \left(t\right):=0$\newline 
\phantom{---------}{\bf else}\newline 
\phantom{------------}$w:=\pravy\left(t\right)$, {\bf Dvojita-rotace$\left(v,t,w\right)$},\newline 
\phantom{------------}{\bf if} $\omega \left(w\right)=0$ {\bf then}\newline 
\phantom{---------------}$\omega \left(t\right):=0$, $\omega \left(v\right):=0$\newline 
\phantom{------------}{\bf else}\newline 
\phantom{---------------}{\bf if} $\omega \left(w\right)=1$ {\bf then}\newline 
\phantom{------------------}$\omega \left(v\right):=0$, $\omega \left(t\right):=-1$\newline 
\phantom{---------------}{\bf else}\newline 
\phantom{------------------}$\omega \left(v\right):=1$, $\omega \left(t\right):=0$\newline 
\phantom{---------------}{\bf endif}\newline 
\phantom{------------}{\bf endif}\newline 
\phantom{------------}$\omega \left(w\right):=0$\newline 
\phantom{---------}{\bf endif}\newline
\phantom{------}{\bf endif}\newline 
\phantom{---}{\bf endif\newline 
else}\newline 
\phantom{---}{\bf if} $\omega \left(v\right)=-1$ {\bf then}\newline 
\phantom{------}$\omega \left(v\right):=0$\newline 
\phantom{---}{\bf else}\newline 
\phantom{------}{\bf if} $\omega \left(v\right)=0$ {\bf then}\newline 
\phantom{---------}$\omega \left(v\right):=1$, $t:=v$, {\bf Kontrola-INSERT}$\left(t\right)$\newline 
\phantom{------}{\bf else}\newline 
\phantom{---------}{\bf if} $\omega \left(t\right)=1$ {\bf then}\newline 
\phantom{------------}{\bf Rotace$\left(v,t\right)$}, $\omega \left(v\right):=0$, $\omega \left(t\right):=0$\newline 
\phantom{---------}{\bf else}\newline 
\phantom{------------}$w:=\levy\left(t\right)$, {\bf Dvojita-rotace$\left(v,t,w\right)$},\newline 
\phantom{------------}{\bf if} $\omega \left(w\right)=0$ {\bf then}\newline 
\phantom{---------------}$\omega \left(t\right):=0$, $\omega \left(v\right):=0$\newline 
\phantom{------------}{\bf else}\newline 
\phantom{---------------}{\bf if} $\omega \left(w\right)=1$ {\bf then}\newline 
\phantom{------------------}$\omega \left(v\right):=0$, $\omega\left(t\right):=-1$\newline 
\phantom{---------------}{\bf else}\newline 
\phantom{------------------}$\omega \left(v\right):=1$, $\omega \left(t\right):=0$\newline 
\phantom{---------------}{\bf endif}\newline 
\phantom{------------}{\bf endif}\newline 
\phantom{------------}$\omega \left(w\right):=0$\newline 
\phantom{---------}{\bf endif}\newline 
\phantom{------}{\bf endif}\newline 
\phantom{---}{\bf endif\newline 
endif}
skip

\flushpar V\v simn\v eme si, \v ze po proveden\'\i\ {\bf Rotace} nebo
{\bf Dvojita-rotace} vyva\v zov\'an\'\i\ v operaci {\bf INSERT} kon\v c\'\i . Tedy 
operace {\bf INSERT} prov\'ad\'\i\ nejv\'y\v se jednu proceduru {\bf Rotace }
nebo {\bf Dvojita-rota\-ce}. Korektnost vyva\v zovac\'\i\ operace je 
zalo\v zena na faktu, \v ze kdy\v z se zv\v et\v s\'\i\ hodnota $
\eta \left(t\right)$, pak 
nem\accent23u\v ze b\'yt $\omega \left(t\right)=0$. Tento fakt se vyu\v z\'\i v\'a  v 
{\bf if}-p\v r\'\i kazu na 5-t\'em a 6-t\'em \v r\'adku a na 13-t\'em a 14-t\'em  
\v r\'adku programu.
\medskip

\flushpar Pop\'\i\v seme vyva\v zovac\'\i\ operaci pro operaci 
{\bf DELETE}.  P\v red\-po\-kl\'adejme, \v ze $t$ je vrchol, jeho\v z otec se 
odstranil (tj.  bratr $t$ byl list) a hodnota $\eta \left(t\right)$ je men\v s\'\i\ 
ne\v z byla hodnota $\eta \left(\otec\left(t\right)\right)$.  Proto zavol\'ame proceduru 
{\bf Kontrola-DELETE$\left(t\right)$}.  Tato procedura zajist\'\i\ spr\'avnou 
hodnotu funkce $\omega$ pro otce $t$.  Nav\'\i c, kdy\v z zjist\'\i , \v ze se 
zmen\v sila hodnota $\eta$ vrcholu otce $t$, pak zavol\'a sama sebe 
na vrchol otec $t$. Pop\'\i\v seme anal\'yzu situace, na n\'\i\v z je 
zalo\v zena korektnost procedury {\bf Kontrola-DELETE$\left(t\right)$}.  
\medskip

\flushpar V anal\'yze je d\accent23ule\v zit\'e, \v ze kdy\v z procedura {\bf Kontrola-DELETE }
p\v resune vrchol $x$ na m\'\i sto vrcholu $y$, pak skute\v cn\'a hodnota $
\eta \left(x\right)$ 
je bu\v d p\accent23uvodn\'\i\ hodnota $\eta \left(y\right)$ nebo je p\v resn\v e o $
1$ men\v s\'\i . V\v simn\v ete si, \v ze to plat\'\i . 
\medskip

\flushpar D\'an vrchol $t$, jeho\v z hodnota $\eta \left(t\right)$ se zmen\v sila (o $
1$).  V 
podstromu ur\v cen\'em vrcholem $t$ jsou hodnoty $\omega$ aktualizov\'any, 
$v=\otec\left(t\right)$ a $\omega \left(v\right)$ je p\accent23uvodn\'\i .  P\v redpokl\'adejme 
$t=\levy\left(v\right)$, $u=\pravy\left(v\right)$ a $\eta \left(t\right)=a$ ($a$ je nezn\'am\'e).  Nast\'avaj\'\i\ 
p\v r\'\i pady:\newline 
A) kdy\v z $\omega \left(v\right)=1$, pak $\eta \left(u\right)=a+2$ a $\eta \left(v\right)=a+3$ (p\accent23uvodn\v e 
bylo $\eta \left(t\right)=a+1$).  Ozna\v cme $u_1=\levy\left(u\right)$, $u_2=\pravy\left(u\right)$.\newline 
A1) $\omega \left(u\right)=1\implies\eta \left(u_1\right)=a,\,\eta \left(u_2\right)=a+1$.  Provedeme 
{\bf Rotace$\left(v,u\right)$}.  
Vrchol $u_1$ je druh\'ym synem $v$ a plat\'\i\ $\eta \left(t\right)=\eta \left(u_
1\right)=a$, 
$\eta \left(v\right)=\eta \left(u_2\right)=a+1$ a $\eta \left(u\right)=a+2$.  Tedy polo\v zme $\omega 
\left(v\right)=\omega \left(u\right)=0$ a 
zavolejme {\bf Kontrola-DELETE} na vrchol $u$.\newline 
A2) $\omega \left(u\right)=0\implies\eta \left(u_1\right)=\eta \left(u_2\right)=a+1$. Provedeme 
{\bf Rotace$\left(v,u\right)$}. Vrchol $u_1$ je druh\'ym synem $v$ a plat\'\i\ 
$\eta \left(t\right)=a$, $\eta \left(u_1\right)=a+1=\eta \left(u_2\right)$, $\eta \left(v\right)=a+2$, $\eta 
\left(u\right)=a+3$. Polo\v zme 
$\omega \left(v\right)=1$, $\omega \left(u\right)=-1$ a kon\v c\'\i me.
\medskip
 
\flushpar A3) $\omega \left(u\right)=-1\implies\eta \left(u_1\right)=a+1,\,\eta \left(u_2\right)=a$.  Provedeme 
{\bf Dvojita-rotace$\left(v,u,u_1\right)$}.  Pro $u_3=\levy\left(u_1\right)$, 
$u_4=\pravy\left(u_1\right)$ nastanou p\v r\'\i pady:\newline 
A3i) $\omega \left(u_1\right)=-1\implies\eta \left(u_3\right)=a,\,\eta \left(u_4\right)=a-1$ a tedy 
$\eta \left(v\right)=\eta \left(u\right)=a+1$ a $\eta \left(u_1\right)=a+2$. Proto polo\v z\'\i me 
$\omega \left(v\right)=\omega \left(u_1\right)=0$, $\omega \left(u\right)=1$ a zavol\'ame proceduru {\bf Kontrola-DELETE }
na vrchol $u_1$.\newline 
A3ii) $\omega \left(u_1\right)=0\implies\eta \left(u_3\right)=\eta \left(u_4\right)=a$ a tedy 
$\eta \left(v\right)=\eta \left(u\right)=a+1$ a $\eta \left(u_1\right)=a+2$. Proto polo\v z\'\i me 
$\omega \left(v\right)=\omega \left(u_1\right)=\omega \left(u\right)=0$ a zavol\'ame proceduru {\bf Kontrola-DELETE }
na vrchol $u_1$.\newline 
A3iii) $\omega \left(u_1\right)=1\implies\eta \left(u_3\right)=a-1,\,\eta \left(u_4\right)=a$ a tedy 
$\eta \left(v\right)=\eta \left(u\right)=a+1$ a $\eta \left(u_1\right)=a+2$. Proto polo\v z\'\i me 
$\omega \left(u\right)=\omega \left(u_1\right)=0$, $\omega \left(v\right)=-1$ a zavol\'ame proceduru {\bf Kontrola-DELETE }
na vrchol $u_1$.\newline 
B) kdy\v z $\omega \left(v\right)=0$, pak $\eta \left(u\right)=a+1$ a $\eta \left(v\right)=a+2$. 
Sta\v c\'\i\ polo\v zit $\omega \left(v\right)=1$ a skon\v cit.\newline 
C) kdy\v z $\omega \left(v\right)=-1$, pak $\eta \left(u\right)=a$ a $\eta \left(v\right)=a+2$. 
Nyn\'\i\ polo\v z\'\i me $\omega \left(v\right)=0$ a 
zavol\'ame proceduru {\bf Kontrola-DELETE} na vrchol $v$.
\medskip

{\bf Kontrola-DELETE$\left(t\right)$\newline 
$v:=\otec\left(t\right)$\newline 
if} $t=\levy\left(v\right)$ {\bf then}\newline 
\phantom{---}{\bf if} $\omega \left(v\right)=1$ {\bf then}\newline 
\phantom{------}$u:=\pravy\left(v\right)$\newline 
\phantom{------}{\bf if} $\omega \left(u\right)\ge 0$ {\bf then}\newline 
\phantom{---------}{\bf Rotace$\left(v,u\right)$}\newline 
\phantom{---------}{\bf if} $\omega \left(v\right)=0$ {\bf then}\newline 
\phantom{------------}$\omega \left(v\right):=1$, $\omega \left(u\right):=-1$\newline 
\phantom{---------}{\bf else}\newline 
\phantom{------------}$\omega \left(u\right):=\omega \left(v\right):=0$, $t:=u$, {\bf Kontrola-DELETE}$\left(t\right)$\newline 
\phantom{---------}{\bf endif}\newline 
\phantom{------}{\bf else}\newline 
\phantom{---------}$w:=\levy\left(u\right)$, {\bf Dvojita-rotace}$\left(v,u,w\right)$\newline 
\phantom{---------}{\bf if} $\omega \left(w\right)=1$ {\bf then}\newline 
\phantom{------------}$\omega \left(u\right):=0$, $\omega \left(v\right):=-1$\newline 
\phantom{---------}{\bf else}\newline 
\phantom{------------}{\bf if} $\omega \left(w\right):=0$ {\bf then}\newline 
\phantom{---------------}$\omega \left(u\right):=0$, $\omega\left(v\right):=0$\newline 
\phantom{------------}{\bf else}\newline 
\phantom{---------------}$\omega \left(u\right):=1$, $\omega \left(v\right):=0$\newline 
\phantom{------------}{\bf endif}\newline 
\phantom{---------}{\bf endif}\newline 
\phantom{---------}$\omega \left(w\right):=0$, $t:=w$, {\bf Kontrola-Delete}$\left(t\right)$\newline 
\phantom{------}{\bf endif}\newline 
\phantom{---}{\bf endif\newline 
else}\newline 
\phantom{---}{\bf if} $\omega \left(v\right)=0$ {\bf then}\newline 
\phantom{------}$\omega \left(v\right):=1$\newline 
\phantom{---}{\bf else}\newline 
\phantom{------}$\omega \left(v\right):=0$, $t:=v$, {\bf Kontrola-DELETE}$\left(t\right)$\newline \phantom{---}{\bf endif\newline 
else}\newline 
\phantom{---}{\bf if} $\omega \left(v\right)=-1$ {\bf then}\newline 
\phantom{------}$u:=\levy\left(v\right)$\newline 
\phantom{------}{\bf if} $\omega \left(u\right)\le 0$ {\bf then}\newline 
\phantom{---------}{\bf Rotace}$\left(v,u\right)$\newline 
\phantom{---------}{\bf if} $\omega \left(u\right)=0$ {\bf then}\newline 
\phantom{------------}$\omega \left(v\right):=-1$, $\omega \left(u\right):=1$\newline 
\phantom{---------}{\bf else}\newline 
\phantom{------------}$\omega \left(u\right):=\omega \left(v\right):=0$, $t:=u$, {\bf Kontrola-DELETE}$\left(t\right)$\newline 
\phantom{---------}{\bf endif}\newline 
\phantom{------}{\bf else}\newline 
\phantom{---------}$w:=\pravy\left(u\right)$, {\bf Dvojita-rotace$\left(v,u,w\right)$}\newline 
\phantom{---------}{\bf if} $\omega \left(w\right)=1$ {\bf then}\newline 
\phantom{------------}$\omega \left(u\right):=-1$, $\omega \left(v\right):=0$\newline 
\phantom{---------}{\bf else}\newline 
\phantom{------------}{\bf if} $\omega \left(w\right):=0$ {\bf then}\newline 
\phantom{---------------}$\omega \left(u\right):=0$, $\omega\left(v\right):=0$\newline 
\phantom{------------}{\bf else}\newline 
\phantom{---------------}$\omega \left(u\right):=0$, $\omega \left(v\right):=1$\newline 
\phantom{------------}{\bf endif}\newline 
\phantom{---------}{\bf endif}\newline 
\phantom{---------}$\omega \left(w\right):=0$, $t:=w$, {\bf Kontrola-Delete}$\left(t\right)$\newline
\phantom{------}{\bf endif}\newline 
\phantom{---}{\bf else}\newline 
\phantom{------}{\bf if} $\omega \left(v\right)=0$ {\bf then}\newline 
\phantom{---------}$\omega \left(v\right):=-1$\newline 
\phantom{------}{\bf else}\newline 
\phantom{---------}$\omega \left(v\right):=0$, $t:=v$, {\bf Kontrola-DELETE}$\left(t\right)$\newline \phantom{------}{\bf endif}\newline 
\phantom{---}{\bf endif\newline 
endif}
\medskip

\flushpar V operaci {\bf DELETE} se m\accent23u\v ze st\'at, \v ze procedury 
{\bf Rotace} nebo {\bf Dvojita-rotace} jsou vol\'any a\v z $\log\left(
|S|\right)$-kr\'at. 
To je v\'yrazn\'y rozd\'\i l proti operaci {\bf INSERT}. Proto operace 
{\bf DELETE} je pomalej\v s\'\i\ ne\v z operace {\bf INSERT}, i kdy\v z 
asymptoticky jsou stejn\v e rychl\'e. Korektnost se 
ov\v e\v r\'\i\ p\v r\'\i mo.

\proclaim{V\v eta}Datov\'a struktura AVL-strom 
umo\v z\v nuje implementaci operac\'\i\ {\bf MEMBER}, {\bf INSERT} a 
{\bf DELETE}, kter\'e vy\v zaduj\'\i\ \v cas $O\left(\log\left(|S|\right)\right)$ (kde $
S$ je 
reprezentovan\'a mno\v zina).  Ope\-race {\bf INSERT} zavol\'a 
nejv\'y\v se jednu proceduru {\bf Rotace} nebo {\bf Dvo\-ji\-ta-rotace}.  
\endproclaim
\medskip

\heading 
\v Cerveno-\v cern\'e stromy
\endheading

\flushpar Bin\'arn\'\i\ vyhled\'avac\'\i\ strom $T$ reprezentuj\'\i c\'\i\ 
mno\v zinu $S$, jeho\v z vrcholy jsou obarveny \v cerven\v e nebo 
\v cern\v e (ka\v zd\'y vrchol m\'a pr\'av\v e jednu barvu) tak, \v ze jsou 
spln\v eny podm\'\i nky:
\roster
\item"{}"
listy jsou obarveny \v cern\v e,
\item"{}"
kdy\v z $v$ je vrchol obarven\'y \v cerven\v e, pak je bu\v d ko\v ren 
stromu nebo jeho otec je obarven \v cern\v e,
\item"{}"
v\v sechny cesty z ko\v rene do list\accent23u maj\'\i\ stejn\'y 
po\v cet \v cer\-n\'ych vrchol\accent23u
\endroster
se naz\'yv\'a \emph{\v cerveno}-\emph{\v cern\'y} \emph{strom}. 
\medskip

\flushpar Nejprve uk\'a\v zeme, \v ze \v cerveno-\v cern\'e stromy jsou 
vyv\'a\v zen\'e stro\-my, tj. $$\hloubka\left(T\right)=O\left(\log\left(|S|\right)\right).$$
P\v redpokl\'adejme, \v ze $T$ je \v cerveno-\v cern\'y strom, kter\'y m\'a 
na cest\v e z ko\v rene do listu pr\'av\v e $k$ \v cern\'ych 
vrchol\accent23u. Pak pro po\v cet vrchol\accent23u $\#T$ 
stromu $T$ plat\'\i\ 
$$2^k-1\le \#T\le 2^{2k}-1.$$
Nejmen\v s\'\i\ takov\'y strom m\'a v\v sechny vrcholy \v cern\v e 
obarven\'e a je to \'upln\'y pravideln\'y bin\'arn\'\i\ strom o 
v\'y\v sce $k-1$, co\v z d\'av\'a doln\'\i\ odhad.  Nejv\v et\v s\'\i\ takov\'y 
strom m\'a v\v sechny vrcholy v sud\'ych hladin\'ach obarveny 
\v cerven\v e a v lich\'ych hladin\'ach \v cern\v e, je to \'upln\'y 
pravideln\'y bin\'arn\'\i\ strom o v\'y\v sce $2k-1$ a t\'\i m je d\'an 
horn\'\i\ odhad. Tedy $k\le\log\left(1+\#T\right)\le 2k$.
Proto\v ze velikost $S$ je po\v cet vnit\v rn\'\i ch vrchol\accent23u, 
dost\'av\'ame, \v ze $\#T=2|S|+1$. Z vlastnost\'\i\ \v cerveno-\v cern\'ych 
strom\accent23u plyne, \v ze 
$$k\le\hloubka\left(T\right)\le 2k.$$
Tedy
\medskip

\proclaim{Tvrzen\'\i}Kdy\v z \v cerveno-\v cern\'y strom $T$ 
reprezentuje mno\v zinu $S$, pak $$\hloubka\left(T\right)\le 2\log\left(2|S|+2\right)=1+\log\left(|S|+1\right).$$
\endproclaim

\flushpar Pro \v cerveno-\v cern\'e stromy navrhneme algoritmy 
realizuj\'\i c\'\i\ ope\-race z uspo\v r\'adan\'eho slovn\'\i\-ko\-v\'eho probl\'emu.  
Operace {\bf MEMBER} pro \v cerveno-\v cern\'e stromy je stejn\'a jako pro 
nevyv\'a\v zen\'e bin\'arn\'\i\ vyhled\'avac\'\i\ stromy. Operace 
{\bf IN\-SERT} a {\bf DELETE} maj\'\i\ dv\v e \v c\'asti:  nejprve se provede 
ope\-race {\bf INSERT} nebo {\bf DELETE} pro nevyv\'a\v zen\'e bin\'arn\'\i\ 
vy\-hled\'avac\'\i\ stromy a pak n\'asleduj\'\i\ vyva\v zovac\'\i\ operace, kter\'e 
zajist\'\i , \v ze v\'ysledn\'y strom spl\v nuje pod\-m\'\i n\-ky pro \v cerveno-\v cern\'e 
stromy (stejn\'e sch\'ema jako pro AVL-stromy). Sch\'ema operac\'\i\ 
{\bf JOIN} a {\bf SPLIT} bude vych\'azet z jejich realizac\'\i\ v $
\left(a,b\right)$-stromech. 
V operaci {\bf JOIN} prohled\'av\'an\'\i m nalezneme m\'\i sto, kde se 
stromy  
daj\'\i\ spojit (a aplikujeme operaci {\bf JOIN} pro 
nevyv\'a\v zen\'e 
bin\'arn\'\i\ vyhled\'a\-va\-c\'\i\ stromy), a pak pou\v zijeme 
vyva\v zovac\'\i\ operace. Algoritmus ope\-race {\bf SPLIT} rozd\v el\'\i\ \v cerveno-\v cern\'y strom 
do n\v ekolika men\v s\'\i ch podle cesty vyhled\'avaj\'\i c\'\i\ $
x$ (podobn\v e jako 
v $\left(a,b\right)$-stromech) a na tyto stromy pak aplikuje operaci {\bf JOIN }
a zkonstruuje hledan\'e \v cerveno-\v cern\'e stromy. Algoritmy pro 
ope\-race {\bf MIN} a {\bf MAX} jsou stejn\'e jako pro nevyv\'a\v zen\'e bin\'arn\'\i\ 
vyhled\'avac\'\i\ stromy.
\medskip

\flushpar Nejprve pop\'\i\v seme vyva\v zovac\'\i\ operace. 
Dvojice $\left(T,v\right)$ se naz\'yv\'a 2-\emph{parci\'aln\'\i}
\emph{\v cerveno}-\emph{\v cern\'y} \emph{strom}, kdy\v z $T$ je 
bin\'arn\'\i\ vyhled\'avac\'\i\ strom, ka\v zd\'y vrchol je obarven 
\v cerven\v e nebo \v cern\v e, $v$ je 
vnit\v rn\'\i\ vrchol stromu $T$ obarven\'y \v cerven\v e a plat\'\i :
\roster
\item"{}"
listy jsou obarveny \v cern\v e,
\item"{}"
kdy\v z $t$ je vrchol obarven\'y \v cerven\v e, pak je bu\v d ko\v ren 
stromu nebo $t=v$ nebo jeho otec je obarven \v cern\v e,
\item"{}"
v\v sechny cesty z ko\v rene do list\accent23u maj\'\i\ stejn\'y 
po\v cet \v cer\-n\'ych vrchol\accent23u.
\endroster
\medskip

\flushpar Vyva\v zov\'an\'\i\ 2-parci\'aln\'\i ho \v cerveno-\v cern\'eho stromu $\left(T',v\right)$ 
pro\-v\'a\-d\'\i\ procedura {\bf Vyvaz-IN\-SERT$\left(v\right)$}.  Po jej\'\i m 
proveden\'\i\ bu\v d do\-sta\-neme \v cer\-ve\-no-\v cern\'y strom nebo 
je procedura {\bf Vyvaz-IN\-SERT} zavol\'ana na vrchol $
v'$ takov\'y, 
\v ze $\left(T',v'\right)$ je 2-parci\'aln\'\i\ \v cerve\-no-\v cern\'y strom a $
v'$ je d\v ed $v$ 
(tj.  je o dv\v e hladiny bl\'\i\v z ke ko\v reni ne\v z vrchol $
v$). 
\medskip

\flushpar Obarven\'\i\ je realizov\'ano roz\v s\'\i\v ren\'\i m struktury vrcholu $v$ 
o boolskou prom\v ennou $b\left(v\right)$, kde $b\left(v\right)=0$ znamen\'a, \v ze $v$ je obarven 
\v cerven\v e, a $b\left(v\right)=1$ znamen\'a, \v ze $v$ je obarven \v cern\v e.
\medskip

\flushpar Pop\'\i\v seme proceduru {\bf Vyvaz-INSERT$\left(v\right)$} (p\v redpokl\'ad\'ame, \v ze 
$v$ je obarven \v cerven\v e). Pro zjednodu\v sen\'\i\ $s\left(v
\right)=levy$, kdy\v z $v=\levy\left(\otec\left(v\right)\right)$, a $s\left(v\right)=pravy$ pro  
$v=\pravy\left(\otec\left(v\right)\right)$.
skip

{\bf Vyvaz-INSERT$\left(v\right)$}.\newline 
{\bf if} $v$ nen\'\i\ ko\v ren $T'$ a $b\left(\otec\left(v\right)\right)=0$ {\bf then}\newline 
\phantom{---}{\bf if} $otec\left(v\right)$ je ko\v ren {\bf then}\newline 
\phantom{------}$b\left(\otec\left(v\right)\right):=1$\newline 
\phantom{---}{\bf else}\newline 
\phantom{------}$w:=\otec\left(v\right)$, $u:=\bratr\left(w\right)$\newline 
\phantom{------}{\bf if} $b\left(u\right)=0$ {\bf then}\newline 
\phantom{---------}$v:=\otec\left(w\right)$, $b\left(w\right):=1$, $b\left(u\right):=1$\newline 
\phantom{---------}$b\left(v\right):=0$, {\bf Vyvaz-INSERT$\left(v\right)$} (Viz Obr. 1)\newline 
\phantom{------}{\bf else}\newline 
\phantom{---------}$t:=\otec\left(w\right)$\newline 
\phantom{---------}{\bf if} $s\left(w\right)=s\left(v\right)$ {\bf then}\newline 
\phantom{------------}{\bf Rotace$\left(t,w\right)$}, $b\left(t\right):=0$, $b\left(w\right):=1$ (Viz Obr. 2)\newline 
\phantom{---------}{\bf else}\newline 
\phantom{------------}{\bf Dvojita-rotace$\left(t,w,v\right)$}, $b\left(t\right):=0$, $b\left(v\right):=1$ (Viz Obr. 3)\newline 
\phantom{---------}{\bf endif}\newline 
\phantom{------}{\bf endif}\newline 
\phantom{---}{\bf endif\newline 
endif}
\medskip

\flushpar Na obr\'azku $b$ zna\v c\'\i\ \v cernou barvu a $r$ zna\v c\'\i\ \v cervenou 
barvu. Otec vrcholu $w$ je ozna\v cen $t$. 
\medskip

\midinsert
\centerline{\input fig3.tex}
\botcaption{Obr. 1}
\endcaption
\endinsert

\midinsert
\centerline{\input fig4.tex}
\botcaption{Obr. 2}
\endcaption
\endinsert
\newpage

\midinsert
\centerline{\input fig5.tex}
\botcaption{Obr. 3}
\endcaption
\endinsert

\flushpar$2$-parci\'aln\'\i\ \v cerveno-\v cern\'e stromy vznikaj\'\i\ p\v ri 
operac\'\i ch {\bf INSERT} a {\bf JOIN.}  P\v ri operaci {\bf DELETE} se poru\v s\'\i\ 
struktura \v cerveno-\v cern\'ych strom\accent23u jin\'ym 
zp\accent23usobem a vznikne 3-parci\'aln\'\i\ \v cerveno-\v cer\-n\'y strom.  
\medskip

\flushpar\v Rekneme, \v ze dvojice $\left(T,v\right)$ je \emph{3}-\emph{parci\'aln\'\i} 
\emph{\v cerveno}-\emph{\v cern\'y} \emph{strom}, kdy\v z $T$ je bin\'arn\'\i\ 
vy\-hled\'avac\'\i\ strom, ka\v zd\'emu vrcholu je p\v ri\-\v razena pr\'av\v e jedna z 
dvojice barev \v cerven\'a -- \v cern\'a, $v$ je vrchol ve stromu $
T$ a 
plat\'\i\ n\'asleduj\'\i c\'\i\ podm\'\i nky: 
\roster
\item"{}"
listy a vrchol $v$ jsou obarveny \v cern\v e,
\item"{}"
kdy\v z $t$ je vrchol obarven\'y \v cerven\v e, pak je bu\v d ko\v ren 
stromu nebo jeho otec je obarven \v cern\v e,
\item"{}"
existuje \v c\'\i slo $k$ takov\'e, \v ze v\v sechny cesty z ko\v rene do 
list\accent23u, kter\'e neobsahuj\'\i\ vrchol $v$, obsahuj\'\i\ pr\'av\v e 
$k$ \v cer\-n\'ych vrchol\accent23u, a v\v sechny cesty z ko\v rene do 
list\accent23u proch\'a\-ze\-j\'\i\-c\'\i\ vrcholem $v$ obsahuj\'\i\ $
k-1$ \v cern\'ych 
vrchol\accent23u.
\endroster
\medskip

\flushpar Pop\'\i\v seme proceduru {\bf Vyvaz-DELETE$\left(v\right)$}, kter\'a se 
pou\v zije na 3-parci\'aln\'\i\ \v cerveno-\v cern\'y strom $\left(T,v\right)$, kdy\v z $
v$ nen\'\i\ jeho 
ko\v ren.  V\'ysledkem procedury bude bu\v d \v cerveno-\v cern\'y strom 
nebo zavol\'an\'\i\ procedury {\bf Vyvaz-DELE\-TE$\left(v'\right)$}, kde $v'$ je otcem 
vrcholu $v$.  Z faktu, \v ze kdy\v z $\left(T,v\right)$ je 3-parci\'aln\'\i\ \v cerveno-\v cern\'y 
strom a $v$ je jeho ko\v ren, pak $T$ je \v cerveno-\v cern\'y strom, plyne, 
\v ze aplikac\'\i\ {\bf Vyvaz-DELETE$\left(v\right)$} na 3-parci\'aln\'\i\ \v cerveno-\v cern\'y strom 
$\left(T,v\right)$ dostaneme \v cerveno-\v cern\'y strom.  
\medskip

{\bf Vyvaz-DELETE$\left(v\right)$\newline 
$u:=\bratr\left(v\right)$}, $t:=\otec\left(v\right)$\newline 
{\bf if} $b\left(u\right)=0$ {\bf then}\newline 
\phantom{---}{\bf Rotace}$\left(t,u\right)$, $b\left(u\right):=1$, $b\left(t\right):=0$, $u:=\bratr\left(v\right)$\newline 
{\bf endif\newline}
(Viz Obr. 4, Koment\'a\v r: nyn\'\i\ $b\left(u\right)=1$)\newline 
$w_1$ je syn $u$ takov\'y, \v ze $s\left(v\right)=s\left(w_1\right)$, $w_2:=\bratr\left(w_1\right)$\newline 
{\bf if} $b\left(w_1\right)=b\left(w_2\right)=1$ {\bf then}\newline 
\phantom{---}$b\left(u\right):=0$\newline 
\phantom{---}{\bf if} $b\left(t\right):=0$ {\bf then}\newline 
\phantom{------}$b\left(t\right):=1$\newline 
\phantom{---}{\bf else}\newline 
\phantom{------}{\bf if} $t$ nen\'\i\ ko\v ren stromu {\bf then}\newline \phantom{---------}$v:=t$, {\bf Vyvaz-DELETE}$\left(v\right)$\newline 
\phantom{------}{\bf endif}\newline 
\phantom{---}{\bf endif} (Viz Obr. 5)\newline 
{\bf else}\newline 
\phantom{---}{\bf if} $b\left(w_1\right)=1$ {\bf then}\newline 
(Koment\'a\v r: $b\left(w_2\right)=0$)\newline 
\phantom{------}{\bf Rotace$\left(t,u\right)$}, $b\left(w_2\right):=1$, $b\left(u\right):=b\left(t\right)$, $b\left(t\right):=1$ (Viz Obr. 6)\newline 
\phantom{---}{\bf else}\newline 
\phantom{------}{\bf Dvojita-rotace}$\left(t,u,w_1\right)$, $b\left(w_1\right):=b\left(t\right)$, $b\left(t\right):=1$ (Viz Obr. 7)\newline
\phantom{---}{\bf endif\newline 
endif}
skip

\flushpar V n\'asleduj\'\i c\'\i ch obr\'azc\'\i ch jsou vrcholy, kter\'e 
nemaj\'\i\ specifikovanou barvu (mohou b\'yt jak \v cerven\'e tak \v cern\'e).
Tyto barvy budeme ozna\v covat $a$, $a'$. D\accent23uvod je, \v ze se 
tato barva m\accent23u\v ze p\v ren\'est do c\'\i lov\'eho stromu, ale 
i na jin\'y vrchol. V tomto smyslu jsou tyto barvy ur\v ceny vstupn\'\i m 
stromem a specifikuj\'\i\ tyto barvy v c\'\i lov\'em strom\v e. V Obr. 5 
se barva $a$ v c\'\i lov\'em strom\v e neobjevuje.
\medskip


\midinsert
\centerline{\input fig6.tex}
\botcaption{Obr. 4}
\endcaption
\endinsert

\midinsert
\centerline{\input fig7.tex}
\botcaption{Obr. 5}
\endcaption
\endinsert
\newpage

\midinsert
\centerline{\input fig8.tex}
\botcaption{Obr. 6}
\endcaption
\endinsert

\midinsert
\centerline{\input fig9.tex}
\botcaption{Obr. 7}
\endcaption
\endinsert


\flushpar Nyn\'\i\ pop\'\i\v seme algoritmy realizuj\'\i c\'\i\ operace {\bf INSERT}, 
{\bf DE\-LE\-TE}, {\bf JOIN3} a {\bf SPLIT} pro \v cerveno-\v cern\'e stromy.  
P\v red\-po\-kl\'a\-dejme, \v ze $T$ je \v cerveno-\v cern\'y strom reprezentuj\'\i c\'\i\ 
mno\v zinu $S$ a pro\-v\'a\-d\'\i\-me operaci {\bf INSERT$\left(x\right)$} pro $
x\notin S$.  Kdy\v z 
ope\-race {\bf IN\-SERT$\left(x\right)$} pro nevyv\'a\v zen\'e bin\'arn\'\i\ vyhled\'avac\'\i\ stro\-my 
vytvo\v r\'\i\ strom $T'$, kde vrchol $v$ reprezentuje $x$, pak $
v$ 
obar\-v\'\i\-me \v cerven\v e a syny $v$ (jsou to listy) obarv\'\i me \v cern\v e.  
Dost\'av\'ame, \v ze $\left(T',v\right)$ je 2-parci\'aln\'\i\ \v cerveno-\v cern\'y strom, a pak 
aplikujeme proceduru {\bf Vyvaz-INSERT}.  
\medskip

\flushpar Operace {\bf INSERT} v \v cerveno-\v cern\'ych stromech 
vol\'a nejv\'y\v se $2+\log\left(|S|\right)$-kr\'at proceduru {\bf Vyvaz-INSERT} a 
provede nejv\'y\v se jednu rotaci nebo dvojitou rotaci.
\medskip

\flushpar Operace {\bf DELETE} je \v re\v sena stejn\'ym zp\accent23usobem 
jako ope\-race {\bf INSERT}, ale p\v ri operaci {\bf DELETE} je poru\v sena t\v ret\'\i\ 
pod\-m\'\i n\-ka v definici \v cerveno-\v cern\'ych stro\-m\accent23u a 
vyva\v zov\'an\'\i\ je technicky n\'aro\v cn\v ej\v s\'\i .  
P\v redpok\'adejme, \v ze $T$ je \v cerveno-\v cern\'y strom. Kdy\v z chceme 
prov\'est operaci {\bf DELETE}, pak nejprve provedeme algoritmus {\bf DELETE} 
pro nevyv\'a\v zen\'e 
bin\'arn\'\i\ vyhled\'avac\'\i\ stro\-my. P\v ri prov\'ad\v en\'\i\ jsme 
odstranili vrchol $u$ a jeho syna $w$, kter\'y je list. Na 
m\'\i sto vrcholu $u$ se dostal jeho druh\'y syn $v$, kter\'y 
obarv\'\i me \v cern\v e. Pak jsou spln\v eny prvn\'\i\ 
dv\v e podm\'\i nky v definici \v cerveno-\v cern\'ych 
strom\accent23u a pokud vrchol $u$ nebo vrchol $v$ byl
obarven \v cerven\v e, pak je spln\v ena i 
t\v ret\'\i\ podm\'\i nka. Pokud vrchol $u$ i vrchol $v$ byly obarveny 
\v cern\v e, pak ka\v zd\'a cesta z ko\v rene do listu obsahuj\'\i c\'\i\ 
vrchol $v$ m\'a o jeden \v cern\'y  vrchol m\'en\v e  ne\v z cesta z 
ko\v rene do listu neobsahuj\'\i c\'\i\ vrchol $v$ (chyb\'\i\ \v cern\'y  
vrchol $u$), a tedy $\left(T,v\right)$ je 3-parci\'aln\'\i\ \v cerveno-\v cern\'y strom. 
Nyn\'\i\ aplikujeme proceduru {\bf Vyvaz-DELETE}.
Anal\'yza poskytuje rychl\'y test na to, zda 
vznikne \v cerveno-\v cern\'y strom nebo 3-parci\'aln\'\i\ 
\v cerveno-\v cern\'y strom (pak $v$ je list). 
\medskip

\flushpar M\v ejme \v cerveno-\v cern\'e stromy $T_1$ a $T_2$ reprezentuj\'\i c\'\i\ 
mno\v ziny $S_1$ a $S_2$ a m\v ejme prvek $x\in U$ takov\'y, \v ze 
$\max S_1<x<\min S_2$.  Nejprve zajist\'\i me, \v ze ko\v reny $T_
1$ i $T_2$ jsou 
obarveny \v cern\v e.  P\v redpokl\'adejme, \v ze $k_i$ je po\v cet \v cern\'ych 
vrchol\accent23u na cest\v e z ko\v rene do list\accent23u ve strom\v e 
$T_i$ pro $i=1,2$.  Kdy\v z $k_1=k_2$, pak sta\v c\'\i\ prov\'est {\bf JOIN3$
\left(T_1,x,T_2\right)$ }
pro nevyv\'a\v zen\'e bin\'arn\'\i\ vyhled\'avac\'\i\ stromy (ko\v ren obarv\'\i me 
\v cerven\v e).  Probl\'em je, kdy\v z $k_1\ne k_2$.  Nap\v r\'\i klad 
p\v redpokl\'adejme, \v ze $k_1>k_2$.  Pak za\v cneme v ko\v reni stromu $
T_1$ a 
jdeme po prav\'ych synech dol\accent23u tak dlouho, a\v z nalezneme 
\v cern\'y vrchol $v$ takov\'y, \v ze v\v sechny cesty z $v$ do list\accent23u 
v $T_1$ obsahuj\'\i\ pr\'av\v e $k_2$ \v cern\'ych vrchol\accent23u.  Pak 
provedeme {\bf JOIN3} pro nevyv\'a\v zen\'e bin\'arn\'\i\ vyhledavac\'\i\ stromy 
na podstrom $T_1$ ur\v cen\'y vrcholem $v$, na $x$ a na $T_2$. Ko\v ren $
w$ 
vznikl\'eho stromu obarv\'\i me \v cerven\v e a tento strom vlo\v z\'\i me do 
$T_1$ m\'\i sto podstromu ur\v cen\'eho vrcholem $v$. Pak $\left(T_1,w
\right)$ je 
$2$-parci\'aln\'\i\ \v cerveno-\v cern\'y strom a aplikujeme proceduru 
{\bf Vyvaz-INSERT}. P\v r\'\i pad $k_2>k_1$ se \v re\v s\'\i\ symetricky. 
\medskip

\flushpar Algoritmus pro operaci {\bf SPLIT} je velmi podobn\'y 
algoritmu pro $\left(a,b\right)$-stromy. Vy\-hled\'av\'a\-me vrchol 
reprezentuj\'\i c\'\i\ $x$. Kdy\v z jsme ve vrcholu $t$ a pokra\v cujme akc\'\i\ 
$t:=\levy\left(t\right)$, pak dvojici $\key\left(t\right)$ a podstrom $T$ ur\v cen\'y prav\'ym 
synem $t$ vlo\v z\'\i me do z\'asobn\'\i ku $Z_2$, kdy\v z pokra\v cujeme akc\'\i\ 
$t:=\pravy\left(t\right)$, pak do z\'asobn\'\i ku $Z_1$ vlo\v z\'\i me dvojici podstrom $
T$ 
ur\v cen\'y lev\'ym synem $T$ a $\key\left(t\right)$. Kdy\v z $\key\left(t\right)=x$, pak do $
Z_1$ 
vlo\v z\'\i me podstrom ur\v cen\'y lev\'ym synem $t$ a do $Z_2$ podstrom 
ur\v cen\'y prav\'ym synem $t$. Kdy\v z $t$ je list, pak do $Z_1$ i $
Z_2$ vlo\v z\'\i me 
jednoprvkov\'e stromy. Ze z\'asobn\'\i ku $Z_1$ pomoc\'\i\ operace {\bf JOIN3 }
vytvo\v r\'\i me strom $T_1$ a ze z\'asobn\'\i ku $Z_2$ pomoc\'\i\ operace {\bf JOIN3 }
dostaneme strom $T_2$. 
\medskip

\flushpar Nyn\'\i\ pop\'\i\v seme algoritmy pro tyto operace.
\medskip

{\bf INSERT$\left(x\right)$\newline 
Vyhledej$\left(x\right)$\newline 
if} $t$ je list {\bf then}\newline 
\phantom{---}$t$ se zm\v en\'\i\ na vnit\v rn\'\i\ vrchol, $\key\left(t\right):=x$\newline 
\phantom{---}pro vrchol $t$ vytvo\v rme syny $\levy\left(t\right)$ a $\pravy\left(t\right)$\newline 
\phantom{---}$b\left(t\right):=0$, $b\left(\levy\left(t\right)\right):=1$, $b\left(\pravy\left(t\right)\right):=1$, {\bf Vyvaz-INSERT$\left(t\right)$\newline
endif
\medskip

DELETE$\left(x\right)$\newline 
Vyhledej$\left(x\right)$\newline 
if} $t$ nen\'\i\ list {\bf then}\newline 
\phantom{---}$vyv:=false$ \newline 
\phantom{---}{\bf if} $\levy\left(t\right)$ je list {\bf then}\newline 
\phantom{------}$v:=\pravy\left(t\right)$\newline 
\phantom{------}{\bf if} $b\left(t\right)=1$ a $b\left(v\right)=1$ {\bf then}\newline 
\phantom{---------}$vyv:=true$\newline 
\phantom{------}{\bf endif}\newline 
\phantom{------}odstran\'\i me vrchol $\levy\left(t\right)$, $\otec\left(v\right):=\otec\left(t\right)$\newline 
\phantom{------}{\bf if} $t=\levy\left(\otec\left(t\right)\right)$ {\bf then}\newline 
\phantom{---------}$\levy\left(\otec\left(t\right)\right):=v$\newline 
\phantom{------}{\bf else}\newline 
\phantom{---------}$\pravy\left(\otec\left(t\right)\right):=v$\newline 
\phantom{------}{\bf endif}\newline 
\phantom{------}$b\left(v\right):=1$, odstran\'\i me vrchol $t$\newline 
\phantom{---}{\bf else}\newline 
\phantom{------}$u:=\levy\left(t\right)$\newline 
\phantom{------}{\bf while} $\pravy\left(u\right)$ nen\'\i\ list {\bf do} $u:=\pravy\left(u\right)$ {\bf enddo}\newline 
\phantom{------}$\key\left(t\right):=\key\left(u\right)$, $v:=\levy\left(u\right)$ \newline 
\phantom{------}{\bf if} $b\left(u\right)=1$ a $b\left(v\right)=1$ {\bf then}\newline 
\phantom{---------}$vyv:=true$\newline 
\phantom{------}{\bf endif}\newline 
\phantom{------}odstran\'\i me vrchol $\pravy\left(u\right)$, $\otec\left(v\right):=\otec\left(u\right)$\newline
\phantom{------}{\bf if} $u=\levy\left(\otec\left(u\right)\right)$ {\bf then}\newline 
\phantom{---------}$\levy\left(\otec\left(u\right)\right):=v$\newline 
\phantom{------}{\bf else}\newline 
\phantom{---------}$\pravy\left(\otec\left(u\right)\right):=v$\newline 
\phantom{------}{\bf endif}\newline 
\phantom{------}$b\left(v\right):=1$, odstran\'\i me vrchol $u$\newline 
\phantom{---}{\bf endif} \newline 
\phantom{---}{\bf if} $vyv$ {\bf then Vyvaz-DELETE$\left(v\right)$ endif\newline 
endif}
\medskip

{\bf JOIN3$\left(T_1,x,T_2\right)$\newline 
if} $b\left(\text{ko\v ren }T_1\right)=0$ {\bf then} $b\left(\text{ko\v ren }T_1\right)
:=1$ {\bf endif\newline 
if} $b\left(\text{ko\v ren }T_2\right)=0$ {\bf then} $b\left(\text{ko\v ren }T_2\right)
:=1$ {\bf endif}\newline 
$k_1$ je po\v cet \v cern\'ych vrchol\accent23u v $T_1$ z ko\v rene do list\accent23u \newline 
$k_2$ je po\v cet \v cern\'ych vrchol\accent23u v $T_2$ z ko\v rene do list\accent23u \newline 
{\bf if} $k_1\ge k_2$ {\bf then}\newline 
\phantom{---}$t:=\text{ ko\v ren }T_1$, $i:=k_1-k_2$ \newline 
\phantom{---}{\bf while} $i>0$ {\bf do}\newline 
\phantom{------}$t:=\pravy\left(t\right)$\newline 
\phantom{------}{\bf if} $b\left(t\right)=1$ {\bf then} $i:=i-1$ {\bf endif}\newline 
\phantom{---}{\bf enddo}\newline 
\phantom{---}vytvo\v r vrchol $u$, $b\left(u\right):=0$, $\key\left(u\right):=x$\newline 
\phantom{---}{\bf if} $t$ nen\'\i\ ko\v ren $T_1$ {\bf then}\newline 
\phantom{------}$\otec\left(u\right):=\otec\left(t\right)$, $\pravy\left(\otec\left(t\right)\right):=u$\newline 
\phantom{---}{\bf endif}\newline 
\phantom{---}$\otec\left(t\right):=u$, $\otec\left(\text{ko\v ren }T_2\right):=u$ \newline 
\phantom{---}$\pravy\left(u\right):=\text{ ko\v ren }T_2$, $\levy\left(u\right):=t$, {\bf Vyvaz-INSERT$\left(T_1,u\right)$\newline 
else}\newline 
\phantom{---}$t:=\text{ko\v ren }T_2$, $i:=k_2-k_1$ \newline 
\phantom{---}{\bf while} $i>0$ {\bf do}\newline 
\phantom{------}$t:=\levy\left(t\right)$\newline 
\phantom{------}{\bf if} $b\left(t\right)=1$ {\bf then} $i:=i-1$ {\bf endif}\newline 
\phantom{---}{\bf enddo}\newline 
\phantom{---}vytvo\v r vrchol $u$, $b\left(u\right):=0$, $\key\left(u\right):=x$ \newline 
\phantom{---}$\otec\left(u\right):=\otec\left(t\right)$, $\levy\left(\otec\left(t\right)\right):=u$, $\otec\left(t\right):=u$ \newline 
\phantom{---}$\otec\left(\text{ko\v ren }T_1\right):=u$, $\levy\left(u\right):=\text{ ko\v ren }T_1$ \newline 
\phantom{---}$\pravy\left(u\right):=t$, {\bf Vyvaz-INSERT}$\left(T_2,u\right)$\newline 
{\bf endif}
\medskip

{\bf SPLIT$\left(x\right)$\newline 
$Z_1$} a $Z_2$ jsou pr\'azdn\'e z\'asobn\'\i ky, $t:=\text{ ko\v ren }T$\newline 
{\bf while} $\key\left(t\right)\ne x$ a $t$ nen\'\i\ list {\bf do}\newline 
\phantom{---}{\bf if} $\key\left(t\right)>x$ {\bf then}\newline 
\phantom{------}vlo\v z $\left(\key\left(t\right),\pravy\left(t\right)\right)$ do $Z_2$, $t:=\levy\left(t\right)$ \newline 
\phantom{---}{\bf else}\newline 
\phantom{------}vlo\v z $\left(\levy\left(t\right),\key\left(t\right)\right)$ do $Z_1$, $t:=\pravy\left(t\right)$ \newline 
\phantom{---}{\bf endif\newline 
enddo \newline 
if} $\key\left(t\right)=x$ {\bf then}\newline 
\phantom{---}{\bf V\'ystup}: $x\in S$, $T_1$ je podstrom $T$ ur\v cen\'y $\levy\left(t\right)$ \newline 
\phantom{---}$T_2$ je podstrom $T$ ur\v cen\'y $\pravy\left(t\right)$ \newline 
{\bf else}\newline 
\phantom{---}{\bf V\'ystup}: $x\notin S$, $T_1$ a $T_2$ jsou jednoprvkov\'e stromy\newline 
{\bf endif \newline 
while} $Z_1\ne\emptyset$ {\bf do}\newline 
\phantom{---}$\left(t,x\right)$ je na vrcholu $Z_1$, odstra\v n $\left(t,x\right)$ ze $Z_1$ \newline 
\phantom{---}$T'$ je podstrom $T$ ur\v cen\'y $t$, $T_1:=${\bf JOIN3$\left(T',x,T_1
\right)$ \newline 
enddo\newline 
while} $Z_2\ne\emptyset$ {\bf do}\newline 
\phantom{---}$\left(x,t\right)$ je na vrcholu $Z_2$, odstra\v n $\left(x,t\right)$ ze $Z_1$ \newline 
\phantom{---}$T'$ je podstrom $T$ ur\v cen\'y $t$, $T_2:=${\bf JOIN3$\left(T_2,x,T'
\right)$ \newline 
enddo}
\medskip

\flushpar Korektnost algoritm\accent23u je vid\v et z 
obr\'azk\accent23u. V\v simn\v eme si p\v ri ope\-raci {\bf DELETE}, \v ze kdy\v z $
u$ je 
obarven \v cerven\v e, pak po proveden\'\i\ {\bf Rotace$\left(t,u\right)$} bude $
\left(T,v\right)$ op\v et 
3-parci\'aln\'\i\ \v cerveno-\v cern\'y strom a vrchol $t$ bude obarven \v cerven\v e.  
Pak z Obr.  5 je vid\v et, \v ze dostaneme \v cerveno-\v cern\'y 
strom. Tedy m\accent23u\v zeme shrnout:


\proclaim{V\v eta}Algoritmy operac\'\i\ {\bf MEMBER}, {\bf INSERT}, 
{\bf DE\-LE\-TE}, {\bf MIN}, {\bf MAX}, {\bf JOIN3} a {\bf SPLIT} pro \v cerveno-\v cern\'e 
stro\-my vy\v zaduj\'\i\ v nejhor\v s\'\i m p\v r\'\i pad\v e \v cas $O
\left(\log\left(|S|\right)\right)$, kde $S$ je 
reprezentovan\'a mno\v zina.  Operace {\bf INSERT} a {\bf JOIN3} zavolaj\'\i\ 
nejv\'y\v se jednou bu\v d {\bf Rotace} nebo {\bf Dvojita-rota\-ce} a operace 
{\bf DE\-LE\-TE} zavol\'a nejv\'y\v se dvakr\'at {\bf Rotace} nebo {\bf Rotace} a 
{\bf Dvojita-ro\-ta\-ce}.  
\endproclaim

\flushpar V\v simn\v ete si, \v ze operace {\bf JOIN3} ve skute\v cnosti 
vy\v zaduje \v cas $O\left(|k_1-k_2|+1\right)$. Proto\v ze $Z_1$ a $Z_2$ obsahuj\'\i\ nejv\'y\v se 
$\log\left(|S|\right)$ polo\v zek, tak se odhad \v casov\'e slo\v zitosti operace {\bf SPLIT }
prove\-de stejn\'ym zp\accent23usobem jako v $\left(a,b\right)$-stromech. 
V ostatn\'\i ch p\v r\'\i\-pa\-dech je odhad \v casov\'e slo\v zitosti vid\v et z toho, 
\v ze $\hloubka\left(T\right)=O\left(\log\left(|S|\right)\right)$ a akce na ka\v zd\'e hladin\v e vy\v zaduj\'\i\ jen 
$O\left(1\right)$ \v casu.
\medskip

\flushpar Pokud chceme m\'\i t i algoritmus pro operaci {\bf ord$
\left(k\right)$}, pak 
mus\'\i me roz\v s\'\i\v rit strukturu o funkci $p$. Pak lze pou\v z\'\i t p\v r\'\i mo 
algoritmus pro {\bf ord$\left(k\right)$} v nevyv\'a\v zen\'ych bin\'arn\'\i ch vyhled\'avac\'\i ch 
stromech. P\v ripome\v nme si, \v ze procedury {\bf Rotace} a {\bf Dvojita-rotace }
mohou aktualizovat funkci $p$ v \v case $O\left(1\right)$. Proto dost\'av\'ame 

\proclaim{V\v eta}Algoritmy operac\'\i\ {\bf MEMBER}, 
{\bf INSERT}, {\bf DE\-LE\-TE}, {\bf MIN}, {\bf MAX}, {\bf JOIN3}, {\bf SPLIT} a {\bf ord$
\left(k\right)$} pro 
roz\-\v s\'\i\-\v re\-nou strukturu \v cerveno-\v cern\'ych strom\accent23u vy\v zaduje 
v nejhor\v s\'\i m p\v r\'\i pad\v e \v cas $O\left(\log\left(|S|\right)\right)$, kde $S$ je reprezentovan\'a 
mno\v zina.  Operace {\bf INSERT} a {\bf JOIN3} zavolaj\'\i\ nejv\'y\v se jednou bu\v d 
{\bf Rotace} nebo {\bf Dvojita-rota\-ce} a operace {\bf DELETE} zavol\'a nejv\'y\v se 
dvakr\'at {\bf Rotace} nebo jednou {\bf Rotace} a {\bf Dvojita-rotace}.  \endproclaim


\flushpar Vznik\'a ot\'azka, pro\v c se tolik pozornosti v\v enuje 
procedur\'am {\bf Rotace} a {\bf Dvojita-rotace}. Sice vy\v zaduj\'\i\ \v cas 
$O\left(1\right)$, ale jsou to nej\-slo\v zit\v ej\v s\'\i\ akce vy\v zaduj\'\i c\'\i\ nejv\'\i ce \v casu. 
V mnoha aplikac\'\i ch (pou\-\v z\'\i vaj\'\i\ se hlavn\v e ve v\'ypo\v cetn\'\i\ geometrii), 
tvar stromu spolu s parametry nesou je\v st\v e 
dal\v s\'\i\ zak\'odovan\'e informace. P\v ri zm\v en\v e tvaru stromu je 
t\v reba je p\v repo\v c\'\i tat. {\bf Rotace} a {\bf Dvojita-rotace} m\v en\'\i\ tvar 
stromu, kde\v zto posun sm\v erem ke ko\v reni pouze m\v en\'\i\ 
obarven\'\i . V tomto p\v r\'\i pad\v e pak {\bf Rotace} nebo {\bf Dvo\-ji\-ta-rotace }
vy\v zaduje \v cas $O\left(|S|\right)$ (obvykle je t\v reba prohl\'ednout cel\'y 
strom) a nikoliv $O\left(1\right)$.
\medskip

\heading
V\'ahov\v e vyv\'a\v zen\'e stromy
\endheading

\flushpar V osmdes\'at\'ych letech se ve v\'ypo\v cetn\'\i\ geometrii 
hodn\v e pou\v z\'\i\-va\-ly $BB\left(\alpha \right)$-stromy, proto se o nich alespo\v n 
orienta\v cn\v e zm\'\i n\'\i me. M\v ejme re\'aln\'e \v c\'\i slo $
\alpha$ takov\'e, \v ze 
$\frac 14<\alpha\;\le\frac {\sqrt 2}2$. Pro strom $T$ ozna\v cme $
p\left(T\right)$ po\v cet 
list\accent23u ve stromu $T$. Bin\'arn\'\i\ vyhled\'avac\'\i\ strom $T$ 
reprezentuj\'\i c\'\i\ mno\v zinu $S$ se naz\'yv\'a $BB\left(\alpha \right)$-\emph{strom}, 
kdy\v z pro ka\v zd\'y vnit\v rn\'\i\ vrchol $v$ plat\'\i :
$$\alpha\le\frac {p\left(T_l\right)}{p \left(T_v\right)}=1-\frac {p\left(T_r\right)}{p\left(T_v\right)}\le 1-\alpha$$
kde $T_v$ je podstrom $T$ ur\v cen\'y vrcholem $v$, $T_l$ je podstrom 
$T$ ur\v cen\'y lev\'ym synem vrcholu $v$, $T_r$ je podstrom $T$ 
ur\v cen\'y prav\'ym synem vrcholu $v$. Plat\'\i


\proclaim{Tvrzen\'\i}Kdy\v z $T$ je $BB\left(\alpha \right)$-strom reprezentuj\'\i c\'\i\ 
$n$-prvkovou mno\v zinu, pak 
$$\hloubka\left(T\right)\le 1+\frac {\log\left(n+1\right)-1}{\log\frac 1{1-\alpha}}.$$
\endproclaim

\flushpar D\accent23usledek je, \v ze $BB\left(\alpha \right)$-stromy pat\v r\'\i\ do 
skupiny vyv\'a\v zen\'ych bin\'arn\'\i ch vyhled\'avac\'\i ch 
stro\-m\accent23u.  Vyva\v zov\'an\'\i\ se provad\'\i\ op\v et pomoc\'\i\ 
{\bf Rotace} a {\bf Dvojita-rotace} a popisuje ho n\'asledu\-j\'\i\-c\'\i\ 
technick\'e tvrzen\'\i. 

\proclaim{Tvrzen\'\i}Pro ka\v zd\'e $\alpha$ existuje konstanta $
d$ 
takov\'a, \v ze $\alpha <d<1-\alpha$ a pro ka\v zd\'y bin\'arn\'\i\ vyhled\'avac\'\i\ 
strom $T$ s ko\v renem $t$ spl\v nuj\'\i c\'\i\ podm\'\i nky
\roster
\item
podstromy $T_l$ a $T_r$ stromu $T$ ur\v cen\'e lev\'ym a prav\'ym 
synem $t$ jsou $BB\left(\alpha \right)$-stromy;
\item
$\frac {p\left(T_l\right)}{p\left(T\right)}<\alpha$, ale $\alpha\le\frac {
p\left(T_l\right)}{p\left(T\right)-1}\le 1-\alpha$ nebo $\alpha\le\frac {
p\left(T_l\right)+1}{p\left(T\right)+1}\le 1-\alpha$
\endroster
plat\'\i :\newline 
kdy\v z $\rho\le d$ a provedeme {\bf Rotace$\left(t,\pravy\left(t\right)\right)$}, nebo kdy\v z 
$\rho >d$ a provedeme proceduru {\bf Dvojita-rotace$\left(t,\pravy\left(t\right),\levy\left(\pravy\left(t\right)\right)\right)$}, pak dostaneme 
$BB\left(\alpha \right)$-strom (zde $\rho =\frac {p\left(T'\right)}{p\left(T_r
\right)}$ a $T'$ je ur\v cen lev\'ym synem 
prav\'e\-ho syna ko\v rene $t$).  
\endproclaim

\flushpar Toto tvrzen\'\i\ a jeho symetrick\'e verze jednozna\v cn\v e ukazuj\'\i , 
jak vyva\v zovat $BB\left(\alpha \right)$-stromy p\v ri aktualiza\v cn\'\i ch 
operac\'\i ch (podstrom $BB\left(\alpha \right)$-stromu je $BB\left(\alpha \right)$-strom). Pak dost\'av\'ame:

\proclaim{V\v eta}Implementace operac\'\i\ {\bf MEMBER}, {\bf INSERT} a {\bf DELETE }v $BB\left(\alpha \right)$-stromech vy\v zaduje v nejhor\v s\'\i m 
p\v r\'\i pad\v e \v cas $O\left(\log\left(|S|\right)\right)$, kde $S$ je 
reprezentovan\'a mno\v zina.
\endproclaim

\flushpar Obliba $BB\left(\alpha \right)$-strom\accent23u byla zap\v r\'\i\v cin\v ena 
platnost\'\i\ n\'asleduj\'\i c\'\i\ v\v ety, kter\'a je analogi\'\i\ v\v ety o 
vyva\v zovac\'\i ch operac\'\i ch pro $\left(a,b\right)$-stromy.

\proclaim{V\v eta}Kdy\v z $\alpha$ je re\'aln\'e \v c\'\i slo takov\'e, \v ze 
$\frac 14<\alpha <1-\frac {\sqrt 2}2$, pak existuje konstanta $c>
0$ z\'avisl\'a jen na 
$\alpha$ takov\'a, \v ze ka\v zd\'a posloupnost operac\'\i\ {\bf INSERT} a {\bf DELETE }
o d\'elce $m$ aplikovan\'a na pr\'azdn\'y $BB\left(\alpha \right)$-strom vol\'a 
nejv\'y\v se $cm$ procedur {\bf Rotace} a {\bf Dvojita-rotace}.
\endproclaim

\heading
Historick\'y p\v rehled:
\endheading

$\left(a,b\right)$-stromy zavedli Bayer a McGreght (1972),\newline 
v\v ety o po\v ctu vyva\v zovac\'\i ch operac\'\i\ pro $\left(a,b\right)$-stromy dok\'azali 
Huddleston a Mehlhorn (1982).\newline 
$A$-sort analyzovali Guibas, McGreight, Plass a Roberts (1977).
\smallskip
\flushpar Anal\'yza interpola\v cn\'\i ho vyhled\'av\'an\'\i\ poch\'az\'\i\ od Perla, 
Itai a Avniho (1978),\newline 
kvadratick\'e vyhled\'av\'an\'\i\ analyzovali Perl a Reingold (1977).
\smallskip
\flushpar Adelson-Velskij a Landis (1962) definovali 
AVL-stromy,\newline 
\v cerveno-\v cern\'e stromy definovali Guibas a Sedgewick 
(1978),\newline 
verze algoritmu {\bf DELETE} poch\'az\'\i\ od Tarjana (1983).
$BB\left(\alpha \right)$-stromy zavedli Nievergelt a Reingold (1973),\newline 
v\v ety o jejich vyva\v zov\'an\'\i\ dok\'azali Blum a Mehlhorn (1980).\newline 
Priorita AVL-strom\accent23u se odr\'a\v z\'\i\ v 
jejim hojn\'em pou\v z\'\i v\'an\'\i , i kdy\v z \v cerveno-\v cern\'e stromy 
jsou efektivn\v ej\v s\'\i . 

\end
