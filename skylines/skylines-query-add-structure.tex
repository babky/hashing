\documentclass[12pt,notitlepage]{report}
\pagestyle{plain}

\usepackage[utf8]{inputenc}
\usepackage[english]{babel}

\usepackage{amsmath}
\usepackage{amsfonts}
\usepackage{amssymb}
\usepackage{amsthm}

\theoremstyle{definition}
\newtheorem{definition}{Definition}

\theoremstyle{plain}
\newtheorem{theorem}[definition]{Theorem}
\newtheorem{lemma}[definition]{Lemma}
\newtheorem{remark}[definition]{Remark}
\newtheorem{corollary}[definition]{Corollary}
\newtheorem{example}[definition]{Example}
\newtheorem{claim}[definition]{Claim}
\newtheorem{statement}[definition]{Statement}
\renewcommand{\theequation}{\arabic{chapter}.\arabic{definition}}

\newcommand{\setdelim}{\mid}
\newcommand{\Prob}[1]{\mathbf{Pr}\left(#1\right)}
\newcommand{\Expect}[1]{\mathbf{E}\left(#1\right)}
\newcommand{\Variance}[1]{\mathbf{Var}\left(#1\right)}
\newcommand{\maxlength}{\mathbf{max\mbox{-}length}}
\newcommand{\length}{\mathbf{length}}
\newcommand{\Indicator}{\mathbf{I}}
\newcommand{\rank}[1]{\mathrm{rank}\left({#1}\right)}
\newcommand{\dimension}[1]{\mathrm{dim}\left({#1}\right)}
\newcommand{\nullspace}[1]{\mathcal{N}\left({#1}\right)}
\newcommand{\matrixkernel}[1]{\mathrm{Ker}\left({#1}\right)}
\newcommand{\vecspan}[1]{\mathrm{span}\left({#1}\right)}
\newcommand{\vecspace}[1]{\mathbb{Z}_{2}^{#1}}

\title{Skyline Merging}
\author{Martin Babka, Jan Bul�nek, Vladim�r Cun�t, V�clav Koubek}

\begin{document}

\chapter{Skylines}

\section{Prerequisites}

\begin{definition}[``Dominates'' relation]
Let $x, y$ be $d$-dimensional vectors. Then we say that $x$ \emph{dominates} $y$, which we denote by $x > y$, whenever $x_i \geq y_i$ for each $i \in [d]$ and $x \neq y$.
\end{definition}

\begin{definition}[Skyline]
Let $X$ be a set of $d$-dimensional vectors.
The \emph{skyline} of $X$ is the set
$\operatorname{Skyline}(X) = \{x \in X \mid \forall y \in X\colon y \not > x \}$.
\end{definition}

\section{Skyline Algorithm}
\begin{itemize}
	\item Presort all the data according to the first dimension.
	\item Split the data to the two sets $A$ and $B$ according to the third dimension so that $||A| - |B|| \leq 1$.
	\item Apply the skyline constructing algorithm recursively to and $A$ and $B$ separately. The algorithm keeps the data presorted according to the first dimension.
	\item Merge the two skylines with the third dimension removed. In the 3D case the merging is done as follows.
		\begin{itemize}
			\item Sweep the data in the direction of the first dimension -- data are presorted. Use the common two set merging algorithm.
			\item Assume that a vector $x$ is merged to the skyline. We just removes all the points which are dominated by $x$.
		\end{itemize}
	\item We reshape the data from the obtained skyline belonging to $A$ with $\operatorname{Skyline}(B)$ so that they are presorted.
\end{itemize}

\section{Fast Merging of Skylines}
In the general case we use a new data structure which keeps the skyline.

The data structure is denoted $\operatorname{SkylineDS}(d, X)$ and stores $\operatorname{Skyline}(X)$. The data structure allows two operations.
\begin{itemize} 
	\item $\operatorname{Add(x)}$ adds a new point into the skyline.
	\item $\operatorname{Query(x)}$ for querying if the given vector $x$ is in the skyline.
\end{itemize}

Using this data structure we are able to design an algorithm for merging two $d$-dimensional skylines $A$ and $B$ which are already presorted according to the first dimension.

The merging algorithm then looks as follows. Remark that it is just rephrasing of the 2D merging algorithm. Assume that we are about to merge $d$-dimensional data. We keep track of $(d - 1)$ dimensional skyline containing all the points belonging to the skyline of the already processed points (the first dimension is removed).

\begin{itemize}
	\item Initialize the data structure for the $(d - 1)$-dimensional skyline to the empty skyline.
	\item Run merging in the dimension according to which the data are presorted.
	\item Whenever a point $x$ comes check if it belongs to the current skyline. If yes, add it to the skyline and report that $x$ is in the merged skyline. Otherwise ignore it.
\end{itemize}

\section{The Skyline Data Structure}
The data structure accompanies the search trees for limited universe having $O(\sqrt{\log n})$ operations. We denote the trees by $\operatorname{SearchTreeDS}$.

The data structure $\operatorname{SkylineDS(d, X)}$ is a recursive structure. The first dimension is indexed by a $\operatorname{SearchTreeDS}$. Assume that  $x \in X$, the tree structure maps each $x_1$, to $\operatorname{SkylineDS(d - 1, X_{x_1})}$ where $X_a =\{y \in X \mid y_1 \leq a \}$.

We are just designing the $\operatorname{Query}(x)$ operation. Search for the successor $s$ of $x_1$ in the $\operatorname{SearchTreeDS}$. The predecessor $p$ of $x$ is the maximal number such that $p \leq x_1$. Then observe that $x \in \operatorname{Skyline}(X \cup \{x\})$ iff. $x \in \operatorname{Skyline}(d - 1, X_p)$.

The running time for query in $\operatorname{SkylineDS(d, X)}$ is $Q_d(n) = O(\sqrt{\log n}) + Q_{d - 1}(n) = O(d \sqrt{\log n}).$

\chapter{Binary skylines}

The elements of the vectors are just zeros or ones. We are interested in searching for minimal elements. To do so we compute the POSET graph (just like Hasse diagram also with the transitive edges).
 
Let $N \leq kd$ be the number of ones in $k$ vectors of dimension $d$.
The running time is $O\left(\frac{N^2 \log \log N}{\log ^2 N}\right)$.
For arbitrary $N$ there are $k$ vectors of dimension $d$ such that they contain $N$ ones and the number of edges in the Hasse diagram is $\Theta\left(\frac{N^2}{\log ^2 N}\right)$.

We are computing the minimal elements as follows.

Let $X, Y, Y'$ be sets of 0-1 vectors and we put $L(X, Y, Y', d) = \{x \in X \mid (\neg \exists y \in Y \colon y >_d x) \wedge (\neg \exists y' \in Y' \colon y' \geq_d x) \}$ where $x >_d$ ($x \geq_d y$) denotes the fact that $x$ (strictly) dominates $y$ when restricted to the last $d$ coordinates.

The computation uses the fact that 
\[
L(X, Y, Y', d) = L(X_1, Y_1, Y'_1, d - 1) \cup L(X_0, Y_0, Y_1 \cup Y'_0 \cup Y'_1, d - 1).
\]
\end{document}
