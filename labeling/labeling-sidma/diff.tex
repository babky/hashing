\documentclass[unicode,review]{siamart1116}
%DIF LATEXDIFF DIFFERENCE FILE
%DIF DEL lab070217-siam-submitted.tex   Sun Jun 10 13:51:53 2018
%DIF ADD lab-fix.tex                    Sat Jun 16 13:34:53 2018

\usepackage{graphicx}
\usepackage{multirow}
\usepackage{booktabs}
\usepackage{amsfonts}
\usepackage{amsmath}
\usepackage{amssymb}
\usepackage{braket}
\usepackage{amsopn}
\usepackage{url}

% May be removed later
\usepackage{todonotes}
% Because of the table, we should try to do it without this workaround.
\addtolength{\voffset}{-1cm}
\addtolength{\hoffset}{-1.5cm}
\addtolength{\textheight}{+2cm}
\addtolength{\textwidth}{+3cm}
 
\newtheorem{claim}[theorem]{Claim}
\newtheorem{fact}[theorem]{Fact}
\newtheorem{remark}[theorem]{Remark}
\newenvironment{proofof}[1]{\noindent{\textbf { Proof of #1:}}} {{\qed}}

\newcommand{\Prob}[1]{\mathbf{Pr}\left(#1\right)}
\newcommand{\E}[1]{\mathbf{E}\left[#1\right]}
\newcommand{\A}{\mathcal{A}}
\newcommand{\N}{\mathbb{N}}
\newcommand{\PTheta}[1]{\Theta\!\left(#1\right)}
\newcommand{\natInt}[2]{ \left\{ #1, \dotsc, #2 \right\} }

\newcommand{\thmA}{{C_0}}
\newcommand{\thmB}{{C_1}}
\newcommand{\thmC}{{C_2}}
\newcommand{\buffer}{{\rm {\bf buffer}}}
\newcommand{\pop}{{\rm {\bf pop}}}
\newcommand{\weight}{{\rm {\bf weight}}}
\newcommand{\midp}{{\rm {\bf midpoint}}}
\newcommand{\depth}{{\rm {\bf depth}}}
\newcommand{\densify}{{\rm {\bf densify}}}
\newcommand{\starts}{{\rm {\bf start}}}
\newcommand{\sends}{{\rm {\bf end}}}
% For upper bound section (added 10-30-11)
\newcommand{\minspace} {s_{\min}}
\def\supp{{\rm{supp}}}
\def\rank{{\rm{rank}}}
\def\new{{\mathrm{new}}}
\def\old{{\mathrm{old}}}

% SIAM template
\numberwithin{theorem}{section}

\newcommand{\TheTitle}{On online labeling with large label set} 
\newcommand{\TheAuthors}{M. Babka, J. Bul\'anek, V. \v{C}un\'at, M. Kouck\'y, and M. Saks}

\headers{\TheTitle}{\TheAuthors}

\title{\TheTitle\footnote{A preliminary version of this paper appeared in the Proceedings of the 2012 European Symposium on Algorithms \cite{BBCKS12}}}

\author{
  Martin Babka\thanks{Charles University, Prague (\email{babkys@gmail.com}, \email{vcunat@gmail.com}).\funding{The first and third authors gratefully acknowledge support by the Czech Science Foundation under grant GA14-10799S}.}
  \and
  Jan Bul\'anek\thanks{Institute of Mathematics, Academy of Sciences CR, Prague (\email{jan.bulanek@gmail.com}).}
  \and
  Vladim\'ir \v{C}un\'at\footnotemark[2]
  \and
  Michal Kouck\'{y}\thanks{Charles University, Prague (\email{koucky@iuuk.mff.cuni.cz}) \funding{Part of the work done while at the Institute of Mathematics, Academy of Sciences CR, Prague and while visiting Aarhus University. Supported in part by Center of Excellence CE-ITI (P202/12/G061 of GA \v{C}R).}}
  \and
  Michael Saks\thanks{Department of Mathematics, Rutgers University (\email{saks@math.rutgers.edu}). \funding{The work of this author was done while on sabbatical at Princeton University and was also supported in part by NSF under grants CCF-0832787 and CCF-1218711.}}
}

\usepackage{amsopn}
\DeclareMathOperator{\diag}{diag}

\ifpdf
\hypersetup{
  pdftitle={\TheTitle},
  pdfauthor={\TheAuthors}
}
\fi
% End SIAM template
%DIF PREAMBLE EXTENSION ADDED BY LATEXDIFF
%DIF UNDERLINE PREAMBLE %DIF PREAMBLE
\RequirePackage[normalem]{ulem} %DIF PREAMBLE
\RequirePackage{color}\definecolor{RED}{rgb}{1,0,0}\definecolor{BLUE}{rgb}{0,0,1} %DIF PREAMBLE
\providecommand{\DIFadd}[1]{{\protect\color{blue}\uwave{#1}}} %DIF PREAMBLE
\providecommand{\DIFdel}[1]{{\protect\color{red}\sout{#1}}}                      %DIF PREAMBLE
%DIF SAFE PREAMBLE %DIF PREAMBLE
\providecommand{\DIFaddbegin}{} %DIF PREAMBLE
\providecommand{\DIFaddend}{} %DIF PREAMBLE
\providecommand{\DIFdelbegin}{} %DIF PREAMBLE
\providecommand{\DIFdelend}{} %DIF PREAMBLE
%DIF FLOATSAFE PREAMBLE %DIF PREAMBLE
\providecommand{\DIFaddFL}[1]{\DIFadd{#1}} %DIF PREAMBLE
\providecommand{\DIFdelFL}[1]{\DIFdel{#1}} %DIF PREAMBLE
\providecommand{\DIFaddbeginFL}{} %DIF PREAMBLE
\providecommand{\DIFaddendFL}{} %DIF PREAMBLE
\providecommand{\DIFdelbeginFL}{} %DIF PREAMBLE
\providecommand{\DIFdelendFL}{} %DIF PREAMBLE
\newcommand{\DIFscaledelfig}{0.5}
%DIF HIGHLIGHTGRAPHICS PREAMBLE %DIF PREAMBLE
\RequirePackage{settobox} %DIF PREAMBLE
\RequirePackage{letltxmacro} %DIF PREAMBLE
\newsavebox{\DIFdelgraphicsbox} %DIF PREAMBLE
\newlength{\DIFdelgraphicswidth} %DIF PREAMBLE
\newlength{\DIFdelgraphicsheight} %DIF PREAMBLE
% store original definition of \includegraphics %DIF PREAMBLE
\LetLtxMacro{\DIFOincludegraphics}{\includegraphics} %DIF PREAMBLE
\newcommand{\DIFaddincludegraphics}[2][]{{\color{blue}\fbox{\DIFOincludegraphics[#1]{#2}}}} %DIF PREAMBLE
\newcommand{\DIFdelincludegraphics}[2][]{% %DIF PREAMBLE
\sbox{\DIFdelgraphicsbox}{\DIFOincludegraphics[#1]{#2}}% %DIF PREAMBLE
\settoboxwidth{\DIFdelgraphicswidth}{\DIFdelgraphicsbox} %DIF PREAMBLE
\settoboxtotalheight{\DIFdelgraphicsheight}{\DIFdelgraphicsbox} %DIF PREAMBLE
\scalebox{\DIFscaledelfig}{% %DIF PREAMBLE
\parbox[b]{\DIFdelgraphicswidth}{\usebox{\DIFdelgraphicsbox}\\[-\baselineskip] \rule{\DIFdelgraphicswidth}{0em}}\llap{\resizebox{\DIFdelgraphicswidth}{\DIFdelgraphicsheight}{% %DIF PREAMBLE
\setlength{\unitlength}{\DIFdelgraphicswidth}% %DIF PREAMBLE
\begin{picture}(1,1)% %DIF PREAMBLE
\thicklines\linethickness{2pt} %DIF PREAMBLE
{\color[rgb]{1,0,0}\put(0,0){\framebox(1,1){}}}% %DIF PREAMBLE
{\color[rgb]{1,0,0}\put(0,0){\line( 1,1){1}}}% %DIF PREAMBLE
{\color[rgb]{1,0,0}\put(0,1){\line(1,-1){1}}}% %DIF PREAMBLE
\end{picture}% %DIF PREAMBLE
}\hspace*{3pt}}} %DIF PREAMBLE
} %DIF PREAMBLE
\LetLtxMacro{\DIFOaddbegin}{\DIFaddbegin} %DIF PREAMBLE
\LetLtxMacro{\DIFOaddend}{\DIFaddend} %DIF PREAMBLE
\LetLtxMacro{\DIFOdelbegin}{\DIFdelbegin} %DIF PREAMBLE
\LetLtxMacro{\DIFOdelend}{\DIFdelend} %DIF PREAMBLE
\DeclareRobustCommand{\DIFaddbegin}{\DIFOaddbegin \let\includegraphics\DIFaddincludegraphics} %DIF PREAMBLE
\DeclareRobustCommand{\DIFaddend}{\DIFOaddend \let\includegraphics\DIFOincludegraphics} %DIF PREAMBLE
\DeclareRobustCommand{\DIFdelbegin}{\DIFOdelbegin \let\includegraphics\DIFdelincludegraphics} %DIF PREAMBLE
\DeclareRobustCommand{\DIFdelend}{\DIFOaddend \let\includegraphics\DIFOincludegraphics} %DIF PREAMBLE
\LetLtxMacro{\DIFOaddbeginFL}{\DIFaddbeginFL} %DIF PREAMBLE
\LetLtxMacro{\DIFOaddendFL}{\DIFaddendFL} %DIF PREAMBLE
\LetLtxMacro{\DIFOdelbeginFL}{\DIFdelbeginFL} %DIF PREAMBLE
\LetLtxMacro{\DIFOdelendFL}{\DIFdelendFL} %DIF PREAMBLE
\DeclareRobustCommand{\DIFaddbeginFL}{\DIFOaddbeginFL \let\includegraphics\DIFaddincludegraphics} %DIF PREAMBLE
\DeclareRobustCommand{\DIFaddendFL}{\DIFOaddendFL \let\includegraphics\DIFOincludegraphics} %DIF PREAMBLE
\DeclareRobustCommand{\DIFdelbeginFL}{\DIFOdelbeginFL \let\includegraphics\DIFdelincludegraphics} %DIF PREAMBLE
\DeclareRobustCommand{\DIFdelendFL}{\DIFOaddendFL \let\includegraphics\DIFOincludegraphics} %DIF PREAMBLE
%DIF END PREAMBLE EXTENSION ADDED BY LATEXDIFF

\begin{document}

\maketitle

\begin{abstract}
In the online labeling problem with parameters $n$ and $m$ we are
presented with a sequence of $n$ {\em items} from a totally ordered
universe $U$ and must assign each arriving item a label from the label set
$\natInt{1}{m}$ so that the order of labels respects the order on $U$. As
new items arrive it may be necessary to change the labels of some items;
such changes may be done at any time at unit cost for each change. The
goal is to minimize the total cost. An alternative formulation of this
problem is the \emph{file maintenance problem}, in which the items are
maintained in sorted order in an array of length $m$, and we pay unit cost
for moving an item.

For the case $m=cn$ for constant $c>1$, an algorithm of Itai et al. (1981)
achieves total cost $O(n (\log n)^2)$, which is asymptotically optimal
(Bul\'anek et al., 2015). For the case of $m=\Theta(n^{1+C})$ for constant
$C>0$, algorithms are known that use $O(n \log n)$ relabelings. A matching
lower bound was provided in Dietz et al. (2004).  The lower bound proof
had two parts: a lower bound for a problem called {\em prefix bucketing}
and a reduction from prefix bucketing to online labeling. We present a
simplified version of their reduction, together with a full proof (which
was not given in Dietz et al. (2004)). We also simplify and improve the
analysis of the prefix bucketing lower bound. This improvement allows us
to extend the lower bounds for online labeling to larger $m$. Our lower
bound for $m$ from $n^{1+C}$ to $2^n$ is $\Omega((n \log n) / (\log \log m
- \log\log n))$.  This reduces to the asymptotically optimal bound
$\Omega(n \log n)$ when $m = \Theta(n^{1+C})$.  We show that our bound is
asymptotically optimal for the case of $m \geq 2^{1+(\log n)^{3}}$ by
giving a matching upper bound.
\end{abstract}

\begin{keywords}
online labeling, lower bounds, file maintenance problem, algorithms
\end{keywords}

\begin{AMS}
  68Q17, % Computational difficulty of problems (lower bounds, completeness, difficulty of approximation, etc.)
  68P05, % Data structures
  68W27  % Online algorithms
\end{AMS}

\section{Introduction}

In the online labeling problem with parameters $n,m,r$,
we are presented with a sequence of $n$ {\em items} from a totally ordered universe $U$ of size $r$
and must assign each arriving item a label from the label set $\natInt{1}{m}$ so that the order of labels respects the ordering on $U$.
As new items arrive it may be necessary to change the labels of some items; such changes may be done at any time at unit cost
for each change.
The goal is to minimize
the total cost.
An alternative formulation of this problem is
the \emph{file maintenance problem}, in which the items, instead of being labeled,
are maintained in sorted order in an  array of length $m$, and we pay unit cost for moving an item.
\iffalse
In this formulation we allow cells of the array to contain a special value that marks unoccupied array cells, and we are
concerned only about the total number of moves of actual items stored in the array.
\fi

The problem, which was introduced by Itai, Konheim and Rodeh \cite{Itaietal},
is natural and intuitively appealing, and has had applications to the design of data structures (see for example
the discussion in \cite{DSZ04}, and the more recent work on cache-oblivious data structures
\cite{BenderetalB-Tree,Brodaletal,BDIW}).  A connection between this problem and distributed resource allocation
was recently shown by Emek and Korman \cite{EK11}.


The parameter $m$, the {\em label space} must be at least the number of items $n$ or else no valid labeling is possible.
There are two natural range of parameters that have received the most attention. In the case of {\em linearly many labels} we have
$m=cn$ for some $c>1$, and in the case of {\em polynomially many labels} we have $m=\theta(n^{1+C})$ for some constant $C>0$.
The problem is trivial if  $|U| \leq m$ , since then we can fix an order preserving
bijection from $U$ to $\natInt{1}{m}$ in advance.   
\iffalse The known upper bounds for this problem, summarized in the next paragraph, put no restriction on the size of $U$.  For the lower bound proved in this paper we will assume that  $|U|$
has size at least exponential in $n$.
\fi

%%MIKE 3/31/15 BEGIN
\paragraph{Known upper bounds}
Itai et al. \cite{Itaietal} gave an algorithm for the case of linearly many labels having worst case total cost
$O(n (\log n)^2)$.  
%This algorithm works in the file maintenance formulation, and the total cost includes even manipulation
%with unoccupied array cells.
Improvements and simplifications were given by Willard \cite{Willard} and Bender et al. \cite{Benderetal}.
In the special case that $m=n$, algorithms with cost $O((\log n)^3)$ per item were given
\cite{Zhang,BirdSadnicki}.  It is also
well known that the  algorithm of Itai et al. can be adapted to give total cost $O(n \log n)$ in the case
of polynomially many labels.   An algorithm with $O(\log n)$ worst case cost per insertion was given
in \cite{Kop12}.
All of these algorithms make no restriction on the size of the
universe $U$ of items.

\paragraph{Known lower bounds}
For the case of polynomially many labels, Dietz at al. \cite{DSZ04} (also in \cite{Zhang}) proved a matching lower bound for the $O(n \log n)$ upper bound.


For the case of linearly many labels ($m=O(n)$), Dietz and Zhang (\cite{DZ-SWAT,Dietz-manuscript}, also available in Zhang's Ph.D. thesis \cite{Zhang}) proved an $\Omega(n \log^2 n)$ lower bound for a restricted class of algorithms, called {\em smooth algorithms}.
A subset of the present authors~\cite{BKS} showed that the same (tight) lower bound holds for any online
labeling algorithm.  These bounds  hold even when the size of the universe $U$ 
is only a constant multiple of $m$.  The bound remains
non-trivial (superlinear in $n$) for
$m  = O(n (\log n)^{2-\varepsilon})$ but
becomes trivial  for $m \in \Omega(2^{\log^2 n})$.


\paragraph{Our results}
In this paper we extend and clarify the lower bound of Dietz et al.~\cite{DSZ04}. Their result consists of two parts; a lower bound for a problem they call {\em prefix bucketing}
and a reduction from prefix bucketing to online labeling.  Our results are:

\begin{itemize}
\item We provide a simpler and more precise lower bound for the prefix bucketing problem. (This is given in \cref{s-lbb}.)
\item We clarify the reduction  from prefix bucketing to online labeling and give a full proof of correctness.
(This is given in \cref{ss-adversary,ss-b to ol}.)
In~\cite{DSZ04}, only a proof sketch is provided.\footnote{Our initial study
of their proof led us to think that it had a significant gap. In the conference
version of this paper we asserted (with the concurrence of one of the authors of
the original paper) that we were filling a significant gap in their proof.  We  have since realized that
their proof, though lacking in detail, is essentially  correct.}
\item Using our improved analysis for prefix bucketing, we extend the range of validity of the
lower bound results from polynomial space to exponential space.  Specifically we prove a lower bound of
$\Omega((n \log n) / (1+\log \log m - \log \log n))$ that is valid
for $m$ between $n$ and $2^{n}$. (This appears in \cref{ss-main}.)
Note that for polynomially many labels this reduces to $\Omega(n \log n)$ which matches the
known upper bound up to a constant factor.
\item We give an algorithm for the case that  
$m \geq 1+2^{(\log n)^{3}}$ 
whose cost is  $\Omega((n \log n) / (\log \log m))$.  (This appears in \cref{s-ub}.)
This
matches our lower bound up to a constant factor.  Thus the lower bound
is tight for the case that $m=n^{1+C}$ and $C>0$ is a constant, and
also for the case $m \geq 1+2^{(\log n)^3}$.\footnote{This algorithm appeared  as a minor part of a conference
paper~\cite{BKS} and is included here rather that in the journal version~\cite{BKSa} of~\cite{BKS} because
of its close connection  to the other results in this paper.}
\end{itemize}


\iffalse
Our initial study of their proof of the reduction from prefix bucketing to online
labeling \cite{DSZ04} led us to think that there is a significant gap in the proof.
We modified their proof and obtained a correct version of the reduction.
In the conference version of the present paper,
we claimed that we were correcting an apparently significant gap in the previous proof;
we made this claim after checking with one of the authors who agreed with it.
Having now done a more careful comparison of our proof to theirs,
we see that while the proof in \cite{DSZ04} has some misleading statements and missing details,
it is essentially correct.
We present the details of our modification in order to clarify the ambiguities that were present in \cite{DSZ04}.
\fi
%%MIKE 3/31/15 END


 We summarize known results about deterministic algorithms in \cref{tab-1}. 
(There is a  brief discussion of randomized algorithms below.)
% (Subsequent to the work presented in this paper, 
%a subset of the authors \cite{BKS2} showed that  $\Omega(n \log(n))$
%lower bound 
%for the case $m=n^{1+C}$ also holds for randomized algorithms).

\paragraph{Further developments and open questions}
The online labeling problem has three parameters: the number  $m$ of possible labels, the number $n$ of items, and the size $r$ of the universe of possible items.   The lower bounds we obtain in this paper for the case  $m=\Omega(n^{1+C})$ for some positive $C$, require that $r$ be exponential in $n$.  This contrasts with the lower bounds obtained  in~\cite{BKS} for the case of   $m=O(n)$, where the lower bounds hold even when $r$ is a
(sufficiently large) linear function of $n$.   This leaves open the intriguing question whether in the case of polynomially many
labels $m=n^{1+C}$, one can improve on the $O(n\log n)$ cost algorithm if the range $r$ is not too large, e.g. polynomial
in $m$, or even $O(m\log n)$.

All of the work mentioned so far (both upper and lower bounds) is for the case of deterministic algorithms.  In
work subsequent to this paper, a subset of the authors~\cite{BKS2} showed that the $\Omega(n \log n)$ lower
bound for the case  $m=n^{1+C}$ of polynomially many labels extends for randomized algorithms.  
The results in that paper do not subsume this one because \begin{itemize}
    \item[(a)] the lower bounds for randomized algorithms don't extend to the range that $m$ is superpolynomial in $n$ (as do the results here) and 
    \item[(b)] the lower bound proofs here are conceptually simpler.
\end{itemize}

In the case of  linearly many labels, the $\Omega(n\log n)$ lower bound for randomized algorithms applies, but
this leaves a gap since the best upper bound is the deterministic $O(n \log^2n)$ algorithm.   Is there a better randomized
algorithm in the case of linearly many labels, or can the lower bound be improved?

\begin{table}
\centering
\caption{Summary of known bounds for deterministic online labeling problem.} \label{tab-1}
\begin{tabular}{l l c c}
	\toprule 
	Array size ($m$) &
	Asymptotic bound &
	Lower bound &
	Upper bound
	\\ \toprule

	$m = n$ &
	$\PTheta{n (\log n)^3}$ &
	\cite{BKS} &
	\cite{Zhang}
	\\ \midrule

	$m = cn$,\,constant $c>1$ &
	$\PTheta{n (\log n)^2}$ &
	\cite{BKS} &
	\cite{Itaietal}
	\\ \midrule

% Michal: I have tentativelly commented this out as we do not talk about it in the intro.
%	
%	$m = n ^ {1 + o(1)}$ &
%	$\PTheta{\frac{n (\log n)^2}{\log m - \log n}}$ &
%	\cite{manuscript} &
%	\cite{Itaietal}
%	\\ \midrule

	$m = n ^ {C}$,\, constant $C>1$ &
	$\PTheta{n \log n}$ &
	\cite{DSZ04} &
	\cite{Itaietal}
	\\ \midrule

	$m = n ^ {\omega(1)}$ &
	$\Omega{\left(\frac{n \log n}{1 + \log \log m - \log \log n}\right)}$ &
	[this paper] & 
\\ \midrule

	$m \geq 2^{1+\log^3 n}$ &
	$\PTheta{\frac{n \log n}{\log \log m}}$ &
	[this paper] & \cite{BKS},[this paper]
	\\ \bottomrule
\end{tabular}
\label{table:table_bounds}
\end{table}

\subsection{Overview of the proof}
\label{s-1.1}

Our proof follows the high level structure of the proof from \cite{DSZ04}.
\DIFaddbegin \DIFadd{To prove the lower bound we construct an adversary generating input for a given online labeling algorithm so that the algorithm is forced to make many relabelings.
A substantial property of the adversary is that the run of any online labeling algorithm against the constructed adversary naturally gives a strategy of a game we call }\emph{\DIFadd{Prefix Bucketing}}\DIFadd{, which is defined further in the text.
We show that the cost of the run of the online labeling algorithm and the cost of the prefix bucket strategy are equal up to a constant factor.
Finally we provide a lower bound on the cost of each prefix bucketing strategy which immediately gives a tight lower bound for }\emph{\DIFadd{Online Labeling}}\DIFadd{.
}\DIFaddend In the remainder of the introduction we sketch the \DIFdelbegin \DIFdel{two
parts }\DIFdelend \DIFaddbegin \DIFadd{parts in more detail}\DIFaddend , and relate our proof to the one in \cite{DSZ04}.

\paragraph{Reducing Prefix Bucketing to Online Labeling}

%%MIKE 3/31/15:BEGIN
Dietz et al. \cite{DSZ04} describe a particular adversary
for the labeling problem and  show that given any algorithm for online labeling, the behavior of the algorithm
against the adversary can be used to construct a strategy for prefix bucketing.   In \cref{s-olp} we will present a
 modification of their adversary, 
and in \cref{ss-b to ol} we give a full proof of the
connection to prefix bucketing.  We now sketch the adversary construction and the reduction.
%%MIKE 3/31/15 END
\iffalse
 If one can show that the cost of the derived bucketing strategy is no more than a constant times  the cost paid by the algorithm for relabelings then a lower bound on bucketing
will give a similar lower bound on the cost of any online labeling algorithm.  Unfortunately, their proof sketch does not show this.
In particular, a single relabeling step may correspond to a bucketing step whose cost is $\Omega(\log n))$, and this
undermines the reduction.
This may happen when inserting $\Theta(\log n)$ items into an empty segment of size $n^\epsilon$ without triggering any relabelings.
We construct a different adversary for which one gets the needed correspondence
between relabeling cost and bucketing steps.
\fi

The goal in constructing an adversary is to force any online algorithm to perform many relabelings during insertion of $n$ items.
%The adversary is described
%in detail in \cref{s-olp} here we provide a high level description.
As the adversary proceeds we will refer to items that have been inserted by the adversary as  
{\em active}, and two active items are {\em adjacent} if there is no active item between them. 
The adversary starts by inserting the seven equally spaced items including the minimum and maximum items 1 and $r=1+3\cdot 2^{n-6}$. 
Each successive inserted item will be  the average (rounded down) of some pair of adjacent
active items.   The central issue in defining the adversary is to determine at each step which
adjacent  pair of active items to choose. 

It is illuminating to think of this in terms of the file maintenance problem mentioned earlier.
In this reformulation the label space $\natInt{1}{m}$ is associated to an array indexed by $\natInt{1}{m}$
and an item  labeled by $j$ is viewed as stored in location $j$.
Intuitively, the adversary wants to choose each successive insertion to be the
average of two adjacent items that appear in a ``crowded'' region of this array.  The hope is that this will
eventually force the algorithm to move many items within the array (which corresponds to relabeling them).
The problem is to make precise the notion of ``crowdedness''.
Crowding within the array occurs at different scales (so a small crowded region may lie inside a large uncrowded region)
and we need to find a pair of adjacent items with the property that all regions containing the pair are somewhat crowded.

With the array picture in mind,
we  call an interval of labels a \emph{segment}, and say that a label is {\em occupied} if there is an item assigned to it.
The \emph{density} of a segment is the fraction of occupied labels.

As a guide to picking each successive item, the adversary maintains a sequence (\emph{hierarchy}) of nested segments.  Each successive segment in the hierarchy
has size at most half the previous segment, and its density  is within a constant factor of the density of the previous segment.
The hierarchy ends with a segment having between 2 and 7 items.  The next item to be inserted is the average (rounded down)
of the two middle items in the lowest segment of the hierarchy.

%%%MIKE BEGIN 5-3-15
In \cite{DSZ04}, the authors show that there is always a \emph{dense point}, which is
a point  with the property that every segment containing it
has density at least half the overall density of the label space.  They use this as the basis for building the hierarchy at each
step.
We build a hierarchy with similar properties to theirs using a sowmewhat simpler argument.


Before the \DIFdelbegin \DIFdel{8th }\DIFdelend \DIFaddbegin \DIFadd{8-th }\DIFaddend insertion, the hierarchy consists of just the single segment $\natInt{1}{m}$ \DIFaddbegin \DIFadd{consisting of the equally spaced 7 items}\DIFaddend .
After each subsequent insertion,
the algorithm $\A$ specifies the label of the next item and (possibly) relabels some items. Say that a label is {\em impacted} if
it is either assigned by the algorithm to some item during this step, or is freed up (because the item previously
assigned to it was  assigned to a different label.) The adversary then updates the hierarchy as follows.  For the hierarchy immediately prior to the insertion, the {\em critical segment} is the smallest segment in the hierarchy that contains all 
labels impacted by the insertion.
The new hierarchy agrees with the previous hierarchy on all segments from the largest segment to 
the critical segment.   Beginning from the critical segment the rest of the hierarchy is ``rebuilt'' one segment at a time.

If the smallest segment selected so far is $S$ and $S$ has at most seven occupied labels then
the hierarchy stops; it will follow from the construction that the final segment has at least
two occupied labels.  Otherwise we choose a successor to $S$ as follows.
Define the
{\em left buffer} of $S$
to be the smallest subsegment of $S$ that starts at the minimum label of $S$ and
includes at least 1/8 of the occupied labels of $S$, and the {\em right buffer} of $S$ to be the smallest subsegment that ends at the maximum label of $S$
and includes at least 1/8 of the occupied \DIFdelbegin \DIFdel{items }\DIFdelend \DIFaddbegin \DIFadd{labels }\DIFaddend of $S$.
Let $S'$ be the segment obtained from $S$ by
deleting the left and right buffers. The successor segment of $S$ in the hierarchy is
a shortest subsegment of $S'$ that contains exactly half (rounded down) of the occupied labels of $S'$.
%%%MIKE END 5-3-15

%%%MIKE BEGIN 5-3-15
It remains to prove that the algorithm will make many relabelings on the sequence
of items produced by the adversary.  Following  \cite{DSZ04}, we 
do this by relating online labeling to the prefix bucketing game.
(Our definition of the game differs slightly from that in \cite{DSZ04}.)
%%%MIKE END 5-3-15

\DIFdelbegin \DIFdel{A }\DIFdelend \emph{\DIFdelbegin \DIFdel{prefix }\DIFdelend \DIFaddbegin \DIFadd{Prefix }\DIFaddend bucketing of $n$ items into $k$ buckets} (numbered 1 to $k$) is a one player game consisting of $n$ steps.
At the beginning of the game
all the buckets are empty. In each step a new item arrives and the player selects an index $p \in \natInt{1}{k}$.
The new item as well as all items in buckets $p+1,\dotsc,k$ are moved into bucket $p$ at a cost equal to the
total number of items in bucket $p$ after the move.   The run of the game is therefore completely specified by
the sequence $p^1,\ldots,p^n$, where $p^t$ is the bin into which the player placed the new item at step $t$. 
The goal is to minimize the total cost of $n$ steps
of the game.   
%%%MIKE BEGIN 5-3-15
Note that the items in this game are indistinguishable (i.e. they have no ``names'').
%%%MIKE END 5-3-15

The lower bound on online labeling is obtained by showing that for any online
labeling algorithm $\A$, the behavior of $\A$ against the adversary described above, can be used to
give a strategy for prefix bucketing.
Consider the run of an arbitrary online labeling algorithm $\A$
against the given adversary.  At each step $t$, the adversary determines a particular
level $p^t$ of the hierarchy to be the critical level (which is always at most $k=\lceil \log m \rceil$,
where in this paper $\log x$ stands for the binary logarithm function.)
 Consider the sequence $p^1,\ldots,p^n$ as a  sequence of placements defining a prefix bucketing
of $n$ items into $k=\lceil \log m \rceil$ buckets. It turns out that
the total cost of the prefix bucketing is within a constant factor of  the total number of relabelings
performed by the online labeling algorithm. Hence, a lower bound on the cost of a prefix bucketing of $n$ items into $k$
buckets will imply a lower bound on the cost of the algorithm against our adversary.

The connection between the cost of $\A$ against the adversary, and the cost of the associated
prefix bucketing is obtained as follows.
We  make the assumption (which can be shown to hold without loss of generality) that the algorithm is \emph{lazy}, which means
that at each  step the set of items that are relabeled is a contiguous block of items that includes
the newly inserted items.    The cost of the bucketing merge step $p^t$ at step $t$ 
is at most the number of items in
the critical segment, so to relate this to the cost incurred by the online labeling algorithm, it is enough
to argue that at step $t$ a constant fraction of the items in the critical segment were relabeled.  This is done
by arguing that  for each successor (sub)segment of the critical segment, either all labels in its left buffer
or all labels in its right buffer were reassigned, and the total number of such items is a constant fraction of the items
in the critical segment. \DIFaddbegin \DIFadd{In the case when the critical segment has no successor (sub)segment, the amount of items 
in the critical segment is less than a constant defined later and the cost of the online labeling step is exactly one.
}\DIFaddend 


\paragraph{\DIFdelbegin \DIFdel{An }\DIFdelend Improved Analysis of Bucketing}

It then remains to give a lower bound
for the cost of prefix bucketing.  This was previously given by Dietz et al. \cite{DSZ04} for $k\in \Theta(\log n)$.
We give a different and simpler proof that gives an asymptotically optimal bound for $k$ between $\log n$ and $O(n^\epsilon)$.
We define a family of trees called $k$-admissible trees and show that the cost
of bucketing  for $n$ and $k$, is between $dn/2$ and $dn$ where $d$ is the minimum depth of a $k$-admissible tree
on $n$ vertices.  We further show that the minimum depth of a $k$-admissible tree on $n$ vertices
is equal $g_k(n)$ which is defined to be the smallest $d$ such that $\binom{k+d-1}{k} \geq n$.
This gives a  characterization of the optimal cost of prefix bucketing (within a factor of 2).  When we apply
this characterization we need to use estimates of $g_k(n)$ in terms of more familiar functions (\cref{lm:lower_bound_d}),
and there is some loss in these estimates.

\section{The Online Labeling Problem}\label{s-olp}

Here we provide the formal definition of the online labeling problem.
Let $m$ be an integer. We have a totally ordered set $U$ of {\em items} which we assume (without loss of generality) is
a set of positive integers. 
An online labeling algorithm $\A$ with range $m$ is an algorithm that on input sequence $y^1,y^2,\dotsc,y^t$ of distinct elements
from $U$ gives
an \emph{allocation $f : \{y^1,y^2,\dotsc,y^t\} \rightarrow \natInt{1}{m}$} that respects the natural ordering of $y^1,\dotsc,y^t$, so that
for $x,y\in \{y^1,y^2,\dotsc,y^t\}$, $f(x) < f(y)$ if and only if $x < y$. We refer to $y^1,y^2,\dotsc,y^t$
as \emph{items}. The trace of $\A$ on a sequence $y^1,y^2,\dotsc,y^n\in U$ is the sequence $f^0,f^1,f^2,\dotsc,f^n$ of functions
such that $f^0$ is the empty mapping and for $t=1,\dotsc,n$, $f^t$ is the output of $\A$ on $y^1,y^2,\dotsc,y^t$.
For the trace $f^0,f^1,f^2,\dotsc,f^n$ and $t=1,\dotsc,n$, we say that $\A$ \emph{relocates $y\in \{y^1,y^2,\dotsc,y^t\}$ at step $t$}
if $f^{t-1}(y)\not=f^t(y)$. In particular, $y^t$ is relocated at step $t$.
For the trace $f^0,f^1,f^2,\dotsc,f^n$ and $t=1,\dotsc,n$, $Rel^t=Rel^t_\A(y^1,\ldots,y^n)$  denotes the set of relocated items at step $t$.
The cost of $\A$ incurred on $y^1,y^2,\dotsc,y^n$ is $\chi_\A(y^1,\dotsc,y^n)=\sum_{t=1}^n |Rel^t|$. 
%where $Rel$ is measured with respect to the trace of $\A$ on $y^1,y^2,\dotsc,y^n$.

 
The maximum cost
$\chi_\A(y^1,\dotsc,y^n)$ over all sequences $y^1,\dotsc,y^n$ of distinct items from $U$  is denoted $\chi_\A(n,U)$;
we write $\chi_\A(n)$ in the case that $U$ is the set of positive integers.
This maximum  is well-defined since the cost of any algorithm on any sequence of length $n$ is at most $\sum_{i=1}^n i = n(n+1)/2$.    
We define $\chi_m(n)$ to be the smallest cost $\chi_\A(n)$ that
can be achieved by any algorithm $\A$ with range $m$.
%% Michal: ``algorithm with range m'' is a technical term, so ``label set $\{1,\dots,m\}$.'' is not good.

\subsection{The Main Theorem}
\label{subsec:main theorem}
Our main lower bound result for  $\chi_m(n)$ is:

\begin{theorem}
\label{thm:main}
There are positive constants $\thmA$, and $\thmB$ so that the following holds: For integers $m,n$ satisfying
$\thmA \leq n \le m \le 2^{n}$.


$$\chi_m(n) \geq   \thmB \cdot \frac{n \log n}{1 + \log \log m - \log \log n}.$$

\end{theorem}


To prove the theorem we fix an online labeling algorithm $\A$ and describe an adversary that, based on the 
behavior of $\A$, selects a sequence $y^1,y^2,\dotsc,y^n$ of items that will cause
the algorithm to incur the desired cost.  The input sequence selected by the adversary will be a subset of
$\{1,\ldots,2^n\}$.

 In the next subsection we  describe the adversary,
and state \cref{l-main}, which asserts a lower bound on the cost incurred
by algorithm $\A$ on  the sequence produced by the adversary. \Cref{thm:main} follows
immediately from this lemma.

\subsection{Adversary Construction}
\label{ss-adversary}

%%%MIKE BEGIN 5-12-15
Fix $n$ and $m$ with $m\ge n$  and 
fix an online labeling algorithm $\A$ with range $m$.  Our adversary will select 
the sequence of items $y^1,\dotsc,y^n$, one by one.    During step $t$  the adversary chooses $y^t$,
and the algorithm specifies the labeling of $y^1,\ldots,y^t$, which is denoted $f^t$.  All of the $y^j$ will be chosen
from the set $\{1,\ldots,2^{n}\}$.

The choice of $y^t$ will depend on
the labeling $f^{t-1}$ of $y^1,\ldots,y^{t-1}$ from the previous step, and we'll need some notation and
observations to describe this choice.   

Any interval $\natInt{a}{b} \subseteq \natInt{1}{m}$ of label values is called a \emph{segment}.
The \emph{population of a segment $S$ after step $t-1$} is $\pop^{t-1}(S) = (f^{t-1})^{-1}(S)$ and the
\emph{weight of $S$ after step $t-1$} is $\weight^{t-1}(S) = |\pop^{t-1}(S)|$. (As usual, $g^{-1}(A) = \{ x : g(x) \in A \}$.)
In particular,  $\pop^0(S) = \emptyset$ and $\weight^0(S) = 0$.
The \emph{density of $S$ after step $t-1$}, denoted $\rho^{t-1}(S)$, is defined to be $\weight^{t-1}(S) / |S|$. For a positive
integer $b$, and for any segment $S$ such that $\weight^{t-1}(S) \geq 2b$, we define
$\densify^{t-1}(S,b)$ to be the minimum  subsegment $T$ of $S$ (with respect to the order that
orders segments by size, and orders segments of the same size by their left endpoint) satisfying:

\begin{itemize}
\item \DIFdelbegin \DIFdel{$pop^{t-1}(T)$ }\DIFdelend \DIFaddbegin \DIFadd{$\pop^{t-1}(T)$ }\DIFaddend does not contain any of the  $b$ largest or smallest elements of $\pop^{t-1}(S)$.
\item  $\weight^{t-1}(T)=\lfloor (\weight^{t-1}(S)-2b)/2 \rfloor$.
\end{itemize} 

Hence, $\densify^{t-1}(S,b)$ is a densest subsegment of $S$ that contains half (rounded down)  of the middle $\weight^{t-1}(S)-2b$ items
stored in $S$.


\iffalse
These $b$ items on either side of $T$ form the left and right buffers described in \cref{s-1.1}.
\fi
%%%BEGIN MIKE 5-13-15
\begin{proposition}
\label{prop:shrinks}
For a segment $S$, if $\weight^{t-1}(S) \geq 2b$\DIFaddbegin \DIFadd{, }\DIFaddend then $|\densify^{t-1}(S,b)| \leq |S|/2$.
\end{proposition}

\begin{proof}
Let $S'$ be the largest subsegment of $S$ such that $\pop^{t-1}(S')$ excludes the
smallest $b$  members and the largest $b$ members of $\pop^{t-1}(S)$.  
Let $L'$ be the smallest subsegment of $S'$ that starts at the left endpoint of $S'$
and satisfies  \DIFdelbegin \DIFdel{$\weight^{t-1}(L)=\lfloor \weight^{t-1}(S')/2 \rfloor$}\DIFdelend \DIFaddbegin \DIFadd{$\weight^{t-1}(L')=\lfloor \weight^{t-1}(S')/2 \rfloor$}\DIFaddend .  Let $R'$
 be the smallest subsegment of $S'$ that ends at the right endpoint of $S'$
and satisfies  \DIFdelbegin \DIFdel{$\weight^{t-1}(L)=\lfloor \weight^{t-1}(S')/2 \rfloor$}\DIFdelend \DIFaddbegin \DIFadd{$\weight^{t-1}(R')=\lfloor \weight^{t-1}(S')/2 \rfloor$}\DIFaddend .
$L'$ and $R'$ are disjoint subsegments of $S$ so the smaller of them has
size at most $|S|/2$. Since $\densify^{t-1}(S,b)$ is the smallest subsegment $T$ of $S'$ with
\DIFdelbegin \DIFdel{$\weight^t{T} = \lfloor \weight^{t-1}(S')/2 \rfloor$}\DIFdelend \DIFaddbegin \DIFadd{$\weight^{t-1}(T) = \lfloor \weight^{t-1}(S')/2 \rfloor$}\DIFaddend , the proposition follows.
\end{proof} 

%%%END MIKE 5-13-15

Suppose $S$ is a segment with $\weight^{t-1}(S)=\ell \geq 2$.  Let $x_1< \ldots < x_\ell$ denote the elements
of $\pop^{t-1}(S)$.  We define $\midp^{t-1}(S)= \lceil (x_{\lceil (\ell-1)/2 \rceil} + x_{\lceil (\ell+1)/2 \rceil} )/ 2 \rceil$. Thus \DIFdelbegin \DIFdel{$\midp^t(S)$ }\DIFdelend \DIFaddbegin \DIFadd{$\midp^{t-1}(S)$ }\DIFaddend is obtained by averaging the two middle items whose labels belong to  $S$. 

Let $y^1,y^2,\dotsc,y^t$ be the first $t$ items inserted  and let $Rel^t$ be the items
that are relabeled by
$\A$ in response to the insertion of $y^t$. The \emph{busy segment $B^t\subseteq \natInt{1}{m}$ at time $t$} is
the smallest segment that contains $f^t(Rel^t)\cup f^{t-1}(Rel^t \setminus \{y^t\})$. \DIFaddbegin \DIFadd{(As usual, $f(A) = \{f(x) : x \in A\}.$)
}\DIFaddend %%%MIKE BEGIN 5/3/15
 (Equivalently, say that a label
is {\em  impacted at step $t$} if either it is unassigned under $f^{t-1}$ and assigned under $f^t$ or assigned under
$f^{t-1}$ and unassigned under $f^t$ or assigned to different items under $f^{t-1}$ and $f^t$. Then $B^t$ is
the smallest segment  of labels that contains all labels that are impacted at step $t$.)
%%%MIKE END 5/3/15

We say
 that the algorithm $\A$ is \emph{lazy} if
all the items that are mapped by $f^{t-1}$ to $B^t$ are relocated at step $t$, i.e., $\pop^{t-1}(B^t) = Rel^t \setminus \{y^t\}$.
Proposition~4 in \cite{BKS} shows that any algorithm $\A$ can be modified to get a lazy algorithm $\A'$
whose cost on any insertion sequence is no more than $\A$. 
%%%MIKE BEGIN 5-3-15
(The intuition behind this fact is that if there is an item $y$
such that $f^t(y)=f^{t-1}(y) \in B^t$ and (without loss of generality) $y<y^t$ then we could defer relocating
all of the items less than $y$, thereby shrinking $B^t$.)
%%%MIKE END 5-3-15
Therefore it suffices to prove our lower bound
for lazy algorithms, and from now on we assume that  $\A$ is lazy.
We record the following simple observation:

\begin{proposition}
\label{l-busy}
Consider an execution of a lazy algorithm $\A$ on $y^1,\ldots,y^n$.  
For any
time $t\in \natInt{1}{n}$, $|Rel^t|=1+\weight^{t-1}(B^t)$.
\end{proposition}



We are now ready to present our adversary.   For the first 7 steps, the adversary chooses
$y^1,y^2,\ldots,y^7$ by $y^i=1+2^{n-7}(i-1)$; note
that the difference between consecutive $y^i$ is $2^{n-7}$ and that $y^7 < 2^n$.

From
then on  the adversary operates at each step $t$ by constructing a nested
sequence of segments $\natInt{1}{m} \supseteq S^t_1 \supseteq \dotsb \supseteq S^t_{\depth(t)}$ called the
{\em hierarchy at step $t$}.   The procedure for specifying this hierarchy is given  below. 
The final (smallest) segment will (necessarily) have weight at least 2,
and $y^t$ is selected to be $\midp^{t-1}(S^t_{\depth(t)})$.   The construction of the hierarchy
requires that when $S^t_i$ is defined, we also specify an integer $b^t_i$ (which is used as the
buffer parameter in the function $\densify$.) 
After $y^t$ is selected, the algorithm $\A$ determines the labeling $f^t$ of $y^1,\ldots,y^t$.
This (together with $f^{t-1}$) determines the busy segment $B^t$ defined earlier.
The adversary defines the integer $p^t$, called the {\em critical level at step $t$},
 to be the largest level such that  $B^t \subseteq S^t_{p^t}$, and $p^t$ is used
in the next iteration to help define the hierarchy at step $t+1$.

\DIFaddbegin \DIFadd{We present the }\emph{\DIFadd{Adversary}} \DIFadd{which generates the input. Recall that the adversary keeps 
the hierarchy of nested }\emph{\DIFadd{segments}} \DIFadd{having, imprecisely said, increasing }\emph{\DIFadd{density}}\DIFadd{. 
The idea is to insert items near a dense point and therefore the next item is chosen as the 
average of median items of the last segment. The relabelings may substantially change the
density of the affected segments and the hierarchy must therefore be rebuilt starting 
from the }\emph{\DIFadd{critical segment}}\DIFadd{.
}

\DIFaddend \medskip\noindent {\bf Adversary($\A,n,m$)} (We assume that $n \geq 8$)

\begin{itemize}
\iffalse
\smallskip\noindent Set $p^0=0$.
\fi
\item  \emph{The first 7 steps:} For $t$ from 1 to 7, set \DIFdelbegin \DIFdel{$y^t= 1+ (i-1)2^{n-7}$}\DIFdelend \DIFaddbegin \DIFadd{$y^t= 1+ (t-1)2^{n-7}$}\DIFaddend .

\item \emph{Initialization for remaining steps:}
\begin{itemize}
\item $\depth(7)=1$.
\item $S^7_1=\natInt{1}{m}$.
\item $p^7=1$.
\item $b^7_1=1$.
\end{itemize}
\item \emph{The remaining steps.} For $t$ from 8 to $n$ do
%\begin{itemize}
%\item Build the hierarchy at step $t$ according to the following rules.
\begin{itemize}
\item
Set $i=1$ (\emph{$i$ indexes the levels of the hierarchy})
\item
\emph{Preservation Rule for first $p^{t-1}$ levels:} While $i \leq p^{t-1}$ 
\begin{itemize}
\item Set $S^t_i=S^{t-1}_{i}$ 
\item Set $b^t_i=b^{t-1}_i$.  
\item Increase $i$ by 1.
\end{itemize}
(\emph{The hierarchy segments and buffer parameters at step $t$ agree with those at step $t-1$ up through level $p^{t-1}$.})
\item
\emph{Rebuilding Rule for $i>p^{t-1}$:} While $\weight^{t-1}(S^t_{i}) \ge 8$
\begin{itemize}
\item Set $S^t_i = \densify^{t-1}(S^t_{i-1},b^t_{i-1})$.
\item Set $b^t_i = \lceil \weight^{t-1}(S^t_i)/8 \rceil$.
\item Increase $i$ by 1.
\end{itemize}
\item \emph{The hierarchy is complete; record the depth.}
 $\depth(t)=i$.
\item \emph{Choose the next item to insert:}
Set $y^t = \midp^{t-1}(S^t_{\depth(t)})$.
\item
\emph{Determine the next critical level:}
Run $\A$ on $y^1,y^2,\dotsc,y^t$ to get $f^t$. Calculate $Rel^t$ and $B^t$.
Set $p^{t}$ to be the largest integer $j \in \natInt{1}{\depth(t)}$ such that $B^{t} \subseteq S^{t}_j$.
\end{itemize}
\end{itemize}

%\smallskip
\noindent\emph{Output:} $y^1,y^2,\dotsc,y^n$.


\medskip
For $t \geq 8$, $y^t$ is the midpoint between two adjacent earlier items.
After step 7, two adjacent items differ by $2^{n-7}$ so an easy induction shows that
after step $t$, any two adjacent items differ by a multiple of $2^{n-t}$.  Therefore the selected items
are distinct integers in the set $\{1,\ldots,2^n\}$.


The following claim about the adversary implies \cref{thm:main}.

\begin{lemma}\label{l-main}
Let $m,n$ be positive integers such that $n \le m$. 
Let $\A$ be a lazy online labeling algorithm with range $m$.
Let $y^1,y^2,\dotsc,y^n$ be the output of {\bf Adversary($\A,n,m$)}.
Then:
$$\chi_\A(y^1,y^2,\dotsc,y^n) \ge \frac{1}{512} \cdot \frac{n \log n}{ 1 + \log \lceil \log m \rceil - \log \log n} - \frac{n}{6}.$$
\end{lemma}

 The constants are chosen for
ease of exposition and can certainly be improved.
%
% MK: if one were to set the bucketing k = \lfloor log m \lfloor then we could
%     drop the ceiling around log m provided that k = \lfloor log m \floor > log n.
%     This comes from a requirement in Lemma 17 for the bucketing lower bound.
%     This would be true for example in m > 10n.


To prove the lemma we model the interaction between the adversary and the algorithm by  
a so-called \emph{prefix bucketing game} and relate the cost of the prefix bucketing to the cost $\chi_\A(y^1,y^2,\dotsc,y^n)$ (\cref{l-b2l}).  Then
we  bound the cost of  prefix bucketing from below (\cref{l-lbb}). These results are combined at the end of \cref{ss-main} to prove \cref{l-main}.

In preparation for this, we prove several useful properties of the adversary.

\begin{lemma}
For any $t\in \natInt{7}{n}$, $\depth(t)\le \log m$.
\end{lemma}

\begin{proof}
%%MIKE 6-27-2015
We will show that for each fixed $t$, $|S^t_{i}| \leq |S^t_{i-1}|/2$ for each $i \in \natInt{2}{\depth(t)}$, from which the result
follows immediately.  We prove this by induction on $t$; the result is vacuous for $t=7$ since $\depth(7)=1$.
Assume $t \geq 8$.  If $i \leq p^{t-1}$, we have $S^t_i=S^{t-1}_i$ and $S^t_{i-1}=S^{t-1}_{i-1}$
so the result follows by induction on $t$.  If $i>p^{t-1}$, then $S^t_i=\densify^{t-1}(S^t_{i-1},b^t_{i-1})$
where $\weight^{t-1}(S^t_{i-1}) \geq 8$ and $b^t_{i-1} \leq \lceil \weight^{t-1}(S^t_{i-1})/8 \rceil < \weight^{t-1}(S^t_{i-1})/2$.
In this case \cref{prop:shrinks} applies. 
\end{proof}


For any $t \in \natInt{1}{n}$ and $i \in \natInt{1}{\depth(t)-1}$, the
difference $S^t_i \setminus S^t_{i+1}$ is a pair of segments 
denoted $L^t_i$ (the portion of $S^t_i$ to the left of $S^t_{i+1}$)
and $R^t_i$ (the portion to the right of $S^t_{i+1}$).

\begin{lemma}
\label{l-b}  For  any $t \in \natInt{1}{n}$ and any $i \in \natInt{1}{\depth(t)-1}$, $\weight^{t-1}(L^t_i) \geq b^t_i$
and $\weight^{t-1}(R^t_i) \geq b^t_i$.
\end{lemma}

\begin{proof}
We prove this by induction on $t$.  

If $i \ge  p^{t-1}$, then $S^t_{i+1}$ is rebuilt at time $t$ and by the \emph{Rebuilding Rule},
$L^t_i$ and $R^t_i$ will each have weight at least $b^t_i$.

If $i < p^{t-1}$ then $S^t_i$ and $S^t_{i+1}$ are preserved at time $t$ and so $L^t_i=L^{t-1}_i$ and $R^t_i=R^{t-1}_i$
are also unchanged, and since $i < p^{t-1}$, 
we have $B^{t-1} \subseteq S^{t-1}_{i+1}=S^{t}_{i+1}$
and consequently, applying the induction hypothesis
we have $\weight^{t-1}(L^t_i)=\weight^{t-1}(L^{t-1}_i) \geq b^{t-1}_i=b^t_i$ 
and $\weight^{t-1}(R^t_i)=\weight^{t-1}(R^{t-1}_i) \geq b^{t-1}_i=b^t_i$.  
\end{proof}

This lemma reflects a subtle point in the adversary.  We defined $b^t_i=\lceil \weight^{t-1}(S^t_i)/8 \rceil$ only for $i >p^{t-1}$,
while it might seem more natural to use this definition for all $i$. 
The given definition which sets $b^t_i=b^{t-1}_i$ for $i \leq p^{t-1}$ is
needed for the induction step in the above proof.

Next we relate the cost of relabelings at step $t$ to $(b^t_i:1 \leq i \leq \depth(t))$:

\begin{lemma}
\label{l-rel}
If $\A$ is lazy then for any $t \in \natInt{1}{n}$, $|Rel^t| \ge \sum_{i=p^t+1}^{\depth(t)} b^t_i$.
\end{lemma}

\begin{proof}
We have $|Rel^t| \geq 1$.
If $p^t = \depth(t)$ then the sum in the inequality evaluates to zero. If $p^t=\depth(t)-1$ then the sum is just $b^t_{\depth(t)}$
which is 1, since
by the termination condition in the \emph{Rebuilding Rule}, $\weight^{t-1}(S^t_{\depth(t)}) \in \natInt{2}{7}$, so $b^t_{\depth(t)} = \lceil \weight^{t-1}(S^t_{\depth(t)})/8 \rceil = 1$.

So assume $p^t < \depth(t)-1$.
First we note that $B^t \cap 
S^t_{\depth(t)} \neq \emptyset$.
By the definition of the adversary, $y^t$
is between two items $u$ and $v$ such that $f^{t-1}(u),f^{t-1}(v) \in S^t_{\depth(t)}$.  If $f^t(y^t)\in S^t_{\depth(t)}$ then
$B^t \cap S^t_{\depth(t)} \neq \emptyset$; otherwise
at least one of $u$ and $v$ is relabeled at step $t$
in which case $f^{t-1}(u)\in B^t$ or $f^{t-1}(v) \in B^t$ and
again $B^t \cap S^t_{\depth(t)} \neq \emptyset$.


\begin{figure}
  \begin{center}
    \includegraphics[scale=1]{lab-fig1}
    \caption{A typical position of $S_L,S_R$ and $B^t$.}
    \label{f-1}
  \end{center}
\end{figure}

The set $S^t_{p^t+1} \setminus S^t_{\depth(t)}$ 
is the union of the left subsegment $S_L$ and the right subsegment $S_R$ (\cref{f-1}). Since $B^t$ is a segment that has nonempty
intersection with $S^t_{\depth(t)}$ and is not
a subset of $S^t_{p^t+1}$ (by the definition of $p^t$),
$B^t$ contains at least one of $S_L$ and $S_R$.

\iffalse
Suppose for contradiction that $B^t$ misses some label $a_L \in S_L$ and some label $a_R \in S_R$. 
Since $B^t$ is  nonempty
intersection with $S^t_{\depth(t)}$ The segment
$S^t_{\depth(t)}$ is a subsegment of $\natInt{a_L}{a_R}$
that contains $y^t$ (since the adversary chose $y^t$ 
to be between two items
$u,v$ that were stored  in $S^{t}_{\depth(t)}$ by $f^{t-1}$.
If neither $u$ nor $v$ was relabeled by $f^t$ then (by the laziness of $\A$),
$Rel^t=\{y^t\}$ and $B^t \subseteq S^t_{\depth(t)}$ which would imply $p^t=\depth(t)$,
contradicting our assumption
that $p^t < \depth(t)-1$.  So at least one of $u,v$ belongs to $Rel^t$. Thus
$B^t$ is a subsegment of $\natInt{a_L}{a_R} \subseteq S^{p^t+1}$ having nonempty intersection with  $S^t_{\depth(t)}$. Since, by assumption $B^t$ is a segment containing neither  $a_L$ nor $a_R$
it is a subset of $\natInt{a_L}{a_R}$ which contradicts the fact (from the definition of $p^t$) 
that $B^t$ is not a subset of $S^t_{p^t+1}$.   
\fi

So assume, without loss of generality, that $S_L \subseteq B^t$.
Since $\A$ is lazy, $|Rel^t|=1+|\pop^{t-1}(B^t)| \geq 1+|\pop^{t-1}(S_L)|$.
Now $S_L$ is the disjoint union of the sets $L^t_i$ defined prior to \cref{l-b}, for $i \in [1+p^t,\depth(t)-1]$.  
By \cref{l-b}, we have $|Rel^t| \geq 1+ \sum_{i=p^t+1}^{\depth(t)-1} b^t_i$.
Since $b^t_{\depth(t)}=1$ we obtain the inequality
of the lemma.
\end{proof}

\iffalse
Let $r_{\mathrm{min}}=\min(Rel^t)$ and $r_{\mathrm{max}}=\max(Rel^t)$ be the minimal and maximal item relocated when going from $f^{t-1}$ to $f^t$.

By the definition of $p^t$ we know that the busy segment $B^t$ is not a subset of $S^t_{p^t+1}$
and so at least one of the following must happen: 
\begin{enumerate}
\item $f^{t-1}(r_{\mathrm{min}}) \in S^t_{p^t} \setminus S^t_{p^t+1}$, or
\item $f^{t}(r_{\mathrm{min}}) \in S^t_{p^t} \setminus S^t_{p^t+1}$, or 
\item $f^{t-1}(r_{\mathrm{max}}) \in S^t_{p^t} \setminus S^t_{p^t+1}$, or 
\item $f^{t}(r_{\mathrm{max}}) \in S^t_{p^t} \setminus S^t_{p^t+1}$. 
\end{enumerate}
Assume that $f^{t-1}(r_{\mathrm{min}}) \in S^t_{p^t} \setminus S^t_{p^t+1}$ or $f^{t}(r_{\mathrm{min}}) \in S^t_{p^t} \setminus S^t_{p^t+1}$,
the other case is symmetric. Let $Y = \{ y \in \{y^1,y^2,\dotsc,y^{t-1}\}:\; r_{\mathrm{min}} \le y < y^t\}$.
Since $\A$ is lazy,  $Y \subset Rel^t$, i.e., all the items between $r_{\mathrm{min}}$ and $y^t$ were relocated when inserting $y^t$.

We claim that all of the items mapped by $S_L$ belong to $Rel^t$, that is the items mapped by $f^{t-1}$ to $S_L$ must all have been relocated
when inserting $y^t$ since they are between $r_{\mathrm{min}}$ and $y^t$ and the algorithm $\A$ is lazy.
For any $i \in \natInt{1}{\depth(t)-1}$, the population of the left subsegment of $S^t_i \setminus S^t_{i+1}$ at time $t-1$ is at least $b^t_i$, by the definition of $S^t_{i+1}$.
These subsegments partition $S_L$.
Hence, $\sum_{i=p^t+1}^{\depth(t)-1} b^t_i \le |\pop^{t-1}(S_L)| \le |Y| < |Rel^t|$. Since $b^t_{\depth(t)}=1$, the lemma follows.
\fi

Next we  bound $b^t_i$ from below.
For $t \in \natInt{1}{n}$ and $i \in \natInt{1}{\depth(t)}$, let $\starts(t,i)$ denote the largest $t' \le t$ 
such that $p^{t'-1} < i$. Thus $S^{\starts(t,i)}_i$ was rebuilt, and $S^u_i$ was preserved for
$u \in (\starts(t,i),t]$. 
Similarly, let $\sends(t,i)$ be the least $t'>t$ such that $p^{t'-1} < i$, if such a $t'$ exists, and $n+1$ otherwise.
Thus, $\sends(t,i)$ is the earliest time $t' > t$
such that $S^{t'}_i$ was rebuilt.


% The next proposition follows from definitions.
%
% \begin{proposition}
% For any $t\in\{1,\dots,n\}$, $i\in[\depth(t)-1]$, and any time $t'\in [\starts(t,i), \sends(t,i)-1]$,
% $B^{t'} \subseteq S^t_i$. Furthermore, if $\sends(t,i)\le n$ then $B^{\sends(t,i)} \setminus S^t_i \not=\emptyset$.
% \end{proposition}


\begin{lemma}\label{l-p1}
For $t \in \natInt{7}{n}$ and $i \in \natInt{2}{\depth(t)}$,
\begin{align*}
b^t_{i} \ge \frac{1}{64}(\weight^{t-1}(S^t_{i-1}) - \weight^{t-1}(S^t_{i})).\
\end{align*}
\end{lemma}


\begin{proof} %% {\cref{l-p1}}
For $t=7$, $\depth(t)=1$ and the lemma holds. 
Assume $t\ge 8$ and $i \in \natInt{2}{\depth(t)}$.
The definitions of $\starts(t,i)$ and $\sends(t,i)$ imply $S^u_{i}=S^t_{i}$ for
$u \in \natInt{\starts(t,i)}{\sends(t,i)-1}$. For any
segment $S$ containing $S^t_{i}$, and any  $s,s'$ satisfying $\starts(t,i)\le s < s' < \sends(t,i)$,
\begin{align*}
\weight^{s'-1}(S) - \weight^{s-1}(S) = s'-s,
\end{align*}
since one item is added to $S$ and none are removed at each step 
$u \in \natInt{\starts(t,i)}{\sends(t,i)-1}$.
Let $s=\starts(t,i)$. Then $\starts(t,i-1) \le s \le t <  \sends(t,i) \le \sends(t,i-1)$ so
\begin{align*}
\weight^{t-1}(S^t_{i-1}) - \weight^{t-1}(S^t_{i}) &= \weight^{s-1}(S^t_{i-1}) - \weight^{s-1}(S^t_{i}) \cr
&= \weight^{s-1}(S^s_{i-1}) - \weight^{s-1}(S^s_{i}).
\end{align*}
Also
\begin{align*}
b^t_{i} = b^s_{i} = \lceil \weight^{s-1}(S^s_{i})/8\rceil \ge  \weight^{s-1}(S^s_{i})/8.
\end{align*}
Since $8 \le \weight^{s-1}(S^t_{i-1})$ and $\weight^{\starts(t,i-1)-1}(S^t_{i-1}) \le \weight^{s-1}(S^t_{i-1})$,
\begin{align*}
b^s_{i-1} = b^{\starts(t,i-1)}_{i-1} &= \lceil \weight^{\starts(t,i-1)-1}(S^t_{i-1})/8 \rceil \cr
&\le \weight^{s-1}(S^s_{i-1})/4
\end{align*}
Hence,
\begin{align*}
\weight^{s-1}(S^s_{i}) = \lfloor  (\weight^{s-1}(S^s_{i-1}) - 2b^s_{i-1} )/ 2 \rfloor &\ge \lfloor \weight^{s-1}(S^s_{i-1})/4 \rfloor \cr &\ge \weight^{s-1}(S^s_{i-1})/8.
\end{align*}
Thus, $b^t_{i} \ge \weight^{s-1}(S^s_{i-1})/64 \ge (\weight^{s-1}(S^s_{i-1}) - \weight^{s-1}(S^s_{i}))/64$. The claim follows.
\end{proof}



\begin{corollary}\label{c-weight}
For any $t \in \natInt{1}{n}$ and $j \in \natInt{1}{\depth(t)-1}$,$$\sum_{i=j+1}^{\depth(t)} b^t_{i} \ge 
\frac{1}{64} \cdot \weight^{t-1}(S^t_j) - \frac{1}{8}.$$
\end{corollary}

\begin{proof}
For fixed $t$, sum the inequality in \cref{l-p1} for $i$ from $j+1$ to $\depth(t)$, and note that
$\weight^{t-1}(S^t_{\depth(t)}) \leq 8$.  Then divide through by 64.
\end{proof}


We now come to the main lower bound of this section.


\begin{corollary}\label{c-main}
Let $\A$ be a lazy algorithm. Then 
$$ \chi_\A(y^1,y^2,\dotsc,y^n)  \ge \frac{1}{64} \sum_{t=1}^n \weight^{t-1}(S^t_{p^t}) - \frac{n}{8}.$$
\end{corollary}

\begin{proof}
$\chi_\A(y^1,\ldots,y^n)$ is at least $\sum_{t=1}^n |Rel^t|$. Now combine \cref{l-rel,c-weight}. (For $t$ such that $p^t=\depth(t)$ use the fact that $|Rel^t|\geq 1> \frac{1}{64} \cdot \weight^{t-1}(S^t_{p^t}) - \frac{1}{8}$.)
\end{proof}


\section{Prefix Bucketing}
\label{s-pb}

Prefix bucketing is a one player game that models the behavior of an algorithm against our adversary.

A {\em  bucket configuration} for $k$ buckets and $t$ items  is a sequence $a_1,\ldots,a_k$ of nonnegative integers
summing to $t$.  One should think of $a_i$ as the number of items in bucket $i$.
Given a bucket configuration
$a_1,\ldots,a_k$, {\em placing a new item in bucket $p$} transforms  the configuration as follows:
A new item is added to bucket $p$ and all items in buckets higher than $p$ are moved to bucket $p$.
Buckets $i < p$ are unchanged.  Formally the configuration $b$ produced from $a$ by the placement $p$ satisfies:

\begin{itemize}
\item $b_i=a_i$ for $i < p$,
\item $b_p=1+\sum_{i \geq p} a_i$, and
\item $b_i=0$ for $i>p$.
\end{itemize}
The {\em cost} of this placement  is the number of items $b_p$ in bucket $p$ after the placement.

%%%BEGIN MIKE 5-13-15
A sequence $p^1,\ldots,p^n$ where each $p^i \in \{1,\ldots,k\}$ is called a {\em placement sequence}, and  corresponds
to placing a sequence of $n$ items in the buckets.  
A placement sequence induces a sequence of configurations
$a^0,a^1,\dotsc,a^n$ where $a^0$ has no items,
and $a^t$ is obtained from $a^{t-1}$ by placing $p^t$ and
applying the above transformation.
The sequence
$a^0,\ldots,a^n$ is called a {\em prefix bucketing}.
%%%END MIKE 5-13-15
\iffalse
satisfying: $a^0 = (0,0,\dotsc,0)$ and for $t=1,2,\dotsc,n$, there exists $p^{t} \in \natInt{1}{k}$ such that
\begin{enumerate}
\item $a^t_i = a^{t-1}_i$, for all $i=1,2,\dotsc,p^{t}-1$,
\item $a^t_{p^{t}} = 1+ \sum_{i\ge p^{t}} a^{t-1}_i$, and
\item $a^t_i = 0$, for all $i=p^{t}+1,\dotsc,k$.
\end{enumerate}
\fi
The \emph{cost} $c(a^0,\dotsc,a^n)=
\sum_{t=1}^n a^t_{p^{t}}$ of $a^0,a^1,\dotsc,a^n$ is the sum of the individual placement costs. 

\subsection{Connecting bucketing to online labeling}
\label{ss-b to ol}

Now we show that for any lazy online labeling algorithm $\A$, the adversary
defined in the previous section can be associated  to a prefix bucketing whose cost provides a lower bound
on the cost of $\A$ in labeling the sequence $Y=\{y_1,\ldots,y_n\}$ produced by the adversary. 

Fix a lazy online labeling algorithm $\A$ and for $1 \leq t \leq n$, let $f^t,S^t_i,B^t,p^{t},y^t$ and $f^0,p^0$ be as
defined by the {\bf Adversary($\A,n,m$)}. Let $Y$ denote the set $\{y^1,y^2,\dotsc,y^n\}$.
Recall that our adversary specifies a sequence of critical levels for steps $7$ to $n-1$.

Set $k=\lceil \log m\rceil$.
For $t=0,1,\dotsc,n$ we define a sequence $(A^t_i:1 \leq i \leq k)$ of subsets of $Y$
as follows:
For $t \in \{0,\ldots,7\}$,
$A^t_1=\{y^1,\ldots,y^t\}$ and,   
for $i \in \{2\dotsc,k\}$, $A^t_i = \emptyset$.  For $t\geq 8$
\begin{itemize}
\item $A^t_i = A^{t-1}_i$, for  $i \in \{1,\dotsc,p^{t}-1\}$,
\item $A^t_{p^{t}} = \{y^t\} \cup \bigcup_{i\ge p^{t}} A^{t-1}_i$ and
\item $A^t_i = \emptyset$, for  $i \in \{p^{t}+1,\dotsc, k\}$.
\end{itemize}

For each $t \in \natInt{0}{n}$, let $a^t=a^t_1,\ldots,a^t_k$ be the bucket configuration defined
by $a^t_i=|A^t_i|$.  It is easy to see (by induction on $n$) that $a^0,\ldots,a^n$ is a prefix bucketing
of $n$ items into $k$ buckets with placement sequence $p^1,\ldots,p^n$ (where we define $p^1=\cdots = p^6 = 1$ since
the adversary does not specify critical levels for $t \leq 6$). 
The cost of this bucketing is $c(a^0,a^1,\dots,a^n)=\sum_{i=1}^n |A^t_{p^t}|$. 
We now relate this cost
to the cost of online labeling. We start by noting:

\begin{lemma}\label{l-p2}
For any $t \in \natInt{1}{n}$ and $i \in \natInt{1}{\depth(t)}$,  $f^{t-1}(A^t_i \setminus \{y^t\}) \subseteq S^t_i$,
and therefore $\weight^{t-1}(S^t_{i}) \ge |A^t_{i}| - 1$.
\end{lemma}

\iffalse
\begin{proof}
We prove this by induction on $t$.  For $t=1$ the result is trivial since  each of the sets $A^1_i \setminus \{y^t\}$ is
empty.

Assume $t>1$.    If $A^t_i\setminus \{y^t\}= \emptyset$  then the conclusion is trivial.  So assume
$A^t_i\setminus \{y^t\} \neq \emptyset$.  In particular, $i \leq p^t$.  We also claim that $i \leq p^{t-1}$.
If not then by definition $A^{t-1}_j = \emptyset$.  If $j< $
For $i=p^t$, by the definition of $p^t$ and  $A^t_{p^t}$ and the induction hypothesis we have:

$$
f^{t-1}(A^t_{p^t}\setminus \{y^t\}) = f^{t-2}(A^t_{p^t}\setminus \{y^t\}) =  \bigcup_{i \geq p^t}f^{t-2}( A^{t-1}_i) \subseteq \bigcup_{i \geq p^t} S^{t-1}_i = S^{t-1}_{p^t}=S^t_{p^t}.
$$

Finally consider the case $i<p^t$.  Then $A^t_i=A^{t-1}_i$.  If $i > p^{t-1}$ then this set is empty, and we're done.
So assume $i \leq p^{t-1}$.    
Recall from the definition of the adversary that $i \leq p^{t-1}$ implies $S^t_i=S^t_{i-1}$, and $i \leq p^t$ implies  
that  the busy segment $B^t$ is a subset of $S^t_i$,  which implies
that $\pop^t(S^t_i) =  \pop^{t-1}(S^t_i)\cup \{y^t\}$.  Let $i \in \natInt{1}{k}$.
If $i<p^t$ then by the definition of $A^t_i$ and the induction hypothesis
we have:
$$A^t_i = A^{t-1}_i \subseteq (f^{t-2})^{-1}(S^{t-1}_i) =  (f^{t-2})^{-1}(S^t_i) \subseteq (f^{t-1})^{-1}(S^t_i),$$
as required.


\end{proof}
\fi

\begin{proof}  %% {\cref{l-p2}}
We prove the claim by induction on $t$. For $t=1$, the only non-empty set is $A^1_1=\{y^1\}$ so the claim is true.  Assume $t \geq 2$.
The {\bf Adversary($\A,n,m$)} chooses the sets $S^t_i$ and the item $y^t$,  the algorithm $\A$ outputs $f^t$ and then
the adversary defines $B^t, Rel^t$ and $p^t$.
We distinguish two cases.


\emph{Case $p^{t-1} \leq p^t$:} Suppose first that $i<p^{t-1}$.
Since $i\le p^{t-1}< p^{t}$, $A^{t}_i = A^{t-1}_i$. 
We have $B^{t-1} \subseteq S^{t-1}_{p^{t-1}} \subseteq S^{t-1}_i = S^t_i$, where
the first containment follows from the definition of $p^{t-1}$ and the last equality follows from the definition of $S^t_i$.
Since  $y^t \not\in A^{t-1}_i = A^t_i$:

$$f^{t-1}(A^t_i \setminus \{y^t\}) = f^{t-1}(A^t_i) \subseteq 
f^{t-2}(A^t_i \setminus \{y^t\}) \cup B^{t-1} = f^{t-2}(A^{t-1}_i\setminus \{y^t\}) \cup B^{t-1} \subseteq S^{t-1}_i = S^t_i,$$
where  the first containment follows from the fact that for any  $y \in
\{y^1,\ldots,y^{t-2}\}$ either $f^{t-1}(y)=f^{t-2}(y)$ or $f^{t-1}(y) \in B^{t-1}$, and the second containment follows from the induction hypothesis.

Next suppose that $i>p^{t-1}$.
Then $A^{t-1}_i=\emptyset$.  Hence, $A^t_{p^t}=\{y^t\}$, and
if $i\not=p^t$ then $A^t_i=\emptyset$.  In either case the lemma follows trivially.

\emph{Case $p^{t-1} \ge p^t$}: For $j < p^{t}$, $A^{t}_j = A^{t-1}_j$, for $j=p^t$, $A^t_i = \{y^t\} \cup \bigcup_{j\ge p^t} A^{t-1}_j$,
and for $j>p^t$, $A^t_j=\emptyset$. Again, for all $i\le p^{t}$, $B^{t-1} \subseteq S^{t-1}_{p^{t-1}} \subseteq S^{t-1}_i = S^t_i$.
For all $i<p^{t}$, $y^{t-1} \not\in A^{t-1}_i = A^t_i$ so using the induction hypothesis
$f^{t-1}(A^t_i) \subseteq f^{t-2}(A^t_i) \cup B^{t-1} \subseteq S^{t-1}_i = S^t_i$.
It only remains to consider the case of $i=p^t$ as the claim is trivial for $i>p^t$.
Since $p^{t-1} \ge p^t$, for $i>p^t$, $S^{t-1}_i \subseteq S^{t-1}_{p^t}=S^t_{p^t}$.
Using the induction hypothesis: 
$$f^{t-1}(A^t_{p^t} \setminus \{y^t\}) \subseteq f^{t-1}(\bigcup_{j\ge p^t} A^{t-1}_j)
\subseteq \bigcup_{j\ge p^t} f^{t-2}( A^{t-1}_j \setminus \{y^{t-1}\}) \cup B^{t-1} \subseteq 
\bigcup_{j \ge p^t} S^{t-1}_{j} \cup B^{t-1} \subseteq S^t_{p^t}.$$
\DIFaddbegin \hfill\DIFadd{${}$
}\DIFaddend \end{proof}


Using this lemma, we can replace $\weight^{t-1}(S^t_{p^t})$ by  $|A^t_{p^t}|-1$  in \cref{c-main} to 
obtain the following connection between
the cost of online labeling and prefix bucketing.

\begin{lemma}\label{l-b2l}
Let $\A$ be a lazy online labeling algorithm and let $y^1,\ldots,y^n$ be the sequence produced by
adversary {\bf Adversary($\A,n,m$)}.  
Let the prefix bucketing $a^0,a^1,\dotsc,a^n$ be defined by $a^t_i = |A^t_i|$, for all $t=0,\dotsc,n$ and $i=1,\dotsc,k$
with $k=\lceil \log m \rceil$, as described above.
Then:
$$\chi_\A(y^1,y^2,\dotsc,y^n) \ge \frac{1}{64} \cdot c(a^0,a^1,\dotsc,a^n) - \frac{9}{64} n.$$
\end{lemma}


\subsection{Lower Bound for Bucketing}\label{s-lbb}

In this subsection we derive a lower bound (\cref{l-lbb}) on the cost of any prefix bucketing.
To do so we map any prefix bucketing to a $k$-tuple of ordered rooted trees.
We prove a lower bound on the sum of the depths of the nodes of the trees,
and this will imply a lower bound on the cost of the bucketing.

\paragraph{Ordered trees}
An {\em ordered rooted tree} is a rooted tree where the children of each node are ordered from left to right.
Since these are the only trees we consider, we refer to them simply as trees.
The \emph{$i$-th subtree of $T$} is the tree rooted at the $i$-th child of the root from the left.
If the root has less than $i$ children, we consider the $i$-th subtree to be empty.
The {\em size} of $T$, denoted $|T|$ is the number of nodes.
The {\em depth} of a node is  one more than its distance to the root, e.g., the root has depth 1.
The depth of a tree is the maximum depth of its nodes.
The \emph{cost} of $T$, denoted $\kappa(T)$, is the sum of the depths of its nodes.
The cost and size of an empty tree is 0.


To each prefix bucketing $\bar{a} = a^0,a^1,\dotsc,a^n$ into $k$ buckets, we inductively associate
the following $k$-tuple of trees $T(\bar{a})_1, T(\bar{a})_2, \dotsc, T(\bar{a})_k$.
The trivial bucketing $\bar{a} = a^0$ is mapped to the $k$-tuple of empty trees.
For bucketing $\bar{a} = a^0,a^1,\dotsc,a^n$ with placement sequence $p^1,\ldots,p^n$
let $\bar{a}'$ be the bucketing $a^0,a^1,\dotsc,a^{n-1}$, and assume $T(\bar{a}')$ has been defined.
We define $T(\bar{a})$ by:
\begin{itemize}
\item For $1 \le i < p^n$,  $T(\bar{a})_i = T(\bar{a}')_i$.
\item $T(\bar{a})_{p^n}$ consists of
a root node whose children are the non-empty trees among \\
$T(\bar{a}')_{p^n},T(\bar{a}')_{p^n+1},\dotsc,T(\bar{a}')_k$ 
ordered left to right by  increasing index.
\item $T(\bar{a})_i$ is an empty tree for $p^n < i \le k$.
\end{itemize}


A straightforward induction on $n$ yields:

\begin{proposition}\label{p-ts}
For any positive integer $k$, if $\bar{a} = a^0,a^1,\dotsc,a^n$ is a prefix bucketing into $k$ buckets
then for each $i \in \natInt{1}{k}$, $|T(\bar{a})_i| = a^n_i$.
\end{proposition}

The next lemma relates the cost of bucketing to the cost of its associated trees.
\begin{lemma}\label{l-tc}
For any positive integer $k$, if $\bar{a} = a^0,a^1,\dotsc,a^n$ is a prefix bucketing into $k$ buckets then $c(\bar{a})=\sum_{i=1}^k \kappa(T(\bar{a})_i)$.
\end{lemma}
\iffalse
\begin{proof}
By induction it suffices to show that
$\kappa(T(\bar{a}))-\kappa(T(\bar{a}'))=c(\bar{a})-c(\bar{a}')$, where $\kappa(T(\bar{a})) = \sum_{i=1}^k \kappa(T(\bar{a})_i)$. By definition of $c(\cdot)$ and \cref{p-ts},
$c(\bar{a})-c(\bar{a}')=a^n_{p^n}=|T(\bar{a})_{p^n}|$.
So it suffices to show $\kappa(T(\bar{a}))-\kappa(T(\bar{a}'))=|T(\bar{a})_{p^n}|$.
By the definition of $\kappa(\cdot)$ and the inductive definition of 
$T(\bar{a})$, 
$$\kappa(T(\bar{a}))-\kappa(T(\bar{a}'))= \kappa(T(\bar{a}_{p^n}))-\sum_{i \geq p^n} \kappa(T(\bar{a}'_{i})).$$ 
This equals $|T(\bar{a})|_{p^n}$, as required, since each node
of $\bigcup_{i \geq p^n} T(\bar{a}')_i$ corresponds (bijectively) to a 
non-root of  $T(\bar{a})_{p^n}$ of depth one greater.
\end{proof}
\fi
\begin{proof}
By induction on $n$. For $n=0$, both sides of the equality are 0.  Suppose $n \geq 1$
and assume that the claim is true for $n-1$.  Let $\bar{a}'=a^0,a^1,\dotsc,a^{n-1}$ and $p^n$ be as in the definition of prefix bucketing.
\begin{align*}
c(\bar{a}) = c(\bar{a}') + 1 + \sum_{i=p^n}^k a^{n-1}_i
%%%% b/c space: %%% &= \sum_{i=1}^k \kappa(T(\bar{a}')_i) + 1 + \sum_{i=p^n}^k a^{t-1}_i \cr
&= \sum_{i=1}^k \kappa(T(\bar{a}')_i) + 1 + \sum_{i=p^n}^k |T(\bar{a}')_i| \cr
&= \sum_{i=1}^{p^n-1} \kappa(T(\bar{a})_i) + 1 + \sum_{i=p^n}^k \left(\kappa(T(\bar{a}')_i) + |T(\bar{a}')_i|\right)
\end{align*}
where the second equality uses the induction hypothesis with \cref{p-ts},
and the last equality follows from the definition of $T(\bar{a})_i$ for $i=1,\dotsc,p^n-1$.
For $i\ge p^n$ the depth of each node in $T(\bar{a}')_i$ increases by one when it becomes a child of $T(\bar{a})_{p^n}$, hence
\begin{align*}
\kappa(T(\bar{a})_{p^n}) =  1 + \sum_{i=p^n}^k \left(\kappa(T(\bar{a}')_i) + |T(\bar{a}')_i|\right).
\end{align*}
For $i > p^n$, $\kappa(T(\bar{a})_i) = 0$ so the lemma follows.
\end{proof}

Thus to get a lower bound on the cost of a prefix bucketing 
it suffices to prove a lower bound on the sum of the costs of
the trees that occur in the associated $k$-tuple. 
The following definition will help describe the structure of
trees that occur in such a $k$-tuple.

\begin{definition}[$k$-admissible]
\label{def:k-admissibility}
Let $k$ be a positive integer. 
The empty tree is $k$-admissible. A non-empty tree $T$ is $k$-admissible if its root has at most $k$ children
and the $i$-th  subtree of $T$ is $(k+1-i)$-admissible.
%%%MIKE 5-13-15 omitted ``nonempty'' before ``subtree'' 
\end{definition}

For example, $T$ is 1-admissible if and only if $T$ is empty or a rooted path.
We collect some basic properties of $k$-admissibility.

\begin{proposition}
\label{lm:k-admissibility}
Let $T$ be a (rooted ordered) tree and $k \geq 1$, and suppose $T$ is $k$-admissible. Let $v$ be a leaf of $T$.
\begin{enumerate}
\item If  $k' > k$ then $T$ is $k'$-admissible.
\item If $v$ is deleted from $T$ then the resulting tree  is  $k$-admissible. 
\item If a new node is added as a child of $v$  then the resulting tree is $k$-admissible.
\item  If $T$ has at least two nodes and $k \geq 2$, 
then the tree obtained from $T$ by removing its first subtree is $(k-1)$-admissible.
\end{enumerate}
\end{proposition}
\begin{proof}
The first and last parts are  immediate from the definition of $k$-admissibility.
We prove the other two parts by induction on $|T|$. 
 Let $T'$ be the tree resulting from deleting $v$
and $T''$ be the tree resulting from adding a child to $v$.  If $|T|=1$ then  $T$, $T'$ and $T''$
are $k$-admissible for all $k \geq 1$.
Suppose $|T| > 1$.  Let $v$ belong to the $i$-th subtree $T_i$ of $T$.
By definition of $k$-admissible, $T_i$ is $(k-i+1)$-admissible, and by induction the corresponding
subtree $T''_i$ is also $(k-i+1)$-admissible.  It follows immediately that $T''$ is $k$-admissible.

To show that $T'$ is $k$-admissible, we split into cases according to whether $|T_i|>1$ or $|T_i|=1$.
If $|T_i|>1$ let $T'_i$ be obtained by deleting $v$.   By induction $T'_i$ is still $(k-i+1)$ admissible,
and every other subtree is unchanged so $T'$ is still $k$-admissible.  
If $|T_i|=1$ then $v$ is a child of the root and removing it eliminates the $i$\DIFdelbegin \DIFdel{th }\DIFdelend \DIFaddbegin \DIFadd{-th }\DIFaddend subtree. Thus for any $j \geq i$, 
the $j$-th subtree $T_j$
of $T$ (which must be $(k-j+1)$-admissible) becomes the $(j-1)$-st subtree of $T'$ (and is $(k-(j-1)+1)$-admissible 
by the first part of the proposition) and so $T'$ is $k$-admissible. 
\end{proof}

The connection of admissibility to prefix bucketing is given by the following proposition:

\begin{proposition}\label{p-ta}
For any positive integer $k$, if $\bar{a} = a^0,a^1,\dotsc,a^n$ is a prefix bucketing into $k$ buckets then
for each $i \in \natInt{1}{k}$, $T(\bar{a})_i$ is $(k+1-i)$-admissible.
\end{proposition}

%%%MIKE BEGIN 5-13-15 ADDED PROOF
\begin{proof}
We proceed by induction on $n$.  The result is immediate for $n=0$.   Assume $n\geq 1$
and let $\bar{a}'=a^0,\ldots,a^{n-1}$.  By induction $T(\bar{a}')_i$ is $(k+1-i)$-admissible.
By the definition of $T(\bar{a})$, for $i <p_n$, $T(\bar{a})_i=T(\bar{a}')_i$ and is
 $k+1-i$-admissible by induction.  We need to show that $T(\bar{a})_{p^n}$ is
$(k+1-p^n)$-admissible.   Its subtrees are $T(\bar{a}'_i)$ for $p^n \leq i \leq k$ (some of which may be empty)
so it has at most $k+1-p^n$ children. We also need that its $i$\DIFdelbegin \DIFdel{th }\DIFdelend \DIFaddbegin \DIFadd{-th }\DIFaddend nonempty subtree is $(k+1-p^n + 1 -i)$-admissible. 
Its $i$\DIFdelbegin \DIFdel{th }\DIFdelend \DIFaddbegin \DIFadd{-th }\DIFaddend nonempty subtree is equal to $T(\bar{a}'_j)$ for
some $j \geq p^n+i-1$. By induction the subtree is $(k+1-j)$ admissible and therefore (by the first part of \cref{lm:k-admissibility}) is $(k+1-p^n +1 - i)$-admissible. 
\end{proof}
%%%MIKE END 5-13-15

Let us define $\mu(n,k)$ to be the minimum cost of a $k$-admissible tree of $n$ vertices.

\begin{proposition}
\label{prop:mu}
For any bucketing $\bar{a}$ of $n$ items into $k$ buckets, we have $c(\bar{a}) \geq \mu(n,k) - n+1$.
\end{proposition}


\begin{proof}
Modify $\bar{a}$ to the bucketing $\bar{b}=b^0,\ldots,b^n$ where $b^i=a^i$
for $i<n$ and $b^n=(n,0,\ldots,0)$.  This corresponds to placing the final item in bucket 1.
This can increase the cost by at most $n-1$ so
 $c(\bar{a}) \geq c(\bar{b}) - n+1$.  The first tree $U$ in the  $k$-tuple  $T(\bar{b})$ has  size $n$ and is $k$-admissible
by \cref{p-ta}.  By \cref{l-tc}, $c(\bar{b}) \geq \kappa(U)$ which
is at least $\mu(n,k)$.  
\end{proof}

It remains to give a  lower bound on $\mu(n,k)$.  

\begin{lemma}
\label{lm:k-d-cost}
Let $d$ be a positive integer and $T$ be an arbitrary rooted tree with all leaves of depth at least $d$. Then 
$\kappa(T) \geq \frac{(d + 1)}{2} \cdot |T|$.
\end{lemma}
\begin{proof} %% {\cref{lm:k-d-cost}}
Let $n_i$ denote the number of nodes of $T$ at the depth $i = 1, \dotsc, d - 1$, and let $n_d$ denote the number of  nodes at depth at least $d$. $|T|=n_1+n_2 + \dotsb + n_d$.  By the hypothesis on $T$ (that every leaf has depth at least $d$),
each node of $T$ at depth $i<d$ has at least one child at depth $i+1$.  Therefore $0\leq n_1 \leq n_2 \leq \dotsb \leq n_d$.
Clearly, $c(T) \geq \sum_{i=1}^d in_i$.
Given the constraints, this sum is minimized (over reals) when $n_1=n_2=\dotsb =n_d = |T| / d $. Thus $c(T) \geq \frac{d(d + 1)}{2} \cdot \frac{|T|}{d}$.
\end{proof}

A $k$-admissible tree may have leaves of low depth, so the above lemma is not immediately useful. We need
the following:

\begin{definition}[Balanced tree]
A tree of depth $d$ is balanced if all its leaves are of depth $d$ or $d - 1$.
\end{definition}


\begin{lemma}
\label{lm:balanced}
Let $T$ be a $k$-admissible tree of size $n$ having cost $\mu(n,k)$. Then $T$ is balanced.
\end{lemma}

\begin{proof}  %% {\cref{lm:balanced}}
Let $d$ be the depth of $T$.  Suppose for contradiction that $T$ is unbalanced.
Let $u$ be a leaf  of depth at most $d-2$ and let $v$ be a leaf of depth $d$.
Let $T'$ be the tree obtained by  
removing $v$ and reattaching $v$ as a child of  $u$. Then $\kappa(T') <\kappa(T)$, and  
by the second and third parts of \cref{lm:k-admissibility}, $T'$ is $k$-admissible. This  contradicts the assumption
that $T$ has minimum cost among $k$-admissible trees. 
Thus $T$ must be balanced.
\end{proof}


\begin{lemma}
\label{lm:k-d-size}
Let $k,d \geq 1$. If $T$ is a $k$-admissible tree of depth $d$,
then $|T| \leq \binom{k + d - 1}{k}$.
\end{lemma}
\begin{proof}
If $k=1$ then $T$ must be a rooted path of depth $d$ and clearly $|T|=d\le \binom{d}{1}$. 
For $k\ge 2$, we prove the result by induction on $|T|$.  If $|T|=1$ the result is trivial so
assume $|T| \geq 2$.  Let $L$ be the first subtree of $T$ and let $R$ be the tree created by removing $L$ from $T$. By the definition of $k$-admissibility
$L$ is a $k$-admissible tree of depth at most $d-1$, and by the last part of
\cref{lm:k-admissibility}, $R$ is a $(k-1)$-admissible tree of depth at most $d$.
By the induction hypothesis, $|T| = |L| + |R| \leq \binom{k + d - 2}{k} + \binom{(k-1) + (d - 1)}{k - 1} = \binom{k + d - 1}{k}$.
\end{proof}





\begin{corollary}
\label{cor:mu}
For any $n \geq k \geq 1$, $\mu(n,k)$ is at least $n(w+1)/2$ 
where $w$ is the smallest integer such that $\binom{k+w}{k} \geq n$.
\end{corollary}

\begin{proof}
Let $T$ be a $k$-admissible tree of size $n$ having minimum cost.  Let $d$ be the depth of $T$.
By \cref{lm:k-d-size} we have
$\binom{k+d-1}{k} \geq n$ and so $d-1 \geq w$.
By \cref{lm:balanced}, $T$ is balanced so all leaves are at depth at least $d-1\geq w$ and so
by \cref{lm:k-d-cost}, $\kappa(T) \geq (w+1)n/2$.  
\end{proof}

\begin{lemma} \label{lm:lower_bound_d}
Let $n,k,w$ be integers such that $n\ge 2$ and $k \ge \log n$.  If $\binom{k+w}{k} \geq n$,
then $w \ge \frac{\log n}{ 4 (\log 8k - \log\log n)}$.
\end{lemma}
\begin{proof}
If $w \geq k$ then the conclusion holds, so assume $w \leq k$.
Recall that $\log\binom{r}{s} \leq {H(s/r) r}$
where $H$ stands for the binary entropy function
defined on $[0,1]$ by $H(x) = x \log\frac{1}{x} + (1-x) \log\frac{1}{1-x}$, where the base of the logarithm is 2.
For $x \in (0,1/2]$, $H(x) \le 2x \log\frac{1}{x}$. Therefore:

\begin{align*}
	\log n \leq \log{\binom{k+w}{k}} 
	= \log {\binom{k+w}{w}}
	\leq (k+w)H\left(\frac{w}{k + w}\right)
	\leq 2w\log\left(\frac{k + w}{w}\right)
	\leq 2w\log\left(\frac{2k}{w}\right).
\end{align*}
Defining $a=\frac{1}{2}\log n$, we get $a \leq 2\log(2ak/\log(n))=2(\log(8k)+\log(a/4)-\log\log n)$
and using $\log x<x$ we get $a \leq 2(\log(8k)+a/4 -\log\log n)$ and so $a/2 \leq 2(\log(8k)-\log\log n)$, which
implies the claimed bound on $w$.
\end{proof}


We  now deduce a lower bound on the cost of prefix bucketing:
\begin{lemma}\label{l-lbb}
Let $k,n$ be positive integers such that $k \ge \log n$.
The cost of any prefix bucketing of $n$ items into $k$ buckets is at least $\frac{n \log n}{8 (\log 8k - \log \log n)} - n$.
\end{lemma}

\begin{proof}
By \cref{prop:mu}, the cost of any such bucketing is at least $\mu(n,k)-n$, and by \cref{cor:mu},
this is at least $\frac{nw}{2}-n$ where $w$ is the smallest integer such that $\binom{k+w}{k} \geq n$, which is at
least $\log n/4(\log 8k -\log\log n)$ by \cref{lm:lower_bound_d}.
\end{proof}

\iffalse
\begin{proof}
For $n=1$ the lemma is straightforward so let us assume $n \ge 2$.
Consider a prefix bucketing $\bar{a} = a^0,a^1,\dotsc,a^n$ of $n$ items into $k$ buckets.
Let $b = (n,0,0,\dotsc,0)$ be a $k$-tuple of integers,
and let  $\bar{b} = a^0,a^1,\dotsc,a^{n-1},b$.
Clearly, $\bar{b}$ is also a prefix bucketing and $c(\bar{b}) \le c(\bar{a}) + n-1$. 
Hence, it suffices to show that $c(\bar{b}) > \frac{n \log n}{8 (\log 8k - \log \log n)}$.

Let $T$ be $T(\bar{b})_1$.
By \cref{p-ts}, $|T|=n$, and by \cref{p-ta}, $T$ is $k$-admissible.
Furthermore, by \cref{l-tc}, $c(\bar{b}) = \sum_{i=1}^k c(T(\bar{b})_i) = c(T)$,
so we only need to bound $c(T)$ from below.
From \cref{lm:balanced} it follows that
there is a balanced $k$-admissible tree $T'$ of size $n$ such that $c(T) \geq c(T')$.

Let the depth of $T'$ be $d$.
By \cref{lm:k-d-size}, $T'$ has at most $\binom{k + d-1}{k}$ nodes.
Using \cref{lm:lower_bound_d}, we see that
$d > \frac{\log n}{4 (\log 8k - \log \log n)}$; otherwise, $T'$ could not contain $n$ nodes.
Since $T'$ is balanced, by \cref{lm:k-d-cost}, $c(T') \ge \frac{(d-1)+1}{2} \cdot n = \frac{dn}{2}$.
Thus, $c(T) \ge c(T') > \frac{n \log n}{8 (\log 8k - \log \log n)}$. The lemma follows.
\end{proof}
\fi


\subsection{Proof of \texorpdfstring{\cref{l-main}}{}}\label{ss-main}
Finally we return to the online labeling problem and prove the main lemma:
Fix a lazy online labeling algorithm $\A$ and let $n \leq m$ be positive integers as in the hypothesis.
\Cref{l-b2l} implies that for the adversary sequence $y^1,\ldots,y^n$ constructed from $\A$, 
the cost $\chi_{\A}(y^1,\ldots,y^n)$ incurred by the algorithm is at least
$ \frac{1}{64} M(n,k) - \frac{9}{64} n$, where $M(n,k)$ is the minimum cost of
any prefix bucketing of $n$ items into  $k=\lceil \log m \rceil$ buckets.  
Substituting the lower bound on $M(n,k)$ given by the previous lemma, we get that
the worst case cost of $\A$ satisfies:
$$\chi_\A(y^1,y^2,\dotsc,y^n) \ge \frac{1}{512} \cdot \frac{n \log n}{ 3 + \log \lceil \log m \rceil - \log \log n} - \frac{n}{6}.$$

\section{An Upper Bound for large array size}
\label{s-ub}

The lower bound of \cref{thm:main} generalizes the $\Omega(n \log n)$ lower bound 
for $m=n^{\Theta(1)}$ of~\cite{DSZ04} to  $\Omega(n\log n \frac{1}{\log\log m -\log\log n})$  for  $m$ up through $m<2^n$.   
When $m=n^{\Theta(1)}$ the upper and lower bounds match at $\Theta(n \log n)$.  On the other hand, when
$m \geq 2^n$,  $n$ items can be inserted without ever moving a placed item (and thus the cost is $n$):
when the $j$\DIFdelbegin \DIFdel{th }\DIFdelend \DIFaddbegin \DIFadd{-th }\DIFaddend item arrives, 
store it in location $L+2^{n-j}$ where $L$ is the location of the largest item less than it
(or is 0 if there is no such item.)

So the lower bound of \cref{thm:main} is tight at the extremes $m=n^{\Theta(1)}$ and $m=2^n$.  What happens for intermediate values of $m$?
In this section we show that the  lower bound of \cref{thm:main} is tight for all $m$ at least some quasi-polynomial function of $n$.  

For a real number $m \geq 2$ and for positive integer $k$, define:

\[
r(m) = \left\lceil \sqrt{\log(m/2)}  \right\rceil \text{ and } 
b_k(m)  =  \binom{r(m)}{k}.
\]


Let $n_k(m)$ be the maximum number of items that can be inserted into an array of size $m$ such that the cost does not exceed $kn$. We prove:

\begin{theorem}
\label{thm:ub}
For positive integers $m$ and $k$, we have $n_k(m) \geq b_k(m)$. 
\end{theorem}

As an immediate consequence of \cref{thm:main,thm:ub} we get:

\begin{corollary}
For $m$ and $n$ satisfying $\log \log (m/2) \ge 12$, $\log n \ge 6$ and $2^{1+\log^3 n} < m \leq 2^n$, $\chi_m(n) = \Theta(n \log n \frac{1}{\log\log m})$.
\end{corollary}

\begin{proof}
The lower bound follows from \cref{thm:main} using $\log\log m - \log\log n \geq \frac{2}{3} \log\log m$
for   $m\geq 2^{\log^3 n}$.
For the upper bound, given $n$ and $m \geq 2^{1+(\log^3 n)}$, let $k= 6\lceil \log n/\log\log (\frac{m}{2}) \rceil$.
Then $n \leq (\log (m/2))^{k/6}$. Since $\binom{a}{k} \geq (a/k)^k$ for $a\ge k$,   
and $k \leq r(m)^{2/3}$, we get
$b_k(m) \geq (\frac{r(m)}{k})^k \geq r(m)^{k/3} \geq (\log (m/2))^{k/6}$. Thus $b_k(m) \ge n$ and so
$\chi_m(n) \leq  k n = 6 n \lceil \log n/\log\log (\frac{m}{2}) \rceil $. 
\end{proof}

The exponent 3 in the hypothesis $m\geq 2^{1+\log^3 n}$ is not crucial; a more careful argument can reduce it to any
number greater than 2, and perhaps even further.

The proof of \cref{thm:ub} will need a few elementary facts about $r(m)$ and $b_k(m)$.

\begin{proposition}
\label{prop:technical}
\begin{enumerate}
\item For $m \geq 2$ and integer $j \in [2,r(m)/2]$, $b_j(m) \leq 2^{2r(m)-3}-1$.
\item $r(\frac{m}{w}) \geq r(m)-1$ provided that $m \geq 2$ and $1 \leq w \leq 2^{2r(m)-3}$.
\end{enumerate}
\end{proposition}
\begin{proof}
For the first part, $2 \leq j \leq r(m)/2$ implies $r(m) \geq 4$ and thus
$r(m) \leq 2r(m)-4$ and so $b_j(m) \leq \binom{r(m)}{j} \leq 2^{r(m)} \leq 2^{2r(m)-4} \leq 2^{2r(m)-3}-1$.

For the second part,  we want $\sqrt{\log (m/2) - \log(w)} \geq \sqrt{\log (m/2)} -1$.  Squaring both sides, it suffices to show $\log(w) \leq 2\sqrt{\log(m/2)}-1$,
which is true  by the assumption $\log w \leq 2r(m)-3 \leq 2\sqrt{\log (m/2)}-1$.
\end{proof}

\begin{proofof}{\Cref{thm:ub}}
We proceed by induction on $k$. We assume (without loss of generality) that throughout the algorithm
cells 1 and $m$ are occupied by items $y_{\min}$ and $y_{\max}$ which are, respectively, lower and upper bounds
on all items.   

At any time certain array cells are occupied.   A segment $S$ of cells whose only occupied cells are the leftmost and rightmost cell is an {\em open segment};
the items stored in these cells
are denoted $y_L(S)$ and $y_R(S)$. The initial open segment has size $m$.  
An open segment $S$ is {\em usable} if $|S| \geq 3$ (so it has at least one unoccupied cell).
For an item $y$ not stored in the array, there is a unique open segment $S$ such that
$y_L(S)<y<y_R(S)$; we say that $S$ is  {\em compatible with $y$}.
Storing $y$ in an unoccupied cell  $c \in S$ splits $S$   into two open
segments that overlap at $c$ and
and $c$ can be chosen to be a {\em middle cell} for which both new segments have size at least $|S|/2$. 

\iffalse
 More generally, it can be checked that given $q-1$ items to be placed in an open
interval $S$ that has at least $q-1$ unoccupied spaces we can place them evenly so that each of the $q$
open segments produced has size at least $|S|/q$.  (The worst case is $|S|=aq+1)$ for some integer $a$, and
in this case each of the $q$ resulting subsegments has length $a+1 \geq |S|/q$.)
\fi


For each $k \geq 1$ we define algorithm $A_k$ whose cost per item
inserted is at most $k$.  We'll show that $A_k$ can insert $b_k(m)$ items when
run on an array of size $m$\DIFaddbegin \DIFadd{.
}\DIFaddend 

\DIFdelbegin \DIFdel{Algorithm $A_1$ is: }\DIFdelend \DIFaddbegin \paragraph{\DIFadd{Algorithm $A_1$}} 
\DIFaddend While  no two adjacent cells are occupied, insert the next item $y$ into the middle cell of the
open segment $S$ compatible with $y$.
A simple induction shows that after inserting $t$ items, each open segment has length at least $m/2^t$. Since $A_1$ can continue as long as this exceeds 2, it can do at least $\lceil \log m/2 \rceil  \geq b_k(m)=\lceil \sqrt{\log (m/2)} \rceil $ insertions.

\DIFaddbegin \paragraph{\DIFadd{Algorithm $A_k$}}
\DIFaddend For $k>1$ we define $A_k$.  If  $m=2$\DIFaddbegin \DIFadd{, }\DIFaddend $b_k(2)=0$ items are inserted,
so assume $m \geq 3$.  If $k > r(m)/2$ then let $k'=\lfloor r(m)/2 \rfloor$ and run $A_{k'}$.
By induction, $b_{k'}(m) \geq b_k(m)$ items can be inserted.
For $m \geq 3$, and $2 \leq k < r(m)/2$, $A_k$ consists of two \DIFaddbegin \DIFadd{main }\DIFaddend {\em phases} \DIFdelbegin \DIFdel{.  In phase 1,  }\DIFdelend \DIFaddbegin \DIFadd{using the algorithm }\DIFaddend $A_{k-1}$ \DIFdelbegin \DIFdel{is
used to insert }\DIFdelend \DIFaddbegin \DIFadd{on the whole array and algorithms $A_{k}$ on smaller arrays.
}\begin{itemize}
	\item {\bf \DIFadd{Phase 1.}} \DIFadd{Insert }\DIFaddend $b_{k-1}(m)$ items \DIFdelbegin \DIFdel{.
We then redistribute }\DIFdelend \DIFaddbegin \DIFadd{using algorithm $A_{k-1}$.
	}\item {\bf \DIFadd{Redistribution step.}} \DIFadd{Redistribute }\DIFaddend the items evenly so that each of the $w=1+b_{k-1}(m)$ segments has size at least $\lceil m/w \rceil$. 
	\DIFdelbegin \DIFdel{We call the segments defined by these items the }\DIFdelend \DIFaddbegin \item \DIFaddend {\DIFdelbegin %DIFDELCMD < \em %%%
\DIFdel{phase 1 segments}\DIFdelend \DIFaddbegin \bf \DIFadd{Phase 2.}\DIFaddend } \DIFdelbegin \DIFdel{.   
The total cost per item inserted in phase 1  is at most $k$.
In phase 2, items }\DIFdelend \DIFaddbegin \DIFadd{Items }\DIFaddend placed during phase 1 do not move. We run $w$ separate instances of $A_k$, one for each  phase 1 segment. Each arriving item is assigned to the phase 1 segment that is compatible with it,
		and is inserted into that segment using $A_k$. 
\DIFdelbegin \DIFdel{Each }\DIFdelend \DIFaddbegin \end{itemize}
\DIFadd{The total cost per item inserted in phase 1 is at most $k$.
We call the segments defined by the items after the redistribution step the }{\em \DIFadd{phase 1 segments}}\DIFadd{.   
Each phase 1 }\DIFaddend segment has length at least $\lceil m/w \rceil $ and so we can handle $b_k(m/w)$ insertions 
even if all items in phase 2 are assigned to the same phase 1 segment. The total
cost per item in phase 2 is at most $k$.

Thus we can handle $b_{k-1}(m)+b_k(m/w)$ insertions in the two phases, where $w=1+b_{k-1}(m)$.  
By part 1 of \cref{prop:technical}, $w \leq 2^{2r(m)-3}$ and by part 2,
$r(m/w) \geq r(m)-1$ and so $b_k(m/w) \geq \binom{r(m)-1}{k}$.
Thus  at least $\binom{r(m)}{k-1}+\binom{r(m)-1}{k} \geq \binom{r(m)}{k}=b_k(m)$
items are inserted.
\end{proofof}

\iffalse
BEGIN 7-14-15 VERSION
\begin{theorem}
\label{thm:ub}
Let $m,n,k$ be positive integers such that:
\begin{enumerate}
\item $m>2^{16}$
\item $k\leq 1/2\sqrt{\log m/\log \log m}$.  
\item  $n \leq (\log m)^{k/3}$.
\end{enumerate}
Then $\chi_m(n) \leq (2k-1)n$, i.e., there is an algorithm that loads $n$ items into an array of size $m$ with amortized cost of $2k-1$ per item. 
\end{theorem}                  

As an immediate consequence of \cref{thm:main,thm:ub} we get:

\begin{corollary}
For $m$ and $n$ satisfying $n^{\log n}\leq m \leq 2^n$, $\chi_m(n) = \Theta(n \log n \frac{1}{\log\log m})$.
\end{corollary}

\begin{proof}
The lower bound follows from \cref{thm:main} and the fact that for $m \geq n^{\log n}$, $\log\log m - \log\log n \geq \frac{1}{2} \log\log m$.

For the upper bound, let
$n$ be sufficiently large and $m \geq n^{\log n}$.  Then $\log n \leq (\log m)^{1/2}$ and so $n = (\log m)^{k/3}$ where $k =\frac{3\log n}{\log\log m} $.
Then $m$, $n$, $k$ satisfy the hypotheses of \cref{thm:ub}, and so $\chi_m(n) \leq (2k-1) n = O(n \log n \frac{1}{\log\log m})$. 
\end{proof}



\begin{proofof}{\Cref{thm:ub}}
We proceed by induction on $k$. To simplify the description we assume (without loss of generality)
cells 1 and $m$ are initially loaded with items $y_{\min}$ and $y_{\max}$ which are, respectively, lower and upper bounds
on all items. Set $Y^0 = \{y_{\min}, y_{\max}\}$.



At any time the array has certain occupied cells.   A segment of cells whose leftmost and rightmost cells are occupied
and all others are unoccupied is called an {\em open segment}; the items in the leftmost and rightmost cells of the open segment
$S$ are denoted $y_L(S)$ and $y_R(S)$ (we include the occupied end cells in the open segment for convenience
in some calculations).  The initial open segment has size $m$.
The segment is said to be {\em usable} if $|S| \geq 3$ (which means there is at least one unoccupied cell).
For any new item $y$ not stored in the array
there is a unique open segment $S$ such that $y_L(S)<y<y_R(S)$; we say that $S$ is  {\em compatible with $y$}.
If item $y$ is assigned to an unoccupied cell in $S$  then the open segment $S$ is split into two open
segments which overlap at the cell containing $y$; the sum of the sizes of these two segments is $|S|+1$.
A {\em middle cell} of $S$ is a cell such that the two segments obtained from $S$ each have size at least $|S|/2$. It's easy
to check that every usable segment has a middle cell.  More generally, it can be checked that given $q-1$ items to be placed in an open
interval $S$ that has at least $q-1$ unoccupied spaces we can place them evenly so that each of the $q$
open segments produced has size at least $|S|/q$.  (The worst case is $|S|=aq+1)$ for some integer $a$, and
in this case each of the $q$ resulting subsegments has length $a+1 \geq |S|/q$.)



We will define algorithms $A_k$ for $k \geq 1$.  It will be obvious from the definitions that the cost per item
loaded is at most $2k-1$.  The main technical question will be how many items $A_k$ can handle.  Let us define $n_k(m)$ to be the
maximum number of items that $A_k$ can handle in an interval of size at least $m$ (the argument $m$ need not be an integer).  
Our goal is to show that $n_k(m) \geq \lfloor (\log m)^{k/3} \rfloor$.



We now define algorithm $A_1$ for $k=1$.   For each successive item $y^t$ ($t \geq 1$), we identify the
open segment $S$ compatible with $y^t$.  If it is usable we store $y^t$ in the middle cell of the segment. 


Let us analyze this algorithm. We never move any loaded item, so the cost per loaded item  is 1.   We want to lower bound the number
of items that can be loaded.  The size of the initial open segment is $m$, so after loading $t-1$ items every open segment
has size at least $m/2^{t-1}$ and we can handle $y^t$ if this is at least 3.  Thus we can load $t$ items provided that
$t-1 \leq \log_2(m/3)+1$. So $n_1(m) \geq \log_2(m/3)+1$, which is at least $\log_2(m)^{1/3}$ for $m$ large enough.



For $k \geq 2$ we define $A_k$, which makes use of $A_{k-1}$.  We initially load $q=\lfloor (\log m)^{(k-1)/3} \rfloor$ 
items using algorithm $A_{k-1}$, which by induction can be done at amortized cost
of $2k-3$ moves per item.  We then move all of the items so that they are as evenly spaced as possible along the array which
increases the amortized cost per item to $2k-2$.

Each of the resulting open segments has size at least $m/(q+1)$.   Let us define
$s_j$ for $j \geq 0$ to be $2^{-j}m/(q+1)$.



Next the algorithm works in \emph{rounds}.  Let $\old^R$ denote the set of items loaded prior to round $R$.
The algorithm will ensure that the open segments defined by the locations of $\old^R$ at the beginning of the phase
all have size at least $s_{R-1}$.  We have already seen that this holds when $R=1$.  

Round $R$ will consist of two phases. During the first
phase we will load  $n_{k-1}(s_{R-1})$ new items without moving any items in $\old^R$.
During the second phase we move items (both old and new) to ensure that all open segments have size
at least $s_R=s_{R-1}/2$.



During the first phase we refer to the open segments defined by the storage function at the beginning of the phase
as {\em working segments}.  We will run $A_{k-1}$ independently on each working segment.  When an item arrives
we assign it to the working segment it is compatible with, and load it into the working segment using $A_{k-1}$.
Since each working segment has length at least $s^{R-1}$ and we load at most $n_{k-1}(s_{R-1})$
items in all of the segments we are guaranteed that each of the independent copies of $A_{k-1}$ successfully
load all of their assigned items at amortized cost of $2k-3$.  



For the second phase we need to rearrange the elements to guarantee that the lower bound on working
segment length decreases by at most a factor of 2.  Classify items as old or new depending on whether
they were added in round $R$.  Say that a segment $S$ is {\em useful} if its first and last cells contain
old items (useful segments are unions of one or more consecutive working segments), and define
the excess of a useful segment to be the number of old items in it minus the number of new items.

We claim there is a collection of disjoint segments each having excess exactly one that cover all of the new items.

To prove the claim,  first note that the entire array is a useful segment with positive excess.  
Now choose a collection $\mathcal{S}$ of disjoint useful segments that together cover all new items, such that $\mathcal{S}$
is as large as possible, and subject to this, the sum of the sizes of segments in $\mathcal{S}$ is as small as possible.

We claim that each $S \in \mathcal{S}$ has excess exactly one.   Consider such an $S$ and suppose that
its excess $m$ is greater than 1.  Let $j_0,j_1,\ldots,j_t$ be the index of the cells of $S$
that store old elements.  If there are no new elements between $j_0$ and $j_1$ then we can shrink $S$
to start at $j_1$ contradicting the minimality of $\sum_{T \in \mathcal{S}} |T|$.

Thus the segment $[j_0,j_1]$ has excess at most 1.  Let $j_r$ be the largest index such that $[j_0,j_r]$
has excess at most 1.   Then $[j_0,j_{r+1}]$ has excess greater than 1, which
implies that $[j_0,j_r]$ has excess exactly 1 and there are no new items in
$[j_r,j_{r+1}]$.   But then we can split
$S$ into $[j_0,j_r]$ and $[j_{r+1},j_t]$ each of which has positive excess, and this
contradicts the maximality of $\mathcal{S}$.  So $\mathcal{S}$ has the desired properties.


Now within each segment $S \in \mathcal{S}$ we redistribute the items (both old and new) that are internal to $S$ uniformly
within $S$.  If $S$ had $u-1$ internal old items then it was originally split into $u$
working segments each of size at least $s_{R-1}$.  

We now have $2u-1$ internal items which will split $S$ into $2u$ working segments (that overlap at their endpoints)
in the next round and after redistributing them all of them will have size at least $s_{R-1}/2=s_R$.



The total work done to accomplish phase 2 is $2u-1$ which is less than twice the number of new items in the segment.
Summing over all segments in $\mathcal{S}$ gives an additional amortized cost of 2 per item.
Thus the total amortized cost of $A_k$ is at most $2k-1$ per item.



It remains to bound the number of items that can be handled by $A_k$.



Let $r$ denotes the number of rounds. We have  $s_1 =m/q  
> \frac{m}{(\log m)^{(k-1)/3} }$. Then the number of items inserted during all rounds is

$$\sum_{i=1}^r \log_2({s_i})^{(k-1)/3} \ge 
\sum_{i=1}^r \log_2\left(\frac{s_1}{2^i}\right)^{(k-1)/3} = $$ 

$$ =\sum_{i=1}^r (\log_2{s_1} - i)^{(k-1)/3} 
\ge r (\log_2{s_1} - r)^{(k-1)/3}$$



Let us chose $r=\sqrt{\log m}$ and we obtain

$$ \sqrt{\log_2 m}\left(\log_2{\frac{m}{3\log_2(m)^{(k-1)/3} }} - \sqrt{\log_2 m}\right)^{(k-1)/3}=$$ 

$$\log_2 (m)^{(2k+1)/6}\left(1 -  \frac{\sqrt{\log_2 m}}{\log_2 m}- \frac{k-1}{3}\frac{\log_2 \log_2 m}{\log_2 m}\right)^{(k-1)/3}$$

                                                                                              

Using the fact that $k<1/2\sqrt{\log_2 m/\log_2 \log_2 m}$ we obtain the lower bound for this expression

$$\left(\log_2(m)^{(2k+1)/6} \right)\left(1 - \sqrt{\frac{\log_2 \log_2 m}{\log_2 m}}\right)^{\frac{1}{6}\cdot 
\sqrt{\frac{\log_2 m}{\log_2 \log_2 m}}}\ge $$

$$\left(\log_2(m)^{(2k+1)/6}\right)\left(\frac{1}{2e}\right)^{\frac{1}{6}}$$

which is for $m>2^{16}$ greater than $\log_2( m)^{2k/6}$. Therefore during all rounds of $A_k$, $\log_2(m)^{k/3}$ items are loaded with an amortized cost $2k-1$ per insertion.
END 7-14-15 VERSION
\end{proofof}
\fi

\bibliographystyle{siamplain}
\bibliography{lab290117}

\end{document}
