% !TEX TS-program = pdflatex
% !TEX encoding = UTF-8 Unicode

% This is a simple template for a LaTeX document using the "article" class.
% See "book", "report", "letter" for other types of document.

\documentclass[11pt]{article} % use larger type; default would be 10pt

\usepackage[utf8]{inputenc} % set input encoding (not needed with XeLaTeX)

%%% Examples of Article customizations
% These packages are optional, depending whether you want the features they provide.
% See the LaTeX Companion or other references for full information.

%%% PAGE DIMENSIONS
\usepackage{geometry} % to change the page dimensions
\geometry{a4paper} % or letterpaper (US) or a5paper or....
% \geometry{margin=2in} % for example, change the margins to 2 inches all round
% \geometry{landscape} % set up the page for landscape
%   read geometry.pdf for detailed page layout information

\usepackage{graphicx} % support the \includegraphics command and options

% \usepackage[parfill]{parskip} % Activate to begin paragraphs with an empty line rather than an indent

%%% PACKAGES
\usepackage{booktabs} % for much better looking tables
\usepackage{array} % for better arrays (eg matrices) in maths
\usepackage{paralist} % very flexible & customisable lists (eg. enumerate/itemize, etc.)
\usepackage{verbatim} % adds environment for commenting out blocks of text & for better verbatim
\usepackage{subfig} % make it possible to include more than one captioned figure/table in a single float
% These packages are all incorporated in the memoir class to one degree or another...


%%% SECTION TITLE APPEARANCE
\usepackage{sectsty}
\allsectionsfont{\sffamily\mdseries\upshape} % (See the fntguide.pdf for font help)
% (This matches ConTeXt defaults)

%%% ToC (table of contents) APPEARANCE
\usepackage[nottoc,notlof,notlot]{tocbibind} % Put the bibliography in the ToC
\usepackage[titles,subfigure]{tocloft} % Alter the style of the Table of Contents
\renewcommand{\cftsecfont}{\rmfamily\mdseries\upshape}
\renewcommand{\cftsecpagefont}{\rmfamily\mdseries\upshape} % No bold!

% Labeling density
\newcommand{\density}[2]{\rho(#1, #2)}
\newcommand{\rdensity}[2]{\delta(#1, #2)}

\usepackage{amsmath}
\usepackage{amsthm}

\newtheorem{definition}{Definition} % [chapter]
\newtheorem{theorem}[definition]{Theorem}
\newtheorem{lemma}[definition]{Lemma}
\newtheorem{remark}[definition]{Remark}
\newtheorem{corollary}[definition]{Corollary}
\newtheorem{example}[definition]{Example}
\newtheorem{claim}[definition]{Claim}
\newtheorem{statement}[definition]{Statement}
%%% END Article customizations

\begin{document}

\section{Labeling}

Label space is between $\Omega(n)$ and $n^{1 + o(1)}$.
We prove only a lower bound for \emph{smooth algorithms}.
The size of the label space is denoted by $m$ and the number of elements is $n$.
We later prove the following lower bound.
\begin{theorem}
\label{thm:lower-bound}
The cost of the labeling for $n$ elements into a labeling space with size $m$, $m \geq 2n$, and $m \in o\left(n^{1 + c}\right)$ for each $c > 0$ is in
\[
\Omega\left(\frac{n(\log m)(\log n)}{\log m - \log n}\right).
\]
\end{theorem}

\subsection{Adversary and algorithm}
Our adversary always inserts an element to be the leftmost one of the already inserted elements.
\begin{definition}[Smoothness]
The algorithm is considered to be \emph{smooth} if it is normalized and after a rearrange of the label space $[a, b]$ the elements of this space are distributed uniformly (the rounding is performed so that the elements are placed more to the left).
\end{definition}

\subsection{Definitions and notation}
The segment $S_i$ is the interval $[1, 2^i]$.
We define the density of a set $S$ at the time $t$, $\density{S}{t}$, as the number of elements with labels in $S$ at the time $t$ divided by $|S|$.
We define the density of the segment $S_i$ as $\density{i}{t}$.
We define $L = \log m$.

\begin{lemma}
\label{lm:descreasing-density}
At each time $t$ it holds that $\density{1}{t} \geq \density{2}{t} \geq \dots \geq \density{L}{t}$.
\end{lemma}
\begin{proof}
Trivial.
\end{proof}

\begin{lemma}
Whenever there is a rearrange, which touches the segment $S_i$, we may assume that the whole segment $S_i$ was rearranged.
\end{lemma}
\begin{proof}
From Lemma~\ref{lm:descreasing-density} it follows that this assumption may only multiply the complexity of the original algorirhtm by at most $2$.
\end{proof}

We define the \emph{critical level} of the mentioned rearrange to be $i$.

\begin{lemma}
After a rearrange with the crtical level $i$ it holds that $\density{1}{t} = \dots = \density{i}{t}$.\footnote{We do not consider possible roundings.}
\end{lemma}
\begin{proof}
Follows from the smoothness property.
\end{proof}

Now we are ready to describe the charging scheme.
Each element is assigned a level $i$. 
A newly inserted element is assigned the level $1$.
All the elements in $S_i$ on the levels less or equal $i$ are reassigned the level $i$ when a rearrange with the crtical level $i$ occured.
Element may be charged only when its level is increased.

\begin{lemma}
Consider a rearrange at the time $t$ with the critical level $i$.
Let $t'$ be the last time when a rearrange with the critical level $i$ occured such that $t' \leq t$.
Then the elements in $S_i$ which are newly inserted (between $t'$ and $t$) are in $S_{i - 1}$.
\end{lemma}
\begin{proof}
Trivial.
\end{proof}

When a rearrange with the critical level $i$ occured at the time $t$ we charge only the most recently inserted half of the elements in $S_{i}$ which resides in $S_{i - 1}$.
These elements are charged the overall work done in the right part of the segment, i.e. $S_{i} - S_{i - 1}$.
The density of the right half of the segment $S_i$ is denoted by $\rdensity{i}{t} = \density{S_i - S_{i - 1}}{t}$.
The work of the charged element $x_t$ for a rearrange with the critical level $i$, denoted by $p(x_t, i)$, is as follows.
\[
p(x_t, i) = \frac{\rdensity{i}{t} l_{i - 1}}{(\density{i - 1}{t} - \rdensity{i}{t})l_{i - 1}} = \frac{\rdensity{i}{t}}{\density{i - 1}{t} - \rdensity{i}{t}}.
\]
The charge is called -- charge of the element at the level $i$.

Because the newly inserted elements in $S_{i - 1}$ pay for the elements in $S_i - S_{i - 1}$ each element adds work to the charged price at most once.
Therefore we are able to charge the whole rearrange cost when summing over all the touched levels.

\begin{lemma}
\label{lm:rdensity-density}
Let $x_t$ be an element inserted at the time $t$.
Let $t' \geq t$ be the first time when a rearrange with the critical level $i$ occured.
Assume that the rearrange charged the element $x_t$.
It holds that $\density{i - 1}{t} - \rdensity{i}{t} \geq \frac{\density{i - 1}{t'} - \rdensity{i}{t'}}{2}$ and $\rdensity{i}{t} = 2\density{i}{t'} - \density{i - 1}{t'}$.
\end{lemma}
\begin{proof}
Since we charge only the last half of the inserted elements and thus the first part follows.
We also have that $\rdensity{i}{t} = \frac{2\density{i}{t} - \density{i - 1}{t}}{l_{i - 1}} = \frac{2\density{i}{t'} - \density{i - 1}{t'}}{l_{i - 1}}$.
\end{proof}

\begin{corollary}
The work of the charged element $x_t$ during a rearrange with the critical level $i$ at the time $t'$ equals
\[
p = \frac{\density{i}{t}}{4(\density{i - 1}{t} - \density{i}{t})} - \frac{1}{4}.
\]
\end{corollary}
\begin{proof}
From Lemma~\ref{lm:rdensity-density} it follows that
\[
\begin{split}
p 	
	& = \frac{\rdensity{i}{t'}}{\density{i - 1}{t'} - \rdensity{i}{t'}} \\
	& = \frac{2(\density{i}{t} - \density{i - 1}{t})}{2(\density{i - 1}{t} - 2\density{i}{t} + \density{i - 1}{t})} \\
	& = \frac{\density{i}{t}}{4(\density{i - 1}{t} - \density{i}{t})} - \frac{\density{i - 1}{t} - \density{i}{t}}{4(\density{i - 1}{t} - \density{i}{t})}.
\end{split}
\]
\end{proof}

\begin{lemma}
For an element $x_t$ it holds that
\[
\sum_{ i =1}^{L} \frac{\density{i}{t}}{\density{i - 1}{t} - \density{i}{t}} \geq \frac{L^2}{\ln m - \ln n} - L. \footnote{We put $\density{0}{t} := 1$.}
\]
If $\density{i - 1}{t} = \density{i}{t}$, then $\frac{\density{i}{t}}{\density{i - 1}{t} - \density{i}{t}} = \infty$.
\end{lemma}
\begin{proof}
For the sake of the proof we define $r_i := \frac{\density{i}{t}}{\density{i - 1}{t}}$.
Now realize the fact that $\density{L}{t} = \density{0}{t} \cdot \prod_{i = 1}^{L} r_i$.

The following computation gives the wanted result.
\[
\begin{split}
\sum_{i =1}^{L} \frac{\density{i}{t}}{\density{i - 1}{t} - \density{i}{t}} 
	& = \sum_{i = 1}^L \frac{\density{i - 1}{t}r_i}{\density{i - 1}{t} - \density{i - 1}{t}r_i} \\
	& = \sum_{i = 1}^L \frac{r_i}{1 - r_i} = \left(\sum_{i = 1}^L \frac{1}{1 - r_i}\right) - L. \\
\end{split}
\]

Now we have the constraints $r_i > 0$ and $\prod_{i = 1}^L r_i = \density{L}{t}$.
The previous sum is minimal when $r_1 = \dots = r_L = \density{L}{t}^{1 / L}$.

Now we compute
\[
\sum_{i = 1}^L \frac{1}{1 - r_i} \geq \frac{L}{1 - \density{L}{t}^{1 / L}} = \frac{L}{1 - e^{\ln(\density{L}{t})/L}} = \frac{L^2}{-\ln \density{L}{t}} = \frac{L^2}{\ln m - \ln n}.
\]

The last inequalit follows from the fact that $\density{L}{t} \leq \frac{n}{2^L}$.
\end{proof}

\begin{corollary}
For $n$ large enough it holds that
\[
\frac{L^2}{L - \ln n} - 2L \in \Omega\left(\frac{L \cdot \log n}{L - \log n}\right).
\]
\end{corollary}
\begin{proof}
For the asymptotic result we use the upper bound on $m \in n^{1 + o(1)}$, i.e. $L \in \log n + o(\log n)$.
\[
\frac{L^2}{\ln m - \ln n} - 2L = \frac{L^2 - 2L(\ln m - \ln n)}{\ln m - \ln n} = \frac{L(L - 2\ln m + 2\ln n)}{\ln m - \ln n}.
\]
The corollary holds for $n$ large enough because $\log m - 2\ln m + 2 \ln n \geq 3\ln n - 2\ln m = 3\ln n - 2(1 + o(1)) \ln n \in \Omega(\log n)$.
\end{proof}

\begin{corollary}
\label{cor:element-charge}
Let $x_t$ be an element and $\mathcal{L}$, $\mathcal{L} \subseteq \{1, \dots, L\}$, be the set of levels where $x_t$ was charged.
If $|\mathcal{L}| \geq L/32$, then the price charged to $x_t$ is at least $\Omega\left(\frac{|\mathcal{L}| \log n}{|\mathcal{L}| - \log n}\right)$.
\end{corollary}
\begin{proof}
The approach is to redefine the levels so that we can reuse the lemma.
We omit the gaps and instead of using $\density{i - 1}{t} - \density{i}{t}$ we use $\density{i'}{t} - \density{i}{t}$ where $i'$ is the greatest integer from $\mathcal{L}$ such that $i' < i$.
The result now follows from the fact that $\density{i'}{t} - \density{i}{t} \geq \density{i - 1}{t} - \density{i}{t}$.
Also the asymptotics is fine -- review the proof of the previous corollary.
\end{proof}

\begin{proof}[Proof of Theorem~\ref{thm:lower-bound}]
To prove the theorem we compute onle the cost of relabelings when $t \geq n/2$ and $t \leq n$ and in the levels from $L/8$ to $L/4$.

Because $m \leq n^2$ for $n$ large enough it is true that $l_{L/4} \leq 2^{\log m / 4} \leq \sqrt{n} \leq \frac{n}{16}$.

Now we compute a lower bound on the number of all charges during the run of the labeling algorithm from the time $n/2$ to $n$.
From now on we count only with the last $n/2$ elements.
It is clear that in the end there are at most $l_i$ elements with the level $i$.
So there are at least $n/2 - l_i$ elements with the level at least $i$.
When an rearrange on the level $i$ occured there are charges on the levels $1, \dots, i$.
But at least half of the elements which crossed the level $i'$, $1 \leq i' \leq i$ was charged at the level $i'$.
Therefore we can conclude that at least half of the elements which were moved outside $S_i$ were charged at the level $i$.
From the previous it follows that from time $n/2$ to $n$ there must be at least $\frac{1}{2} \cdot \sum_{i = L/8}^{L/4}(n/2 - l_i)$ charges of the elements $x_{n/2}, \dots, x_n$ at the levels from $L/8$ to $L/4$.
Now we compute that
\[
\frac{1}{2} \cdot \sum_{i = L/8}^{L/4}(n/2 - l_i) = \frac{Ln}{32} (n - o(n)) \geq \frac{Ln}{64}.
\]
So there are at least $Ln/64$ charges during the run of the algorithm.

Each element can be charged on at most $L/8$ levels.
At least $n/64$ elements must be charged on $L/32$ levels.
If not, then there would be at most $n/64 \cdot L/8 + 63/64nL/64$ relabelings which is contradictory.

These elements add at least $\Omega\left(\frac{n(\log m)(\log n)}{\log m - \log n}\right)$ to the overall cost by Corollary~\ref{cor:element-charge}.
\end{proof}
\end{document}
