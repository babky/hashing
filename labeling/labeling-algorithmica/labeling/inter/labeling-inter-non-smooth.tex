% !TEX TS-program = pdflatex
% !TEX encoding = UTF-8 Unicode

% This is a simple template for a LaTeX document using the "article" class.
% See "book", "report", "letter" for other types of document.

\documentclass[11pt]{article} % use larger type; default would be 10pt

\usepackage[utf8]{inputenc} % set input encoding (not needed with XeLaTeX)

%%% Examples of Article customizations
% These packages are optional, depending whether you want the features they provide.
% See the LaTeX Companion or other references for full information.

%%% PAGE DIMENSIONS
\usepackage{geometry} % to change the page dimensions
\geometry{a4paper} % or letterpaper (US) or a5paper or....
% \geometry{margin=2in} % for example, change the margins to 2 inches all round
% \geometry{landscape} % set up the page for landscape
%   read geometry.pdf for detailed page layout information

\usepackage{graphicx} % support the \includegraphics command and options

% \usepackage[parfill]{parskip} % Activate to begin paragraphs with an empty line rather than an indent

%%% PACKAGES
\usepackage{booktabs} % for much better looking tables
\usepackage{array} % for better arrays (eg matrices) in maths
\usepackage{paralist} % very flexible & customisable lists (eg. enumerate/itemize, etc.)
\usepackage{verbatim} % adds environment for commenting out blocks of text & for better verbatim
\usepackage{subfig} % make it possible to include more than one captioned figure/table in a single float
% These packages are all incorporated in the memoir class to one degree or another...


%%% SECTION TITLE APPEARANCE
\usepackage{sectsty}
\allsectionsfont{\sffamily\mdseries\upshape} % (See the fntguide.pdf for font help)
% (This matches ConTeXt defaults)

%%% ToC (table of contents) APPEARANCE
\usepackage[nottoc,notlof,notlot]{tocbibind} % Put the bibliography in the ToC
\usepackage[titles,subfigure]{tocloft} % Alter the style of the Table of Contents
\renewcommand{\cftsecfont}{\rmfamily\mdseries\upshape}
\renewcommand{\cftsecpagefont}{\rmfamily\mdseries\upshape} % No bold!

% Labeling density
\newcommand{\density}[2]{\rho(#1, #2)}
\newcommand{\zdensity}[2]{\bar\rho(#1, #2)}
\newcommand{\rdensity}[2]{\delta(#1, #2)}
\newcommand{\length}[1]{\ell(#1)}
\newcommand{\charge}[2]{c(#1, #2)}
\newcommand{\new}[2]{\operatorname{new}(#1, #2)}
\newcommand{\scount}[2]{n(#1, #2)}
\newcommand{\segment}[2]{S(#1, #2)}
\newcommand{\segmentt}[2]{S^+(#1, #2)}
\newcommand{\segments}[1]{L(#1)}

\usepackage{amsmath}
\usepackage{amsthm}

\newtheorem{definition}{Definition} % [chapter]
\newtheorem{theorem}[definition]{Theorem}
\newtheorem{lemma}[definition]{Lemma}
\newtheorem{remark}[definition]{Remark}
\newtheorem{corollary}[definition]{Corollary}
\newtheorem{example}[definition]{Example}
\newtheorem{claim}[definition]{Claim}
\newtheorem{statement}[definition]{Statement}
%%% END Article customizations

\begin{document}

\section{Labeling}

We prove a lower bound for the non-smooth case of Labeling problem with super-linear and sub-polynomial array sizes.
The size of the label space, denoted $m$, is therefore between $2n$ and $n^{1 + o(1)}$.
The number of elements is denoted by $n$.
We prove the lower bound stated in the following theorem.
\begin{theorem}
\label{thm:lower-bound}
The cost of the labeling for $n$ elements into a labeling space with size $m$, $m \geq 2n$, and $m \in n^{1 + o(1)}$ is in
\[
\Omega\left(\frac{n\log^2 n}{\log m - \log n}\right).
\]
\end{theorem}

\subsection{Adversary and Algorithm}
Our adversary builds a segment hierarchy after each insertion. 
The time $t$ refers to the situation after insertion of the $t$-th element.
Our adversary always inserts an element into the last segment of the segment hierarchy so that the inserted element is its median.

\subsection{Definitions and Notation}
Each segment is a subinterval of the interval $[1, m]$ with the integral bounds.
The length of a segment $S = [i, j]$ is denoted by $\length{S}$ and equals $j - i + 1$, the number of elements inside $S$ at the time $t$ is denoted by $\scount{S}{t}$.
We define the density of a segment $S$ at the time $t$ as $\density{S}{t} := \scount{S}{t}/\length{S}$.\footnote{We also use the same notation for disjunction of two disjoint segments. We extend it in the way that $\length{S_1 \cup S_2} = \length{S_1} + \length{S_2}$ and $\scount{S_1 \cup S_2}{t} = \scount{S_1}{t} + \scount{S_2}{t}$.}

The number of segments at the time $t$ is denoted by $\segments{t}$.
The hierarchy at each time is a sequence of nested intervals $\segment{1}{t}, \dots, \segment{\segments{t}}{t}$, i.e. $S_{i + 1} \subset S_{i}$.
The topmost interval equals $[1, m]$ at each time $t$ and is denoted by $\segment{1}{t}$.
The segment at the $i$-th level is denoted by $\segment{i}{t}$.

\subsection{Rebuilding of the Hierarchy}
We distinguish between two types of segments -- bad segments and good segments.
According to the segment type we choose the subsegment on the next level.
Before we proceed we show a technical lemma stating certain facts about the density of the segments.

\begin{lemma}
\label{lm:density-estimate}
Let $S' \subset S$ be a subsegment of the segment $S$ such that $\frac{n(S, t)}{2} \leq n(S', t)$. 
Assume that $\density{S - S'}{t} \leq c \cdot \density{S}{t}$ where $c \geq 1$, then $\density{S'}{t} \geq \frac{\density{S}{t}}{c}$.
\end{lemma}
\begin{proof}
We compute the density of the segment $S'$.
\[
\begin{split}
\density{S'}{t}
	& = \frac{\scount{S'}{t}}{\length{S'}} = \frac{\scount{S}{t} - \scount{S - S'}{t}}{\length{S'}} = \frac{\density{S}{t}\length{S} - \density{S-S'}{t}(\length{S} - \length{S'})}{\length{S'}} \\
	& \geq \frac{\length{S}\density{S}{t}(1 - c) + c\density{S}{t}\length{S'}}{\length{S'}} = \frac{-(c - 1)\scount{S}{t}}{\length{S'}} + c\density{S}{t} \\
	& \geq -2(c - 1)\density{S'}{t} + c\density{S}{t}.
\end{split}
\]

Now use the fact that $\frac{c}{(2c - 1)} \geq \frac{1}{c}$ since $c^2 \geq 2c -1$.
We get that
\[
\density{S'}{t} \geq \frac{c\density{S}{t}}{2c - 1} \geq \frac{\density{S}{t}}{c}.
\]
\end{proof}

From now on we fix a constant $c$ which is in the interval $[1, 2]$ for $n$ large enough.
Our choice of $c$ is $c := 2^{\frac{8\log \frac{m}{n}}{\log n}}$.
Remember that from now on we use this $c$ throughout the proof. 

Let $B_L(S, t)$ and $B^2_L(S, t)$ be the densest possible subsegments of $S$ consisting of the leftmost $\lceil\scount{S}{t}/8\rceil$ and respectively $\lceil\scount{S}{t}/4\rceil$ elements of $S$. Similarly we define the subsegments $B_R(S, t)$ and $B^2_R(S, t)$.

\begin{definition}[Good and bad segments]
Segment $S$ is \emph{bad} if there is a segment $S'$ such that $B_L(S, t) \subseteq S' \subseteq B^2_L(S, t)$ or $B_R(S, t) \subseteq S' \subseteq B^2_L(R, t)$ and $\density{S'}{t} \geq c \cdot \density{S}{t}$
Otherwise we say that the segment $S$ is \emph{good}.
\end{definition}

We also need the following notation. The last creation time of the segment at the level $i$ before $t$ is denoted by $b(i, t)$. Similarly the time of the closest dismiss of $\segment{i}{t}$ from the hierarchy at the time $t$ is denoted by $d(i, t)$.

\begin{definition}[Touched subsegment]
For each good segment $\segment{i}{t}$ we keep a subsegment $\segmentt{i}{t} \subset \segment{i}{t}$ which we refer to as the touched subsegment.
\end{definition}
Remember that the touched subsegment changes in time. And we also build the hierarchy so that $\segment_{i + 1}{t} \subset \segmentt{i}{t}$ for a good segment $\segment{i}{t}$.

\begin{definition}[Critical level]
Assume that there is a rearrange at the time $t$. We define the \emph{critical level} of the mentioned rearrange to be the maximal number $i$ such that 
\begin{itemize}
	\item the $\segment{i}{t}$ is be good at $b(i, t)$ and
	\item the rearrange is strictly in contained in $\segmentt{i}{t}$ and
	\item the rearrange touches $B_L(\segment{j}{t}, b(j, t)) \cup B_R(\segment{j}{t}, b(j, t))$ for the minimal $j > i$ such that $\segment{j}{t}$ was good at $b(j, t)$.
\end{itemize}
\end{definition}

Now we are ready to show the reconstruction of the hierarchy. 
The hierarchy is built for the first time at the time $\frac{n}{2}$ with the critical level $1$ and $\segment{1}{\frac{n}{2}$} is put to be $[1, m]$.

Now we may assume that there is a rearrange at the time $t$ with the critical level $i$. We show a procedure creating the hierarchy at $t + 1$.
\begin{itemize}
	\item The segment at the critical level is preserved, i.e. $\segment{i}{t + 1} := \segment{i}{t}$.
	\item The touched subsegment of the segment is changed so that $\segmentt{i}{t + 1} := \segmentt{i}{i} \cup R$ where $R$  is the segment of the rearrange.
\end{itemize}

Now we rebuild the hierarchy recursively in the following way.
Assume that the segment $\segment{i}{t + 1}$ is already built.
\begin{itemize}
	\item If $\scount{\segment{i}{t + 1}}{t} \leq 16$, then we stop.
	\item We build the segment on the next level according to the segment type of the segment on the previous level.
		\begin{itemize}
			\item[\emph{The segment $\segment{i}{t + 1}$ is good at $b(i, t + 1)$.}] The segment $\segment{i + 1}{t + 1}$ will be the densest subsegment of $\segmentt{i}{t}$ containing $\lfloor\scount{\segmentt{i}{t}}{t}/2\rfloor$ elements.
			\item[\emph{The segment $\segment{i}{t + 1}$ is bad at $b(i, t + 1)$.}] The segment $\segment{i + 1}{t + 1}$ will be the subsegment of $S'$ of $\segment{i}{t + 1}$ showing that $\segment{i}{t + 1}$ is bad.
		\end{itemize}
		If the created segment $\segment{i + 1}{t + 1}$ is good we define the touched subsegment $\segmentt{i + 1}{t + 1} := \segment{i + 1}{t + 1} - (B^2_L(i, t) \cup B^2_R(i, t))$.
\end{itemize}

\begin{lemma}
\label{lm:pigeonhole-density}
If $\segment{i}{t}$ is good, then $\density{\segment{i + 1}{t}}{t} \geq \density{\segmentt{i}{t}}{t}$.
\end{lemma}
\begin{proof}
Split $\segmentt{i}{t}$ into two halves according to the number of elements.
If $\scount{\segmentt{i}{t}}{t}$ is even the denser part contains the wanted number of elements and has the desired density.
If $\scount{\segmentt{i}{t}}{t}$ is odd and the denser part contains $\lceil\scount{\segmentt{i}{t}}{t}/2\rceil$ elements move the border element to the other part and thus obtain the wanted subsegment.
\end{proof}

\begin{lemma}
If $\segment{i}{t}$ is good, then $\density{\segment{i + 1}{t}}{t} \geq \frac{\density{\segment{i}{t}}{t}}{\delta}$.
\end{lemma}
\begin{proof}
Realize that $\density{\segment{i}{t} - \segmentt{i}{t}}{t} = \density{\segment{i}{t} - \segmentt{i}{t}}{b(i, t)} \leq c \density{\segment{i}{t}}{b(i, t)} \leq \density{\segment{i}{t}}{t}$.
Lemmas~\ref{lm:pigeonhole-density}~and~\ref{lm:density-estimate} used for $S' = \segmentt{i}{t}$ imply that $\density{\segmentt{i + 1}{t}}{t} \geq \frac{\density{\segment{i}{t}}{t}}{\delta}$.
\end{proof}

\begin{lemma}
From the construction it holds that $\length{\segment{i + 1}{t}} \leq \length{\segment{i}{t}}/2$.
\end{lemma}
\begin{proof}
We choose the densest subsegment for a good segment $\segment{i}{t}$ and therefore the length is at least halved.
In a bad segment we choose a subsegment denser than the whole segment containing less than half of the elements nested in the segment.
\end{proof}

\begin{lemma}
\label{lm:L-bounds}
It holds that $\log_8 (n/2) - 2 \leq L(t) \leq \log_4 n$.
\end{lemma}
\begin{proof}
On each level $i$ we distribute at least $\lceil \scount{\segment{i}{i}}{b(i, t)}/8 \rceil$ elements to the next level.
Since we stop when $\scount{\segment{i}{i}}{b(i, t)} \leq 16$ and the hierarchy is built when $t \geq n/2$ we get the wanted lower bound.
The same distribution argument used with the fact that at most one quarter of elements is distributed to the next level implies the upper bound.
\end{proof}

\begin{corollary}
\label{cor:L-bound}
For $n$ large enough it holds that $L(t) \geq \log(n) / 4$.
\end{corollary}
\begin{proof}
Follows from Lemma~\ref{lm:L-bounds}.
\end{proof}

\subsection{Charging Scheme} 
Now we are ready to describe the charging scheme which distributes the cost of each rearrange among the inserted items.
For each inserted element $x$ and level $i$ we have the function $\charge{x}{i}$ which returns the \emph{charge} of the element $x$ on the level $i$.
First we have to show that cost of the labeling sequence is bounded from below by $\sum_{x, i} \charge{x}{i}$.
Then we also bound $\charge{x}{i}$.

\begin{definition}[New element]
An element $x$ is \emph{new} in the segment $\segment{i}{t}$ if it was inserted at the time $t$ such that $b(i, t) \leq t \leq d(i, t)$.
The number of new elements in $\segment{i}{t}$ at the time $t$ is denoted by $\new{\segment{i}{t}}{t}$.
\end{definition}

\begin{definition}[Charge]
Let $x$ be an element inserted at the time $t$.
Assume that the segment $\segment{i}{t}$ was good at $b(i, t)$ and  $t \in \left[\left\lceil \frac{b(i, t) + d(i, t)}{2}\right\rceil, d(i, t) \right]$.
Then we define the \emph{charge} of the element $x$ on the level $i$ as 
\[
	\charge{x}{i} := \frac{\scount{\segment{i}{t}}{b(i, t)}}{4 \cdot \new{\segment{i}{t}}{d(i, t)}}.
\]
Otherwise we put $\charge{x}{i} := 0$.
\end{definition}

Notice that $\new{\segment{i}{t}}{t} = t - b(i, t) + 1$ and the new elements in the segment are contained in $\segment{i + 1}{t} \subset \segmentt{i}{t}$.

\begin{lemma}
\label{lm:charges-disjoint}
Let $\operatorname{cost}$ be the cost of the generated labeling sequence. Then $\operatorname{cost} \geq \sum_{x, i} \charge{x}{i}$.
\end{lemma}
\begin{proof}
Assume that there is a rearrange at the time $t$ with the critical level $i$. Then for every level $j$, such that $j > i$ and $\segment{j}{t}$ is good, the rearrange came out of $B^2_L(\segment{j}{t}, b(j, t))$ (or $B^2_R(\segment{j}{t}, b(j, t))$).
Therefore we are allowed to charge the cost of $\frac{\scount{\segment{i}{t}}{b(i, t)}}{8}$ on the level $j$.
We distribute the cost of only among the last half of the new elements on each level $j$.
Hence we get that the work on the level $j$ is distributed properly on the new elements for the rearrange.
The lemma follows from the fact that the charged work is ``disjoint'' for the different levels.
\end{proof}

\subsection{Estimating a Single Charge}

In this subsection we are going to estimate $\charge{x}{i}$ for a single element $x$ inserted at the time $t$ using the quantities depending only on $t$ and $i$.
\begin{lemma}
\label{lm:charge-estimate}
Assume that the element $x$ is inserted at the time $t$ and is charged on the level $i$.
Then $\charge{x}{i} \geq -\frac{c}{7} + \frac{c \density{\segmentt{i}{t}}{t}}{56 \cdot \left(c\density{\segmentt{i}{t}}{t} - \density{\segment{i}{t}}{t}\right)}$.
\end{lemma}
\begin{proof}
For simplicity in this proof we denote $\segment{i}{t}$ by $S$, $\segmentt{i}{t}$ by $S^+$, $b(i, t)$ by $b$ and $d(i, t)$ by $d$.
\begin{claim}
For the element $x$ it holds that $\new{S}{t} \geq \new{S}{d} / 2$.
\end{claim}
\begin{proof}
Follows from the fact that only the last half of the new elements is charged.
\end{proof}

\begin{claim}
It holds that $\new{S}{t} \leq \length{S^+}(\density{S^+}{t} - \density{S}{b}/c)$.
\end{claim}
\begin{proof}
We can compute the number of new elements as 
\[
	\new{S}{t} = \density{S^+}{t}\length{S^+} - \density{S^+}{b}\length{S^+}.
\]
Since $S$ is good at the time of its birth from Lemma~\ref{lm:density-estimate} we get that $\density{S^+}{b} \geq \density{S}{b}/c$. 
When we plug the inequality in we get the wanted claim.
\end{proof}

Now we can derive that
\[
\begin{split}
\charge{x}{i} 
	& = \frac{\scount{S}{b}}{4 \cdot \new{S}{d}}
	\geq \frac{\scount{S}{b}}{8 \cdot \new{S}{t}}
	\geq \frac{\scount{S}{b}}{8 \cdot \length{S^+}(\density{S^+}{t} - \density{S}{b}/c)}.
\end{split}
\]

\begin{claim}
It holds that
\[\density{S}{b} \geq \frac{\density{S}{t}\length{S} - \length{S^+}\density{S^+}{t}}{\length{S} - \length{S^+}/c}.\]
\end{claim}
\begin{proof}
It is clear that $\scount{S}{t} = \scount{S}{b} + \new{S}{t}$.
Thus we have that $\density{S}{t}\length{S} \leq \density{S}{b}\length{S} + \length{S^+}\left(\density{S^+}{t} - \frac{\density{S}{b}}{c}\right)$.
Hence $\density{S}{b} \geq \frac{\density{S}{t}\length{S} - \length{S^+}\density{S^+}{t}}{\length{S} - \length{S^+}/c}$.
\end{proof}

\begin{claim}
For $n$ large enough it holds that $c \leq 2$.
\end{claim}
\begin{proof}
Since for $n$ large enough it holds that $m \leq n^{1 + 1/8}$ we get that
\[
c = 2^{\frac{8\log \frac{m}{n}}{\log n}} \leq 2^{\frac{8\log n^{\frac{1}{8}}}{\log n}} = 2.
\]
\end{proof}

\begin{claim}
For $n$ large enough it holds that
\[
\length{S^+} \leq \frac{7}{8} \cdot \length{S}.
\]
\end{claim}
\begin{proof}
Since $S$ is good we get that $\density{S - S^+}{b} \leq c\density{S}{b} \leq 2\density{S}{b}$.
Now $l(S - S^+) = \frac{\scount{S - S^+}{b}}{\density{S - S^+}{b}} \geq \frac{\scount{S}{b}}{8\density{S}{b}} = \frac{\length{S}}{8}$.
\end{proof}

With there claims we are able to derive the final expression.
\[
\begin{split}
\charge{x}{i} 
	& \geq \frac{\scount{S}{b}}{8 \cdot \length{S^+}(\density{S^+}{t} - \density{S}{b}/c)} \\
	& \geq \frac{c\density{S}{b}\length{S}}{7 \cdot \length{S}(c\density{S^+}{t} - \density{S}{b})} \\
	& \geq \frac{c \cdot \frac{\density{S}{t}\length{S} - \length{S^+}\density{S^+}{t}}{\length{S} - \length{S^+}/c} \cdot \length{S}}{7 \cdot \length{S}\left(c\density{S^+}{t} - \frac{\density{S}{t}\length{S} - \length{S^+}\density{S^+}{t}}{\length{S} - \length{S^+}/c}\right)} \\
	& = \frac{c \length{S} \cdot (\density{S}{t}\length{S} - \length{S^+}\density{S^+}{t})}{7 \cdot \length{S}\left(c\density{S^+}{t}(\length{S} - \length{S^+}/c) - (\density{S}{t}\length{S} - \length{S^+}\density{S^+}{t})\right)} \\
	& = \frac{c \cdot (\density{S}{t}\length{S} - \length{S^+}\density{S^+}{t})}{7 \cdot \left(\density{S^+}{t}(c\length{S} - \length{S^+}) - (\density{S}{t}\length{S} - \length{S^+}\density{S^+}{t})\right)} \\
	& = \frac{c \cdot (\density{S}{t}\length{S} - \length{S^+}\density{S^+}{t})}{7 \cdot \left(c\density{S^+}{t}\length{S} - \density{S}{t}\length{S}\right)} \\
	& \geq \frac{c \cdot (\density{S}{t}\length{S} - \frac{7}{8}\length{S}\density{S^+}{t})}{7 \cdot \left(c\density{S^+}{t}\length{S} - \density{S}{t}\length{S}\right)} = \frac{c \cdot (8\density{S}{t} - 7\density{S^+}{t})}{56 \cdot \left(c\density{S^+}{t} - \density{S}{t}\right)} \\
	& = \frac{c \cdot (8(\density{S}{t} - c\density{S^+}{t}) + 8c\density{S^+}{t} - 7\density{S^+}{t})}{56 \cdot \left(c\density{S^+}{t} - \density{S}{t}\right)} \\
	& = -\frac{c}{7} + \frac{c \cdot (8c - 7)\density{S^+}{t}}{56 \cdot \left(c\density{S^+}{t} - \density{S}{t}\right)} \\
	& \geq -\frac{c}{7} + \frac{c \density{S^+}{t}}{56 \cdot \left(c\density{S^+}{t} - \density{S}{t}\right)}. \\
\end{split}
\]
\end{proof}

Now we show an estimate on the sum of the charges for a single element $x$.
\begin{lemma}
\label{lm:element-charge}
If $x$ is the element inserted at the time $t$, then
\[
\sum_{i} \charge{x}{i} \geq -\frac{L}{4} + \frac{L^2}{2^{17} (\log m - \log n)}.
\]
If $\density{i - 1}{t} = \density{i}{t}$, then $\frac{\density{i}{t}}{\density{i - 1}{t} - \density{i}{t}} = \infty$.
\end{lemma}
\begin{proof}
Define $\mathcal{L} = \{i \mid \charge{x}{i} > 0\}$ and $|\mathcal{L}| = L$.
We also put $W := \sum_{i \in \mathcal{L}} \frac{\density{\segmentt{i}{t}}{t}}{c\density{\segmentt{i}{t}}{t} - \density{\segment{i}{t}}{t}}$.
Then by Lemma~\ref{lm:charge-estimate} it is clear that $\sum_{i \in \mathcal{L}} \charge{x}{i} \geq -\frac{cL}{7} + \frac{cW}{56}$.

For the sake of the proof we define $r_i := \frac{\density{\segmentt{i}{t}}{t}}{\density{\segment{i}{t}}{t}}$ for every $i \in \mathcal{L}$.
Since we charge only for good levels it is clear that $r_i \geq 1/c$.

Let $R = \prod_{i\in\mathcal{L}} r_i$, $M = \max{\mathcal{L}}$ and $m = \min{\mathcal{L}}$.
It is clear that $\rho_M \leq 1$ and $\rho_{m} \geq \frac{m}{2nc^{L(t)}}$.
From the definition of $c$ and Lemma~\ref{lm:L-bounds} it follows that $c^{L(t)} \leq \frac{m^4}{n^4}$.
We can bound $R$ as follows since each good segment in the hierarchy may multiply the estimate by $c$.
\[
R = \prod_{i\in\mathcal{L}} \frac{\density{\segmentt{i}{t}}{t}}{\density{\segment{i}{t}}{t}} \leq \frac{c^{L(t)}\rho_M}{\rho_m} \leq \left(\frac{m}{n}\right)^8 \cdot \frac{2m}{n} \leq \frac{2m^9}{n^9}.
\]

The following computation gives the wanted result.
\[
\begin{split}
	W & = \sum_{i \in \mathcal{L}} \frac{\density{\segmentt{i}{t}}{t}}{c\density{\segmentt{i}{t}}{t} - \density{\segment{i}{t}}{t}} \\ 
	& = \sum_{i \in \mathcal{L}} \frac{r_i}{cr_i - 1} \geq \frac{1}{c} \cdot \sum_{i \in \mathcal{L}} \frac{1}{cr_i - 1}. \\
\end{split}
\]
The sum is minimal when $r_1 = \dots = r_L = R^{1 / L}$.
Now we plug this fact in and get
\[
\sum_{i \in \mathcal{L}} \frac{1}{cr_i - 1} \geq \frac{L}{cR^{1 / L} - 1} \geq \frac{L}{2^{\log c + (\log R)/L} - 1}.
\]

Finally we use the fact that $2^x - 1 \leq 64 x$ for $x$ small enough.
By Lemma~\ref{lm:L-bounds} we have that $\log c + \frac{\log R}{L} \leq \frac{8\log \frac{m}{n}}{\log n} + \frac{72 \log \frac{m}{n}}{\log n} = o(1)$ and therefore smaller than any prescribed constant for $n$ large enough.

Now we may estimate 
\[
\sum_{i \in \mathcal{L}} \frac{1}{cr_i - 1} \geq \frac{L^2}{64(L\log c + \log R)} \geq \frac{L^2}{64 \cdot 32 (\log m - \log n)}.
\]
\end{proof}

\begin{proof}[Proof of Theorem~\ref{thm:lower-bound}]
To prove the theorem we count only with the cost of relabelings when $t \geq n/2$ and $t \leq n$ and in the levels from $\frac{L}{32}$ to $\frac{31L}{32}$.

First we compute the lower bound on the number of good segments at each time $t \geq n/2$.

\begin{claim}
There are at most $\frac{7}{8} L(t)$ bad segments at each time $t$, $t \geq n/2$.
\end{claim}
\begin{proof}
Since $\density{\segment{i}{t}}{t} \geq n/2m$, $\density{\segment{L(t)}{t}}{t} \leq 1$.
Recall that each bad segment multiplies the density by at least $c$ and a good segment lowers the density by at most $1/c$.
For the sake of contradiction assume that there are more than $\frac{7}{8} L(t)$ bad segments.
Then we would get
\[
\density{\segment{L(t)}{t}}{t} \geq \frac{n}{2m} \cdot c^{(7L(t) - L(t))/8} = \frac{n}{2m} \cdot c^{3L(t)/4} = \frac{n}{2m} \cdot 2^{\frac{3\log\frac{m}{n} \log n}{2\log n}} = \frac{n}{2m} \cdot \left(\frac{m}{n}\right)^{3/2} > 1. 
\]
\end{proof}

So we can conclude that at least $L(t)/16$ segments are in the charged levels.

The size of the on the first considered level from the top of the hierarchy is at most $n / 128$ for $n$ large enough.
So the overall number of charges on the considered levels is at least
\[
\frac{1}{2 \cdot 16} \cdot \sum_{i = \frac{L}{32}}^{\frac{31 L}{32}} \left(\frac{n}{2} - \frac{n}{128}\right) \geq \frac{nL}{128}.
\]

So each element is charged on at most $\frac{15 L}{16}$ levels.
We state that at least $n/1024$ elements must be charged on at least $L/1024$ levels.
If not, then there would be at most $n/1024 \cdot 15L/16 + 1023nL/1024^2$ relabelings which is contradictory.

These elements add at least $\Omega\left(\frac{n\log^2 n}{\log m - \log n}\right)$ to the overall cost by Lemma~\ref{lm:element-charge}.
\end{proof}
\end{document}
