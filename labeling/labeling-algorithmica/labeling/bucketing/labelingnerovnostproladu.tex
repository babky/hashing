\documentclass[11pt,a4paper]{article}

\usepackage{epsfig}
\sloppy
%\usepackage{fullpage}
%\usepackage{a4wide}

%\usepackage{showlabels}
\usepackage{amsmath}
\usepackage{amssymb}
\usepackage{bbold}
   
%\usepackage{algorithm}
%\usepackage{algorithmic}
%\usepackage{xcolor}

\usepackage{latexsym}
\usepackage{graphics}
%\usepackage[dvips]{graphics}
%\usepackage{epic}
%\usepackage{eepic}
%\usepackage{color}

\newtheorem{theorem}{Theorem} %***
\newtheorem{corollary}[theorem]{Corollary} 
\newtheorem{claim}[theorem]{Claim} 
\newtheorem{lemma}[theorem]{Lemma} 
\newtheorem{theorem*}{Theorem} 
\newtheorem{fact}[theorem]{Fact}
\newtheorem{remark}[theorem]{Remark}
\newtheorem{definition}[theorem]{Definition} 
\newtheorem{proposition}[theorem]{Proposition} 
\newcommand{\qed}{$\Box$}
%\newcommand{\qed}{\rule{1em}{0in} \hspace*{\fill}$\square$\vspace{1ex}\par}%{\rule{7pt}{7pt}}
\newenvironment{proof}{\noindent {\bf Proof:}}{\hfill \qed \smallskip}
\newenvironment{proofof}[1]{\noindent{\it Proof of #1. }} {{\qed}}

\newcommand\bel[1]{\begin{equation}\label{#1}}
\newcommand\ee{\end{equation}} 

\newcommand\rest{\restriction}


\date{\today} 

\begin{document}


%\def\R{\mathbb{R}}
\def\1{\mathbb{1}}
\def\R{\mathbb{R}}
%\def\1{1}
\def\<{\left<}
\def\>{\right>}
\newcommand{\RL}{\mbox{{\sf RL}}}
\newcommand{\BPL}{\mbox{{\sf BPL}}}
\newcommand{\Log}{\mbox{{\sf L}}}
\newcommand{\NCo}{\mbox{{\sf NC$^1$}}}
\newcommand{\RNC}{\mbox{{\sf RNC$^1$}}}
\newcommand{\SC}{\mbox{{\sf SC}}}

% \def\D{\widetilde{D}}
% \def\hD{\widehat{D}}
% \def\hR{\widehat{R}}

 \def\D{\tilde{D}}
\def\hD{\hat{D}}
\def\hR{\hat{R}}

%\def\supp{{\mathrm{supp}}}
%\def\rank{{\mathrm{rank}}}
\def\supp{{\rm{supp}}}
\def\rank{{\rm{rank}}}

\def\prl{\|}
Vime ze plati ${{b}\choose{a}} <2^{h(b/a)b}$.
case 1 je jasny, vrhneme se tedy na case 2.

case 2. $d-1<k$  chceme najit dolni odhad na d takove, ze 
$$ n \leq {k+d-1 \choose d-1}$$
Ale protoze je to rostouci fce (na tomto intervalu), tak jiste staci najit $d$ t.ze 
$$ n \ge {k+d-1 \choose d-1}$$
Protoze (detaily viz mike) 
$$\log {k+d-1 \choose d-1}\leq 2(d-1) \log\frac{d+k-1}{d-1}$$
staci najit $d$ splnujici
$$\log n\ge 2(d-1) \log\frac{d+k-1}{d-1}$$
Tato nerovnost se da upravit jako 
$$\frac{\log n}{\log\frac{d+k-1}{d-1}}\ge 2(d-1) $$
$$\frac{\log n}{\log({d+k-1})-\log({d-1})}\ge 2(d-1) $$
a protoze $k>d-1$
$$\frac{\log n}{\log({2k})-\log({d-1})}\ge 2(d-1) $$
nyni magicky zvolme
$$d-1 = \frac {\log n }{6\log (4k) - 6 \log \log n}$$
a potom dostanem
$$\log n \ge \frac{2\log n}{6(\log (4k) -\log \log n)} \log \frac{2k6(\log (4k)- \log \log n)}{\log n}$$
coz je 
$$ 1\ge \frac{1}{3(\log (4k) -\log \log n)} (\log 4k + \log 3(\log (4k)- \log \log n) - \log \log n)$$
to jde upravit nasledovne
$$\frac{1}{3(\log (4k) -\log \log n)} (\log 4k + \log 3(\log (4k)- \log \log n) - \log \log n)=$$
$$\frac{\log (3(\log (4k)- \log \log n))}{3(\log (4k) -\log \log n)} + \frac{\log 4k - \log \log n}{3(\log (4k) -\log \log n)}>$$
$$1/3+\frac{\log(3\log\frac{4k}{\log n})}{3\log\frac{4k}{\log n}}<1/3+1/2 = 5/6$$
takze takova volba d je ok


% Hence
% 
% log(n)/2(log(2k)-log(d-1)) < = d-1
% 
% Now if d-1 > = log(n) we're happy.  Otherwise
% we can bound the LHS below by replacing log(d-1) by loglog(n) and we get
% 
% d-1 >= log(n)/2(log(2k)-loglog(n))
% Chceme dokazat, ze plati nasledujici nerovnost
% $$\log(n) \leq 2(d-1)\log ((d+k-1)/(d-1))$$
% volme $d = \frac{\log n}{2(\log(2k) - \log \log n)}+1$.Potom
% $$2(d-1)\log ((d+k-1)/(d-1)) = $$
% $$2\frac{\log n}{2\log (2k) - 2\log \log n}\log\left(\frac{2(\log (2k) - \log \log n) k}{\log n }+1\right)$$
% a tedy otazka zni
% $$\frac{\log n}{\log (2k) - \log \log n}\log\left(\frac{2(\log (2k) - \log \log n) k}{\log n }+1\right)
% \overset{?}{>}\log n$$
% 
% $$\frac{1}{\log (2k) - \log \log n}\log\left(\frac{2(\log (2k) - \log \log n) k}{\log n }+1\right)
% \overset{?}{>}1$$
% 
% A pro $k>(1/2)\log n $
% $$\log\left(\frac{2(\log (2k) - \log \log n) k}{\log n }+1\right)
% \overset{?}{>}\log \left(\frac{2k}{\log n}\right)$$
% $$\left(\frac{2(\log (2k) - \log \log n) k}{\log n }+1\right)
% \overset{?}{>}\left(\frac{2k}{\log n}\right)$$
% 
% $$\left(2(\log (2k) - \log \log n) k+\log n \right) \overset{?}{>}2k$$
% $$\left(2k \log \frac{2k}{\log n } + \log n \right) \overset{?}{>}2k$$
% a jelikoz pro $k>\log n$ je $\log \frac{2k}{\log n }\ge 1$
% jsme hotovi


\end{document}

