\documentclass[11pt]{article}


\setcounter{tocdepth}{3}
\usepackage{graphicx}

\addtolength{\voffset}{-1cm}
\addtolength{\hoffset}{-1.5cm}
\addtolength{\textheight}{+2cm}
\addtolength{\textwidth}{+3cm}


% Packages because of tables.
\usepackage{multirow}
\usepackage{booktabs}
%\usepackage{fullpage}
\usepackage{amsmath}
\usepackage{amssymb}

\newtheorem{theorem}{Theorem} %***
\newtheorem{corollary}[theorem]{Corollary}
\newtheorem{claim}[theorem]{Claim}
\newtheorem{lemma}[theorem]{Lemma}
\newtheorem{theorem*}{Theorem}
\newtheorem{fact}[theorem]{Fact}
\newtheorem{remark}[theorem]{Remark}
\newtheorem{definition}[theorem]{Definition}
\newtheorem{proposition}[theorem]{Proposition}
\newcommand{\qed}{$\Box$}
%\newcommand{\qed}{\rule{1em}{0in} \hspace*{\fill}$\square$\vspace{1ex}\par}%{\rule{7pt}{7pt}}
\newenvironment{proof}{\noindent {\bf Proof:}}{\hfill \qed \smallskip}
\newenvironment{proofof}[1]{\noindent{\it Proof of #1. }} {{\qed}}

\usepackage{url}

\newcommand{\Prob}[1]{\mathbf{Pr}\left(#1\right)}
\newcommand{\E}[1]{\mathbf{E}\left[#1\right]}
\newcommand{\A}{\mathcal{A}}
\newcommand{\N}{\mathbb{N}}
\newcommand{\PTheta}[1]{\Theta\!\left(#1\right)}
\newcommand{\natInt}[2]{ \left\{ #1, \dotsc, #2 \right\} }

\newcommand{\thmA}{{C_0}}
\newcommand{\thmB}{{C_1}}
\newcommand{\thmC}{{C_2}}
\newcommand{\buffer}{{\rm {\bf buffer}}}
\newcommand{\pop}{{\rm {\bf pop}}}
\newcommand{\weight}{{\rm {\bf weight}}}
\newcommand{\midp}{{\rm {\bf midpoint}}}
\newcommand{\depth}{{\rm {\bf depth}}}
\newcommand{\densify}{{\rm {\bf densify}}}
\newcommand{\starts}{{\rm {\bf start}}}
\newcommand{\sends}{{\rm {\bf end}}}

\begin{document}

\title{On online labeling with large label set}

\author{
Martin Babka\thanks{The first three authors gratefully acknowledge a support by the Charles University Grant Agency (grant No. 265 111 and 344 711) and SVV project no. 265 314.}\\
{Charles University, Prague}\\
\tt{babkys@gmail.com}
\and
Jan Bul\'anek\\
%%\thanks{Partially supported by student project GAUK 344711, grant P202/10/0854 of GA \v{C}R and RVO: 67985840.}\\
{Charles University and }\\
{Institute of Mathematics}\\
{Academy of Sciences CR, Prague}\\
\tt{honyai@seznam.cz}
\and
Vladim\'ir \v{C}un\'at\\
{Charles University, Prague}\\
\tt{vcunat@gmail.com}
\and
Michal Kouck\'{y}
\thanks{Part of the work done while visiting Aarhus University, partially supported by the Sino-Danish Center CTIC (funded under the grant 61061130540). Supported in part by GA \v{C}R P202/10/0854, grant IAA100190902 of GA AV \v{C}R, Center of Excellence CE-ITI (P202/12/G061 of GA \v{C}R) and RVO: 67985840.}\\
{Institute of Mathematics}\\
{Academy of Sciences CR, Prague}\\
\tt{koucky@math.cas.cz}
\and
Michael Saks\thanks{The work of this author was done while on sabbatical at Princeton University and
was also supported in part by NSF under  grant CCF-0832787.}\\
{Department of Mathematics}\\
{Rutgers University}\\
\tt{saks@math.rutgers.edu}
}

\maketitle



\begin{abstract}

In the online labeling problem with parameters $n$ and $m$
we are presented with a sequence of $n$ {\em keys} from a totally ordered universe $U$
and must assign each arriving key a label from the label set $\natInt{1}{m}$
so that the order of labels respects the ordering on $U$.
As new keys arrive it may be necessary to change the labels of some items; such changes may be done at any time at unit cost
for each change.
The goal is to minimize
the total cost.
An alternative formulation of this problem is
the \emph{file maintenance problem}, in which the items, instead of being labeled,
are maintained in sorted order in an  array of length $m$, and we pay unit cost for moving an item.

%%MIKE 6-27-12
For the case $m=cn$ for constant $c>1$,
there are known algorithms that use at most $O(n (\log n)^2)$ relabelings in total \cite{Itaietal},
and it was shown recently that this is asymptotically optimal \cite{BKS}. For the case of $m=\Theta(n^{C})$ for constant $C>1$,
algorithms are known that use $O(n \log n)$ relabelings. A matching lower bound was provided in \cite{DSZ04}.  That
proof involved two distinct steps: a lower bound for a problem they call {\em prefix bucketing}
and a reduction from  prefix bucketing to online labeling. In this paper we present a simplified version of their reduction,
together with a full proof (which was missing in the previous work).
Furthermore we give a simplified and improved analysis of the prefix bucketing lower bound.
This improvement allows us to extend the lower bounds for online labeling to the case where the number $m$ of labels is superpolynomial in $n$.
In particular, for superpolynomial $m$ we get an asymptotically optimal lower  bound
$\Omega((n \log n) / (\log \log m - \log\log n))$.
\end{abstract}

\section{Introduction}

In the online labeling problem with parameters $n,m,r$,
we are presented with a sequence of $n$ {\em keys} from a totally ordered universe $U$ of size $r$
and must assign each arriving key a label from the label set $\natInt{1}{m}$ so that the order of labels respects the ordering on $U$.
As new keys arrive it may be necessary to change the labels of some items; such changes may be done at any time at unit cost
for each change.
The goal is to minimize
the total cost.
An alternative formulation of this problem is
the \emph{file maintenance problem}, in which the items, instead of being labeled,
are maintained in sorted order in an  array of length $m$, and we pay unit cost for moving an item.
In this formulation we allow cells of the array to contain a special value that marks unoccupied array cells, and we are
concerned only about the total number of moves of actual items stored in the array.

The problem, which was introduced by Itai, Konheim and Rodeh \cite{Itaietal},
is natural and intuitively appealing, and has had applications to the design of data structures (see for example
the discussion in \cite{DSZ04}, and the more recent work on cache-oblivious data structures
\cite{BenderetalB-Tree,Brodaletal,BDIW}).  A connection between this problem and distributed resource allocation
was recently shown by Emek and Korman \cite{EK11}.


The parameter $m$, the {\em label space} must be at least the number of items $n$ or else no valid labeling is possible.
There are two natural range of parameters that have received the most attention. In the case of {\em linearly many labels} we have
$m=cn$ for some $c>1$, and in the case of {\em polynomially many labels} we have $m=\theta(n^{C})$ for some constant $C>1$.
The problem is trivial if  $|U| \leq m$ , since then we can fix an order preserving
bijection from $U$ to $\natInt{1}{m}$ in advance.   The known upper bounds for this problem, summarized in the next paragraph, put no restriction on the size of $U$.  For the lower bound proved in this paper we will assume that  $|U|$
has size at least exponential in $n$.

Itai et al. \cite{Itaietal} gave an algorithm for the case of linearly many labels having worst case total cost
$O(n (\log n)^2)$.  This algorithm works in the file maintenance formulation, and the total cost includes even manipulation
with unoccupied array cells.
Improvements and simplifications were given by Willard \cite{Willard} and Bender et al. \cite{Benderetal}.
In the special case that $m=n$, algorithms with cost $O((\log n)^3)$ per item were given
\cite{Zhang,BirdSadnicki}.  It is also
well known that the  algorithm of Itai et al. can be adapted to give total cost $O(n \log n)$ in the case
of polynomially many labels.
All of these algorithms make no restriction on the size of
universe $U$ of keys.

For lower bounds, a subset of the present authors
recently proved \cite{BKS} a tight $\Omega(n (\log n)^2)$ lower bound for the case of linearly many labels and tight bound $\Omega(n (\log n)^3)$ for
the case $m=n$. These bounds hold even when the size of the universe $U$ is only a constant multiple of $m$.  The bound remains
non-trivial (superlinear in $n$) for
%%MIKE 6-27-2012
$m  = O(n (\log n)^{2-\varepsilon})$ but
becomes trivial  for $m \in \Omega(n (\log n)^2)$.


For the case of polynomially many labels, Dietz at al. \cite{DSZ04} (also in \cite{Zhang}) proved a matching lower bound for the $O(n \log n)$ upper bound.
In this paper we extend the lower bound so that it applies to larger label spaces,
and we also clarify some significant details that were missing or ambiguous in the original proof.

Their result consists of two parts; a lower bound for a problem they call {\em prefix bucketing}
and a reduction from prefix bucketing to online labeling.
We provide a simpler and more precise lower bound for prefix bucketing,
which allows us to extend the lower bounds for online labeling to the case where the label size is as large as $2^{n}$.
Specifically we prove a lower bound of
$\Omega((n \log n) / (1+\log \log m - \log \log n))$ that is valid
for $m$ between $n$ and $2^{n}$.
Note that for polynomially many labels this reduces to $\Omega(n \log n)$.

Our initial study of their proof of the reduction from prefix bucketing to online
labeling \cite{DSZ04} led us to think that there is a significant gap in the proof.
We modified their proof and obtained a correct version of the reduction.
In the conference version of the present paper,
we claimed that we were correcting an apparently significant gap in the previous proof;
we made this claim after checking with one of the authors who agreed with it.
Having now done a more careful comparison of our proof to theirs,
we see that while the proof in \cite{DSZ04} has some misleading statements and missing details,
it is essentially correct.
We present the details of our modification in order to clarify the ambiguities that were present in \cite{DSZ04}.


We remark that, unlike the bounds of \cite{BKS} for the case of linearly many labels, our lower bound proof requires that the universe
$U$ is at least exponential in $n$. It is an interesting question whether one could design
a better online labeling algorithm for $U$ of size say $m \log n$. We summarize known results in Table \ref{tab-1}. All the results
are for deterministic algorithms. (Subsequent to the work presented in this paper, 
a subset of the authors \cite{BKS2} showed that  $\Omega(n \log(n))$
lower bound 
for the case $m=n^C$ also holds for randomized algorithms).

\begin{table}
\centering
\caption{Summary of known lower and upper bounds for the online labeling problem.} \label{tab-1}
\begin{tabular}{l l c c}
	\toprule
	Array size ($m$) &
	Asymptotic bound &
	Lower bound &
	Upper bound
	\\ \toprule

	$m = n$ &
	$\PTheta{n (\log n)^3}$ &
	\cite{BKS} &
	\cite{Zhang}
	\\ \midrule

	$m = cn$,\,constant $c>1$ &
	$\PTheta{n (\log n)^2}$ &
	\cite{BKS} &
	\cite{Itaietal}
	\\ \midrule

% Michal: I have tentativelly commented this out as we do not talk about it in the intro.
%	
%	$m = n ^ {1 + o(1)}$ &
%	$\PTheta{\frac{n (\log n)^2}{\log m - \log n}}$ &
%	\cite{manuscript} &
%	\cite{Itaietal}
%	\\ \midrule
	
	$m = n ^ {C}$,\, constant $C>1$ &
	$\PTheta{n \log n}$ &
	\cite{DSZ04} &
	\cite{Itaietal}
	\\ \midrule
	
	$m = n ^ {\omega(1)}$ &
	$\PTheta{\frac{n \log n}{1 + \log \log m - \log \log n}}$ &
	[this paper] &
	\cite{BKS}
	\\ \bottomrule
\end{tabular}
\label{table:table_bounds}
\end{table}

Our proof follows the high level structure of the proof from \cite{DSZ04}.  In the remainder of the introduction we sketch the two
parts, and relate our proof to the one in \cite{DSZ04}.

\subsection{Reducing Prefix Bucketing to Online Labeling}
\label{s-1.1}

Dietz et al. \cite{DSZ04} sketched a reduction from prefix bucketing to online labeling.  In their reduction they describe an adversary
for the labeling problem.  They show that given any algorithm for online labeling, the behavior of the algorithm
against the adversary can be used to construct a strategy for prefix bucketing.   As mentioned above,
while the idea of their adversary and the reduction from prefix bucketing to online labeling are essentially correct,
the description is incomplete and ambiguous.  In Section \ref{s-olp} we will present a modification of their adversary, 
and in Section \ref{ss-b to ol} we give a full proof of the
connection to prefix bucketing.  We now sketch the adversary construction and the reduction.

\iffalse
 If one can show that the cost of the derived bucketing strategy is no more than a constant times  the cost paid by the algorithm for relabelings then a lower bound on bucketing
will give a similar lower bound on the cost of any online labeling algorithm.  Unfortunately, their proof sketch does not show this.
In particular, a single relabeling step may correspond to a bucketing step whose cost is $\Omega(\log n))$, and this
undermines the reduction.
This may happen when inserting $\Theta(\log n)$ items into an empty segment of size $n^\epsilon$ without triggering any relabelings.
We construct a different adversary for which one gets the needed correspondence
between relabeling cost and bucketing steps.
\fi

The goal in constructing an adversary is to force any online algorithm to perform many relabelings during insertion on $n$ keys.
%The adversary is described
%in detail in Section \ref{s-olp} here we provide a high level description.
As the adversary proceeds we will refer to keys that have been inserted by the adversary as  
{\em active}, and two active keys are {\em adjacent} if there is no active key between them. 
The adversary starts by inserting the seven equally spaced keys including the minimum and maximum key.
Each successive inserted key will be  the average (rounded down) of some pair of adjacent
active keys.   The central issue in defining the adversary is to determine at each step which
adjacent  pair of active keys to choose. 

It is illuminating to think of this in terms of the file maintenance problem mentioned earlier.
In this reformulation the label space $\natInt{1}{m}$ is associated to an array indexed by $\natInt{1}{m}$
and a key labeled by $j$ is viewed as stored in location $j$.
Intuitively, the adversary wants to choose each successive insertion to be the
average of two adjacent keys that appear in a ``crowded'' region of this array.  The hope is that this will
eventually force the algorithm to move many keys within the array (which corresponds to relabeling them).
The problem is to make precise the notion of ``crowdedness''.
Crowding within the array occurs at different scales (so a small crowded region may lie inside a large uncrowded region)
and we need to find a pair of adjacent keys with the property that all regions containing the pair are somewhat crowded.

With the array picture in mind,
we  call an interval of labels a \emph{segment}, and say that a label is {\em occupied} if there is a key assigned to it.
The \emph{density} of a segment is the fraction of occupied labels.

As a guide to picking each successive key, the adversary maintains a sequence (\emph{hierarchy}) of nested segments.  Each successive segment in the hierarchy
has size at most half the previous segment, and its density  is within a constant factor of the density of the previous segment.
The hierarchy ends with a segment having between 2 and 7 keys.  The next key to be inserted is the average (rounded down)
of the two middle keys in the lowest segment of the hierarchy.


Initially, and during the first 8 insertions, the hierarchy consists of just the single segment $\natInt{1}{m}$.
After each subsequent insertion,
the algorithm $\A$ specifies the label of the next key and (possibly) relabels some keys. The adversary then updates
its hierarchy.  For the hierarchy immediately prior to the insertion, the adversary specifies the {\em critical segment} to be the smallest segment
of the hierarchy that contains the label assigned to the inserted key and the old and new labels
of all keys that were relabeled.  The new hierarchy agrees with the previous hierarchy from the largest segment up to and including
the critical segment.   Beginning from the critical segment the hierarchy is extended as follows.
Having chosen segment $S$ from the hierarchy, define its
{\em left buffer}
to be the smallest subsegment of $S$ that starts at the minimum label of $S$ and
includes at least 1/8 of the occupied labels of $S$, and its {\em right buffer} to be the smallest subsegment that ends at the maximum label of $S$
and includes at least 1/8 of the occupied keys of $S$.
Let $S'$ be the segment obtained from $S$ by
deleting the left and right buffers. The successor segment of $S$ in the hierarchy is
a shortest subsegment of $S'$ that contains exactly half (rounded down) of the occupied labels of $S'$.
The hierarchy ends when we reach a segment with at most seven occupied labels; such a segment necessarily has at least two occupied
labels.

In \cite{DSZ04}, the authors show that there is always a \emph{dense point}, which is
a point  with the property that every segment containing it
has density at least half the overall density of the label space.  They use this as the basis for building the hierarchy.
We build a hierarchy with similar properties using a simpler argument.

It remains to prove that the algorithm will make lot of relabels on the sequence
of keys produced by the adversary.  Following  Dietz et al. \cite{DSZ04}, we 
do this by relating online labeling to the prefix bucketing game.
(Our definition of the game differs slightly from that in \cite{DSZ04}.)

%%MIKE 6-27-2012
A \emph{prefix bucketing of $n$ items into $k$  buckets} (numbered 1 to $k$) is a one player game consisting of $n$ steps.
At the beginning of the game
all the buckets are empty. In each step a new item arrives and the player selects an index $p \in \natInt{1}{k}$.
The new item as well as all items in buckets $p+1,\dotsc,k$ are moved into bucket $p$ at a cost equal to the
total number of items in bucket $p$ after the move.   The run of the game is therefore completely specified by
the sequence $p^1,\ldots,p^n$, where $p^t$ is the bin into which the player placed the new item at step $t$. 
The goal is to minimize the total cost of $n$ steps
of the game.

The lower bound on online labeling is obtained by the following correspondence.
Consider the run of an arbitrary online labeling algorithm $\A$
against the given adversary.  At each step $t$, the adversary determines a particular
level $p^t$ of the hierarchy to be the critical level (which is always at most $k=\lceil \log m \rceil$,
where in this paper $\log x$ stands for the binary logarithm function.)
 Consider the sequence $p^1,\ldots,p^n$ as a  sequence of placements defining a prefix bucketing
of $n$ items into $k=\lceil \log m \rceil$ buckets. It turns out that
the total cost of the prefix bucketing will be within a constant factor of  the total number of relabelings
performed by the online labeling algorithm. Hence, a lower bound on the cost of a prefix bucketing of $n$ items into $k$
buckets will imply a lower bound on the cost of the algorithm against our adversary.
The connection between the cost of $\A$ against the adversary, and the cost of the associated
prefix bucketing is obtained as follows.
We  make the assumption (which can be shown to hold without loss of generality) that the algorithm is \emph{lazy}, which means
that at each time step the set of keys that are relabeled is a contiguous block of keys that includes
the newly inserted keys.    The cost of the bucketing merge step $p^t$ at step $t$ 
is at most the number of keys in
the critical segment, so to relate this to the cost incurred by the online labeling algorithm, it is enough
to argue that at step $t$ a constant fraction of the keys in the critical segment were relabeled.  This is done
by arguing that  for each successor (sub)segment of the critical segment, either all labels in its left buffer
or all labels in its right buffer were reassigned, and the total number of such keys is a constant fraction of the keys
in the critical segment.


\subsection{An Improved Analysis of Bucketing}

It then remains to give a lower bound
for the cost of prefix bucketing.  This was previously given by Dietz et al. \cite{DSZ04} for $k\in \Theta(\log n)$.
We give a different and simpler proof that gives asymptotically optimal bound for $k$ between $\log n$ and $O(n^\epsilon)$.
We define a family of trees called $k$-admissible trees and show that the cost
of bucketing  for $n$ and $k$, is between $dn/2$ and $dn$ where $d$ is the minimum depth of a $k$-admissible tree
on $n$ vertices.  We further show that the minimum depth of a $k$-admissible tree on $n$ vertices
is equal $g_k(n)$ which is defined to be the smallest $d$ such that $\binom{k+d-1}{k} \geq n$.
This gives a  characterization of the optimal cost of prefix bucketing (within a factor of 2).  When we apply
this characterization we need to use estimates of $g_k(n)$ in terms of more familiar functions (Lemma \ref{lm:lower_bound_d}),
and there is some loss in these estimates.

\section{The Online Labeling Problem}\label{s-olp}

Here we provide the formal definition of the online labeling problem.
Let $m$ and $r\ge 1$ be integers. We assume without loss of generality $U=\natInt{1}{r}$.
An online labeling algorithm $\A$ with range $m$ is an algorithm that on input sequence $y^1,y^2,\dotsc,y^t$ of distinct elements
from $U$ gives
an \emph{allocation $f : \{y^1,y^2,\dotsc,y^t\} \rightarrow \natInt{1}{m}$} that respects the natural ordering of $y^1,\dotsc,y^t$, so that
for $x,y\in \{y^1,y^2,\dotsc,y^t\}$, $f(x) < f(y)$ if and only if $x < y$. We refer to $y^1,y^2,\dotsc,y^t$
as \emph{keys}. The trace of $\A$ on a sequence $y^1,y^2,\dotsc,y^n\in U$ is the sequence $f^0,f^1,f^2,\dotsc,f^n$ of functions
such that $f^0$ is the empty mapping and for $i=1,\dotsc,n$, $f^i$ is the output of $\A$ on $y^1,y^2,\dotsc,y^i$.
For the trace $f^0,f^1,f^2,\dotsc,f^n$ and $t=1,\dotsc,n$, we say that $\A$ \emph{relocates $y\in \{y^1,y^2,\dotsc,y^t\}$ at time $t$}
if $f^{t-1}(y)\not=f^t(y)$. So each $y^t$ is relocated at time $t$.
For the trace $f^0,f^1,f^2,\dotsc,f^n$ and $t=1,\dotsc,n$, $Rel^t$ denotes the set of relocated keys at step $t$.
The cost of $\A$ incurred on $y^1,y^2,\dotsc,y^n\in U$ is $\chi_\A(y^1,\dotsc,y^n)=\sum_{i=1}^n |Rel^i|$ where $Rel$ is
measured with respect to the trace of $\A$ on $y^1,y^2,\dotsc,y^n$. The maximum cost
$\chi_\A(y^1,\dotsc,y^n)$ over all sequences $y^1,\dotsc,y^n$ is denoted $\chi_\A(n)$. We write $\chi_m(n)$ for the smallest cost $\chi_\A(n)$ that
can be achieved by any algorithm $\A$ with range $m$.
%% Michal: ``algorithm with range m'' is a technical term, so ``label set $\{1,\dots,m\}$.'' is not good.

\subsection{The Main Theorem}
\label{subsec:main theorem}
In this section, we state our lower bound results for  $\chi_m(n)$.

\begin{theorem}
\label{thm:main}
There are positive constants $\thmA$, and $\thmB$ so that the following holds. Let $m,n$ be integers satisfying
$\thmA \leq n \le m \le 2^{n}$. Let the size of $U$ be more than $2^{n+4}$.
Then $\chi_m(n) \geq \thmB \cdot \frac{n \log n}{1 + \log \log m - \log \log n}$.
\end{theorem}

To prove the theorem we fix a labeling algorithm $\A$ and describe an adversary who 
builds a sequence $y^1,y^2,\dotsc,y^t$ of keys based on the behavior of $\A$  that will cause
the algorithm to incur the desired cost. In the next subsection we will describe the adversary,
and state Lemma \ref{l-main}, which proves a lower bound on the cost incurred
by algorithm $\A$ on the the sequence produced by the adversary.  Theorem \ref{thm:main} follows
immediately from this lemma.

\subsection{Adversary Construction}

Any interval $\natInt{a}{b} \subseteq \natInt{1}{m}$ of label values is called a \emph{segment}. Fix $n,m,r>1$ such that $m\ge n$ and $r>2^{n+4}$.
Fix some online labeling algorithm $\A$ with range $m$.
To pick the sequence of keys $y^1,\dotsc,y^n$, the adversary will maintain a sequence of
nested segments $S^t_{\depth(t)} \subseteq \dotsb \subseteq S^t_2 \subseteq S^t_1 = \natInt{1}{m}$, updating them after each time step $t$.
The adversary will choose the next element $y^{t}$ to fall between the keys in
the smallest interval $S^t_{\depth(t)}$. In what follows, $f^t$ is the allocation of $y^1,\dotsc,y^t$ by the algorithm $\A$.

The \emph{population of a segment $S$ at time $t$} is $\pop^t(S) = (f^t)^{-1}(S)$ and the
\emph{weight of $S$ at time $t$} is $\weight^t(S) = |\pop^t(S)|$. (Here, we use $g^{-1}$ to stand for the preimage of a function $g$, i.e., $g^{-1}(A) = \{ x : g(x) \in A \}$.)
For $t=0$, we extend the definition by $\pop^0(S) = \emptyset$ and $\weight^0(S) = 0$.
The \emph{density of $S$ at time $t$} is $\rho^t(S) = \weight^t(S) / |S|$. For a positive
integer $b$, let $\densify^t(S,b)$ be the minimum  subsegment $T$ of $S$ (with respect to the order that
orders segments by size, and orders segments of the same size by their left endpoint) satisfying:

\begin{itemize}
\item $pop^t(T)$ does not contain any of the  $b$ largest or smallest elements of $\pop^t(S)$.
\item  $\weight^t(T)=\lfloor (\weight^t(S)-2b)/2 \rfloor$.
\end{itemize} 

Hence, $\densify^t(S,b)$ is a densest subsegment $T$ of $S$ that contains 
the appropriate number of items but which excludes at least $b$ items labeled within $S$ both above and below $T$.
These $b$ items on either side of $T$ form the left and right buffers described in Section \ref{s-1.1}.

If $\pop^t(S)=\{x_1<x_2 < \dotsb <x_\ell\}$ define $\midp^t(S)= \lceil (x_{\lceil (\ell-1)/2 \rceil} + x_{\lceil (\ell+1)/2 \rceil} )/ 2 \rceil$.

Let $y^1,y^2,\dotsc,y^t$ be the first $t$ keys inserted  and let $Rel^t$ be the keys
that are relabeled by
$\A$ in response to the insertion of $y^t$. The \emph{busy segment $B^t\subseteq \natInt{1}{m}$ at time $t$} is
the smallest segment that contains $f^t(Rel^t)\cup f^{t-1}(Rel^t \setminus \{y^t\})$.
We say that the algorithm $\A$ is \emph{lazy} if
all the keys that are mapped by $f^{t-1}$ to $B^t$ are relocated at step $t$, i.e., $\pop^{t-1}(B^t) = Rel^t \setminus \{y^t\}$.
Proposition~4 in \cite{BKS} shows that any algorithm $\A$ can be modified to get a lazy algorithm $\A'$
whose cost on any insertion sequence is no more than $\A$.  Therefore it suffices to prove our lower bound
for lazy algorithms, and from now on we assume that  $\A$ is lazy.
We record the following simple observation:

\begin{proposition}
\label{l-busy}
Consider an execution of a lazy algorithm $\A$ on $y^1,\ldots,y^n$.  
For any
time $t\in \natInt{1}{n}$, $|Rel^t|=1+\weight^{t-1}(B^t)$.
\end{proposition}

We are now ready to present our adversary. In the adversary, $b^t_i$ denotes the number of items in the left and right buffers
of $S^t_i$. This number changes only when $S^t_i$ is redefined. Also, $p^t$ denotes the critical level of the hierarchy
after inserting item $y^t$, i.e., the level from which the hierarchy will be rebuilt in the next step.
 

\medskip\noindent {\bf Adversary($\A,n,m,r$)}

\smallskip\noindent Set $p^0=0$.

\smallskip\noindent For $t=1,2,\dotsc,n$ do
\begin{itemize}
\item
If $t < 8$, let $S^t_1 = \natInt{1}{m}$, $b^t_1=1$, $\depth(t)=1$, and $p^{t}=1$.
Set $y^t = 1 + \lceil (t-1) \cdot (r-1)/6 \rceil$, and run $\A$ on $y^1,y^2,\dotsc,y^t$ to get $f^t$.
Continue to next $t$.

\item
\emph{Preservation Rule:}
For $i=1,\dotsc,p^{t-1}$, let $S^t_i=S^{t-1}_{i}$ and $b^t_i=b^{t-1}_i$ .
(For $t\ge 8$, copy the corresponding segments from the previous time step.)

\item
\emph{Rebuilding Rule:}

Set $i=p^{t-1}+1$.

While $\weight^{t-1}(S^t_{i-1}) \ge 8$

\begin{itemize}
\item Set $S^t_i = \densify^{t-1}(S^t_{i-1},b^t_{i-1})$.
\item Set $b^t_i = \lceil \weight^{t-1}(S^t_i)/8 \rceil$.
\item Increase $i$ by one.
\end{itemize}

\item
Set $\depth(t)=i-1$.

\item
Set $y^t = \midp^{t-1}(S^t_{\depth(t)})$.

\item
\emph{The critical level:}
Run $\A$ on $y^1,y^2,\dotsc,y^t$ to get $f^t$. Calculate $Rel^t$ and $B^t$.
Set the critical level $p^{t}$ to be the largest integer $j \in \natInt{1}{\depth(t)}$ such that $B^{t} \subseteq S^{t}_j$.

\end{itemize}

%\smallskip
\noindent\emph{Output:} $y^1,y^2,\dotsc,y^n$.


\medskip

A crucial assumption for this adversary is that $y^1,\ldots,y^t$ are distinct.  At step $t \geq 8$, $y^t$ is defined
to be the average of two adjacent keys, so it is enough to ensure that for $t \leq n-1$, 
the difference between any two
adjacent keys is always at least 2.  To do this, we will restrict the size $r$ of the universe to be at least
$2^{n+4}$ so that for any $t \in \natInt{8}{n-1}$ the smallest pair-wise difference between integers 
$y^1,y^2,\dotsc,y^t$ is
at least $2^{n+1-t}$. This property is maintained inductively.

We make the following claim about the adversary which implies Theorem \ref{thm:main}.

\begin{lemma}\label{l-main}
Let $m,n,r$ be positive integers such that $n \le m$ 
and $2^{n+4}<r$. %ToDo: re-check we no longer need the removed requirements anywhere
Let $\A$ be a lazy online labeling algorithm with the range $m$.
Let $y^1,y^2,\dotsc,y^n$ be the output of {\bf Adversary($\A,n,m,r$)}.
Then the cost $$\chi_\A(y^1,y^2,\dotsc,y^n) \ge \frac{1}{512} \cdot \frac{n \log n}{ 1 + \log \lceil \log m \rceil - \log \log n} - \frac{n}{6}.$$
\end{lemma}

 The constants are chosen for
ease of exposition and can certainly be improved.
%
% MK: if one were to set the bucketing k = \lfloor log m \lfloor then we could
%     drop the ceiling around log m provided that k = \lfloor log m \floor > log n.
%     This comes from a requirement in Lemma 17 for the bucketing lower bound.
%     This would be true for example in m > 10n.


To prove the lemma we will design a so called \emph{prefix bucketing game} from the interaction between the
adversary and the algorithm, we will relate the cost of the prefix bucketing to the cost $\chi_\A(y^1,y^2,\dotsc,y^n)$ (Lemma \ref{l-b2l})
and we will lower bound the cost of the prefix bucketing (Lemma \ref{l-lbb}). The proof of Lemma \ref{l-main} is on page \pageref{pf-main}.

In preparation for this, we prove several useful properties of the adversary.

\begin{lemma}
For any $t\in \natInt{1}{n}$, $\depth(t)\le \log m$.
\end{lemma}

\begin{proof}
%%MIKE 6-27-2012
The lemma is immediate for $t <8$.
For $t\ge 8$, it suffices to show that the hierarchy $S^t_1,S^t_2,\dotsc$ satisfies that for each $i \in \natInt{1}{\depth(t)-1}$,
$|S^t_{i+1}| \leq |S^t_{i}|/2$.   Let $S'$ be the largest subsegment of $S^t_{i}$ that excludes the
smallest $b$  labels in $S^t_i$ that are used by $f^{t-1}$ 
and the largest $b$ labels  in $S^t_i$ that are used by $f^{t-1}$ and let $w= \lfloor \weight^{t-1}(S')/2 \rfloor$.
 Recall that $S^t_i$ is a subsegment of $S'$ of minimum size having weight $w$.
Let $L'$ be the smallest segment having the same
left endpoint as $S'$ such that $\weight(L')=w$ and $R'$ be the smallest
segment having the same right endpoint as $S$ such that $\weight(R')=w$. Then
$R'$ and $L'$ are disjoint subsegments of $S'$ and $|S^t_{i+1}| \leq |L'|$ and $|S^t_{i+1}| \leq |R'|$,
and so
 $|S^t_{i+1}| \leq (|L'|+|R'|)/2 \leq |S'|/2 < |S^t_i|/2$.
\end{proof}


The next lemma reflects a rather subtle point in the definition of the adversary.
For any $t \in \natInt{1}{n}$ and $i \in \natInt{1}{\depth(t)-1}$, the
difference $S^t_i \setminus S^t_{i+1}$ is a pair of segments denoted $L^t_i$ (to the left of $S^t_{i+1}$)
and $R^t_i$.  

\begin{lemma}
\label{l-b}  For  any $t \in \natInt{1}{n}$ and any $i \in \natInt{1}{\depth(t)-1}$, $\weight^{t-1}(L^t_i) \geq b^t_i$
and $\weight^{t-1}(R^t_i) \geq b^t_i$.
\end{lemma}

\begin{proof}
We prove this by induction on $t$.  

If $i \geq  p^{t-1}$, then $S^t_{i+1}$ is rebuilt at time $t$ and by the \emph{Rebuilding Rule},
$L^t_i$ and $R^t_i$ will each have weight at least $b^t_i$.

If $i < p^{t-1}$ then $S^t_i$ and $S^t_{i+1}$ are preserved at time $t$ and so $L^t_i=L^{t-1}_i$ and $R^t_i=R^{t-1}_i$
are also unchanged, and since $i+1 \leq p^{t-1}$, we have $B^{t-1} \subseteq S^{t-1}_{i+1}=S^{t}_{i+1}$
and consequently, applying the induction hypothesis
we have $\weight^{t-1}(L^t_i)=\weight^{t-1}(L^{t-1}_i) \geq b^{t-1}_i=b^t_i$ 
and $\weight^{t-1}(R^t_i)=\weight^{t-1}(R^{t-1}_i) \geq b^{t-1}_i=b^t_i$.  
\end{proof}
   
This lemma reflects a subtle point in the adversary.  In the definition of $b^t_i$
we set it to $\lceil \weight^{t-1}(S^t_i)/8 \rceil$ only in the case that $i >p^{t-1}$,
while it might seem more natural to always define $b^t_i$ in this way.  However,
our actual definition which sets $b^t_i=b^{t-1}_i$ for $i \leq p^{t-1}$ is
needed for the induction step in the above proof.

We now use this to relate the cost of relabelings at step $t$ to the quantities $b^t_i$:

\begin{lemma}
\label{l-rel}
If $\A$ is lazy then for any $t \in \natInt{1}{n}$, $|Rel^t| \ge \sum_{i=p^t+1}^{\depth(t)} b^t_i$.
\end{lemma}


\begin{proof}
If $p^t = \depth(t)$ then the sum on the right hand side of the inequality evaluates to zero so the lemma is true. 
By the termination condition in the \emph{Rebuilding Rule}, $\weight^{t-1}(S^t_{\depth(t)}) \in \natInt{2}{7}$, so $b^t_{\depth(t)} = \lceil \weight^{t-1}(S^t_{\depth(t)})/8 \rceil = 1$.
Thus, if $p^t = \depth(t)-1$ then the lemma is true as well
since $|Rel^t|\ge 1$ and $b^t_{\depth(t)}=1$. 

So let us assume that $p^t < \depth(t)-1$.
Consider the set $S^t_{p^t+1} \setminus S^t_{\depth(t)}$ which
is the union of two subsegments, the left one $S_L$ and the right one $S_R$ (Fig.~\ref{f-1}).  We claim that the
busy segment $B^t$ contains one of the sets $S_L$ and $S_R$.

\begin{figure}
  \begin{center}
    \includegraphics[scale=1]{lab-fig1}
    \caption{A typical position of $S_L,S_R$ and $B^t$.}
    \label{f-1}
  \end{center}
\end{figure}

Suppose for contradiction that $B^t$ misses some label $a_L \in S_L$ and some label $a_R \in S_R$.
Recall that $y^t$ was chosen by the adversary to be between two keys
$u,v$ that were stored  in $S^{t}_{\depth(t)}$ by $f^{t-1}$.
If neither $u$ nor $v$ was relabeled by $f^t$ then (by the laziness of $\A$),
$Rel^t=\{y^t\}$ and $B^t \subseteq S^t_{\depth(t)}$ which would imply $p^t=\depth(t)$,
contradicting our assumption
that $p^t < \depth(t)-1$.  So at least one of $u,v$ belongs to $Rel^t$. Thus
$B^t$ is a segment having nonempty intersection with the segment $S^t_{\depth(t)}$ which is a subset
of the segment $\natInt{a_L}{a_R}$. Since, by assumption $B^t$ is a segment containing neither  $a_L$ nor $a_R$
it is a subset of $\natInt{a_L}{a_R}$ which contradicts the fact (from the definition of $p^t$) 
that $B^t$ is not a subset of $S^t_{p^t+1}$.   

So assume, without loss of generality, that $S_L \subseteq B^t$.
Since $\A$ is lazy, $|Rel^t|=1+|\pop^{t-1}(B^t)| \geq 1+|\pop^{t-1}(S_L)|$.
Now $S_L$ is the disjoint union of sets $L^t_i$ for $i \in \natInt{p^t+1}{\depth(t)-1}$  and so
by Lemma \ref{l-b}, we have $|Rel^t| \geq 1+ \sum_{i=p^t+1}^{\depth(t)-1} b^t_i$.
Since $b^t_{\depth(t)}=1$ we obtain the inequality
of the lemma.
\end{proof}
\iffalse

Let $r_{\mathrm{min}}=\min(Rel^t)$ and $r_{\mathrm{max}}=\max(Rel^t)$ be the minimal and maximal key relocated when going from $f^{t-1}$ to $f^t$.


By the definition of $p^t$ we know that the busy segment $B^t$ is not a subset of $S^t_{p^t+1}$
and so at least one of the following must happen: 
\begin{enumerate}
\item $f^{t-1}(r_{\mathrm{min}}) \in S^t_{p^t} \setminus S^t_{p^t+1}$, or
\item $f^{t}(r_{\mathrm{min}}) \in S^t_{p^t} \setminus S^t_{p^t+1}$, or 
\item $f^{t-1}(r_{\mathrm{max}}) \in S^t_{p^t} \setminus S^t_{p^t+1}$, or 
\item $f^{t}(r_{\mathrm{max}}) \in S^t_{p^t} \setminus S^t_{p^t+1}$. 
\end{enumerate}
Assume that $f^{t-1}(r_{\mathrm{min}}) \in S^t_{p^t} \setminus S^t_{p^t+1}$ or $f^{t}(r_{\mathrm{min}}) \in S^t_{p^t} \setminus S^t_{p^t+1}$,
the other case is symmetric. Let $Y = \{ y \in \{y^1,y^2,\dotsc,y^{t-1}\}:\; r_{\mathrm{min}} \le y < y^t\}$.
Since $\A$ is lazy,  $Y \subset Rel^t$, i.e., all the keys between $r_{\mathrm{min}}$ and $y^t$ were relocated when inserting $y^t$.

We claim that all of the keys mapped by $S_L$ belong to $Rel^t$, that is the keys mapped by $f^{t-1}$ to $S_L$ must all have been relocated
when inserting $y^t$ since they are between $r_{\mathrm{min}}$ and $y^t$ and the algorithm $\A$ is lazy.
For any $i \in \natInt{1}{\depth(t)-1}$, the population of the left subsegment of $S^t_i \setminus S^t_{i+1}$ at time $t-1$ is at least $b^t_i$, by the definition of $S^t_{i+1}$.
These subsegments partition $S_L$.
Hence, $\sum_{i=p^t+1}^{\depth(t)-1} b^t_i \le |\pop^{t-1}(S_L)| \le |Y| < |Rel^t|$. Since $b^t_{\depth(t)}=1$, the lemma follows.
\end{proof}
\fi

The next step in deriving a lower bound on the cost is to derive a lower bound on $b^t_i$.
For convenience we define the following quantities.
For $t \in \natInt{1}{n}$ and $i \in \natInt{1}{\depth(t)}$, define $\starts(t,i)$ to be the largest $t' \le t$ 
such that $p^{t'-1} < i$. Hence, $\starts(t,i)$ denotes the time when the segment $S^t_i$ that is current at time $t$ was created.
Similarly, let $\sends(t,i)$ be the smallest $t'>t$ such that $p^{t'-1} < i$ if such a $t'$ exists and $n+1$ otherwise.
Thus, $\sends(t,i)$ denotes the time when the segment $S^t_i$ was replaced or removed from the hierarchy. 


% The next proposition follows from definitions.
%
% \begin{proposition}
% For any $t\in\{1,\dots,n\}$, $i\in[\depth(t)-1]$, and any time $t'\in [\starts(t,i), \sends(t,i)-1]$,
% $B^{t'} \subseteq S^t_i$. Furthermore, if $\sends(t,i)\le n$ then $B^{\sends(t,i)} \setminus S^t_i \not=\emptyset$.
% \end{proposition}


\begin{lemma}\label{l-p1}
For any $t \in \natInt{1}{n}$ and $i \in \natInt{1}{\depth(t)-1}$, $64\cdot b^t_{i+1} \ge \weight^{t-1}(S^t_i) - \weight^{t-1}(S^t_{i+1})$.
\end{lemma}

\begin{proof} %% {Lemma \ref{l-p1}}
For $t<8$, there is no $i\in \natInt{1}{\depth(t)-1}$ so the lemma is true trivially. Thus, assume that $t\ge 8$.
During the time interval $\natInt{\starts(t,i)}{\sends(t,i)}$, $\weight(S^t_i)$ increases by exactly
one for each step and so for any integers $s,s'$ such that $\starts(t,i)\le s < s' < \sends(t,i)$,
\begin{align*}
\weight^{s-1}(S^t_i) = \weight^{s'-1}(S^t_i) + (s'-s).
\end{align*}
Let $s=\starts(t,i+1)$. Then $\starts(t,i) \le s \le t <  \sends(t,i+1) \le \sends(t,i)$ so
\begin{align*}
\weight^{t-1}(S^t_i) - \weight^{t-1}(S^t_{i+1}) &= \weight^{s-1}(S^t_i) - \weight^{s-1}(S^t_{i+1}) \cr
&= \weight^{s-1}(S^s_i) - \weight^{s-1}(S^s_{i+1}).
\end{align*}
Also
\begin{align*}
b^t_{i+1} = b^s_{i+1} = \lceil \weight^{s-1}(S^s_{i+1})/8\rceil \ge  \weight^{s-1}(S^s_{i+1})/8.
\end{align*}
Since $8 \le \weight^{s-1}(S^t_{i})$ and $\weight^{\starts(t,i)-1}(S^t_{i}) \le \weight^{s-1}(S^t_{i}) $
\begin{align*}
b^s_{i} = b^{\starts(t,i)}_{i} &= \lceil \weight^{\starts(t,i)-1}(S^t_{i})/8 \rceil \cr
&\le \weight^{s-1}(S^s_{i})/4
\end{align*}
Hence,
\begin{align*}
\weight^{s-1}(S^s_{i+1}) = \lfloor  (\weight^{s-1}(S^s_{i}) - 2b^s_i )/ 2 \rfloor &\ge \lfloor \weight^{s-1}(S^s_{i})/4 \rfloor \cr &\ge \weight^{s-1}(S^s_{i})/8.
\end{align*}
Thus, $b^t_{i+1} \ge \weight^{s-1}(S^s_{i})/64 \ge (\weight^{s-1}(S^s_i) - \weight^{s-1}(S^s_{i+1}))/64$. The claim follows.
\end{proof}



\begin{corollary}\label{c-weight}
For any $t \in \natInt{1}{n}$ and $j \in \natInt{1}{\depth(t)-1}$,$$\sum_{i=j+1}^{\depth(t)} b^t_{i} \ge 
\frac{1}{64} \cdot \weight^{t-1}(S^t_j) - \frac{1}{8}.$$
\end{corollary}

\begin{proof}
For fixed $t$, sum the inequality in Lemma \ref{l-p1} for $i$ from $j$ to $\depth(t)-1$, and note that
$\weight^{t-1}(S^t_{\depth(t)}) \leq 8$.  Then divide through by 64.
\end{proof}


We now come to the main lower bound of this section.


\begin{corollary}\label{c-main}
Let $\A$ be a lazy algorithm. Then 
$$ \chi_\A(y^1,y^2,\dotsc,y^n)  \ge \frac{1}{64} \sum_{t=1}^n \weight^{t-1}(S^t_{p^t}) - \frac{n}{8}.$$
\end{corollary}

\begin{proof}
$\chi_\A(y^1,\ldots,y^n)$ is at least $\sum_{t=1}^n |Rel^t|$.  Now combine Lemmas \ref{l-rel} and Corollary 
\ref{c-weight}. (For $t$ such that $p^t=\depth(t)$ use the fact that $|Rel^t|>0> \frac{1}{64} \cdot \weight^{t-1}(S^t_{p^t}) - \frac{1}{8}$.)
\end{proof}


\section{Prefix Bucketing}
\label{s-pb}

We start by a definition of the prefix bucketing.
A {\em  bucket configuration} for $k$ buckets and $t$ items  is a sequence $a_1,\ldots,a_k$ of nonnegative integers
summing to $t$.  One should think of $a_i$ as the number of items in bucket $i$.

Given a bucket configuration
$a_1,\ldots,a_k$, {\em placing a new item in bucket $p$} transforms  the configuration as follows:
A new item is added to bucket $p$ and all items in buckets higher than $p$ are moved to bucket $p$.
Buckets $i < p$ are unchanged.  Formally the configuration $b$ produced from $a$ by the placement $p$ satisfies:

\begin{itemize}
\item $b_i=a_i$ for $i < p$,
\item $b_p=1+\sum_{i \geq p} a_i$, and
\item $b_i=0$ for $i>p$.
\end{itemize}


A {\em prefix bucketing} of $n$ items into $k$ buckets is a sequence $a^0,a^1,\dotsc,a^n$ such that
each $a^t$ is a bucket configuration of $t$ items into $k$ buckets, and there is a sequence
$p^1,\dots p^n$ of integers in $\natInt{1}{k}$ such that for each $t \in \natInt{1}{n}$,
$a^t$ is obtained from $a^{t-1}$ by placing a new item in bucket $p^t$ according to the transformation
defined above.  The sequence $p^1,\ldots,p^n$ is called the {\em placement sequence} of the bucketing.

\iffalse
satisfying: $a^0 = (0,0,\dotsc,0)$ and for $t=1,2,\dotsc,n$, there exists $p^{t} \in \natInt{1}{k}$ such that
\begin{enumerate}
\item $a^t_i = a^{t-1}_i$, for all $i=1,2,\dotsc,p^{t}-1$,
\item $a^t_{p^{t}} = 1+ \sum_{i\ge p^{t}} a^{t-1}_i$, and
\item $a^t_i = 0$, for all $i=p^{t}+1,\dotsc,k$.
\end{enumerate}
\fi

The \emph{cost of the bucketing $a^0,a^1,\dotsc,a^n$} is $c(a^0,a^1,\dotsc,a^n) = \sum_{t=1}^n a^t_{p^{t}}$. 

\subsection{Connecting bucketing to online labeling}
\label{ss-b to ol}

In this subsection we show that given any lazy online labeling algorithm $\A$, we can use the adversary
defined in the previous section to construct a prefix bucketing whose cost provides a lower bound
on the cost of $\A$ in labeling the sequence $Y=\{y_1,\ldots,y_n\}$ produced by the adversary. 


Fix a lazy online labeling algorithm $\A$ and for $1 \leq t \leq n$, let $f^t,S^t_i,B^t,p^{t},y^t$ and $f^0,p^0$ be as
defined by the {\bf Adversary($\A,n,m,r$)}. Let $Y$ denote the set $\{y^1,y^2,\dotsc,y^n\}$.


Set $k=\lceil \log m\rceil$.
For $t=0,1,\dotsc,n$ we define a sequence $(A^t_i:1 \leq i \leq k)$ of subsets of $Y$
as follows:
For all $i=1,\dotsc,k$, $A^0_i = \emptyset$.  For $t>0$:
\begin{itemize}
\item $A^t_i = A^{t-1}_i$, for all $i=1,\dotsc,p^{t}-1$,
\item $A^t_{p^{t}} = \{y^t\} \cup \bigcup_{i\ge p^{t}} A^{t-1}_i$, and
\item $A^t_i = \emptyset$, for all $i=p^{t}+1,\dotsc, k$.
\end{itemize}

For each $t \in \natInt{1}{k}$, let $a^t=a^t_1,\ldots,a^t_k$ be the sequence defined
by $a^t_i=|A^t_i|$.  It is easy to see (by induction on $t$) that $a^0,\ldots,a^n$ is a prefix bucketing
of $n$ items into $k$ buckets with placement sequence $p^1,\ldots,p^n$. 
The cost of this bucketing is $c(a^0,a^1,\dots,a^n)=\sum_{i=1}^n |A^t_{p^t}|$.  Our next step is to relate this cost
to the cost of online labeling.

We start by observing:


\begin{lemma}\label{l-p2}
For any $t \in \natInt{1}{n}$ and $i \in \natInt{1}{\depth(t)}$,  $f^{t-1}(A^t_i \setminus \{y^t\}) \subseteq S^t_i$.
\end{lemma}

\iffalse
\begin{proof}
We prove this by induction on $t$.  For $t=1$ the result is trivial since  each of the sets $A^1_i \setminus \{y^t\}$ is
empty.

Assume $t>1$.    If $A^t_i\setminus \{y^t\}= \emptyset$  then the conclusion is trivial.  So assume
$A^t_i\setminus \{y^t\} \neq \emptyset$.  In particular, $i \leq p^t$.  We also claim that $i \leq p^{t-1}$.
If not then by definition $A^{t-1}_j = \emptyset$.  If $j< $
For $i=p^t$, by the definition of $p^t$ and  $A^t_{p^t}$ and the induction hypothesis we have:

$$
f^{t-1}(A^t_{p^t}\setminus \{y^t\}) = f^{t-2}(A^t_{p^t}\setminus \{y^t\}) =  \bigcup_{i \geq p^t}f^{t-2}( A^{t-1}_i) \subseteq \bigcup_{i \geq p^t} S^{t-1}_i = S^{t-1}_{p^t}=S^t_{p^t}.
$$

Finally consider the case $i<p^t$.  Then $A^t_i=A^{t-1}_i$.  If $i > p^{t-1}$ then this set is empty, and we're done.
So assume $i \leq p^{t-1}$.    
Recall from the definition of the adversary that $i \leq p^{t-1}$ implies $S^t_i=S^t_{i-1}$, and $i \leq p^t$ implies  
that  the busy segment $B^t$ is a subset of $S^t_i$,  which implies
that $\pop^t(S^t_i) =  \pop^{t-1}(S^t_i)\cup \{y^t\}$.  Let $i \in \natInt{1}{k}$.
If $i<p^t$ then by the definition of $A^t_i$ and the induction hypothesis
we have:
$$A^t_i = A^{t-1}_i \subseteq (f^{t-2})^{-1}(S^{t-1}_i) =  (f^{t-2})^{-1}(S^t_i) \subseteq (f^{t-1})^{-1}(S^t_i),$$
as required.


\end{proof}
\fi

\begin{proof}  %% {Lemma \ref{l-p2}}
We prove the claim by induction on $t$. For $t=1$, the only non-empty set is $A^1_1=\{y^1\}$ so the claim is true.
Let assume that it is true for $t-1$ and we prove it for $t$.
The {\bf Adversary($\A,n,m,r$)} produces the sets $S^t_i$ and the key $y^t$, and then the algorithm $\A$ outputs $f^t$.
Based on it the adversary defines $B^t, Rel^t$ and $p^t$.
We distinguish two cases.

\emph{Case $p^{t-1} < p^t$:} For all $i\le p^{t-1}$, $A^{t}_i = A^{t-1}_i$, and for all $i> p^{t-1}$, if $i\not=p^t$ then $A^t_i=\emptyset$;
otherwise, $A^t_i = \{y^t\}$. For all $i\le p^{t-1}$, $B^{t-1} \subseteq S^{t-1}_{p^{t-1}} \subseteq S^{t-1}_i = S^t_i$, where
the first containment follows from the definition of $p^{t-1}$ and the last equality follows from the definition of $S^t_i$.
For all $i<p^{t-1}$, $y^{t-1} \not\in A^{t-1}_i = A^t_i$ so using the induction hypothesis
$f^{t-1}(A^t_i) \subseteq f^{t-2}(A^t_i) \cup B^{t-1} \subseteq S^{t-1}_i = S^t_i$.
Similarly for $i=p^{t-1}$, $f^{t-1}(A^t_i) \subseteq f^{t-2}(A^t_i \setminus \{y^{t-1}\}) \cup B^{t-1} \subseteq S^{t-1}_i = S^t_i$.
For each $i>p^{t-1}$, either $A^t_i = \emptyset$ or $i=p^t$ and $A^t_i=\{y^t\}$. In either case the lemma follows trivially.

\emph{Case $p^{t-1} \ge p^t$} is similar: For all $i < p^{t}$, $A^{t}_i = A^{t-1}_i$, for $i=p^t$, $A^t_i = \{y^t\} \cup \bigcup_{j\ge p^t} A^{t-1}_j$,
and for $i>p^t$, $A^t_i=\emptyset$. Again, for all $i\le p^{t}$, $B^{t-1} \subseteq S^{t-1}_{p^{t-1}} \subseteq S^{t-1}_i = S^t_i$.
For all $i<p^{t}$, $y^{t-1} \not\in A^{t-1}_i = A^t_i$ so using the induction hypothesis
$f^{t-1}(A^t_i) \subseteq f^{t-2}(A^t_i) \cup B^{t-1} \subseteq S^{t-1}_i = S^t_i$.
It only remains to consider the case of $i=p^t$ as the claim is trivial for $i>p^t$.
Since $p^{t-1} \ge p^t$, for $i>p^t$, $S^{t-1}_i \subseteq S^t_{p^t}$.
Using the induction hypothesis, $f^{t-1}(A^t_{p^t} \setminus \{y^t\}) \subseteq f^{t-1}(\bigcup_{j\ge p^t} A^{t-1}_j)
\subseteq \bigcup_{j\ge p^t} f^{t-2}( A^{t-1}_j \setminus \{y^{t-1}\}) \cup B^{t-1} \subseteq S^t_{p^t}$. The lemma follows.
\end{proof}

This lemma implies that for each $t \in \natInt{1}{n}$, $\weight^{t-1}(S^t_{p^t}) \ge |A^t_{p^t}| - 1$.
Combining this with  Corollary \ref{c-main} we obtain the following connection between
the cost of online labeling and prefix bucketing.

\begin{lemma}\label{l-b2l}
Let the prefix bucketing $a^0,a^1,\dotsc,a^n$ be defined by $a^t_i = |A^t_i|$, for all $t=0,\dotsc,n$ and $i=1,\dotsc,k$.
The cost $c(a^0,a^1,\dotsc,a^n)$ of the bucketing satisfies:
$$\chi_\A(y^1,y^2,\dotsc,y^n) \ge \frac{1}{64} \cdot c(a^0,a^1,\dotsc,a^n) - \frac{9}{64} n.$$
\end{lemma}


\subsection{Lower Bound for Bucketing}\label{s-lbb}

In this subsection we derive a lower bound (Lemma \ref{l-lbb}) on the cost of any prefix bucketing.
To do so we map any prefix bucketing to a $k$-tuple of ordered rooted trees.
We prove a lower bound on the sum of the depths of the nodes of the trees,
and this will imply a lower bound for the cost of the bucketing.

\paragraph{Ordered trees}
An {\em ordered rooted tree} is a rooted tree where the children of each node are ordered from left to right.
Since these are the only trees we consider, we refer to them simply as trees.
The \emph{$i$-th subtree of $T$} is the tree rooted in the $i$-th child of the root from the left.
If the root has less than $i$ children, we consider the $i$-th subtree to be empty.
The number of nodes of $T$ is called its {\em size} and is denoted $|T|$.
The {\em depth} of a node is  one more than its distance to the root, e.g., the root has depth 1.
The depth of a tree is the maximum depth of its nodes.
The \emph{cost} of $T$, denoted $\kappa(T)$, is the sum of the depths of its nodes.
The cost and size of an empty tree is defined to be zero.


To each prefix bucketing $\bar{a} = a^0,a^1,\dotsc,a^n$ into $k$ buckets, we associate
a $k$-tuple of trees $T(\bar{a})_1,T(\bar{a})_2,\dotsc,T(\bar{a})_k$ inductively 
as follows:
The trivial bucketing $\bar{a} = a^0$ is mapped to the $k$-tuple of empty trees.
For bucketing $\bar{a} = a^0,a^1,\dotsc,a^n$ with placement sequence $p^1,\ldots,p^n$
let $\bar{a}'$ be the bucketing $a^0,a^1,\dotsc,a^{n-1}$, and assume $T(\bar{a}')$ has been defined.
We define $T(\bar{a})$ by:
\begin{itemize}
\item For $1 \le i < p^n$,  $T(\bar{a})_i = T(\bar{a}')_i$.
\item $T(\bar{a})_{p^n}$ consists of
a root node whose children are the non-empty trees among \\
$T(\bar{a}')_{p^n},T(\bar{a}')_{p^n+1},\dotsc,T(\bar{a}')_k$ 
ordered left to right by  increasing index.
\item $T(\bar{a})_i$ is an empty tree for $p^n < i \le k$.
\end{itemize}

We make several simple observations about the trees assigned to a bucketing.

\begin{proposition}\label{p-ts}
For any positive integer $k$, if $\bar{a} = a^0,a^1,\dotsc,a^n$ is a prefix bucketing into $k$ buckets
then for each $i \in \natInt{1}{k}$, $|T(\bar{a})_i| = a^n_i$.
\end{proposition}
\begin{proof}
The proof is straightforward by induction on $n$.
\end{proof}

The next lemma relates the cost of bucketing to the cost of its associated trees.
\begin{lemma}\label{l-tc}
For any positive integer $k$, if $\bar{a} = a^0,a^1,\dotsc,a^n$ is a prefix bucketing into $k$ buckets then
 $c(\bar{a})=\sum_{i=1}^k \kappa(T(\bar{a})_i)$.
\end{lemma}

\iffalse
\begin{proof}
By induction it suffices to show that
$\kappa(T(\bar{a}))-\kappa(T(\bar{a}'))=c(\bar{a})-c(\bar{a}')$, where $\kappa(T(\bar{a})) = \sum_{i=1}^k \kappa(T(\bar{a})_i)$. By definition of $c(\cdot)$ and Proposition \ref{p-ts},
$c(\bar{a})-c(\bar{a}')=a^n_{p^n}=|T(\bar{a})_{p^n}|$.
So it suffices to show $\kappa(T(\bar{a}))-\kappa(T(\bar{a}'))=|T(\bar{a})_{p^n}|$.
By the definition of $\kappa(\cdot)$ and the inductive definition of 
$T(\bar{a})$, 
$$\kappa(T(\bar{a}))-\kappa(T(\bar{a}'))= \kappa(T(\bar{a}_{p^n}))-\sum_{i \geq p^n} \kappa(T(\bar{a}'_{i})).$$ 
This equals $|T(\bar{a})|_{p^n}$, as required, since each node
of $\bigcup_{i \geq p^n} T(\bar{a}')_i$ corresponds (bijectively) to a 
non-root of  $T(\bar{a})_{p^n}$ of depth one greater.
\end{proof}
\fi

\begin{proof}
By induction on $n$. For $n=0$, both sides of the equality are 0.  Suppose $n \geq 1$
and assume that the claim is true for $n-1$.  Let $\bar{a}'=a^0,a^1,\dotsc,a^{n-1}$ and $p^n$ be as in the definition of prefix bucketing.
\begin{align*}
c(\bar{a}) = c(\bar{a}') + 1 + \sum_{i=p^n}^k a^{n-1}_i
%%%% b/c space: %%% &= \sum_{i=1}^k \kappa(T(\bar{a}')_i) + 1 + \sum_{i=p^n}^k a^{t-1}_i \cr
&= \sum_{i=1}^k \kappa(T(\bar{a}')_i) + 1 + \sum_{i=p^n}^k |T(\bar{a}')_i| \cr
&= \sum_{i=1}^{p^n-1} \kappa(T(\bar{a})_i) + 1 + \sum_{i=p^n}^k \left(\kappa(T(\bar{a}')_i) + |T(\bar{a}')_i|\right)
\end{align*}
where the second equality uses the induction hypothesis with Proposition~\ref{p-ts},
and the last equality follows from the definition of $T(\bar{a})_i$ for $i=1,\dotsc,p^n-1$.
For $i\ge p^n$ the depth of each node in $T(\bar{a}')_i$ increases by one when it becomes a child of $T(\bar{a})_{p^n}$, hence
\begin{align*}
\kappa(T(\bar{a})_{p^n}) =  1 + \sum_{i=p^n}^k \left(\kappa(T(\bar{a}')_i) + |T(\bar{a}')_i|\right).
\end{align*}
For $i > p^n$, $\kappa(T(\bar{a})_i) = 0$ so the lemma follows.
\end{proof}

Thus to get a lower bound on the cost of a prefix bucketing 
it suffices to prove a lower bound on the sum of the costs of
the trees that occur in the associated $k$-tuple. 
The following definition will help describe the structure of
trees that occur in such a $k$-tuple.
 
\begin{definition}[$k$-admissible]
\label{def:k-admissibility}
Let $k$ be a positive integer. 
The empty tree is $k$-admissible. A non-empty tree $T$ is $k$-admissible if its root has at most $k$ children
and the $i$-th non-empty subtree of $T$ is $(k+1-i)$-admissible. 
\end{definition}

For example, $T$ is 1-admissible if and only if $T$ is empty or a rooted path.
We collect some basic properties of $k$-admissibility.

\begin{proposition}
\label{lm:k-admissibility}
Let $T$ be a (rooted ordered) tree and $k \geq 1$, and suppose $T$ is $k$-admissible. Let $v$ be a leaf of $T$.
\begin{enumerate}
\item If  $k' > k$ then $T$ is $k'$-admissible.
\item If $v$ is deleted from $T$ then the resulting tree  is  $k$-admissible. 
\item If a new node is added as a child of $v$  then the resulting tree is $k$-admissible.
\item  If $T$ has at least two nodes and $k \geq 2$, 
then the tree obtained from $T$ by removing its first subtree is $(k-1)$-admissible.
\end{enumerate}
\end{proposition}
\begin{proof}
The first and last parts are essentially immediate from the definition of $k$-admissibility.
We prove the other two parts by induction on $|T|$. 
 Let $T'$ be the tree resulting from deleting $v$
and $T''$ be the tree resulting from adding a child to $v$.  If $|T|=1$ then  $T$, $T'$ and $T''$
are $k$-admissible for all $k \geq 1$.
Suppose $|T| > 1$.  Let $v$ belong to the $i$-th subtree $T_i$ of $T$.
By definition of $k$-admissible, $T_i$ is $(k-i+1)$-admissible, and by induction the corresponding
subtree $T''_i$ is also $(k-i+1)$-admissible.  It follows immediately that $T''$ is $k$-admissible.
If $T_i$ consists of a single vertex, namely $v$, then by removing it, the $j$-th subtree
of $T$ (which must be $(k-j+1)$-admissible) becomes the $(j-1)$-st subtree of $T'$ (and is $(k-(j-1)+1)$-admissible 
by the first part) so $T'$ is $k$-admissible. Otherwise, the subtree $T'_i$ that corresponds to $T_i$ is non-empty and 
by induction it is $(k-i+1)$-admissible. Since all other subtrees of $T'$ are the same as in $T$, $T'$ is $k$-admissible.
\end{proof}

The connection of admissibility to prefix bucketing is given by the following proposition, which is obtained by 
an easy induction on $n$ (using the first observation in Lemma \ref{lm:k-admissibility}):


\begin{proposition}\label{p-ta}
For any positive integer $k$, if $\bar{a} = a^0,a^1,\dotsc,a^n$ is a prefix bucketing into $k$ buckets then
for each $i \in \natInt{1}{k}$, $T(\bar{a})_i$ is $(k+1-i)$-admissible.
\end{proposition}

Let us define $\mu(n,k)$ to be the minimum cost of a $k$-admissible tree of $n$ vertices.

\begin{proposition}
\label{prop:mu}
For any bucketing $\bar{a}$ of $n$ items into $k$ buckets, we have $c(\bar{a}) \geq \mu(n,k) - n+1$.
\end{proposition}

\begin{proof}
Modify $\bar{a}$ to the bucketing $\bar{b}=b^0,\ldots,b^n$ where $b^i=a^i$
for $i<n$ and $b^n=(n,0,\ldots,0)$.  This corresponds to placing the final item in bucket 1.
This can increase the cost by at most $n-1$ so
 $c(\bar{a}) \geq c(\bar{b}) - n+1$.  The first tree $U$ in the  $k$-tuple  $T(\bar{b})$ has  size $n$ and is $k$-admissible
by Proposition \ref{p-ta}.  By  Lemma \ref{l-tc}, $c(\bar{b}) \geq \kappa(U)$ which
is at least $\mu(n,k)$.  
\end{proof}

It remains to give a  lower bound on $\mu(n,k)$.  

\begin{lemma}
\label{lm:k-d-cost}
Let $d$ be a positive integer and $T$ be an arbitrary rooted tree with all leaves of depth at least $d$. Then 
$\kappa(T) \geq \frac{(d + 1)}{2} \cdot |T|$.
\end{lemma}
\begin{proof} %% {Lemma \ref{lm:k-d-cost}}
Let $n_i$ denote the number of nodes of $T$ at the depth $i = 1, \dotsc, d - 1$, and let $n_d$ denote the number of all nodes at depth at least $d$. $|T|=n_1+n_2 + \dotsb + n_d$.
Since each node of $T$ at depth $i<d$ has at least one child at depth $i+1$, we have $0\leq n_1 \leq n_2 \leq \dotsb \leq n_d$.
Clearly, $c(T) \geq \sum_{i=1}^d in_i$.
Given the constraints, this sum is minimized (over reals) when $n_1=n_2=\dotsb =n_d = |T| / d $. Thus $c(T) \geq \frac{d(d + 1)}{2} \cdot \frac{|T|}{d}$.
\end{proof}

A $k$-admissible tree may have some leaves at low depth, so the above bound is not immediately useful, so we need
the following:

\begin{definition}[Balanced tree]
A tree of depth $d$ is balanced if all its leaves are of depth $d$ or $d - 1$.
\end{definition}


\begin{lemma}
\label{lm:balanced}
Let $T$ be a $k$-admissible tree of size $n$ having the minimum cost $\mu(n,k)$. Then $T$ is balanced.
\end{lemma}

\begin{proof}  %% {Lemma \ref{lm:balanced}}
Let $d$ be the depth of $T$.  Suppose $T$ is unbalanced.
Let $u$ be a leaf  of depth at most $d-2$ and let $v$ be a leaf of depth $d$.
Let $T'$ be the tree obtained by  
removing $v$ and reattaching $v$ as a child of  $u$. Then $\kappa(T') <\kappa(T)$, and  
by the second and third parts of Proposition \ref{lm:k-admissibility}, $T'$ is $k$-admissible which contradicts the minimality of $\kappa(T)$.
Thus $T$ is balanced.
\end{proof}


\begin{lemma}
\label{lm:k-d-size}
Let $k,d \geq 1$. If $T$ is a $k$-admissible tree of depth $d$,
then $|T| \leq \binom{k + d - 1}{k}$.
\end{lemma}
\begin{proof}
If $k=1$ then $T$ must be a rooted path of depth $d$ and clearly $|T|=d\le \binom{d}{1}$. 
For $k\ge 2$, we prove the result by induction on $|T|$.  If $|T|=1$ the result is trivial so
assume $|T| \geq 2$.  Let $L$ be the first subtree of $T$ and let $R$ be the tree created by removing $L$ from $T$. By the definition of $k$-admissibility
$L$ is a $k$-admissible tree of depth at most $d-1$, and by the last part of
Proposition \ref{lm:k-admissibility}, $R$ is a $(k-1)$-admissible tree of depth at most $d$.
By the induction hypothesis, $|T| = |L| + |R| \leq \binom{k + d - 2}{k} + \binom{k + d - 2}{k - 1} = \binom{k + d - 1}{k}$.
\end{proof}

\begin{corollary}
\label{cor:mu}
For any $n \geq k \geq 1$, $\mu(n,k)$ is at least $n(w+1)/2$ 
where $w$ is the smallest integer such that $\binom{k+w}{k} \geq n$.
\end{corollary}

\begin{proof}
Let $T$ be a $k$-admissible tree of size $n$ having minimum cost.  Let $d$ be the depth of $T$.
By Lemma \ref{lm:k-d-size} we have
$\binom{k+d-1}{k} \geq n$ and so $d-1 \geq w$.
By Lemma \ref{lm:balanced}, $T$ is balanced so all leaves are at depth at least $d-1\geq w$ and so
by Lemma \ref{lm:k-d-cost}, $\kappa(T) \geq (w+1)n/2$.  
\end{proof}

\begin{lemma} \label{lm:lower_bound_d}
Let $n,k,w$ be integers such that $n\ge 2$ and $k \ge \log n$.  If $\binom{k+w}{k} \geq n$,
then $w \ge \frac{\log n}{ 4 (\log 8k - \log\log n)}$.
\end{lemma}
\begin{proof}
If $w \geq k$ then the conclusion holds, so assume $w \leq k$.
Recall that $\log\binom{r}{s} \leq {H(s/r) r}$
where $H$ stands for the binary entropy function
defined on $[0,1]$ by $H(x) = x \log\frac{1}{x} + (1-x) \log\frac{1}{1-x}$, where the base of the logarithm is 2.
For $x \in (0,1/2]$, $H(x) \le 2x \log\frac{1}{x}$. Therefore:
	
	\begin{align*}
	\log(n) \leq \log{k+w \choose k} = \log {k+w \choose w} \leq (k+w)H\left(\frac{w}{k + w}\right)  \leq 
2w\log\left(\frac{k + w}{w}\right) \leq 2w\log\left(\frac{2k}{w}\right).
	\end{align*}
Defining $a=\log(n)/w$, we get $a \leq 2\log(2ak/\log(n))=2(\log(8k)+\log(a/4)-\log\log(n))$
and using $\log(x)<x$ we get $a \leq 2(\log(8k)+a/4 -\log\log(n))$ and so $a/2 \leq 2(\log(8k)-\log\log(n))$, which
implies the claimed bound on $w$.
\end{proof}


We  now deduce the following lower bound on the cost of prefix bucketing:
\begin{lemma}\label{l-lbb}
Let $k,n$ be positive integers such that $k \ge \log n$.
The cost of any prefix bucketing of $n$ items into $k$ buckets is at least $\frac{n \log n}{8 (\log 8k - \log \log n)} - n$.
\end{lemma}

\begin{proof}
By Proposition \ref{prop:mu}, the cost of any such bucketing is at least $\mu(n,k)-n$, and by Corollary \ref{cor:mu},
this is at least $\frac{nw}{2}-n$ where $w$ is the smallest integer such that $\binom{k+w}{k} \geq n$, which is at
least $\log n/4(\log 8k -\log\log n)$ by Lemma \ref{lm:lower_bound_d}.
\end{proof}

\iffalse
\begin{proof}
For $n=1$ the lemma is straightforward so let us assume $n \ge 2$.
Consider a prefix bucketing $\bar{a} = a^0,a^1,\dotsc,a^n$ of $n$ items into $k$ buckets.
Let $b = (n,0,0,\dotsc,0)$ be a $k$-tuple of integers,
and let  $\bar{b} = a^0,a^1,\dotsc,a^{n-1},b$.
Clearly, $\bar{b}$ is also a prefix bucketing and $c(\bar{b}) \le c(\bar{a}) + n-1$. 
Hence, it suffices to show that $c(\bar{b}) > \frac{n \log n}{8 (\log 8k - \log \log n)}$.

Let $T$ be $T(\bar{b})_1$.
By Proposition~\ref{p-ts}, $|T|=n$, and by Proposition~\ref{p-ta}, $T$ is $k$-admissible.
Furthermore, by Lemma~\ref{l-tc}, $c(\bar{b}) = \sum_{i=1}^k c(T(\bar{b})_i) = c(T)$,
so we only need to bound $c(T)$ from below.
From Lemma~\ref{lm:balanced} it follows that
there is a balanced $k$-admissible tree $T'$ of size $n$ such that $c(T) \geq c(T')$.

Let the depth of $T'$ be $d$.
By Lemma~\ref{lm:k-d-size}, $T'$ has at most $\binom{k + d-1}{k}$ nodes.
Using Lemma~\ref{lm:lower_bound_d}, we see that
$d > \frac{\log n}{4 (\log 8k - \log \log n)}$; otherwise, $T'$ could not contain $n$ nodes.
Since $T'$ is balanced, by Lemma \ref{lm:k-d-cost}, $c(T') \ge \frac{(d-1)+1}{2} \cdot n = \frac{dn}{2}$.
Thus, $c(T) \ge c(T') > \frac{n \log n}{8 (\log 8k - \log \log n)}$. The lemma follows.
\end{proof}
\fi

\begin{proofof}{Lemma \ref{l-main}}\label{pf-main}
By Lemma \ref{l-b2l}, the cost incurred by a lazy algorithm on the adversary sequence $y^1,y^2,\dotsc,y^n$
is lower-bounded by
$$\chi_\A(y^1,y^2,\dotsc,y^n) \ge \frac{1}{64} \cdot c(a^0,a^1,\dotsc,a^n) - \frac{9}{64} n$$
where  $a^0,a^1,\dotsc,a^n$ is a certain  prefix bucketing of $n$ items into $k=\lceil \log m \rceil$ buckets.
The previous lemma provides a lower bound on the cost of any such bucketing. Using that lower bound we get:
$$\chi_\A(y^1,y^2,\dotsc,y^n) \ge \frac{1}{512} \cdot \frac{n \log n}{ 3 + \log \lceil \log m \rceil - \log \log n} - \frac{n}{6}.$$
\end{proofof}

\newcommand{\etalchar}[1]{$^{#1}$}
\begin{thebibliography}{32}

\bibitem{manuscript}
{Martin Babka, Jan Bul\'{a}nek, Vladim\'{i}r \v{C}un\'{a}t, Michal Kouck\'{y}, and Michael Saks}.
\newblock On Online Labeling with Superlinearly Many Labels.
\newblock Manuscript 2012.

\bibitem{BKS}
Jan Bul{\'a}nek, Michal Kouck{\'y}, and Michael Saks.
\newblock Tight lower bounds for the online labeling problem.
\newblock In  \emph{Proc. of 66th Symp. of Theory of Computation}, {\em (STOC'12)}, Howard~J. Karloff and Toniann Pitassi, editors, pages 1185--1198. ACM, 2012.

\bibitem{BKS2}
Jan Bul{\'a}nek, Michal Kouck{\'y}, and Michael Saks.
\newblock  On Randomized Online Labeling with Polynomially Many Labels.
\newblock In \emph{Proc. of 4oth  ICALP}, Lecture Notes in Computer Science 7965,
pages 291-302, Springer, 2013.


\bibitem{Benderetal}
Michael~A. Bender, Richard Cole, Erik~D. Demaine, Martin Farach-Colton, and Jack Zito.
\newblock Two simplified algorithms for maintaining order in a list.
\newblock In \emph{Proc. of 10th Annual European Symposium on Algorithms}, {\em (ESA)}, Rolf~H. M{\"o}hring and Rajeev Raman, editors, volume 2461 of {\em LNCS}, pages 152--164. Springer, 2002.

\bibitem{BenderetalB-Tree}
Michael~A. Bender, Erik~D. Demaine, and Martin Farach-Colton.
\newblock Cache-oblivious {B}-trees.
\newblock {\em Journal on Computing}, 35(2):341--358, 2005.

\bibitem{BDIW}
Michael~A. Bender, Ziyang Duan, John Iacono, and Jing Wu.
\newblock A locality-preserving cache-oblivious dynamic dictionary.
\newblock {\em Journal of Algorithms}, 53(2):115--136, 2004.

\bibitem{Brodaletal}
Gerth~St{\o}lting Brodal, Rolf Fagerberg, and Riko Jacob.
\newblock Cache oblivious search trees via binary trees of small height.
\newblock In \emph{Proc. of 13th ACM-SIAM Symp. on Discrete Algorithms}, {\em (SODA)}, D. Eppstein, editor, pages 39--48. ACM/SIAM, 2002.

\bibitem{BirdSadnicki}
Richard~S. Bird and Stefan Sadnicki.
\newblock Minimal on-line labelling.
\newblock {\em Information Processing Letters}, 101(1):41--45, 2007.

\bibitem{entropy}
Thomas M. Cover, Joy A. Thomas. 
\newblock {\em Elements of Information Theory.}
\newblock  New York: Wiley, 1991.

\bibitem{DSZ04}
Paul~F. Dietz, Joel~I. Seiferas, and Ju~Zhang.
\newblock A tight lower bound for online monotonic list labeling.
\newblock {\em SIAM J. Discrete Mathematics}, 18(3):626--637, 2004.

% \bibitem{DZ-SWAT}
% Paul~F. Dietz and Ju~Zhang.
% \newblock Lower bounds for monotonic list labeling.
% \newblock In {\em SWAT}, pages 173--180, 1990.

\bibitem{EK11}
Yuval Emek and Amos Korman.
\newblock New bounds for the controller problem.
\newblock {\em Distributed Computing}, 24(3-4):177--186, 2011.

\bibitem{Itaietal}
Alon Itai, Alan~G. Konheim, and Michael Rodeh.
\newblock A sparse table implementation of priority queues.
\newblock In \emph{Proc. of 8th International Colloquium on Automata, Languages and Programming}, {\em (ICALP'81)} Shimon Even and Oded Kariv, editors, volume 115 of {\em LNCS}, pages 417--431. Springer, 1981.

\bibitem{Willard}
Dan~E. Willard.
\newblock A density control algorithm for doing insertions and deletions in a sequentially ordered file in good worst-case time.
\newblock {\em Information and Computation}, 97(2):150--204, 1992.

\bibitem{Zhang}
Ju~Zhang.
\newblock Density Control and On-Line Labeling Problems.
\newblock {\em PhD thesis}, University of Rochester, 1993.
\end{thebibliography}
\end{document}








