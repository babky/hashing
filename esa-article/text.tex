% This is the LaTeX source for the article dealing with Worst Case 
% Aware Universal Hashing. The source code is a modification of the 
% file typeinst.tex which may be found at http://www.springer.com/lncs
% and ftp://ftp.springer.de/data/pubftp/pub/tex/latex/llncs/latex2e

\documentclass[runningheads,a4paper]{llncs}
\newtheorem{definition}{Definition}
\newtheorem{theorem}{Theorem}
\newtheorem{lemma}{Lemma}
\newtheorem{remark}{Remark}

\usepackage{amsmath}
\usepackage{amsfonts}
\usepackage{amssymb}
\setcounter{tocdepth}{3}
\usepackage{graphicx}
\usepackage{url}

\urldef{\mail}\path|babkys@gmail.com,vcunat@matfyz.cz|
\newcommand{\keywords}[1]{\par\addvspace\baselineskip
\noindent\keywordname\enspace\ignorespaces#1}
\usepackage[utf8]{inputenc}

\begin{document}

\mainmatter

% TODO: Coslo GAUKu dopisat.
\author{Martin Babka\thanks{Moj GAUK} \and Vladimír Čunát\thanks{This research was partially supported by SVV project number 263 314.}}
\authorrunning{Martin Babka, Vladimír Čunát}

\title{A Worst Case Aware Version of Universal Hashing}
\titlerunning{A Worst Case Aware Version of Universal Hashing}
\toctitle{A Worst Case Aware Version of Universal Hashing}

\institute{Faculty of Mathematics and Physics, Charles University,\\ Prague, Czech Republic \\
\mail\\
\url{http://www.ktiml.mff.cuni.cz/~babka/hashing}}

\tocauthor{Martin Babka, Vladimír Čunát}
\maketitle

\begin{abstract}
In this work we propose and analyze a simple hash table giving a worst case time guarantee in case of Find operation with all operations running in constant expected amortized time.
To achieve this goal we exploit known properties of some universal families of functions.
Since the mentioned guarantee strongly depends on the quality of the underlying universal system, we show various systems capable of interesting bounds.
With reasonable assumptions we can also combine the proposed model with the "two choices paradigm" and obtain $\bigo(\log \log n)$ worst case bound.
The main advantage of the method is its simplicity and genericness because it may be used with separating chaining as well as with linear probing.

\keywords{universal hashing, data structures, worst case lookup}
\end{abstract}

\section{Introduction}
Dictionary is a data structure which allows storing and querying data associated with a given key. If there are no assumptions on the set of the keys, e.g. it is not ordered, then we often represent such dictionaries by \emph{hash tables}. We only need a suitable function, that maps the keys to positions in a table and a hash table provides operations \emph{Find}, \emph{Insert} and \emph{Delete}. 

We assume that the given hash function behaves randomly in some way. Provided its full randomness each operation has expected constant running time. There are two major ways how to deal with the randomness. The first approach is to assume uniformly and independently distributed input data. The formal description of these assumptions may be found in \cite{DBLP:books/sp/Mehlhorn84}. These requirements are not necessarily satisfied in some situations although weaker conditions may hold. In this case we switch to a randomization provided by the uniform selection of a hash function from a family of functions. This method is known as \emph{universal hashing} and was pioneered by Carter and Wegman in \cite{DBLP:journals/jcss/CarterW79}. 

The main difference between universal and plain hashing is that we switch to a different probability space. In the area of plain hashing the probability space is formed by a uniform and independent selection of the input data. On the other in case of universal hashing a random hash function from is chosen uniformly from a finite family of functions. Unfortunately dealing with common universal systems brings up a lot of dependencies when considering probabilities of collisions.

Although universal hashing guarantees expected constant time this is not enough when we need a worst case warranty. Perfect hashing \cite{Fredman:1984:SST:828.1884} is an extension of universal hashing which addresses this problem for static sets. In order to add update operations various dynamizations of perfect hashing were proposed \cite{DBLP:journals/siamcomp/DietzfelbingerKMHRT94} and \cite{DBLP:conf/icalp/DietzfelbingerH90}. Dietzfelbinger with Meyer auf der Heide designed real time hash tables \cite{DBLP:conf/icalp/DietzfelbingerH90} with update operations running in expected constant time. Real time hash tables together with cuckoo hashing \cite{DBLP:conf/esa/PaghR01} guarantee a constant look up and the other operations run in amortized constant times in the expected case.

Our goal is such a simple modification of uniform hashing so that we obtain a time warranty in the worst case. We chose a different way compared to cuckoo hashing which more or less resembles open addressing. Compared to perfect hashing our method does not use doubled hash tables. Let us note that representing chains this way significantly improves the provided warranty if separate chaining is used. It is clear that running time of Find operation is proportional to the length of the longest. Our method bounds its length by a predefined limit function and rehashes the table if this rule is violated.

\subsection{Notation}
$U$ denotes the universe, the set of possible key values. $V$ denotes the addresses of the hash table. We refer to $S \subset U$ as to the set of the stored set keys. The size of the hash table is denoted by $m = |V|$ and the number of stored elements is $n = |S|$, usually $n \ll |U|$. The hash table's load factor is denoted by $\alpha = \frac{n}{m}$. It is kept in a predefined interval so that the space is not wasted, usually its maximal value is less than one.

Mapping $f\colon U \rightarrow V$ denotes a currently used hash function. We will discuss probabilistic probabilities of hash functions and their families in more detail.

We often work with the following two random variables. The length of the chain at the address $y \in V$ for a fixed stored set $S$ is denoted by $\psl(y, S)$. The length of the longest chain, $\lpsl(S)$, is defined as $\lpsl = \max_{y \in V} \psl(y, S).$ We omit the parameter $S$ when we talk about bounds that hold for each $S \subseteq U$.

\subsection{Universal hashing}
Universal hashing solves dictionary problem using explicit randomization. Hash functions are picked from a universal system which ensures that for each stored there are many functions in it behaving "properly".

This proper behavior may be understood differently and leads to various definitions of universal systems. However, from each of the definitions it follows that there is only a constant number of elements colliding with a given element. Hence the expected length of a chain is constant and thus expected time complexity of Find operation is constant, too.

\begin{definition}[$c$-universal system \cite{DBLP:journals/jcss/CarterW79}]
\label{definition-c-universal-system}
Let $H$ be a multiset of hash functions $f\colon U \rightarrow V$. We say that $H$ is a \emph{$c$-universal system of functions} if there is a constant $c > 0$ such that for different $x, y \in U$
\[
\left|\lbrace f \in H \setdelim f(x) = f(y) \rbrace\right| \leq \frac{c|H|}{m}.
\]
\end{definition}

To be precise the set on the left side of the above expression is also a multiset. Since each function $f$ is chosen uniformly from $H$ we equivalently restate our definition in terms of probability as
\[
\Prob{f(x) = y} \leq \frac{c}{m}.
\]

From now we assume the uniform choice of a hash function $f$ from a universal system $H$ without any special referral.

More powerful definitions include strong \emph{$k$-universality}, which is also called \emph{$k$-wise independence}, cover \emph{strongly $\omega$-universal} systems and \emph{uniform} systems.
\begin{definition}
Let $k > 0$ be an integer. System of functions $H$ is 
\begin{itemize}
	\item \emph{almost strongly $k$-universal} with constant $c > 0$ if for any sequence of pairwise different elements $x_1, \dots, x_k \in U$ and arbitrary $y_1, \dots, y_k \in V$ \[\Prob{f(x_1) = y_1, \dots, f(x_k) = y_k} \leq \frac{c}{m^k}\mbox{,}\]
	\item \emph{strongly $k$-universal} \cite{DBLP:conf/focs/WegmanC79} if it is nearly strongly $k$-universal with $c = 1$,
	\item \emph{strongly $\omega$-universal} \cite{DBLP:conf/focs/WegmanC79} if it is strongly $k$-universal for each $k \in \bbbn$,
	\item \emph{(almost) uniform} \cite{DBLP:journals/siamcomp/PaghP08} if it is (almost) strongly $n$-universal.
\end{itemize}
\end{definition}

Notice that strongly $k$-universal systems are fully random for up to $k$ different elements and provide a limited randomness for more than $k$ elements. On the other hand strongly $\omega$-universal systems behave fully randomly. When we need estimates only for the $n$ stored elements, then concept of uniform systems is as powerful as full randomness.

\section{Model of Universal Hashing}
\label{section-model}
In this section we provided the analysis in case of universal hashing with separate chaining. 
The way we manage to achieve worst-case lookup time is to keep all chains shorter than the prescribed limit. 
First we introduce the concept of limit function which bounds length of the longest chain with a reasonable probability. 
It turns out that time needed to keep chains short can be amortized to a constant.

\subsection{Limit function}
Limit functions play a crucial role in our model. 
The lower the values of a~limit function are the better warranty is obtained. Apparently each limit function depends on the size of the table, the load factor. 
When thinking about the probability with which the limit holds there is a tradeoff between the obtained worst case guarantee and the time spent per update.

\begin{definition}[Limit function, trimming rate, suitable function]
\label{definition-limit-function}
The function $l\colon \bbbn \times \mathbb{R}_0^+ \times (0, 1) \rightarrow \bbbn$ is called a \emph{limit function} with \emph{trimming rate} $p$, for $p \in (0, 1]$, if $\Prob{\lpsl > l(m, \alpha, p)} \leq p$.

Let $l$ be a limit function with a~trimming rate $p$.
We say that a function $f \in H$ is \emph{suitable}, or \emph{does not create a long chain}, for $S$ if $\lpsl(S) \le l(m, \alpha, p)$. 
We consider the value of the random variable $\lpsl$ for the the function $f$.
\end{definition}

\subsection{Algorithms}
The comprehensive list of all parameters of the hash table:
\begin{itemize}
	\item {$m_0$}, initial size of the hash table;
	\item {$\alpha_l$, $\alpha_u$, $\alpha_l < \alpha_u$}, lower and upper bound on the load factor, lower bound may be violated when $m = m_0$;
	\item {$\alpha_m$, $\alpha_m \in (\alpha_l, \alpha_u)$}, target load factor, when rehashing because of having more than $\alpha_u m$ or less than $\alpha_l m$ elements, we choose new $m$ so that $\alpha = \alpha_m$;
	\item \textbf{$\alpha'$, $\alpha' < \alpha_u$}, the load factor for which we bound $\lpsl$, this parameter sets the tradeoff between the worst case guarantee and the trimming rate.
\end{itemize}
Following three invariants briefly describe the model. 
\begin{itemize}
\item[(1)] \textbf{Universal class and limit function.} The chosen class $H$ must be at least a $c$-universal system which has a limit function $l$ with a trimming rate $p$.
\item[(2)] \textbf{Load Factor Rule.} The load factor of the table is kept in a predefined interval $[\alpha_l, \alpha_u]$ except when $m = m_0$.
\item[(3)] \textbf{Chain Length Limit Rule.} If there is a chain longer than $l(m, \alpha', p)$, then the table is rehashed using a new chosen uniformly at random from $H$.
\end{itemize}

\begin{algorithm}[ht]
\caption{Implementation of the hash table.}
\label{algorithm-hash-table}
\begin{minipage}[t]{0.5\linewidth}
\begin{algorithmic}
\Procedure{Find}{$x$}
	\If{$x$ is inside chain $t[f(x)]$}
		\State \textbf{return} \textbf{true} \Comment{successful case}
	\Else
		\State \textbf{return} \textbf{false} \Comment{unsuccessful case}
	\EndIf
\EndProcedure
\end{algorithmic}
\vspace{0.1cm}
\begin{algorithmic}
\Procedure{Insert}{$x$}
	\If{$x$ is not inside chain $t[f(x)]$}
		\State insert $x$ into the chain $t[f(x)]$
		\If{$n/m > \alpha_u$}
			\State \Call{Rehash}{\textbf{true}}
		\Else
			\State $l_c \leftarrow $ length of the chain $t[f(x)]$
			\If{$l_c > l(m, \alpha', p)$}
				\State \Call{Rehash}{\textbf{false}}
			\EndIf
		\EndIf
		\State \textbf{return} \textbf{true} \Comment{successful case}
	\Else
		\State \textbf{return} \textbf{false} \Comment{unsuccessful case}
	\EndIf
\EndProcedure
\end{algorithmic}
\end{minipage}
\hfill
\begin{minipage}[t]{0.49\linewidth}
\begin{algorithmic}
\Procedure{Delete}{$x$}
	\If {$x$ is inside chain $t[f(x)]$}	
		\State delete $x$ from chain $t[f(x)]$
		\If{$n/m < \alpha_l$ \textbf{and} $m > m_0$}
			\State \Call{Rehash}{\textbf{true}}
		\EndIf
		\State \textbf{return} \textbf{true} \Comment{successful case}
	\Else
		\State \textbf{return} \textbf{false} \Comment{unsuccessful case}
	\EndIf
\EndProcedure
\end{algorithmic}
\vspace{0.1cm}
\begin{algorithmic}
\Procedure{Rehash}{$Resize$}
	\State \Comment{Also finds suitable hash function.}
	\If{$Resize$}
		\State $m \leftarrow n / \alpha_m$
	\EndIf
	\State $t_{old} \leftarrow t$
	\Repeat
		\State create a new table $t$ of size $m$
		\State choose a new function $h$ from $H$
		\ForAll{$x$ in the $t_{old}$}
			\State add $x$ into the chain $f(x)$ in $t$
		\EndFor
	\Until{$\lpsl < l(m, \alpha', p)$}
\EndProcedure
\end{algorithmic}
\end{minipage}
\begin{minipage}[t]{\linewidth}
\vspace{0.3cm}
The available variables include the parameters of the model, the variable $t$ denotes the array used to store the chains and $f \in H$ is the used hash function.
\end{minipage}
\end{algorithm}

\subsection{Consequences of Trimming Long Chains}
Now we deal with with the effects of Chain Length Limit Rule on the expected length of a chain.
We begin by definition of the system of all suitable functions for arbitrary stored set $S$.
\begin{definition}[$(l, p)$ - trimmed system]
\label{definition-trimmed-system}
Let $l$ be a limit function with a~trimming rate $p$.
The \emph{$(p, l)$-trimmed system} is the multiset of functions \[ H(p, l, S) = \{ f \in H \setdelim f \textit{ does not create a long chain for the limit function $l$} \}. \]
\end{definition}

The restriction of $H$ to $H(p, l, S)$ is informed about the stored set since it has a feedback if a function is suitable or not. 
Furthermore, the uniform choice of function from the original system $H$ with refusal of the unsuitable functions is equivalent to the uniform choice of a function from the restricted system.

We show that $(l, p)$-trimmed systems are still universal with a higher constant of universality. 
\begin{lemma}
\label{lemma-trimmed-system}
If $H$ is a universal system with a constant $c$ and $l$ is a limit function with a trimming rate $p$, then $|H(p, l, S)| \geq (1 - p)|H|$ and $H(p, l, S)$ is $c/(1 - p)$ universal.
\end{lemma}
\begin{proof}
From Definitions \ref{definition-trimmed-system} and \ref{definition-limit-function} it is clear that $\Prob{f \in H(p, l, S)} \geq 1 - p$ and, as a result, $|H(p, l, S)| \geq (1 - p)|H|$.
Moreover for different $x, y \in U$ it holds that
\[
\begin{split}
& \Prob{f(x) = f(y) \mid f \in H(p, l, S)} 
	= \frac{\Prob{f(x) = f(y),\, f \in H(p, l, S)}}{\Prob{f \in H(p, l, S)}} \\
	& \qquad \leq \frac{\Prob{f(x) = f(y),\, f \in H}}{1 - p} = \frac{\Prob{f(x) = f(y) \mbox{ for } f \in H}}{1 - p}. \\
\end{split}
\]
\qed
\end{proof}

Because the expected length of a chain is upper bounded by $c\alpha$ for $c$-universal systems, for $(l, p)$-trimmed systems we obtain that $\Expect{\psl} \leq c\alpha/(1-p)$.

Lemma \ref{lemma-no-trials} gives an estimate on the expected number of trials needed to find a function $f \in H(p, l, S)$ if $f$ is chosen uniformly from $H$.
\begin{lemma}
\label{lemma-no-trials}
If $l$ is a limit function with a trimming rate $p$, then the expected number of trials needed to find a function from $H(p, l, S)$ is at most ${1}/{(1 - p)}$.
\end{lemma}
\begin{proof}
Because the subsequent choices are independent, we know that the probability of having at least $k$ trials is at most $p^{k - 1}$. Hence the expected value is upper bounded by $\sum_{k = 1}^{\infty} p^{k - 1} = \sum_{k = 0}^{\infty} p^k = {1}/{(1 - p)}.$
\qed
\end{proof}

\subsection{Amortized Analysis}
We analyze the amortized complexity of the model with the potential method. Let $p_i$ be the potential of the hash table after performing $i$\textsuperscript{th} operation and $t_i$ be the time consumed by the operation. We define the amortized complexity of the operation as $a_i = t_i + p_i - p_{i - 1}$. The expected amortized complexity of the sequence $A$ equals
\[
\Expect{A} = \sum_{i=1}^{k} \Expect{a_i} = \sum_{i = 1}^{k} \left(\Expect{t_i} + \Expect{p_i} - \Expect{p_{i - 1}}\right) = \Expect T + \Expect{p_k} - \Expect{p_0}.
\]

%TODO: Staticke odhady.
%Although deletion of an element can not prolong chains, such static bound may be invalid since the assumed static model is no longer preserved. Let us note that dynamic extensions of Theorem \ref{theorem-universal-hashing-two-choices} are available and discussed in \cite{DBLP:journals/jacm/Vocking03}. However, it is easy to see that allowing deletions with the static estimates does not cause any problem in our model .
%TODO: Kolaps lemy 3?
%Lemma \ref{lemma-sets} states an upper bound on the expected number of trials needed to enforce Chain Length Limit Rule for a sequence of sets $S_1, \dots, S_k$.
%\begin{lemma}
%\label{lemma-sets}
%Let $S_1 \subset \dots \subset S_k$ be a sequence of sets such that $|S_k| \leq \alpha' m$ and $h_0 \in H$ be an initial function. Let $h_1, \dots, h_l \in H$ be a sequence of independently and uniformly chosen functions tried to enforce Chain Length Limit Rule. Assume that $0 = i_0 < \dots < i_k = l$ is a sequence of integers such that for each $j \in \{0, \dots, k - 1\}$
%\begin{itemize}
%\item the functions $h_{i_{j}}, h_{i_{j} + 1}, \dots, h_{i_{j + 1} - 1}$ are not suitable for the set $S_{j + 1}$ and 
%\item the function $h_{i_{j}}$ is suitable for the set $S_j$.
%\end{itemize}
%Then $\Expect{l} \leq {1}/{(1 - p)}$.
%\end{lemma}
%\begin{proof}
%A function $h$ is suitable for the set $S_k$ only if it is suitable for all the sets $S_1, \dots, S_k$. Using Lemma \ref{lemma-no-trials} for the set $S_k$ completes the proof.
%\qed
%\end{proof}

\begin{theorem}
\label{theorem-amortised-expected-time}
Assume the hash table described in Algorithm \ref{algorithm-hash-table} obeying the invariants (1) -- (3). Then the expected amortized time complexity of each operation is constant and the worst case running time of Find operation is $\bigo(1 + l(m, \alpha', p))$.
\end{theorem}
\begin{proof}
To prove the theorem assume that we are given a sequence of operations Insert, Delete and Find. We consider Insert and Delete to be successful or unsuccessful according to the comments in Algorithm \ref{algorithm-hash-table}. Since Find, unsuccessful Delete and unsuccessful Insert do not change the stored set, they do not have any effect on the dictionary and its potential and thus we may omit them from the sequence. The worst case complexity of Find follows from Chain Length Limit Rule. The expected running time of Find and unsuccessful updates follows from Lemma \ref{lemma-trimmed-system}. 

Now we have to prove the statement for successful update operations. To do so we partition the sequence into two types of cycles and for both of them we define a potential. The first cycle type is used to amortize violations of Load Factor Rule. With the second one we are able to distribute the time required to repair violations of Chain Length Limit Rule.

\begin{definition}[$\alpha$-cycle, $l$-cycle]
Cycles partition operations of the analyzed sequence.
Each \emph{$\alpha$-cycle} ends just after the operation causing violation of Load Factor Rule.
Each \emph{$l$-cycle} ends after $(\alpha'-\alpha_u)m$-th successful insertion in the $l$-cycle or after an operation which causes violation of Load Factor Rule (both cycles end).
\end{definition}

Let us define $e = 1/(1-p)$. Let ($i_{\alpha}$, $d_\alpha$, $i_l$) be the number of successful (insertions, deletions, insertions) performed in the current ($\alpha$, $\alpha$, l)-cycle. Let $r$ be the number of performed trials of a hash function caused by violation of Chain Limit Rule and $c$ be the number of started $l$-cycles. Both $r$ and $c$ are counted from the initial state. We define the potential $p_1 = {2ei_{\alpha}}/{(\alpha_u - \alpha_m)} + {2ed_{\alpha}}/{(\alpha_m - \alpha_l)}$ and the potential $p_2 = {ei_{l}}/{(\alpha' - \alpha_u)} + (ce - r) m$.  The overall potential is $p = p_1 + p_2$.

We decompose each update operation into an actual update and a possible rehashing part. Since the expected length of a chain is constant the expected running time of an update is constant as well. Observe that the overall potential is increased by a constant if the analyzed operation is not the last one in a cycle. Hence the amortized complexity of an update is constant. 

Each trial of a function from $H$ caused by Chain Limit Rule violation is amortized by incrementing $r$. When we break Load Factor Rule, the $\alpha$-cycle ends so $p_1$ decreases by at least $2em$ and $p_2$ is increased by at most $em$, hence $\Delta p \leq -em$ which can pay for the rehashing work. Similarly when an $l$-cycle is ended because $i_l = (\alpha' - \alpha_u)m$, then $\Delta p_2 = 0$. 

Now we argue that $\Expect{ce - r} \geq 0$. If we omitted deletes from each $l$-cycle, we would get a sequence of sets created only be insertions. Deletes may not prolong chains so from Lemma \ref{lemma-sets} it follows that the expected number of hash function trials in an $l$-cycle is at most $e$. Since $c$ is the number of started $l$-cycles and $r$ is the total number of such trials, we have $\Expect{ce} \geq \Expect{r}$ and thus $\Expect{p_k} \geq 0$. The theorem now follows because our potential is expected to be non-negative, $p_0 = \bigo(1)$ and as we have already seen $\Expect{T} \leq \Expect{A} - \Expect{p_k} + \Expect{p_0}$.
\qed
\end{proof}

In case of "insertion only" limit functions Delete operation may be allowed as well without any further change. If each $l$-cycle started with a rehash, then we would obtain an insertion only hash table. Observe that one additional rehash in each $l$-cycle can be amortized to constant time by introducing another potential $p_3 = {ei_{l}}/{(\alpha' - \alpha_u)}$. Moreover, performing the mentioned rehash is not necessary and, as a result, the first rehash in each $l$-cycle may be caused by invalidity of the limit function.

With trimming rates which tend to zero with growing $n$, e.g. in case of the two choices paradigm, the amortization overhead caused by Chain Length Limit Rule gradually disappears. It is caused by the fact that for a sufficiently large $n$ almost each function is suitable for every $S$.

\section{Obtaining the limit function}
\label{section-limit}

So far we have studied the possibility of providing a worst case guarantee for Find if we are given a limit function. 
In this section we show various examples of sublinear limit functions.

\subsection{Linear hash functions}
Representing the universe $U$ and the addresses of the hash table $V$ as vector spaces is natural when keys and addresses are interpreted as zero-padded  bitstrings of a fixed length. Alon, Dietzfelbinger, Bro Miltersen, Petrank and Tardos\cite{DBLP:journals/jacm/AlonDMPT99} found an interesting upper bound on $\Expect \lpsl$ with the system of all linear functions between two vector spaces over the field $\mathbb{Z}_2$.

\begin{theorem}
\label{theorem-linear-hash-functions-dietzefelbinger}
Suppose universal hashing with the system of linear functions from $U$ to $V$. 
If $m \log m$ elements are stored in a table of size $m$, then $\Expect{\lpsl} = \bigo(\log m \log \log m)$. 
\end{theorem}

The major problem of Theorem \ref{theorem-linear-hash-functions-dietzefelbinger} is a high multiplicative constant. However, it can be significantly reduced by a refinement of the original proof. The improved result is stated in Theorem \ref{theorem-linear-refined} \cite{babka-mt}.

\begin{theorem}
\label{theorem-linear-refined}
Assume universal hashing with the system of all linear transformations between vector spaces over $\mathbb{Z}_2$. 
Let $p \in (0, 1)$ be the trimming rate and $\alpha > 0$. 
If $n = \alpha m$ elements are stored inside the hash table of size $m$, then $$\Expect{\lpsl} \leq 538 \alpha \log n \log \log n + 44\mbox{ and}$$ $$\Prob{\lpsl \geq a_{\alpha, p} \log m \log \log m + b_{\alpha, p}\log m} < p.$$ where the values $a$ and $b$ depend only on the choice of $\alpha$ and $p$.
\end{theorem}

Let us note that the exact constants for different trimming rates and load factors can be found by a simple computer program.
For example the choice of $\alpha = 1.5$ and $p = 0.5$ yields $a = 57.29$ and $b = 0$.
In addition, constant $a$ can get arbitrarily close to $1$ but such estimates hold only for large values of $n$.

\subsection{Two choices paradigm}
A recent study\cite{DBLP:conf/stoc/AzarBKU94} of balls and bins systems discovered that hashing with at least two independent fully random hash functions brings remarkable results if it is done properly. 
If each stored element is put inside a least loaded chain of $d$ ones, then with a high probability the most loaded bin contains ${\ln \ln n}/\ln d + \bige(1)$ balls.
Further improvements of the states result may be derived using witness tree analysis\cite{DBLP:journals/jacm/Vocking03}. 

\begin{theorem}
\label{theorem-universal-hashing-two-choices}
Let $H$ be an $\omega$-universal system and $d \in \mathbb{N}$, $d \geq 2$. Assume that each stored element $x$ is placed into a least loaded chain of chains $f_1(x), \dots, f_d(x)$ where hash functions $f_1, \dots, f_d$ are chosen uniformly and independently from $H$. If $n \leq m$, then $\Prob{\lpsl > \frac{\ln \ln n}{\ln d} + 5} \in \littleo\left(1\right).$
\end{theorem}

Bounds for $k$-wise independence with constant number of functions and for general $S \subset U$ do not follow from the stated result and are not trivial.
On the other hand the stated result holds in case of uniform or almost uniform systems.
In addition a convenient example of a uniform system appeared in\cite{DBLP:journals/siamcomp/PaghP08}.
The system consists of functions which can be computed in a constant time. 
Let us note that a slight increase of the trimming rate occurs when using the uniform system since its uniformity is probabilistic.

The problem of using $k$-wise independent functions is addressed by Woelfel\cite{InProc-Woe2006a}. 
This work shows how to use simple universal classes with the two choices paradigm.
So we can conclude that there is a way how obtain an impressive doubly logarithmic worst case warranty with the two choices paradigm.

\chapter{Conclusion}
In this work we present a model of universal hashing which preserves the constant expected length of a chain. The running time of the find operation is then $O(1)$, too. The model uses the system of linear transformations and exploits its remarkable properties shown in \cite{DBLP:journals/jacm/AlonDMPT99}. In addition, these results are substantially improved so that they allow the construction of the model. We are also able to show that the expected amortised running time of the insert and delete operations is constant. Not only that, our model bounds the worst case running time of the find operation in $O(\log m \log \log m)$. 

Because the model is based on the system of linear transformations, the time to compute the hash value of an element is worsened. The solution that we propose is to store once computed hash values within the object. This optimisation takes the advantage of the warranties provided by the model, if the find operation is dominant. 

\section{Future Work}
There are many ways how to improve the model. For example, we can go the way of the tighter estimates. Although, it may also be more interesting to describe the behaviour of double hashing when it is used with a universal class of functions. Combined with the system of linear transformations it may be possible to obtain a similar worst case bound without violating the expected running times. 

Our next option is to use ideas of perfect hashing. Every chain may be represented by a hash table allowed to have a small load factor. Whenever the elements in the bucket can not be accessed in a constant time, then it might be possible to rehash only the small table instead of the large one. This approach may bring another optimisation.

Another brilliant idea comes from the area of load balancing. We can hash by two functions simultaneously and a newly stored element is placed into a smaller bucket. The find operation has to search in both buckets associated with an element. However, as stated in \cite{1076315} the expected worst case time in classic hashing is then substantially better and the expected complexity is preserved.

Unfortunately, the thesis does not show the experimental results. A high quality benchmark of the model is required. The benchmark needs to be performed with and without the mentioned cache optimisation. This should show the influence of the linear system on the running times. The benchmark has to show when the warranty is needed in dependence on the operations composition and the input distribution. If we try using inputs created by a random number generator, then they are uniformly distributed. So the obtained chains are short even for classic hashing. On the other hand such good inputs are seldom present. The real inconvenient inputs should be another output of the benchmark.When the chains are longer and the find operation is the most frequent one, then it is convenient to have a worst case warranty especially for large number of stored elements. The question is, how frequent the find operation has to be when compared to the modifying ones.

Of course, usage of simpler classes brings faster times of computing the hash codes. Linear classes are quite similar to each other. Maybe we could find a correspondence allowing the results found for the class of linear transformations to be brought into a faster class.

Models providing a reasonable worst case warranty with a good expected time complexity may be a suitable solution for various set representation problems. Current models of hashing may provide such qualities when enriched by simple rules. Also the approach of relaxing the models providing warranties may be helpful when achieving similar results.


\subsubsection*{Acknowledgments.}
% TODO: Lepsia formulacia (others).
We would like to thank Václav Koubek and some others for their helpful comments.

\begin{thebibliography}{99}
 \addcontentsline{toc}{chapter}{Literatúra}
 
 \bibitem{linear-hash-functions}Alon N., Dietzfelbinger M., Miltersen P. B., Petrank E., Tardos G.: {\em Linear Hash Functions}, JACM, 1999.

 \bibitem{universal-hashing}
  \begin{flushleft}
	Wikipedia: {\em Universal hashing},\\
	http://en.wikipedia.org/wiki/Universal{\_}hashing, október 2009
  \end{flushleft}

\end{thebibliography}


\clearpage
% \section*{Appendix -- proof of Theorem \ref{theorem-amortised-expected-time}}

Let us describe how we have chosen the potential function. In the proof we partition the sequence of performed operations into two types of cycles. We distinguish between the work required to enforce Load Factor Rule and the work needed by keeping Chain Limit Rule. During so called $\alpha$-cycles we gather potential needed to rehash the table to enforce Load Factor Rule. From this potential we pay the needed \emph{Rehash} operation at the end of the cycle. The second type of cycles, l-cycle, is essential for analysis of Chain Limit Rule. Every l-cycle has its potential charged at the beginning and from this potential we are able to pay the expected time spent by keeping Chain Limit Rule.

We deal with the amortised time of Find and unsuccessful Insert or Delete operations in advance. Their expected running time is proportional to the expected chain length. From Theorem \ref{theorem-expected-chain-length-universal} it follows that this value is constant. Since chains are bounded by $l(m, \alpha', p)$ we have that the worst case time of Find operation is $O(1 + l(m, \alpha', p))$. We require that these operations do not change the potential and with our potential this is true. Our analysis is thus simplified by omitting \emph{Find} and unsuccessful \emph{Delete} and \emph{Insert} operations from the sequence of operations. So let the sequence $o = \{o_i\}_{i=1}^{k}$ denote the successful \emph{Insert} and \emph{Delete} operations, $o_i \in \{Insert, Delete\}$ for $i = 1, \dots, k$.

\begin{definition}[$\alpha$-cycle]
Every \emph{$\alpha$-cycle} ends just after the operation causing violation of Load Factor Rule.
\end{definition}
Notice that it is not important if load factor violates the upper or the lower bound.

\begin{definition}[l-cycle]
The \emph{l-cycles} are the partitioning of the sequence $\{o\}_{i = 1}^{k}$ such that every l-cycle ends after the operation satisfying either of the following conditions is satisfied.
\begin{enumerate}
\item The operation causes the violation of the load factor rule.
\item The operation is the $(\alpha' - \alpha_u) m$\textsuperscript{th} successful insertion from the beginning of the l-cycle.
\end{enumerate}
\end{definition}
Notice that if an $\alpha$-cycle ends after an operation, the corresponding l-cycle also ends after the same operation, too. 

The potential $p$ consists is the sum of two parts $p_1$ and $p_2$ so $p = p_1 + p_2$. The first part of potential is used to distribute the time needed for rehashing the table at the end of an $\alpha$-cycle across operations inside it. The second parts deals with the expected time needed to enforce Chain Limit Rule.

Let $e$ denote the expected number of trails when finding a suitable function for a set, $i_{\alpha}$ be the number of insertions and $d_{\alpha}$ be the number of deletions performed successfuly so far in the current $\alpha$-cycle. The value $i_l$ denotes the number of insertions performed so far in the current l-cycle. The variable $r$ denotes the number of performed \emph{Rehash} operations, which are caused by Chain Limit Rule violation counted from the initial state. The variable $c$ denotes the number of current l-cycle counted from the beginning starting at one. We define the parts $p_1$ and $p_2$ as
\[
\begin{split}
p_1 & = \frac{2ei_{\alpha}}{\alpha_u - \alpha_k} + \frac{2ed_{\alpha}}{\alpha_m - \alpha_l}, \\
p_2 & = \frac{ei_{l}}{\alpha' - \alpha_u} + (ce - r) m.
\end{split}
\]

Remark that the initial potential $p_0$ equals $em_0$ where $m_0$ is the initial size of the hash table and hence $p_0 \in O(1)$. Execution of a single operation, without possible subsequent rehash, is expected to take $O(1)$ time because we iterate through a chain with a constant expected length and obtaining the hash value takes only constant time. Possible rehash seeks for a suitable function every try requires $O(m)$ time and by Lemma \ref{lemma-linear-transformations-trials} we expect $e$ trials. In the analysis we just compute the potential difference and assume that rehash takes $O(em)$ time and the operation itself runs in $O(1)$ time. In the proof we use the notation that values of variables $c, r, i_\alpha, d_\alpha, i_l$ refer to the state just before the execution of the analysed operation.

The analysis of \emph{Delete} operation is simpler and is shown first. When a deletion is performed we have to discuss the following two cases.
\begin{itemize}
\item \textbf{\emph{Delete} operation is not the last one in its $\alpha$-cycle.} The potential difference is constant since $\Delta p = \Delta p_1 + \Delta p_2 = \frac{2e}{\alpha_m - \alpha_l} + 0 \in O(1)$.

\item \textbf{\emph{Delete} operation is the last one in its $\alpha$-cycle.} Notice that at the end of the cycle $d_\alpha = (\alpha_m - \alpha_l)m$ and after the operation values of $i_\alpha$ and $d_\alpha$ are zeroed. The expected amortised complexity of the operation is constant since
\[
\begin{split}
a
	& = O(1) + O(em) + \Delta p_1 + \Delta p_2 \\
	& \leq O(em) -2em + ((c + 1)e - r)m - (ce - r)m \\
	& = O(em) - em.
\end{split}
\]

After rescaling the potential the claim holds.
\end{itemize}

The analysis of \emph{Insert} now follows.
\begin{itemize}
\item \textbf{The operation is not last in neither of its $\alpha$-cycle or l-cycle and Chain Limit Rule is not violated.}
We have already shown that the expected running time is constant and the potential change is constant, too, since the potential change is constant, 
$\Delta p = \Delta p_1 + \Delta p_2 = \frac{2e}{\alpha_u - \alpha_k} + \frac{e}{\alpha' - \alpha_u} \in O(1)$.

\item \textbf{The operation is last in its $\alpha$-cycle.} 
Since at the end of the $\alpha$-cycle $i_\alpha = (\alpha_u - \alpha_m)m$ the expected amortised time required to execute the whole operation may be bounded from above as
\[
\begin{split}
a
	& = O(1) + O(em) + \Delta p  \\
	& \leq O(em) - 2em + ((c + 1)e - r)m - (ce - r)m \\
	& = O(em) - em.
\end{split}
\]

Scaling of the potential from the analysis of \emph{Delete} operation is sufficient for this case and the claim thus holds. 

\item \textbf{Operation is the last one in the l-cycle and Chain Limit Rule is not violated.} Under these assupmtions it follows that $i_l = (\alpha' - \alpha_u)m$ hence $\Delta p_2 = ((c + 1)e + r)m - em - rm = 0$. Since $\Delta p_1 = \frac{2e}{\alpha_u - \alpha_m}$ the expected amortised time of the operation is constant.

\item \textbf{Chain Limit Rule was violated during the performed insertion.}
The operation took $O(1) + O(\Delta r m)$ time. Whole potential change is equal to \[ \frac{2e}{\alpha_u - \alpha_m} + \frac{e}{\alpha' - \alpha_u} - \Delta r m .\] The already performed rescaling of the potential deals with the time needed to rehash the table. The expected amortised complexity of the operation is constant.
\end{itemize}

In order to be properly able to estimate the expected running time of the sequence of operations we have to show that $\Expect{p_k} \geq 0$. If it holds, then from Remark \ref{remark-expected-sequence} it follows that $\Expect{T} = \Expect{A} + O(1)$. In our analysis we have already shown that $\Expect{A} + O(1) = O(k) + O(1) = O(k)$. 

At first notice that the variable $c$ is incremented by one at the beginning of every $l$-cycle. The part $p_2$ of potential is thus increased by $em$; the potential is ``charged''. This ``charge'' is paid by the operations from the previous $l$-cycle or from $p_0$ if we are in the initial state. Now consider the sequence of sets $S_1, S_2, \dots$. Let $S_1$ be equal to the set stored at the beginning of the l-cycle. $S_2$ is the union of $S_1$ and the set stored immediately after the first violation of Chain Limit Rule. $S_3$ is the union of $S_2$ and the set stored after the second violation and so on. Realise that there are at most $(\alpha' - \alpha_u)m$ successful insertions in an l-cycle and $|S_1| \leq \alpha_u m$. So the last set of the sequence contains at most $\alpha'm$ elements. We can use Lemma \ref{lemma-sets} for the sequence and immediately obtain that the expected number of trials in an l-cycle equals $e$. From this fact it is clear that during an l-cycle $\Expect{\Delta r} = e$ and at the end of every l-cycle $\Expect{ce - r} = 0$. Realising the obvious fact that during an l-cycle the value of $r$ may only grow we conclude that
\[
\Expect{p_k} = \Expect p_1 + \Expect p_2 \geq m\Expect{ce - r} \geq 0.
\]
\qed
\section*{Appendix -- proof of Theorem \ref{theorem-linear-refined}}
\setcounter{theorem}{1}

\begin{theorem}
Assume universal hashing with the system of all linear transformations between vector spaces over $\bbbz_2$. When storing $n = \bige(m)$ elements, then $$\Expect{\lpsl} \leq 538 \alpha \log n \log \log n + 44,$$ $$\Prob{\lpsl \geq 57.29 \log m \log \log m} < 0.5.$$
\end{theorem}

Although the proof of Theorem \ref{theorem-linear-refined} is rather technical we shall enclose it for further possible verification of the results. It may be found in Chapters 5 and 6.3 at \url{http://ktiml.mff.cuni.cz/~babka/hashing/thesis.pdf}. There is one change in the notation, the set of all addresses of the hash table is denoted by $B$ instead of $V$.
\section*{Appendix -- proof of Theorem \ref{theorem-universal-hashing-two-choices}}
\setcounter{theorem}{3}

\begin{theorem}
Let $H$ be an almost uniform universal system with a constant $c$ and $d \in \bbbn$, $d > 1$. Assume that each stored element $x$ is placed into a least loaded chain of chains $f_1(x), \dots, f_d(x)$ where hash functions $f_1, \dots, f_d$ are chosen uniformly and independently from $H$. If $n \leq m$, then $$\Prob{\lpsl > \frac{\ln \ln n}{\ln d} + 3c^2 + 4} \in \littleo\left(1\right).$$
\end{theorem}

We state that the original result \cite{DBLP:journals/jacm/Vocking03} of Vöcking holds in case of $c > 1$.
A more detailed proof can be found in the original paper. 
First we state the result when the size of the table is the same as the number of elements, $m = n$.
Here we briefly describe the construction but emphasize all the changes.
It is obvious that in case of universal system with $c = 1$ the statement holds since we may assume full independence up to $n$ elements and the probability bounds used in the proof are automatically correct.
When $c > 1$ we have to compensate higher collision probability by slightly increasing the height of the witness tree.

First, we show that if there is a chain containing $L + 3c^2$ elements, then we are able to construct an activated pruned witness tree of order $L$.
The next step is to estimate from above the probability of existence of an activated pruned witness tree of order $L$.
Hence we obtain the bound on the probability of having a chain consisting of at least $L + 3c^2$ elements.

\subsection{Witness tree}
\begin{definition}[Witness tree]
\emph{Witness tree} of order $L$ is a complete oriented $d$-ary tree with edges oriented in the direction from root to the leaves. Each node of the tree is assigned an element of the stored set $S$.
\end{definition}

We freely exchange the roles of a node and the element assigned to it because this simplification does not cause any mistake.
We refer to the element assigned to a node $v$ as to the variable $v$ without any further notification.
The value $f(x)$ denotes the chain where the element $x \in S$ is placed.

\begin{definition}[Edge event, leaf event, activated witness tree]
We associate two types of events with the tree:
\begin{itemize}
	\item \textbf{Edge event} Consider en edge $(u, v)$. The \emph{edge event associated with the edge $(u, v)$} occurs if $\exists i \in \{1, \dots, d \} \colon f(u) = f_i(v)$.
	\item \textbf{Leaf event} Consider a leaf $v$. The \emph{leaf event associated with the leaf $v$} occurs if the chain $f(v)$ at the insertion time of $v$ contains at least $3c^2$ elements.
\end{itemize}

We say that \emph{witness tree is activated} if all the edge and leaf events occur.
\end{definition}

\subsection{Construction of the activated witness tree}
We just repeat the original construction. 
Let $y$ be the chain having at least $L + 3c^2$ elements. 
We create an activated witness tree of order $L$ witnessing the event of having a chain of length $L + 3c^2$ elements.
We begin by assigning the topmost element $x$ of the chain $y$ to the root. 
Then we look at the chains $f_1(x), \dots, f_d(x)$ and they must contain at least $L + 3c^2 - 1$ elements. 
The topmost elements in these chains are assigned to the children of the root and the process continues recursively. 

Notice that leaves are assigned elements placed into the chains containing at least $3c^2$ elements at their insertion times.
Hence all the leaf events occur.
Realize that all the edge events, too, because on each edge $(u, v)$ the element $v$ is chosen to be from the chains $f_1(u), \dots, f_d(u)$.

\subsection{Probability of the activation of a witness tree}
We estimate from above the probability that there is an activate witness tree of order $L$.
First we make a simplifying assumption that elements assigned to the nodes of the tree are different.
In the following we will get rid of the assumption and obtain a complete proof of the statement.
Let $N$ be the number of all nodes in the tree of order $L$. The number of edges in the tree is thus $N - 1$ and the number of leaves is $d^L$.
\begin{itemize}
	\item The number of all assignments of elements to the nodes is at most $n^N$.
	\item The probability that en edge event occurs is at most ${cd}/{n}$ because we use an almost uniform system with the constant $c$.
	\item 
		Since we have $n$ elements, there are at most ${n}/{3c^2}$ chains consisting of at least $3c^2$ elements.
		The probability that an leaf event occurs is hence at most $({3c})^{-d}$ because each of $d$ function must place the element inside a chain having at least $3c^2$ elements.
\end{itemize}

We use the following estimates in the computation $N \leq 2d^L$ and $2d^2 \leq 3^d$.
Since we assume hashing with a uniform system and that the nodes are assigned different elements the leaf and edge events are independent.
Thus we can conclude that the probability of existence of an activated witness tree of order $L$ is at most:
\[
\begin{split}
n^N \cdot \left(\frac{cd}{n}\right)^{N - 1} \cdot \left({3c})^{-d}\right)^{d^L} 
	& \leq n \cdot \left(\frac{c^2 d^2}{(3c)^{d}}\right)^{d^L} \\
	& \leq n \cdot \left(\frac{c^2 3^d}{2(3c)^{d}}\right)^{d^L} \\
	& \leq n \cdot 2^{-d^L}.
\end{split}
\]

To make the probability estimate a function from $\littleo(1)$ it is sufficient to choose $L = \log_d \log_2 n + \log_d(1 + \alpha)$ for any constant $\alpha > 0$.

\subsection{Pruned witness tree}
Our witness tree of order $L$ witness the event of having a chain consisting of $L + 3c^2$ elements, which is slightly more than in the original result when $c > 1$.
This is the way how we managed to get the same upper bound on the probability of existence of a activated witness tree.
Now we remove the assumption that the nodes of the tree are assigned different elements.

For more comprehensive definitions consult the original proof, however these definitions are satisfactory in our situation.

\begin{definition}[Full witness tree, edge events]
Let $\mathcal{K} \geq 2$ be an integer. Full witness tree of order $L$ is a tree which root has exactly $\mathcal{K}$ children. Each of these children has exactly one child that is the parent of a witness tree of order $L$. Labeling of nodes is done in this way:
\begin{itemize}
	\item Root is assigned one of the $n$ chains. 
	\item Each child of the root is assigned an element $x \in S$ and a number $i \in \{1, \dots, d\}$.
	\item Elements assigned to the children of the root are different.
	\item Grandchildren of the root conform to the assignment defined for witness trees.
\end{itemize}

Let $r$ be the root of the full witness tree.
The edge events for the added edges need to be defined. 
\begin{itemize}
	\item Edge event on the edge $(r, u)$ occurs if $\exists i: f_i(u) = r$.
	\item Assume that $u$ is a child of $r$ and $i$ is the number assigned to the node $u$. Edge event on the edge $(u, v)$ occurs if $\exists j: f_j(v) = f_i(u)$.
\end{itemize}
\end{definition}

We prune full witness tree so that the nodes in the remaining tree are assigned different elements.
The tree which is obtained after pruning is called \emph{pruned witness tree}.
The algorithm for pruning the full witness tree is a BFS procedure which cuts off each edge $(u, v)$ if the element assigned to the node $v$ is has already been encountered before visiting the node $v$.
The edge $(u, v)$ is called \emph{cutoff}.
The pruning procedure performs at most $\mathcal{K}$ cutoffs and then stops.
The visited nodes create the pruned witness tree.
With each witness tree we also remember the cutoff edges.

First observe that if there is a chain $y$ having $L + \mathcal{K} + 3c^2$ elements, then we can create an activated full witness tree.
Root is assigned the chain $y$.
Children of the root are assigned the $\mathcal{K}$ topmost balls of the chain $y$.
Each child $u$ also needs to have a value $i$ -- it is assigned $i \in \{1, \dots, d\}$ such that $f_i{u} = y$. 
The child $v$ of the node $u$ is assigned the topmost element in the chain $f_{(i + 1) \bmod d + 1}(u)$ at the insertion time of $u$.
Also observe that the constructed full witness tree is activated.

Now we estimate the probability of existence of an activated pruned witness tree.

First, assume that there are less than $\mathcal{K}$ cutoff edges in the pruned witness tree.
Then at least one witness tree rooted at a grandchild of the root has not been pruned at all.
The probability of its activation is $o(1)$ and this completes the proof for this case.

Now assume that there are exactly $\mathcal{K}$ cutoff edges, the pruned witness tree consists of $q$ leaves and $N$ nodes and full witness tree from which we started contains $M$ nodes.

We estimate, as in the previous case, the number of assignments and the probability of activation of each event.
\begin{itemize}
	\item There are at most $M^{\mathcal{K}}$ ways how to created a pruned witness tree from the full one.
	\item The number of all assignments to the nodes is bounded from above by $n^N \cdot d^\mathcal{K}$.
	\item The probability that a leaf event occurs is at most $(3c)^{-d}$.
	\item The probability of each type of edge event is ${cd}/{n}$.
	\item The probability of existence of a cutoff edge $(u, v)$ is at most ${Mcd}{n}$.
\end{itemize}

We choose $L = \log_d \log_2 n + \log_d(1 + \alpha)$, as in the case of plain witness trees.
From that we obtain that $M \leq 1 + \mathcal{K} + \mathcal{K} \cdot d^L \leq 2\mathcal{K}(1+\alpha)\log_2 n$.
In the following we also need that $N \leq 2q$ and $d^2 \leq 3^d$. 

Hence the probability of existence of an activated pruned witness tree is at most
\[
\begin{split}
M^{\mathcal{K}} \cdot n^N & \cdot d^\mathcal{K} \cdot \left(\frac{cd}{n}\right)^{N - 1} \cdot (3c)^{-dq} \cdot \left(\frac{Mcd}{n}\right)^{\mathcal{K}} \\
	& \leq n \cdot (cd)^N \cdot (3c)^{-dq} \cdot \left(\frac{M^2 c d^2}{n}\right)^{\mathcal{K}} \\
	& \leq n \cdot \left((cd)^{2} \cdot (3c)^{-d}\right)^{q} \cdot \left(\frac{M^2 c d^2}{n}\right)^{\mathcal{K}} \\
	& \leq n \cdot \left(\frac{M^2 c d^2}{n}\right)^{\mathcal{K}} = n^{-\mathcal{K} + 1 + \littleo(1)} = \littleo(1). \\
\end{split}
\]

The theorem is the statement for $\alpha = 3$ and $\mathcal{K} = 2$.

\subsection{Hashing more than $m$ elements}

So far we have considered hashing of $n$ elements into a table of size $n$. For storing $n$, $n < m$, elements the result clearly holds.
In case of having more than $m$ elements, $\alpha > 1$, we can repeat the same argument as in \cite{DBLP:journals/jacm/Vocking03}.

\qed


\end{document}
