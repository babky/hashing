The exact definition of algorithms now follows. 
Variables available inside the procedures include the parameters of the model, variable $t$ denotes the array used to store the chains and $h \in H$ is the current hash function.

\begin{algorithmic}
\Procedure{Find}{$x$}
	\If{$x$ is inside chain $t[h(x)]$}
		\State \textbf{return} \textbf{true} \Comment{Successful Find}
	\Else
		\State \textbf{return} \textbf{false} \Comment{Unsuccessful Find}
	\EndIf
\EndProcedure
\State
\end{algorithmic}
\begin{algorithmic}
\Procedure{Insert}{$x$}
	\If{$x$ is not inside chain $t[h(x)]$}
		\State insert $x$ into chain $t[h(x)]$
		\If{$\alpha > \alpha_u$ \textbf{or} chain $t[h(x)]$ violates Chain Length Limit Rule}
			\State \Call{Rehash}{}
		\EndIf
		\State \textbf{return} \textbf{true} \Comment{Successful Insert}
	\Else
		\State \textbf{return} \textbf{false} \Comment{Unsuccessful Insert}
	\EndIf
\EndProcedure
\end{algorithmic}
\begin{algorithmic}
\Procedure{Delete}{$x$}
	\If {$x$ is inside chain $t[h(x)]$}	
		\State delete $x$ from chain $t[h(x)]$
		\If{$\alpha < \alpha_l$}
			\State \Call{Rehash}{}
		\EndIf
		\State \textbf{return} \textbf{true} \Comment{Successful Delete}
	\Else
		\State \textbf{return} \textbf{false} \Comment{Unsuccessful Delete}
	\EndIf
\EndProcedure
\end{algorithmic}
\begin{algorithmic}
\Procedure{Rehash}{}\Comment{Also finds a suitable hash function.}
	\State $t_{old} \leftarrow t$
	\State choose a new function $h$ from $H$
	\If{$\alpha < \alpha_l$ \textbf{or} $\alpha > \alpha_u$}
		\State $m \leftarrow \lceil\alpha_m m\rceil$
		\State $\alpha = n / m$
	\EndIf
	\Repeat
	\State create a new table $t$ with the size $m$
	\ForAll{$x$ in the $t_{old}$}
		\State place $x$ into chain $h(x)$ in $t$
	\EndFor
	\Until{all rules are satisfied}
\EndProcedure
\end{algorithmic}
